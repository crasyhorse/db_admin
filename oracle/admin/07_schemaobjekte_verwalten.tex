  \chapter{Schemaobjekte verwalten}
    \setcounter{page}{1}\kapitelnummer{chapter}
    \minitoc
\newpage
    \section{Tabellen}
      \subsection{Einf\"uhrung}
      Oracle kennt mehrere verschiedene Arten von Tabellen.
      \begin{itemize}
        \item Heap-Organized Tables (Standardtabellen)
        \item Index-Organized Tables
        \item Partitioned Tables
        \item Temporary Tables
      \end{itemize}
        \subsubsection{Heap-Organized Tables}
          Heap-Organized Tables sind die Basisdatenstruktur einer Oracle
          Datenbank. Daten werden in Zeilen und Spalten abgelegt. Jede Tabelle
          wird mit einem Namen und einer Anzahl von Spalten definiert. Jede
          Spalte hat einen Bezeichner, einen Datentyp und eine L\"ange.

          Bei dieser Art von Tabelle werden die Zeilen unsortiert, in der
          Reihenfolge ihrer Erstellung abgelegt. Dies ist vergleichbar mit einem
          \enquote{Haufen} (engl. Heap) bei dem alle Elemente einfach
          aufeinander geworfen werden.
          \bild{Heap-Organized Table}{heap_organized_table}{1.5}
          \abbildung{heap_organized_table} zeigt, dass die Tabellenzeilen, hier
          durch rote, blaue, gr\"une und gelbe K\"astchen dargestellt, einfach
          nacheinander abgelegt werden.
        \subsubsection{Index-Organized-Tables}
          In einer Index-Organized Table dagegen, werden die Daten als sortierte
          Baumstruktur abgelegt.
          \bild{Index-Organized-Table}{Index}{0.9}
          Die unterste Ebene dieses Baumes, auch \enquote{Leaf Level} genannt,
          dient zur Ablage der Daten. Die oberen Beiden k\"onnen als
          \enquote{Inhaltsverzeichnis} oder \enquote{Stichwortverzeichnis}
          verstanden werden. Dort werden Metadaten gehalten, mit deren Hilfe,
          die Daten in der Leaf Ebene aufgefunden werden k\"onnen. Nachteilig
          ist, dass der Baum bei jedem \"Anderungsvorgang gepflegt werden muss.
          Daraus folgt, dass eine solche Tabelle besonders da geeignet ist, wo
          hohe Lesegeschwindigkeit gefordert und ein geringes \"Anderungsvolumen
          vorhanden ist.
        \subsubsection{Partitioned Tables}
          Wie bereits bekannt, werden Tabellen in Form von Segmenten in einer
          Datendatei abgelegt. W\"achst eine Tabelle stark an, vergr\"o\ss{}ert
          sich folglich auch ihr Segment. Dadurch wird mit der Zeit das Arbeiten
          mit den betroffenen Daten sehr langsam. Um bei solch gro\ss{}en
          Tabellen Abhilfe zu schaffen, kann eine Tabelle in mehrere Segmente,
          so genannte Partitionen, zerteilt werden. Dies ist 1:1 vergleichbar
          mit einer Festplatte, die in mehrere Partitionen aufgeteilt wird.

          Ein gro\ss{}er Vorteil der Partitionierung ist, dass sie f\"ur
          Anwendungen v\"ollig transparent ist. Das bedeutet, dass
          SQL-Statements in keiner Weise ge\"andert werden m\"ussen. Es sieht
          nach wie vor alles so aus, als w\"are die Tabelle in \enquote{einem
          St\"uck} gespeichert. Ein weiterer Vorteil ist, dass die Datenbank
          nicht mehr die komplette Tabelle verarbeiten muss, wenn ein Nutzer nur
          einen Teil der Daten abrufen muss (h\"ohere
          Verarbeitungsgeschwindigkeit, aufgrund des niedrigeren Datenvolumens).
          Die Datenbank entscheidet selbstst\"andig, auf welchen Partitionen
          gearbeitet werden muss.
          \bild{Partitionierte Tabelle}{partitioned_table}{1}
        \subsubsection{Tempor\"are Tabellen}
          Wie der Name dieser Tabellenart bereits sagt, dienen sie zur tempor\"aren Aufnahme von Daten. Ihr Vorteil besteht darin, dass sie dynamisch wachsen und schrumpfen und keinerlei Redo-Informationen erzeugen. Dadurch sind sie ideal f\"ur Daten geeignet, die nur f\"ur kurze Zeit existieren.
      \subsection{Oracle 11g Advanced Compression}
        Oracle 11g Advanced Compression ist die Weiterentwicklung der Oracle Basic Compression Technologie, die seit Oracle 9i im Einsatz ist. Es ist nun m\"oglich:
        \begin{itemize}
          \item DIRECT-Load-Operationen,
          \item DML-Operationen,
          \item Unstrukturierte Daten (LOBs),
          \item Backups und
          \item Networktraffic
        \end{itemize}
        zu komprimieren.

        Die unter Oracle 9i eingef\"uhrte Basic Compression erm\"oglichte nur die transparente Komprimierung von Daten, die mittels eines Direct Path Load Vorganges (mittels SQL*Loader) in die Datenbank geschrieben wurden. Da aber die Masse der Daten mittels SQL oder des imp-tools geladen wurden, konnte die Kompression nur sehr selten zum Einsatz gebracht werden, was ihre Akzeptanz stark verringerte.
        \begin{merke}
          Die Advanced Compression ist eine lizenz- und kostenpflichtige Zusatzoption f\"ur die Enterprise Edition der Datenbank!
        \end{merke}
        \subsubsection{Wie funktioniert die Kompression?}
          Die Kompression von Tabellendaten erfolgt, indem mehrfach auftretende Werte eliminiert werden. Oracle speichert komprimierte Daten in einem Block als \enquote{selbst erhaltende Daten}, d. h. alles was zur Rekonstruktion der Daten ben\"otigt wird, ist im gleichen Block gespeichert, in dem auch die Daten selbst abgelegt sind. Die Datenkompression ist transparent, was bedeutet, dass alle Features, die auf unkomprimierte Bl\"ocke angewendet werden k\"onnen, auch auf Komprimierte anwendbar sind.
          \bild{Ein unkom\-pri\-mierter Datenblock}{uncompressed_block}{1}

          Der Aufbau eines unkomprimierten Datenblockes ist bereits bekannt. Nach dem Header, der alle Metainformationen zum Block enth\"alt, werden die Nutzdaten eingetragen.

          In einem komprimierten Block kommt eine neue Struktur hinzu, die \enquote{Symbol table}, die in \abbildung{compressed_block} blau dargestellt wird. Alle Werte, die vorher im unkomprimierten Block redundant waren, werden in ihr gespeichert. Im Datenbereich werden nur noch Zeiger (dargestellt durch rote X) abgelegt, die auf die jeweiligen Werte verweisen.
          \bild{Ein komprimierter Datenblock}{compressed_block}{1}
          \begin{merke}
            Zu beachten ist, dass sich die Kompression immer nur auf neue Datens\"atze auswirkt. Daten die schon vor dem Aktivieren der Kompression gespeichert waren, bleiben davon unber\"uhrt. Das bedeutet, dass ein Block sowohl komprimierte als auch unkomprimierte Datens\"atze enthalten kann.
          \end{merke}
        \subsubsection{Konfigurieren der Kompression}
          Die Komprimierung f\"ur Tabellen kann auf Ebene der Tablespaces oder der, der Tabellen selbst konfiguriert werden. Wird sie auf Tablespace ebene konfiguriert, werden automatisch alle Tabellen in diesem Tablespace komprimiert.
          \begin{lstlisting}[caption={Kompression auf Tablespace ebene aktivieren},label=admin300,language=oracle_sql]
-- Basic Compression
SQL> CREATE TABLESPACE data
  2  DATAFILE '/u02/oradata/orcl/data01.dbf' SIZE 100M
  3  DEFAULT  COMPRESS BASIC;

--Advanced Compression
SQL> CREATE BIGFILE TABLESPACE hr_data
  2  DATAFILE '/u02/oradata/orcl/hr_data01.dbf' SIZE 500M
  3  DEFAULT  COMPRESS FOR OLTP;
          \end{lstlisting}
          \begin{merke}
            Die Advanced Compression kann in smallfile und in bigfile Tablespaces zum Einsatz kommen!
          \end{merke}
          Auf Tabellenebene wird die Kompression ebenfalls mit der \languageorasql{COMPRESS}-Klausel aktiviert. Dies kann schon bei der Tabellenerstellung oder auch noch im Nachhinein erfolgen.
\clearpage
          \begin{lstlisting}[caption={Eine Tabelle mit OLTP Compression erstellen},label=admin301,language=oracle_sql]
SQL> CREATE TABLE buchung_hist (
  2    Buchungs_ID         NUMBER,
  3    Betrag              NUMBER(12, 2),
  4    Buchungsdatum       DATE,
  5    Konto_ID            NUMBER NOT NULL,
  6    Transaktions_ID     NUMBER NOT NULL,
  7    Archivierungsdatum  DATE NOT NULL,
  8    CONSTRAINT buchung_hist_pk PRIMARY KEY (Buchungs_ID)
  9  )
 10  COMPRESS FOR OLTP;
          \end{lstlisting}
          Die Tabelle \identifier{buchung\_hist} ist so konfiguriert, dass von Anfang an, alle Datens\"atze komprimiert werden. Das es sich hier um eine komprimierte Tabelle handelt, kann der View \identifier{dba\_tables} entnommen werden.
          \begin{lstlisting}[caption={Eine Tabelle mit OLTP Compression erstellen},label=admin302,language=oracle_sql]
SQL> SELECT table_name, compression, compress_for
  2  FROM   dba_tables
  3  WHERE  table_name LIKE 'BUCHUNG_HIST';

TABLE_NAME                      &COMPRESS& COMPRESS_FOR
------------------------------- -------- ------------
BUCHUNG_HIST                    ENABLED  &OLTP&
          \end{lstlisting}
          Existiert die Tabelle bereits und die Kompression soll nachtr\"aglich genutzt werden, gibt es zwei M\"oglichkeiten:
          \begin{itemize}
            \item Aktivieren der Kompression (nur neue Datens\"atze sind betroffen)
            \item Die gesamte Tabelle komprimieren (alte Datens\"atze werden ebenfalls komprimiert)
          \end{itemize}
          \begin{lstlisting}[caption={Nachtr\"agliches aktivieren der OLTP Kompression},label=admin303,language=oracle_sql]
SQL> ALTER TABLE buchung COMPRESS FOR OLTP;
          \end{lstlisting}
          Soll die gesamte Tabelle komprimiert werden, muss sie neu aufgebaut werden. Dies geschieht mit der \languageorasql{MOVE}-Klausel. Diese sorgt, in Verbindung mit der \languageorasql{COMPRESS}-Klausel daf\"ur, dass die Tabelle reorganisiert/defragmentiert und komprimiert wird.
          \begin{lstlisting}[caption={Komprimieren einer Tabelle},label=admin304,language=oracle_sql]
SQL> ALTER TABLE buchung MOVE COMPRESS FOR OLTP;
          \end{lstlisting}
          \begin{merke}
            Die Reorganisation einer Tabelle ist sehr ressourcenintensiv und sollte daher nur bei niedriger Grundlast im System durchgef\"uhrt werden.
          \end{merke}
          Muss die Kompression aus irgendeinem Grund wieder deaktiviert werden, geschieht dies mit der \languageorasql{NOCOMPRESS}-Klausel.
          \begin{lstlisting}[caption={Deaktivieren der Kompression},label=admin305,language=oracle_sql]
SQL> ALTER TABLE buchung NOCOMPRESS;
          \end{lstlisting}
          \begin{lstlisting}[caption={Dekomprimieren einer Tabelle},label=admin306,language=oracle_sql]
SQL> ALTER TABLE buchung MOVE NOCOMPRESS;
          \end{lstlisting}
        \subsubsection{Ein Experiment}
          Im nachfolgenden Experiment soll gezeigt werden, wie sich die Kompression auf den Platzbedarf einer Tabelle auswirkt.

          Im ersten Schritt wird die Anzahl der Datenbl\"ocke und die Anzahl der Tabellenzeilen pro Datenblock, f\"ur die unkomprimierte Tabelle \identifier{Buchung} ermittelt.
          \begin{lstlisting}[caption={Anzahl der Datenbl\"ocke + durchschnittliche Anzahl der Zeilen pro Block},label=admin307,language=oracle_sql,alsolanguage=sqlplus]
col segment_name format a20
SQL> SELECT segment_name, blocks
  2  FROM   dba_segments
  3  WHERE  segment_name LIKE 'BUCHUNG';

SEGMENT_NAME               BLOCKS
---------------------- ----------
BUCHUNG                      2176

SQL> SELECT AVG(Anzahl) As Mittelwert
  2  FROM   (SELECT  DBMS_ROWID.ROWID_BLOCK_NUMBER(rowid) AS Block_ID,
  3                  COUNT(*) AS Anzahl
  4          FROM    Buchung
  5          GROUP BY DBMS_ROWID.ROWID_BLOCK_NUMBER(rowid));

 MITTELWERT
-----------
 405,120428
          \end{lstlisting}
          Gem\"a\ss{} der Abfrage aus \beispiel{admin307} besteht die Tabelle \identifier{Buchung} aus 2176 Datenbl\"ocken mit durchschnittlich 405 Zeilen pro Datenblock.

          Der zweite Schritt besteht darin, die Tabelle \identifier{Buchung} zu komprimieren.
          \begin{lstlisting}[caption={Komprimieren der Tabelle \identifier{Buchung}},label=admin308,language=oracle_sql]
SQL> ALTER TABLE buchung MOVE COMPRESS FOR OLTP;
          \end{lstlisting}
          Zu guter Letzt, wird nun die gleiche Statistik, wie in \beispiel{admin307} erhoben.
          \begin{lstlisting}[caption={Anzahl der Datenbl\"ocke + durchschnittliche Anzahl der Zeilen pro Block nach der Kompression},label=admin309,language=oracle_sql,alsolanguage=sqlplus]
col segment_name format a20
SQL> SELECT segment_name, blocks
  2  FROM   dba_segments
  3  WHERE  segment_name LIKE 'BUCHUNG';

SEGMENT_NAME               BLOCKS
---------------------- ----------
BUCHUNG                      2048

SQL> SELECT AVG(Anzahl) AS Mittelwert
  2  FROM   (SELECT  DBMS_ROWID.ROWID_BLOCK_NUMBER(rowid) AS Block_ID,
  3                  COUNT(*) AS Anzahl
  4          FROM    Buchung
  5          GROUP BY DBMS_ROWID.ROWID_BLOCK_NUMBER(rowid));

 MITTELWERT
-----------
 445,672228
          \end{lstlisting}
          Die neue Statistik beweist es. Durch die Kompression besteht die Tabelle \identifier{Buchung} aus ca. 6 \% weniger Datenbl\"ocken und es werden im arithmetischen Mittel 10 \% mehr Tabellenzeilen pro Datenblock gespeichert.
          \begin{merke}
            Durch die Nutzung von komprimierten Tabellen wird der Platzbedarf an Arbeitsspeicher und Speicher auf dem Datentr\"ager reduziert. Daraus resultiert oft eine bessere Performance f\"ur Leseoperationen, die Kosten f\"ur Schreiboperationen steigen aber.
          \end{merke}
      \subsection{Tabellen reorganisieren}
        Eine Tabelle wird reorganisiert, indem sie in ein neues Segment verschoben wird. Dies kann auf zwei unterschiedliche Arten geschehen:
        \begin{itemize}
          \item Die Tabelle wird in ein neues Segment, im gleichen Tablespace, verschoben.
          \item Die Tabelle wird in ein neues Segment, in einem anderen Tablespace, verschoben.
        \end{itemize}
        Hierzu einige Beispiele:
        \begin{lstlisting}[caption={Reorganisieren der Tabelle \identifier{Buchung}},label=admin310,language=oracle_sql]
SQL> ALTER TABLE buchung MOVE;
        \end{lstlisting}
        \begin{lstlisting}[caption={Verschieben einer Tabelle in einen anderen Tablespace},label=admin311,language=oracle_sql]
SQL> ALTER TABLE buchung MOVE
  2  TABLESPACE users;
        \end{lstlisting}
        Das Verschieben einer Tabelle hat zur Konsequenz, dass die RowIDs aller Tabellenzeilen ver\"andert werden. Daraus resultiert:
        \begin{itemize}
          \item Das alle Indizes, die auf der Tabelle liegen, ung\"ultig werden und neu erstellt oder reorganisiert werden m\"ussen.
          \item W\"ahrend des Verschiebevorganges sind keine DML-Operationen auf der Tabelle m\"og\-lich.
          \item Es werden alle Statistiken f\"ur das automatische Performance Tuning ung\"ultig und m\"ussen neu gesammelt werden.
        \end{itemize}
        Dies zeigt, dass das Verschieben einer Tabelle eine sehr teure Angelegenheit ist und gut \"uberlegt sein sollte.
      \subsection{Informationen \"uber Tabellen sammeln}
        \begin{literaturinternet}
          \item \cite{ADMIN015}
        \end{literaturinternet}
    \section{Indizes}
      Indizes sind optionale Strukturen, die mit einer Tabelle verbunden sind. Sie werden benutzt, um die Geschwindigkeit von Abfragen zu erh\"ohen. Dies geschieht, indem sie die Anzahl der Datentr\"agerzugriffe pro Abfrage verringern.

      Es k\"onnen beliebig viele Indizes zu einer Tabelle erstellt werden, solange sich die Spaltenkombinationen f\"ur die Indizes unterscheiden. Dabei kann eine Spalte in mehreren Kombinationen vorkommen und ein Index kann sich \"uber mehrere Spalten erstrecken.

      Oracle stellt die unterschiedlichsten Arten von Indizes bereit:
      \begin{itemize}
        \item \textbf{B-Baum} Index: Die meist genutzte Variante (Standard)
        \item \textbf{Reverse key} Index: Werden haupts\"achlich im Oracle Real Application Cluster genutzt
        \item \textbf{Bitmap} Index: Sehr kompakte Variante, die am besten auf Spalten mit sehr wenigen Werten funktioniert
        \item \textbf{Function-based} Index: Enthalten den berechneten Ergebniswert einer Funktion
      \end{itemize}
\clearpage
      Indizes sind logisch und physisch unabh\"angig von den Daten in der Tabelle, an die sie angeh\"angt sind. Deshalb ben\"otigen Sie ihren eigenen Speicherplatz auf dem Datentr\"ager. Ein Index kann erstellt und gel\"oscht werden, ohne das dadurch Tabellen oder andere Speicherstrukturen beeinflusst werden. Ihre Verwaltung geschieht automatisch durch die Datenbank, wenn DML-Operationen auf ihren Basistabellen stattfinden. Auch Anwendungen werden durch das L\"oschen eines Index nicht beeintr\"achtigt, nur der Zugriff auf die Nutzdaten kann  langsamer werden.
      \subsection{B\"aume in der Informatik}
        B\"aume stellen in der Informatik eine wichtige hierarchische Datenstruktur dar, die f\"ur unterschiedlichste Zwecke genutzt werden kann. Das bekannteste Beispiel f\"ur B\"aume sind die \enquote{Verzeichnisb\"aume} eines Dateisystems.
        \begin{center}
          \scalebox{1}{
            \begin{tikzpicture}[->, >=stealth',level/.style={sibling distance = 3cm/#1, level distance = 1.5cm}]
              \node[arn_root] {/}
                child {node[arn_first] {etc}
                  child {node [arn_second] {httpd}}
                  child {node [arn_second] {mail}}
                }
                child {node [arn_first] {home}
                  child {node [arn_second] {bin}}
                }
                child {node [arn_first] {var}
                  child{node[arn_second] {run}}
                }
                child {node[arn_first] {usr}
                  child{node[arn_second]{bin}}
                  child{node[arn_second]{share}}
                };
            \end{tikzpicture}
          }
        \end{center}
        Die Abbildung zeigt den Unix-Verzeichnisbaum, anhand dessen einige Merkmale eines Baumes erkennbar sind.
        \begin{itemize}
          \item Jedes Element eines Baumes wird als \enquote{Knoten} bezeichnet.
          \item Die Verbindungslinien/Pfeile zwischen den Knoten sind \enquote{Kanten}.
          \item Ein Knoten, der keinen Vorg\"anger hat, wird als \enquote{Wurzel} bezeichnet (im Diagramm rot).
          \item Alle Knoten ohne Nachfolger sind \enquote{Bl\"atter} bzw. \enquote{Leafs} (im Diagramm gr\"un)
          \item Knoten, die sowohl einen Vorg\"anger als auch einen Nachfolger, besitzen sind \enquote{Zweige} oder \enquote{Branches} (im Diagramm blau).
        \end{itemize}
        \subsubsection{Wichtige Eigenschaften von B\"aumen am Beispiel des Bin\"arbaumes}
          Es gibt diverse Arten von B\"aumen in der Informatik (z. B. Bin\"arbaum, B-Baum, B$^+$-Baum oder B$^*$-Baum, uvm.). Die einfachste Form ist der Bin\"arbaum. Daher wird dieser hier dazu benutzt werden, um die wichtigsten Grundlagen f\"ur das Verst\"andnis von B\"aumen zu legen.

          Der im Folgenden abgebildete Bin\"arbaum dient als Grundlage f\"ur alle weiteren Erl\"auterungen.
          \begin{center}
            \scalebox{0.8}{
              \begin{tikzpicture}[->, >=stealth',level/.style={sibling distance = 8cm/#1, level distance = 1.5cm}]
                \node[arn_root] {24}
                  child{node[arn_first] {17}
                    child{node[arn_second] {15}
                      child{node[arn_third] {13}}
                      child{node[arn_third] {16}}
                    }
                    child{node[arn_second] {21}
                      child{node[arn_third] {19}}
                      child{node[arn_third] {23}}
                    }
                  }
                  child {node [arn_first] {30}
                    child {node [arn_second] {28}
                      child{node[arn_third] {26}}
                      child{node[arn_third] {29}}
                    }
                    child {node [arn_second] {34}
                      child{node[arn_third] {31}}
                      child{node[arn_third] {38}}
                    }
                  };
              \end{tikzpicture}
            }
          \end{center}
          Wichtige Eigenschaften eines Baumes sind:
          \begin{itemize}
            \item \textbf{Tiefe}: Die Tiefe eines Knotens ist sein Abstand zur Wurzel (Anzahl der Kanten zwischen ihm und der Wurzel).
            \item \textbf{Ebene/Level}: Die Menge aller Knoten der gleichen Tiefe wird als Ebene bzw. Level bezeichnet.
            \item \textbf{Baumh\"ohe}: Die H\"ohe eines Baumes wird durch die maximale Tiefe, die ein Knoten erreichen kann, bestimmt.
            \item \textbf{Vollst\"andigkeit}: Ein Baum gilt als vollst\"andig, wenn alle Ebenen, bis auf die Unterste, komplett gef\"ullt sind. Die unterste Ebene muss von links nach rechts gef\"ullt sein.
          \end{itemize}
          Daraus ergibt sich, f\"ur den Bin\"arbaum, folgendes:
          \begin{itemize}
            \item Die blauen Knoten bilden eine Ebene mit der Tiefe 1.
            \item Die gr\"unen Knoten bilden eine Ebene mit der Tiefe 2.
            \item Die gelben Knoten bilden eine Ebene mit der Tiefe 3.
            \item Die H\"ohe des Baumes ist mit dem Wert 3 anzugeben. Dabei werden alle Ebenen, au\ss{}er der Wurzel gez\"ahlt.
            \item Der Baum ist vollst\"andig, da die blaue Ebene mit zwei Knoten und die gr\"une Ebene mit vier Knoten voll besetzt ist.
            \item B\"aume wachsen in der Informatik von oben nach unten.
          \end{itemize}
          Die folgende Abbildung zeigt einen unvollst\"andigen Bin\"arbaum.
          \begin{center}
            \scalebox{0.8}{
              \begin{tikzpicture}[->, >=stealth',level/.style={sibling distance = 8cm/#1, level distance = 1.5cm}]
                \node[arn_root] {24}
                  child{node[arn_first] {17}
                    child{node[arn_second] {15}
                      child{node[arn_third] {13}}
                      child{node[arn_third] {16}}
                    }
                    child{node[arn_second] {21}
                      child{node[arn_third] {19}}
                      child{node[arn_third] {23}}
                    }
                  }
                  child {node [arn_first] {30}
                    child {node [arn_second] {28}}
                  };
              \end{tikzpicture}
            }
          \end{center}
          In der zweiten Ebene (gr\"un) fehlt ein Element. Sie ist nicht mehr voll besetzt, was bedeutet, dass der Baum unvollst\"andig ist.
        \subsubsection{Elemente und H\"ohe des Bin\"arbaumes}
          Um die Anzahl der Elemente eines beliebigen vollst\"andigen Bin\"arbaumes zu berechnen, m\"ussen verschiedene Informationen bekannt sein:
          \begin{itemize}
            \item In einem Bin\"arbaum hat jeder Knoten h\"ochstens zwei Nachfolger.
            \item Die H\"ohe des Baumes ist f\"ur die Berechnung notwendig.
            \item Die Ebenen werden nummeriert, beginnend mit dem Wert 0 bei der Wurzel.
          \end{itemize}
          Mit Hilfe dieser Informationen kann nun die maximale Anzahl der Elemente des Baumes (hier mit dem Buchstaben $n$ bezeichnet) berechnet werden:
          \begin{center}
            $ n = 2^0 + 2^1 + 2^2 + 2^3$ \\
            $ n = 15$
          \end{center}
          Der Wert 2 in dieser Formel r\"uhrt daher, dass jeder Knoten h\"ochstens zwei Nachfolger haben kann. Die Exponenten 0, 1, 2 und 3 sind die Nummern der Ebenen. Berechnet man die einzelnen Zweierpotenzen ergibt sich:
          \begin{center}
          $ n= 1 + 2 + 4 + 8$
          \end{center}
          Die Wurzelebene hat genau ein Element. Die zweite Ebene hat h\"ochstens zwei, die dritte Ebene h\"ochstens 4 und die Blatt ebene h\"ochstens 8 Elemente. Diese Formel l\"asst sich auf folgende Schreibweise verk\"urzen: $n = 2^{h + 1} - 1$. Denn es gilt:
          \begin{center}
          $ 2^{h+1}-1 = (2^0 + 2^1 + 2^2 + \dots + 2^h) - 1$
          \end{center}
          Der Buchstabe $h$ steht f\"ur die H\"ohe des Baumes.

          Dieses Rechenbeispiel zeigt, dass die H\"ohe des Baumes und die Anzahl der Elemente in einer direkten Beziehung zu einander stehen. Daraus folgt, wenn die Anzahl der Elemente des Bin\"arbaumes bekannt ist, kann die H\"ohe des Baumes errechnet werden. Dies geschieht durch Umstellen der Formel: $n = 2^{h + 1} - 1$.
          \begin{align}
            &2^{h + 1} - 1 &&= n            &&| + 1\\
            &2^{h + 1}     &&= n + 1        &&| log_2\\
            &h + 1         &&= log_2(n + 1) &&| - 1\\
            &h             &&= log_2(n + 1) - 1 && = log_2(n) abgerundet
          \end{align}
          Das $log_2(n + 1) - 1 = log_2(n) abgerundet$ gilt l\"asst sich beweisen:
          \begin{align}
            &n                 &&= 15 \\
            &log_2(15 + 1) - 1 &&= 3 \\
            &log_2(15)         &&= 3.91 (Abrunden!)
          \end{align}
          \begin{merke}
            Mit Hilfe der Formel $log_2(n)$ kann die H\"ohe eines Bin\"arbaumes
            berechnet werden. Dabei ist zu beachten, dass das Ergebnis immer auf
            die ganze Zahl \textbf{abgerundet} werden muss.
          \end{merke}
          Zur Wiederholung: Die H\"ohe eines Baumes wird durch die maximale
          Tiefe, die ein Knoten erreichen kann, bestimmt. Die Tiefe eines
          Knotens ist sein Abstand von der Wurzel (Anzahl der Kanten zwischen
          ihm und der Wurzel).

          Die Zahl 3 aus dem vorangegangenen Beispiel gibt somit die Anzahl der
          Kanten zwischen dem Wurzelknoten und einem Blattknoten an. Damit ist
          aber noch nicht die Frage gekl\"art: \enquote{Wie viele Knoten
          m\"ussen maximal gelesen werden, um ein bestimmtes Objekt
          aufzufinden?}. Die Antwort lautet: Die maximale Anzahl der
          Lesezugriffe ist die H\"ohe des Baumes + 1.

          Dies kann am Beispiel des Knotens mit der Nummer \enquote{19}
          nachgewiesen werden. Um diesen Knoten aufzufinden, m\"ussen die Knoten
          \enquote{24}, \enquote{17} und \enquote{21} gelesen werden.
          Anschlie\ss{}end noch die \enquote{19} selbst. Dies sind vier
          Zugriffe. Die H\"ohe des Baumes wird mit $h = log_2(n) = 3.91$
          berechnet, was abgerundet $3$ ergibt. F\"ur die Anzahl der
          Lesezugriffe muss die Zahl $3.91$ aufgerundet werden, was den Wert $4$
          ergibt.
        \subsubsection{B\"aume als Hilfsmittel zur Suche von Datens\"atzen}
          Um demonstrieren zu k\"onnen, welch wichtige Rolle B\"aume beim
          Auffinden von Datens\"atzen spielen, wird zu aller erst eine Tabelle
          ben\"otigt. Dies soll hier die Tabelle \identifier{Kunde} aus dem
          Schema \identifier{bank} sein. Sie umfasst insgesamt 561 Zeilen.

          Um nun einen bestimmten Datensatz zu finden, gibt es zwei
          unterschiedliche Verfahren:
          \begin{itemize}
            \item \textbf{Full Table Scan}: Das zeilenweise Durchsuchen der
            Tabelle, bis zum Auffinden des richtigen Datensatzes wird als Full
            Table Scan bezeichnet.
            \item \textbf{Index Scan}: Bei einem Index Scan wird der
            Suchbaum/Index nach dem gew\"unschten Datensatz durchsucht.
          \end{itemize}
          Diese beiden Methoden unterscheiden sich wesentlich in der Anzahl der
          Lesevorg\"ange, die zum Auffinden eines Datensatzes ben\"otigt werden.
          Beim Full Table Scan gilt die Formel: $x = \frac{n}{2}$, wobei $x$ die
          Anzahl der Lesevorg\"ange und $n$ die Anzahl der Datens\"atze darstellt.
          F\"ur die Tabelle \identifier{Kunde} bedeutet dies konkret: $x =
          \frac{561}{2} \rightarrow x \approx 280$. Es werden also ca. 280
          Lesevorg\"ange ben\"otigt, um in der Tabelle \identifier{Kunde} einen
          ganz bestimmten Datensatz zu finden.

          Anders sieht dies bei der Nutzung eines Suchbaumes aus. Wie zuvor
          beschrieben, wird in einem Bin\"arbaum die Anzahl der Lesevorg\"ange
          mit der Formel: $log_2(n)$ angegeben, wobei $n$ die Anzahl der
          Datens\"atze darstellt. Bezogen auf die Tabelle \identifier{Kunde}
          bedeutet dies: $x = log_2(561) \rightarrow log_2(561) = 9.13
          \rightarrow x \approx 10$. Mit Hilfe eines Suchbaumes w\"urden also
          nur durchschnittlich 10 Zugriffe, statt der 280 ben\"otigt. Dies
          stellt eine Kostenreduzierung von ca. 96 \% dar.
          \begin{merke}
            In modernen Datenbanken kommen keine Bin\"arb\"aume, sondern
            B*-B\"aume zum Einsatz! Der Bin\"arbaum wurde hier nur als einfaches
            Beispiel gew\"ahlt.
          \end{merke}
      \subsection{B*-Baum Indizes}
        \subsubsection{Aufbau und Funktionsweise}
          B*-B\"aume (gesprochen B-Stern Baum) sind eine vielfach
          weiterentwickelte Variante der Bin\"arb\"aume. Urheber dieser Baumart
          ist Donald Ervin Knuth\footnote{Donal Ervin Knuth: US-Amerikanischer
          Informatiker und Professor an der Stanford University. Autor der
          Buchserie: \enquote{The Art of Computer Programming} und Erfinder des
          Textsatzsystems \enquote{\TeX}}. Einer der wesentlichsten Unterschiede
          zu den Bin\"arb\"aumen ist, dass in einem B*-Baum ein Knoten beliebig
          viele Nachfolger haben kann, nicht mehr nur Zwei. Das bedeutet, dass
          die Kosten f\"ur die Suche eines Datensatzes noch weiter reduziert
          werden, da der Baum sehr viel breiter und damit flacher wird.

          B*-Baum Indizes sind die h\"aufigste Index-Variante in einer
          Datenbank. Sie bestehen aus einer Reihe von Oracle-Bl\"ocken und
          enthalten die indizierten Werte in sortierter Reihenfolge. Jedem Wert
          muss dabei ein Schl\"ussel zugeordnet werden, mit dessen Hilfe der
          Wert gefunden werden kann.

				 Wird beispielsweise die Spalte \identifier{Name} der Tabelle
				 \identifier{Bank} indiziert, legt Oracle die RowID als Wert und den Namen
				 der jeweiligen Bank als Schl\"ussel ab. Da mittels der RowID eine
				 Tabellenzeile direkt aus einer Datendatei gelesen werden kann, bedeutet
				 dies, dass nach dem Auffinden des Schl\"usselwertes direkt die gew\"unschte
				 Zeile gelesen wird.
          \begin{merke}
            Die RowID stellt in einem B*-Baum-Index das Hilfsmittel zum
            Auffinden der Tabellenzeilen dar. Die indizierten Tabellenspalten
            werden als Schl\"ussel benutzt.
          \end{merke}
          \abbildung{b_tree} zeigt beispielhaft, wie ein B*-Baum Index aussehen k\"onnte.
          \bild{Ein B*-Baum Index}{b_tree}{1.2}
          Der Rootnode und die Branchnodes enthalten die Schl\"usselwerte, in
          der Form: von - bis und die Adresse des dazugeh\"origen Indexblockes.
          Zum Beispiel bedeutet 0 -7 0, dass die Schl\"usselwerte null bis
          sieben im Branchnode Nummer null zu finden sind. Die erste Zeile in
          den Branchnodes ist die Indexblocknummer (0, 1 und 2).

          In den Leafnodes stehen dann die indizierten Schl\"usselwerte und die
          jeweilige RowID. Alle Leafnodes sind untereinander verbunden, so dass
          ein direkter Wechsel von einem Leafnode in den n\"achsten stattfinden
          kann, ohne dabei \"uber den zust\"andigen Branchnode gehen zu
          m\"ussen.
        \subsubsection{B*-Tree Indizes erstellen}
          B*-Tree Indizes werden in Oracle mit dem Kommando \languageorasql{CREATE INDEX} erstellt.
          \begin{lstlisting}[caption={Einen B*-Tree Index erstellen},label=admin312,language=oracle_sql]
SQL> CREATE INDEX idx_transactions
  2  ON bank.buchung(buchungs_ID, transaktions_ID)
  3  TABLESPACE bank;
          \end{lstlisting}
          Das SQL-Statement aus \beispiel{admin312} erstellt den Index
          \identifier{idx\_transactions} auf den beiden Spalten
          \identifier{buchungs\_ID} und \identifier{transaktions\_ID}. Die
          \languageorasql{TABLESPACE}-Klausel ist optional und beeinflusst wo
          der Index abgelegt wird.
          \begin{merke}
            Ein Index der auf eine Kombination von zwei oder mehr Spalten gelegt
            wird, wird als \enquote{Composite Index} bezeichnet.
          \end{merke}
          Um sich den soeben erstellten Index n\"aher betrachten zu k\"onnen,
          stellt Oracle die View \identifier{dba\_indexes} bereit.
          \begin{lstlisting}[caption={Der Index unter der Lupe - Die View
          dba\_indexes},label=admin313,language=oracle_sql,alsolanguage=sqlplus]
SQL> col index_name format  a20
SQL> col index_type format  a10
SQL> col uniqueness format  a10
SQL> col blevel format      99
SQL> col leaf_blocks format 9999999
SQL> SELECT index_name, index_type, blevel, leaf_blocks, uniqueness
  2  FROM   dba_indexes
  3  WHERE  index_name LIKE 'IDX_TRANSACTIONS';

INDEX_NAME       INDEX_TYPE  BLEVEL LEAF_BLOCKS UNIQUENES
---------------- ----------- ------ ----------- ---------
IDX_TRANSACTIONS  &NORMAL&           2        1325 NONUNIQUE
          \end{lstlisting}
          Die Spalte \identifier{index\_type} gibt Auskunft dar\"uber, welche Art von Index erstellt wurde. Die Angabe \enquote{normal} besagt, dass es sich um einen B*-Tree Index handelt. Aus den Spalten \identifier{blevel} (B*-Tree-Level) und \identifier{leaf\_blocks} geht hervor, das der Baum eine H\"ohe von 2 hat und 1325 Leafnodes besitzt.

          Die interessanteste Spalte, d\"urfte die Spalte \identifier{uniqueness} sein. Sie zeigt den Wert \enquote{nonunique}, der besagt, dass die Werte im Index nicht eindeutig sind. Da ein B*-Tree nur mit eindeutigen Schl\"usselwerten funktionieren kann, muss Oracle an dieser Stelle einen Trick anwenden.

          Der Trick besteht darin, die RowID als zus\"atzliche Indexspalte mit abzulegen. Das hei\ss{}t, obwohl der Index auf nur zwei Spalten gelegt wurde, wird intern die RowID als dritte Indexspalte gef\"uhrt. Dadurch werden zum einen alle Eintr\"age eindeutig und k\"onnen zum anderen nach der RowID sortiert werden.

          Bei einem Unique Index ist ein solcher Trick nicht n\"otig. Hier werden nur die Schl\"usselspalten gespeichert. Die RowID wird als eigenst\"andiges Attribut zu den Schl\"usselspalten abgelegt.
          \begin{merke}
            Letztlich ist der einzige Unterschied zwischen einem Unique Index und einen Nonunique Index das interne Format, in dem die Daten abgespeichert werden.
          \end{merke}
          Um einen Unique Index zu erstellen, muss dem \languageorasql{CREATE INDEX}-Kommando noch das Schl\"usselwort \languageorasql{UNIQUE} hinzugef\"ugt werden.
          \begin{lstlisting}[caption={Einen B*-Tree Index erstellen},label=admin314,language=oracle_sql]
SQL> CREATE UNIQUE INDEX idx_sozversnr
  2  ON bank.mitarbeiter(sozversnr)
  3  TABLESPACE bank;
          \end{lstlisting}
          Eine Abfrage auf \identifier{dba\_indexes} beweist, dass dieser Index Unique ist.
          \begin{lstlisting}[caption={Ein Unique Index unter der Lupe - Die View dba\_indexes - 2},label=admin315,language=oracle_sql,alsolanguage=sqlplus]
SQL> col index_name format  a20
SQL> col index_type format  a10
SQL> col uniqueness format  a10
SQL> col blevel format      99
SQL> col leaf_blocks format 999999
SQL> SELECT index_name, index_type, blevel, uniqueness
  2  FROM   dba_indexes
  3  WHERE  index_name LIKE 'IDX_SOZVERSNR';

INDEX_NAME       INDEX_TYPE  BLEVEL UNIQUENES
---------------- ----------- ------ ---------
IDX_SOZVERSNR     &NORMAL&           0  &UNIQUE&
          \end{lstlisting}
        \subsubsection{Einen Index f\"ur ein Constraint erstellen}
          Primary Key und Unique Constraints ben\"otigen einen eigenen Unique Index, um arbeitsf\"ahig zu sein. Dieser Index wird, beim Anlegen des Constraints, automatisch mit erstellt. Der DBA hat jedoch die M\"oglichkeit, einen Index selbst anzulegen und diesen mit dem Constraint zu verbinden. Dies ist beispielsweise dann sehr n\"utzlich, wenn der Index, getrennt von der Tabelle, in einem anderen Tablespace liegen soll.
          \begin{lstlisting}[caption={Ein Unique Constraint mit einem Index verbinden},label=admin316,language=oracle_sql]
SQL> ALTER TABLE bank.mitarbeiter
  2  ADD CONSTRAINT sozversnr_uk UNIQUE (sozversnr)
  3  USING INDEX IDX_SOZVERSNR;
          \end{lstlisting}
          Der Quellcode aus den Zeilen eins und zwei ist bereits bekannt. Es wird der Tabelle \identifier{mitarbeiter} ein Unique-Constraint, auf der Spalte \identifier{sozversnr}, hinzugef\"ugt. Die \languageorasql{USING INDEX}-Klausel in Zeile drei, ist nun daf\"ur verantwortlich, dass das Constraint \identifier{sozversnr\_uk} mit dem Unique Index \identifier{idx\_sozversnr} verbunden wird.

          Mit Hilfe dieser Klausel kann ein Index aber auch direkt, zusammen mit dem Constraint angelegt werden.
\clearpage
          \begin{lstlisting}[caption={Ein Unique Constraint zusammen mit einem Index erstellen},label=admin317,language=oracle_sql]
SQL> ALTER TABLE bank.bank
  2  ADD CONSTRAINT bank_name_uk UNIQUE (name)
  3  USING INDEX (
  4    CREATE UNIQUE INDEX bank.bank_name_uk
  5    ON bank.bank(name));
          \end{lstlisting}
          Das die beiden Constraints tats\"achlich mit den gew\"unschten Indizes verbunden sind, kann mittels der View \identifier{dba\_constraints} gepr\"uft werden.
          \begin{lstlisting}[caption={Die View \identifier{dba\_constraints}},label=admin318,language=oracle_sql]
SQL> SELECT constraint_name, constraint_type, index_name
  2  FROM   dba_constraints
  3  WHERE  constraint_name IN ('SOZVERSNR_UK', 'BANK_NAME_UK');

CONSTRAINT_NAME                C INDEX_NAME
------------------------------ - ------------------------------
SOZVERSNR_UK                   U IDX_SOZVERSNR
BANK_NAME_UK                   U IDX_BANK
          \end{lstlisting}
        \subsubsection{Index Clustering Factor}
          Der Index Clustering Factor ist ein Wert, der Auskunft dar\"uber gibt, wie viele Leseoperationen ben\"otigt werden, um die gesuchten Tabellenzeilen zu holen. Im Idealfall zeigen alle RowIDs eines Index-Leafnodes auf genau einen Tabellen block. Dies h\"atte einen Index Clustering Factor von 1 zur Folge. Im schlechtesten Fall, zeigt jede RowID eines Index-Leafnodes auf einen anderen Tabellen block. Dies w\"urde dann mit einem sehr hohen Index Clustering Factor ausgedr\"uckt werden.
          \begin{merke}
            Je h\"oher der Index Clustering Factor, desto schlechter ist die Ausnutzung des Indizes m\"oglich.
          \end{merke}
					In \abbildung{index_clustering_factor_height} ist, anhand der ersten beiden Indexbl\"ocke, ein hoher Index Clustering Factor dargestellt. Nahezu jede RowID zeigt auf einen anderen Tabellen block. Sollten in diesem Zustand die vier Banken aus Index block Nummer 2 (zweiter von links) geholt werden, m\"usste Oracle vier Tabellenbl\"ocke lesen, obwohl alle vier Zeilen auch in einem einzigen gespeichert werden k\"onnten. Der Leseaufwand ist also viermal so hoch, wie n\"otig.
\clearpage
					\bild{Ein hoher Index Clustering Factor}{index_clustering_factor_height}{1.6}

          In \abbildung{index_clustering_factor_low} wird nun ein niedriger Clustering Factor gezeigt. Die ersten drei Indexbl\"ocke ben\"otigen jeweils nur einen Tabellen block, f\"ur alle Tabellenzeilen. Dies ist der theoretische Idealfall.

          \bild{Ein sehr niedriger Index Clustering Factor}{index_clustering_factor_low}{1.6}

          Der Clustering Factor kann mit Hilfe der View
          \identifier{dba\_indexes} abgefragt werden.
          \begin{lstlisting}[caption={Den Index Clustering Factor anzeigen},label=admin319,language=oracle_sql]
SQL> SELECT index_name, clustering_factor
  2  FROM   dba_indexes
  3  WHERE  owner LIKE 'BANK';

INDEX_NAME                     CLUSTERING_FACTOR
------------------------------ -----------------
BANK_PK                                        1
BANKFILIALE_PK                                 1
BANKKUNDE_PK                                   1
BUCHUNGS_PK                                 7857
DEPOT_PK                                       1
					\end{lstlisting}
\clearpage
\begin{lstlisting}[language=oracle_sql]
EIGENKUNDEN_PK                               337
EIGENKUNDEKONTO_PK                           832
EIGENKUNDEMITARBEITER_PK                       1
FREMDKUNDEN_PK                                 1
FREMDKUNDEKONTO_PK                             1
GIROKONTO_PK                                   2

INDEX_NAME                     CLUSTERING_FACTOR
------------------------------ -----------------
KONTO_PK                                      11
KUNDEN_PK                                      4
MITARBEITER_PK                                 2
SPARBUCH_PK                                    1
\end{lstlisting}
          \beispiel{admin319} zeigt das drei der 15 Indizes aus dem \identifier{bank}-Schema einen sehr hohen Clustering Factor haben (\identifier{buchungs\_pk}, \identifier{eigenkunden\_pk} und \identifier{eigenkundekonto\_pk}). Diese drei Indizes sollten neu aufgebaut werden, um den Clustering Factor zu verringern.
      \subsection{Reverse Key Indizes}
        Reverse Key Indizes sind eine Subvariante der B*-Tree Indizes. Ihre Besonderheit ist, dass sie die Bytes der Indexschl\"usselspalten in umgekehrter Reihenfolge speichern. Durch das Speichern der Bytes in umgekehrter Reihenfolge wird erreicht, dass bei Einf\"ugevorg\"angen die neuen Werte \"uber den gesamten Index verteilt werden. Dadurch wird verhindert, das viele Serverprozesse gleichzeitig auf ein und den selben Index block zugreifen, was bei einem normalen B*-Tree Index der Fall w\"are.

        Das folgende Beispiel soll die Funktionsweise, die Vorteile, aber auch die Nachteile von Reverse Key Indizes erl\"autern.

        F\"ur die Tabelle \identifier{Buchung} wird auf der Spalte \identifier{Buchungs\_ID} ein neuer Reverse Key Index angelegt. Der bestehende B*-Tree Index wurde vorher gel\"oscht.
        \begin{lstlisting}[caption={Einen Reverse Key Index anlegen},label=admin320,language=oracle_sql]
SQL> CREATE UNIQUE INDEX IDX_Buchung
  2  ON Buchung(Buchungs_ID)
  3  REVERSE;
        \end{lstlisting}
        \begin{merke}
          Das Schl\"usselwort \languageorasql{REVERSE}, in Zeile 3, macht aus einem normalen B*-Tree Index einen Reverse Key Index.
        \end{merke}
\clearpage
        Nun werden 100 neue Datens\"atze, mit den Buchungs-IDs 469197 bis 469297 in die Tabelle eingef\"ugt. Da es sich bei den Buchungs-IDs um fortlaufende Werte handelt, werden alle in den gleichen Index block eingef\"ugt. Dieses Verhalten kann kritisch sein, wenn sehr viele Nutzer gleichzeitig Datens\"atze in eine Tabelle einf\"ugen. Es entsteht eine gro\ss{}e Konzentration von Schreibzugriffen auf einen Index block, was dazu f\"uhren kann, dass die Nutzer auf einander warten m\"ussen.

        Um ein solches Warten zu verhindern, \enquote{dreht Oracle die Werte einfach um}. Das bedeutet, aus 469197 wird 791964 und aus 469198 wird 891964, usw. Somit werden Werte produziert, die nicht mehr fortlaufend sind und in unterschiedliche Indexbl\"ocke eingetragen werden.

        Der Vorteil ist, eine h\"ohere Geschwindigkeit bei Schreibzugriffen, da die Nutzer nicht auf einander warten m\"ussen. Der Nachteil ist ein extrem hoher Index Clustering Factor, da keine Synchronisation zwischen Index block und Tabellen block erreicht werden kann.
      \subsection{Bitmap Indizes}
        Bitmap Indizes unterscheiden sich von B*-Tree und Reverse Key Indizes dahingehend, dass sie nicht als Baumstruktur, sondern als Bitmap angelegt werden. Ein solcher Index ist immer dann sinnvoll, wenn eine Tabellenspalte eine niedrige Kardinalit\"at hat (wenig unterschiedliche Werte, z. B. \enquote{M} f\"ur m\"annlich und \enquote{W} f\"ur weiblich) und nur wenige \"Anderungen an der Tabelle vorgenommen werden.

        Eine Tabellenspalte, die diese Kriterien erf\"ullt, befindet sich in der Tabelle \identifier{eigenkunde}. Es handelt sich um die Spalte \identifier{staatsangehoerigkeit}. Sie  kennt nur vier Werte: \enquote{Belgisch}, \enquote{T\"urkisch}, \enquote{D\"anisch} und \enquote{Deutsch}.

        Um Suchvorg\"ange auf dieser Spalte zu beschleunigen, wird ein Bitmap Index angelegt.
        \begin{lstlisting}[caption={Einen Bitmap Index anlegen},label=admin321,language=oracle_sql]
SQL> CREATE BITMAP INDEX idx_bm_staatsangehoerigkeit
  2  ON eigenkunde(staatsangehoerigkeit);
        \end{lstlisting}
        Die View \identifier{dba\_indexes} zeigt, dass es sich bei \identifier{idx\_bm\_staatsangehoerigkeit} tats\"achlich um einen Bitmap Index handelt.
        \begin{lstlisting}[caption={Die Eigenschaften eines Bitmap Indexes},label=admin322,language=oracle_sql,alsolanguage=sqlplus]
SQL> col index_name format a30
SQL> col index_type format a15
SQL> SELECT index_name, index_type, blevel, leaf_blocks
  2  FROM   dba_indexes
  3  WHERE  index_name LIKE 'IDX_BM_STAATS%'

INDEX_NAME                     INDEX_TYPE          BLEVEL LEAF_BLOCKS
------------------------------ --------------- ---------- -----------
IDX_BM_STAATSANGEHOERIGKEIT    &BITMAP&                   0           1
        \end{lstlisting}
        Die Angabe \identifier{leaf\_blocks} = 1 sagt aus, dass der Bitmap Index aktuell aus nur einem Index block besteht.
        \begin{merke}
          Auch bei Bitmap Indizes werden die Indexbl\"ocke als Leaf blocks bezeichnet!
        \end{merke}
        N\"utzlich wird dieser Index, wenn z. B. eine Anfrage nach allen
        belgischen Kunden gestellt wird. In diesem Fall benutzt Oracle den
        erstellten Bitmap Index.
        \bild{Benutzung eines Bitmap Index}{use_of_bitmap_index}{0.8}
        \abbildung{use_of_bitmap_index} zeigt die schematische Darstellung eines Bitmap Index. F\"ur jeden Wert der Spalte \identifier{staatsangehoerigkeit} wird eine Bitmap erstellt, die von links nach rechts gelesen werden muss. Jedes Bit stellt eine Tabellenzeile dar. Eine 0 bedeutet, dass in dieser Zeile der betreffende Wert nicht steht. Laut \abbildung{use_of_bitmap_index} steht also in Zeile 1 der Wert \enquote{D\"anisch} (eine 1 hinter d\"anisch). In der zweiten Spalte ist ein t\"urkischer Staatsb\"urger erfasst und in Spalte drei ein Belgier. Oracle kann so sehr schnell und effizient die gew\"unschten Zeilen heraussuchen.
      \subsection{Visibility und Usability}
        \subsubsection{Visibility}
          Seit Oracle 11g ist es m\"oglich einen Index als sichtbar bzw. unsichtbar zu deklarieren. Sichtbare Indizes werden:
          \begin{itemize}
            \item vom SQL-Optimizer bei der Erstellung von Ausf\"uhrungspl\"anen ber\"ucksichtigt,
            \item bei allen DML-Operationen mit gepflegt.
          \end{itemize}
          Dies gibt dem Admin die M\"oglichkeit, die Auswirkung eines Indizes auf einen Ausf\"uhrungsplan zu testen, indem er ihn mit \languageorasql{ALTER INDEX} sichtbar bzw. unsichtbar macht.
          \begin{lstlisting}[caption={Einen Index unsichtbar werden lassen},label=admin323,language=oracle_sql]
SQL> ALTER INDEX IDX_SOZVERSNR INVISIBLE;
          \end{lstlisting}

          \begin{lstlisting}[caption={Und so wird er wieder sichtbar},label=admin324,language=oracle_sql]
SQL> ALTER INDEX IDX_SOZVERSNR VISIBLE;
          \end{lstlisting}
          Die View \identifier{dba\_indexes} hilft dem Admin dabei, zu \"uberpr\"ufen, ob ein Index sichtbar oder unsichtbar ist.
          \begin{lstlisting}[caption={Ist der Index sichtbar oder unsichtbar?},label=admin325,language=oracle_sql]
SQL> SELECT index_name, visibility
  2  FROM   dba_indexes
  3  WHERE  index_name LIKE 'IDX_SOZVERSNR'

INDEX_NAME                     VISIBILITY
------------------------------ ----------
IDX_SOZVERSNR                  &VISIBLE&
          \end{lstlisting}
          Das folgende Beispiel zeigt, was passiert, wenn der Admin einen Index unsichtbar werden l\"asst.

          W\"ahrend der Index \identifier{idx\_sozversnr} sichtbar ist, wird die folgende Abfrage ausgef\"uhrt.
          \begin{lstlisting}[caption={Abfrage mit sichtbarem Index},label=admin326,language=oracle_sql]
SQL> SELECT vorname, nachname
  2  FROM   bank.mitarbeiter
  3  WHERE  sozversNr IN ('5679B983-694-22FA34D','17211682-BA6-D9C0B30');
          \end{lstlisting}
          Das Ergebnis besteht aus zwei Zeilen. Mit Hilfe des \languagesqlplus{set autotrace traceonly}-Kommandos kann der Ausf\"uhrungsplan, f\"ur dieses Statement, angezeigt werden.
          \begin{lstlisting}[caption={Ausf\"urungsplan f\"ur die Abfrage mit sichtbarem Index},label=admin327,language=oracle_sql,alsolanguage=sqlplus]
SQL> set autotrace traceonly
SQL> SELECT vorname, nachname
  2  FROM   Mitarbeiter
  3  WHERE  sozversNr IN ('5679B983-694-22FA34D','17211682-BA6-D9C0B30');

Execution Plan
----------------------------------------------------------
Plan hash value: 2743979585

--------------------------------------------------------------------------------
|Id| Operation                   |Name         |Rows|Bytes|Cost (%CPU)| Time   |
--------------------------------------------------------------------------------
|0 |&SELECT& STATEMENT              |             |  2 |  72 |    2   (0)|00:00:01|
|1 | INLIST ITERATOR             |             |    |     |           |        |
|2 |  &\textbf{\textcolor{red}{TABLE ACCESS BY INDEX ROWID}}&     |MITARBEITER  |  2 |  72 |    2   (0)|00:00:01|
|3 |   &\textbf{\textcolor{red}{INDEX UNIQUE SCAN}}&            |IDX_SOZVERSNR|  2 |     |    1   (0)|00:00:01|
--------------------------------------------------------------------------------

Predicate Information (identified by operation id):
---------------------------------------------------

   3 - access("SOZVERSNR"='17211682-BA6-D9C0B30' &OR&
              "SOZVERSNR"='5679B983-694-22FA34D')
          \end{lstlisting}
\clearpage
          \begin{lstlisting}[caption={Und so wird er wieder sichtbar -
          Fortsetzung},language=terminal]
Statistics
----------------------------------------------------------
          1  recursive calls
          0  db block gets
          &\textbf{\textcolor{red}{4}}&  &\textbf{\textcolor{red}{consistent gets}}&
          0  physical reads
          0  redo size
        666  bytes sent via SQL*Net to client
        524  bytes received via SQL*Net from client
          2  SQL*Net roundtrips to/from client
          0  sorts (memory)
          0  sorts (disk)
          2  rows processed
          \end{lstlisting}
          Interessant am Ausf\"uhrungsplan aus \beispiel{admin327} sind die rot markierten Stellen. Die Angaben \enquote{TABLE ACCESS BY INDEX ROWID} und \enquote{INDEX UNIQUE SCAN} zeigen, dass der Index \identifier{idx\_sozversnr}, f\"ur die Suche der gew\"unschten Ergebniszeilen benutzt wurde. Die Angabe \enquote{4 consistent gets} sagt aus, dass f\"ur die Suche nur vier Lesevorg\"ange ben\"otigt wurden.

          Nun wird der Index unsichtbar geschaltet und die Abfrage erneut ausgef\"uhrt.
          \begin{lstlisting}[caption={Der Index wird unsichtbar und das Statement wird wiederholt},label=admin328,language=oracle_sql]
SQL> ALTER INDEX IDX_SOZVERSNR INVISIBLE;
SQL> SELECT vorname, nachname
  2  FROM   Mitarbeiter
  3  WHERE  sozversNr IN ('5679B983-694-22FA34D','17211682-BA6-D9C0B30');

Execution Plan
----------------------------------------------------------
Plan hash value: 414804864

---------------------------------------------------------------------
|Id| Operation        |Name       |Rows|Bytes|Cost (%CPU)| Time     |
---------------------------------------------------------------------
| 0|&SELECT& STATEMENT   |           |  2 |  72 |    3   (0)| 00:00:01 |
| 1| &\textbf{\textcolor{red}{TABLE ACCESS FULL}}&   |MITARBEITER|  2 |  72 |    3   (0)| 00:00:01 |
---------------------------------------------------------------------

Predicate Information (identified by operation id):
---------------------------------------------------

   1 - filter("SOZVERSNR"='17211682-BA6-D9C0B30' OR
              "SOZVERSNR"='5679B983-694-22FA34D')
          \end{lstlisting}
\clearpage
          \begin{lstlisting}[caption={Der Index wird unsichtbar und das
          Statement wird wiederholt - Fortsetzung},language=terminal]
Statistics
----------------------------------------------------------
        238  recursive calls
          0  db block gets
          &\textbf{\textcolor{red}{56}}&  &\textbf{\textcolor{red}{consistent gets}}&
          0  physical reads
          0  redo size
        666  bytes sent via SQL*Net to client
        524  bytes received via SQL*Net from client
          2  SQL*Net roundtrips to/from client
          6  sorts (memory)
          0  sorts (disk)
          2  rows processed
          \end{lstlisting}
          Der Ausf\"uhrungsplan aus \beispiel{admin328} beweist, dass der unsichtbare Index nicht mehr benutzt wird. Als Suchmethode wurde ein \enquote{Full Table Scan} genutzt, f\"ur den 56 Lesevorg\"ange ben\"otigt wurden.

          Um die Anzeige von Ausf\"uhrungspl\"anen wieder zu deaktivieren, muss in SQL*Plus das Kommando \languagesqlplus{set autotrace off} eingegeben werden.
        \subsubsection{Usability}
          Es ist m\"oglich, einen Index in Oracle nicht nur unsichtbar, sondern auch unbenutzbar werden zu lassen. Der Unterschied ist, dass der unsichtbare Index durch DML-Statements mit gepflegt wird, w\"ahrend der Unbenutzbare nicht gepflegt wird. Beide werden bei der Ausf\"uhrung von \languageorasql{SELECT}-Anfragen nicht ber\"ucksichtigt.

          Ein Index kann durch den Administrator oder die Datenbank selbst als unbenutzbar markiert werden. H\"aufig werden Indizes als unbenutzbar markiert, um gro\ss{}e Data-Load Vorg\"ange zu beschleunigen.
          \begin{lstlisting}[caption={Einen Index als unbenutzbar markieren},label=admin329,language=oracle_sql]
SQL> ALTER INDEX IDX_SOZVERSNR UNUSABLE;
          \end{lstlisting}
          Um zu \"uberpr\"ufen, in welchem Zustand sich ein Index befindet, kann wiederum die View \identifier{dba\_indexes} herangezogen werden.
          \begin{lstlisting}[caption={Pr\"ufen, ob ein Index nutzbar ist},label=admin330,language=oracle_sql]
SQL> SELECT index_name, visibility, status
  2  FROM   dba_indexes
  3  WHERE  index_name LIKE 'IDX_SOZVERSNR'

INDEX_NAME                     VISIBILIT STATUS
------------------------------ --------- --------
IDX_SOZVERSNR                  &INVISIBLE&    &UNUSABLE&
          \end{lstlisting}
          Die Spalte \identifier{status} zeigt, dass sich der Index in einem unbenutzbaren Zustand befindet. Dieser Zustand kann auf zwei Arten korrigiert werden:
          \begin{itemize}
            \item Den Index l\"oschen und neu aufbauen
            \item Den Index reparieren (Index rebuild)
          \end{itemize}
          Den Index zu l\"oschen und neu aufzubauen, w\"urde bedeuten, dass evtl. Constraints (Unique oder Primary Key) mitgel\"oscht und neu erstellt werden m\"ussten. Au\ss{}erdem m\"usste der Admin die genauen Parameter des Indexes kennen, um ihn korrekt neu zu erstellen.

          Bei der Methode \enquote{Index rebuild} wird der noch vorhandene, aber unbenutzbare Index, mit den gespeicherten Parametern neu aufgebaut und Constraints bleiben erhalten.
          \begin{lstlisting}[caption={Einen Index reparieren},label=admin331,language=oracle_sql]
SQL> ALTER INDEX IDX_SOZVERSNR REBUILD ONLINE;
          \end{lstlisting}
          Mit der Angabe \languageorasql{REBUILD ONLINE} wird der Index aufgebaut und die Tabelle bleibt f\"ur SQL-Abfragen offen. Eine erneute Abfrage der View \identifier{dba\_indexes} zeigt, dass der Index wieder benutzbar ist.
          \begin{lstlisting}[caption={Der Index ist wieder benutzbar},label=admin332,language=oracle_sql]
SQL> SELECT index_name, visibility, status
  2  FROM   dba_indexes
  3  WHERE  index_name LIKE 'IDX_SOZVERSNR';

INDEX_NAME                     VISIBILIT STATUS
------------------------------ --------- --------
IDX_SOZVERSNR                  &INVISIBLE&    VALID
          \end{lstlisting}
      \subsection{Indizes l\"oschen}
        Beim L\"oschen eines Indizes werden alle durch den Index belegten Extents frei und k\"onnen wieder verwendet werden.
				
        Wie ein Index gel\"oscht werden kann, h\"angt davon ab, wie er angelegt wurde. Ein Index der explizit angelegt wurde, kann durch das Kommando \languageorasql{DROP INDEX} gel\"oscht werden. Bei einem Index der mit einem Primary Key oder Unique Constraint mit erstellt wurde, muss das Constraint deaktiviert oder gel\"oscht werden, bevor der Index mit \languageorasql{DROP INDEX} gel\"oscht werden kann.
        \begin{lstlisting}[caption={Einen Index L\"oschen},label=admin333,language=oracle_sql]
DROP INDEX idx_sozversnr;
        \end{lstlisting}
    \section{Informationen}
      \subsection{Verzeichnis der relevanten Initialisierungsparameter}
        In diesem Kapitel wurden keine Initialisierungsparameter angesprochen!
      \subsection{Verzeichnis der relevanten Data Dictionary Views}
        \begin{literaturinternet}
          \item \cite{sthref2545}
          \item \cite{sthref2528}
          \item \cite{sthref2202}
          \item \cite{sthref2191}
        \end{literaturinternet}
\clearpage
