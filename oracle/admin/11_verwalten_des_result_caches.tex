  \chapter{Verwalten des Result Caches}
    \setcounter{page}{1}\kapitelnummer{chapter}
    \minitoc
\newpage
    Zusammen mit der Version 11g seiner Datenbank hat Oracle ein neues Feature
    auf den Markt gebracht, dass unter dem Namen Result Cache firmiert. Es
    handelt sich dabei um eine SGA-Komponente, die im Shared Pool angesiedelt
    ist. Ihre Aufgabe ist es, die Arbeit mit h\"aufig wiederkehrenden
    SQL-Statements bzw. PL/SQL-Funktionen zu beschleunigen, in dem die
    Ergebnisse von SQL-Abfragen zwischengespeichert und direkt von dort wieder
    ausgeliefert werden. Der Result Cache unterscheidet sich vom Library Cache,
    da er keine Ausf\"uhrungspl\"ane sondern ganze Result Sets (Ergebniszeilen)
    puffert.

    \begin{merke}
      Um dieses Feature sinnvoll nutzen zu k\"onnen, muss der Datenbankserver \"uber eine gro\ss{}e Menge Arbeitsspeicher verf\"ugen.
    \end{merke}
    \section{Bestandteile des Result Caches}
      \subsection{SQL Query Result Cache}
        Der SQL Query Result Cache ist eine der beiden serverseitigen Komponenten, die den Result Cache bilden. Seine Aufgabe ist die session\"ubergreifende Zwischenspeicherung von SQL-Result Sets. Mit seiner Hilfe reduziert sich die Ausf\"uhrungszeit, eines wiederholten SQL-Statements, meist auf den Bruchteil einer Sekunde. \"Andert sich jedoch etwas an den Basistabellen der Abfrage, wird das im Cache gespeicherte Result Set automatisch invalidiert und das Ergebnis muss neu berechnet werden.
        \begin{merke}
          Optimal einsetzbar ist der SQL Query Result Cache nur bei Tabellen mit geringer Volatilit\"at\footnote{volatilit\"at = \"Anderungsh\"aufigkeit}.
        \end{merke}

        Einmal konfiguriert, steht der SQL Query Result Cache allen Anwendungen transparent zur Verf\"ugung. Das hei\ss{}t, dass sich durch seine Nutzung nichts an den SQL-Statements der Anwendungen \"andert.
      \subsection{Der PL/SQL Function Cache}
        Der PL/SQL-Funktion Cache ist der Bruder des SQL Query Result Caches. Er speichert Ausf\"uhrungsergebnisse von PL/SQL-Funktionen. Damit eine Funktion diesen Cache nutzt, muss sie in ihrer Definition die Klausel \languageplsql{RESULT_CACHE RELIES_ON (tabellenliste)} enthalten.
    \section{Arbeiten mit dem Result Cache}
      \subsection{Aktivieren des Result Caches}
        Es existieren insgesamt 4 Initialisierungsparameter, zur
        Konfiguration des Result Caches.
        \begin{lstlisting}[caption={Die Result Cache Parameter},label=admin700,language=oracle_sql,alsolanguage=sqlplus]
SQL> col name format a30 
SQL> SELECT name, issys_modifiable
  2  FROM   v$system_parameter
  3  WHERE  LOWER(name) LIKE 'result%';

NAME                           ISSYS_MOD
------------------------------ ---------
result_cache_mode              &IMMEDIATE&
result_cache_max_size          &IMMEDIATE&
result_cache_max_result        &IMMEDIATE&
result_cache_remote_expiration &IMMEDIATE&
        \end{lstlisting}
        Aktiviert wird der Result Cache durch das Setzen des Parameters \parameter{result\_cache\_max\_size}, auf einen Wert gr\"o\ss{}er 0.
        \begin{lstlisting}[caption={Den Result Cache aktivieren},label=admin701,language=oracle_sql,alsolanguage=sqlplus]
SQL> col value format a10
SQL> col name format a30
SQL> SELECT name, value
  2  FROM   v$system_parameter
  3  WHERE  LOWER(name) LIKE 'result_cache_max_size';

NAME                           VALUE
------------------------------ ----------
result_cache_max_size          2097152

SQL> ALTER SYSTEM
  2  SET result_cache_max_size = 300M;

System altered.
        \end{lstlisting}
        Um nun zu pr\"ufen, ob der Result Cache tats\"achlich aktiviert wurde, kann die Funktion \identifier{status()} aus dem PL/SQL-Paket \identifier{dbms\_result\_cache} ausgef\"uhrt werden.

        \begin{lstlisting}[caption={Welchen Status hat der Result Cache?},label=admin702,language=oracle_sql,alsolanguage=sqlplus]
SQL> col status format a10
SQL> SELECT DBMS_RESULT_CACHE.STATUS() AS status
  2  FROM   dual;

STATUS
----------
ENABLED
        \end{lstlisting}
        Der \parameter{result\_cache\_max\_size} Parameter beeinflu\ss{}t die Gr\"o\ss{}e des Result Caches. Hat er den Wert 0, ist der Result Cache deaktiviert. Wird das Automatic Memory Management genutzt (Parameter \parameter{memory\_target}), aktiviert Oracle den Result Cache automatisch und es werden bis zu 25 \% von \parameter{memory\_target} f\"ur ihn reserviert. Ist der Parameter \parameter{sga\_target} in Benutzung, wird Oracle bis zu 0,5 \% von \parameter{sga\_target} reservieren. Wird der Shared Pool direkt durch \parameter{shared\_pool\_size} dimensioniert, allokiert Oracle bis zu 1 \% von \parameter{shared\_pool\_size} f\"ur den Result Cache.

        Insgesamt werden aber nie mehr als 75 \% des Shared Pools f\"ur den Result Cache freigegeben.
      \subsection{Konfigurieren des Result Caches}
        \subsubsection{Result\_Cache\_Mode}
          Neben \parameter{result\_cache\_max\_size} existieren noch drei weitere Parameter, die zur Konfiguration des Result Caches dienen. \parameter{result\_cache\_mode} legt fest, ob der Optimizer automatisch versuchen soll, alle Result Sets im Cache zu speichern (Force) oder ob dies bei jedem Statement einzeln festgelegt werden muss (Manual). Der Standardwert f\"ur diesen Parameter ist \enquote{manual}.

          Um ein Result Set manuell im Result Cache abzulegen, muss der Result Cache Hint \languageorasql{/*+ result_cache */}, im SQL-Statement angegeben werden.
          \begin{lstlisting}[caption={Den Result Cache manuell benutzen},label=admin703,language=oracle_sql,alsolanguage=sqlplus]
SQL> set autotrace traceonly explain
SQL> SELECT /*+ result_cache */ vorname, nachname
  2  FROM   bank.mitarbeiter;

Execution Plan
----------------------------------------------------------
Plan hash value: 414804864
--------------------------------------------------------------------------------
|Id| Operation     | Name                     |Rows|Bytes|Cost(%CPU)|Time      |
--------------------------------------------------------------------------------
|0 |&SELECT& STATEMENT |                          |100 |1500 |   3   (0)| 00:00:01|
|1 |&\textcolor{red}{RESULT CACHE}&      |cy4n5s0cg4tbpabs63n3u2k5sa|    |     |          |         |
|2 |&TABLE ACCESS FULL&  | MITARBEITER              |100 |1500 |   3   (0)| 00:00:01|
--------------------------------------------------------------------------------

Result Cache Information (identified by operation id):
------------------------------------------------------

   1 - &column&-count=2; dependencies=(BANK.MITARBEITER);
   name="&SELECT& /*+ result_cache */ vorname, nachname
&FROM&   bank.mitarbeiter"
          \end{lstlisting}
          Dass das Result Set des Statements, aus \beispiel{admin703}, tats\"achlich im SQL Query Result Cache gespeichert wurde, ist aus der zweiten Zeile des Ausf\"uhrungsplanes ersichtlich. Die Angabe: \enquote{RESULT CACHE cy4n5s0cg4tbpabs63n3u2k5sa} zeigt an, dass unter der ID \enquote{cy4n5s0cg4tbpabs63n3u2k5sa}, ein Statement im Result Cache abgelegt wurde. Mit Hilfe dieser ID kann in der View \identifier{v\$result\_cache\_objects} nach dem dazugeh\"orenden SQL-Statement gesucht werden.
          \begin{lstlisting}[caption={Objekte im Result Cache suchen},label=admin704,language=oracle_sql,alsolanguage=sqlplus]
SQL> col name format a35
SQL> col namespace format a9
SQL> col cache_id format a40
SQL> set linesize 100
SQL> SELECT type, name, namespace, row_count, cache_id
  2  FROM   v$result_cache_objects
  3  WHERE  cache_id = 'cy4n5s0cg4tbpabs63n3u2k5sa';

&TYPE&       NAME                                NAMESPACE  ROW_COUNT
---------- ----------------------------------- --------- ----------
CACHE_ID
----------------------------------------
Result     &SELECT& /*+ result_cache */ vorname, SQL              100
            nachname
           &FROM&   bank.mitarbeiter
cy4n5s0cg4tbpabs63n3u2k5sa
          \end{lstlisting}
          Soll der Query Optimizer versuchen alle SQL Result Sets im Result Cache abzulegen, muss der Parameter \parameter{result\_cache\_mode} auf den Wert \enquote{force} eingestellt werden. Ein solches Vorgehen d\"urfte jedoch in nur ganz wenigen F\"allen sinnvoll sein.
        \subsubsection{Result\_Cache\_Max\_Result}
          Mit \parameter{result\_cache\_max\_result} wird die maximale
          Gr\"o\ss{}e eines Result Sets, prozentual von
          \parameter{result\_cache\_max\_size} festgelegt. Statements deren
          Result Sets mehr Speicherplatz ben\"otigen, als 
          \parameter{result\_cache\_max\_result} freigibt, werden nicht im
          Result Cache gespeichert.

          Mit \languageorasql{result_cache_max_result = 5} wurde festgelegt,
          dass ein einzelnes Result Set nicht mehr als 5 \% der
          Gesamtgr\"o\ss{}e des Result Caches (hier 5 \% von 300M = 15M)
          verwenden darf. Der Ausf\"uhrungsplan aus \beispiel{admin705} zeigt,
          dass das Statement 268 MB Arbeitsspeicher ben\"otigen w\"urde. Daher
          wird es nicht im Result Cache aufgenommen.
\clearpage
          \begin{lstlisting}[caption={Maximalgr\"o\ss{}e von Objekten im Result Cache},label=admin705,language=oracle_sql,alsolanguage=sqlplus]
SQL> ALTER SYSTEM
  2  SET result_cache_max_result = 5;

SQL> set autotrace traceonly explain
SQL> SELECT *
  2  FROM   bank.mitarbeiter a, bank.mitarbeiter b, bank.mitarbeiter c;

Execution Plan
----------------------------------------------------------
Plan hash value: 2203449472

--------------------------------------------------------------------------------
|Id| Operation             | Name        |Rows  |Bytes | Cost (%CPU)| Time     |
--------------------------------------------------------------------------------
|0 | &SELECT& STATEMENT      |             | 1000K|  268M| 13707   (1)| 00:02:45 |
|1 |  MERGE &JOIN& CARTESIAN |             | 1000K|  268M| 13707   (1)| 00:02:45 |
|2 |   MERGE &JOIN& CARTESIAN|             |10000 | 1835K|   140   (0)| 00:00:02 |
|3 |    &TABLE ACCESS FULL&    | MITARBEITER |  100 | 9400 |     3   (0)| 00:00:01 |
|4 |    BUFFER SORT       |             |  100 | 9400 |   137   (0)| 00:00:02 |
|5 |     &TABLE ACCESS FULL&   | MITARBEITER |  100 | 9400 |     1   (0)| 00:00:01 |
|6 |   BUFFER SORT        |             |  100 | 9400 | 13705   (1)| 00:02:45 |
|7 |    &TABLE ACCESS FULL&    | MITARBEITER |  100 | 9400 |     1   (0)| 00:00:01 |
--------------------------------------------------------------------------------

          \end{lstlisting}
        \subsubsection{DBMS\_RESULT\_CACHE}
          Das PL/SQL-Paket \identifier{dbms\_result\_cache} enth\"alt eine Reihe von Funktionen zur Administration des Result Caches.
          \begin{itemize}
            \item \identifier{flush()}: L\"oscht den kompletten Inhalt des Result Caches
            \item \identifier{status()}: Zeigt den Status (enabled/disabled) des Result Caches an
            \item \identifier{memory\_report()}: Zeigt einen Nutzungsbericht zum Result Cache an
          \end{itemize}
      \subsection{Einschr\"ankungen bei der Nutzung des Result Caches}
        Ob der Result Cache f\"ur das Result Set eines SQL-Statements genutzt werden kann, ist von verschiedenen Faktoren abh\"angig. Er kann nicht genutzt werden wenn:
        \begin{itemize}
          \item die Abfrage Tabellen des Data Dictionary oder tempor\"are Tabellen enth\"alt,
          \item eine Sequenz im SQL-Statement genutzt wird,
          \item nicht deterministische SQL- oder PL/SQL-Funktionen genutzt werden.
        \end{itemize}
        \begin{merke}
          Der Begriff \enquote{deterministisch} bedeutet, dass das Ergebnis
          einer Funktion, bei gleichbleibenden Eingabeparamtern,
          unver\"anderlich ist. Beispielsweise ist das Ergebnis von
          \languageorasql{SQRT(9)} immer 3, da die Quadratwurzel von 9
          unver\"anderlich ist. Das Ergebnis der Funktion
         \languageorasql{SYSTIMESTAMP}  \"andert sich jedoch mit jeder Nanosekunde. Daher gilt sie als nichtdeterministisch.
        \end{merke}
      \subsection{Monitoring des SQL Query Result Caches}
        F\"ur die \"Uberwachung des Result Caches gibt es zwei Quellen:
        \begin{itemize}
          \item Die View \identifier{v\$result\_cache\_statistics}
          \item Die Funktion \identifier{dbms\_result\_cache.memory\_report()}
        \end{itemize}
        Die View \identifier{v\$result\_cache\_statistics} zeigt eine Kurzform der Ausgabe der Funktion \identifier{dbms\_result\_cache.memory\_report()}.
        \begin{lstlisting}[caption={\identifier{v\$result\_cache\_statistics}},label=admin706,language=oracle_sql]
SQL> SELECT *
  2  FROM   v$result_cache_statistics;

        ID NAME                           VALUE
---------- ------------------------------ ----------
         1 Block Size (Bytes)             1024
         2 Block Count Maximum            307200
         3 Block Count Current            1952
         4 Result Size Maximum (Blocks)   15360
         5 Create Count Success           4
         6 Create Count Failure           0
         7 Find Count                     0
         8 Invalidation Count             0
         9 Delete Count Invalid           1
        10 Delete Count Valid             0
        11 Hash Chain Length              1
        \end{lstlisting}
        Das Ergebnis aus \beispiel{admin706} sagt aus, dass der Result Cache aus
        1 KB gro\ss{}en Bl\"ocken besteht (Block Size (Bytes)) und das er 307200
        Bl\"ocke umfasst (Block Count Maximum). Von diesen 307200 Bl\"ocken sind
        derzeit 1952 allokiert (Block Count Current). Ein Statement darf maximal
        15360 Bl\"ocke umfassen (Result Size Maximum (Blocks)). Der Wert
        \enquote{Create Count Success} sagt aus, dass sich aktuell 4 Statements
        im Cache befinden. Einen etwas ausf\"uhrlicheren \"Uberblick bekommt
        man, wenn man die PL/SQL-Funktion
        \identifier{dbms\_result\_cache.memory\_report()} ausf\"uhrt.
\clearpage
        \begin{lstlisting}[caption={Der Result Cache Report - \identifier{dbms\_result\_cache.memory\_report()}},label=admin707,language=oracle_sql,alsolanguage=sqlplus]
SQL> set serveroutput on
SQL> exec DBMS_RESULT_CACHE.MEMORY_REPORT();
R e s u l t   C a c h e   M e m o r y   R e p o r t
[Parameters]
Block Size          = 1K bytes
Maximum Cache Size  = 300M bytes (300K blocks)
Maximum Result Size = 15M bytes (15K blocks)
[Memory]
Total Memory = 2194328 bytes [0.872% of the Shared Pool]
... Fixed Memory = 10696 bytes [0.004% of the Shared Pool]
... Dynamic Memory = 2183632 bytes [0.868% of the Shared Pool]
....... Overhead = 184784 bytes
....... Cache Memory = 1952K bytes (1952 blocks)
........... Unused Memory = 715 blocks
........... Used Memory = 1237 blocks
............... Dependencies = 1 blocks (1 count)
............... Results = 1236 blocks
................... SQL     = 2 blocks (1 count)
................... Invalid = 1234 blocks (2 count)

PL/SQL procedure successfully completed.
        \end{lstlisting}
        \begin{center}
          \begin{small}
            \label{memoryreportstatistic}
            \tablecaption{\identifier{Memory\_Report} und \identifier{v\$result\_cache\_statistics} im Vergleich}
            \tablefirsthead{%
              \hline
              \multicolumn{1}{|c}{\textbf{\identifier{v\$result\_cache\_statistics}}}&
              \multicolumn{1}{|c|}{\textbf{\identifier{Memory\_Report}}} \\
              \hline
            }
            \tablehead{%
              \hline
              \multicolumn{1}{|c}{\textbf{\identifier{v\$result\_cache\_statistics}}}&
              \multicolumn{1}{|c|}{\textbf{\identifier{Memory\_Report}}} \\
              \hline
            }
            \tabletail{
              \hline
            }
            \begin{supertabular}[h]{|p{6.5cm}|p{8.5cm}|}
              \texttt{Block Size (Bytes) 1024} & \texttt{Block Size = 1K bytes} \\
              \hline
              \texttt{Block Count Maximum 307200} & \texttt{Maximum Cache Size  = 300M bytes (300K blocks)} \\
              \hline
              \multirow{3}{*}{\texttt{Block Count Current 1952}} & \texttt{Cache Memory = 1952K bytes (1952 blocks)} \\
              & \texttt{Unused Memory = 715 blocks} \\
              & \texttt{Used Memory = 1237 blocks} \\
              \hline
              \texttt{Result Size Maximum (Blocks) 15360} & \texttt{Maximum Result Size = 15M bytes (15K blocks)} \\
              \hline
              \multirow{3}{*}{\texttt{Create Count Success 4}} & \texttt{Dependencies = 1 blocks (1 count)} \\
              & \texttt{SQL = 2 blocks (1 count)} \\
              & \texttt{Invalid = 1234 blocks (2 count)} \\
              \hline
              \texttt{Delete Count Invalid 1} & N. a.\\
            \end{supertabular}
          \end{small}
        \end{center}
        \tabelle{memoryreportstatistic} zeigt, dass im Bericht aus
        \beispiel{admin707} die gleichen Werte wiederzufinden sind, wie in der
        View  \identifier{v\$result\_cache\_statistics}.
\clearpage
    \section{Informationen}
      \subsection{Verzeichnis der relevanten Initialisierungsparameter}
        \begin{literaturinternet}
          \item \cite{REFRN10285}
          \item \cite{REFRN10256}
          \item \cite{REFRN10202}
          \item \cite{REFRN10298}
          \item \cite{REFRN10272}
          \item \cite{REFRN10270}
          \item \cite{REFRN10294}
        \end{literaturinternet}
      \subsection{Verzeichnis der relevanten Data Dictionary Views}
        \begin{literaturinternet}
          \item \cite{REFRN30438}
          \item \cite{REFRN30439}
        \end{literaturinternet}
\clearpage