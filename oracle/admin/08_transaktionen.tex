  \chapter{Transaktionen}
    \setcounter{page}{1}\kapitelnummer{chapter}
    \minitoc
\newpage
      Eine Transaktionen ist eine logische Arbeitseinheit, die eines oder mehrere SQL-State\-ments enth\"alt. Transaktionen sind in sich geschlossene Einheiten. Die Ergebnisse aller SQL-Statements einer Transaktionen k\"onnen entweder in die Datenbank \"ubernommen (committed) oder r\"uckg\"angig gemacht (rolled back) werden.

      Eine Transaktion beginnt implizit mit der ersten DML- oder DDL-Anweisung  und endet mit einer der Anweisungen \languageorasql{COMMIT} (\"ubernahme der Daten) oder \languageorasql{ROLLBACK} (r\"uckg\"angig machen), bzw. implizit wenn eine DDL-Anweisung abgesetzt wird (auto commit).

      Um das Konzept einer Transaktion bildlich darzustellen, stelle man sich die Datenbank eines Kreditinstitutes vor. Wenn ein Kunde Geld von seinem eigenen Konto auf ein anderes \"uberweist, geschehen drei verschiedene Dinge:
      \begin{enumerate}
        \item Kontostand des Zahlenden herabsetzen
        \item Kontostand des Zahlungsempf\"angers anpassen
        \item Die Transaktion in einem Journal dokumentieren
      \end{enumerate}
      Die Datenbank muss zwei verschiedene F\"alle abdecken k\"onnen:
      \begin{enumerate}
        \item Alle drei SQL-Statements k\"onnen erfolgreich abgesetzt werden
        \item Durch ein Problem kann mindestens eines der drei Statements nicht korrekt abgesetzt werden (falsche Kontonummer, Hardwarefehler, usw.)
      \end{enumerate}
      Im ersten Fall muss die Datenbank die \"Anderungen der Transaktion in der Datenbank speichern, damit die Bankkonten der Kunden korrekt verwaltet werden. Tritt jedoch wie in Fall zwei ein Fehler auf, muss die gesamte Transaktion zur\"uckgerollt werden.
    \section{Eigenschaften einer Transaktion (ACID)}
      Damit ein transaktionsbasiertes System, funktionieren kann, m\"ussen alle
      Transaktionen grundlegende Eigenschaften aufweisen. Diese k\"onnen durch das Akronym
      \enquote{ACID} beschrieben werden. ACID steht f\"ur \enquote{atomicity},
     \enquote{consistency}, \enquote{isolation} und \enquote{durability}. Im
     Deutschen wird statt ACID auch h\"aufig AKID verwendet.
      \begin{itemize}
        \item \textbf{atomicity} (Atomarit\"at): Eine Transaktion gilt als
        atomar, wenn Sie ganz oder gar nicht ausgef\"uhrt wird. F\"ur den
        Benutzer muss es so aussehen, als w\"are eine Transaktion eine einzelne
        elementare Anweisung, die nicht unterbrochen werden kann. Da die
        einzelnen Anweisungen, aus denen sich eine Transaktion zusammensetzt,
        tats\"achlich aber nacheinander ausgef\"uhrt werden m\"ussen, muss im
        Falle dessen, dass die Transaktion nicht vollst\"andig ausgef\"uhrt
        werden kann, jede einzelne Anweisung wieder r\"uckg\"angig gemacht
        werden.
        \item \textbf{consistency} (Konsistenz): Konsistenz bedeutet, dass sich die Datenbank nach der Aus\-f\"uh\-rung einer Transaktion in einem konsistenten Zustand befinden muss, davon ausgehend, dass die Datenbank auch vor der Transaktion schon konsistent war. F\"ur die Konsistenz einer Datenbank sorgen die Integrit\"ats Constraints, die bei der Ausf\"uhrung einer Transaktion nicht verletzt werden d\"urfen.
        \item \textbf{isolation} (Isolation): Das Prinzip der Isolation
        bedeutet, dass parallel ausgef\"uhrte Transaktionen nicht sich nicht       
        gegenseitig beeinflussen d\"urfen. Umgesetzt wird dies durch
        verschiedene Mechanismen, wie z. B.~Sperren, Zeitstempelverfahren oder,
        im Falle von Oracle, das Multiversioning.
        \item \textbf{durability} (Dauerhaftigkeit): Das Ergebnis einer
        abgeschlossenen Transaktion muss dauerhaft in der Datenbank verf\"ugbar
        sein, auch nach System\-ab\-st\"ur\-zen.
      \end{itemize}
    \section{Transaktionen und ihre Ph\"anomene}
      Die Isolation von Transaktionen ist eine der wesentlichen ACID-Eigenschaften, die an eine Transaktion gestellt werden. Fehlt die Isolation vollst\"andig oder ist diese nur mangelhaft umgesetzt, k\"onnen Probleme bei der Bearbeitung und dem Abfragen von Datens\"atzen auftreten. Diese Ph\"anomene sind im ANSI/ISO SQL-Standard (SQL92) definiert.
      \subsection{Dirty Reads}
        Gr\"unde f\"ur Dirty Reads sind:
        \begin{itemize}
          \item Das DBMS implementiert keine oder nur mangelhafte Isolation f\"ur Transaktionen.
          \item Konkurrierende Lese- und Schreibzugriffe
        \end{itemize}
        \begin{center}
          \tablecaption{Dirty Reads}
          \tablefirsthead{
            \multicolumn{2}{c}{Transaktion 1} &
            \multicolumn{2}{c}{Transaktion 2}\\
            \hline
          }
          \tablehead{}
          \tabletail{}
          \tablelasttail{}
          \begin{supertabular}{lr|ll}
            \small{\texttt{SELECT * FROM employees;}} & \scriptsize{t1} & &\\
            & & \scriptsize{t2} & \small{\texttt{UPDATE employees SET salary= 1000;}}\\
            \small{Anzeigen der Datens\"atze} & & & \\
            \small{mit einem Gehalt von 1000} & \scriptsize{t3} & & \\
            & & \scriptsize{t4} & \small{\texttt{ROLLBACK;}}\\
          \end{supertabular}
        \end{center}
        Das vorangegangene Beispiel zeigt zwei konkurrierende Transaktionen. Transaktion 1 liest die Daten der Tabelle \identifier{employees} zum Zeitpunkt t1. W\"ahrend Transaktion 1 noch liest, beginnt Transaktion 2 zum Zeitpunkt t2 diese Daten zu ver\"andern. Zum Zeitpunkt t3 erfolgt f\"ur Transaktion 1 die Ausgabe der Daten.

				Aufgrund nicht vorhandener Isolation werden auch die noch nicht best\"atigten \"Anderungen von Transaktion 2 mit ausgegeben. Zum Zeitpunkt t4 macht die Transaktion 2 ihr \languageorasql{UPDATE}-Statement wieder r\"uckg\"angig.

        Die Schlussfolgerung aus diesem Szenario zeigt, dass Transaktion 1 zum
        Zeitpunkt t3 nicht committete Daten ausgegeben hat.
        \begin{merke}
          Dieses Szenario kann in Oracle nicht durchgef\"uhrt werden!
        \end{merke}
      \subsection{Non-Repeatable Reads}
        Non-Repeatable Reads sind ein Phänomen, dass immer dann auftritt, wenn:
        \begin{itemize}
          \item das DBMS keine oder nur mangelhafte Isolation f\"ur Transaktionen implementiert
          \item zwei gleiche Lesevorg\"ange \textit{eines Datensatzes} in einer Transaktion unterschiedliche Ergebnisse liefern.
        \end{itemize}
        \begin{center}
          \tablecaption{Non-Repeatable Reads}
          \tablefirsthead{
            \multicolumn{2}{c}{Transaktion 1} &
            \multicolumn{2}{c}{Transaktion 2}\\
            \hline
          }
          \tablehead{
          }
          \tabletail{}
          \begin{supertabular}{lr|ll}
            \small{\texttt{SELECT salary FROM employees}} & & & \\
            \small{\texttt{WHERE employee\_id = 100;}} & \scriptsize{t1} & & \\
            \small{Anzeigen des Datensatzes (24000)} & \scriptsize{t2} & & \\
            & & \scriptsize{t3} & \small{\texttt{UPDATE employees SET salary = 25000}} \\
            & & & \small{\texttt{WHERE employee\_id = 100;}} \\
            & & & \texttt{COMMIT;} \\
            \cline{3-4}
            \small{\texttt{SELECT salary FROM employees}} & & & \\
            \small{\texttt{WHERE employee\_id = 100;}} & \scriptsize{t4} & & \\
            \small{Anzeigen des Datensatzes (25000)} & \scriptsize{t5} & & \\
          \end{supertabular}
        \end{center}
        Transaktion 1 f\"uhrt in diesem Beispiel zweimal den gleichen
        Lesevorgang durch. Dieser liefert zum Zeitpunkt t2 ein Gehalt von 24.000
        \$. Bei Zeitpunkt t3 ver\"andert Transaktion 2 die Basistabelle. Der
        erneute Lesevorgang liefert zum Zeitpunkt t5  ein Gehalt von
        25.000~\$. Der gleiche Lesevorgang \textit{ein und des selben
        Datensatzes} konnte also nicht zweimal mit dem gleichen Ergebnis
        durchgef\"uhrt werden, was als Non-Repeatable Read bezeichnet wird.
\clearpage
      \subsection{Phantom Reads}
        Phantom Reads treten immer dann auf, wenn:
        \begin{itemize}
          \item das DBMS nicht die h\"ochst m\"ogliche Isolation f\"ur Transaktionen implementiert und
          \item zwei gleiche Lesevorg\"ange in einer Transaktion eine \textit{unterschiedliche Menge an Ergebniszeilen} liefern. Dies bedeutet, dass zu den bereits gelesenen Zeilen neue hinzugekommen oder bestehende weggefallen sind, sich also die Menge der gelesenen Zeilen ver\"andert hat.
        \end{itemize}
        \begin{center}
          \tablecaption{Phantom Reads}
          \tablefirsthead{
            \multicolumn{2}{c}{Transaktion 1} &
            \multicolumn{2}{c}{Transaktion 2}\\
            \hline
          }
          \tabletail{}
          \begin{supertabular}{lr|ll}
            \small{\texttt{SELECT * FROM employees}} & & & \\
            \small{\texttt{WHERE department\_id = 30;}} & \scriptsize{t1} & & \\
            \small{Anzeigen der Datens\"atze: 6 St\"uck} & \scriptsize{t2} & & \\
            & & & \small{\texttt{INSERT INTO employees}} \\
            & & \scriptsize{t3} & \small{\texttt{VALUES (...);}}\\
            & & & \texttt{COMMIT;} \\
            \cline{3-4}
            \small{\texttt{SELECT * FROM employees}} & & & \\
            \small{\texttt{WHERE department\_id = 30;}} & \scriptsize{t4} & & \\
            \small{Anzeigen der Datens\"atze: 7 St\"uck} & \scriptsize{t5} & & \\
          \end{supertabular}
        \end{center}
        Transaktion 1 f\"uhrt in diesem Beispiel zweimal den gleichen Lesevorgang aus. Lesevorgang 1 liefert zum Zeitpunkt t2 6 Zeilen. Zum Zeitpunkt t3 ver\"andert Transaktion 2 diese Tabelle. Der erneute Lesevorgang von Transaktion 1 liefert jetzt, zum Zeitpunkt t5 ein ver\"andertes Ergebnis. Es sind nun 7 Zeilen. Der gleiche Lesevorgang konnte also \textit{nicht zweimal mit der gleichen Ergebnismenge} durchgef\"uhrt werden.
        \begin{merke}
          Der Unterschied zwischen Non-Repeatable Reads und Phantom Reads ist der, dass bei den Non-Repeatable Reads bestehende Datens\"atze ver\"andert werden. Dadurch kann sich auch die Ergebnismenge \"andern. Bei den Phantom Reads werden neue Datens\"atze hinzugef\"ugt oder bestehende gel\"oscht. Auch hier wird die Ergebnismenge ge\"andert, aber auf eine andere Art.
        \end{merke}
    \section{Transaktionslevel}
      Um die beschriebenen Transaktionsph\"anomene zu umgehen, wurden im ANSI/ISO SQL-Standard (SQL92) vier verschiedene Transaktionslevel festgelegt. Diese Level legen unterschiedliche Einschr\"ankungen fest, um die genannten Ph\"anomene zu unterdr\"ucken.
\clearpage
      \begin{small}
        \tablecaption{Transaktionslevel gem\"a\ss{} SQL92-Standard}
        \tablefirsthead{
          \hline
          \multicolumn{1}{|c}{\textbf{Isolationslevel}} &
          \multicolumn{1}{|c}{\textbf{Dirty Reads}} &
          \multicolumn{1}{|c}{\textbf{Non-Repeatable Reads}} &
          \multicolumn{1}{|c|}{\textbf{Phantom Reads}}\\
          \hline
        }
        \begin{supertabular}{|l|c|c|c|}
          Read Uncommitted & \textcolor{red}{Ja} & \textcolor{red}{Ja} & \textcolor{red}{Ja} \\
          \hline
          Read Committed & \textcolor{green}{Nein} & \textcolor{red}{Ja} & \textcolor{red}{Ja} \\
          \hline
          Repeatable Read & \textcolor{green}{Nein} & \textcolor{green}{Nein} & \textcolor{red}{Ja} \\
          \hline
          Serializable & \textcolor{green}{Nein} & \textcolor{green}{Nein} & \textcolor{green}{Nein}  \\
          \hline
        \end{supertabular}
      \end{small}
      \subsection{Read Uncommitted}
        Dies ist der Transaktionslevel mit den geringsten Einschr\"ankungen (es gibt n\"amlich keine). Es findet keinerlei Isolation statt, so dass eine Transaktion unbest\"atigte Informationen einer anderen Transaktion lesen kann. Dieser Level ist in Oracle nicht implementiert!
      \subsection{Read Committed}
        Diese Stufe bringt die ersten Einschr\"ankungen mit sich. Es k\"onnen nur noch best\"atigte Informationen anderer Transaktionen gelesen werden. Eine Transaktion wird also lediglich vor fehlerhaften Daten einer anderen Transaktion gesch\"utzt, da fehlerhafte Daten meist zur\"uckgerollt werden. Dieser Level ist der Standard in Oracle.
      \subsection{Repeatable Read}
        Durch eine verbesserte Isolation der Transaktionen wird in diesem Level sichergestellt, dass auch das Ph\"anomen der Non-Repeatable Reads verhindert werden kann. Dieser Level ist in Oracle nicht implementiert!
      \subsection{Serializable}
        Serializable ist der strengste Transaktionslevel. Er verhindert jegliche Trans\-akt\-ions\-ph\"a\-no\-me\-ne. Er tut dies allerdings auf Kosten der Performance, da, wie sein Name sagt, eine Serialisierung der einzelnen Transaktionen durchgef\"uhrt wird, d. h. Oracle versucht zwei parallel ausgef\"uhrte Transaktionen so auszuf\"uhren, als w\"urden sie hintereinander ausgef\"uhrt.

        Zus\"atzlich zu den Transaktionsleveln die der SQL92-Standard festgelegt hat, kennt Oracle noch einen weiteren Level, den Read Only Level. Wird eine Transaktion in diesem Level gestartet, kann sie nur Abfragen, aber kein DML Statement durchf\"uhren. Er hat die gleichen Auswirkungen wie der Level Serializable und ist f\"ur sehr lange laufende Abfragen gedacht, die ein hohes Ma\ss{} an Lesekonsistenz ben\"otigen.

        \bild{Se\-ri\-a\-li\-sie\-rung von Trans\-ak\-tio\-nen}{serializable}{1.2}

    \section{Transaktionssteuerung}
      \subsection{Eine Transaktion starten}
        Eine Transaktion kann auf zwei verschiedene Arten gestartet werden: implizit oder explizit. Implizit wird eine Transaktion durch ein beliebiges DML-Kommando gestartet. D. h. eine Transaktion beginnt implizit, sobald der Nutzer ein DML-Statement abgesetzt hat.

        Um das Transaktionslevel einer Transaktion zu setzen, kann das
        \languageorasql{SET TRANSACTION ISOLATION LEVEL}-Kommandos benutzt
        werden. Die drei m\"oglichen Isolationslevel werden wie folgt gesetzt:
        \begin{lstlisting}[caption={Isolationslevel einer Transaktion
        w\"ahlen},label=admin401,language=oracle_sql]
SET TRANSACTION ISOLATION LEVEL READ COMMITTED;

SET TRANSACTION ISOLATION LEVEL SERIALIZABLE;

SET TRANSACTION READ ONLY;
        \end{lstlisting}
        Soll das Transaktionslevel für eine gesamte Session geändert werden,
        geschieht dies mit \languageorasql{ALTER SESSION}.
\clearpage
        \begin{lstlisting}[caption={Isolationslevel einer Session w\"ahlen},label=admin402,language=oracle_sql]
ALTER SESSION SET isolation_level = READ COMMITTED;

ALTER SESSION SET isolation_level = SERIALIZABLE;
        \end{lstlisting}
      \subsection{Eine Transaktion beenden}
        Eine Transaktion kann auf unterschiedliche Art und Weise beendet werden:
        \begin{itemize}
          \item Durch das Kommando \languageorasql{COMMIT}.

          Ein \languageorasql{COMMIT} sorgt daf\"ur, dass eine Transaktion beendet und ihre \"Anderungen in der Datenbank dauerhaft gemacht werden. Ein \languageorasql{COMMIT} kann nur dann erfolgreich sein, wenn keine Verletzung der Datenkonsistenz vorliegt.
          \item Durch das Kommando \languageorasql{ROLLBACK}.

          Mit \languageorasql{ROLLBACK} werden alle \"Anderungen, die eine Transaktion an einer Datenbank vorgenommen hat, r\"uckg\"angig gemacht. Die Datenbank wird in den letzten konsistenten Zustand zur\"uckversetzt.
          \item Durch das Abbrechen einer Session.

          Wird eine Session unerwartet abgebrochen, werden alle offnen Transaktionen der Session beendet. Es erfolgt kein \languageorasql{COMMIT}.
        \end{itemize}
      \subsection{Transaktionsteile zur\"uckrollen}
        Innerhalb einer Transaktion k\"onnen Marken gesetzt werden, die als Savepoints bezeichnet werden. Dadurch wird eine Transaktion in einzelne Teile zerlegt. Der Nutzer hat so die M\"oglichkeit eine Transaktion nur teilweise, bis zu einem bestimmten Savepoint, zur\"uckzurollen. Dies kann bei langen und komplexen Transaktion sehr n\"utzlich sein.

        Wird ein Rollback zu einem Savepoint durchgef\"uhrt, hebt Oracle nur die Sperren auf, die f\"ur die zur\"uckgerollten Statements notwendig waren. Die Transaktion bleibt, trotz der teilweisen Rollbacks, erhalten. Andere Transaktionen, die Zugriff auf die bisher gesperrten Daten ben\"otigen, k\"onnen dann mit ihrer Arbeit fortfahren.

        \begin{lstlisting}[caption={Einen Savepoint setzen},label=admin403,language=oracle_sql]
UPDATE departments
SET    department_ID = 11
WHERE  department_ID = 10;

SAVEPOINT dept;
        \end{lstlisting}
        \begin{lstlisting}[caption={Rollback zu einem Savepoint},label=admin404,language=oracle_sql]
        
UPDATE departments
SET    department_ID = 21
WHERE  department_ID = 20;

SAVEPOINT dept2;

ROLLBACK TO SAVEPOINT dept;
        \end{lstlisting}
        \begin{literaturinternet}
          \item \cite[Konkurrierende Zugriffe und Datenkonsistenz]{dataconcurrencyandconsistency}.
          \item \cite[Transaktionsverwaltung]{transactionmanagement}.
        \end{literaturinternet}
      \subsection{Deadlocks}
        Ein Deadlock tritt immer dann auf, wenn zwei oder mehr Nutzer Sperren auf ein und die selbe Ressource legen m\"ochten. Deadlocks verhindern, dass die betreffenden Transaktionen weiterarbeiten k\"onnen. Diese Situation wird deshalb als Deadlock bezeichnet, weil es egal ist, wie lange jede Transaktion warten w\"urde, da jeder auf den anderen wartet.

        Oracle kann automatisch Deadlock-Situationen erkennen und aufl\"osen. Dies geschieht, in dem die Transaktion, die den Deadlock bemerkt auf Statementebene zur\"uckgerollt wird (Statement-level rollback). So werden die betreffenden Sperren freigegeben und die andere Transaktion kann weiter arbeiten.

        \begin{literaturinternet}
          \item \cite[Wie Oracle Daten sperrt]{howoraclelocksdata}
        \end{literaturinternet}
    \section{Multiversion Concurrency Control}
      Multiversion Concurrency Control ist ein Mechanismus, den Oracle zur Ge\-w\"ahr\-leis\-tung der Lesekonsistenz nutzt. Dabei werden verschiedene Versionen von Datenbankobjekten aufbewahrt, so dass jeder Nutzer die ben\"otigte Sicht seiner Daten bekommt. Die Umsetzung dieses Verfahrens wird durch verschiedene Methoden, wie z. B. Zeitstempelverfahren oder Snapshots erreicht. Oracle benutzt hierf\"ur sogenannte Before Images seiner Datenbl\"ocke.
\clearpage
      \subsection{Lesekonsistenz}
        \subsubsection{Lesekonsistenz auf Statementebene (Statementlevel Read Consistency)}
          \begin{merke}
            Unter dem Begriff \enquote{Statementlevel Read Consistency} versteht man, dass eine Abfrage nur die Daten sieht, die zum Startzeitpunkt der Abfrage g\"ultig (committed) waren. Die L\"ange der Laufzeit der Abfrage darf dabei keine Rolle spielen.
          \end{merke}
          \tablecaption{Statementlevel Read Consistency}
          \tablefirsthead{
            \multicolumn{2}{c}{Transaktion 1} &
            \multicolumn{2}{c}{Transaktion 2}\\
            \hline
          }
          \tabletail{}
          \begin{supertabular}{lr|ll}
            \label{statementlevelreadconsistency}
            \small{\texttt{SELECT * FROM employees}} & & & \\
            \small{\texttt{WHERE department\_id = 30;}} & \scriptsize{t1} & & \\
            & & & \small{\texttt{DELETE employees}} \\
            & & \scriptsize{t2} & \small{\texttt{WHERE employee\_id = 117;}}\\
            & & & \texttt{COMMIT;} \\
            \cline{3-4}
            \small{Anzeigen der Datens\"atze: 6 St\"uck} & \scriptsize{t3} & & \\
            \small{\texttt{SELECT * FROM employees}} & & & \\
            \small{\texttt{WHERE department\_id = 30;}} & \scriptsize{t4} & & \\
            \small{Anzeigen der Datens\"atze: 5 St\"uck} & \scriptsize{t5} & & \\
          \end{supertabular}

          Das \beispiel{statementlevelreadconsistency} zeigt die beiden Transaktionen 1 und 2. Transaktion 1 startet zum Zeitpunkt t1. Unmittelbar nach dem Start von Transaktion 1 startet Transaktion 2 zum Zeitpunkt t2. Der Delete-Vorgang von Transaktion 2 darf die Abfrage von Transaktion 1 nicht beeinflussen und die Ausgabe der Abfrageergebnisse zum Zeitpunkt t3 zeigt auch korrekte 6 Datens\"atze an. Erst die zum Zeitpunkt t4 gestartete Abfrage in Transaktion 1 registriert die durch Transaktion 2 vorgenommenen \"Anderungen. Die Lesekonsistenz auf Statementebene ist gew\"ahrleistet.
        \subsubsection{Lesekonsistenz auf Transaktionsebene (Transactionlevel Read Consistency)}
          \begin{merke}
            Unter dem Begriff \enquote{Transactionlevel Read Consistency} versteht man, dass eine Abfrage nur die Daten sieht, die zum Startzeitpunkt der Transaktion g\"ultig (committed) waren. Die L\"ange der Laufzeit der Transaktion darf dabei keine Rolle spielen. Um Transactionlevel Read Consistency zu erwirken, muss das Isolationslevel Serializable verwendet werden.
          \end{merke}
\clearpage
          Das \beispiel{transactionlevelreadconsistency} zeigt die beiden Transaktionen 1 und 2. Transaktion 1 startet zum Zeitpunkt t1. Unmittelbar nach dem Start von Transaktion 1 startet Transaktion 2 zum Zeitpunkt t2. Der Delete-Vorgang von Transaktion 2 darf die gesamte Transaktion 1 nicht beeinflussen und die beiden Ausgaben der Abfrageergebnisse, zu den Zeitpunkten t4 und t6, zeigen auch korrekte 6 Datens\"atze an. Die Lesekonsistenz auf Transaktionsebene ist gew\"ahrleistet.
          \tablecaption{Transactionlevel Read Consistency}
          \tablefirsthead{
            \multicolumn{2}{c}{Transaktion 1} &
            \multicolumn{2}{c}{Transaktion 2}\\
            \hline
          }
          \tabletail{}
          \begin{supertabular}{lr|ll}
            \label{transactionlevelreadconsistency}
            \small{\texttt{SET TRANSACTION ISOLATION LEVEL}} & & & \\
            \small{\texttt{SERIALIZABLE;}} & \scriptsize{t1} & & \\
            \small{\texttt{SELECT * FROM employees}} & & & \\
            \small{\texttt{WHERE department\_id = 30;}} & \scriptsize{t2} & & \\
            & & & \small{\texttt{DELETE employees}} \\
            & & \scriptsize{t3} & \small{\texttt{WHERE employee\_id = 117;}}\\
            & & & \texttt{COMMIT;} \\
            \cline{3-4}
            \small{Anzeigen der Datens\"atze: 6 St\"uck} & \scriptsize{t4} & & \\
            \small{\texttt{SELECT * FROM employees}} & & & \\
            \small{\texttt{WHERE department\_id = 30;}} & \scriptsize{t5} & & \\
            \small{Anzeigen der Datens\"atze: 6 St\"uck} & \scriptsize{t6} & & \\
          \end{supertabular}

      \subsection{Undo-Segmente}
        Um die Lesekonsistenz f\"ur Abfragen zu gew\"ahrleisten, benutzt Oracle Undo-Segmente. Undo-Segmente sind spezielle Segmente, die anders als z. B.~Tabellensegmente oder Clustersegmente, nicht direkt durch den Nutzer bearbeitet werden k\"onnen. Seit Oracle 9i sind alle Undo-Segmente in einem Undo-Tablespace zusammengefasst, der nur minimale Administration ben\"otigt.

        Undo-Segmente werden von der Datenbank genutzt, um Before Images von Datenbl\"ocken zu speichern.
        \subsubsection{Before Images}
          Unter dem Begriff Before Image versteht Oracle eine Kopie eines Oracle blocks, bevor dieser ge\"andert wird. Ein Before Image kann im weitesten Sinne als eine \enquote{Kopie der Originalwerte} verstanden werden.
\clearpage
          Before Images werden f\"ur verschiedene Zwecke ben\"otigt. Dies sind im wesentlichen:
          \begin{itemize}
            \item Zur\"uckrollen einer Transaktion, wenn ein \languageorasql{ROLLBACK}-Kommando ausgef\"uhrt werden soll
            \item Recovery der Datenbank
            \item Lesekonsistenz
            \item Oracle Flashback Query
            \item Oracle Flashback Table
          \end{itemize}
          Wird ein \languageorasql{ROLLBACK}-Kommando am Ende einer Transaktion abgesetzt, m\"ussen alle durch die Transaktion verursachten \"Anderungen r\"uckg\"angig gemacht werden. Hierzu werden die Originalwerte aus den Before Images, die in den Undo-Segmente liegen, benutzt.

          Zu beachten ist, dass es mehrere Before Images eines Oracle blocks, mit unterschiedlichen Versionsst\"anden geben kann. Da jeder Oracle block im Database Buffer Cache mit einer SCN versehen wird, kann anhand dieser das Alter des Blocks (und somit seine Version) bestimmt werden. Je h\"oher die SCN, desto neuer ist der Block.

          \subsubsection{Multiversion Concurrency Control durch Before Images}
            \abbildung{readconsistency} zeigt die Nutzung der Before Images f\"ur das Erzeugen von Lesekonsistenz.

            \bild{Lesekonsistenz durch Multiversion Concurrency Control}{readconsistency}{0.7}

            Eine Transaktion wird bei SCN 4711 gestartet. Um Lesekonsistenz zu gew\"ahrleisten, muss die Datenbank daf\"ur sorgen, dass diese Transaktion nur solche Datenbl\"ocke liest, deren SCN kleinergleich 4711 lautet. Ein Oracle block tr\"agt bereits die SCN 4866. F\"ur diesen muss Oracle ein Before Image, mit einer SCN kleiner oder gleich 4711, aus den Undo-Segmenten holen.
