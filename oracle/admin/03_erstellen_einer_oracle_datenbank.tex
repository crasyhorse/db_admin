  \chapter{Erstellen einer Oracle Datenbank}
    \setcounter{page}{1}\kapitelnummer{chapter}
    \minitoc
\newpage
    \section{\"Uberlegungen vor der Erstellung}
      Eine Oracle Datenbank kann auf zwei verschiedene Arten erstellt werden:
      \begin{itemize}
        \item \textbf{Database Configuration Assistant (DBCA)}: Der DBCA kann w\"ahrend der Installation durch den OUI oder sp\"ater von Hand gestartet werden. Er stellt ein grafisches Interface f\"ur die Erstellung von Datenbanken zur Verf\"ugung.
        \item \textbf{CREATE DATABASE}: Das SQL-Kommando \languageorasql{CREATE DATABASE} bietet die M\"oglichkeit zur manuellen Erstellung einer Datenbank. Wird eine Datenbank auf diesem Weg erstellt, anstatt mit dem DBCA, m\"ussen zus\"atzliche Schritte ausgef\"uhrt werden, bis die Datenbank voll funktionsf\"ahig ist.
      \end{itemize}
      Bevor aber eine dieser beiden M\"oglichkeiten genutzt werden kann, muss zuerst einiges an Vorarbeit geleistet werden. Die folgenden \"Uberlegungen sollen dabei helfen, eine Oracle Datenbank planvoll zu erstellen.
      \subsection{Kapazit\"atenplanung}
        F\"ur jedes Datenbankobjekt, wie z. B. eine Tabelle, kann anhand der verwendeten Datentypen, der Spaltenanzahl und der gesch\"atzten maximal Anzahl Zeilen eine durchschnittliche Gr\"o\ss e berechnet werden. Um den Speicherbedarf einer Datenbank realistisch einsch\"atzen zu k\"onnen sollte dies f\"ur jedes Objekt geschehen. Folgendes Beispiel verdeutlicht die Kapazit\"atsplanung einer einzelnen Tabelle.
          \begin{verbatim}
Tabelle: tblKunden
Spalten:
  1.)    KundenID   NUMBER(4)           3 Byte
  2.)    Name       VARCHAR2(30)       30 Byte
  3.)    Vorname    VARCHAR2(25)       25 Byte
  4.)    Strasse    VARCHAR2(50)       50 Byte
  5.)    PLZ        VARCHAR2(5)         5 Byte
  6.)    Ort        VARCHAR2(30)       30 Byte
Voraussichtliche Anzahl Zeilen   : 300
          \end{verbatim}
          Die Gesamtgr\"o\ss e der Tabelle \identifier{tblKunden} wird somit wie folgt berechnet:

          \centerline{$300*(3+30+25+50+5+30)=42.900$ Byte}

      \subsection{Planen des physischen Datenbanklayouts}
        Unter dem physischen Layout einer Oracle Datenbank versteht man die Aufteilung der selben in einzelne Dateien und die Verteilung dieser Dateien auf verschiedene Datentr\"ager. Der Einsatz von Mirroring- und Striping-Mechanismen wird ebenfalls zum physischen Datenbanklayout gez\"ahlt. Eine vern\"unftige Planung kann sich entscheidend auf Performance und Ausfallsicherheit der Datenbank auswirken.
        \subsubsection{Oracle Managed Files (OMF) und Automatic Storage Management (ASM)}
          OMF und ASM sind zwei Mechanismen, die den Administrator bei der Verwaltung einer Datenbank unterst\"utzen sollen. Beide sind aber nicht zwingend notwendig f\"ur den Betrieb.
        \subsubsection{Ausw\"ahlen des globalen Datenbanknamens}
          Der globale Datenbankname ist der Name der Datenbank, der sie eindeutig identifiziert, z. B. \enquote{ORCL}.
        \subsubsection{Initialisierungsparameter}
          Die Initialisierungsparameter beeinflussen das Verhalten einer Oracle-Instanz. Eine ung\"unstige Konfiguration kann sich negativ auf die Performance der Datenbank auswirken.
        \subsubsection{Der Datenbankzeichensatz}
          Je nach dem welcher Zeichensatz f\"ur die Datenbank ausgew\"ahlt wurde kann diese verschiedene Sprachen darstellen und andere nicht. Dieser Punkt gewinnt im multinationalen Betrieb eine erh\"ohte Bedeutung. F\"ur die Auswahl des richtigen Datenbankzeichensatzes sind die folgenden Punkte entscheidend:
          \begin{itemize}
            \item Welche Sprachen sollen in der Datenbank dargestellt werden (jetzt und in Zukunft)?
            \item Ist der Zeichensatz im verwendeten Betriebssystem verf\"ugbar?
            \item Welchen Zeichensatz nutzen die Clients?
          \end{itemize}

          \begin{literaturinternet}
            \item \cite{NLSPG002}
          \end{literaturinternet}

        \subsubsection{Ausw\"ahlen der korrekten Zeitzone(n)}
          Die gew\"ahlte Zeitzone ist dann entscheidend, wenn Clients in unterschiedlichen Zeitzonen stehen. Sie ist daf\"ur zust\"andig, eine korrekte Umrechnung der Datum/Uhrzeit-Angaben des Clients in die Zeitzone des Servers sicherzustellen.

          \begin{literaturinternet}
            \item \cite{i1006705}
          \end{literaturinternet}

        \subsubsection{Festlegen der Oracle Datenblockgr\"o\ss e}
          Die Standard Oracle Blockgr\"o\ss e betr\"agt 8 Kb. Abh\"angig vom verwendeten Dateisystem und dem Verwendungszweck der Datenbank muss dieser Wert angepasst werden, um die Datenbank performant zu machen.
        \subsubsection{Absch\"atzen der passenden Gr\"o\ss e f\"ur den Sysaux-Tablespace}
          Der \identifier{Sysaux}-Tablespace wird automatisch bei der Datenbankerstellung generiert. Er beinhaltet Nutzdaten f\"ur verschiedene Oracle Features wie z. B. Oracle Text, Ultra Search, Log Miner, Oracle Spatial und andere. Erstellt wird dieser Tablespace abh\"angig von den gew\"ahlten Oracle Features, in der richtigen Gr\"o\ss e. Der Administrator muss sich darum k\"ummern, dass dieser Tablespace auf einem Datentr\"ager mit gen\"ugend freiem Speicher erstellt wird.
        \subsubsection{Einplanen von Standardtablespaces f\"ur Nutzer}
          Wenn Nutzer neue Datenbankobjekte anlegen, werden diese im \enquote{Defaulttablespace} des jeweiligen Nutzers abgelegt. Wurde einem Nutzer kein Defaulttablespace zugewiesen oder existiert kein Tablespace, au\ss{}er dem \identifier{Sysaux}-Tablespace, wird automatisch dieser verwendet. SYSTEM stellt jedoch das Herzst\"uck einer Oracle Datenbank dar und sollte deshalb niemals als Defaulttablespace dienen.
    \section{Der Database Configuration Assistant (DBCA)}
      Der DBCA stellt ein grafisches Werkzeug dar, mit dessen Hilfe folgende T\"atigkeiten durchgef\"uhrt werden k\"onnen:
      \begin{itemize}
        \item Eine Datenbank erstellen
        \item Konfigurieren einer Datenbank
        \item Datenbanken l\"oschen
        \item Datenbanktemplates erstellen und verwalten
      \end{itemize}
      Gestartet wird der DBCA unter Windows mit Hilfe des Startmen\"us. Unter Linux muss, in einem Terminalfenster, der gesamte Pfad zur Datei \oscommand{dbca}, also \oscommand{\$ORACLE\_HOME/bin/dbca} angegeben werden. Dies kann jedoch nur dann funktionieren, wenn bereits die Umgebungsvariable \oscommand{\$ORACLE\_HOME} gesetzt wurde. Andernfalls muss der Pfad manuell angegeben werden: \oscommand{/u01/app/oracle/product/11.2.0/<SID>/bin/dbca} (<SID> stellt hier einen Platzhalter f\"ur die SID der Datenbank dar.). Wie gewohnt, wird der Nutzer zu aller erstbegr\"u\ss{}t.

      \bild{Willkommen im DBCA}{dbca_welcome}{3}

      Nach einem Klick auf den Button \enquote{Weiter} muss zuerst die gew\"unschte T\"atigkeit ausgew\"ahlt werden.

      \bild{Schritt 1 - Vorg\"ange}{dbca_step_1}{3}
      \subsection{Eine Datenbank mit dem DBCA erstellen}
        \label{db_mit_dbca_erstellen}
        F\"ur die Erstellung einer Datenbank bietet der DBCA zwei Varianten:
        \begin{itemize}
          \item Erstellen einer benutzerdefinierten Datenbank
          \item Erstellen einer Datenbank mittels einer Vorlage
        \end{itemize}
        Die erste Option erlaubt es, eine Datenbank komplett selbst zu
        gestalten, jedoch mit dem Nachteil, dass hier sehr viel Arbeit zu
        leisten ist. Die Erstellung einer Datenbank aus einer Vorlage hingegen
        bietet nahezu genauso viele M\"oglichkeiten, entlastet aber den
        Ersteller deutlich.
        \subsubsection{Schritt 2 - Datenbankvorlagen}
        \bild{Schritt 2 - Datenbankvorlagen}{dbca_step_2}{2.5}
          Standardm\"a\ss{}ig werden zwei Vorlagen durch den DBCA angeboten:
          \begin{itemize}
            \item Allgemeiner Gebrauch oder Transaktionsverarbeitung
            \item Data Warehouse
          \end{itemize}
          Diese beiden unterscheiden sich haupts\"achlich in den gespeicherten Initialisierungsparametern, mit denen die fertige Datenbank gestartet wird. Welche der beiden Vorlagen genutzt wird h\"angt davon ab, welchen Einsatzzweck die Datenbank haben soll.

          Ein Datawarehouse, auch als Decision Support System bezeichnet, ist eine Datenbank, die oft eine sehr gro\ss{}e Datenmenge h\"alt und durch aufwendige SQL-Anfragen analysiert wird. Das hei\ss{}t, dass f\"ur eine solche Datenbank eine Optimierung der Lesevorg\"ange im Vordergrund steht. Schreibvorg\"ange haben hier lediglich eine untergeordnete Rolle, da sie selten erfolgen und meist gro\ss{}e Datenmengen am St\"uck, in die Datenbank importiert werden.

          In einer OLTP - Oline Transaction Processing - Datenbank ist das Verh\"altnis zwischen Lese- und Schreibzugriffen ganz anders. Daraus folgt, dass die Datenbank f\"ur beide Vorg\"ange so optimal wie m\"oglich eingestellt werden sollte (Kompromissl\"osung).

          Ein Klick auf den Button \enquote{Details zeigen\dots} zeigt die Einstellungen der Vorlagen.

          \bild{Schritt 2 - Vorlagen-Details}{dbca_step_2_template_details}{3}
        \subsubsection{Schritt 3 - Datenbank-ID}
          Nach einem Klick auf \enquote{Weiter} muss in Schritt 3 der Datenbank ein Name gegeben werden. Zu diesem Thema gibt es jedoch einiges zu sagen.

          Eine Oracle-Datenbank hat, \"ahnlich wie ein Mensch, einen Vornamen und einen Nachnamen. Korrekt ausgedr\"uckt hat eine Oracle-Datenbank einen Datenbanknamen und eine Datenbankdom\"ane. Der Datenbankname kann maximal acht Zeichen umfassen und sollte daf\"ur genutzt werden, um den Einsatzzweck der Datenbank wieder zu spiegeln. Datenbanknamen wie \identifier{ORCL} oder \identifier{ORA11G} sind maximal in Trainingsumgebungen sinnvoll. In einer Produktivumgebung sollten Namen wie \identifier{dss} f\"ur ein Decision Support System oder \identifier{pers} f\"ur eine Personalverwaltung genutzt werden. Evtl. ist es auch n\"utzlich, dem Datenbanknamen eine Versionsnummer anzuh\"angen, um im Falle einer Migration die alte und die neue Datenbank am Namen unterscheiden zu k\"onnen.

          Die Datenbankdom\"ane ist der zweite Namensbestandteil. Sie kann dazu genutzt werden, um dem Datenbanknamen zus\"atzliche Informationen hinzuzuf\"ugen. Wenn beispielsweise zu einer produktiven Datenbank ein Testsystem hinzugef\"ugt werden soll, k\"onnten die Dom\"anen \identifier{prod.oracle.local} und \identifier{test.oracle.local} dazu genutzt werden, um beide Systeme zu unterscheiden. F\"ugt man beide Informationen zusammen, den Datenbanknamen und die Datenbankdom\"ane, so erh\"alt man den \enquote{Globalen Datenbanknamen}.
          \begin{merke}
            Der Globale Datenbankname besteht aus dem Datenbanknamen und der Datenbankdom\"ane. Er wird im Format \oscommand{db\_name.db\_domain} angegeben.
          \end{merke}
          In der Welt der Oracle-Datenbanken hat jedoch nicht nur die Datenbank einen Namen, sondern auch die zu ihr geh\"orende Instanz. Der Instanzname wird im Betriebssystem, in einer Umgebungsvariablen namens \identifier{ORACLE\_SID} hinterlegt. Die Abk\"urzung SID (gesprochen S-ID) steht f\"ur System Identifier. Welche L\"ange die SID haben darf ist Betriebssystemspezifisch, jedoch sind auf nahezu allen Platformen acht Zeichen ohne Probleme m\"oglich. Da die Instanz keine Dom\"ane als Erg\"anzung hat, kann hier eine Unterscheidung, z. B. zwischen Test- und Produktivsystem nur im Instanznamen selbst erfolgen, beispielsweise \enquote{ppers} oder \enquote{tpers}.
          \begin{merke}
            Der Instanznamen muss nicht gleich dem Datenbanknamen sein. Es empfiehlt sich jedoch beide Namen gleich zu w\"ahlen, um Datenbank und Instanz zusammen finden zu k\"onnen.
          \end{merke}
          \bild{Schritt 3 - Datenbank-ID}{dbca_step_3}{3}
        \subsubsection{Schritt 4 - Verwaltungsoptionen}
          Dieser Schritt gliedert sich in zwei Registerkarten. Auf dem Register \enquote{Enterprise Manager} kann gew\"ahlt werden, wie die Datenbank verwaltet werden soll.
          \begin{itemize}
            \item \textbf{Enterprise Manager Grid Control}: Grid Control ist eine zentrale Verwaltungskonsole f\"ur verschiedenste Oracle-Produkte, wie z. B. die Datenbank oder den Application Server. Sie kann f\"ur die Verwaltung beliebig vieler Systeme genutzt werden.
            \item \textbf{Enterprise Manager Database Control}: Hierbei handelt es sich um eine lokale Verwaltungskonsole f\"ur nur eine Datenbank.
            \item \textbf{Kein Enterprise Manager}: Es ist nicht zwingend notwendig f\"ur die Verwaltung einer Oracle-Datenbank den Enterprise Manager einzusetzen. Soll lediglich SQL*Plus genutzt werden oder sind Produkte von Drittanbietern im Einsatz, kann der Enterprise Manager einfach weggelassen werden.
          \end{itemize}
          \bild{Schritt 4 - Verwaltungs\-optionen}{dbca_step_4}{2.7}
          \begin{merke}
            Die Option \enquote{Enterprise Manager konfigurieren} kann seit Oracle 11g erst genutzt werden, wenn ein Listener (Siehe \enquote{Konfigurieren der Oracle Netzwerkumgebung}) konfiguriert und eine Datenbank erstellt wurde.
          \end{merke}
          Die Registerkarte \enquote{Automatische Wartungs-Tasks} bietet dem Administrator die M\"oglichkeit, automatische Wartungs-Tasks der Datenbank zu aktivieren. Dabei handelt es sich um Jobs, wie das Sammeln von Performance-Statistiken oder das Generieren von Berichten.
          \bild{Schritt 4 - Verwaltungs\-optionen}{dbca_step_4_automatic_maintenance_tasks}{3}
        \subsubsection{Schritt 5 - Datenbank-ID-Daten}
          In Schritt 5 m\"ussen Passw\"orter f\"ur die beiden Datenbankbenutzer \identifier{SYS} und \identifier{SYSTEM} festgelegt werden. Der Nutzer \identifier{SYS} ist ein Account mit uneingeschr\"ankten Rechten. Er ist von zentraler Bedeutung in einer Oracle-Datenbank. Diese Tatsache sollte sich in der Komplexit\"at seines Passwortes reflektieren.

          Das Benutzerkonto \identifier{SYSTEM} hat ebenfalls administrative Berechtigungen, ist jedoch st\"arker eingeschr\"ankt, als \identifier{SYS}. F\"ur diesen Nutzer sollte ebenso ein sicheres Passwort gew\"ahlt werden, da er das Recht hat, den gesamten Datenbankinhalt zu exportieren.

          \bild{Schritt 5 - Datenbank-ID-Daten}{dbca_step_5}{2.7}
        \subsubsection{Schritt 6 - Speicherort von Datenbankdateien}
          Eine Oracle Datenbank besteht aus einer Vielzahl unterschiedlicher Dateien. Damit diese erstellt werden k\"onnen, muss ein Speicherort vorgegeben werden. Der DBCA bietet dazu die beiden grunds\"atzlichen M\"oglichkeiten:
          \begin{itemize}
            \item Speichern der Dateien in einem Dateisystem
            \item Nutzung der Oracle-Eigenen Technologie \enquote{Automatic Storage Management}
          \end{itemize}
          Egal welche der beiden Methoden gew\"ahlt wurde, es muss nun noch die Auswahl des genauen Speicherortes getroffen werden. Es kann der Speicherort aus der Vorlage genutzt, ein eigenes Verzeichnis f\"ur alle Dateien angegeben oder die Techonologie \enquote{Oracle Managed Files}, die an sp\"aterer Stelle noch besprochen wird, genutzt werden.

          Falls keine dieser drei M\"oglichkeiten die richtige ist, kann am Ende des Assistenten der Speicherort einer jeden einzelnen Datei ge\"andert werden.
          \bild{Schritt 6 - Speicherort von Daten\-bank\-dateien}{dbca_step_6}{3}

          Um herauszufinden, was hinter der Angabe \oscommand{\textbraceleft{}ORACLE\_BASE\textbraceright{}/oradata} steckt, kann der Button \enquote{Variablen f\"ur Dateispeicherort} angeklickt werden.
          \bild{Schritt 6 - Variablen f\"ur Dateispeicherort}{dbca_step_6_file_locations}{4}
        \subsubsection{Schritt 7 - Recovery-Konfiguration}
          Dieser Dialog bietet zwei Optionen:
          \begin{itemize}
            \item \textbf{Flash Recovery-Bereich angeben}: Hierbei handelt es sich um ein von der Datenbank \"uberwachtes Verzeichnis, welches als Puffer f\"ur Backups dient.
            \item \textbf{Archivierung aktivieren}: Diese Option aktiviert den Archiver-Hintergrundprozess, der daf\"ur sorgt, dass automatisch Kopien der benutzten Redo Log Dateien angelegt werden.
          \end{itemize}
          \bild{Schritt 7 - Recovery-Konfiguration}{dbca_step_7}{2.5}
        \subsubsection{Schritt 8 - Datenbankinhalt}
          Hier ist es m\"oglich, die Datenbank mit Beispielinhalten f\"ur Test-
          und Schulungszwecke zu f\"ullen. Produktivsysteme sollten niemals die
          Beispielschemata enthalten, da diese bekannt sind und eine
          zus\"atzliche Angriffsfl\"ache darstellen.

          Auf der zweiten Registerkarte \enquote{Benutzerdefinierte Skripts}
          k\"onnen eigene SQL-Skripte angegeben werden, die der DBCA automatisch
          nach Erstellung der Datenbank ausf\"uhrt.
          \bild{Schritt 8 - Daten\-bank\-inhalt}{dbca_step_8}{2.8}
        \subsubsection{Schritt 9 - Initialisierungsparameter}
          Dies ist der wohl umfangreichste Dialog des DBCA. Auf der
          Registerkarte \enquote{Speicher} muss die Entscheidung getroffen
          werden, wie der Arbeitsspeicher der Instanz verwaltet werden soll.
          Folgende Methoden stehen zur Verf\"ugung:
          \begin{itemize}
            \item \textbf{Automatic Memory Management}: Bei dieser Option wird der Instanz eine Speichermenge zugeteilt, die diese dann vollst\"andig autark verwaltet. Es wird keine Unterscheidung in SGA und PGA gemacht. Der Standard sind 40 \% der gesamten Arbeitsspeichermenge des Servers.
            \bild{Schritt 9 - Automatic Memory Management}{dbca_step_9_automatic_memory_management}{3}
            \item \textbf{Automatic Shared Memory Management}: Dies ist der Vorg\"anger zum Automatic Memory Management. Hier werden der Instanz zwei getrennte Werte f\"ur die Speichermenge der SGA und der aggregierten PGAs gegeben. Beide Speicherbereiche werden unabh\"angig von einander von der Instanz verwaltet.
            \bild{Schritt 9 - Automatic Shared Memory Management}{dbca_step_9_automatic_shared_memory_management}{0.35}
\clearpage
            \item \textbf{Manual Shared Memory Management}: Mit dieser Variante
            muss f\"ur jeden Speicherpool in der SGA und f\"ur die aggregierten
            PGAs ein eigener Wert angegeben werden. Auch wenn der DBA hier die
            gr\"o\ss{}ten Einflussm\"oglichkeiten hat, so sollte doch eine der
            beiden automatischen Verwaltungsoptionen genutzt werden, da diese
            sich immer wieder den aktuellen Gegebenheiten anpassen.
            \bild{Schritt 9 - Manual Shared Memory Management}{dbca_step_9_manual_shared_memory_management}{0.3}
          \end{itemize}
          Nach der Einstellung der Speicherverwaltung kann nur auf der zweiten
          Registerkarte, mit dem Namen \enquote{Skalierung}, die Option
          \enquote{Prozesse} ge\"andert werden. Dieser Wert ist aus zwei
          Gr\"unden interessant:
          \begin{itemize}
            \item lizenzrechtlich: F\"ur jeden Client muss eine
            Cient-Access-Licence vorliegen.
            \item Serverauslastung: Damit der Server nicht \"uberlastet wird.
          \end{itemize}
          Der zweite Wert, die \enquote{Blockgr\"o\ss{}e} kann nicht ge\"andert
          werden, wenn die Datenbank aus einer Vorlage heraus erstellt wird. Nur
          bei einer benutzerdefinierten DB ist dies m\"oglich.
          \bild{Schritt 9 - Skalierung}{dbca_step_9_scalability}{2.8}
          Auf der dritten Registerkarte k\"onnen zwei Zeichens\"atze f\"ur die
          Datenbank ausgew\"ahlt werden, der Datenbankzeichensatz und der
          l\"anderspezifische Zeichensatz.
          \begin{merke}
            Ein Zeichensatz besteht, wie sein Name besagt, aus einer Menge von Zeichen (Buchstaben,               Ziffern, Sonderzeichen). Jedem Zeichen wird ein numerischer Code zugeordnet. Dieser wird vom Computer ben\"otigt, um die Zeichen verarbeiten zu k\"onnen.
          \end{merke}
          \bild{Schritt 9 - Zeichens\"atze}{dbca_step_9_character_sets}{2.5}
          Der Datenbankzeichensatz wird f\"ur die folgenden Aufgaben verwendet:
          \begin{itemize}
            \item Speichern von Daten in den Datentypen \identifier{CHAR}, \identifier{VARCHAR2}, \identifier{CLOB} und \identifier{LONG},
            \item Speichern von Objektbezeichnern, Spaltennamen und Variablen\-be\-zeich\-nern,
            \item Speichern von SQL und PL/SQL-Quellcode.
          \end{itemize}
          Bevor der Datenbankzeichensatz ausgew\"ahlt wird, sollten die folgenden \"Uberlegungen angestellt werden:
          \begin{itemize}
            \item Welche Landessprachen muss die Datenbank jetzt und in zukunft unterst\"utzen?
            \item Ist der gew\"unschte Zeichensatz auch auf dem Betriebssystem des Datenbankservers verf\"ugbar und welche Zeichens\"atze nutzen die Clients?
            \item Kommen die genutzten Anwendungen mit dem Zeichensatz zurecht?
            \item Gibt es Performance-Probleme oder andere Einschr\"ankungen bei der Nutzung dieses Zeichensatzes?
          \end{itemize}
          \begin{merke}
            Der Datenbankzeichensatz kann nach der Datenbankerstellung nur unter einer Bedingung ge\"andert werden: Der neue Zeichensatz muss eine strikte Obermenge des Aktuellen sein. Oft ist dies nicht der Fall, weshalb der Datenbankzeichensatz, auf Empfehlung von Oracle, immer \identifier{AL32UTF8} sein sollte, da dieser am umfassendsten ist.
          \end{merke}
          Der l\"anderspezifische Zeichensatz dient dazu, um Unicode-Zeichen in einer Datenbank zu speichern, die keinen Unicode-Datenbankzeichensatz nutzt. Nur die Datentypen \identifier{NCHAR}, \identifier{NVARCHAR2} und \identifier{NCLOB} unterst\"utzen diesen alternativen Zeichensatz.
          Auf der Registerkarte \enquote{Verbindungsmodus} kann die Art und Weise gew\"ahlt werden, wie sich Clients standardm\"assig mit der Datenbank verbinden sollen. Die beiden Modi \enquote{Dedizierter Server} und \enquote{Shared Server} werden sp\"ater im Skript noch n\"aher erl\"autert.
        \subsubsection{Schritt 10 - Datenbankspeicherung}
          In diesem vorletzten Schritt k\"onnen die Speicherorte aller Datenbankdateien ge\"andert werden. Des Weiteren ist es m\"oglich, verschiedene Optionen f\"ur einzelne Dateiarten zu \"andern.
          \bild{Schritt 10 - Datenbankspeicher}{dbca_step_10}{3.2}
        \subsubsection{Schritt 11 - Optionen f\"ur das Erstellen}
          In Schritt 11 von 11 bleibt nur die Auswahl, was mit den soeben get\"atigten Einstellungen geschehen soll. Soll damit eine neue Datenbank erstellt, ein neues Template kreiert oder ein SQL-Skript generiert werden. Diese Optionen sind kombinierbar.

          An dieser Stelle muss der Nutzer auf den \enquote{Beenden}-Button klicken, um eine Zusammenfassung des Erstellvorgangs angezeigt zu bekommen.
\clearpage
          \bild{Schritt 11 - Optionen f\"ur das Erstellen}{dbca_step_11}{2.5}
					\bild{Schritt 12 - Best\"atigung}{dbca_step_12_confirmation}{2.5}
          Nach einem Klick auf \enquote{OK} bleibt nur noch abzuwarten, bis die Datenbank fertig ist.
          \bild{Schritt 13 - Daten\-bank\-erstellung}{dbca_step_13_database_creation}{2.7}
\clearpage
				\subsection{Konfigurieren einer Datenbank mit dem DBCA}
          Der DBCA bietet die M\"oglichkeit, Konfigurationseinstellungen von Datenbankoptionen zu \"andern. Folgende Optionen stehen zur Verf\"ugung:
          \begin{itemize}
            \item Einbinden von installierten Komponenten in die Datenbank
            \item \"Andern des Verbindungsmodus
          \end{itemize}
          \subsubsection{Schritt 1 - Starten des DBCA}
%             \bild{Schritt 1 - Vorg\"ange}{dbca_configure_step_1}{2.8}
            Nach dem Start des DBCA muss die Option \enquote{Datenbankoptionen konfigurieren} ausgew\"ahlt werden.
          \subsubsection{Schritt 2 - Datenbank }
            Im zweiten Schritt wird eine Auswahl aller installierten Datenbanken geboten. Hier ist die Richtige auszuw\"ahlen.
            \bild{Schritt 2 - Datenbank}{dbca_configure_step_2}{2.8}

            Nachdem die Datenbank ausgew\"ahlt wurde, werden deren Einstellungen eingelesen.
            \bild{Schritt 2 - Datenbank}{dbca_configure_step_2b}{2.8}
          \subsubsection{Schritt 3 - Verwaltungsoptionen}
            Wie unter Schritt 4 der Datenbankerstellung beschrieben kann hier der Enterprise Manager f\"ur die Datenbank konfiguriert werden.
            \bild{Schritt 3 - Verwaltungs\-optionen}{dbca_configure_step_3}{0.375}
          \subsubsection{Schritt 4 - Datenbankinhalt}
            Hier k\"onnen verschiedene Datenbankoptionen hinzugef\"ugt oder entfernt werden. Dies ist nur m\"oglich, wenn es sich um eine benutzerdefinierte Datenbank handelt.
            \bild{Schritt 4 - Datenbankinhalt}{dbca_configure_step_4}{0.375}
          \subsubsection{Schritt 5 - Verbindungsmodus}
            Im letzten Schritt ist das \"Andern des Verbindungsmodus m\"oglich.
\clearpage
            \bild{Schritt 5 - Verbindungsmodus}{dbca_configure_step_5}{0.375}
