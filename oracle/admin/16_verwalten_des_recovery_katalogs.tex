  \chapter{Verwalten des Recovery Katalogs}
    \setcounter{page}{1}\kapitelnummer{chapter}
    \minitoc
\newpage
    \section{Einrichten eines Recovery Katalogs}
    \label{createrecoverycatalog}
      Das Einrichten eines Recovery Katalogs umfasst vier Schritte:
      \begin{itemize}
        \item Erstellen/Konfigurieren der Datenbank, die den Recovery Katalog aufnimmt
        \item Schaffen einer Netzwerkverbindung (TNS) zwischen der Zieldatenbank und der Recovery Katalog Datenbank.
        \item Den Eigent\"umer des Recovery Katalogs erstellen
        \item Den Recovery Katalog erstellen
      \end{itemize}
      Bevor der Katalog genutzt werden kann, muss noch die Zieldatenbank registriert werden.
      \subsection{Erstellen der Katalogdatenbank}
        F\"ur die Planung des Recovery Katalogs ist es wichtig zu wissen, wie viel Speicherplatz zur Verf\"ugung gestellt werden muss. Dies ist davon abh\"angig, wie viele Datenbanken durch den Katalog verwaltet werden oder wie gro\ss{} die Anzahl der Archive Logs und der Backupdateien einer jeden Datenbank ist. Als drittes werden auch noch RMAN stored Scripts im Recovery Katalog gespeichert, f\"ur die ebenfalls geringe Platzreserven ben\"otigt werden.

        \begin{merke}
          Laut Oracle sind ca. $15 MB * Anzahl\ reg\ DB = Wachstum/Jahr$ ein guter Ansatz.
        \end{merke}

        Der typische Speicherplatzverbrauch f\"ur einen Katalog teilt sich wie folgt auf:
        \begin{center}
          \tablecaption{Speicherplatzbedarf einer Katalogdatenbank}
          \tablefirsthead{%
          \hline
          \multicolumn{1}{|c}{\textbf{Art des Speicherplatzes}} &
          \multicolumn{1}{|c|}{\textbf{Verbrauch/Jahr}} \\
          \hline
          }
          \tablehead{%
          \hline
          \multicolumn{1}{|c}{\textbf{Art des Speicherplatzes}} &
          \multicolumn{1}{|c|}{\textbf{Verbrauch/Jahr}} \\
          \hline
          }
          \tabletail{%
            \hline
          }
          \begin{supertabular}[h]{|l|c|c|c|p{7cm}|}
          System-Tablespace & 90 MB \\
          \hline
          Temp-Tablespace & 5 MB \\
          \hline
          Undo-Tablespace & 5 MB \\
          \hline
          Recovery Katalog Tablespace & 15 MB pro registrierter Datenbank \\
          \hline
          Online Redo Logs & 10 MB pro Member (3 Gruppen mit je 2 Membern) \\
          \end{supertabular}
        \end{center}
\clearpage
      \subsection{Die TNS-Netzwerkverbindung erstellen}
        Um eine Netzwerkverbindung mittles TNS zwischen der Zieldatenbank und dem Recovery Katalog zu erm\"oglichen, sollte das Local Naming oder das Directory Naming genutzt werden. Im Folgenden wird eine Beispielkonfiguration f\"ur das Local Naming gezeigt.

        Es werden folgende Annahmen getroffen:
        \begin{itemize}
          \item Der Rechner auf dem sich der Recovery Katalog befindet benutzt TCP/IP.
          \item Der Name des Rechners, der den Recovery Katalog enth\"alt ist: FEA11-119CAT.
          \item Der Listener dieses Rechners l\"auft auf Port 1521.
          \item Der Servicename der Katalogdatenbank ist: CATDB.
        \end{itemize}
        Die Datei \oscommand{tnsnames.ora} der Zieldatenbank sollte folgenden Abschnitt beinhalten:
        \begin{lstlisting}[caption={Der Net Service Name der CATDB},label=admin1200,language=configfile]
CATDB =
  (DESCRIPTION=
    (ADDRESS_LIST=
     (ADDRESS= (PROTOCOL=tcp)(HOST=FEA11-119CAT)(PORT=1521))
    )
  (CONNECT_DATA=
    (SERVICE_NAME=CATDB)))
        \end{lstlisting}
      \subsection{Den Katalogeigent\"umer erstellen}
        Ist die Katalogdatenbank erst einmal erstellt und die TNS-Netzwerkverbindung geschaffen, kann der dritte Schritt, das Erstellen des Katalogeigent\"umers vorgenommen werden.
        \begin{enumerate}
          \item Als Nutzer sys mit der Datenbank verbinden.
            \begin{lstlisting}[caption={Mit der Katalogdatenbank verbinden},label=admin1201,language=terminal]
[oracle@FEA11-119SRV ~]$ sqlplus sys/oracle@CATDB as sysdba
            \end{lstlisting}
          \item Erstellen des Katalogtablespaces
            \begin{lstlisting}[caption={Den Katalogtablespace erstellen},label=admin1202,language=oracle_sql]
SQL> CREATE TABLESPACE catts
  2  DATAFILE '/u02/oradata/catdb/catts01.dbf' SIZE 15 M
  3  AUTOEXTEND ON MAXSIZE 30 M;
            \end{lstlisting}
\clearpage
          \item Erstellen des Katalogeigent\"umers.
            \begin{lstlisting}[caption={Den Katalogeigent\"umer erstellen},label=admin1203,language=oracle_sql]
SQL> CREATE USER catowner
  2  IDENTIFIED BY catpass
  3  DEFAULT TABLESPACE catts
  4  QUOTA unlimited ON catts;
            \end{lstlisting}
          \item Dem Katalogeigent\"umer die Rolle \privileg{recovery\_catalog\_owner} geben.
            \begin{lstlisting}[caption={Die Rolle \privileg{recovery\_catalog\_owner} \"ubergeben},label=admin1204,language=oracle_sql]
SQL> GRANT create session, recovery_catalog_owner TO catowner;
            \end{lstlisting}
        \end{enumerate}
        \begin{merke}
          Damit ein Nutzer Eigent\"umer des Recovery Katalogschemas sein kann, muss er zwingend die Rolle \privileg{recovery\_catalog\_owner} besitzen.
        \end{merke}

      \subsection{Den Recovery Katalog erstellen}
        Das Erstellen des Kataloges geschieht im RMAN.
        \begin{enumerate}
          \item RMAN starten und mit der Katalogdatenbank verbinden
            \begin{lstlisting}[caption={Mit der Katalogdatenbank verbinden},label=admin1205,language=rman,language=terminal]
[oracle@FEA11-119SRV ~]$ rman catalog catowner/catpass@CATDB
            \end{lstlisting}
          \item Den Katalog erstellen
            \begin{lstlisting}[caption={Katalog erstellen},label=admin1206,language=rman]
RMAN> CREATE CATALOG;
            \end{lstlisting}
        \end{enumerate}
        Das Kommando \languagerman{CREATE CATALOG} erstellt den Recovery Katalog im Default Tablespace des Katalogeigent\"umers. Wahlweise kann dieses Kommando auch um die \languageorasql{TABLESPACE}-Klausel erweitert werden.
      \subsection{Registrieren einer Datenbank}
        Der letzte Schritt, bevor eine Datenbank mit dem Recovery Katalog
        genutzt werden kann, ist die Datenbank zu registrieren.
\clearpage
        \begin{enumerate}
          \item Bei  Zieldatenbank und  Recovery Katalog anmelden
            \begin{lstlisting}[caption={Mit Ziel- und Katalogdatenbank verbinden},label=admin1207,language=terminal]
[oracle@FEA11-119SRV ~]$ rman target / catalog catowner/catpass@CATDB
            \end{lstlisting}
          \item Die Zieldatenbank in den Mount-Status versetzen
          \item Registrieren der Zieldatenbank
            \begin{lstlisting}[caption={Zieldatenbank registrieren},label=admin1208,language=rman]
RMAN> REGISTER DATABASE;
            \end{lstlisting}
        \end{enumerate}
        W\"ahrend dieses Vorgangs erstellt RMAN Zeilen in den Katalogtabellen, die alle notwendigen Informationen aus der Kontrolldatei der Zieldatenbank enthalten. So wird der Recovery Katalog mit der Zieldatenbank synchronisiert. Um zu \"uberpr\"ufen, ob die Synchronisation erfolgreich war, kann das Kommando \languagerman{REPORT SCHEMA} genutzt werden.
            \begin{lstlisting}[caption={Schemainformationen anzeigen},label=admin1209,language=rman]
RMAN> REPORT SCHEMA;

File Size(MB)   Tablespace       RB segs Datafile Name
---- ---------- ---------------- ------- -------------------
1        307200 SYSTEM             NO    /u02/oradata/ORCL/system01.dbf
2        303500 SYSAUX             NO    /u02/oradata/ORCL/sysaux01.dbf
3         20480 UNDOTBS            YES   /u02/oradata/ORCL/undotbs01.dbf
4         10240 USERS              NO    /u02/oradata/ORCL/users01.dbf
5         10240 EXAMPLE            NO    /u02/oradata/ORCL/example01.dbf
            \end{lstlisting}
   \section{Datenbanken verwalten}
      \subsection{Eine weitere Datenbank registrieren}
        Es k\"onnen mehrere Zieldatenbank in einem Recovery Katalog verwaltet werden, wenn sie unterschiedliche Datenbank-IDs aufweisen. Beim Vorgang des Duplizierens einer neuen Datenbank, mit dem RMAN-Kommando \languagerman{DUPLICATE} oder bei der Erstellung mit Hilfe des SQL-Statements \languageorasql{CREATE DATABASE}, wird automatisch eine neue DBID generiert. Lediglich bei Datenbankkopien, die auf anderem Wege erstellt wurden, kann es zu Problemen kommen. Hierzu muss zuerst mit dem RMAN-Kommando \languagerman{DBNEWID} eine neue DatenbankID erstellt werden.
        \begin{merke}
          Eine Datenbank kann auch in mehreren Recovery Katalogen registriert werden.
        \end{merke}
\clearpage
      \subsection{Aufheben einer Datenbankregistrierung}
        Zum Aufheben einer Datenbankregistrierung im Recovery Katalog gibt es das Kommando \languagerman{UNREGISTER DATABASE}. Dabei werden alle Eintr\"age zu einer Datenbank aus dem Katalog gel\"oscht.
        \begin{merke}
          Informationen die zwar im Recovery Katalog gespeichert waren, aber nicht in der Kontroldatei (\parameter{control\_file\_record\_keep\_time}) gespeichert sind, gehen bei diesem Vorgang verloren.
        \end{merke}
        Die Registrierung einer Datenbank wird wie folgt aufgehoben:

        \begin{enumerate}
          \item RMAN starten und eine Verbindung zur betreffenden Zieldatenbank und dem Recovery Katalog herstellen.
            \begin{lstlisting}[caption={Starten des RMAN},label=admin1210,language=rman]
[oracle@FEA11-119SRV ~]$ rman target / catalog catowner/catpass@CATDB
            \end{lstlisting}
            Es ist nicht zwingend notwendig, sich bei der Zieldatenbank anzumelden. Sind jedoch mehrere Datenbanken im Recovery Katalog registriert, muss die Zieldatenbank durch ihre DBID identifiziert werden.
          \item Auflisten aller Backup Sets und Image Copies der Zieldatenbank. Dies sollte aus Sicherheitsgr\"unden erfolgen, damit Backups wieder katalogisiert werden k\"onnen.
            \begin{lstlisting}[caption={Backup Sets und Image Copies auflisten},label=admin1211,language=rman]
RMAN> LIST BACKUP SUMMARY;
RMAN> LIST COPY;
            \end{lstlisting}
          \item Soll die Datenbank dauerhaft aus dem Recovery Katalog gel\"oscht werden, m\"ussen auch alle Backups der Zieldatenbank gel\"oscht werden.
            \begin{lstlisting}[caption={L\"oschen aller Backups},label=admin1212,language=rman]
RMAN> DELETE BACKUP DEVICE TYPE sbt;
RMAN> DELETE BACKUP DEVICE TYPE DISK;
RMAN> DELETE COPY;
            \end{lstlisting}
            \begin{merke}
              Dieser Schritt darf nur dann ausgef\"uhrt werden, wenn auch die Zieldatenbank selbst gel\"oscht werden soll!!!
            \end{merke}
\clearpage
          \item Aufheben der Datenbankregistrierung
            \begin{lstlisting}[caption={Aufheben der Datenbankregistrierung},label=admin1213,language=rman]
RMAN> UNREGISTER DATABASE;
            \end{lstlisting}
        \end{enumerate}
        Wird eine Datenbankregistrierung aufgehoben, ohne dass eine Verbindung zur Zieldatenbank m\"oglich ist, muss Schritt 4 wie folgt ver\"andert werden.
        \begin{lstlisting}[caption={Aufheben der Datenbankregistrierung ohne Verbindung zur Zieldatenbank},label=admin1214,language=rman]
RMAN> RUN {
2>      SET DBID=93457265485;
3>      UNREGISTER DATABASE;
4>    }
        \end{lstlisting}
    \section{Synchronisation des Recovery Katalogs}
      Wenn RMAN eine Synchronisation des Recovery Katalogs durchf\"uhrt, vergleicht er den Inhalt der Kontrolldatei der Zieldatenbank, mit dem Inhalt des Recovery Katalogs und gleicht beide einander an. RMAN f\"uhrt dabei folgende Schritte durch:
      \begin{enumerate}
        \item Erstellen eines Snapshot Controlfiles
        \item Snapshot Controlfile und Recovery Katalog vergleichen
        \item Angleichen von Snapshot Controlfile und Recovery Katalog
      \end{enumerate}
      RMAN f\"uhrt automatisch die Synchronisation des Recovery Katalogs durch. Dies geschieht in verschiedenen Situationen, wie z. B. bei Ausf\"uhrung des \languagerman{BACKUP}-Kommandos. Die Synchronisation kann jedoch auch im Bedarfsfall manuell durchgef\"uhrt werden. Dies geschieht mit dem \languagerman{RESYNC CATALOG}-Kommando.
      \subsection{Vollst\"andige und teilweise Resynchronisation}
        Die Synchronisation des Recovery Katalogs kann vollst\"andig oder nur teilweise erfolgen. In einer teilweisen Resynchronisation liest RMAN die Kontrolldatei der Zieldatenbank und wertet diese nach neuen Backups, neuen Archive Logs und der Redo Log Historie aus. Es werden keine Schemadaten der Zieldatenbank synchronisiert. Diese werden nur bei einer vollst\"andigen Synchronisation abgeglichen.
      \subsection{Wann sollte manuell synchronisiert werden?}
        Da RMAN die Synchronisation des Recovery Katalogs automatisch durchf\"uhrt, ist ein manuelles Synchronisieren meist nicht notwendig. Es gibt jedoch einige Ausnahmesituationen, die im Folgenden beschrieben werden.
        \begin{merke}
          Grundsatz: Manuelle Synchronisation ist immer dann notwendig, wenn \"uber einen l\"angeren Zeitraum keine Verbindung zwischen dem Katalog und der Zieldatenbank bestanden hat.
        \end{merke}
        \subsubsection{Resynchronisation wenn der Recovery Katalog nicht verf\"ugbar war}
          Es kann vorkommen, das Backup oder Recovery Arbeiten notwendig sind, ohne dass der Recovery Katalog verf\"ugbar ist. In so einem Fall kann keine automatische Synchronisation erfolgen.
        \subsubsection{Resynchronisation bei unregelm\"a\ss{}igen Backups}
          Folgendes Beispiel:
          \begin{itemize}
            \item Die Zieldatenbank l\"auft im Archive Logs Modus.
            \item Die Datenbank wird in unregelm\"a\ss{}igen Abst\"anden gesichert.
            \item Zwischen den einzelnen Backups wird eine hohe Anzahl Redo Log Switches erzeugt.
          \end{itemize}
          Bezugnehmend auf das obige Beispiel, kann es sinnvoll sein, den Recovery Katalog manuell zu synchronisieren, da er nicht automatisch bei jedem Redo Log Switch aktualisiert wird. Informationen \"uber Log Switches werden nur im Controlfile der Zieldatenbank gesichert. Wie aktuell der Recovery Katalog gehalten wird, h\"angt dann davon ab, wie oft er manuell synchronisiert wird.
        \subsubsection{Resynchronisation nach einer \"Anderung der Datenbankstruktur}
          Wie schon im vorhergehenden Fall, wird der Recovery Katalog auch bei
          einer Struk\-tur\-\"an\-der\-ung der Datenbank nicht automatisch
          resynchronisiert. Nach einer Struktur\"anderung an der Datenbank
          sollte wie folgt vorgegangen werden:
\clearpage
          \begin{enumerate}
            \item RMAN starten und eine Verbindung zur betreffenden Zieldatenbank und dem Recovery Katalog herstellen.
              \begin{lstlisting}[caption={Starten des RMAN},label=admin1215,language=rman]
[oracle@FEA11-119SRV ~]$ rman target / catalog catowner/catpass@CATDB
              \end{lstlisting}
            \item Resynchronisieren des Recovery Katalogs
              \begin{lstlisting}[caption={Resynchronisieren des Recovery Katalogs},label=admin1216,language=rman]
RMAN> RESYNC CATALOG;
              \end{lstlisting}
          \end{enumerate}
      \subsection{CONTROL\_FILE\_RECORD\_KEEP\_TIME}
        Der Initialisierungsparameter
        \parameter{control\_file\_record\_keep\_time} legt fest, wie lange die
        Eintr\"age des RMAN Repositories in der Kontrolldatei der Zieldatenbank
        erhalten bleiben. \"Uberschreitet ein Eintrag diesen Wert, so wird er
        aus der Kontrolldatei gel\"oscht. Deshalb sollte darauf geachtet werden,
        dass die Synchronisationsvorg\"ange innerhalb der angegebenen Zeitspanne
        bleiben. Das hei\ss{}t, dass der Wert f\"ur den
        Initialisierungsparameter \parameter{control\_file\_record\_keep\_time}
        so gew\"ahlt werden sollte, dass das RMAN Repository ohne Verlust in den
        Recovery Katalog synchronisiert werden kann.
