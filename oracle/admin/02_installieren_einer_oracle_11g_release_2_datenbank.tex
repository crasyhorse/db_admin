  \chapter{Installieren einer Oracle 11g Release 2 Datenbank}
    \setcounter{page}{1}\kapitelnummer{chapter}
    \minitoc
\newpage
    \section{Aufgaben und Verantwortungsbereich eines DBA}
      Der Verantwortungsbereich eines Datenbankadministrators kann die folgenden Aufgaben einschlie\ss en:
      \begin{itemize}
        \item Installation und Upgrade des Datenbankservers und der Nutzeranwendungen
        \item Planen und \"Uberwachen des Speicherbedarfs der Datenbank auf dem Datentr\"ager
        \item Erstellen von Storage Strukturen in der Datenbank (z. B. Tabellen und Indizes)
        \item Nutzerkonten verwalten und \"uberwachen der Sicherheit
        \item Performance Monitoring und Tuning
        \item Entwickeln von Backup- und Recoverystrategien
        \item Verwalten von archivierten Daten
        \item Durchf\"uhren von Backup und Recovery
      \end{itemize}
      \subsection{Planen und durchf\"uhren einer Oracle Installation}
        \subsubsection{Lesen der Releasenotes}
          Die Releasenotes sind mit Sicherheit der am wenigsten beachtete Teil einer Software Dokumentation. So wenig Beachtung sie jedoch finden, um so wichtiger sind sie tats\"achlich. Sie enthalten wichtige Neuerungen zur Vorg\"angerversion und andere Hinweise, die bei einem Update auf eine neue Version, der Auswahl der notwendigen Hardware und anderen Preinstallation Tasks von gro\ss er Bedeutung sein k\"onnen.

          \begin{literaturinternet}
            \item \cite{e23557}
            \item \cite{e23558}
          \end{literaturinternet}

        \subsubsection{Evaluieren der vorhandenen Hardware}
          Vor der Installation sollte der DBA pr\"ufen, wie die vorhandene
          Hardware bestm\"oglich durch die Oracle-Datenbank und die ben\"otigten
          Anwendungsprogramme genutzt werden kann. Diese \"Uberpr\"ufung sollte
          die folgenden \"Uberlegungen einschlie\ss en:
          \begin{itemize}
            \item Wie viele Datentr\"ager sind f\"ur die Datenbank verf\"ugbar?
            \item Wie viel Arbeitsspeicher ist f\"ur die Oracle Instanz verf\"ugbar?
          \end{itemize}

          \begin{literaturinternet}
            \item \cite{BABFDGHJ}
            \item \cite{i1011417}
          \end{literaturinternet}

        \subsubsection{Planen der Datenbank}
          Zum Planen einer Datenbank geh\"oren die folgenden Aufgaben:
          \begin{itemize}
            \item Planen der logischen Speicherstrukturen der Datenbank
            \item Pr\"ufen des Datenbankdesigns
            \item Entwickeln von Backupstrategien
          \end{itemize}
          Die Planung der logischen Speicherstrukturen der Datenbank ist wichtig, um deren Einfl\"usse auf die Systemperformance erkennen zu k\"onnen. Beispielsweise ist es entscheidend, ob in einer Datenbank gro\ss e Objekte, wie z. B. Bilder oder ISO-Images gespeichert werden oder nur normale Text/Zahlen-Werte. Auch das Datenbankdesign ist entscheidend f\"ur die Performance der Datenbank. Meist hat der DBA zwar keinen direkten Einfluss\ auf das Design, er kann aber Hinweise und Verbesserungsvorschl\"age geben.
      \subsection{Download und Installation von Patches}
        Um die Datenbank sicherer zu machen und um Bugs zu beheben, muss der DBA von Zeit zu Zeit Patches oder Patchsets installieren. Ein Patch (auch \enquote{single interim patch} genannt) behebt ein einzelnes spezielles Problem und ist nicht bei jeder Installation notwendig. Ein Patchset ist eine Sammlung von Patches, die so erstellt wurde, das sie f\"ur jeden Kunden passt. Ein Patchset hat eine Releasenummer. Wurde beispielsweise die Oracle Version 11.2.0.0 installiert, dann hat das erste Patchset die Versionsnummer 11.2.0.1.
    \section{OFA - Die Optimal Flexible Architecture}
      \subsection{\"Uberblick \"uber die OFA}
        Die Optimal Flexible Architecture ist eine Menge von Namenskonventionen und Konfigurationsrichtlinien die zuverl\"assige Oracle Installationen mit minimalem Wartungsaufwand sicherstellen sollen. Der OFA-Standard wurde entwickelt um:
        \begin{itemize}
          \item eine gro\ss e Menge komplexer Anwendungen zusammen mit deren Nutzdaten auf einem Datentr\"ager zu speichern und dabei Bottlenecks und schlechte Performance zu vermeiden,
          \item administrative Routineaufgaben zu erleichtern, wie z. B. Backups,
          \item das Wechseln zwischen verschiedenen Oracle Instanzen zu erleichtern,
          \item es dem Administrator zu erm\"oglichen angemessen auf das Wachstum einer Datenbank reagieren zu k\"onnen,
          \item die Fragmentierung von Teilen der Datenbank m\"oglichst gering zu halten.
        \end{itemize}
        Die OFA kann als eine Art \enquote{Sammlung von guten Manieren} bei der Erstellung von Verzeichnissen f\"ur Oracle angesehen werden. Der Oracle Universal Installer platziert automatisch alle Oracle Komponenten in Verzeichnissen, die gem\"a\ss{} OFA erstellt wurden. Auch wenn die Nutzung einer OFA kein Muss ist, wird dies in jedem Falle von Oracle empfohlen.

        Der Oracle Universal Installer (OUI) trennt die Datenbanksoftware von den Nutzdaten. Er legt die Software im Oracle Homeverzeichnis \oscommand{\$ORACLE\_HOME} ab. Die Nutzdaten werden in \oscommand{\$ORACLE\_BASE/oradata} abgelegt. \oscommand{\$ORACLE\_BASE} und \oscommand{\$ORACLE\_HOME} sind Umgebungsvariablen die Verzeichispfade enthalten.

        Der Vorteil an dieser Methode ist, wenn eine neue Version der Datenbanksoftware installiert wird, hat diese ein neues Oracle Homeverzeichnis. Sie kann auf Zuverl\"assigkeit getestet werden und die alte Version kann nach erfolgreichen Tests der neuen Version gel\"oscht werden, ohne dass dabei Probleme entstehen.
      \subsection{Charakteristika einer OFA-Konformen Installation}
        Eine OFA-Konforme Installation  besitzt die folgenden Charakterristika:
        \begin{itemize}
          \item Unabh\"angige Unterverzeichnisse

            Dateien unterschiedlicher Arten werden in unterschiedliche Unterverzeichnisse gelegt. Somit ist eine Dateiart wenig bis gar nicht betroffen, wenn eine Andere ver\"andert oder gel\"oscht wird.
          \item Konsequente Namenskonventionen f\"ur die Dateien der Datenbank

            Die Dateien der Oracle-Datenbank k\"onnen einfach anhand der Verzeichnisstruktur von anderen Dateien unterschieden werden. Datenbankdateien unterschiedlicher Oracle Versionen k\"onnen auf die gleiche Art und Weise sehr einfach von einander unterschieden werden.
          \item Integrit\"at der Oracle-Home-Verzeichnisse

            Oracle-Home-Verzeichnisse k\"onnen hinzugef\"ugt, verschoben oder gel\"oscht werden, ohne das die betroffene Software dadurch besch\"adigt werden w\"urde.
\clearpage
					\item Metadaten unterschiedlicher Datenbanken von einander trennen

            Die strikte Trennung administrativer Daten von einer Datenbank gegen\"uber allen anderen erleichtert administrative Arbeiten deutlich und schafft eine \"ubersichtliche und zuverl\"a{}ssige Struktur.
          \item Strikte Trennung unterschiedlicher Arten von Nutzdaten

            Nutzdaten unterschiedlicher Arten k\"onnen von einander getrennt werden, um eine bessere Performance zu schaffen
          \item Verteilung von I/O-Last auf alle verf\"ugbaren Datentr\"ager
        \end{itemize}
      \subsection{Die OFA-Konventionen}
        Die OFA schreibt folgende Dateiendungen f\"ur Datenbankdateien vor:
        \begin{itemize}
          \item *.ctl Kontrolldatei
          \item *.dbf Datendatei
          \item *.log Redo Log Datei
        \end{itemize}
        \subsubsection{Das Oracle Base-Verzeichnis}
          Die Umgebungsvariable \oscommand{\$ORACLE\_BASE} enth\"alt den Pfad des Oracle Basisverzeichnisses. Es stellt die Wurzel des Oracleverzeichnisbaumes dar. Der OUI setzt f\"ur \oscommand{\$ORACLE\_BASE} automatisch den Wert

          \oscommand{/u01/app/oracle}

        \subsubsection{Das Oracle Homeverzeichnis}
          Die Umgebungsvariable \oscommand{\$ORACLE\_HOME} enth\"alt das Oracle Homeverzeichnis. Es ist der Speicherort f\"ur die Oracle Datenbanksoftware. \oscommand{\$ORACLE\_HOME} ist ein Unterverzeichnis von \oscommand{\$ORACLE\_BASE}. Zum Beispiel:

          \oscommand{/u01/app/oracle/product/11.2.0/orcl}

          \begin{literaturinternet}
            \item \cite{BABHAIIJ}
          \end{literaturinternet}

    \section{Durchf\"uhren einer Oracle Installation auf Linux}
      In dieser Sektion wird die Installation einer Oracle 11g R2 Datenbank auf einem Oracle Enterprise Linux 6 System beschrieben. Alle Voreinstellungen die am Betriebssystem gemacht werden m\"ussen, sind bereits get\"atigt worden. An dieser Stelle wird nur noch die reine Softwareinstallation beschrieben.
      \subsection{Der Installationsbeginn}
        Das Starten der Installation erfolgt durch den Aufruf: \oscommand{/media/CDROM/runInstaller}. Die Pfadangabe \oscommand{/media/CDROM} kann von System zu System variieren. Hierbei handelt es sich um den Pfad, an dem das CD/DVD-Laufwerk oder die ISO-Datei gemountet wurde. \oscommand{runInstaller} ist das Kommando zum Starten des Oracle Universal Installer, kurz OUI.
        \subsubsection{Schritt 1- Sicherheitsupdates}
          Der OUI ist ein grafisches Werkzeug zur Installation der Oracle-Datenbank-Software. Der Setup-Vorgang umfasst insgesamt neun Schritte.

          \bild{Schritt 1 - Sicherheitsupdates}{oui_step_1}{0.45}
\clearpage
          In Schritt 1  der Installation wird das Fenster f\"ur die Konfiguration von Sicherheitsupdates angezeigt. Hier kann der Administrator seine E-Mail-Adresse eingeben, um \"uber Sicherheitsupdates informiert zu werden. Zus\"atzlich kann er auch sein \enquote{My Oracle Support} Kennwort angeben, so dass Sicherheitsupdates direkt bezogen werden k\"onnen.

          Sollte der Administrator in diesem Schritt \enquote{vergessen haben}, seine E-Mail-Adresse anzugeben, wird er prompt auf diesen \enquote{Fehler} hingewiesen.

          Ein Klick auf den \enquote{Ja}-Button l\"ost das Problem.

          \bild{E-Mail-Adresse vergessen?}{oui_step_1_no_mail_address}{0.5}
        \subsubsection{Schritt 2 - Installationsoptionen}
          In diesem Installationsschritt wird abgefragt, ob:
          \begin{itemize}
            \item Die Software installiert und eine neue Datenbank angelegt werden soll,
            \item nur die Software installiert werden soll oder
            \item ob ein Upgrade einer bestehenden Datenbank durchgef\"uhrt werden soll.
          \end{itemize}
          \bild{Schritt 2 - Installationsoptionen}{oui_step_2}{0.35}
        \subsubsection{Schritt 3 - Grid-Installation Optionen}
          Dieser Schritt ist seit Oracle 11g neu. Die Version 11 der Oracle-Datenbank beinhaltet eine Software, die als \enquote{Grid Infrastructure} bezeichnet wird. Diese wird jedoch nur dann ben\"otigt, wenn eine Real Application Cluster Installation durchgef\"uhrt werden soll.
          \bild{Schritt 3 - Grid-Optionen}{oui_step_3}{0.38}
        \subsubsection{Schritt 4 - Produktsprachen}
          Oracle 11g R2 wird standardm\"a\ss{}ig in den Sprachen Deutsch und Englisch installiert. Zus\"atzlich stehen weitere 45 Sprachen zur Verf\"ugung.
          \bild{Schritt 4 - Produktsprachen}{oui_step_4}{0.38}
        \subsubsection{Schritt 5 - Datenbank Edition}
          Die Oracle-Datenbank existiert in verschiedenen Editionen. Es gibt die:
          \begin{itemize}
            \item Enterprise Edition
            \item Standard Edition
            \item Standard Edition One
            \item Personal Edition (nur unter Windows)
          \end{itemize}
          \bild{Schritt 5 - Datenbank Edition}{oui_step_5}{0.42}
          Der Unterschied zwischen diesen Editionen liegt in deren Funktionsumfang und in den Lizenzkosten. Die Enterprise Edition ist das umfangreichste Paket. Sie enth\"alt alle Features und sie bietet die M\"oglichkeit \enquote{Datenbank Optionen} hinzu zu installieren, um die F\"ahigkeiten der Datenbank noch mehr zu erweitern.

          Die Standard Edition hat einen eingeschr\"ankten Funktionsumfang. Sie ist nur noch f\"ur  mittlere Unternehmen geeignet, da Features fehlen, die im Umgang mit gro\ss{}en Datenmengen absolut unverzichtbar sind. F\"ur diese Edition ist es auch nicht m\"oglich, Datenbank Optionen hinzuzuf\"ugen.

          Die Standard Edition One ist das kleinste Softwarepaket. Sie hat einen sehr stark eingeschr\"ankten Funktionsumfang und ist somit nur noch f\"ur kleine Unternehmen oder Abteilungen gedacht.

          Sollte die Enterprise Edition ausgew\"ahlt worden sein, so k\"onnen \"uber den Button \enquote{Optionen w\"ahlen} weitere Datenbankoptionen hinzugef\"ugt werden.
          \bild{Schritt 5 - Datenbank Optionen hinzuf\"ugen}{oui_step_5_database_options}{0.42}
          \begin{merke}
            Einige Datenbankoptionen m\"ussen eigenst\"andig lizensiert werden, wodurch zus\"atzliche Lizenzkosten entstehen!
          \end{merke}
        \subsubsection{Schritt 6 - Installationsspeicherort}
          Um diesen Schritt erfolgreich zu beenden, m\"ussen zwei Angaben gemacht werden:
          \begin{itemize}
            \item \textbf{Oracle Base}: Das Oracle Base-Verzeichnis ist die Wurzel des Installationspfades. Alle weiteren Angaben beziehen sich standardm\"a\ss{}ig auf dieses Verzeichnis.
            \item \textbf{Softwareverzeichnis}: Dies ist das \oscommand{\$ORACLE\_HOME}-Verzeichnis. Hierhinein wird die Oracle-Software installiert. Bei diesem Verzeichnis handelt es sich um ein Unterverzeichnis von \oscommand{\$ORACLE\_BASE}.
          \end{itemize}
          \bild{Schritt 6 - Installationsspeicherort}{oui_step_6}{0.39}
        \subsubsection{Schritt 7 - Bestandsverzeichnis erstellen (nur Linux)}
          Beim Oracle Bestandsverzeichnis handelt es sich um ein Verzeichnis, in dem alle installierten Oracle-Produkte Bestandsdateien anlegen. Es wird bei der ersten Installation eines beliebigen Oracle-Produktes angelegt. Alle Produkte erstellen dort Unterverzeichnisse f\"ur ihre eigenen Bestandsdaten.
          \bild{Schritt 7 - Bestandsverzeichnis erstellen}{oui_step_7}{0.45}
        \subsubsection{Schritt 8 - Berechtigte Betriebssystemgruppen}
          Bei der Installation der Oracle-Datenbank werden zwei Betriebssystemgruppen angelegt, welche f\"ur die \enquote{Betriebssystemauthentifizierung} relevant sind: OSDBA und OSOPER. Mitglieder der Gruppe OSDBA haben innerhalb der Datenbank alle Berechtigungen. Wie der Name der Gruppe sagt, sollten nur DBAs Mitglied sein.

          Alle Mitglieder der Betriebssystemgruppe OSOPER haben eingeschr\"ankte administrative Rechte. F\"ur die t\"agliche Arbeit eines Oracle-DBAs ist diese Gruppe faktisch irrelevant.
\clearpage
          \bild{Schritt 8 - Berechtigte Betriebssystemgruppen}{oui_step_8}{0.39}
        \subsubsection{Schritt 9 - Voraussetzungen pr\"ufen}
          In diesem Schritt werden alle Bedingungen gepr\"uft, die betriebssystemseitig vor der Installation erf\"ullt sein m\"ussen.
          \bild{Schritt 9 - Voraussetzungen pr\"ufen}{oui_step_9}{0.39}
        \subsubsection{Schritt 10 - \"Uberblick}
          Bevor der Nutzer mit einem Klick auf den Button \enquote{Fertig stellen} die Installation startet wird ein \"Uberblick \"uber alle gew\"ahlten Einstellungen gegeben. Erstmalig wird hier auch die M\"oglichkeit geboten eine Antwortdatei mit allen get\"atigten Einstellungen erstellen zu lassen. Diese kann an sp\"aterer Stelle f\"ur eine automatisierte Installation genutzt werden.
          \bild{Schritt 10 - \"Uberblick}{oui_step_10}{0.45}
        \subsubsection{Schritt 11 - Produkt installieren}
          Hier erfolgt nun die eigentliche Installation.

          Unter Linux/Unix ist es notwendig zwei Aufgaben als Benutzer \enquote{root} durchzuf\"uhren, da diese lokale Administratorrechte erfordern. Beide Aufgaben liegen in Form von Shell-Skripten vor. Um diese Skripte ausf\"uhren zu k\"onnen, muss ein neues Terminalfenster ge\"offnet und die Identit\"at des Nutzers root angenommen werden.

          Nach dem Ausf\"uhren beider Skripte kann das neue Terminalfenster wieder geschlossen und das Fenster \enquote{Konfigurationsskripte ausf\"uhren} mit einem Klick auf \enquote{OK} geschlossen werden.

          \bild{Schritt 11 - Produkt installieren}{oui_step_11}{0.45}

          \bild{Schritt 11 - Das Skript \enquote{orainstRoot.sh}}{oui_step_11_orainstRoot}{0.45}
\clearpage
          \bild{Schritt 11 - Das Skript \enquote{root.sh}}{oui_step_11_root}{0.42}
        \subsubsection{Schritt 12 - Beenden}
          Zu guter Letzt erh\"alt man noch die Best\"atigung, dass die Installation erfolgreich verlaufen ist. Mit einem Klick auf \enquote{Schlie\ss{}en} kann der Oracle Universal Installer nun geschlossen werden.
          \bild{Schritt 12 - Beenden}{oui_step_12}{0.42}
    \section{Software installieren/deinstallieren}
      W\"ahrend der Installation der Oracle-Software wird auch eine Kopie des Oracle Universal Installers installiert. Diese befindet sich im Oraclesoftwareverzeichnis, Unterordner \oscommand{oui/bin}. Mit Hilfe dieses OUI kann weitere Software nachinstalliert bzw. k\"onnen Komponenten deinstalliert werden.

      Unter Microsoft Windows kann der OUI aus dem Startmen\"u ge\"offnet werden. F\"ur Linux-Systeme muss die Kommandozeile genutzt werden.

      \oscommand{/u01/app/oracle/product/11.2.0/ORCL/oui/bin/runInstaller}

      Nach dem Start erfolgt die \"ubliche Begr\"u\ss{}ung.
      \bild{Willkommen im OUI}{local_oui_welcome}{0.45}

      Hier erfolgt nun die Auswahl, was als n\"achstes geschehen soll. Mit einem Klick auf den Button \enquote{Installierte Produkte} kann das Oracle Bestandsverzeichnis abgefragt werden.

      Durch anhaken/markieren einzelner Komponenten und anklicken der Schaltfl\"ache \enquote{Entfernen} k\"onnen Teile der Software deinstalliert werden.

      Zur\"uck im Willkommensfenster kann mit einem Klick auf die Schaltfl\"ache \enquote{Weiter}, die Installation einer weiteren Oracle-Instanz begonnen werden. Die Installationsschritte bleiben dabei die gleichen, wie sie soeben beschrieben wurden.
\clearpage
      \bild{Auflistung aller installierten Produkte}{local_oui_orainventory}{0.45}
