  \chapter{Undo-Daten verwalten}
    \setcounter{page}{1}\kapitelnummer{chapter}
    \label{undodata}
    \minitoc
\newpage
    \section{Undo-Tablespaces verwalten}
      \subsection{Automatic Undo Management aktivieren}
        Oracle ist in der Lage, Undo-Daten voll automatisiert zu verwalten. Der
        dazu verwendete Mechanismus hei\ss{}t \enquote{Automatic Undo
        Management}. Wird er benutzt, muss nur ein Undo-Tablespace erstellt
        werden und Oracle k\"ummert sich um den Rest.

        Zur Aktivierung dieses Features, muss der Parameter \parameter{undo\_management} auf den Wert \enquote{auto} gesetzt werden. Dadurch wird bei der Erstellung der Datenbank automatisch ein \enquote{Default Undo-Tablespace} angelegt. Ebenfalls werden auch alle Initialisierungsparameter, die mit manuellem Undo-Management zu tun haben ignoriert.
        \begin{merke}
          Manuelles Undo Management ist zwar noch immer m\"oglich, wird aber von Oracle keinesfalls mehr empfohlen!
        \end{merke}
        Ist der Undo-Tablespace nicht verf\"ugbar, kann die Datenbank nicht
        geöffnet werden. Tritt dieser Fall ein, wird ein Eintrag in der Alert
        log Datei gemacht.
        \begin{lstlisting}[caption={Fehlermeldung bei nicht vorhandenem
        Undo-Tablespace},label=admin501,language=terminal]
Mon Sep 23 15:52:19 2007 
Errors in file /u01/app/oracle/diag/rdbms/orcl/orcl/trace/orcl_ora_544.trc:
ORA-30012: Undo Tablespace 'UNDOTBS01' ist nicht vorhanden oder hat den falschen
Typ
Mon Sep 23 15:52:19 2007
Error 30012 happened during db &open&, shutting down database
&user&: terminating instance due to error 30012
Mon Sep 23 15:52:19 2007
Error in file /u01/app/oracle/diag/rdbms/orcl/orcl/trace/orcl/orcl_pmon_3000.trc
ORA-30012: undo tablespace '' does not exist or of wrong type
        \end{lstlisting}
        Der Parameter \parameter{undo\_tablespace} legt den Undo-Tablespace der Instanz fest.
        \begin{lstlisting}[caption={Der Parameter \parameter{undo\_management}},label=admin502,language=oracle_sql]
UNDO_TABLESPACE = undotbs_01
        \end{lstlisting}
      \subsection{Einen Undo-Tablespace erstellen}
        Es gibt zwei M\"oglichkeiten einen Undo-Tablespace zu erstellen:
        \begin{itemize}
          \item Durch das \languageorasql{CREATE DATABASE}-Kommando, beim Anlegen der Datenbank.
          \item Mit dem \languageorasql{CREATE UNDO TABLESPACE}-Statement
        \end{itemize}
        Wird ein Undo-Tablespace mit dem \languageorasql{CREATE UNDO TABLESPACE}-Statement erstellt, sind die beiden einzigen Parameter die dabei ge\"andert werden k\"onnen:
        \begin{itemize}
          \item die Datafile-Klausel
          \item die Extent-Management-Klausel
        \end{itemize}
        Im folgenden Beispiel wird ein automatisch wachsender Undo-Tablespace erstellt.
        \begin{lstlisting}[caption={Undo-Tablespace erstellen},label=admin503,language=oracle_sql]
SQL> CREATE UNDO TABLESPACE undotbs02
  2  DATAFILE '/u02/oradata/orcl/undotbs02_01.dbf' SIZE 10M
  3  AUTOEXTEND ON;
        \end{lstlisting}
        \begin{merke}
          Es k\"onnen mehrere Undo-Tablespaces erstellt werden, jedoch ist immer
          nur einer aktiv.
        \end{merke}
      \subsection{Einen Undo-Tablespace ver\"andern}
        Undo-Tablespaces werden mit dem Kommando \languageorasql{ALTER TABLESPACE} ver\"andert. Da die meisten Parameter f\"ur einen Undo-Tablespace direkt vom System verwaltet werden, muss sich der DBA nur um folgende Dinge k\"ummern:
        \begin{itemize}
          \item Datendateien zum Tablespace hinzuf\"ugen
          \item Datendateien des Tablespaces umbenennen
          \item Datendateien des Tablespaces on-/offline setzen
          \item (De-)Aktivieren der Retention Guarantee
        \end{itemize}
        Andere Parameter k\"onnen nicht durch den Administrator ver\"andert werden.

        Um zu verhindern, dass der Platz in einem Undo-Tablespace nicht mehr ausreicht, kann eine Datendatei hinzugef\"ugt werden.
          \begin{lstlisting}[caption={Datendatei zum Undo-Tablespace hinzuf\"ugen},label=admin504,language=oracle_sql]
SQL> ALTER TABLESPACE undotbs02
  2  ADD DATAFILE '/u01/app/oracle/oradata/orcl/undotbs02_02.dbf' SIZE 100M
  3  AUTOEXTEND ON MAXSIZE 5G;
          \end{lstlisting}
          Eine andere M\"oglichkeit besteht darin, eine bestehende Datendatei mit dem \languageorasql{ALTER DATABASE}-Kommando in ihrer Gr\"o\ss{}e zu ver\"andern.
          \begin{lstlisting}[caption={Datendatei in ihrer Gr\"o\ss{}e ver\"andern},label=admin505,language=oracle_sql]
SQL> ALTER DATABASE
  2  DATAFILE '/u01/app/oracle/oradata/orcl/undotbs02_02.dbf'
  3  RESIZE 250M;
          \end{lstlisting}
      \subsection{Einen Undo-Tablespace l\"oschen}
        Um einen Undo-Tablespace zu l\"oschen, wird das \languageorasql{DROP TABLESPACE}-Kommando verwendet.
        \begin{lstlisting}[caption={Undo-Tablespace l\"oschen},label=admin506,language=oracle_sql]
SQL> DROP TABLESPACE undotbs02;
        \end{lstlisting}
        \begin{merke}
          Ein Undo-Tablespace kann nur gel\"oscht werden, wenn keine offenen Transaktionen ihn mehr verwenden. Zu beachten ist dabei, dass ein Undo-Tablespace, trotz der Tatsache, dass als unexpired markierte Undo-Records existieren, gel\"oscht werden kann. Daher ist beim L\"oschen von Undo-Tablespaces gr\"o\ss{}te Vorsicht geboten.
        \end{merke}
      \subsection{Den Undo-Tablespace wechseln}
        Da der \parameter{undo\_tablespace}-Initialisierungsparameter dynamisch ist, ist es m\"oglich den Undo-Tablespace im laufenden Betrieb zu wechseln.
        \begin{lstlisting}[caption={Undo-Tablespace wechseln},label=admin507,language=oracle_sql]
SQL> ALTER SYSTEM SET undo_tablespace = undotbs02;
        \end{lstlisting}
        \subsubsection{Auswirkungen des Wechsels}
          Hat der Wechsel funktioniert, wird der neue Undo-Tablespace, ohne f\"ur die Nutzer sp\"urbare \"Anderungen, verwendet.

          Ein Fehlschlagen des Wechsels kann aus folgenden Gr\"unden geschehen:
          \begin{itemize}
            \item Der neue Tablespace existiert nicht
            \item Der neue Tablespace ist kein Undo-Tablespace
            \item Der neue Undo-Tablespace wird bereits von einer anderen Instanz benutzt (RAC)
          \end{itemize}
          Wenn nach dem Wechsel neue Transaktionen durch Nutzer gestartet werden, werden die Undo-Daten hierf\"ur im neuen Undo-Tablespace verwaltet. Der alte Undo-Tablespace bekommt den Status \enquote{pending offline}. Transaktionen, die vor dem Wechsel bereits bestanden, werden noch aus dem alten Undo-Tablespace bedient, es k\"onnen jedoch keine neuen mehr hinzukommen.

          W\"ahrend ein Undo-Tablespace im pending offline-Modus ist, kann er nicht gel\"oscht werden. Erst wenn alle diesen Undo-Tablespace betreffenden Transaktionen beendet wurden, wird der Status des Tablespaces von pending offline auf offline gesetzt. Jetzt kann dieser Undo-Tablespace entweder einer anderen Instanz zur Verf\"ugung gestellt oder gel\"oscht werden.
    \section{Undo Retention}
      \label{undoretention}
      Durch das Ausf\"uhren von Transaktionen sammeln sich Undo-Daten an, die f\"ur das Zur\"uckrollen der Transaktionen oder f\"ur Recoveryzwecke gebraucht werden. Auch nach dem Abschluss einer Transaktion ist es notwendig, diese Daten weiterhin vorzuhalten, da sie zur Aufrechterhaltung der Lesekonsistenz anderer Transaktionen wichtig sind.

      Die Undo Retention bestimmt, wie lange Undo-Daten im Undo-Tablespace verweilen. Je nach dem, wie die Datenbank genutzt wird (viele Schreibvorg\"ange, lange Lesevorg\"ange, usw.) ist es wichtig diesen Wert h\"oher oder niedriger einzustellen. Mit der Aktivierung der automatischen Undo-Verwaltung wird auch ein Standardwert von 900 Sekunden f\"ur die Undo Retention gesetzt.      Abh\"angig von der Undo Retention und davon, wie lange ein Datenblock bereits im Undo-Tablespace verweilt, kann er zwei verschiedene Markierungen erhalten:
      \begin{itemize}
        \item \textbf{expired}: Alte Undo-Daten, deren Transaktionen bereits best\"atigt wurden und deren Alter gr\"o\ss{}er ist, als die aktuelle Undo Retention, werden als expired\footnote{expired = engl. abgelaufen} markiert.
        \item \textbf{unexpired}: Undo-Daten, deren Transaktionen bereits best\"atigt wurden, die aber noch j\"unger sind, als die aktuelle Undo Retention, werden als unexpired\footnote{unexpired = engl. noch nicht abgelaufen} markiert.
      \end{itemize}

      Die Undo Retention wird durch Oracle automatisch, basierend auf der Gr\"o\ss{}e des Undo-Tablespace und der aktuellen Aktivit\"aten im System, richtig gesetzt. Eine manuelle Einstellung der Undo Retention kann \"uber den Parameter \parameter{undo\_retention} geschehen. Die Angabe erfolgt in Sekunden. Vorausgesetzt der Undo-Tablespace ist gro\ss{} genug, h\"alt die Datenbank diese Vorgabe ein.

      Wird der Platz f\"ur neue Transaktionen zu knapp, werden als expired markierte Undo-Daten \"uberschrieben. Sollte dann immer noch nicht genug Platz vorhanden sein, beginnt die Datenbank, die als unexpired markierten Daten zu \"uberschreiben. Dies kann dazu f\"uhren, dass die \"uberschriebenen Daten, noch von einer lang laufenden Abfrage ben\"otigt w\"urden, weshalb die Datenbank die Abfrage mit der Fehlermeldung \enquote{Snapshot too old} abbricht.

      Die beiden folgenden Punkte beschreiben noch einmal kurz, wie sich die Undo Retention auf die Lesekonsistenz der Datenbank auswirkt.
\clearpage
			\bild{Die Fehler\-meldung Snapshot too old}{snapshot_too_old}{0.2}
			\begin{itemize}
        \item Hat der Undo-Tablespace eine vordefinierte Gr\"o\ss{}e und kann nicht wachsen (autoextend wurde nicht benutzt), wird der Parameter \parameter{undo\_retention} ignoriert, sobald der Platz im Undo-Tablespace zu knapp wird. In dieser Situation werden als unexpired markierte Undo-Daten \"uberschrieben.
        \item Kann der Undo-Tablespace wachsen, versucht die Datenbank sich an den Vorgabewert f\"ur die Undo Retention zu halten. Wird der Platz im Undo-Tablespace zu knapp wird der Tablespace vergr\"o\ss{}ert, bis eine evtl. definierte Maximalgr\"o\ss{}e erreicht wird. Danach werden unter Umst\"anden wieder als unexpired markierte Undo-Records gel\"oscht.
      \end{itemize}
      \subsection{Automatisches Tuning der Undo Retention}
        Wie bereits erw\"ahnt, versucht Oracle den Wert f\"ur die Undo Retention automatisch zu tunen. Dies geschieht nach den folgenden Vorgaben:
        \begin{itemize}
          \item Hat der Undo-Tablespace eine feste Gr\"o\ss{}e, tuned Oracle die Undo Retention so, das sie f\"ur die Gr\"o\ss{}e des Undo-Tablespaces und die aktuelle Transaktionslast optimal ist.
          \item Wurde der Undo-Tablespace so konfiguriert, das er wachsen kann, versucht Oracle die Undo Retention so zu tunen, das sie gro\ss\ genug f\"ur die bisher gr\"o\ss{}te gelaufene Abfrage ist.
        \end{itemize}
\clearpage
				Die aktuelle \textit{Tuned Undo Retention} kann in der dynamischen Performance View\\ \identifier{v\$undostat} in 10-Minuten-Intervallen \"uber die letzten 4 Tage hinweg abgefragt werden. Der Wert wird dort in Sekunden angegeben. Soll die Tuned Undo Retention \"uber einen l\"angeren Zeitraum, als 4 Tage, hinweg beobachtet werden, kann hierzu die View \identifier{dba\_hist\_undostat} verwendet werden.
        \begin{lstlisting}[caption={Die Tuned Undo Retention abfragen},label=admin508,language=oracle_sql]
SQL> SELECT   TO_CHAR(begin_time, 'DD.MM.YY HH24:MI') AS "Begin Time",
  2           TO_CHAR(end_time,   'DD.MM.YY HH24:MI') AS "End Time",
  3           tuned_undoretention
  4  FROM     v$undostat
  5  ORDER BY end_time;
        \end{lstlisting}
      \bild{Undo Retention}{undo_retention}{0.58}

      \subsection{Retention Guarantee}
        Um den Erfolg von besonders lang laufenden Abfragen garantieren zu k\"onnen, bietet Oracle die \enquote{Retention Guarantee}. Sie wird beim Anlegen des Undo-Tablespaces durch die Klausel \languageorasql{RETENTION GUARANTEE} aktiviert. Ist dies der Fall, werden die als unexpired markierten Undo-Records solange gespeichert, wie sie ben\"otigt werden, um die Lesekonsistenz f\"ur alle laufenden Abfragen aufrecht zu erhalten.
\clearpage
        Die Aufrechterhaltung der Lesekonsistenz geschieht jedoch nicht ohne Kosten. Durch die Retention Guarantee ist es m\"oglich, dass f\"ur DML-Statements bzw. deren Transaktionen nicht mehr gen\"ugend Speicher im Undo-Tablespace vorhanden ist. Eine solche Transaktion muss dann abgebrochen und zur\"uckgerollt werden. Aus diesem Grund wird von Oracle empfohlen, diesen Mechanismus mit Bedacht zu benutzen.

        Um die Retention Guarantee zu deaktivieren gibt es das \languageorasql{ALTER TABLESPACE}-Statement zusammen mit der \languageorasql{RETENTION NOGUARANTEE}-Klausel.
        \begin{lstlisting}[caption={(De-)aktivieren der Retention Guarantee},label=admin509,language=oracle_sql]
SQL> ALTER TABLESPACE undotbs02 RETENTION GUARANTEE;

SQL> ALTER TABLESPACE undotbs02 RETENTION NOGUARANTEE;
        \end{lstlisting}
        Ob f\"ur den Undo-Tablespace die Retention Guarantee aktiviert wurde, kann der View \identifier{dba\_tablespaces} entnommen werden.
        \begin{lstlisting}[caption={Den Status der Retention Guarantee abfragen},label=admin510,language=oracle_sql]
SQL> SELECT tablespace_name, retention
  2  FROM   dba_tablespaces
  3  WHERE  contents = 'UNDO';

TABLESPACE_NAME                &RETENTION&
------------------------------ -----------
UNDOTBS1                       &NOGUARANTEE&
        \end{lstlisting}
    \section{Informationen}
      \subsection{Verzeichnis der relevanten Initialisierungsparameter}
        \begin{literaturinternet}
          \item \cite{REFRN10224}
          \item \cite{REFRN10225}
          \item \cite{REFRN10227}
        \end{literaturinternet}
\clearpage
      \subsection{Verzeichnis der relevanten Data Dictionary Views}
        \begin{literaturinternet}
          \item \cite{sthref3583}
          \item \cite{sthref3800}
          \item \cite{sthref2575}
          \item \cite{sthref3804}
          \item \cite{REFRN23460}
        \end{literaturinternet}
\clearpage
