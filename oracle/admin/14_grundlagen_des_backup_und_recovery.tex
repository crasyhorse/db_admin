  \chapter{Grundlagen des Backup und Recovery}
    \setcounter{page}{1}
    \kapitelnummer{chapter}
    \minitoc
\newpage
    \section{Was ist Backup and Recovery?}
      Im Allgemeinen stehen die Begriffe Backup und Recovery f\"ur verschiedenste Strategien und Arbeitsabl\"aufe, mit deren Hilfe versucht wird, eine Datenbank gegen Datenverlust zu sch\"utzen.
      \subsection{Physische und logische Backups}
        Ein Backup ist eine Kopie der Daten der Datenbank, das dazu genutzt werden kann, die Datenbank wiederherzustellen. Backups werden in physische und logische Backups unterteilt.
        \begin{itemize}
          \item \textbf{Physische Backups}: Dies sind Kopien von Datenbankdateien, d. h. sie enthalten sowohl die Datenbankstruktur, als auch die Daten selbst. Sie k\"onnen in den verschiedensten Formen vorliegen, z. B. komprimiert oder auch als inkrementelles Backup.
          \item \textbf{Logische Backups}: Solche Backups enthalten Nutz- und Metadaten von Datenbank\-objekten, die mit Hilfe von Oracle Utilities aus der Datenbank exportiert wurden. Hierbei handelt es sich nur um Datenexporte, nicht aber um vollst\"andige Backups.
        \end{itemize}
        Physische Backups sind die Grundlage f\"ur jede Backup and Recovery Strategie. Logische Backups stellen in einigen Situationen eine hilfreiche Erg\"anzung zu den physischen Backups dar, sind jedoch kein ernstzunehmender Schutz f\"ur eine Datenbank gegen Datenverlust.

        Dieser Teil der Unterrichtsunterlage k\"ummert sich im Wesentlichen nur um physische Backups. Daher wird mit dem Begriff Backup, wenn nicht anders angegeben, auch immer auf ein physisches Backup verwiesen.
      \subsection{Ursachen die ein Recovery notwendig machen}
        Obwohl es viele Ursachen gibt, die die normale Funktion einer Oracle Datenbank beeintr\"achtigen k\"onnen, gibt es im Wesentlichen nur zwei, die das Eingreifen des Administrators erfordern:
        \begin{itemize}
          \item Medienfehler
          \item Bedienerfehler
        \end{itemize}
        \subsubsection{Bedienerfehler}
          Bedienerfehler liegen immer dann vor, wenn es entweder durch eine fehlerhafte Anwendungslogik oder durch eine direkte Fehleingabe eines Nutzers zu Datenverlust kommt. Dies \"au\ss{}ert sich meist in f\"alschlicherweise ge\"anderten oder gel\"oschten Daten. Obwohl durch Nutzerschulungen, sowie dem vorsichtigen und umsichtigen Umgang mit Privilegien, die meisten Bedienerfehler vermieden werden k\"onnen, bestimmt die Backup und Recovery Strategie, wie einfach und unkompliziert der Administrator die verlorenen Daten wiederherstellen kann.
        \subsubsection{Medienfehler}
          Unter dem Begriff Medienfehler versteht man den Verlust von Daten, aufgrund der Fehlfunktion eines Datentr\"agers. Keine Datenbankdatei ist gegen diese Art von Fehler gesch\"utzt. Die passende Recovery Strategie f\"ur solche F\"alle h\"angt davon ab, welche Datenbankdateien betroffen sind.
    \section{Unterschiedliche Arten physischer Backups mit RMAN}
      Physische Backups k\"onnen nach den unterschiedlichsten Gesichtspunkten unterschieden werden, z. B. in welchem Zustand (ge\"offnet oder geschlossen) war die Datenbank zum Zeitpunkt des Backups, welche Teile der Datenbank wurden gesichert und in welcher Form wird das Backup gespeichert.

      \subsection{Image Copies, Backup Sets und Backup Pieces}
        \subsubsection{Image Copy}
          RMAN (Recovery Manager) kann ein Backup entweder als Image Copy oder als Backup Set speichern. Eine Image Copy ist eine 1:1 Kopie einer Datei, die auch mit Hilfe von Betriebssystemmitteln erstellt werden k\"onnte. Der Vorteil der Nutzung von RMAN bei der Erstellung von Image Copies ist, dass er bei diesem Vorgang den Inhalt der Image Copy auf korrupte Bl\"ocke pr\"ufen kann. Image Copies werden im RMAN Repository registriert.

        \subsubsection{Backup Set}
          Eine andere Form der Speicherung von Backups sind Backup Sets. Ein Backup Set besteht aus mehreren Backup Pieces. Ein Backup Piece ist eine bin\"are Archivdatei, welche mit einer ZIP-Datei verglichen werden kann. Ein mit RMAN durchgef\"uhrter Backup-Job kann eines oder mehrere Backup Sets erzeugen. Backup Sets k\"onnen auch nur durch RMAN wieder zur\"uckgeschrieben werden.

        \begin{merke}
          RMAN kann nur Backup Sets auf Bandlaufwerke \"ubertragen.
        \end{merke}

        \bild{Backup Set und Image Copy}{backupset_and_imagecopy}{0.5}

      \subsection{Konsistente und inkonsistente Backups}
        Physische Backups werden in konsistente Backups (Cold Backup) und inkonsistente Back\-ups (Hot Backup) unterschieden. Ein Backup wird als konsistent bezeichnet, wenn die Datenbank zum Zeitpunkt des Backups in einem konsistenten Zustand war. Das ist dann der Fall, wenn alle \"Anderungen der Redo Logs in die Datendateien \"ubertragen wurden. Dies geschieht nur, wenn die Datenbank ordnungsgem\"a\ss{} heruntergefahren wurde. Eine Datenbank, die aus einem konsistenten Backup wiederhergestellt wurde, kann sofort ohne weiteres Recovery ge\"offnet werden.

        Es k\"onnen aber auch Backups einer ge\"offneten Datenbank durchgef\"uhrt werden. Dabei handelt es sich dann um inkonsistente Backups, sogenannte Hot Backups. Wurde eine Datenbank aus einem inkonsistenten Backup wiederhergestellt, muss in jedem Falle ein Media Recovery durchgef\"uhrt werden, um alle \"Anderungen der Redo Logs in die Datendateien zu \"ubernehmen. Da hier auch archivierte Redo Logs ben\"otigt werden, muss sich die Datenbank im Archivelog-Modus befinden.
      \subsection{Vollst\"andige und inkrementelle Backups}
        Als vollst\"andig werden Backups dann bezeichnet, wenn sie komplette
        Datendateien enthalten. Solche Backups k\"onnen mit dem RMAN oder mit
        Betriebssystemmitteln erzeugt werden. Inkrementelle Backups basieren auf
        der Idee, nur die Datenbl\"ocke einer Datendatei zu sichern, die sich
        seit dem letzten vollst\"andigen Backup ge\"andert haben. Der Vorteil
        dieser Methode ist, dass durch das Zur\"uckschreiben kompletter
        Datenbl\"ocke der Zeitaufwand f\"ur das Zur\"uckspielen der Redo Logs
        erheblich reduziert und somit die gesamte Recovery Phase verringert
        wird. Inkrementelle Backups k\"onnen nur mit RMAN durchgef\"uhrt werden.
\clearpage
    \section{Oracle Backup and Recovery L\"osungen}
      Oracle kennt zwei verschiedene M\"oglichkeiten physische Backups zu erstellen.
      \begin{itemize}
        \item \textbf{Recovery Manager} (RMAN): Der RMAN ist ein Programm, das sowohl \"uber die Kommandozeile als auch \"uber den Enterprise Manager bedient werden kann. Er erstellt eigene Sessions auf dem Datenbankserver, um seine Backup- oder Recovery-Operationen durchzuf\"uhren.

        \item \textbf{User-Managed Backup and Recovery}: Bei dieser Methode wird mit Hilfe von SQL*Plus und Betriebssystemkommandos ein Backup der Datenbank erstellt bzw. zur\"uckgespielt. Hierbei ist der Administrator selbst f\"ur die Verwaltung der Backups zust\"andig.
      \end{itemize}
      Beide Methoden werden von Oracle unterst\"utzt und sind vollst\"andig dokumentiert. Die bevorzugte Variante stellt jedoch der RMAN dar. Er liefert eine einheitliche, betriebssystemunabh\"angige Bedienoberfl\"ache und bietet zu dem M\"oglichkeiten die das User-Managed Backup and Recovery nicht kennt.
      \subsection{Backup and Recovery Features des RMAN}
        Wie bereits erw\"ahnt, besitzt der RMAN einige wesentliche Vorteile gegen\"uber dem User-Managed Backup and Recovery. Die Wissenswertesten davon sind:
        \begin{itemize}
          \item \textbf{Inkrementelle Backups}: Inkrementelle Backups sind kompakter als vollst\"andige Backups und k\"onnen schneller durchgef\"uhrt werden. Sie verk\"urzen auch die Recovery Phase, da durch inkrementelle Backups weniger Redo Log-Informationen zur\"uckgespielt werden m\"ussen.
          \item \textbf{Block media recovery}: Eine Datendatei welche nur einige wenige zerst\"orte Bl\"ocke enth\"alt, kann online repariert werden.
          \item \textbf{Unused block compression}: RMAN kann in bestimmten F\"allen unbenutzte Bl\"ocke bei einem Backup auslassen.
          \item \textbf{Binary compression}: Durch einen integrierten Kompressionsalgorithmus werden Backups komprimiert.
          \item \textbf{Encrypted backups}: Backups k\"onnen verschl\"usselt werden.
        \end{itemize}
\clearpage
        Der RMAN reduziert den administrativen Aufwand bei der Verwaltung von
        Backups, da er ein eigenes Verzeichnis, das \textit{RMAN Repository}
        f\"uhrt, in dem  Informationen \"uber Backups gespeichert werden. RMAN
        kann mit Hilfe dieses Verzeichnisses genau bestimmen, welches das f\"ur
        diese Situation optimalste Backup ist, das zum Restore benutzt werden
        soll. Des Weiteren hat der Administrator die M\"oglichkeit, sich
        Berichte aus dem RMAN Repository ausgeben zu lassen.

        Prim\"ar speichert der RMAN seine ben\"otigten Informationen in der Kontrolldatei der Datenbank. Es kann jedoch auch ein \textit{Recovery Catalog} erstellt werden, um das RMAN Repository zu erweitern. Dies ist ein Datenbankschema, in das der RMAN Informationen \"uber eine oder mehrere Datenbanken speichern kann. F\"ur den RMAN Catalog sollte eine eigenst\"andige Datenbank verwendet werden.
      \subsection{Welche Dateien kann RMAN sichern?}
        RMAN kann alle f\"ur ein Recovery der Datenbank ben\"otigten Dateien sichern. Im Einzelnen sind dies folgende:
        \begin{itemize}
          \item Datendateien und Image Copies von Datendateien
          \item Kontrolldateien und Image Copies von Kontrolldateien
          \item Archivierte Redo Logs
          \item Das aktuelle SPFile
          \item Backup pieces anderer RMAN Backups
        \end{itemize}
        Andere Dateien die f\"ur den Betrieb der Datenbank ben\"otigt werden, wie z. B. Netzwerkkonfigurationsdateien (tnsnames.ora, listener.ora sqlnet.ora) oder die Passwortdatei, k\"onnen nicht durch den RMAN gesichert werden. F\"ur diese Dateien muss ein anderer Backupmechanismus eingesetzt werden.

        Der RMAN kann Backups sowohl auf einem Storage-Laufwerk (Festplatte, RAID usw.), als auch auf Tapes erstellen. Bandlaufwerke werden of auch als SBT\footnote{SBT = System Backup to Tape}-Ger\"ate bezeichnet. RMAN kommuniziert mit SBT-Ger\"aten \"uber einen sogenannten \textit{Media Management Layer}.
    \section{Unterschiedliche Formen des Recovery}
      \subsection{Data File Media Recovery}
        Das Data File Media Recovery oder kurz Media Recovery ist die am h\"aufigsten ben\"otigte Form des Recovery. Es wird durchgef\"uhrt, um verlorene Datendateien, SPFiles oder Kontrolldateien wiederherzustellen. In den folgenden Situation ist ein Media Recovery notwendig:
        \begin{itemize}
          \item Es musste das Backup einer Datendatei wieder hergestellt werden.
          \item Es wurde eine Backup-Kontrolldatei wieder hergestellt.
          \item Eine Datendatei wurde ohne die Option \languageorasql{OFFLINE NORMAL} offline geschaltet
        \end{itemize}
        Damit das Recovery einer Datendatei durchgef\"uhrt werden kann, muss mindestens eine der folgenden Bedingungen erf\"ullt sein:
        \begin{itemize}
          \item Die betreffende Datenbank ist heruntergefahren
          \item Die betreffende Datendatei ist offline, falls die Datenbank nicht heruntergefahren werden kann.
        \end{itemize}
        Eine Datendatei die ein Recovery ben\"otigt, kann erst in den Online-Status gebracht werden, wenn das Recovery abgeschlossen wurde. Eine Datenbank kann nicht ge\"offnet werden, wenn eine ihrer Datendateien ein Recovery ben\"otigt.
        \subsubsection{Die Phasen eines Recovery-Prozesses}
          Das Wiederherstellen von Teilen einer Datenbank oder einer kompletten Datenbank wird in zwei Phasen unterteilt:
          \begin{itemize}
            \item \textbf{Restore}: Unter dem Begriff Restore versteht man das Zur\"uckspielen eines Backups auf den Datentr\"ager.
            \item \textbf{Recovery}: Als Recovery bezeichnet man den Prozess des Aktualisierens der zu\-r\"uck\-ge\-spielten Datenbank\-dateien, mit den Informationen aus den (archivierten) Redo Log Dateien.
          \end{itemize}
          \abbildung{backup_and_recovery_basics} zeigt das Grundprinzip von Backup, Restore und Recovery.

          \bild{Grund\-prin\-zi\-pien des Backup and Recovery}{backup_and_recovery_basics}{1}

          In diesem Beispiel wird bei SCN\footnote{SCN = System Change Number} 75 ein vollst\"andiges Backup der Datenbank gemacht. Die von der Datenbank erzeugten Redo Logs enthalten alle \"Anderungen von SCN 75 bis SCN 666. Gef\"ullte Redo Logs werden archiviert. Bei SCN 666 gehen Datendateien der Datenbank durch einen Medienfehler verloren. Durch ein Restore wird die Datenbank dann auf den Stand von SCN 75 zur\"uckgebracht. Beim anschlie\ss{}enden Recovery, mit Hilfe der Redo Logs und der archivierten Redo Logs, wird die Datenbank wieder auf den Stand von SCN 666 \"uberf\"uhrt.
        \subsubsection{Vollst\"andiges und unvollst\"andiges Recovery}
          Beim Data File Media Recovery unterscheidet man zwei verschiedene Arten:
          \begin{itemize}
            \item Vollst\"andiges Recovery
            \item Unvollst\"andiges Recovery
          \end{itemize}
          \begin{merke}
            Unter einem vollst\"andigen Recovery versteht man das Recovern der gesamten Datenbank oder auch nur von Teilen der Datenbank, bis auf den neuesten Stand. D. h. die Datenbank wird ohne den Verlust von abgeschlossenen Transaktionen wiederhergestellt.
          \end{merke}

          In einigen F\"allen kann es jedoch notwendig sein, die Datenbank bis
          zu einem ganz bestimmten Zeitpunkt zur\"uckzusetzen. Wenn
          beispielsweise durch einen Bedienerfehler eine Tabelle gel\"oscht
          wurde, die noch ben\"otigte Informationen enth\"alt, muss die
          Datenbank bis zu dem Zeitpunkt direkt vor dem Bedienerfehler
          zur\"uckgesetzt werden, so dass die betreffende Tabelle wieder
          existiert.
\clearpage
          Diese Form des Recovery wird als unvollst\"andiges Recovery
          oder als Point-In-Time-Recovery bezeichnet. Ziel dieser Recoveryform
          ist es, die Datenbank auf eine ganz bestimmte SCN oder einen
          bestimmten Zeitpunkt zur\"uckzusetzen, um die Folgen von
          Bedienerfehlern zu beheben.

          Point-In-Time-Recovery stellt auch die einzige Reaktionsm\"oglichkeit dar, wenn der Administrator feststellt, das eines oder mehrere archivierte Redo Logs verloren gegangen sind, die f\"ur ein vollst\"andiges Recovery ben\"otigt w\"urden.

          \begin{merke}
            Bemerkt der Administrator im laufenden Betrieb der Datenbank, dass Redo Log Dateien verloren gegangen sind, sollte augenblicklich ein vollst\"andiges Backup der Datenbank durchgef\"uhrt werden.
          \end{merke}
      \subsection{Instance Recovery / Crash Recovery}
        Wird eine Oracle Datenbank neu gestartet, pr\"uft der SMON, ob ein automatisches Recovery der Datendateien notwendig ist. Ziel dieser Form des Recovery ist es, die Datenbank vor dem \"Offnen auf einen konsistenten Stand zu bringen, ohne abgeschlossene Transaktionen zu verlieren.

        Instance Recovery und Media Recovery haben \"ahnliche Ziele. Es existieren jedoch auch einige Unterschiede zwischen den beiden Formen.
        \begin{itemize}
          \item Media Recovery muss durch einen Nutzer eingeleitet werden, es wird niemals automatisch durchgef\"uhrt.
          \item Media Recovery behandelt nur die durch ein Restore wiederhergestellten Datendateien, jedoch keine Datendateien die w\"ahrend des Crashes online waren.
          \item Media Recovery ben\"otigt die Online Redo Logs und die archivierten Redo Logs um die Datenbank vollst\"andig wiederherstellen zu k\"onnen.
        \end{itemize}
        Im Gegensatz zum Media Recovery ben\"otigt das Instance Recovery nur die Online Redo Logs und es wirkt sich nur auf die Datendateien aus, die w\"ahrend des Neustarts der Datenbank den Status online hatten. Es werden keine archivierten Redo Logs ben\"otigt und es wird auch kein Restore durchgef\"uhrt.

        Die Datenbank rollt beim Instance Recovery alle offenen Transaktionen zur\"uck (Rollback-Phase) und wendet anschlie\ss{}end den Inhalt der Online Redo Logs auf die Datendateien an (Rollforward-Phase). So wird die Datenbank auf den neuesten Stand vor dem Crash gebracht.
