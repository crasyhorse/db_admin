    \item Setzen Sie das folgende SQL-Statement, unter Benutzung des Result
    Caches ab. Da Sie dieses Statement mehrfach werden ausf\"uhren m\"ussen,
    sollten Sie es in einer SQL-Datei speichern.
    \begin{lstlisting}[language=oracle_sql,alsolanguage=sqlplus]
SQL> SELECT /*+ result_cache */ *
  2  FROM   (SELECT TO_CHAR(Buchungsdatum, 'Q') AS Quartal,
  3                 TO_CHAR(Buchungsdatum, 'YYYY') AS Datum, Betrag
  4          FROM   bank.Buchung)
  5  PIVOT  (SUM(Betrag) FOR Datum IN ('1985', '1986', '1987', '1988', 
                                       '1989'));
    \end{lstlisting}
