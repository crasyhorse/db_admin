\section{L\"osungen - Recovery mit dem RMAN}
  \begin{enumerate}
        \item Ermitteln Sie, ob der serverseitige Result Cache aktiviert ist!

      \begin{lstlisting}[caption={Ermitteln des betroffenen Redo Log Members im Alert Log},language=terminal]
Errors in file /u01/app/oracle/diag/rdbms/orcl/orcl/trace/orcl_arc0_3513.trc
ORA-00313: &open& failed for members of log group 5 of thread 1
ORA-00312: online log 5 thread 1: '/u01/app/oracle/oradata/orcl/redo05a.log'
ORA-27037: unable to obtain file status
Linux-x86_64 Error: 2: No such file or directory
      \end{lstlisting}
      \begin{lstlisting}[caption={Recovern des Members},language=terminal]
[oracle@FEA11-119SRV ~]$ cd /u01/app/oracle/oradata/orcl
[oracle@FEA11-119SRV ~]$ cp /u02/oradata/orcl/redo05b.log redo05a.log
      \end{lstlisting}
        \item Welchen temporären Tablespace nutzt alice jetzt und welcher View können Sie diese Angabe entnehmen?

        \item Erstellen Sie den Nutzer \identifier{bob} mit dem Passwort \enquote{Pass1/gH,3word}.

        \item Welchen Default Tablespace benutzt \identifier{bob} und aus welcher View können Sie diese Angabe entnehmen?

      \begin{lstlisting}[caption={Ermitteln des betroffenen Tempfiles im Alert Log},language=terminal]
Errors in file /u01/app/oracle/diag/rdbms/orcl/orcl/trace/orcl_m000_7480.trc
ORA-01116: error in opening database file 201
ORA-01110: data file 201: '/u01/app/oracle/oradata/orcl/temp01.dbf'
ORA-27041: unable to &open& file
Linux-x86_64 Error: 2: No such file or directory
Additional information: 3
      \end{lstlisting}
      \begin{lstlisting}[caption={Neues Tempfile erstellen},language=oracle_sql]
SQL> ALTER TABLESPACE temp
  2  ADD TEMPFILE '/u02/oradata/ORCL/temp02.dbf'
  3  SIZE 20 M AUTOEXTEND ON MAXSIZE 500M;
      \end{lstlisting}
\clearpage
      \begin{lstlisting}[caption={Besch\"adigtes Tempfile l\"oschen},language=oracle_sql]
SQL> ALTER TABLESPACE temp
  2  DROP TEMPFILE '/u02/oradata/ORCL/temp01.dbf';
      \end{lstlisting}
        \item F\"uhren Sie ein Recovery bei Verlust einer Kontrolldatei durch.
      \begin{enumerate}
        \item Starten Sie das Skript \oscommand{lab\_delete\_ctlfl.sql}. Es l\"oscht eine Ihrer Kontrolldateien.
          \begin{lstlisting}[language=terminal]
SQL> @/home/oracle/labs/lab_delete_ctlfl.sql
          \end{lstlisting}
        \item Analysieren Sie das Problem und f\"uhren Sie ein geeignetes Recovery durch.
      \end{enumerate}

      \begin{lstlisting}[caption={Ermitteln der betroffenen Kontrolldatei im Alert Log},language=terminal]
Errors in file /u01/app/oracle/diag/rdbms/orcl/orcl/trace/orcl_ora_2765.trc:
ORA-00210: cannot &open& the specified control file
ORA-00202: control file: '/u01/app/oracle/oradata/orcl/control01.ctl'
ORA-27041: unable to &open& file
Linux-x86_64 Error: 2: No such file or directory
      \end{lstlisting}
      \begin{lstlisting}[caption={Recovern der Kontrolldatei},language=terminal]
[oracle@FEA11-119SRV ~]$ cd /u01/app/oracle/oradata/orcl
[oracle@FEA11-119SRV ~]$ cp control02.ctl control01.ctl
      \end{lstlisting}
        \item Machen Sie alle Auditingeinstellungen r\"uckg\"angig und l\"oschen Sie den Inhalt des Auditingtrails!

      \begin{lstlisting}[caption={Recovern der Datendatei},language=rman]
RMAN> VALIDATE database;
Starting validate at 03-NOV-13
using target database control file instead of recovery catalog
allocated channel: ORA_DISK_1
channel ORA_DISK_1: SID=21 device type=&DISK&
allocated channel: ORA_DISK_2
channel ORA_DISK_2: SID=144 device type=&DISK&
RMAN-06169: could not read file header for datafile 6 error reason 5
RMAN-00571: ===========================================================
RMAN-00569: =============== ERROR MESSAGE STACK FOLLOWS ===============
RMAN-00571: ===========================================================
RMAN-03002: failure of validate command at 11/03/2013 09:24:26
RMAN-06056: could not access datafile 6
      \end{lstlisting}
\clearpage
      \begin{lstlisting}[language=rman,emph={[9]ALTER,DATABASE, DATAFILE,OFFLINE,ONLINE},emphstyle={[9]\color{magenta}\bfseries}]
RMAN> SQL 'ALTER DATABASE DATAFILE 6 OFFLINE';

RMAN> RESTORE datafile 6;

RMAN> RECOVER datafile 6;

RMAN> SQL 'ALTER DATABASE DATAFILE 6 ONLINE';
      \end{lstlisting}
    \item Konfigurieren Sie Ihre Instanz MSSQLSERVER so, dass Sie immer
mindestens 200M Arbeitsspeichern zur Verfügung hat und nie mehr als 2G
benutzt!

      \begin{lstlisting}[caption={Recovern der Systemdatendatei},language=rman,alsolanguage=sqlplus,emph={[9]ALTER,DATABASE,OPEN},emphstyle={[9]\color{magenta}\bfseries}]
RMAN> VALIDATE database; 
Starting validate at 03-NOV-13
released channel: ORA_SBT_TAPE_1
using channel ORA_DISK_1
using channel ORA_DISK_2
channel ORA_DISK_1: starting validation of datafile
channel ORA_DISK_1: specifying datafile(s) for validation
RMAN-00571: ===========================================================
RMAN-00569: =============== ERROR MESSAGE STACK FOLLOWS ===============
RMAN-00571: ===========================================================
RMAN-03009: failure of validate command on ORA_DISK_1 channel at 11/03/2013
ORA-01122: database file 1 failed verification check
ORA-01110: data file 1: '/u01/app/oracle/oradata/orcl/system01.dbf'
ORA-01565: error in identifying file 
           '/u01/app/oracle/oradata/orcl/system01.dbf'
ORA-27037: unable to obtain file status
Linux-x86_64 Error: 2: No such file or directory
Additional information: 3

RMAN> shutdown abort

RMAN> startup mount

RMAN> RESTORE datafile 1;

RMAN> RECOVER datafile 1;

RMAN> ALTER DATABASE OPEN;
      \end{lstlisting}
\clearpage
    \item Legen Sie einen nicht gruppierten Index auf die Tabelle
\identifier{eigenkunde} (Spalte \identifier{ablaufdatum}). Der Index soll nur
die Zeilen enthalten, die einen Personalausweis zeigen, der vor dem 01.01.2016
abgelaufen ist. Schließen Sie die Spalte \identifier{personalausweisnr} als
Nichtschlüsselspalte in den Index ein.
      \begin{lstlisting}[caption={Herunterfahren der Datenbank in SQL*Plus},language=oracle_sql,alsolanguage=sqlplus]
SQL> shutdown abort
      \end{lstlisting}
      \begin{lstlisting}[caption={Recovern der Kontrolldateien},language=rman,alsolanguage=sqlplus,emph={[9]ALTER,DATABASE, MOUNT,OPEN,RESETLOGS},emphstyle={[9]\color{magenta}\bfseries}]
RMAN> SET DBID 1351916467;
RMAN> startup nomount
RMAN> RESTORE controlfile
2>    FROM AUTOBACKUP
3>    RECOVERY AREA '/u05/fast_recovery_area'
4>    DB_NAME = 'orcl';

Starting restore at 03-NOV-13
allocated channel: ORA_DISK_1
channel ORA_DISK_1: SID=134 device type=&DISK&

recovery area destination: /u05/fast_recovery_area
database name (or database unique name) used for search: ORCL
channel ORA_DISK_1: &AUTOBACKUP& /u05/fast_recovery_area/ORCL/autobackup/
channel ORA_DISK_1: looking for &AUTOBACKUP& on day: 20131103
channel ORA_DISK_1: restoring control file from &AUTOBACKUP&
channel ORA_DISK_1: control file restore from &AUTOBACKUP& complete
output file name=/u01/app/oracle/oradata/orcl/control01.ctl
output file name=/u05/fast_recovery_area/orcl/control02.ctl
Finished restore at 03-NOV-13

RMAN> SQL 'ALTER DATABASE MOUNT';

RMAN> RECOVER database;

RMAN> SQL 'ALTER DATABASE OPEN RESETLOGS';
      \end{lstlisting}

        \item Konfigurieren Sie zwei Net Service Names, einen f\"ur eine Dedicated Server Verbindung und einen f\"ur eine DRCP-Verbindung. Testen Sie beide Verbindungen (Hinweis: Um die DRCP-Verbindung testen zu k\"onnen, muss vorher das Connection Pooling aktiviert werden, siehe: \myhref{http://docs.oracle.com/cd/B28359_01/server.111/b28318/process.htm#CIHBIFCI}{Database Resident Connection Pooling, Database Concepts 11g Release 2, Kapitel 9})

        \item Schalten Sie das SQL-Autotracing und das Timing wieder aus: \languagesqlplus{set autotrace off} und \languagesqlplus{set timing off}

      \begin{lstlisting}[caption={Validieren der Datenbank},language=rman]
RMAN> VALIDATE database;

      \end{lstlisting}
        \item Weisen Sie den Nutzern \identifier{alice}, \identifier{bob} und \identifier{chloe} das Profil \identifier{p\_clerk} zu und testen Sie die Auswirkungen!

      \begin{lstlisting}[caption={Validieren der Datenbank},language=rman]
RMAN> RUN {
2>      SET MAXCORRUPT FOR datafile 7 TO 999;
3>      BACKUP AS BACKUPSET DATAFILE 7;
4>    }
      \end{lstlisting}
        \item Melden Sie sich als Nutzer \identifier{alice} (Passwort: Welcome01\#) an Ihrer Datenbank an und versuchen Sie das folgende SQL-Statement abzusetzen:
    \begin{lstlisting}[language=oracle_sql]
&SQL>& INSERT INTO full
  2 VALUES (1,1,1);
    \end{lstlisting}
    Die Ausf\"uhrung des Statements wird scheitern. Pr\"ufen Sie im Alert.log, warum das Statement gescheitert ist.

      \begin{lstlisting}[caption={Recovern der Datendatei},language=rman]
RMAN> BLOCKRECOVER CORRUPTION LIST;
      \end{lstlisting}
  \end{enumerate}