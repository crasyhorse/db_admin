  \section{L\"osungen - Verwalten einer Oracle Instanz}
    \begin{enumerate}
          \item Ermitteln Sie, ob der serverseitige Result Cache aktiviert ist!

        \begin{lstlisting}[language=sqlplus]
sqlplus / as sysdba
        \end{lstlisting}
          \item Welchen temporären Tablespace nutzt alice jetzt und welcher View können Sie diese Angabe entnehmen?

        \begin{lstlisting}[language=sqlplus]
SQL> startup nomount
        \end{lstlisting}
          \item Erstellen Sie den Nutzer \identifier{bob} mit dem Passwort \enquote{Pass1/gH,3word}.

        \begin{lstlisting}[language=oracle_sql]
SQL> ALTER DATABASE MOUNT;
        \end{lstlisting}
          \item Welchen Default Tablespace benutzt \identifier{bob} und aus welcher View können Sie diese Angabe entnehmen?

        \begin{lstlisting}[language=oracle_sql]
SQL> ALTER DATABASE OPEN;
        \end{lstlisting}
          \item F\"uhren Sie ein Recovery bei Verlust einer Kontrolldatei durch.
      \begin{enumerate}
        \item Starten Sie das Skript \oscommand{lab\_delete\_ctlfl.sql}. Es l\"oscht eine Ihrer Kontrolldateien.
          \begin{lstlisting}[language=terminal]
SQL> @/home/oracle/labs/lab_delete_ctlfl.sql
          \end{lstlisting}
        \item Analysieren Sie das Problem und f\"uhren Sie ein geeignetes Recovery durch.
      \end{enumerate}

        \begin{lstlisting}[language=sqlplus]
SQL> shutdown immediate
        \end{lstlisting}
          \item Machen Sie alle Auditingeinstellungen r\"uckg\"angig und l\"oschen Sie den Inhalt des Auditingtrails!

        \begin{lstlisting}[language=sqlplus]
SQL> startup
        \end{lstlisting}
      \item Konfigurieren Sie Ihre Instanz MSSQLSERVER so, dass Sie immer
mindestens 200M Arbeitsspeichern zur Verfügung hat und nie mehr als 2G
benutzt!

        \begin{lstlisting}[language=sqlplus]
SQL> show parameter memory_target
        \end{lstlisting}
       \item Legen Sie einen nicht gruppierten Index auf die Tabelle
\identifier{eigenkunde} (Spalte \identifier{ablaufdatum}). Der Index soll nur
die Zeilen enthalten, die einen Personalausweis zeigen, der vor dem 01.01.2016
abgelaufen ist. Schließen Sie die Spalte \identifier{personalausweisnr} als
Nichtschlüsselspalte in den Index ein.
        \begin{lstlisting}[language=oracle_sql]
SQL> ALTER SYSTEM
  2  SET memory_target = 800M;
        \end{lstlisting}
          \item Konfigurieren Sie zwei Net Service Names, einen f\"ur eine Dedicated Server Verbindung und einen f\"ur eine DRCP-Verbindung. Testen Sie beide Verbindungen (Hinweis: Um die DRCP-Verbindung testen zu k\"onnen, muss vorher das Connection Pooling aktiviert werden, siehe: \myhref{http://docs.oracle.com/cd/B28359_01/server.111/b28318/process.htm#CIHBIFCI}{Database Resident Connection Pooling, Database Concepts 11g Release 2, Kapitel 9})

        \begin{lstlisting}[language=oracle_sql]
SQL> ALTER SYSTEM
  2  SET sga_target = 400M
  3  SCOPE=spfile;
SQL> ALTER SYSTEM
  2  SET pga_aggregate_target = 100M
  3  SCOPE=spfile;
        \end{lstlisting}
\clearpage
          \item Schalten Sie das SQL-Autotracing und das Timing wieder aus: \languagesqlplus{set autotrace off} und \languagesqlplus{set timing off}

        \begin{lstlisting}[language=sqlplus]
SQL> shutdown immediate
SQL> startup
        \end{lstlisting}
          \item Weisen Sie den Nutzern \identifier{alice}, \identifier{bob} und \identifier{chloe} das Profil \identifier{p\_clerk} zu und testen Sie die Auswirkungen!


      Antwort: Beide werden als Mindestwerte f\"ur die Gr\"o\ss{}e der SGA
      (\parameter{sga\_target}) und der PGAs
      (\parameter{pga\_aggregate\_target}) herangezogen. Der restliche
      Speicherplatz von 300M (\parameter{memory\_target} -
      \parameter{sga\_target} - \parameter{pga\_aggregate\_target}) kann
      dynamisch auf beide Strukturen verteilt werden.
          \item Melden Sie sich als Nutzer \identifier{alice} (Passwort: Welcome01\#) an Ihrer Datenbank an und versuchen Sie das folgende SQL-Statement abzusetzen:
    \begin{lstlisting}[language=oracle_sql]
&SQL>& INSERT INTO full
  2 VALUES (1,1,1);
    \end{lstlisting}
    Die Ausf\"uhrung des Statements wird scheitern. Pr\"ufen Sie im Alert.log, warum das Statement gescheitert ist.

        Beachten Sie: Die Zahlen in der folgenden L\"osung k\"onnen bei Ihnen anders sein.
        \begin{lstlisting}[language=oracle_sql]
SQL> SELECT name, bytes
  2  FROM   v$sgainfo
  3  WHERE  name IN ('Buffer Cache Size', 'Shared Pool Size',
  4                  'Large Pool Size', 'Maximum SGA Size');

NAME                                  BYTES
-------------------------------- ----------
Buffer Cache Size                 272629760
Shared Pool Size                  125829120
Large Pool Size                     4194304
Maximum SGA Size                  835104768
        \end{lstlisting}
          \item Ändern Sie die Sperrdauer des Accounts im Profil \identifier{p\_clerk} auf unbegrenzt!

        \begin{lstlisting}[language=oracle_sql]
SQL> ALTER SESSION
  2  SET NLS_LANGUAGE='GERMAN';
        \end{lstlisting}
          \item L\"oschen Sie das Nutzerprofil \identifier{p\_clerk} in einem Arbeitsschritt!

        \begin{lstlisting}[language=oracle_sql]
SQL> CREATE PFILE = '/home/oracle/initorcl.ora'
  2  FROM   SPFILE;
        \end{lstlisting}
      \item Lesen Sie den Artikel \parencite{utilUDPaC} und notieren Sie sich
mindestens drei sinnvolle Fragen!

  \rule{0.94\textwidth}{0.5pt}

  \rule{0.94\textwidth}{0.5pt}

  \rule{0.94\textwidth}{0.5pt}

  \rule{0.94\textwidth}{0.5pt}

  \rule{0.94\textwidth}{0.5pt}

  \rule{0.94\textwidth}{0.5pt}

  \rule{0.94\textwidth}{0.5pt}

  \rule{0.94\textwidth}{0.5pt}

        \begin{lstlisting}[language=oracle_sql]
SQL> CREATE SPFILE = '/home/oracle/spfileorcl.ora'
  2  FROM   MEMORY;
        \end{lstlisting}
\clearpage
          \item Verschieben Sie die Datendatei \oscommand{uebungs\_ts02.dbf} nach \oscommand{/u02/oradata/orcl/}.

        \begin{lstlisting}[language=oracle_sql,alsolanguage=sqlplus]
SQL> shutdown immediate
SQL> startup mount
SQL> col name format a50
SQL> SELECT name
  2  FROM   v$controlfile;

NAME
-------------------------------------------------------------
/u01/app/oracle/oradata/orcl/control01.ctl
/u05/fast_recovery_area/orcl/control02.ctl

SQL> shutdown immediate
SQL> host cp /u01/app/oracle/oradata/orcl/control01.ctl
            /u02/oradata/orcl/control03.ctl
SQL> startup nomount
SQL> ALTER SYSTEM
  2  SET control_files = '/u01/app/oracle/oradata/orcl/control01.ctl',
                         '/u05/fast_recovery_area/orcl/control02.ctl',
                         '/u02/oradata/orcl/control03.ctl'
  3  SCOPE=spfile;
SQL> shutdown immediate
SQL> startup
        \end{lstlisting}
            \item Um die weiteren Übungsaufgeben bearbeiten zu können, führen Sie bitte das Skript \oscommand{lab\_java.sql} als Nutzer \identifier{SYS} aus. Das SQL-Skript befindet sich im Verzeichnis \oscommand{/home/oracle/labs}.
        \begin{lstlisting}[language=terminal]
SQL> @/home/oracle/labs/lab_java.sql
        \end{lstlisting}

      \item Ermitteln Sie welchen Wert die Datenbankoption \identifier{auto\_close}
hat und recherchieren Sie, welche Bedeutung diese Option hat bzw. was sie
bewirkt.

        \begin{lstlisting}[language=oracle_sql,alsolanguage=sqlplus]
SQL> col member format a50
SQL> SELECT   group&\#&, member
  2  FROM     v$logfile
  3  ORDER BY 1;

    GROUP&\#& MEMBER
---------- --------------------------------------------------
         1 /u01/app/oracle/oradata/orcl/redo01.log
         2 /u01/app/oracle/oradata/orcl/redo02.log
         3 /u01/app/oracle/oradata/orcl/redo03a.log
         4 /u01/app/oracle/oradata/orcl/redo04a.log
         5 /u02/oradata/orcl/redo05b.log
         5 /u01/app/oracle/oradata/orcl/redo05a.log
        \end{lstlisting}
\clearpage
            \item Passen Sie das Setup Ihrer Redo Log Gruppen so an, dass Sie drei Gruppen mit je 3 Membern haben. Verteilen Sie die Member sinnvoll \"uber die vorhandenen Datentr\"ager \oscommand{/u01} bis \oscommand{/u05} und notieren Sie sich die \"Anderungen in der obigen Tabelle.

        Beachten Sie: Die im folgenden gezeigte L\"osung muss f\"ur Ihr System u. U. angepasst werden!
        \begin{lstlisting}[language=oracle_sql,alsolanguage=sqlplus]
SQL> shutdown immediate
SQL> host mv /u01/app/oracle/oradata/orcl/redo01.log
             /u01/app/oracle/oradata/orcl/redo01a.log
SQL> host mv /u01/app/oracle/oradata/orcl/redo02.log
             /u02/oradata/orcl/redo02a.log
SQL> host mv /u01/app/oracle/oradata/orcl/redo03.log
             /u03/oradata/orcl/redo03a.log
SQL> startup mount
SQL> ALTER DATABASE
  2  RENAME FILE '/u01/app/oracle/oradata/orcl/redo01.log'
  3           TO '/u01/app/oracle/oradata/orcl/redo01a.log';
SQL> ALTER DATABASE
  2  ADD LOGFILE MEMBER '/u02/oradata/orcl/redo01b.log'
  3  TO GROUP 1;
SQL> ALTER DATABASE
  2  ADD LOGFILE MEMBER '/u03/oradata/orcl/redo01c.log'
  3  TO GROUP 1;
SQL> ALTER DATABASE
  2  RENAME FILE '/u01/app/oracle/oradata/orcl/redo02.log'
              TO '/u02/oradata/orcl/redo02a.log';
SQL> ALTER DATABASE
  2  ADD LOGFILE MEMBER '/u03/oradata/orcl/redo02b.log'
  3  TO GROUP 2;
SQL> ALTER DATABASE
  2  ADD LOGFILE MEMBER '/u01/app/oracle/oradata/orcl/redo02c.log'
  3  TO GROUP 2;
SQL> ALTER DATABASE
  2  RENAME FILE '/u01/app/oracle/oradata/orcl/redo03a.log'
              TO '/u03/oradata/orcl/redo03a.log';
SQL> ALTER DATABASE
  2  ADD LOGFILE MEMBER '/u01/app/oracle/oradata/orcl/redo03b.log'
  3  TO GROUP 3;
SQL> ALTER DATABASE
  2  ADD LOGFILE MEMBER '/u02/oradata/orcl/redo03c.log'
  3  TO GROUP 3;
        \end{lstlisting}
\clearpage
          \item Pr\"ufen Sie im RMAN welche Backups nun nicht mehr zur Verf\"ugung stehen.

        \begin{lstlisting}[language=oracle_sql,alsolanguage=sqlplus]
SQL> SELECT log_mode
  2  FROM   v$database;

LOG_MODE
------------
&ARCHIVELOG&

-- Nur falls notwendig ausf&\"u&hren
SQL> shutdown immediate
SQL> startup mount
SQL> ALTER DATABASE ARCHIVELOG;
        \end{lstlisting}
            \item Ändern Sie die Parameter der Archivierung wie folgt:
        \begin{center}
          \begin{small}
            \tablefirsthead{
              \multicolumn{1}{c}{\textbf{Optional}} &
              \multicolumn{1}{c}{\textbf{/u02}} &
              \multicolumn{1}{c}{\textbf{/u03}} &
              \multicolumn{1}{c}{\textbf{/u04}} &
              \multicolumn{1}{c}{\textbf{/u05}} \\
              \hline
            }
            \tablefirsthead{
              \multicolumn{1}{c}{\textbf{Optional}} &
              \multicolumn{1}{c}{\textbf{/u02}} &
              \multicolumn{1}{c}{\textbf{/u03}} &
              \multicolumn{1}{c}{\textbf{/u04}} &
              \multicolumn{1}{c}{\textbf{/u05}} \\
              \hline
            }
            \tabletail{
              \hline
            }
            \tablelasttail {
              \hline
            }
            \begin{supertabular}{| >{\centering\arraybackslash}m{1cm}| >{\centering\arraybackslash}m{2cm}| >{\centering\arraybackslash}m{2cm}| >{\centering\arraybackslash}m{2cm}| >{\centering\arraybackslash}m{2cm}|}
               Ja &  dest 1 &  - &  - & - \\
              \hline
               Nein &  - &  dest 2 &  - & - \\
              \hline
               Ja &  - &  - &  dest 3 & - \\
            \end{supertabular}
          \end{small}
        \end{center}
        \begin{itemize}
          \item Die Archivierung muss an mindestens zwei Speicherorten erfolgreich sein.
          \item Das Dateinamensformat der Archive Log Files muss wie folgt aufgebaut sein: \oscommand{archive\_\%d\_\%t\_\%s\_\%r.arc}
        \end{itemize}

        \begin{lstlisting}[language=oracle_sql,alsolanguage=sqlplus]
SQL> ALTER SYSTEM
  2  SET log_archive_dest_1 = 'LOCATION=/u02/archive OPTIONAL';
SQL> ALTER SYSTEM
  2  SET log_archive_dest_2 = 'LOCATION=/u03/archive MANDATORY';
SQL> ALTER SYSTEM
  2  SET log_archive_dest_3 = 'LOCATION=/u04/archive OPTIONAL';
SQL> ALTER SYSTEM
  2  SET log_archive_min_succeed_dest=2;
SQL> ALTER SYSTEM
  2  SET log_archive_format='archive_%d_%t_%s_%r.arc' SCOPE=spfile;
SQL> shutdown immediate
SQL> startup
        \end{lstlisting}
    \end{enumerate}
\clearpage
