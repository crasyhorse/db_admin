\chapter{Entwicklung eines Datenmodells}
\chaptertoc{}
\cleardoubleevenpage

In diesem Abschnitt erfolgt die Vermittlung der Kenntnisse darüber, wie man aus einer großen Menge von Informationen und Anforderungen an eine neue Datenbank eine Datenstruktur entwirft. Im Zuge der Datenmodellierung soll ein exaktes und vollständiges Modell des betrachteten Realitätsausschnitts erarbeitet werden, welches den Rahmen für die Entwicklung neuer oder die Erweiterung bestehender Anwendungssysteme bildet.

Als Werkzeug für die Erstellung konzeptioneller Datenmodelle wird die Entity-Relation\-ship Modellierung (ER-Modellierung), zuerst in ihrer einfachsten Art und später in einer erweiterten Fassung, verwendet. Es werden dabei alle vier Phasen des Datenmodellierungsprozesses anhand eines durchgängigen Beispiels beschrieben.

Die ER-Modellierung stellt eine spezielle \enquote{Information-Engineering-Technik} dar, die zur Erstellung von Datenmodellen hoher Qualität benutzt wird. Entworfen in den 70er Jahren von Peter Pin-Shan Chen wurde sie seither vielfach erweitert und verbessert.

Ziel eines konzeptionellen Datenmodells ist es, die typmäßige Struktur der Daten in der Datenbank ohne deren Inhalt zu beschreiben. Als Beispiel aus der realen Welt wäre  die Ausstattung eines Büros mit Möbeln zu nennen. Wer später das Büro nutzt und mit welchem Inhalt die Schränke gefüllt werden, ist für die Erstellung des Modells Bedeutungslos.

Die Entwicklung eines Datenmodells teilt sich in die folgenden vier Phasen ein:
\begin{enumerate}
    \item Klassifizierung von Objekten
    \item Festlegung relevanter Eigenschaften
    \item Bestimmung identifizierender Eigenschaften
    \item Beschreibung sachlogischer Zusammenhänge zwischen den einzelnen Objekten
\end{enumerate}
\section{Die Modellierungsinformationen}
Die Struktur der Bundeswehr soll in einem ER-Modell dargestellt werden.
Es wird jedoch nur ein Ausschnitt aus der Realität betrachtet, um die
Übersichtlichkeit der Datenstrukturen zu gewährleisten. Im Folgenden
werden die dafür notwendigen Objekte festgelegt.
\clearpage
\begin{enumerate}
    \item Die Bundeswehr besitzt zahlreiche Dienststellen an den unterschiedlichsten Standorten, welche sich untereinander über- oder untergeordnet sind. Jede Dienststelle besteht aus Dienstposten. Diese wiederum werden mit Soldaten besetzt. Jeder Soldat empfängt bei Einstieg in die Bundeswehr seine persönliche Ausrüstung und besitzt diese dann für die Dauer seines Dienstverhältnisses.
    \item Jeder Dienststelle der Bundeswehr ist eine Dienststellennummer zugeordnet, über die diese identifiziert werden kann. Des Weiteren hat jede Dienststelle eine bestimmte Größe sowie Bezeichnung, wie z. B. Führungsunterstützungsschule der Bundeswehr.
    \item Für die Standorte, an denen sich die Dienststellen befinden, müssen in der Datenbank Postleitzahl (PLZ), Ort, Straße und die Hausnummer hinterlegt sein. Hierbei ist zu beachten, dass sich eine Dienststelle auch an mehreren Standorten befinden kann.
    \item Die den Dienststellen untergeordneten Dienstposten werden durch ihre Dienstposten\_ID und ein Beginn- und Enddatum charakterisiert. Als eine weitere Eigenschaft soll eine kurze Dienstpostenbeschreibung hinterlegt werden.
    \item Jeder in der Datenbank aufgeführte Soldat soll dort mit seiner Personanlnummer, einer Personenkennziffer, sowie Vor- und Nachnamen und Dienstgrad aufgeführt werden. Weiterhin kann der Anwender auch die aktuelle Adresse, bestehend aus PLZ, Wohnort, Straße und Hausnummer, aus der Datenbank heraussuchen.
    \item Das letzte Objekt in der Datenbank stellt die persönliche Ausrüstung des Soldaten dar. Diese besitzt eine Bezeichnung, ist aus einem bestimmten Material gefertigt und hat eine Farbe. Für eine bessere Zuordnung ist jeder Aurüstungsgegenstand mit einer Versorgungsnummer versehen.
\end{enumerate}
\section{Klassifizierung von Objekten}
Das im vorigen Abschnitt vorgestellte Beispiel stellt nur einen kleinen Ausschnitt aus der Realität dar. Komplexer gestaltete Datenbanken können aus weit mehr Objekten bestehen.
Um dabei nicht den Überblick über diese Flut von Objekten zu verlieren, werden diese in Klassen gruppiert. Diese Objektklassen enthalten dann die Objekte, die von ihrer Art her gleich sind und über die die gleichen Informationen gesammelt werden. Damit wird eine Abstraktionsebene gebildet, die es ermöglicht, von den Besonderheiten der einzelnen Objekte abzusehen und nur das typische der gebildeten Objektklassen zu berücksichtigen.
\subsection{Definitionen und Syntaxregeln}
Für die Darstellung der Objekttypen gelten folgende syntaktische
Regeln:
\begin {enumerate}
\item Ein Objekttyp wird durch ein Rechteck dargestellt, in dessen
Mitte der Objekttypname eingetragen wird.
\begin{center}
    \scalebox{1}{
        \begin{tikzpicture}[node distance=1.5cm, every edge/.style={link}]
            \node[entity](e1){Soldat};
        \end{tikzpicture}
    }
\end{center}
\item Die Größe und Position des Rechtecks sind bedeutungslos.
\item Der Objekttypname steht im Singular und muss für
das gesamte Datenmodell eindeutig sein.
\end{enumerate}
\begin{merke}
    \begin{itemize}
        \item \textbf{Objekt}: Ein Objekt (engl. Entity) ist ein Exemplar von Personen, Gegenständen oder nichtmateriellen Dingen, über das Informationen gespeichert wird, z. B. der konkrete Soldat Max Mustermann.
        \item \textbf{Objekttyp}: Ein Objekttyp (engl. Entity type) ist eine durch einen Objekttyp-namen eindeutig benannte Klasse von Objekten, über die dieselben Informationen gespeichert und die prinzipiell in gleicher Weise verarbeitet werden, wie z. B. die benannte Klasse bzw. der Objekttyp SOLDAT.
    \end{itemize}
\end{merke}

Die Bildung von Objekttypen hängt entscheidend von den Anforderungen
des jeweils zu modellierenden Gegenstandsbereiches ab. Aus der Sicht
eines Großhändlers kann ein Unternehmen mit all seinen Bereichen
als ein einziger Objekttyp gesehen werden. Dasselbe Unternehmen dagegen
wird aus seiner eigenen Sicht detailliert mit seinen Bereichen,
Abteilungen, Mitarbeitern, Werkshallen, Fahrzeugen usw. zu modellieren
sein. Diese Modellierung ist ebenfalls nicht eine einmalige, in sich
abgeschlossene Tätigkeit, denn im Laufe der Zeit müssen Änderungen
der Realität auch im Modell eingearbeitet werden.
\subsection{Traditionelles Pendant}
Die Informationsverarbeitung mit Hilfe elektronischer Datenverarbeitung
hat hinsichtlich der Datenspeicherung nur wenig prinzipiell neue
Methoden entwickelt. Fast alle Konzepte, in Bezug auf das
Entity-Relationship-Modell, haben ihr Pendant in der traditionellen
Informationsspeicherung. Zum besseren Verständnis der eingeführten
Begriffe wird auf diese Zusammenhänge an den entsprechenden Stellen
hingewiesen.

So entsprechen die Objekttypen den traditionellen Karteikästen und die
Objekte eines Objekttyps den Karteikarten, die in einem Karteikasten
eingeordnet sind. In der traditionellen Arbeitsweise würde für
\enquote{Axel Schweiss} eine Karteikarte angelegt werden und z. B. im
Karteikasten \enquote{Soldat} abgelegt werden
\section{Festlegung der relevanten Eigenschaften}
Im ersten Schritt, der Klassifizierung von Objekten, wurden Objekte, die
in gleicher Art und Weise verarbeitet werden, in Objekttypen
zusammengefasst. Um die Verarbeitung dieser Objekte automatisiert
durchführen zu können, ist es Voraussetzung, dass jeder Objekttyp
bestimmte Angaben speichert. Diese Angaben werden für die elektronische
Verarbeitung entweder als Eingabeinformation oder als Ausgabeinformation
benötigt.

Aus diesem Grund ist es notwendig, für jeden Objekttyp die relevanten
Eigenschaften der Objekte, die in ihm zusammengefasst werden, anzugeben.
Damit wird der durch den Objekttyp definierte Begriff auf einen Satz
relevanter Eigenschaften reduziert. Die Festlegung dieser Eigenschaften
ist aber nur bei genauer Kenntnis der Geschäfts- und
Verarbeitungsprozesse des Auftraggebers möglich. Ohne diese Kenntnisse
befindet man sich in jedem Falle im Bereich von Spekulationen.
\subsection{Definitionen und Syntaxregeln}
\begin{merke}
    \begin{itemize}
        \item \textbf{Eigenschaft:} Eine Eigenschaft (engl. Attribute) ist
              die Benennung für ein relevantes Merkmal aller Objekte, die in
              einem Objekttyp zusammengefaßt werden, z. B. Soldaten haben die
              Eigenschaft Dienstgrad.
        \item \textbf{Eigenschaftswert:} Ein Eigenschaftswert (engl.
              Attribute value) ist eine spezielle Ausprägung, die eine
              Eigenschaft für ein konkretes Objekt annimmt, z. B. der Dienstgrad
              Hauptfeldwebel.
    \end{itemize}
\end{merke}

Für die Darstellung der Eigenschaften gelten folgende Regeln:
\begin {enumerate}
\item Die Benennung der Eigenschaft wird als Blase an den Objekttyp
angehängt.

\begin{center}
    \scalebox{1}{
        \begin{tikzpicture}[node distance=1.5cm, every edge/.style={link}]
            \node[entity](e1){Soldat};
            \node[attribute](a1)[right = of e1]{Dienstgrad} edge (e1);
        \end{tikzpicture}
    }
\end{center}
\item Die Reihenfolge der Eigenschaften ist bedeutungslos.
\item Die Benennung der Eigenschaft steht im Singular und muss für
den Objekttyp eindeutig sein.
\end{enumerate}
Auch an dieser Stelle wird eine Abstraktion der realen Welt
vorgenommen, in dem statt der Eigenschaftswerte, die jedes einzelne
Objekt besitzt, nur die Eigenschaften der Objekttypen angegeben werden.
Um diesen Abstraktionsprozess korrekt durchführen zu können, bedarf
es der Einhaltung einiger Regeln:
\begin{itemize}
    \item Es darf niemals ein Eigenschaftswert als Eigenschaftsname
          verwendet werden (beispielsweise anstatt 02.05.1985 die Bezeichnung
          \enquote{Geburtsdatum} oder statt \enquote{männlich} die Bezeichnung
          \enquote{Geschlecht}).
    \item Die Bezeichnung eines Objekttyps sollte niemals im
          Eigenschaftsnamen auftauchen, da dieser immer nur im Kontext
          des Objekttyps gilt (z. B. Person mit der Eigenschaft Name, nicht
          \enquote{Personname}, sondern die Eigenschaft \enquote{Name}
          wählen).
    \item Bei komplexen Eigenschaften wie z. B. einer Adresse oder einer
          PK stellt sich immer wieder die Frage, ob diese zerlegt werden
          müssen oder ob sie als atomar \footnote{atomare Information = eine
              einzige, nicht mehr teilbare Information} betrachtet werden. Die
          Antwort auf diese Frage ist abhängig von den einzelnen
          Verarbeitungsprozessen, denen diese Daten unterliegen. Muss
          beispielsweise die Information verfügbar sein, von welchem
          Kreiswehrersatzamt ein Soldat betreut wird, so muss  die PK in ihre
          Bestandteile zerlegt werden. Ist diese Information irrelevant, kann
          die PK im ganzen als atomar betrachtet werden.
    \item Problematisch ist auch die Entscheidung, ob speicherwürdige
          Informationen als Eigenschaften oder als ein eigenständiger
          Objekttyp betrachtet werden sollen. Um einen eigenständigen
          Objekttyp handelt es sich immer dann, wenn er für das Unternehmen
          bedeutsame Objekte enthält, die relevante individuelle Eigenschaften
          besitzen.
\end{itemize}
\subsection{Traditionelle Datenspeicherung}
Die Eigenschaften eines Objekttyps A entsprechen den Feldern, die auf
einer Karteikarte zur Aufnahme der relevanten Informationen angelegt
werden. Durch die gewählten Eigenschaften wird also die einheitliche
Struktur aller Karteikarten eines Kastens festgelegt.

Welche Eigenschaften können Sie, bezogen auf das Beispiel
\enquote{Bundeswehr}, den identifizierten Objekttypen zuordnen?
\clearpage
\subsubsection{Lösungsvorschlag}
\begin{itemize}
    \item Objekttyp \enquote{Dienststelle}: Dienststellennummer,
          Bezeichung
    \item Objekttyp \enquote{Standort}: PLZ, Ort, Straße, Hausnummer
    \item Objekttyp \enquote{Dienstposten}: Dienstposten\_ID,
          Beginndatum, Enddatum, Dienstpostenbeschreibung
    \item Objekttyp \enquote{Soldat}: Personalnummer, PK, Name, Vorname,
          Dienstgrad, PLZ, Ort, Straße, Hausnummer
    \item Objekttyp \enquote{Ausrüstung}: Versorgungsnummer, Material,
          Farbe
\end{itemize}
\section{Festlegung der Identifizierung}
Ein Objekttyp stellt die Zusammenfassung mehrerer gleichartiger Objekte
dar. Gleichartig bedeutet, dass diese Objekte die gleichen Eigenschaften
aufweisen und die Verarbeitungsprozesse für all diese Objekte gleich
sind. Die einzelnen Objekte eines Objekttyps müssen aber voneinander
unterscheidbar sein. Deshalb muss festgelegt werden, auf welche Weise ein
Objekt innerhalb des Objekktyps identifiziert werden kann.

\subsection{Identifizierungsvarianten}
Für die Identifizierung eines Objekts innerhalb eines Objekttyps
stehen die konkreten Eigenschaftswerte des Objekts zur Verfügung. Es werden drei Varianten zur Identifizierung von Objekten
unterschieden.
\subsubsection{Identifizierung eines Objekts durch eine einzelne
    Eigenschaft}
Es kann vorkommen, dass die Eigenschaftswerte einer Eigenschaft eines
Objekttyps eindeutig ist. D. h. jeder Eigenschaftswert dieser
Eigenschaft ist so geartet, dass jedes Objekt dieses Objekttyps einen
unterschiedlichen Wert für diese Eigenschaft hat. Durch Angabe
dieses Eigenschaftswertes ist dann das Objekt eindeutig identifiziert.

Beispiel: Soldaten werden i. d. R. durch ihre PK eindeutig
identifiziert

\subsubsection{Identifizierung eines Objekts durch eine Kombination
    mehrerer Eigenschaften}
In einigen Fällen ist keine der Eigenschaften eines Objekttyps
geeignet, alleine als identifizierende Eigenschaft zu fungieren. Um
dieses Problem zu lösen kann versucht werden, eine
\underline{minimale} Kombination von Eigenschaften eines Objekttyps
als Identifikationsmerkmal zu benutzen. Dies funktioniert dann, wenn
die Kombination der Werte der entsprechenden Eigenschaften bei jedem
Objekt eindeutig sind.

Beispiel: Die Kombination der beiden Attribute PLZ und Ortsbezeichnung ist eindeutig,
eine Eigenschaft alleine nicht.

\subsubsection{Identifizierung eines Objekts durch eine organisatorische
    Eigenschaft}
Sollte es dennoch vorkommen, dass weder eine einzelne Eigenschaft, noch eine
Kombination von Eigenschaften zur Identifikation der Objekte eines
Objekttyps geeignet ist, kann eine künstliche Eigenschaft
eingeführt werden, bei der die Eindeutigkeit durch organisatorische
Maßnahmen gewährleistet wird.

Ein Beispiel hierfür wäre eine Ort\_ID, die es möglich
macht, einen Ort eindeutig zu bestimmen ohne die Kombination aus PLZ
und Ortsnamen heranziehen zu müssen.

Es ist aber auch möglich, dass eine organisatorische Eigenschaft
gewählt wird, da abzusehen ist, dass eine identifizierende
Eigenschaft durch eine andere ersetzt werden soll. Die
Umstrukturierung der Datenbank wäre zu aufwendig und zu komplex.

Ein Beispiel aus der Praxis ist die PK und die Personalnummer eines
jeden Soldaten. Die Personalnummer wird in geraumer Zeit die PK
ersetzen. Es ist aus diesem Grund von Vorteil eine organisatorische
Eigenschaft, wie die Personen\_ID zu wählen, um eine
Umstrukturierung zu vermeiden.

Weitere Beispiele für organisatorische Eigenschaften sind Attribute wie Artikelnummer und LfdNr.

Die Identifizierung der Objekte eines Objekttyps mittels einer
\enquote{organisatorischen Eigenschaft} ist bei der automatisierten
Datenverarbeitung eine beliebte Vorgehensweise. Sie wird häufig
selbst dann angewendet, wenn natürliche identifizierende
Eigenschaften vorhanden sind. Meist handelt es sich um eine laufende
Nummer.

Dies hat den Vorteil, dass kurze identifizierende Eigenschaftswerte
entstehen. Der Nachteil ist, dass Werte wie z. B. eine laufende
Nummer meist keinerlei Aussagekraft haben und somit Gefahren wie
Verwechslung oder Fehleingabe entstehen.

Die Festlegung der Identifizierungsform für einen Objekttyp muss mit
großer Sorgfalt erfolgen. Relationale Datenbank-Managementsysteme
für die wir unsere Modellierung durch\-füh\-ren, lassen es
nämlich nicht zu, dass zwei Objekte desselben Objekttyps in der
Kombination ihrer identifizierenden Merkmale übereinstimmen. Bleibt
ein Restrisiko hinsichtlich der Unikalität der identifizierenden
Merkmale und treten dann bei der praktischen Datenbankarbeit
tatsächlich zwei Objekte mit übereinstimmenden Werten ihrer
identifizierenden Merkmale auf, lehnt das Datenbank-Managementsystem
die Speicherung des zweiten Objekts ab.

\begin{merke}
    Um ein Modell übersichtlich und verständlich zu halten, sollte
    konsequent immer nur eine der drei zur Verfügung stehenden
    Methoden für die Identifizierung von Objekten genutzt werden!
\end{merke}
\subsection{Markierung der Identifizierenden Eigenschaft}
Für die Markierung der (teil)identifizierenden Eigenschaft gelten
die folgenden syntaktischen Regeln (diese werden u. U. in Kombination
mit anderen angewendet):
\begin {enumerate}
\item Eine identifizierende Eigenschaft (bzw. jede teilidentifizierende Eigenschaft) wird durch Unterstreichung kenntlich gemacht.

\begin{center}
    \scalebox{1}{
        \begin{tikzpicture}[node distance=1.5cm, every edge/.style={link}]
            \node[entity](e1){Soldat};
            \node[attribute](a1)[right = of e1]{\key{Personalnummer}} edge
            (e1);
        \end{tikzpicture}
    }
\end{center}
\item Die Position der (teil)identifizierenden Eigenschaft innerhalb der Liste der Eigenschaften ist bedeutungslos. Sie sollte jedoch aus Gründen der Übersichtlichkeit immer zu erst genannt werden.
\end{enumerate}
\subsection{Klassisches Pendant}
Bei der traditionellen Informationsspeicherung mit Hilfe von Karteikästen entspricht der Identifizierung die Festlegung eines Kennbegriffs: Üblicherweise wird im Kopf einer Karteikarte für das betreffende Objekt ein Wert angegeben, der das Objekt innerhalb des Karteikastens identifiziert\footnote{Kennbegriff wird oft auch als \enquote{Reiter} bezeichnet}. Um eine bestimmt Karteikarte manuell schneller finden zu können, werden die Karteikarten des Karteikastens nach diesem Begriff sortiert.

Betrachten wir wieder das Beispiel \enquote{Bundeswehr}. Welche Objekteigenschaften kön\-nen Ihrer Meinung nach als identifizierende Merkmale verwendet werden?
\subsubsection{Lösungsvorschlag}
Eine mögliche Antwort auf diese Frage könnte die folgende Aufzählung sein:
\begin{itemize}
    \item Objekttyp \enquote{Dienststelle}: Dienststellennummer
    \item Objekttyp \enquote{Standort}: PLZ, Ort
    \item Objekttyp \enquote{Dienstposten}: Dienstposten\_ID
    \item Objekttyp \enquote{Soldat}: Personalnummer
    \item Objekttyp \enquote{Ausrüstung}: Versorgungsnummer
\end{itemize}
An dieser Stelle wird die Entscheidung getroffen, dass für die Fortführung dieses Modells organisatorische Einheiten als identifizierende Eigenschaften eingeführt werden! Daraus ergeben sie die folgenden Änderungen:
\begin{itemize}
    \item Objekttyp \enquote{Dienststelle}: Dienststellen\_ID
    \item Objekttyp \enquote{Standort}: Ort\_ID
    \item Objekttyp \enquote{Dienstposten}: DP\_ID
    \item Objekttyp \enquote{Soldat}: Personen\_ID
    \item Objekttyp \enquote{Ausrüstung}: Ausruestungs\_ID
\end{itemize}

\section{Modellierungsformen}
\begin{center}
    \scalebox{.6}{
        \begin{tikzpicture}[node distance=1.5cm, every edge/.style={link}]
            \node[entity](dienstposten){Dienstposten};
            \node[attribute](dpid)[above = of dienstposten]{\key{DP\_ID}} edge (dienstposten);
            \node[attribute](dienstpostenid)[above left = of dienstposten]{Dienstposten\_ID} edge (dienstposten);
            \node[attribute](begindatum)[above right = of dienstposten]{Beginndatum} edge (dienstposten);
            \node[attribute](enddatum)[below right = of dienstposten]{Enddatum} edge (dienstposten);
            \node[attribute](dienstpostenbeschreibung)[below left = of dienstposten]{Dienstpostenbeschreibung} edge (dienstposten);
            \node[entity](soldat)[below left = 6cm of dienstposten]{Soldat};
            \node[attribute](personenid)[above = 1.6cm of soldat]{\key{Personen\_ID}} edge (soldat);
            \node[attribute](pk)[above right = of soldat]{PK} edge (soldat);
            \node[attribute](name)[below right = of soldat]{Name} edge (soldat);
            \node[attribute](vorname)[below left = of soldat]{Vorname} edge (soldat);
            \node[attribute](dienstgrad)[above left = of soldat]{Dienstgrad} edge (soldat);
            \node[entity](ausruestung)[below = 6cm of soldat]{Ausrüstung};
            \node[attribute](versorgungsnummer)[above left = of ausruestung]{Versorgungsnummer} edge (ausruestung);
            \node[attribute](bezeichnung)[above = 1.7cm of ausruestung]{Bezeichnung} edge (ausruestung);
            \node[attribute](ausruestungsid)[above right = of ausruestung]{\key{Ausruestungs\_ID}} edge (ausruestung);
            \node[attribute](material)[below right = of ausruestung]{Material} edge (ausruestung);
            \node[attribute](farbe)[below left = of ausruestung]{Farbe} edge (ausruestung);
            \node[entity](dienststelle)[below right = 6cm of dienstposten]{Dienststelle};
            \node[attribute](dienststellennummer)[below left = of dienststelle]{\key{Dienststellen\_ID}} edge (dienststelle);
            \node[attribute](dienststellennummer)[above left = of dienststelle]{Dienststellennummer} edge (dienststelle);
            \node[attribute](bezeichung)[above right = of dienststelle]{Bezeichnung} edge (dienststelle);
            \node[attribute](groesse)[below right = of dienststelle]{Größe} edge (dienststelle);
            \node[entity](standort)[below = 6cm of dienststelle]{Standort};
            \node[attribute](ortid)[above = of standort]{\key{Ort\_ID}} edge (standort);
            \node[attribute](plz)[above left = of standort]{PLZ} edge (standort);
            \node[attribute](ort)[above right = of standort]{Ort} edge (standort);
            \node[attribute](strasse)[below right = of standort]{Straße} edge (standort);
            \node[attribute](hausnummer)[below left = of standort]{Hausnummer} edge (standort);
        \end{tikzpicture}
    }
\end{center}
Hinweis: Das Objekt Soldat enthält nicht alle Attribute. Es fehlen die Attribute Personalnummer, Plz, Ort, Straße und Hausnummer.
