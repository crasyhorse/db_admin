\subsection{Übungsaufgabe Autohändler}
Entwerfen Sie, basierend auf der folgenden Lage, ein ER-Modell, inklusive der Beziehungen zwischen
den Entitäten.

Der Eigentümer eines namhaften Autohauses der Region beauftragt Sie eine Datenbank zu entwerfen.
Bei einem ersten Gespräch erfahren Sie, dass die Datenbank benötigt wird, um das Autohaus besser zu
verwalten. Des Weiteren wird die Grobgliederung der Datenbank festgehalten.
\begin{itemize}
    \item Die Datenbank soll alle Angestellten des Autohauses aufführen. Diese unterteilen sich in Verkäufer und Mechaniker. Verkäufer sind für die Betreuung der Kunden verantwortlich und führen  Autoverkäufe durch. Auß erdem nehmen die Verkäufer auch Aufträge für Reparaturen an. Mechaniker sind nur für die Durchführung der Reparaturen zuständig.
    \item In der Datenbank sollen Vor- und Nachname, Geburtsdatum und die Adresse des jeweiligen Mitarbeiters
          hinterlegt werden können. Jeder Verkäufer kann mehrere Kunden betreuen, einen Verkauf vornehmen oder eine Reparatur annehmen. Allerdings wird ein Kunde nur von genau einem Verkäufer betreut. Dies gilt auch im Bezug auf einen Autoverkauf und die Annahme einer Reparatur. Der Besitzer des Autohauses bittet Sie auch zu berücksichtigen, dass ein neu eingestellter Verkäufer noch nicht all diese Tätigkeiten durchführen kann.
\end{itemize}
Bei einem zweiten Treffen mit dem Autohausbesitzer werden die Details der Datenbank besprochen:
\begin{itemize}
    \item Kunden des Autohauses sollen mit Vor- und Nachnamen sowie ihrer Adresse im System erfasst werden. Es ist dabei zu bedenken, dass Personen, die ihr Auto nur an das Autohaus verkaufen, auch als Kunden erfasst werden, selbst wenn diese dann kein anderes Auto im Autohaus, bei Ihrem Auftraggeber, kaufen.
    \item Ihr Auftraggeber bittet Sie weiterhin, in der Datenbank seine gesamten Autos, inklusive der Autos
          seiner Kunden zu berücksichtigen. Autos können schon einem Kunden gehören oder werden an einen Kunden verkauft. Manche Kunden besitzen kein Auto z. B. Neukunden, andere haben zwei oder mehr Autos.
    \item Oft werden Autos zur Reparatur gebracht. Reparaturaufträge
          werden genau einem Auto zugeordnet. Manche Autos, z. B. Neuwagen haben
          noch keine Reparaturen, andere dagegen schon mehrere.
          \clearpage
    \item Die Fahrzeuge werden mit der Automarke, dem Modell, dem
          Marktpreis und der Fahrleistung erfasst. Es ist zu erwähnen, dass zu
          einem Kaufvertrag ein Datum und immer nur ein Auto gehören. Hier
          kann davon ausgegangen werden, dass jedes Auto höchstens einmal
          verkauft wird.
    \item Als Letztes soll festgehalten werden, dass ein Mechaniker eine oder mehrere Reparaturen durchführen kann. Zu jeder Reparatur müssen deren Datum, der Rechnungspreis für die geleisteten Arbeiten und die Ersatzteile, sowie die benötigten Arbeitsstunden gespeichert werden. Eine Reparatur wird auch nur von einem Mechaniker durchgeführt.
\end{itemize}
\clearpage