	\chapter{Transformation}
    \setcounter{page}{1}\kapitelnummer{chapter}
    \minitoc
\newpage
    Bei der Beschreibung der Realit\"at, mit Hilfe eines Entity-Relationship-Modells, bestand die Zielrichtung darin, Objekttypen und Beziehungstypen unabh\"angig vom sp\"ater einzusetzenden Datenbankmanagementsystem und somit auch unabh\"angig von einem speziellen Datenbankmodell zu beschreiben. Nun muss der Versuch unternommen werden, das durch die Modellierung entstandene Datenmodell m\"oglichst ohne semantische\footnote{Semantik = Bedeutungslehre, bezeichnen, anzeigen} Einbu\ss en mit den Strukturierungsmitteln der verf\"ugbaren Datenbankmanagementsysteme wiederzugeben. Dieser Vorgang wird als Transformation bezeichnet. Es ist die Umsetzung des konzeptionellen Datenmodells, welches nur die m\"oglichen Beziehungen zwischen den Objekttypen beschreibt, in das physische Datenmodell. Das physische Datenmodell ber\"ucksichtigt, wie die Beziehungen zwischen den Objekttypen auf Datenbankebene effektiv umgesetzt werden. Das physische Datenmodell wird in verschiedenen Quellen auch als relationales Modell
bezeichnet.

    Eine komplette Gleichsetzung ist je nach Interpretation nicht m\"oglich, hier soll jedoch nicht weiter unterschieden werden.

    Im Vorfeld sind diese beiden Fragen zu beantworten.
    \begin{enumerate}
      \item Welches Datenbankmodell soll der zu erstellenden Datenbank zugrunde liegen?
      \item Welche M\"oglichkeiten bietet das zu verwendende Datenbankmanagementsystem f\"ur die Gestaltung der Datenstrukturen?
    \end{enumerate}
    Hinsichtlich der ersten Frage hat heutzutage das relationale Datenbankmodell die meiste Relevanz. Was die zweite Fragestellung anbetrifft, so kommen hier nur die beiden Datenbankmanagementsysteme \textit{ORACLE} und \textit{Microsoft SQL Server} zum Einsatz. Bei beiden Systemen handelt es sich um relationale Datenbankmanagementsysteme (RDBMS), welche ihre Vor- und Nachteile besitzen, auf die im Unterricht kurz eingegangen werden soll.

    Dieses Kapitel schafft einheitliche Begriffe und stellt Regeln f\"ur die Transformation eines ER-Modells in ein relationales Modell auf.
\clearpage
    \section{Grundlagen}
      \subsection{Begriffsdefinitionen}
        \label{basics_definitions}
        Im vorangegangenen Kapitel wurde der Begriff \enquote{Schl\"ussel} erl\"autert, der an dieser Stelle aber noch weiter differenziert werden muss.
        \subsubsection{Identifikationsschl\"ussel (ID-Schl\"ussel)}
          Jedes Objekt einer Objektmenge muss eindeutig identifizierbar sein. Dies kann durch eine Eigenschaft/Attribut oder eine Kombination von Eigenschaften/Attributen gew\"ahrleistet werden. Beispielsweise ist ein Soldat der Bundeswehr eindeutig durch die Personalnummer identifizierbar. Der Name einer Person kann \textbf{kein} Identifikationsschl\"ussel sein, weil es sehr wahrscheinlich ist, dass es mehrere Personen mit dem gleichen Namen gibt (z. B. mehrere Meier).

          Der Identifikationsschl\"ussel muss folgende Kriterien erf\"ullen:
          \begin{itemize}
           \item Jedes Objekt muss eindeutig identifizierbar sein. Es d\"urfen nicht mehrere Objekte einen ID-Schl\"ussel mit dem gleichen Wert aufweisen.
           \item Jedem neuen Objekt muss augenblicklich ein Identifikationsschl\"ussel zugeteilt werden k\"onnen, da sonst keine Speicherung des Objektes erfolgen darf.
           \item Der ID-Schl\"usselwert eines Objektes darf sich w\"ahrend dessen Existenz nicht ver\"an\-dern.
          \end{itemize}
        \subsubsection{Prim\"arschl\"ussel}
          Der Prim\"arschl\"ussel wird h\"aufig mit dem Begriff \enquote{Identifikationsschl\"ussel} gleichgesetzt. Diese beiden Begriffe sind aber nicht geichbedeutend. Der Prim\"arschl\"ussel (PK - primary key) wird direkt in die Speicherorganisation einbezogen und ist somit dem physischen Datenmodell zugeordnet. Der ID-Schl\"ussel hingegen ist dem konzeptionellen Datenmodell zugeordnet. Ansonsten gelten f\"ur die Eigenschaftswerte eines PK dieselben Bedingungen, wie beim ID-Schl\"ussel beschrieben. Jeder transformierte Objekttyp kann nur einen PK haben. Da jedes Objekt eindeutig \"uber den PK identifiziert werden soll, ergibt sich automatisch die Bedingung, dass der Wert des PK einmalig (unikal) und nicht NULL (leer) sein muss. Daneben kann es aber auch weitere Eigenschaften mit eindeutigen Werten geben, die nicht zum PK geh\"oren.
\clearpage
        \subsubsection{Fremdschl\"ussel}
          Ein Fremdschl\"ussel (FK = Foreign Key) ist ein Attribut, welches zur Verkn\"upfung zweier Tabellen dient. Ein Fremdschl\"usselattribut wird in eine Tabelle eingef\"ugt, welche sich als untergeordnete Tabelle auf eine andere bezieht. Das Fremschl\"usselattribut bezieht sich dabei immer auf den Prim\"arschl\"ussel oder aber auf ein anderes, eindeutiges Attribut der \"ubergeordneten Tabelle. Genauso wie der Prim\"arschl\"ussel kann auch der Fremdschl\"ussel aus einer Kombination von Attributen bestehen. Im Gegensatz zum PK kann der FK NULL-Werte beinhalten.
        \subsubsection{NULL bzw. NOT NULL}
          Ein Eigenschaftswert kann in einer Datenbank verschiedene Einschr\"ankungen haben, um die Konsistenz der Daten zu gew\"ahrleisten. Eine dieser Einschr\"ankungen (engl. constraint) ist das NOT NULL constraint, welches im SQL Teil n\"aher erl\"autert wird. Es wird damit festgelegt, ob der Eigenschaftswert beim Anlegen eines Datensatzes vorhanden sein muss (NOT NULL, Abk. [NN]) oder ob er leer sein darf (NULL, Standard). Damit kann in der Datenbank die Forderung nach der Eigenschaft \enquote{eingabepflichtig} (Teil der Transformation) realisiert werden.
        \subsubsection{UNIQUE}
          Bei dem Begriff UNIQUE (Abk. [UN]) handelt es sich ebenfalls um ein constraint (Er\-l\"au\-terung\-en im SQL Teil). Mit diesem constraint kann die Forderung nach der \enquote{Unikalit\"at}, die sich durch die Transformation ergibt, erf\"ullt werden. Das UNIQUE (einmalig) constraint sorgt daf\"ur, dass die Eigenschaftswerte eines Objekttyps nicht doppelt vorkommen. Im RDBMS Oracle bildet hier der NULL-Wert eine Ausnahme, d.h. es k\"onnen mehrere NULL Werte innerhalb einer Spalte vorkommen. In Microsoft SQL Server ist dies nicht der Fall.
      \subsection{Kurzschreibweise der Tabellen}
        Den Aufbau einer Tabelle kann man mit folgender Kurzschreibweise darstellen:

        \centerline{Tabellenname(\pk{ID-Schl\"ussel}, Attribut$_1$, Attribut$_2$, Attribut$_3$, ..., Attribut$_n$)}

        Falls der PK aus zusammengesetzten Attributen besteht, werden alle erforderlichen Attribute unterstrichen:\\
        \hspace*{12mm}Tabellenname(\pk{Teil-ID-Schl\"ussel$_1$}, \pk{Teil-ID-Schl\"ussel$_2$}, Attribut$_1$, Attribut$_2$,\\
        \hspace*{40mm}Attribut$_3$, ..., Attribut$_n$)
    \section{Transformation von Objekttypen}
      Die Objekttypen des ER-Modells stellen das Haupt-Ordnungsprinzip der Datenmodellierung dar. Sie beschreiben die Klassen-Struktur, in die die speicherrelevanten Objekte der abzubildenden Realit\"at eingeordnet werden. Im physischen Datenbankmodell wird diese Klassen-Struktur durch einen Satz von Tabellen wiedergegeben.

      F\"ur jeden Objekttyp des ER-Modells wird eine Tabelle vereinbart. Dabei gilt die Transformationsregel T01.
      \subsubsection{Transformationsregel T01 (Objekttyp)}
      \tablefirsthead{%
        \hline
          \multicolumn{1}{|l}{\textbf{konzeptionelles Datenmodell}} &
          \multicolumn{1}{|l}{\textbf{ }} &
          \multicolumn{1}{|l|}{\textbf{physisches Datenmodell}} \\
        \hline
      }
      \begin{supertabular}[h]{|l|l|p{7.5cm}|}
        \hline
        \footnotesize Objekttypname & $\Rightarrow$ & \footnotesize Tabellen-Bezeichnung\\
        \hline
        \footnotesize (Teil-)Identifizierende Eigenschaft & $\Rightarrow$ & \footnotesize Prim\"arschl\"ussel-Attribut\\
        \hline
        \footnotesize Beschreibende Eigenschaft & $\Rightarrow$ & \footnotesize Spalten-Bezeichnung\\
        \hline
      \end{supertabular}

      \begin{center}
        \scalebox{.7}{
          \begin{tikzpicture}[node distance=1.5cm, every edge/.style={link}]
            \node[entity](obj1){Objekttypname};
              \node[attribute](teil1)[above left = of obj1]{\key{Teilidentifizierendes Attribut 1}} edge (obj1);
              \node[attribute](teil2)[above right = of obj1]{\key{Teilidentifizierendes Attribut 2}} edge (obj1);
              \node[attribute](beschr)[below = of obj1]{Beschreibendes Attribut} edge (obj1);
        \end{tikzpicture}
        }
      \end{center}
    \section{Transformation bin\"arer Beziehungstypen}
      In vorangegangenen Abschnitten wurde festgestellt, dass sich Beziehungstypen im physischen Datenbankmodell nur dadurch repr\"asentieren lassen, dass der Prim\"arschl\"ussel (PK) einer Tabelle \enquote{gedoppelt} und als Fremdschl\"ussel an \enquote{anderer Stelle} aufgenommen wird. Handelt es sich bei dieser \enquote{anderen Stelle} um eine andere Tabelle, wird ein bin\"arer Beziehungstyp dargestellt, der den sachlogischen Zusammenhang zwischen zwei Objekten aus verschiedenen Objekttypen beschreibt.

      Liegt diese \enquote{andere Stelle} dagegen in derselben Tabelle, aus der der PK stammt, wird ein rekursiver Beziehungstyp repr\"asentiert. Hier wird der sachlogische Zusammenhang zwischen zwei Objekten widergespiegelt, die demselben Objekttyp angeh\"oren. Nachfolgend werden die zehn m\"oglichen bin\"aren und die sieben rekursiven Beziehungstypen in (Min,Max)-Notation erl\"autert.
      \subsection{Der (1,1):(1,1) Beziehungstyp}
        Beim (1,1):(1,1) Beziehungstyp ist jedes Objekt des Objekttyps A mit genau einem Objekt des Objekttyps B verbunden und umgekehrt. Je ein A-Objekt und ein B-Objekt gehen eine feste Paarung ein, wie \abbildung{rel1111} zeigt. %oder auch in Abbildung \ref{fig:rel1111}.

        %\begin{figure}[!htb]
        \begin{center}
          \scalebox{1}{
            \begin{tikzpicture}
              \label{rel1111}
              \node[circleA](A) at (0,0){Objekttyp A};
                \node[redspot](a1) at (0, 1.3) {a1};
                \node[redspot](a2) at (0.5, 0.8){a2};
                \node[redspot](a3) at (0.1, 0.3){a3};
                \node[redspot](a4) at (-0.8, -0.4){a4};
                \node[redspot](a5) at (-0.2, -0.8){a5};
              \node[circleB](B) at (4.5, 0){Objekttyp B};
                \node[bluespot](b1) at (4.5, 1.3) {b1} edge (a1);
                \node[bluespot](b2) at (5, 0.8) {b2} edge (a2);
                \node[bluespot](b3) at (4.6, 0.3) {b3} edge (a3);
                \node[bluespot](b4) at (3.7, -0.4) {b4} edge (a4);
                \node[bluespot](b5) at (4.3, -0.8) {b5} edge (a5);
            \end{tikzpicture}
          }
        \end{center}
        (1,1):(1,1) Beziehungstypen sind bereits im ER-Modell kritisch zu betrachten, weil sie meist \"uber\-fl\"us\-sig sind. Ist n\"amlich jedes Objekt-A mit genau einem Objekt-B - und umgekehrt - verbunden, dann bildet sich eine derart feste Kopplung, dass sie als ein einziges, komplexes Objekt betrachtet werden k\"onnen. Die Umsetzung erfolgt, in dem alle Attribute von B in die Tabelle A eingef\"ugt werden. Die Kopplung wird dadurch erzwungen, dass der Schl\"ussel von B in der Tabelle A als eingabepflichtig (NOT NULL, NN) und als unikal (UNIQUE, UN) deklariert wird. Ein B-Objekt kann nun nicht losgel\"ost von \enquote{seinem} Objekt A gespeichert werden.

        \subsubsection{Beispiel (1,1):(1,1) Beziehungstyp}
          Betrachten wir zun\"achst einen \enquote{\"uberfl\"ussigen} (1,1):(1,1) Beziehungstyp: Mitarbeiter eines Unternehmens, von denen jeder genau einen Dienstausweis besitzt. Ein gegebener Dienstausweis ist nat\"urlich f\"ur genau einen Mitarbeiter ausgestellt.

          Da das Attribut \enquote{Ausweisnummer} eingabepflichtig ist, muss jeder gespeicherte Mitarbeiter einen Ausweis haben. Andererseits ist das Attribut unikal, so dass eine Ausweisnummer nur einem Mitarbeiter zugeordnet sein kann.

          Es gibt aber auch F\"alle, bei denen ein (1,1):(1,1) Beziehungstyp durchaus sinnvoll ist. Will man beispielsweise die Informationen \"uber Objekte in \"offentliche (z.B. Mitarbeiter-\\Offen(\pk{Personalnummer}, Name, TelNr, Abteilung)) und vertrauliche Daten (z.B. Mitar-\\beiterVS(\pk{Personalnummer}, Gehalt, Konfession)) unterteilen, kann man zwei Objekttypen A und B mit demselben Schl\"ussel \pk{S} ins Datenmodell aufnehmen. Dann wird sowohl der Schl\"ussel \pk{S} von A in B als unikaler eingabepflichtiger Fremdschl\"ussel vereinbart und umgekehrt, der Schl\"ussel \pk{S} von B wird in A als unikaler eingabepflichtiger FK deklariert. Die folgende Tabelle zeigt die entsprechende Transformationsregel T02.
\clearpage
        \subsubsection{Transformationsregel T02 f\"ur sinnvolle (1,1):(1,1) Beziehungen}
          \tablefirsthead{%
          \hline
            \multicolumn{1}{|c}{\textbf{konzeptionelles Datenmodell}} &
            \multicolumn{1}{|l}{\textbf{ }} &
            \multicolumn{1}{|c|}{\textbf{physisches Datenmodell}} \\
          \hline
          }
          \tabletail{%
          \hline
          }
          \begin{supertabular}[h]{|l|l|p{7.5cm}|}
            \footnotesize Objekttyp A mit Schl\"ussel \pk{S} & $\Rightarrow$ & \footnotesize Tabelle A mit PK \pk{S}\\
            \hline
            \footnotesize Objekttyp B mit Schl\"ussel \pk{S} & $\Rightarrow$ & \footnotesize Tabelle B mit PK \pk{S}\\
            \hline
            \footnotesize (1,1):(1,1) Beziehungstyp & $\Rightarrow$ & \footnotesize \pk{S} wird sowohl in A als auch in B als unikaler [UN], eingabepflichtiger [NN] Fremdschl\"ussel vereinbart (\fk{S})\\
            \hline
            \multicolumn{3}{|l|}{Alternative bei unterschiedlichen Schl\"usseln:} \\
            \hline
            \footnotesize Objekttyp A mit Schl\"ussel \pk{SA} & $\Rightarrow$ & \footnotesize Tabelle A mit PK \pk{SA}\\
            \hline
            \footnotesize Objekttyp B mit Schl\"ussel \pk{SB} & $\Rightarrow$ & \footnotesize Tabelle B mit PK \pk{SB}\\
            \hline
            \footnotesize (1,1):(1,1) Beziehungstyp & $\Rightarrow$ & \footnotesize Es wird \pk{SA} in B und \pk{SB} in A als unikaler [UN], eingabepflichtiger [NN] Fremdschl\"ussel eingef\"ugt (\fk{SA} und \fk{SB})\\
          \end{supertabular}

          Ein Fremdschl\"ussel wird in die Verweispfeile eingeschlossen - \fk{xyz}. Sollte der Fremd\-schl\"us\-sel seinerseits wieder Fremd\-schl\"us\-sel enthalten, werden f\"ur die inneren Fremd\-schl\"us\-sel die Verweispfeile weggelassen. Ist der Fremdschl\"ussel in der Tabelle f\"ur sich wiederum ein Prim\"arschl\"ussel, so wird der FK unterstrichen und fett gedruckt - \fk{\pk{xyz}}.

        \begin{center}
          \scalebox{.7}{
            \begin{tikzpicture}[node distance=1.5cm, every edge/.style={link}]
              \node[entity](A){A};
                \node[attribute](S)[left = of A]{\key{S}} edge (A);
              \node[relationship](rel1)[right = of A]{} edge node[auto,swap] {(1,1)}(A);
              \node[entity](B)[right = of rel1]{B} edge node[auto,swap] {(1,1)} (rel1);
                \node[attribute](S)[right = of B]{\key{S}} edge (B);
          \end{tikzpicture}
          }
        \end{center}

        \bild{}{transformationsregel_02_tabellen}{0.45}

          Bei diesem Beziehungstyp kann ein neues Objekt weder allein in A noch allein in B gespeichert werden. Das w\"urde der Forderung nach Nichtoptionalit\"at beider Beziehungstyprichtungen (siehe den jeweiligen Min-Wert) widersprechen. Deswegen muss vom Anwendungsprogramm im Rahmen einer Transaktion erreicht werden, dass zu einem neuen Schl\"usselwert je eine Zeile in die Tabellen A und B eingetragen wird.
          \begin{merke}
            Eine Transaktion ist eine Folge von Operationen, bei der sichergestellt wird, dass entweder alle Operationen fehlerfrei beendet werden oder das keine der Operationen ausgef\"uhrt wird.
          \end{merke}
        \subsubsection{Beispiel - Mitarbeiteradresse}
          Werden in einer Datenbank Adressen f\"ur mehrere Objekttypen gespeichert, z. B. Mitarbeiter und Kunden, kann es sinnvoll sein, einen eigenen Objekttyp \enquote{Adresse} zu erstellen.

					Da die Fremdschl\"ussel jeweils auf die andere Tabelle verweisen, muss es nach den Regeln der referentiellen Integrit\"at zu jedem FK genau einen Datensatz in der anderen Tabelle geben.

          \begin{center}
            \scalebox{.7}{
              \begin{tikzpicture}[node distance=1.2cm, every edge/.style={link}]
                \node[entity](A){Mitarbeiter};
                  \node[attribute](a1)[left = of A]{\key{Personalnummer}} edge (A);
                  \node[attribute](a2)[above left = of A]{Name} edge (A);
                  \node[attribute](a3)[below left = of A]{Telefon} edge (A);
                \node[relationship](rel1)[right = of A]{wohnt an} edge node[auto,swap] {(1,1)}(A);
                \node[entity](B)[right = of rel1]{Adresse} edge node[auto,swap] {(1,1)} (rel1);
                  \node[attribute](b1)[right = of B]{\key{Adresse\_ID}} edge (B);
                  \node[attribute](b2)[above right = of B]{Strasse} edge (B);
            \end{tikzpicture}
            }
          \end{center}
          \begin{small}
            Mitarbeiter(\pk{Personalnummer}, Name, Telefon, \un{\nn{\fk{Adresse\_ID}}})

            Adresse(\pk{Adresse\_ID}, Strasse, Hausnummer, PLZ, Ort, \un{\nn{\fk{Personalnummer}}})
          \end{small}
      \subsection{Der (1,1):(0,1) Beziehungstyp}
        Beim (1,1):(0,1) Beziehungstyp ist jedes Objekt des Objekttyps A mit genau einem Objekt des Objekttyps B verbunden. Ein Objekt aus B kann aber nur mit h\"ochstens einem Objekt aus A gekoppelt sein. Die Information(en) im A-Objekt k\"onnen als fakultative\footnote{wahlfreie, beliebige, optionale} erg\"anzende Angaben zum B-Objekt interpretiert werden. \abbildung{rel1101} verdeutlicht dies.

        \begin{center}
        \label{rel1101}
          \scalebox{1}{
            \begin{tikzpicture}
              \node[circleA](A) at (0,0){Objekttyp A};
                \node[redspot](a1) at (0, 1.3) {a1};
                \node[redspot](a2) at (0.5, 0.8){a2};
                \node[redspot](a3) at (0.1, 0.3){a3};
              \node[circleB](B) at (4.5, 0){Objekttyp B};
                \node[bluespot](b1) at (4.5, 1.3) {b1} edge (a1);
                \node[bluespot](b2) at (5, 0.8) {b2};
                \node[bluespot](b3) at (4.6, 0.3) {b3} edge (a3);
                \node[bluespot](b4) at (3.7, -0.4) {b4} edge (a2);
                \node[bluespot](b5) at (4.3, -0.8) {b5};
            \end{tikzpicture}
          }
        \end{center}

        Die Art der Transformation richtet sich nun danach, wie hoch der Anteil jener B-Objekte ist, f\"ur die erg\"anzende Angaben gemacht werden, die also in Beziehung zu einem A-Objekt stehen. Oder anders gefragt, ob der Objekttyp B wesentlich mehr Objekte enth\"alt als der Objekttyp A.
        \subsubsection{Fall 1:\#A unwesentlich kleiner als \#B}
          Gibt es bei einem (1,1):(0,1) Beziehungstyp nur unwesentlich mehr B-Objekte als A-Objekte, k\"onnen die Daten beider Objekttypen in einer Tabelle zusammengefasst werden. Die Eigenschaften von A werden dabei als nicht-eingabepflichtig deklariert.
\clearpage
          Der Schl\"ussel \pk{SA} von A wird in der Tabelle B jedoch als unikal deklariert. Ein A-Objekt, das damit nur zu einem einzigen B-Objekt geh\"oren kann, kann nicht losgel\"ost von \enquote{seinem} B-Objekt gespeichert werden. Bei den wenigen B-Objekten, die nicht mit einem A-Objekt verbunden sind, nehmen die A-Attribute den Wert NULL an.
        \subsubsection{Transformationsregel T03 f\"ur den (1,1):(0,1) Beziehungstyp (selten realisierte Optionalit\"at)}
          \tablefirsthead{%
          \hline
            \multicolumn{1}{|c}{\textbf{konzeptionelles Datenmodell}} &
            \multicolumn{1}{|l}{\textbf{ }} &
            \multicolumn{1}{|c|}{\textbf{physisches Datenmodell}} \\
          \hline
          }
          \begin{supertabular}[h]{|l|l|p{7.5cm}|}
            \footnotesize Objekttyp A mit Schl\"ussel \pk{SA} & $\Rightarrow$ & \footnotesize Alle Eigenschaften von A werden als nicht-eingabe-pflichtige Attribute in B aufgenommen. SA wird au\ss{}erdem als unikal [UN] deklariert\\
            \hline
            \footnotesize Objekttyp B mit Schl\"ussel \pk{SB} & $\Rightarrow$ & \footnotesize Tabelle B mit PK \pk{SB} \\
            \hline
            \footnotesize (1,1):(0,1) Beziehungstyp & $\Rightarrow$ & \footnotesize wird nicht gesondert dargestellt\\
          \end{supertabular}

          \begin{center}
            \scalebox{.7}{
              \begin{tikzpicture}[node distance=1.5cm, every edge/.style={link}]
                \node[entity](A){A};
                  \node[attribute](SA)[left = of A]{\key{SA}} edge (A);
                \node[relationship](rel1)[right = of A]{} edge node[auto,swap] {(1,1)}(A);
                \node[entity](B)[right = of rel1]{B} edge node[auto,swap] {(0,1)} (rel1);
                  \node[attribute](SB)[right = of B]{\key{SB}} edge (B);
            \end{tikzpicture}
            }
          \end{center}
        \subsubsection{Beispiel - Mitarbeiterhandy}
          Werden fast alle Mitarbeiter eines Unternehmens mit genau einem Handy ausgestattet, so sind die erg\"anzenden Angaben f\"ur das Handy bei nahezu allen Mitarbeitern erforderlich. Die entsprechende Transformation zeigt die folgende Abbildung.

          \begin{center}
            \scalebox{.7}{
              \begin{tikzpicture}[node distance=1.5cm, every edge/.style={link}]
                \node[entity](A){Handy};
                  \node[attribute](Inventarnummer)[left = of A]{\key{Inventarnummer}} edge (A);
                  \node[attribute](Mobilfunknummer)[above left = of A]{Mobilfunknummer} edge (A);
                \node[relationship](rel1)[right = of A]{hat} edge node[auto,swap] {(1,1)}(A);
                \node[entity](B)[right = of rel1]{Mitarbeiter} edge node[auto,swap] {(0,1)} (rel1);
                  \node[attribute](Personalnummer)[right = of B]{\key{Personalnummer}} edge (B);
                  \node[attribute](Name)[above right = of B]{Name} edge (B);
            \end{tikzpicture}
            }
          \end{center}
          \begin{small}
            Mitarbeiter(\pk{Personalnummer}, Name, \un{Handy-Inventarnummer}, Mobilfunknummer)
          \end{small}

          Die urspr\"ungliche Eigenschaftsbezeichnung \enquote{Inventarnummer}
          wurde durch den Zusatz\\ \enquote{Handy} in ihrem neuen Kontext
          \enquote{sprechender} gew\"ahlt. Die Handyspalten der Tabelle
          Mitarbeiter sind nicht-eingabepflichtig, d.h. sie nehmen f\"ur jene
          Mitarbeiter, die nicht mit einem Handy ausgestattet sind, den
          NULL-Wert an. Dadurch, dass das Attribut
          \enquote{Handy-Inventarnummer} als unikal deklariert ist, kann ein und
          dasselbe Handy nicht mehreren Mitarbeitern zugeordnet werden.
\clearpage
        \subsubsection{Fall 2: \#A ist wesentlich kleiner als \#B}
          Betrachten wir nun den Fall, dass es wesentlich mehr B-Objekte als A-Objekte gibt. W\"urde man jetzt die beiden Objekttypen in einer Tabelle vereinen, h\"atten die Attribute der A-Objekte in vielen Zeilen den NULL-Wert. Um dies zu vermeiden, werden zwei Tabellen angelegt: eine B-Tabelle f\"ur alle B-Objekte und eine A-Tabelle f\"ur die seltenen A-Objekte, die durch einen eingabepflichtigen Fremdschl\"ussel jeweils auf \enquote{ihr} B-Objekt verweisen. Die Kardinalit\"at \enquote{1} von (0,1) auf der B-Seite wird dadurch erzwungen, dass der Fremdschl\"ussel in A als unikal deklariert wird. Diese Umsetzung entspricht der Transformationsregel T04.
        \subsubsection{Transformationsregel T04 f\"ur den (1,1):(0,1) Beziehungstyp (oft realisierte Optionalit\"at)}
          \tablefirsthead{%
          \hline
            \multicolumn{1}{|c}{\textbf{konzeptionelles Datenmodell}} &
            \multicolumn{1}{|l}{\textbf{ }} &
            \multicolumn{1}{|c|}{\textbf{physisches Datenmodell}} \\
          \hline
          }
          \begin{supertabular}[h]{|l|l|p{7,25cm}|}
            \footnotesize Objekttyp A mit Schl\"ussel \pk{SA} & $\Rightarrow $ & \footnotesize Tabelle A mit PK \pk{SA}\\
            \hline
            \footnotesize Objekttyp B mit Schl\"ussel \pk{SB} & $\Rightarrow $ & \footnotesize Tabelle B mit PK \pk{SB}\\
            \hline
            \footnotesize (1,1):(0,1) Beziehungstyp & $\Rightarrow $ & \footnotesize \pk{SB} wird in A als ein\-ga\-be\-pflich\-tiger uni\-kaler Fremd\-schl\"ussel auf\-ge\-nommen (\fk{SB}[NN, UN])\\
          \end{supertabular}

          \begin{center}
            \scalebox{.7}{
              \begin{tikzpicture}[node distance=1.5cm, every edge/.style={link}]
                \node[entity](A){A};
                  \node[attribute](SA)[left = of A]{\key{SA}} edge (A);
                \node[relationship](rel1)[right = of A]{} edge node[auto,swap] {(1,1)}(A);
                \node[entity](B)[right = of rel1]{B} edge node[auto,swap] {(0,1)} (rel1);
                  \node[attribute](SB)[right = of B]{\key{SB}} edge (B);
            \end{tikzpicture}
            }
          \end{center}
        \subsubsection{Beispiel - Projektleiter}
          Soll beispielsweise festgehalten werden, welcher Mitarbeiter ein - und h\"ochstens ein - Projekt leitet, wobei f\"ur jedes Projekt genau ein Mitarbeiter verantwortlich ist, so wird es viele Mitarbeiter ohne Projektverantwortung geben. Daher ist in diesem Fall die Transformationsregel T04 anzuwenden.

          \begin{center}
            \scalebox{.7}{
              \begin{tikzpicture}[node distance=1.5cm, every edge/.style={link}]
                \node[entity](A){Projekt};
                  \node[attribute](a1)[left = of A]{\key{Projeknummer}} edge (A);
                  \node[attribute](a2)[above left = of A]{Bezeichnung} edge (A);
                \node[relationship](rel1)[right = of A]{leitet} edge node[auto,swap] {(1,1)}(A);
                \node[entity](B)[right = of rel1]{Mitarbeiter} edge node[auto,swap] {(0,1)} (rel1);
                  \node[attribute](Personalnummer)[right = of B]{\key{Personalnummer}} edge (B);
                  \node[attribute](Name)[above right = of B]{Name} edge (B);
            \end{tikzpicture}
            }
          \end{center}
          \begin{small}
            (A:) Projekt(\pk{Projeknummer}, Bezeichnung, \fk{Personalnummer} [NN, UN])

            (B:) Mitarbeiter(\pk{Personalnummer}, Name)
          \end{small}

          Da der Fremdschl\"ussel \fk{ Personalnummer } eingabepflichtig ist, muss jedes Projekt genau einen Projektleiter haben. Andererseits ist der Fremdschl\"ussel unikal, so dass ein Mitarbeiter nur f\"ur ein Projekt als Leiter ausgewiesen sein kann. Da es nicht m\"oglich ist sicherzustellen, dass jede Personalnummer eines Mitarbeiters auch tats\"achlich als Wert in der Fremdschl\"usselspalte auftritt, kann es Mitarbeiter geben, die mit keinem Projekt als Leiter verbunden sind.
      \subsection{Der (0,1):(0,1) Beziehungstyp}
        Beim (0,1):(0,1) Beziehungstyp ist jedes Objekt des Typs A mit keinem oder einem Objekt des Typs B verbunden und umgekehrt. Es gibt also A-Objekte ohne B-Partner und B-Objekte ohne A-Partner. Jedes der Objekte kann aber h\"ochstens einen Partner haben. Schematisch ist dies in der folgenden Abbildung dargestellt.

        \begin{center}
          \scalebox{1}{
            \begin{tikzpicture}
              \node[circleA](A) at (0,0){Objekttyp A};
                \node[redspot](a1) at (0, 1.3) {a1};
                \node[redspot](a2) at (0.5, 0.8){a2};
                \node[redspot](a3) at (0.1, 0.3){a3};
                \node[redspot](a4) at (-0.8, -0.4){a4};
                \node[redspot](a5) at (-0.2, -0.8){a5};
              \node[circleB](B) at (4.5, 0){Objekttyp B};
                \node[bluespot](b1) at (4.5, 1.3) {b1} edge (a1);
                \node[bluespot](b2) at (5, 0.8) {b2} edge (a3);
                \node[bluespot](b3) at (4.6, 0.3) {b3};
                \node[bluespot](b4) at (3.7, -0.4) {b4};
                \node[bluespot](b5) at (4.3, -0.8) {a4};
            \end{tikzpicture}
          }
        \end{center}
        Wir nehmen f\"ur die folgenden Betrachtungen - ohne Beschr\"ankung der Allgemeinheit - an, dass es h\"ochstens so viele B-Objekte gibt wie A-Objekte (\#A $\geq$ \#B, andernfalls m\"ussen die Objekttpypen einfach die Seiten tauschen). Da sowohl die A-Objekte als auch die B-Objekte unabh\"angig voneinander existieren k\"onnen, m\"ussen sie jeweils in einer eigenen Tabelle gespeichert werden.

        Bei den B-Objekten wird ein Verweis auf \enquote{ihren} A-Partner hinterlegt. Bei den \enquote{partnerlosen} B-Objekten hat dieser Verweis dann den NULL-Wert. Deshalb wird der Pri\-m\"ar\-schl\"us\-sel von A in der Tabelle B als nicht-eingabepflichtiger Fremdschl\"ussel aufgenommen. Fordert man f\"ur den Fremdschl\"ussel ausserdem die Unikalit\"at, kann auf ein A-Objekt nur von h\"ochstens einem B-Objekt aus verwiesen werden. Die Regel T05 zeigt diesen Sachverhalt.
        \subsubsection{Transformationsregel T05 f\"ur den (0,1):(0,1) Beziehungstyp}
          \tablefirsthead{%
          \hline
            \multicolumn{1}{|c}{\textbf{konzeptionelles Datenmodell}} &
            \multicolumn{1}{|l}{\textbf{ }} &
            \multicolumn{1}{|c|}{\textbf{physisches Datenmodell}} \\
          \hline
          }
          \begin{supertabular}[h]{|l|l|p{7,25cm}|}
            \footnotesize Objekttyp A mit Schl\"ussel \pk{SA} & $\Rightarrow $ & \footnotesize Tabelle A mit PK \pk{SA}\\
            \hline
            \footnotesize Objekttyp B mit Schl\"ussel \pk{SB} & $\Rightarrow $ & \footnotesize Tabelle B mit PK \pk{SB}\\
            \hline
            \footnotesize (0,1):(0,1) Beziehungstyp & $\Rightarrow $ & \footnotesize \pk{SA} wird in B als nicht-ein\-gabe\-pflichtiger uni\-kaler Fremd\-schl\"ussel auf\-ge\-nommen (\un{\fk{SA}})\\
          \end{supertabular}
          \begin{center}
            \scalebox{.7}{
              \begin{tikzpicture}[node distance=1.5cm, every edge/.style={link}]
                \node[entity](A){A};
                  \node[attribute](SA)[left = of A]{\key{SA}} edge (A);
                \node[relationship](rel1)[right = of A]{} edge node[auto,swap] {(0,1)}(A);
                \node[entity](B)[right = of rel1]{B} edge node[auto,swap] {(0,1)} (rel1);
                  \node[attribute](SB)[right = of B]{\key{SB}} edge (B);
            \end{tikzpicture}
            }
          \end{center}
          \bild{}{transformationsregel_05_tabellen}{0.35}

        \subsubsection{Beispiel - Ehe ohne Scheidung}
          In einem hier nicht n\"aher genannten Land wird die monogame Ehe praktiziert, jedoch mit dem Unterschied zu Deutschland, dass eine Scheidung rechtlich nicht vorgesehen ist. Die Eheschlie\ss{}ung zweier Personen lie\ss{}e sich somit als bin\"arer (0,1):(0,1)-Beziehungstyp darstellen, da jede Frau h\"ochstens einen Mann und jeder Mann h\"ochstens eine Frau heiraten kann.
          \begin{center}
            \scalebox{.7}{
              \begin{tikzpicture}[node distance=1.5cm, every edge/.style={link}]
                \node[entity](A){Frau};
                  \node[attribute](a1)[left = of A]{\key{Personen\_ID}} edge (A);
                  \node[attribute](a2)[above left = of A]{Name} edge (A);
                \node[relationship](rel1)[right = of A]{heiratet} edge node[auto,swap] {(0,1)}(A);
                \node[entity](B)[right = of rel1]{Mann} edge node[auto,swap] {(0,1)} (rel1);
                  \node[attribute](b1)[right = of B]{\key{Personen\_ID}} edge (B);
                  \node[attribute](b2)[above right = of B]{Name} edge (B);
            \end{tikzpicture}
            }
          \end{center}
          \begin{small}
            (A:) Frau(\pk{Personen\_ID}, Name)

            (B:) Mann(\pk{Personen\_ID}, Name, \un{\fk{ Ehegatten\_ID}})
          \end{small}

         Nicht verheiratete M\"anner haben, statt des Verweises auf die \fk{Personen\_ID} des Ehegatten, einen NULL-Wert. Wegen der Unikalit\"at des Fremdschl\"ussels kann eine Frau h\"ochs\-tens einen Mann heiraten.
      \subsection{Der (1,1):(0,*) Beziehungstyp}
        Beim (1,1):(0,*) Beziehungstyp ist jedes Objekt B mit keinem, einem oder
        mehreren Objekten des Objekttyps A verbunden. Ein A-Objekt ist dagegen
        immer mit genau einen B-Objekt gekoppelt. Die folgende Abbildung zeigt
        dies schematisch.

        \begin{center}
          \scalebox{0.97}{
            \begin{tikzpicture}
              \node[circleA](A) at (0,0){Objekttyp A};
                \node[redspot](a1) at (0, 1.3) {a1};
                \node[redspot](a2) at (0.5, 0.8){a2};
                \node[redspot](a3) at (0.1, 0.3){a3};
                \node[redspot](a4) at (-0.8, -0.4){a4};
                \node[redspot](a5) at (-0.2, -0.8){a5};
              \node[circleB](B) at (4.5, 0){Objekttyp B};
                \node[bluespot](b1) at (4.5, 1.3) {b1};
                \node[bluespot](b2) at (5, 0.8) {b2};
                \node[bluespot](b3) at (4.6, 0.3) {b3};
                \node[bluespot](b4) at (3.7, -0.4) {b4};
                \node[bluespot](b5) at (4.3, -0.8) {b5};
              \path (a1) edge (b1)
                (a2) edge (b2)
                (a3) edge (b3)
                (a4) edge (b3)
                (a5) edge (b4);
            \end{tikzpicture}
          }
        \end{center}
        Der Beziehungstyp wird im physischen Datenbankmodell durch je eine
        Tabelle f\"ur die A-Objekte und die B-Objekte repr\"asentiert, wobei der
        PK \pk{SB} von B als eingabepflichtiger Fremdschl\"ussel in A
        aufgenommen wird. Jedes A-Objekt muss dann auf genau ein B-Objekt
        verweisen. Da der Fremdschl\"ussel aber nicht als unikal vereinbart ist,
        k\"onnen mehrere A-Objekte mit demselben B-Objekt gekoppelt sein.
\clearpage
        \subsubsection{Transformationsregel T06 f\"ur den (1,1):(0,*) Beziehungstyp}
          \tablefirsthead{%
          \hline
            \multicolumn{1}{|c}{\textbf{konzeptionelles Datenmodell}} &
            \multicolumn{1}{|l}{\textbf{ }} &
            \multicolumn{1}{|c|}{\textbf{physisches Datenmodell}} \\
          \hline
          }
          \begin{supertabular}[h]{|l|l|p{7.5cm}|}
            \footnotesize Objekttyp A mit Schl\"ussel \pk{SA} & $\Rightarrow $ & \footnotesize Tabelle A mit PK \pk{SA}\\
            \hline
            \footnotesize Objekttyp B mit Schl\"ussel \pk{SB} & $\Rightarrow $ & \footnotesize Tabelle B mit PK \pk{SB}\\
            \hline
            \footnotesize (1,1):(0,*) Beziehungstyp & $\Rightarrow $ & \footnotesize \pk{SB} wird in A als ein\-ga\-be\-pflich\-tiger nicht-unikaler Fremd\-schl\"ussel auf\-ge\-nom\-men (\nn{\fk{SB}})\\
          \end{supertabular}
          \begin{center}
            \scalebox{.7}{
              \begin{tikzpicture}[node distance=1.5cm, every edge/.style={link}]
                \node[entity](A){A};
                  \node[attribute](a1)[left = of A]{\key{SA}} edge (A);
                \node[relationship](rel1)[right = of A]{} edge node[auto,swap] {(1,1)}(A);
                \node[entity](B)[right = of rel1]{B} edge node[auto,swap] {(0,*)} (rel1);
                  \node[attribute](b1)[right = of B]{\key{SB}} edge (B);
            \end{tikzpicture}
            }
          \end{center}
        \subsubsection{Beispiel - Gehaltsgruppen}
          Beispielsweise kann eine Gehaltsgruppe noch keinem, erst einem Mitarbeiter oder bereits mehreren Mitarbeitern zugeordnet sein. Andererseits wird jeder Mitarbeiter in genau eine Gehaltsgruppe eingeordnet.
          \begin{center}
            \scalebox{.7}{
              \begin{tikzpicture}[node distance=1.5cm, every edge/.style={link}]
                \node[entity](A){Mitarbeiter};
                  \node[attribute](a1)[left = of A]{\key{Personalnummer}} edge (A);
                  \node[attribute](a2)[above left = of A]{Name} edge (A);
                \node[relationship](rel1)[right = of A]{hat} edge node[auto,swap] {(1,1)}(A);
                \node[entity](B)[right = of rel1]{Gehaltsgruppe} edge node[auto,swap] {(0,*)} (rel1);
                  \node[attribute](b1)[right = of B]{\key{Gruppennummer}} edge (B);
                  \node[attribute](b2)[above right = of B]{Gehalt} edge (B);
            \end{tikzpicture}
            }
          \end{center}
          \begin{small}
            (B:) Gehaltsgruppe(\pk{Gruppennummer}, Gehalt)

            (A:) Mitarbeiter(\pk{Personalnummer}, Name, \nn{\fk{Gruppennummer}})
          \end{small}

          Da der Fremdschl\"ussel \fk{Gruppennummer} eingabepflichtig ist, muss jedem Mitarbeiter genau eine Gehaltsgruppe zugeordnet werden. Andererseits ist der Fremdschl\"ussel nichtunikal, so dass mehrere Mitarbeiter auf dieselbe Gehaltsgruppe verweisen k\"onnen. Da nicht zu fordern ist, dass jeder Wert des Prim\"arschl\"ussels \pk{Gruppennummer} auch tats\"achlich als Wert des Fremdschl\"ussels \fk{Gruppennummer} auftreten muss, kann es Gehaltsgruppen geben, auf die noch nicht verwiesen wird.
      \subsection{Der (0,1):(0,*) Beziehungstyp}
        Beim (0,1):(0,*) Beziehungstyp kann ein Objekt des Typs B mit keinem, einem oder mehreren A-Objekten gekoppelt sein. Ein A-Objekt kann aber zu h\"ochstens einem B-Objekt in Beziehung stehen.

        \begin{center}
          \scalebox{1}{
            \begin{tikzpicture}
              \node[circleA](A) at (0,0){Objekttyp A};
                \node[redspot](a1) at (0, 1.3) {a1};
                \node[redspot](a2) at (0.5, 0.8){a2};
                \node[redspot](a3) at (0.1, 0.3){a3};
                \node[redspot](a4) at (-0.8, -0.4){a4};
                \node[redspot](a5) at (-0.2, -0.8){a5};
              \node[circleB](B) at (4.5, 0){Objekttyp B};
                \node[bluespot](b1) at (4.5, 1.3) {b1};
                \node[bluespot](b2) at (5, 0.8) {b2};
                \node[bluespot](b3) at (4.6, 0.3) {b3};
                \node[bluespot](b4) at (3.7, -0.4) {b4};
                \node[bluespot](b5) at (4.3, -0.8) {b5};
              \path (a1) edge (b1)
                (a2) edge (b2)
                (a3) edge (b3)
                (a4) edge (b3);
            \end{tikzpicture}
          }
        \end{center}
        Die Transformation dieses Beziehungstyps erfolgt, indem der Prim\"arschl\"ussel von B als nicht-ein\-ga\-be\-pflich\-ti\-ger und nicht-unikaler Fremdschl\"ussel in A aufgenommen wird.
        \subsubsection{Transformationsregel T07 f\"ur den (0,1):(0,*) Beziehungstyp}
          \tablefirsthead{%
          \hline
            \multicolumn{1}{|c}{\textbf{konzeptionelles Datenmodell}} &
            \multicolumn{1}{|l}{\textbf{ }} &
            \multicolumn{1}{|c|}{\textbf{physisches Datenmodell}} \\
          \hline
          }
          \begin{supertabular}[h]{|l|l|p{7.5cm}|}
            \footnotesize Objekttyp A mit Schl\"ussel \pk{SA} & $\Rightarrow $ & \footnotesize Tabelle A mit PK \pk{SA}\\
            \hline
            \footnotesize Objekttyp B mit Schl\"ussel \pk{SB} & $\Rightarrow $ & \footnotesize Tabelle B mit PK \pk{SB}\\
            \hline
            \footnotesize (0,1):(0,*) Beziehungstyp & $\Rightarrow $ & \footnotesize \pk{SB} wird in A als nicht\-ein\-ga\-be\-pflich\-tiger nicht-uni\-kaler Fremd\-schl\"ussel auf\-ge\-nommen (\fk{SB})\\
          \end{supertabular}
          \begin{center}
            \scalebox{.7}{
              \begin{tikzpicture}[node distance=1.5cm, every edge/.style={link}]
                \node[entity](A){A};
                  \node[attribute](a1)[left = of A]{\key{SA}} edge (A);
                \node[relationship](rel1)[right = of A]{} edge node[auto,swap] {(0,1)}(A);
                \node[entity](B)[right = of rel1]{B} edge node[auto,swap] {(0,*)} (rel1);
                  \node[attribute](b1)[right = of B]{\key{SB}} edge (B);
            \end{tikzpicture}
            }
          \end{center}
        \subsubsection{Beispiel - Abteilungsmitarbeiter}
          Betrachten wir eine Abteilung, die noch keinen, schon einen oder bereits mehrere Mitarbeiter hat. Die meisten Mitarbeiter sind in eine Abteilung eingeordnet, einige wenige - mit zentralen Aufgaben - geh\"oren jedoch zu keiner Abteilung.
          \begin{center}
            \scalebox{.7}{
              \begin{tikzpicture}[node distance=1.5cm, every edge/.style={link}]
                \node[entity](A){Mitarbeiter};
                  \node[attribute](a1)[left = of A]{\key{Personalnummer}} edge (A);
                  \node[attribute](a2)[above left = of A]{Name} edge (A);
                \node[relationship](rel1)[right = of A]{} edge node[auto,swap] {(0,1)}(A);
                \node[entity](B)[right = of rel1]{Abteilung} edge node[auto,swap] {(0,*)} (rel1);
                  \node[attribute](b1)[right = of B]{\key{Abteilungsnummer}} edge (B);
                  \node[attribute](b2)[above right = of B]{Jahresbudget} edge (B);
                  \node[attribute](b3)[below right = of B]{Bezeichnung} edge (B);
            \end{tikzpicture}
            }
          \end{center}
          \begin{small}
            (B:) Abteilung(\pk{Abteilungsnummer}, Jahresbudget, Bezeichnung)

            (A:) Mitarbeiter(\pk{Personalnummer}, Name, \fk{Abteilungsnummer})
          \end{small}

          Bei den mit zentralen Aufgaben betrauten Mitarbeitern wird in der Fremdschl\"usselspalte \fk{Abteilungsnummer} ein NULL-Wert stehen. Da der Fremdschl\"ussel nicht-unikal ist, k\"onnen mehrere Mitarbeiter mit derselben Abteilung in Verbindung gebracht werden.

          Da es ohnehin nicht m\"oglich ist zufordern, dass jeder Wert eines Prim\"arschl\"ussels auch in der Spalte des Fremdschl\"ussels \fk{Abteilungsnummer} auftritt, kann es Abteilungen ohne Mitarbeiter geben.
      \subsection{Der (1,1):(1,*) Beziehungstyp}
        Der (1,1):(1,*) Beziehungstyp unterscheidet sich vom (1,1):(0,*) Beziehungstyp nur dadurch, dass jedes Objekt des Objekttyps B mit mindestens einem Objekt des Objekttyps A verbunden sein muss.
        \begin{center}
          \scalebox{1}{
            \begin{tikzpicture}
              \node[circleA](A) at (0,0){Objekttyp A};
                \node[redspot](a1) at (0, 1.3) {a1};
                \node[redspot](a2) at (0.5, 0.8){a2};
                \node[redspot](a3) at (0.1, 0.3){a3};
                \node[redspot](a4) at (-0.8, -0.4){a4};
                \node[redspot](a5) at (-0.2, -0.8){a5};
              \node[circleB](B) at (4.5, 0){Objekttyp B};
                \node[bluespot](b1) at (4.5, 1.3) {b1};
                \node[bluespot](b2) at (5, 0.8) {b2};
                \node[bluespot](b3) at (4.6, 0.3) {b3};
                \node[bluespot](b4) at (3.7, -0.4) {b4};
              \path (a1) edge (b1)
                (a2) edge (b2)
                (a3) edge (b3)
                (a4) edge (b3)
                (a5) edge (b4);
            \end{tikzpicture}
          }
        \end{center}
        In der Literatur zum physischen Datenbankmodell (auch relationales Modell) wird der Beziehungstyp (1,1):(1,*) gew\"ohnlich als \enquote{meistverbreitetste} Art von Beziehungstypen bezeichnet. Das ist aber falsch, denn dieser Beziehungstyp l\"asst sich im physikalischen Datenbankmodell gar nicht re\-pr\"as\-en\-tie\-ren. Der Grund daf\"ur liegt darin, dass es auf der Ebene der Tabellen-Typ\-be\-schrei\-bung\-en keine M\"oglichkeit gibt, die Nichtoptionalit\"at der Beziehungstyprichtung von B zu A zu repr\"asentieren.

        Betrachten wir die Situation im Einzelnen. Die letzten L\"osungen zur Transformation eines Beziehungstyps bestanden darin, den PK der B-Tabelle in die A-Tabelle als FK aufzunehmen. Soll nun jedes B-Objekt mit mindestens einem A-Objekt gekoppelt sein, so ist das gleichbedeutend mit der Forderung, das jeder Wert, den der PK B annimmt, wenigstens einmal als Wert des FK in A auftauchen muss. Diese Forderung hat nichts mit der referentiellen Integrit\"at zu tun. Diese fordert nur, dass jeder Wert des FK in A entweder der NULL-Wert sein oder aber als Wert des PK in B vorhanden sein muss. Die Umkehrrichtung, dass jeder Wert des PK auch als Wert des FK auftauchen muss, l\"asst sich im physischen Datenbankmodell nicht formulieren.

        Die Transformation des (1,1):(1,*) Beziehungstyps in das physische Datenbankmodell erfolgt deshalb auf die gleiche Weise, wie die des (1,1):(0,*) Beziehungstyps, gem\"a\ss\ T06. Dabei muss allerdings in Kauf genommen werden, dass wichtige semantische Informationen, die im konzeptionellen Datenmodell repr\"asentiert werden, verloren gehen. Diese Informationen k\"onnen nur im Rahmen der Anwendungsprogrammierung ber\"ucksichtigt werden.
        \subsubsection{Beispiel Kundenbestellung}
          Betrachten wir ein Unternehmen, in dem eine Gesch\"aftsregel besagt, dass ein Kunde erst dann gespeichert wird, wenn er die erste Bestellung vornimmt. Im Laufe der Zeit k\"onnen einem Kunden nat\"urlich mehrere Bestellungen zugeordnet werden. Jede Bestellung kommt von genau einem Kunden. Ein Kunde, f\"ur den keine Bestellung mehr besteht (weil er alle seine Bestellungen storniert hat), wird wieder gel\"oscht. Das Datenmodell muss nach der Transformationsregel T06, entsprechend dem folgenden Bild, umgesetzt werden.
          \begin{center}
            \scalebox{.7}{
              \begin{tikzpicture}[node distance=1.5cm, every edge/.style={link}]
                \node[entity](A){Bestellung};
                  \node[attribute](a1)[left = of A]{\key{Bestellnummer}} edge (A);
                  \node[attribute](a2)[above left = of A]{Datum} edge (A);
                \node[relationship](rel1)[right = of A]{} edge node[auto,swap] {(1,1)}(A);
                \node[entity](B)[right = of rel1]{Kunde} edge node[auto,swap] {(1,*)} (rel1);
                  \node[attribute](b1)[right = of B]{\key{Kundennummer}} edge (B);
                  \node[attribute](b2)[above right = of B]{Name} edge (B);
            \end{tikzpicture}
            }
          \end{center}
          \begin{small}
            (B:) Kunde(\pk{Kundennummer}, Name)

            (A:) Bestellung(\pk{Bestellnummer}, Datum, \nn{\fk{Kundennummer}})
          \end{small}

          Der FK \fk{Kundennummer} ist eingabepflichtig: Jede Bestellung wird also genau einem Kunden zugeordnet. Der FK ist nicht-unikal: Mehrere Bestellungen k\"onnen also auf denselben Kunden verweisen. Es wird aber nicht gesichert, dass jeder Kunde mit mindestens einer Bestellung verkn\"upft ist. Diese Gesch\"aftsregel kann nicht beim Datenbankentwurf \enquote{festgeschrieben} werden, sondern muss durch die Anwendungssoftware erzwungen werden.
      \subsection{Der (0,1):(1,*) Beziehungstyp}
        Der (0,1):(1,*) Beziehungstyp unterscheidet sich vom (1,1):(1,*) Beziehungstyp dadurch, dass nicht jedes Objekt des Objekttyps A mit einem Objekt des Objekttyps B gekoppelt sein muss. Es m\"ussen aber wieder alle Objekte aus B mindestens eine Beziehung zu Objekten aus A haben.
        \begin{center}
          \scalebox{1}{
            \begin{tikzpicture}
              \node[circleA](A) at (0,0){Objekttyp A};
                \node[redspot](a1) at (0, 1.3) {a1};
                \node[redspot](a2) at (0.5, 0.8){a2};
                \node[redspot](a3) at (0.1, 0.3){a3};
                \node[redspot](a4) at (-0.8, -0.4){a4};
                \node[redspot](a5) at (-0.2, -0.8){a5};
              \node[circleB](B) at (4.5, 0){Objekttyp B};
                \node[bluespot](b1) at (4.5, 1.3) {b1};
                \node[bluespot](b2) at (5, 0.8) {b2};
                \node[bluespot](b3) at (4.6, 0.3) {b3};
              \path (a1) edge (b1)
                (a2) edge (b2)
                (a3) edge (b3)
                (a4) edge (b3);
            \end{tikzpicture}
          }
        \end{center}
\clearpage
        Wie schon beim (1,1):(1,*) Beziehungstyp, so ist auch hier die Nichtoptionalit\"at der Beziehungstyprichtung von B zu A im physischen Datenbankmodell nicht zu erzwingen. Die Transformation des (0,1):(1,*) Beziehungstyps kann nur wie beim (0,1):(0,*) Beziehungstyp erfolgen, also gem\"a\ss\ T07. Es muss auch hier der Verlust von semantischen Informationen in Kauf genommen werden.
       \subsubsection{Beispiel - Exklusivangebotsbestellung}
          Betrachten wir als Beispiel einen Versandhandel, der Exklusivangebote f\"ur seine Kunden erstellt. Jeder Kunde kann zu einem Angebot h\"ochstens eine Bestellung abschicken, er muss es aber nicht tun (Kardinalit\"at (0,1)). In einer Bestellung kann der Kunde sich aber nicht nur auf ein Angebot beziehen, sondern auch auf mehrere (Kardinalit\"at (1,*)).
          \begin{center}
            \scalebox{.7}{
              \begin{tikzpicture}[node distance=1.5cm, every edge/.style={link}]
                \node[entity](A){Angebot};
                  \node[attribute](a1)[left = of A]{\key{Angebotsnummer}} edge (A);
                  \node[attribute](a2)[above left = of A]{Angebotsdatum} edge (A);
                \node[relationship](rel1)[right = of A]{enth\"alt} edge node[auto,swap] {(0,1)}(A);
                \node[entity](B)[right = of rel1]{Bestellung} edge node[auto,swap] {(1,*)} (rel1);
                  \node[attribute](b1)[right = of B]{\key{Bestellnummer}} edge (B);
                  \node[attribute](b2)[above right = of B]{Bestelldatum} edge (B);
            \end{tikzpicture}
            }
          \end{center}
          \begin{small}
            (B:) Bestellung(\pk{Bestellnummer}, Bestelldatum)

            (A:) Angebot(\pk{Angebotsnummer}, Angebotsdatum, \fk{Bestellnummer})
          \end{small}
          Der FK \fk{Bestellnummer} ist nicht-eingabepflichtig: Ein Angebot muss also nicht unbedingt eine Bestellung hervorrufen. Der FK ist nicht-unikal: Mehrere Angebote k\"onnen in einer Bestellung genutzt werden. Es l\"asst sich aber nicht erzwingen, dass jede Bestellung sich auf mindestens ein Angebot bezieht. Die Datenbank w\"urde also durchaus zulassen, dass ein Kunde eine Bestellung t\"atigt, ohne ein entsprechendes Angebot vorliegen zu haben. Will man diesen, in der Praxis nicht hinnehmbaren Fehler vermeiden, kann dies nur durch eine entsprechende Gestaltung der Anwendungssoftware erreicht werden.
      \subsection{Der (0,*):(0,*) Beziehungstyp}
        Beziehungstypen, die in beiden Richtungen die Kardinalit\"at \enquote{*} aufweisen, lassen sich nicht direkt im physischen Datenbankmodell darstellen. Der Grund daf\"ur liegt in der Tatsache, dass Attributwerte nur atomare Werte haben d\"urfen. Der Wert eines FK kann somit nur auf eine Zeile und nicht auf mehrere Zeilen verweisen.

        Beim (0,*):(0,*) Beziehungstyp kann ein Objekt des Objekttyps A jedoch nicht nur mit keinem oder einem, sondern auch mit mehreren Objekten des Objekttyps B verbunden sein. Ebenso wie ein B-Objekt mit keinem, einem oder mehreren A-Objekten gekoppelt sein kann. Folgende Abbildung zeigt daf\"ur ein Beispiel.
        \begin{center}
          \scalebox{1}{
            \begin{tikzpicture}
              \node[circleA](A) at (0,0){Objekttyp A};
                \node[redspot](a1) at (0, 1.3) {a1};
                \node[redspot](a2) at (0.5, 0.8){a2};
                \node[redspot](a3) at (0.1, 0.3){a3};
                \node[redspot](a4) at (-0.8, -0.4){a4};
                \node[redspot](a5) at (-0.2, -0.8){a5};
              \node[circleB](B) at (4.5, 0){Objekttyp B};
                \node[bluespot](b1) at (4.5, 1.3) {b1};
                \node[bluespot](b2) at (5, 0.8) {b2};
                \node[bluespot](b3) at (4.6, 0.3) {b3};
                \node[bluespot](b4) at (3.7, -0.4) {b4};
                \node[bluespot](b5) at (4.3, -0.8) {b5};
              \path (a1) edge (b1)
                (a1) edge (b2)
                (b3) edge (a3)
                (b3) edge (a4)
                (a5) edge (b4);
            \end{tikzpicture}
          }
        \end{center}
        Die Repr\"asentation ist im physischen Datenbankmodell nur dadurch m\"oglich, dass man einen neuen - rein technisch bedingten - Hilfsobjekttyp A/B einf\"uhrt. In mancher Literatur wird auch von einem Koppel-Objekttyp gesprochen. A/B wird mit den Objekttypen A und B jeweils durch einen (0,*):(1,1) Beziehungstyp verbunden.
          \begin{center}
          \scalebox{1}{
            \begin{tikzpicture}
              \node[circleA](A) at (0,0){Objekttyp A};
                \node[redspot](a1) at (0, 1.3) {a1};
                \node[redspot](a2) at (0.5, 0.8){a2};
                \node[redspot](a3) at (0.1, 0.3){a3};
                \node[redspot](a4) at (-0.8, -0.4){a4};
                \node[redspot](a5) at (-0.2, -0.8){a5};
              \node[circleC](C) at (4.5, 0){Hilfstyp AB};
                \node[yellowspot](c1) at (4.5, 1.3) {c1};
                \node[yellowspot](c2) at (5.0, 0.8) {c2};
                \node[yellowspot](c3) at (4.6, 0.3) {c3};
                \node[yellowspot](c4) at (3.7, -0.4) {c4};
                \node[yellowspot](c5) at (4.3, -0.8) {c5};
              \node[circleB](B) at (9, 0){Objekttyp B};
                \node[bluespot](b1) at (9.0, 1.3) {b1};
                \node[bluespot](b2) at (9.5, 0.8) {b2};
                \node[bluespot](b3) at (9.1, 0.3) {b3};
                \node[bluespot](b4) at (8.2, -0.4) {b4};
                \node[bluespot](b5) at (8.8, -0.8) {b5};
              \path (a1) edge (c1)
                    (a1) edge (c2)
                    (a3) edge (c3)
                    (a4) edge (c4)
                    (a5) edge (c5)
                    (c1) edge (b1)
                    (c2) edge (b2)
                    (c3) edge (b3)
                    (c4) edge (b3)
                    (c5) edge (b4);
            \end{tikzpicture}
          }
        \end{center}
        Die beiden (1,1):(0,*) Beziehungstypen werden gem\"a\ss\ T06 umgewandelt. Zusammenfassend gilt f\"ur den (0,*):(0,*) Beziehungstyp die Transformationsregel T08.
        \subsubsection{Transformationsregel T08 f\"ur den (0,*):(0,*) Beziehungstyp}
          \tablefirsthead{%
          \hline
            \multicolumn{1}{|c}{\textbf{ER-Modell}} &
            \multicolumn{1}{|l}{\textbf{ }} &
            \multicolumn{1}{|c|}{\textbf{physisches Datenmodell}} \\
          \hline
          }
          \begin{supertabular}[h]{|l|l|p{7.75cm}|}
            \footnotesize Objekttyp A mit Schl\"ussel \pk{SA} & $\Rightarrow $ & \footnotesize Tabelle A mit PK \pk{SA}\\
            \hline
            \footnotesize Objekttyp B mit Schl\"ussel \pk{SB} & $\Rightarrow $ & \footnotesize Tabelle B mit PK \pk{SB}\\
            \hline
            \footnotesize (0,*):(0,*) Beziehungstyp & $\Rightarrow $ & \footnotesize Hilfstabelle A/B. \pk{SA} und \pk{SB} werden als eingabepflichtige nicht-unikale Fremdschl\"ussel in A/B aufgenommen. Die Kombination der FK \fk{SA} und \fk{SB} als unikal vereinbart. Sie bildet den PK von A/B\\
          \end{supertabular}
          \begin{center}
            \scalebox{.7}{
              \begin{tikzpicture}[node distance=1.5cm, every edge/.style={link}]
                \node[entity](A){A};
                  \node[attribute](a1)[left = of A]{\key{SA}} edge (A);
                \node[relationship](rel1)[right = of A]{} edge node[auto,swap] {(0,*)}(A);
                \node[entity](B)[right = of rel1]{B} edge node[auto,swap] {(0,*)} (rel1);
                  \node[attribute](b1)[right = of B]{\key{SB}} edge (B);
            \end{tikzpicture}
            }
          \end{center}
          \bild{}{transformationsregel_08_tabellen}{0.85}
        \subsubsection{Beispiel - Projektmitarbeiter}
          Als Beispiel betrachten wir noch einmal die Mitarbeiter, die an keinem, einem oder mehreren Projekten beteiligt sein k\"onnen, wobei ein Projekt (noch) von keinem, einem oder auch von mehreren Mitarbeitern bearbeitet wird.
          \begin{center}
            \scalebox{.7}{
              \begin{tikzpicture}[node distance=1.5cm, every edge/.style={link}]
                \node[entity](A){Projekt};
                  \node[attribute](a1)[left = of A]{\key{Projeknummer}} edge (A);
                  \node[attribute](a2)[above left = of A]{Bezeichnung} edge (A);
                \node[relationship](rel1)[right = of A]{arbeitet} edge node[auto,swap] {(0,*)}(A);
                \node[entity](B)[right = of rel1]{Mitarbeiter} edge node[auto,swap] {(0,*)} (rel1);
                  \node[attribute](b1)[right = of B]{\key{Personalnummer}} edge (B);
                  \node[attribute](b2)[above right = of B]{Name} edge (B);
            \end{tikzpicture}
            }
          \end{center}
          \begin{small}
            (B:) Mitarbeiter(\pk{Personalnummer}, Name)

            (A:) Projekt(\pk{Projektnummer}, Bezeichnung)

            (A/B:) MitarbeiterProjekt(\fk{\pk{Personalnummer + Projektnummer}})
          \end{small}
          Eine Zeile der Hilfstabelle \enquote{MitarbeiterProjekt} verweist auf genau einen Mitarbeiter und auf genau ein Projekt. Da der FK \fk{Personalnummer} f\"ur sich alleine nichtunikal ist, kann es mehrere Zeilen der Hilfstabelle geben, die auf denselben Mitarbeiter verweisen. Die Nichtunikalit\"at des FK \fk{Projektnummer} erm\"oglicht es, dass mehrere Zeilen der Hilfstabelle mit demselben Projekt verkn\"upft sind. Erst die Kombination \fk{Personalnummer +  Projektnummer} ist als unikal vereinbart und bildet den PK der Hilfstabelle.
      \subsection{Der (1,*):(0,*) Beziehungstyp}
        Beim (1,*):(0,*) Beziehungstyp muss ein Objekt des Objekttyps A mit mindestens einem oder mehreren Objekten des Objekttyps B verbunden sein, ein B-Objekt jedoch mit keinem, einem oder aber mehreren A-Objekten.

        \begin{center}
          \scalebox{1}{
            \begin{tikzpicture}
              \node[circleA](A) at (0,0){Objekttyp A};
                \node[redspot](a1) at (0, 1.3) {a1};
                \node[redspot](a2) at (0.5, 0.8){a2};
                \node[redspot](a3) at (0.1, 0.3){a3};
                \node[redspot](a4) at (-0.8, -0.4){a4};
                \node[redspot](a5) at (-0.2, -0.8){a5};
              \node[circleB](B) at (4.5, 0){Objekttyp B};
                \node[bluespot](b1) at (4.5, 1.3) {b1};
                \node[bluespot](b2) at (5, 0.8) {b2};
                \node[bluespot](b3) at (4.6, 0.3) {b3};
                \node[bluespot](b4) at (3.7, -0.4) {b4};
                \node[bluespot](b5) at (4.3, -0.8) {b5};
                \node[bluespot](b6) at (5.2, -1) {b6};
              \path (a1) edge (b1)
                    (a1) edge (b2)
                    (b3) edge (a3)
                    (b3) edge (a4)
                    (a2) edge (b4)
                    (a5) edge (b5);
            \end{tikzpicture}
          }
        \end{center}

        Analog zum (0,*):(0,*) Beziehungstyp muss der (1,*):(0,*) Beziehungstyp vor seiner Transformation in das physische Datenbankmodell durch die Einf\"uhrung eines neuen Objekttyps A/B in einen (1,1):(0,*) Beziehungstyp und einen (1,*):(1,1) Beziehungstyp umgewandelt werden.
\clearpage
        \begin{merke}
          Es wurde bereits gezeigt, dass der (1,*):(1,1) Beziehungstyp im physischen Datenbankmodell lediglich als (0,*):(1,1) Beziehungstyp dargestellt werden kann. Die Transformation des (1,*):(0,*) Beziehungstyps erfolgt damit nach der Transformationsregel T08.
        \end{merke}
        \subsubsection{Beispiel - \"Arzte mit Operation(en)}
          Als Beispiel betrachten wir \"Arzte, die Operationen durchf\"uhren. Eine Operation ohne \"Arzte gibt es nicht. Mitunter wird eine Operation aber von mehreren \"Arzten ausgef\"uhrt.
          \begin{center}
            \scalebox{.65}{
              \begin{tikzpicture}[node distance=1.5cm, every edge/.style={link}]
                \node[entity](A){Operation};
                  \node[attribute](a1)[left = of A]{\key{Operationsnummer}} edge (A);
                  \node[attribute](a2)[above left = of A]{Bezeichnung} edge (A);
                \node[relationship](rel1)[right = of A]{f\"uhrt durch} edge node[auto,swap] {(1,*)}(A);
                \node[entity](B)[right = of rel1]{Arzt} edge node[auto,swap] {(0,*)} (rel1);
                  \node[attribute](b1)[right = of B]{\key{Personalnummer}} edge (B);
                  \node[attribute](b2)[above right = of B]{Name} edge (B);
            \end{tikzpicture}
            }
          \end{center}
          \begin{small}
            (B:) Arzt(\pk{Personalnummer}, Name)

            (A:) Operation(\pk{Operationsnummer}, Bezeichnung)

            (B/A:) ArztOperation(\fk{\pk{Personalnummer + Operationsnummer}})
          \end{small}
          Durch jede Zeile der Hilfstabelle \enquote{ArztOperation} wird ein Arzt mit einer Operation in Verbindung gebracht. Keiner der beiden Fremdschl\"ussel \fk{Personalnummer} beziehungsweise \fk{Operationsnummer} muss f\"ur sich genommen unikal sein. Damit k\"onnen mehrere Zeilen der Hilfstabelle auf denselben Arzt verweisen. Ebenso k\"onnen mehrere Zeilen dieselbe Operation betreffen.

          Es kann aber durch die Tabellentypbeschreibungen nicht erzwungen werden, dass jede Operationsnummer auch tats\"achlich als FK in der Hilfstabelle auftaucht. In der DB kann also die falsche Aussage gespeichert werden, dass einer Operation kein Arzt zugeordnet ist. Dieser Fehler kann wiederum nur software-technisch vermieden werden.
      \subsection{Der (1,*):(1,*) Beziehungstyp}
        Beim (1,*):(1,*) Beziehungstyp ist jedes Objekt des Objekttyps A mit einem oder mehreren Objekten des Objekttyps B verbunden, ebenso wie jedes B-Objekt mit einem oder mehreren A-Objekten gekoppelt ist.

        \begin{center}
          \scalebox{1}{
            \begin{tikzpicture}
              \node[circleA](A) at (0,0){Objekttyp A};
                \node[redspot](a1) at (0, 1.3) {a1};
                \node[redspot](a2) at (0.5, 0.8){a2};
                \node[redspot](a3) at (0.1, 0.3){a3};
                \node[redspot](a4) at (-0.8, -0.4){a4};
                \node[redspot](a5) at (-0.2, -0.8){a5};
              \node[circleB](B) at (4.5, 0){Objekttyp B};
                \node[bluespot](b1) at (4.5, 1.3) {b1};
                \node[bluespot](b2) at (5, 0.8) {b2};
                \node[bluespot](b3) at (4.6, 0.3) {b3};
                \node[bluespot](b4) at (3.7, -0.4) {b4};
                \node[bluespot](b5) at (4.3, -0.8) {b5};
              \path (a1) edge (b1)
                (a2) edge (b2)
                (a3) edge (b3)
                (a4) edge (b3)
                (a5) edge (b4)
                (a5) edge (b5);
            \end{tikzpicture}
          }
        \end{center}
        Dieser Beziehungstyp muss vor der Transformation in einen (1,*):(1,1) und einen\\ (1,1):(1,*) Beziehungstyp umgeformt werden. Beide neuen Beziehungstypen m\"ussen wiederum nur unter Semantikverlust in (0,*):(1,1) bzw. (1,1):(0,*) Beziehungstypen \"uber\-f\"uhrt werden. Die Transformation des (1,*):(1,*) Beziehungstyps erfolgt damit - so wie f\"ur den (1,*):(0,*) Beziehungstyp - nach der Transformationsregel T08.
        \subsubsection{Beispiel - B\"uchersachgruppe}
          Als Beispiel betrachten wir eine Bibliothek, in der die B\"ucher den
          einzelnen Sachgruppen zugeordnet werden. Ein Buch wird meist nur f\"ur
          eine Sachgruppe, mitunter aber auch zu mehreren Sachgruppen
          zugeordnet. Eine Sachgruppe wird erst dann eingef\"uhrt, wenn sie
          wenigstens ein Buch enth\"alt. Die Transformation sieht folgenderma\ss
          en aus.

          \begin{center}
            \scalebox{.7}{
              \begin{tikzpicture}[node distance=1.5cm, every edge/.style={link}]
                \node[entity](A){Buch};
                  \node[attribute](a1)[left = of A]{\key{ISBN}} edge (A);
                  \node[attribute](a2)[above left = of A]{Titel} edge (A);
                \node[relationship](rel1)[right = of A]{geh\"ort zu} edge node[auto,swap] {(1,*)}(A);
                \node[entity](B)[right = of rel1]{Sachgruppe} edge node[auto,swap] {(1,*)} (rel1);
                  \node[attribute](b1)[right = of B]{\key{Sachgruppencode}} edge (B);
                  \node[attribute](b2)[above right = of B]{Bezeichnung} edge (B);
            \end{tikzpicture}
            }
          \end{center}
          \begin{small}
            (A:) Buch(\pk{ISBN}, Titel)

            (B:) Sachgruppe(\pk{Sachgruppencode}, Bezeichnung)

            (A/B:) BuchSachgruppe(\fk{\pk{ISBN + Sachgruppencode}})
          \end{small}
          
          Eine Zeile der Hilfstabelle \enquote{Buch/Sachgruppe} ordnet ein Buch
          einer Sachgruppe zu. Da weder der FK \fk{ISBN} noch der FK
          \fk{Sachgruppencode} f\"ur sich unikal sind, k\"onnen mehrere Zeilen
          der Hilfstabelle auf dasselbe Buch oder auch auf dieselbe Sachgruppe
          verweisen. Die Tabellentypbeschreibungen sichern aber weder, dass
          einem Buch wenigstens eine Sachgruppe zugeordnet wird, noch k\"onnen
          sie garantieren, dass es keine \enquote{leere} Sachgruppe gibt. Diese
          Gesch\"aftsregeln m\"ussen von der Anwendungssoftware durchgesetzt
          werden.
      \subsection{Transformation von Beziehungsattributen}
        Bei der Transformation von Eigenschaften eines Beziehungstyps (Beziehungsattributen) kann folgende Verallgemeinerung angewandt werden:
        \begin{itemize}
          \item Das Beziehungsattribut \enquote{wandert} in die Tabelle auf der 1-Seite des Beziehungstyps.
          \item Gibt es auf beiden Seiten einen * in der Kardinalit\"at, dann \enquote{wandert/wandern} das/die Beziehungsattribut(e) in das Hilfsobjekt.
          \item Gibt es keine *-Seite, dann muss die angewandte Transformationsregel betrachtet und gepr\"uft werden, in welcher Tabelle das Beziehungsattribut am sinnvollsten ist. Normalerweise ist dies die Tabelle mit dem FK.
        \end{itemize}
        \subsubsection{Transformation von Beziehungsattributen bei einer m:n-Beziehung}
          \begin{center}
            \scalebox{.7}{
              \begin{tikzpicture}[node distance=1.5cm, every edge/.style={link}]
                \node[entity](A){Maschine};
                  \node[attribute](a1)[left = of A]{\key{Maschinennummer}} edge (A);
                  \node[attribute](a2)[above left = of A]{Bezeichnung} edge (A);
                \node[relationship](rel1)[right = of A]{stellt her} edge node[auto,swap] {(1,*)}(A);
                  \node[attribute](rel11)[above = of rel1]{Zeitstempel} edge (rel1);
                \node[entity](B)[right = of rel1]{Bauteil} edge node[auto,swap] {(1,*)} (rel1);
                  \node[attribute](b1)[right = of B]{\key{Teilenummer}} edge (B);
                  \node[attribute](b2)[above right = of B]{Bezeichnung} edge (B);
            \end{tikzpicture}
            }
          \end{center}
          \begin{small}
            (A:) Maschine(\pk{Maschinennummer}, Bezeichnung)

            (B:) Bauteil(\pk{Teilenummer}, Bezeichnung)

            (A/B:) Herstellung(\fk{\pk{Maschinennummer + Teilenummer}}, Zeitstempel)
          \end{small}
        \subsubsection{Transformation von Beziehungsattributen bei einer 1:n-Beziehung}
          \begin{center}
            \scalebox{.7}{
              \begin{tikzpicture}[node distance=1.5cm, every edge/.style={link}]
                \node[entity](A){Maschine};
                  \node[attribute](a1)[left = of A]{\key{Maschinennummer}} edge (A);
                  \node[attribute](a2)[above left = of A]{Bezeichnung} edge (A);
                \node[relationship](rel1)[right = of A]{stellt her} edge node[auto,swap] {(1,*)}(A);
                  \node[attribute](rel11)[above = of rel1]{Zeitstempel} edge (rel1);
                \node[entity](B)[right = of rel1]{Bauteil} edge node[auto,swap] {(1,1)} (rel1);
                  \node[attribute](b1)[right = of B]{\key{Teilenummer}} edge (B);
                  \node[attribute](b2)[above right = of B]{Bezeichnung} edge (B);
              \end{tikzpicture}
            }
          \end{center}
          \begin{small}
            (B): Maschine(\pk{Maschinennummer}, Bezeichnung)

            (A): Bauteil(\pk{Teilenummer}, Bezeichnung, \fk{Maschinennummer}, Zeitstempel)
          \end{small}
        \subsubsection{Transformation von Beziehungsattributen bei einer 1:1-Beziehung}
          Da bei Beziehungen des Typs 1:1 nur extrem selten Beziehungsattribute vorkommen, muss in solchen F\"allen eine Einzelfallbetrachtung durchgef\"uhrt werden. Daher wird an dieser Stelle auf ein Transformationsbeispiel verzichtet.
\clearpage
			\section{Transformation des Begleitbeispiels Bundeswehr}
        In diesem Abschnitt sollen die bis hier erlernten Dinge praktisch ge\"ubt werden. Aufgabe soll es sein, f\"ur das begleitende Beispiel \enquote{Bundeswehr}, ein ER-Modell zu konzipieren und dieses anschlie\ss end, mittels der in den vorigen Abschnitten kennengelernten Tranformationsregeln, in ein relationales Modell umzuandeln.

        Nachfolgendes ER-Modell kann als L\"osungsansatz f\"ur das Beispiel \enquote{Bundeswehr} gew\"ahlt werden.

					\begin{tabular}{>{\textbf\bgroup}p{5cm}<{\egroup}>{\small}p{10.1cm}}
						Dienstposten & (\pk{Dienstposten\_ID}, Beginndatum, Enddatum, Aufgabenbeschreibung) \\
						Soldat & (\pk{Personen\_ID}, PK, Vorname, Nachname, Dienstgrad) \\
						Ausr\"ustung & (\pk{Versorgungsnummer}, Bezeichnung, Verwendungszweck, Material, Farbe) \\
						Dieststelle & (\pk{Dienststellennummer}, Bezeichnung, Typ, Gr\"o\ss e) \\
						Standort & (\pk{Ort\_ID}, PLZ, Ort, Stra\ss e, Hausnummer) \\
						& Aufl\"osen des Beziehungstyps (besetzt) mit T04) \\
						& Aufl\"osen des Beziehungstyps (besteht aus) mit T06) \\
						& Aufl\"osen der Beziehungstypen (besitzt, befindet sich an) mit T08) \\
						& Aufl\"osen des Beziehungstyps (ist untergeordnet) mit T11) \\
						DienststelleStandort & (\fk{\pk{Dienststellennummer + Ort\_ID}}) \\
						SoldatAusr\"ustung & (\fk{\pk{Personen\_ID + Versorgungsnummer}}) \\
					\end{tabular}

        \begin{center}
            \scalebox{.5}{
              \begin{tikzpicture}[node distance=1.5cm, every edge/.style={link}]
                \node[entity](dienstposten){Dienstposten};
                  \node[attribute](dienstpostenid)[above left = of dienstposten]{\key{Dienstposten\_ID}} edge (dienstposten);
                  \node[attribute](begindatum)[above right = of dienstposten]{Beginndatum} edge (dienstposten);
                  \node[attribute](enddatum)[below right = of dienstposten]{Enddatum} edge (dienstposten);
                  \node[attribute](aufgabenbeschreibung)[below left = of dienstposten]{Aufgabenbeschreibung} edge (dienstposten);
                \node[relationship](besetzt)[left = 5.1cm of dienstposten]{besetzt} edge node[auto,swap] {(0,1)}(dienstposten);
                \node[entity](soldat)[below left = 7cm of dienstposten]{Soldat} edge node [auto,swap] {(1,1)}(besetzt);
                  \node[attribute](personenid)[above left = 1.6cm of soldat]{\key{Personen\_ID}} edge (soldat);
                  \node[attribute](pk)[above right = of soldat]{PK} edge (soldat);
                  \node[attribute](name)[below right = of soldat]{Nachname} edge (soldat);
                  \node[attribute](vorname)[below left = of soldat]{Vorname} edge (soldat);
                  \node[attribute](dienstgrad)[left = of soldat]{Dienstgrad} edge (soldat);
                \node[relationship](besitzt)[below = 3cm of soldat]{besitzt} edge node[auto,swap] {(0,*)}(soldat);
                \node[entity](ausruestung)[below = 8cm of soldat]{Ausr\"ustung} edge node[auto,swap] {(0,*)}(besitzt);
                  \node[attribute](versorgungsnummer)[above left = of ausruestung]{\key{Versorgungsnummer}} edge (ausruestung);
                  \node[attribute](bezeichnung)[left = of ausruestung]{Bezeichnung} edge (ausruestung);
                  \node[attribute](verwendungszweck)[above right = of ausruestung]{Verwendungszweck} edge (ausruestung);
                  \node[attribute](material)[below right = of ausruestung]{Material} edge (ausruestung);
                  \node[attribute](farbe)[below left = of ausruestung]{Farbe} edge (ausruestung);
                \node[relationship](bestehtaus)[right = 4.7cm of dienstposten]{besteht aus} edge node[auto,swap] {(1,1)}(dienstposten);

                \node[entity](dienststelle)[below right = 7cm of dienstposten]{Dienststelle} edge node [auto,swap] {(1,*)}(bestehtaus);

                  \node[attribute](dienststellennummer)[above left = of dienststelle]{\key{Dienststellennummer}} edge (dienststelle);
                  \node[attribute](bezeichung)[below left = of dienststelle]{Bezeichnung} edge (dienststelle);
                  \node[attribute](groesse)[below right = 0.9cm of dienststelle]{Gr\"o\ss e} edge (dienststelle);
                  \node[attribute](typ)[above right = 0.9cm of dienststelle]{Typ} edge (dienststelle);

                \node[relationship](istuntergeordnet)[right = 2cm of dienststelle]{ist untergeordnet};
                  \node[auto,swap](l1) at (11,-3.6) {(1,1)};
                  \node[auto,swap](l2) at (11,-8.7) {(0,*)};
                  \path [draw, -] (dienststelle) |- ($(dienststelle.south) +(1.5, -1.5)$) -| (istuntergeordnet);
                  \path [draw, -] (dienststelle) |- ($(dienststelle.north) +(1.5, 1.5)$) -| (istuntergeordnet);

                \node[relationship](befidetsichan)[below = 2cm of dienststelle]{befindet sich an} edge node[auto,swap] {(1,*)}(dienststelle);
                \node[entity](standort)[below = 8cm of dienststelle]{Standort} edge node[auto,swap] {(1,*)}(befidetsichan);
                  \node[attribute](ortid)[above left = of standort]{\key{Ort\_ID}} edge (standort);
                  \node[attribute](plz)[left = of standort]{PLZ} edge (standort);
                  \node[attribute](ort)[above right = of standort]{Ort} edge (standort);
                  \node[attribute](strasse)[below right = of standort]{Stra\ss e} edge (standort);
                  \node[attribute](hausnummer)[below left = of standort]{Hausnummer} edge (standort);
              \end{tikzpicture}
            }
            \end{center}
      Bei weiterer Betrachtung des ER-Modells sollte auffallen, dass die Anzahl an Ausr\"ustungsgegenst\"anden eines Soldaten nicht ber\"ucksichtigt wurde. In der Tabelle \enquote{SoldatAusr\"ustung} kann ein Attribut \enquote{Anzahl} aufgenommen werden, welches die jeweilige Anzahl an Ausr\"ustungsgegenst\"anden eines Soldaten enth\"alt. Die neue Tabelle sieht dann folgenderma\ss en aus:

      \begin{small}
        SoldatAusr\"ustung(\fk{\pk{Personen\_ID + Versorgungsnummer}}, Anzahl)
      \end{small}

      Dieser Fall muss sich dann mit einem Beziehungsattribut (vgl. Abschnitt \ref{attributes_of_entities}) im ER-Modell wiederfinden. Das angepasste ER-Modell sieht folgenderma\ss en aus:
      \begin{center}
        \scalebox{.5}{
        \begin{tikzpicture}[node distance=1.5cm, every edge/.style={link}]
          \node[entity](dienstposten){Dienstposten};
            \node[attribute](dienstpostenid)[above left = of dienstposten]{\key{Dienstposten\_ID}} edge (dienstposten);
            \node[attribute](begindatum)[above right = of dienstposten]{Begindatum} edge (dienstposten);
            \node[attribute](enddatum)[below right = of dienstposten]{Enddatum} edge (dienstposten);
            \node[attribute](aufgabenbeschreibung)[below left = of dienstposten]{Aufgabenbeschreibung} edge (dienstposten);
          \node[relationship](besetzt)[left = 5.1cm of dienstposten]{besetzt} edge node[auto,swap] {(0,1)}(dienstposten);
          \node[entity](soldat)[below left = 7cm of dienstposten]{Soldat} edge node [auto,swap] {(1,1)}(besetzt);
            \node[attribute](personenid)[above left = 1.6cm of soldat]{\key{Personen\_ID}} edge (soldat);
            \node[attribute](pk)[above right = of soldat]{PK} edge (soldat);
            \node[attribute](name)[below right = of soldat]{Name} edge (soldat);
            \node[attribute](vorname)[below left = of soldat]{Vorname} edge (soldat);
            \node[attribute](dienstgrad)[left = of soldat]{Dienstgrad} edge (soldat);
          \node[relationship](besitzt)[below = 3cm of soldat]{besitzt} edge node[auto,swap] {(0,*)}(soldat);
            \node[attribute](anzahl)[right  = of besitzt]{Anzahl} edge (besitzt);

          \node[entity](ausruestung)[below = 8cm of soldat]{Ausr\"ustung} edge node[auto,swap] {(0,*)}(besitzt);
            \node[attribute](versorgungsnummer)[above left = of ausruestung]{\key{Versorgungsnummer}} edge (ausruestung);
            \node[attribute](bezeichnung)[left = of ausruestung]{Bezeichnung} edge (ausruestung);
            \node[attribute](verwendungszweck)[above right = of ausruestung]{Verwendungszweck} edge (ausruestung);
            \node[attribute](material)[below right = of ausruestung]{Material} edge (ausruestung);
            \node[attribute](farbe)[below left = of ausruestung]{Farbe} edge (ausruestung);
          \node[relationship](bestehtaus)[right = 4.7cm of dienstposten]{besteht aus} edge node[auto,swap] {(1,1)}(dienstposten);

          \node[entity](dienststelle)[below right = 7cm of dienstposten]{Dienststelle} edge node [auto,swap] {(1,*)}(bestehtaus);

            \node[attribute](dienststellennummer)[above left = of dienststelle]{\key{Dienststellennummer}} edge (dienststelle);
            \node[attribute](bezeichung)[below left = of dienststelle]{Bezeichnung} edge (dienststelle);
            \node[attribute](groesse)[below right = 0.9cm of dienststelle]{Gr\"o\ss e} edge (dienststelle);
            \node[attribute](typ)[above right = 0.9cm of dienststelle]{Typ} edge (dienststelle);

          \node[relationship](istuntergeordnet)[right = 2cm of dienststelle]{ist untergeordnet};
            \node[auto,swap](l1) at (11,-3.6) {(0,1)};
            \node[auto,swap](l2) at (11,-8.7) {(0,*)};
            \path [draw, -] (dienststelle) |- ($(dienststelle.south) +(1.5, -1.5)$) -| (istuntergeordnet);
            \path [draw, -] (dienststelle) |- ($(dienststelle.north) +(1.5, 1.5)$) -| (istuntergeordnet);

          \node[relationship](befidetsichan)[below = 2cm of dienststelle]{befindet sich an} edge node[auto,swap] {(1,*)}(dienststelle);
          \node[entity](standort)[below = 8cm of dienststelle]{Standort} edge node[auto,swap] {(1,*)}(befidetsichan);
            \node[attribute](ortid)[above left = of standort]{\key{Ort\_ID}} edge (standort);
            \node[attribute](plz)[left = of standort]{PLZ} edge (standort);
            \node[attribute](ort)[above right = of standort]{Ort} edge (standort);
            \node[attribute](strasse)[below right = of standort]{Stra\ss e} edge (standort);
            \node[attribute](hausnummer)[below left = of standort]{Hausnummer} edge (standort);
        \end{tikzpicture}
        }
      \end{center}

    \section{Transformation rekursiver Beziehungstypen}
      \label{recursive_relatiaons}
      Da bei Rekursiv-Beziehungstypen die miteinander gekoppelten Objekte beide im selben Objekttyp liegen, k\"onnen sie nicht - wie bei den bin\"aren Beziehungstypen - durch ihre Zugeh\"origkeit zum jeweiligen Objekttyp unterschieden werden. Bei Rekursiv-Be\-ziehungs\-typen m\"ussen die Objekte hinsichtlich ihrer Rolle unterschieden werden. Deshalb soll folgende Sprachregelung eingef\"uhrt werden. Wir sprechen bei einem Rekursiv-Beziehungstyp allgemein von einem \enquote{Sender} (rechte Seite), der eine Nachricht an einen \enquote{Empf\"anger} (linke Seite) sendet. Als Beispiel daf\"ur ist in der folgenden Abbildung ein (0,1):(0,*) Rekursiv-Beziehungstyp dargestellt.

      \begin{center}
        \scalebox{.7}{
          \begin{tikzpicture}[node distance=1.5cm, every edge/.style={link}]
            \node[entity](A){A};
              \node[attribute](a1)[left = of A]{\key{SA}} edge (A);
            \node[relationship](rel1)[right = of A]{};
            \node[auto,swap](l1) at (5.6,-1.4) {(0,*) empf\"angt Nachricht};
            \node[auto,swap](l2) at (5.6,1.4) {(0,1) sendet Nachricht};
            \path [draw, -] (A) |- ($(A.south) + (0.5,-0.5)$) -| (rel1);
            \path [draw, -] (A) |- ($(A.north) +(0.5, 0.5)$) -| (rel1);
          \end{tikzpicture}
        }
      \end{center}

      An dieser Stelle sei auf den Abschnitt \ref{syntaxofrecursiverelationtypes} mit der \tabelle{combinationsrecurisverelationtyps} verwiesen. In der \tabelle{combinationsrecurisverelationtyps} sind die 7 m\"oglichen Rekursiv-Beziehungstypen aufgelistet.

      \subsection{Der (1,1):(1,1) Rekursiv-Beziehungstyp}
        Beim (1,1):(1,1) Beziehungstyp muss jedes Objekt des Objekttyps genau eine Nachricht an ein Objekt desselben Objekttyps senden. Umgekehrt muss jedes Objekt genau eine Nachricht von einem anderen Objekt empfangen. Die folgende Abbildung zeigt die vorliegenden Verh\"altnisse.

        \begin{center}
          \scalebox{1}{
            \begin{tikzpicture}[]
              \node[circleA](A){Objekttyp A};
              \node[redspot](a1) at (1.0, 0.5){a1};
              \node[redspot](a2) at (0.0, 1.7){a2};
              \node[redspot](a3) at (-1.0, 0.5){a3};
              \node[redspot](a4) at (-0.5, -0.4) {a4};

      \path[->] (a1) edge[bend right=45] (a2)
                (a2) edge[bend right=45] (a3)
                (a3) edge[bend right=45] (a1)
                (a4) edge[loop right, out=270, in = 0, looseness=18] (a4);
            \end{tikzpicture}
          }
        \end{center}

        Wie man sehen kann, k\"onnen ausschlie\ss lich Objekt-Zyklen gebildet werden ($a_1 \rightarrow a_2 \rightarrow a_3 \rightarrow a_1$). Im Minimalfall gibt es nur ein Objekt, welches dann einen Ein-Objekt-Zyklus bildet ($a_1 \rightarrow a_1$).

        Bei der Repr\"asentation von Rekursiv-Beziehungstypen im physischen Datenbankmodell muss zu\-n\"ach\-st jedes Objekt des Objekttyps A - unabh\"angig von den anderen Objekten - in einer Tabellenzeile gespeichert werden. Die Beziehung zwischen zwei Objekten $a_1$ und $a_2$ kann nun dadurch ausgedr\"uckt werden, dass in der Tabellenzeile von $a_2$ (beim Empf\"anger) auf das Objekt $a_1$ (auf den Sender) verwiesen wird. Der Verweis wird, wie schon bei den bin\"aren Beziehungstypen, durch Abspeichern des Indentifikators von $a_1$ realisiert. Dieser Identifikator ist aber der Wert, den der Prim\"arschl\"ussel von A im Falle des Objekts $a_1$ annimmt. Die Tabelle f\"ur den Objekttyp A muss somit den Schl\"ussel des Objekttyps A in doppelter Ausf\"uhrung aufnehmen:
\clearpage
        \begin{enumerate}
          \item als Prim\"arschl\"ussel, um eine Zeile der Tabelle eindeutig identifizieren zu k\"onnen;
          \item als Fremdschl\"ussel, um auf eine andere (oder auch auf dieselbe) Zeile der Tabelle, n\"amlich auf den Sender verweisen zu k\"onnen.
        \end{enumerate}
        Da die Spaltenbezeichnungen einer Tabelle voneinander verschieden sein m\"ussen, ist es erforderlich, f\"ur das Attribut (bzw. die Attribute) des PK im Falle ihrer \enquote{Wiedergeburt} als FK, andere Bezeichnungen zu vergeben.

        Im physischen Datenbankmodell l\"asst sich die Forderung, dass jedes Objekt genau eine Nachricht empfangen muss, einfach dadurch sichern, dass der FK, der auf den Sender verweist, als \textit{eingabepflichtig} deklariert wird, dass also jedes Objekt auf \enquote{sein} Senderobjekt verweisen muss. Dass ein und dasselbe Objekt nicht von mehreren Empf\"angern als ihr Sender angegeben werden kann, wird durch die Unikalit\"at des FK erreicht.
        \subsubsection{Transformationsregel T09 f\"ur den (1,1):(1,1) Rekursiv-Beziehungstyp}
          \tablefirsthead{%
          \hline
            \multicolumn{1}{|c}{\textbf{ER-Modell}} &
            \multicolumn{1}{|l}{\textbf{ }} &
            \multicolumn{1}{|c|}{\textbf{physisches Datenmodell}} \\
          \hline
          }
          \begin{supertabular}[h]{|l|l|p{7.5cm}|}
            \footnotesize Objekttyp A mit Schl\"ussel \pk{SA} & $\Rightarrow $ & \footnotesize Tabelle A mit PK \pk{SA}\\
            \hline
            \footnotesize (1,1):(1,1) Rekursiv-Beziehungstyp & $\Rightarrow $ & \footnotesize \pk{SA} wird in A mit einer anderen Attributbezeichnung \enquote{gedoppelt} und als eingabepflichtiger unikaler FK \fk{SA'} vereinbart, der auf den Sender verweist.
            (\fk{SA'} [NN, UN])\\
          \end{supertabular}

      \begin{center}
        \scalebox{.7}{
          \begin{tikzpicture}[node distance=1.5cm, every edge/.style={link}]
            \node[entity](A){A};
              \node[attribute](a1)[left = of A]{\key{SA}} edge (A);
            \node[relationship](rel1)[right = of A]{};
            \node[auto,swap](l1) at (5.6,-1.4) {(1,1) empf\"angt Nachricht};
            \node[auto,swap](l2) at (5.6,1.4) {(1,1) sendet Nachricht};
            \path [draw, -] (A) |- ($(A.south) + (0.5,-0.5)$) -| (rel1);
            \path [draw, -] (A) |- ($(A.north) +(0.5, 0.5)$) -| (rel1);
          \end{tikzpicture}
        }
      \end{center}

      \bild{}{transformationsregel_09_tabellen}{0.35}

        \subsubsection{Beispiel - Kreis einer Turnergruppe}
          Als Beispiel betrachten wir eine Turnergruppe, die bei einem Sportfest einen Kreis bilden soll. F\"ur jeden Turner wird festgelegt, wer sein rechter Nachbar ist.

          Von jedem Turner wird obligatorisch auf seinen (einzigen) rechten Nachbarn verwiesen. Durch die Unikalit\"at des FK wird au\ss erdem gesichert, dass ein Turner nur f\"ur einen anderen Turner rechter Nachbar sein kann. Da jedem Turner ein rechter Nachbar zugeordnet wird, muss auch jeder Turner rechter Nachbar genau eines anderen Turners sein. Damit ist die Kreisstruktur der Aufstellung der Turner erzwungen. Allerdings l\"asst sich durch die Tabellentypbeschreibung nicht ausschlie\ss en, dass ein Turner als sein eigener rechter Nachbar eingesetzt wird. (Ein-Objekt-Zyklus).

          Schwierig wird bei diesem Beziehungstyp die Speicherung der Daten so vorzunehmen, dass keine Integrit\"atsverletzung auftritt. Es sind entsprechende software-technische Ma\ss nahmen zu treffen, die hier nicht weiter erl\"autert werden sollen.
          \begin{center}
            \scalebox{.7}{
              \begin{tikzpicture}[node distance=1.5cm, every edge/.style={link}]
                \node[entity](A){Turner};
                  \node[attribute](a1)[left = of A]{\key{Mitgliedsnummer}} edge (A);
                  \node[attribute](a2)[above left= of A]{Name} edge (A);
                  \node[attribute](a3)[below left = of A]{rechter-Nachbar} edge (A);
                \node[relationship](rel1)[right = of A]{};
                \node[auto,swap](l1) at (5.6,-1.4) {(1,1) ist rechter Nachbar};
                \node[auto,swap](l2) at (5.6,1.4) {(1,1) hat rechten Nachbar};
                \path [draw, -] (A) |- ($(A.south) + (0.5,-0.5)$) -| (rel1);
                \path [draw, -] (A) |- ($(A.north) +(0.5, 0.5)$) -| (rel1);
              \end{tikzpicture}
            }
          \end{center}
          \begin{small}
            Turner(\pk{Mitgliedsnummer}, Name, \fk{rechter-Nachbar} [NN, UN])
          \end{small}
      \subsection{Der (0,1):(0,1) Rekursiv-Beziehungstyp}
        Beim (0,1):(0,1) Rekursiv-Beziehungstyp kann ein Objekt keine oder eine Nachricht an ein anderes Objekt (oder an sich selbst) senden. Andererseits kann ein Objekt keine oder eine Nachricht empfangen.
        \begin{center}
          \scalebox{1}{
            \begin{tikzpicture}[]
              \node[circleA](A){Objekttyp A};
              \node[redspot](a1) at (1.0, 0.5){a1};
              \node[redspot](a2) at (0, 1.7){a2};
              \node[redspot](a3) at (-1.0, 0.5){a3};
              \node[redspot](a4) at (-1, -0.6) {a4};
              \node[redspot](a5) at (1, -0.4) {a5};
              \node[redspot](a6) at (0.5, -0.8) {a6};
              \node[redspot](a7) at (0, -1.2) {a7};
              \node[redspot](a8) at (0, 0.85){a8};

      \path[->] (a1) edge[bend right=45] (a2)
                (a2) edge[bend right=45] (a3)
                (a3) edge[bend right=45] (a1)
                (a4) edge[loop right, out=270, in = 0, looseness=18] (a4)
                (a5) edge (a6)
                (a6) edge (a7);
            \end{tikzpicture}
          }
        \end{center}
        Beim (0,1):(0,1) Rekursiv-Beziehungstyp sind also - wie schon beim (1,1):(1,1) Rekursiv-Be\-zieh\-ungs\-typ - Objekt-Zyklen ($a_1 \rightarrow a_2 \rightarrow a_3 \rightarrow a_1$) m\"oglich, die im Minimalfall Ein-Objekt-Zyklen ($a_4 \rightarrow a_4$) sind. Dar\"uber hinaus kann es Objekt-Ketten ($a_5 \rightarrow a_6 \rightarrow a_7$) geben, die gegebenenfalls nur ein einziges Objekt enthalten ($a_8$). Ein solches Objekt ist dann weder Sender noch Empf\"anger.

        F\"ur die Umsetzung im physischen Modell, betrachten wir den Fall, dass die Objekte einen Verweis auf \enquote{ihren} Sender ben\"otigen. Dieser Sender kann der Empf\"anger selbst oder ein anderes A-Objekt sein. Da nicht jedes Objekt ein Empf\"anger ist, muss der FK, der auf den Sender verweist, als nicht-eingabepflichtig deklariert werden. Andererseits muss er unikal sein, damit ein $a_1$ Objekt nur von h\"ochstens einem Objekt $a_2$ als \enquote{sein} Sender ausgewiesen werden kann. Zusammengefasst ist dies die Transformationsregel T10.
\clearpage
        \subsubsection{Transformationsregel T10 f\"ur den (0,1):(0,1) Rekursiv-Beziehungstyp}
          \tablefirsthead{%
          \hline
            \multicolumn{1}{|c}{\textbf{ER-Modell}} &
            \multicolumn{1}{|l}{\textbf{ }} &
            \multicolumn{1}{|c|}{\textbf{physisches Datenmodell}} \\
          \hline
          }
          \begin{supertabular}[h]{|l|l|p{7.5cm}|}
            \footnotesize Objekttyp A mit Schl\"ussel \pk{SA} & $\Rightarrow $ & \footnotesize Tabelle A mit PK \pk{SA}\\
            \hline
            \footnotesize (0,1):(0,1) Rekursiv-Beziehungstyp & $\Rightarrow $ & \footnotesize \pk{SA} wird in A mit einer anderen Attributbezeichnung \enquote{gedoppelt} und als nicht-eingabepflichtiger unikaler FK \fk{SA'} vereinbart, der auf den Sender verweist. (\un{\fk{SA'}})\\
          \end{supertabular}
          \begin{center}
            \scalebox{.7}{
              \begin{tikzpicture}[node distance=1.5cm, every edge/.style={link}]
                \node[entity](A){A};
                  \node[attribute](a1)[left = of A]{\key{SA}} edge (A);
                \node[relationship](rel1)[right = of A]{};
                \node[auto,swap](l1) at (5.6,-1.4) {(0,1) empf\"angt Nachricht};
                \node[auto,swap](l2) at (5.6,1.4) {(0,1) sendet Nachricht};
                \path [draw, -] (A) |- ($(A.south) + (0.5,-0.5)$) -| (rel1);
                \path [draw, -] (A) |- ($(A.north) +(0.5, 0.5)$) -| (rel1);
              \end{tikzpicture}
            }
          \end{center}

          \bild{}{transformationsregel_10_tabellen}{0.35}

        \subsubsection{Beispiel - Ersatzbus}
          Nehmen wir als Beispiel an, dass in einem Busreiseunternehmen festgelegt wird, welcher Bus als Ersatz verwendet werden soll, wenn ein Bus defekt ist. Dabei soll ein Bus h\"ochstens f\"ur einen anderen als Ersatz dienen. F\"ur die wenigen neuen Busse mit geringer Ausfallwahrscheinlichkeit wird kein Ersatz vorgesehen und sie k\"onnen auch nicht als Ersatz dienen, da sie st\"andig im Einsatz sind. F\"ur andere Busse wird festgelegt, dass sie bei Sch\"aden schnell repariert werden; sie sind dann Ersatz f\"ur sich selbst (Ein-Objekt-Zyklen). Typischerweise wird f\"ur jeden Bus ein anderer als Ersatz festgelegt und jeder Bus kann als Ersatz f\"ur einen anderen Bus dienen (Objekte in einem Mehr-Objekte-Zyklus oder \enquote{innere Objekte} in einer Objekt-Kette). Es ist in seltenen F\"allen aber auch nicht auszuschlie\ss en, dass ein Bus zwar nicht als Ersatz dienen kann, aber einen Ersatz braucht (Anfangsglied einer Kette). Ebenso kann ein Bus in Reserve gehalten werden, also nur als Ersatz
dienen, ohne selbst einen Ersatz zu ben\"otigen (Endglied einer Kette).
          \begin{center}
            \scalebox{.7}{
              \begin{tikzpicture}[node distance=1.5cm, every edge/.style={link}]
                \node[entity](A){Bus};
                  \node[attribute](a1)[left = of A]{\key{PolKennzeichen}} edge (A);
                  \node[attribute](a2)[above left = of A]{Sitzplatzanzahl} edge (A);
                  \node[attribute](a3)[below left = of A]{ErsatzbusKennzeichen} edge (A);
                \node[relationship](rel1)[right = of A]{};
                \node[auto,swap](l1) at (5.6,-1.4) {(0,1) ist Ersatz f\"ur};
                \node[auto,swap](l2) at (5.6,1.4) {(0,1) hat als Ersatz};
                \path [draw, -] (A) |- ($(A.south) + (0.5,-0.5)$) -| (rel1);
                \path [draw, -] (A) |- ($(A.north) +(0.5, 0.5)$) -| (rel1);
              \end{tikzpicture}
            }
          \end{center}
          \begin{small}
            Bus(\pk{PolKennzeichen}, Sitzplatzanzahl, \un{\fk{ErsatzbusKennzeichen}})
          \end{small}
\clearpage
          Da der FK \fk{ErsatzbusKennzeichen} als nicht-eingabepflichtig deklariert ist, kann ein Bus auf keinen oder auf einen anderen Bus (evtl. auf sich selbst) als Ersatzbus verweisen. Da der FK unikal ist, kann auf einen Ersatzbus nur von \textit{einem} anderen Bus verwiesen werden. Andererseits ist nicht gefordert - und kann auch gar nicht gefordert werden -, dass jeder Wert des PK \pk{PolKennzeichen} auch tats\"achlich als FK auftritt. Somit kann es Busse geben, die nicht durch einen FK-Wert als Ersatzbusse ausgewiesen sind.
      \subsection{Der (1,1):(0,*) Rekursiv-Beziehungstyp}
        Beim (1,1):(0,*) Rekursiv-Beziehungstyp kann ein Objekt keine, eine oder mehrere Nachrichten senden, es muss aber stets genau eine Nachricht empfangen werden. Das entspricht den Bedingungen des (1,1):(1,1) Rekursiv-Beziehungstyps, zuz\"uglich der M\"oglichkeit eines Objekts, entweder gar nicht als Sender oder aber als Sender mehrerer Nachrichten aufzutreten.

        Da jedes Objekt des Objekttyps A genau eine Nachricht empfangen muss, m\"ussen auch \#A Nachrichten gesendet werden. F\"ur jedes Nicht-Sender-Objekt, muss somit ein anderes - gewisserma\ss en sein Vertreter - eine zus\"atzliche Nachricht senden.
        \begin{center}
          \scalebox{1}{
            \begin{tikzpicture}[]
              \node[circleA](A){Objekttyp A};
              \node[redspot](a1) at (1.0, 0.5){a1};
              \node[redspot](a2) at (0.0, 1.6){a2};
              \node[redspot](a3) at (-1.0, 0.5){a3};
              \node[redspot](a4) at (-1, -0.6) {a4};
              \node[redspot](a5) at (1, -0.4) {a5};
              \node[redspot](a6) at (0.5, -0.8) {a6};
              \node[redspot](a7) at (-0.2, -0.5) {a7};
              \node[redspot](a8) at (0.5, 1) {a8};
              \node[redspot](a9) at (0, 0.4){a9};

      \path[->] (a1) edge[bend right=45] (a2)
                (a2) edge[bend right=45] (a3)
                (a3) edge[bend right=45] (a1)
                (a4) edge[loop right, out=270, in = 0, looseness=18] (a4)
                (a8) edge[bend right=45] (a9)
                (a1) edge[bend left=45] (a5)

                (a5) edge (a6)
                (a5) edge (a7)
                (a1) edge (a8);
            \end{tikzpicture}
          }
        \end{center}
        Bei der Transformation, muss der Identifikator des Senders eingabepflichtig beim Empf\"anger hinterlegt werden. Wird der FK au\ss erdem als nicht-unikal deklariert, kann ein Objekt von mehreren Empf\"anger-Objekten als \enquote{ihr} Sender ausgewiesen werden. Folgend ist die Transformationsregel T11 dargestellt.
\clearpage
        \subsubsection{Transformationsregel T11 f\"ur den (1,1):(0,*) Rekursiv-Beziehungstyp}
          \tablefirsthead{%
          \hline
            \multicolumn{1}{|c}{\textbf{ER-Modell}} &
            \multicolumn{1}{|l}{\textbf{ }} &
            \multicolumn{1}{|c|}{\textbf{physisches Datenmodell}} \\
          \hline
          }
          \begin{supertabular}[h]{|l|l|p{7.5cm}|}
            \footnotesize Objekttyp A mit Schl\"ussel \pk{SA} & $\Rightarrow $ & \footnotesize Tabelle A mit PK \pk{SA}\\
            \hline
            \footnotesize (1,1):(0,*) Rekursiv-Beziehungstyp & $\Rightarrow $ & \footnotesize \pk{SA} wird in A mit anderen Attributbezeichnungen \enquote{gedoppelt} und als eingabepflichtiger nicht-unikaler FK \fk{SA'} vereinbart, der auf den Sender verweist. (\nn{\fk{SA'}})\\
          \end{supertabular}
          \begin{center}
            \scalebox{.7}{
              \begin{tikzpicture}[node distance=1.5cm, every edge/.style={link}]
                \node[entity](A){A};
                  \node[attribute](a1)[left = of A]{\key{SA}} edge (A);
                \node[relationship](rel1)[right = of A]{};
                \node[auto,swap](l1) at (5.6,-1.4) {(1,1) empf\"angt Nachricht};
                \node[auto,swap](l2) at (5.6,1.4) {(0,*) sendet Nachricht};
                \path [draw, -] (A) |- ($(A.south) + (0.5,-0.5)$) -| (rel1);
                \path [draw, -] (A) |- ($(A.north) +(0.5, 0.5)$) -| (rel1);
              \end{tikzpicture}
            }
          \end{center}
          \bild{}{transformationsregel_11_tabellen}{0.35}

        \subsubsection{Beispiel - Fotos der Vereinsmitglieder}
          Nehmen wir als Beispiel an, dass f\"ur die Festzeitschrift eines Vereins von allen Mitgliedern ein Foto ben\"otigt wird. Im Interesse der Kostend\"ampfung wird kein externer Fotograf hinzugezogen. Jedes Mitglied muss genau einmal fotografiert werden. Manche Mitglieder fotografieren gar nicht, andere m\"ussen daf\"ur umso mehr Fotos machen. Damit auch wirklich jeder Fotograf selbst auf einem Foto zu sehen ist, ist letztlich erforderlich, dass sich ein Fotograf entweder selbst fotografiert (Ein-Objekt-Zyklus) oder dass ihn jemand fotografiert, der selbst bereits fotografiert wurde (Mehr-Objekte-Zyklus). Die erforderliche Transformation ist in folgender Abbildung dargestellt.
          \begin{center}
            \scalebox{.7}{
              \begin{tikzpicture}[node distance=1.5cm, every edge/.style={link}]
                \node[entity](A){Mitglied};
                  \node[attribute](a1)[left = of A]{\key{Mitgliedsnummer}} edge (A);
                  \node[attribute](a2)[above left = of A]{Name} edge (A);
                  \node[attribute](a3)[below left = of A]{FotografenNr} edge (A);
                \node[relationship](rel1)[right = of A]{};
                \node[auto,swap](l1) at (5,-1.4) {(0,*) fotografiert};
                \node[auto,swap](l2) at (5.6,1.4) {(1,1) wird fotografiert von};
                \path [draw, -] (A) |- ($(A.south) + (0.5,-0.5)$) -| (rel1);
                \path [draw, -] (A) |- ($(A.north) +(0.5, 0.5)$) -| (rel1);
              \end{tikzpicture}
            }
          \end{center}
          \begin{small}
            Mitglied(\pk{Mitgliedsnummer}, Name, \nn{\fk{FotografenNr}})
          \end{small}
      \subsection{Der (0,1):(0,*) Rekursiv-Beziehungstyp}
        Der (0,1):(0,*) Rekursiv-Beziehungstyp unterscheidet sich vom (1,1):(0,*) Rekursiv-Be\-zieh\-ungs\-typ nur durch die Aufhebung der Forderung, dass jedes Objekt eine Nachricht empfangen muss. Die folgende Abbildung zeigt daf\"ur ein Beispiel.
        \begin{center}
          \scalebox{1}{
            \begin{tikzpicture}[]
              \node[circleA](A){Objekttyp A};
              \node[redspot](a1) at (1.2, 0.75){a1};
              \node[redspot](a2) at (0.0, 1.6){a2};
              \node[redspot](a3) at (-1.3, 0.5){a3};
              \node[redspot](a4) at (-1, -0.6) {a4};
              \node[redspot](a5) at (1.3, -0.4) {a5};
              \node[redspot](a6) at (0.6, -0.9) {a6};
              \node[redspot](a7) at (-0.2, -0.5) {a7};
              \node[redspot](a8) at (0.5, 1) {a8};
              \node[redspot](a9) at (0, 0.6) {a9};
              \node[redspot](a10) at (-0.7, 1) {a10};
              \node[redspot](a11) at (-0.2, -1.4){a11};


      \path[->] (a1) edge[bend right=45] (a2)
                (a2) edge[bend right=45] (a3)
                (a3) edge[bend right=45] (a1)
                (a4) edge[loop right, out=270, in = 0, looseness=18] (a4)
                (a5) edge (a6)
                (a5) edge (a7)
                (a1) edge (a8)
                (a9) edge (a10);
            \end{tikzpicture}
          }
        \end{center}
        Dieser Rekursiv-Beziehungstyp ist pr\"adestiniert f\"ur die Repr\"asentation von Monohierarchien, wie sie in der Praxis in Form von Organigrammen oder St\"ucklisten auftreten. Als Sonderfall eines \enquote{Baumes} ist eine Liste ($a_{9} \rightarrow a_{10}$) m\"oglich, die evtl. auch nur aus einem Element ($a_{11}$) bestehen kann.

        Bei der Transformation dieses Rekursiv-Beziehungstyps in das physische Datenbankmodell wird der FK als nicht-eingabepflichtig und nicht-unikal deklariert. Die entsprechende Transformation ist in T12 zusammengefasst.
        \subsubsection{Transformationsregel T12 f\"ur den (0,1):(0,*) Rekursiv-Beziehungstyp}
          \tablefirsthead{%
          \hline
            \multicolumn{1}{|c}{\textbf{ER-Modell}} &
            \multicolumn{1}{|l}{\textbf{ }} &
            \multicolumn{1}{|c|}{\textbf{physisches Datenmodell}} \\
          \hline
          }
          \begin{supertabular}[h]{|l|l|p{7.5cm}|}
            \footnotesize Objekttyp A mit Schl\"ussel \pk{SA} & $\Rightarrow $ & \footnotesize Tabelle A mit PK \pk{SA}\\
            \hline
            \footnotesize (0,1):(0,*) Rekursiv-Beziehungstyp & $\Rightarrow $ & \footnotesize \pk{SA} wird in A mit anderen Attributbezeichnungen \enquote{gedoppelt} und als nicht-eingabepflichtiger nicht-unikaler FK \fk{SA} vereinbart, der auf den Sender verweist. (\fk{SA'})\\
          \end{supertabular}
          \begin{center}
            \scalebox{.7}{
              \begin{tikzpicture}[node distance=1.5cm, every edge/.style={link}]
                \node[entity](A){A};
                  \node[attribute](a1)[left = of A]{\key{SA}} edge (A);
                \node[relationship](rel1)[right = of A]{};
                \node[auto,swap](l1) at (5.6,-1.4) {(0,1) empf\"angt Nachricht};
                \node[auto,swap](l2) at (5.6,1.4) {(0,*) sendet Nachricht};
                \path [draw, -] (A) |- ($(A.south) + (0.5,-0.5)$) -| (rel1);
                \path [draw, -] (A) |- ($(A.north) +(0.5, 0.5)$) -| (rel1);
              \end{tikzpicture}
            }
          \end{center}
          \bild{}{transformationsregel_12_tabellen}{0.35}
\clearpage
				\subsubsection{Beispiel - Unternehmenshierarchie}
          Betrachten wir als Beispiel die Leitungshierarchie im Unternehmen. Ein Mitarbeiter kann ohne Leitungsfunktion sein, er kann aber auch mehrere andere Mitarbeiter anleiten. Dies kann \"uber mehrere Hierarchiestufen erfolgen. Andererseits haben die meisten Mitarbeiter genau einen Chef. Mitarbeiter der obersten Leitungsebene haben keinen Chef. Die Transformation ist in der folgenden Abbildung dargestellt.
          \begin{center}
            \scalebox{.7}{
              \begin{tikzpicture}[node distance=1.5cm, every edge/.style={link}]
                \node[entity](A){Mitarbeiter};
                  \node[attribute](a1)[left = of A]{\key{Personalnummer}} edge (A);
                  \node[attribute](a2)[above left = of A]{Name} edge (A);
                  \node[attribute](a2)[below left = of A]{Chef-Personalnummer} edge (A);
                \node[relationship](rel1)[right = of A]{};
                \node[auto,swap](l1) at (5.6,-1.4) {(0,1) wird angeleitet von};
                \node[auto,swap](l2) at (5.6,1.4) {(0,*) leitet an};
                \path [draw, -] (A) |- ($(A.south) + (0.5,-0.5)$) -| (rel1);
                \path [draw, -] (A) |- ($(A.north) +(0.5, 0.5)$) -| (rel1);
              \end{tikzpicture}
            }
          \end{center}
          \begin{small}
            Mitarbeiter(\pk{Personalnummer}, Name, $\Uparrow$Chef-Personalnummer$\Uparrow $)
          \end{small}

          Der FK \fk{Chef-Personalnummer} ist nicht-eingabepflichtig, somit kann es Mitarbeiter ohne Chef geben. Da der FK nicht-unikal ist, k\"onnen mehrere Mitarbeiter auf denselben Chef verweisen. An diesem Beispiel sieht man, dass der (0,1):(0,*) Rekursiv-Beziehungstyp wesentlich mehr Objekt-Strukturen umfasst, als f\"ur die Repr\"asentation monohierarchischer Zusammenh\"ange ben\"otigt werden:

          \begin{itemize}
            \item \textit{Mehr-Objekt-Zyklen:} Der Mitarbeiter mit der Personalnummer \enquote{0815} kann Chef des Mitarbeiters \enquote{0816} sein. Dieser kann den Mitarbeiter \enquote{0817} anleiten, der wiederum Chef des Mitarbeiters \enquote{0815} ist.
            \item \textit{Ein-Objekt-Zyklen:} Ein Mitarbeiter kann als sein eigener Chef ausgewiesen werden, wenn n\"am\-lich in einem Datensatz die Werte von PK und FK identisch sind.
            \item \textit{Mehr-Objekt-Ketten:} Ein Mitarbeiter kann einen einzigen Mitarbeiter anleiten, der wiederum Chef eines einzigen Mitarbeiter ist.
            \item \textit{Ein-Objekt-Ketten:} Ein Mitarbeiter, der keinen Chef hat, leitet selbst auch niemanden an.
          \end{itemize}
          In der Praxis sind solche Situationen gew\"ohnlich verboten, sie k\"onnen aber durch die Tabellentypbeschreibung nicht verhindert werden. Die Anwendungssoftware muss sichern, dass die beschriebenen F\"alle nicht realisiert werden, sofern es nicht gewollt ist.
\clearpage
      \subsection{Der (0,*):(0,*) Rekursiv-Beziehungstyp}
        Der (0,*):(0,*) Rekursiv-Beziehungstyp stellt keinerlei einschr\"ankende Bedingungen an die Struktur, nach der die Objekte eines Objekttyps A miteinander in Beziehung stehen. Jedes Objekt kann keine, eine oder mehrere Nachrichten senden oder empfangen. Die nachstehende Abbildung vermittelt davon einen Eindruck, ohne dass sie alle M\"oglichkeiten ber\"ucksichtigen kann.

        Es lassen sich beliebig strukturierte Netzwerke darstellen. Insbesondere k\"onnen dies Polyhierarchien sein. In einer Polyhierarchie hat jedes Objekt keines, eines oder mehrere untergeordnete Objekte. Jedes Objekt kann keines, eines oder mehrere \"ubergeordnete Objekte besitzen. Als Sonderfall sind nat\"urlich auch Monohierarchien m\"oglich, diese entarten aber mitunter zur Objektkette, die evtl. auch nur aus einem Element bestehen kann. Die Objektkette kann in sich geschlossen sein und wird dann zum Objekt-Zyklus, der im Minimalfall ein Ein-Objekt-Zyklus ist.

        \begin{center}
          \scalebox{1}{
            \begin{tikzpicture}[]
              \node[circleA](A){Objekttyp A};
              \node[redspot](a1) at (1.2, 0.75){a1};
              \node[redspot](a2) at (0.0, 1.6){a2};
              \node[redspot](a3) at (-1.3, 0.5){a3};
              \node[redspot](a4) at (-1, -0.6) {a4};
              \node[redspot](a5) at (1.3, -0.4) {a5};
              \node[redspot](a6) at (0.6, -0.9) {a6};
              \node[redspot](a7) at (-0.2, -0.5) {a7};
              \node[redspot](a8) at (0.5, 1) {a8};
              \node[redspot](a9) at (0, 0.6) {a9};
              \node[redspot](a10) at (-0.7, 1) {a10};
              \node[redspot](a11) at (-0.2, -1.4){a11};
              \node[redspot](a12) at (0.4, -0.2){a12};

      \path[->] (a1) edge[bend right=45] (a2)
                (a2) edge[bend right=45] (a3)
                (a3) edge[bend right=45] (a1)
                (a5) edge[bend right=45] (a1)
                (a4) edge[loop right, out=270, in = 0, looseness=18] (a4)
                (a5) edge (a6)
                (a6) edge (a7)
                (a1) edge (a8)
                (a9) edge (a10)
                (a6) edge (a11);
            \end{tikzpicture}
          }
        \end{center}

        Der (0,*):(0,*) Rekursiv-Beziehungstyp l\"asst sich in das physische Datenbankmodell nur nach vorheriger Umwandlung in einen neuen Objekttyp \"uberf\"uhren. Diese Umwandlung erfolgt mit Hilfe eines Koppel-Objekttyps, der als \enquote{Sendung} bezeichnet werden soll. Er repr\"asentiert die Sender-Empf\"anger-Beziehung zwischen den Objekten des Objekttyps A. Der Objekttyp A ist mit dem Objekttyp \enquote{Sendung} durch zwei (1,1):(0,*) Beziehungen zu verbinden.
        \begin{center}
          \scalebox{.7}{
            \begin{tikzpicture}[node distance=1.5cm, every edge/.style={link}]
              \node[entity](A){A};
                \node[attribute](a1)[left = of A]{\key{SA}} edge (A);
              \node[relationship](rel1)[right = of A]{};
              \node[auto,swap](l1) at (5.6,-1.4) {(0,*) empf\"angt Nachricht};
              \node[auto,swap](l2) at (5.6,1.4) {(0,*) sendet Nachricht};
              \path [draw, -] (A) |- ($(A.south) + (0.5,-0.5)$) -| (rel1);
              \path [draw, -] (A) |- ($(A.north) +(0.5, 0.5)$) -| (rel1);
            \end{tikzpicture}
          }
        \end{center}
        wird umgewandelt in:
        \begin{center}
          \scalebox{.7}{
            \begin{tikzpicture}[node distance=1.5cm, every edge/.style={link}]
              \node[entity](A){A};
                \node[attribute](a1)[left = of A]{\key{SA}} edge (A);
              \node[relationship](rel1) at (4, 0){1};
              \node[entity](B)[right = of rel1]{Sendung};
              \node[relationship](rel2) at (4, -2){2};
              \node[auto,swap](l1) at (1, -1.5){(0,*)};
              \node[auto,swap](l2) at (6, -1.5){(1,1)};
              \path (A) edge node[auto,swap]{(0,*)} (rel1)
                (rel1) edge node[auto,swap]{(1,1)} (B);
              \path[draw, -] (A.south) |- (rel2.west)
                (rel2.east) -| (B.south);
            \end{tikzpicture}
          }
        \end{center}
        Stellt man den Objekttyp \enquote{Sendung} im physischen Datenbankmodell als Koppel-Tabelle A/A dar, so enth\"alt er den PK des Objekttyps A zweimal, als eingabepflichtige FK, die auf den Empf\"anger bzw. auf den Sender einer Nachricht verweisen. Beide FK sind f\"ur sich genommen nicht-unikal, so dass ein Objekt in mehreren Tabellenzeilen als Empf\"anger bzw. als Sender auftreten kann. Die Kombination dieser beiden FK stellt den PK der Koppel-Tabelle dar. Dieses Vorgehen ist in der Transformationsregel T13 dargestellt.
        \subsubsection{Transformationsregel T13 f\"ur den (0,*):(0,*) Rekursiv-Beziehungstyp}
          \tablefirsthead{%
          \hline
            \multicolumn{1}{|c}{\textbf{ER-Modell}} &
            \multicolumn{1}{|l}{\textbf{ }} &
            \multicolumn{1}{|c|}{\textbf{physisches Datenmodell}} \\
          \hline
          }
          \begin{supertabular}[h]{|l|l|p{7.5cm}|}
            \footnotesize Objekttyp A mit Schl\"ussel \pk{SA} & $\Rightarrow $ & \footnotesize Tabelle A mit PK \pk{SA}\\
            \hline
            \footnotesize (0,*):(0,*) Rekursiv-Beziehungstyp & $\Rightarrow $ & \footnotesize Koppel-Tabelle A/A, die \pk{SA} in zwei Exemplaren, als eingabepflichtige, nicht-unikale Fremdschl\"ussel enth\"alt: als Empf\"anger-Fremdschl\"ussel \fk{SA} und als Sender-Fremdschl\"ussel \fk{SA'}. Die Kombination beider Fremdschl\"ussel bildet den Prim\"arschl\"ussel von A/A.\\
          \end{supertabular}

          \bild{}{transformationsregel_13_tabellen}{0.35}

        \subsubsection{Beispiel - komplexes Bauteil}
          Zur Veranschaulichung der Transformationsregel T13 betrachten wir eine St\"uckliste, die den Aufbau eines komplexen Bauteils - z.B. eines Autos - aus kleineren Bauteilen beschreibt. Ein Bauteil kann elementar sein, kann sich also nicht mehr in kleinere Bauteile zerlegen lassen (z.B. Schraube). Andere Bauteile lassen sich, wie z.B. der Motor, in kleinere Einzelteile zerlegen. Des Weiteren ist es m\"oglich, dass ein gegebenes Bauteil nicht Bestandteil eines gr\"o\ss eren Bauteils ist, wenn es n\"amlich beispielsweise das komplette Auto darstellt. Es kann aber auch Bestandteil mehrerer gr\"o\ss erer Bauteile sein, wenn ein bestimmter Motor in mehrere Modelle eingebaut werden kann. Die erforderliche Transformation zeigt die folgende Abbildung.
          \begin{center}
            \scalebox{.7}{
              \begin{tikzpicture}[node distance=1.5cm, every edge/.style={link}]
                \node[entity](A){Bauteil};
                  \node[attribute](a1)[left = of A]{\key{Bauteilnummer}} edge (A);
                  \node[attribute](a1)[above left = of A]{Bezeichnung} edge (A);
                \node[relationship](rel1)[right = of A]{};
                \node[auto,swap](l1) at (5.6,-1.4) {(0,*) ist zerlegbar in};
                \node[auto,swap](l2) at (5.6,1.4) {(0,*) ist Bestandteil von};
                \path [draw, -] (A) |- ($(A.south) + (0.5,-0.5)$) -| (rel1);
                \path [draw, -] (A) |- ($(A.north) +(0.5, 0.5)$) -| (rel1);
              \end{tikzpicture}
            }
          \end{center}
          \begin{small}
            Bauteil(\pk{Bauteilenummer}, Bezeichnung)

            BauteilBauteil(\pk{\fk{Grossteilnummer + Kleinteilnummer}})
          \end{small}

          Jeder der beiden FK ist f\"ur sich nicht-unikal, so dass mehrere Bauteile jeweils als Gro\ss teil bzw. Kleinteil in der St\"uckliste auftreten k\"onnen. Es kann sein, dass eine bestimmte Bauteilnummer nicht als Wert des FK \fk{Grossteilnummer} auftritt. Dann ist dieses Bauteil nicht weiter zerlegbar. Ist die Bauteilnummer nie Wert des FK \fk{Kleinteilnummer}, dann geh\"ort dies zu keinem gr\"o\ss eren Bauteil.

          Die Tabellentypbeschreibungen lassen allerdings wiederum einige Situationen zu, welche man im angegebenen Praxisfall eigentlich ausschlie\ss en m\"ochte. Dazu geh\"oren beispielsweise:
          \begin{itemize}
            \item \textit{Mehr-Objekte-Zyklen:} Das Bauteil \enquote{1001} hat als eines seiner Bestandteile das Bauteil \enquote{1002}, f\"ur das wiederum als eines seiner Bestandteile das Bauteil \enquote{1003} angegeben ist. F\"ur das Bauteil \enquote{1003} ist aber angegeben, dass es das Bauteil \enquote{1001} als Bestandteil enth\"alt. Die Koppel-Tabelle \enquote{BauteilBauteil} enth\"alt dann folgende Zeilen.

            \tablefirsthead{%
            \hline
              \multicolumn{1}{|l}{\textbf{Grossteilnummer}} &
              \multicolumn{1}{|l|}{\textbf{Kleinteilnummer}} \\
            \hline
            }
            \begin{supertabular}[h]{|l|p{5cm}|}
              \hline
              .... & ....\\
              \hline
              1001 & 1002\\
              \hline
              1002 & 1003\\
              \hline
              1003 & 1001\\
              \hline
              ... & ...\\
              \hline
            \end{supertabular}
            \item \textit{Ein-Objekt-Zyklen:} Ein Bauteil kann als sein eigenes Bestandteil ausgewiesen werden, wenn seine Bauteilnummer in einer Zeile der Koppel-Tabelle als Wert beider FK angegeben wird.
            \item \textit{Mehr-Objekte-Ketten:} Ein Bauteil kann als Bestandteil nur ein einziges anderes Bauteil haben. Seine Nummer kommt dann in der Koppel-Tabelle \enquote{BauteilBauteil} nur einmal als Wert des FK \fk{Grossteilnummer} vor. Das ist eine unsinnige Konstruktion, denn ein Bauteil l\"asst sich entweder nicht zerlegen oder es ist mindestens in zwei Bestandteile zerlegbar.
            \item \textit{Ein-Objekt-Ketten:} Die Typbeschreibungen lassen die Speicherung eines Bauteils zu, das weder zerlegbar, noch Bestandteil eines gr\"o\ss eren Bauteils ist. Das w\"are beispielsweise eine Schraube, die in keinem weiteren gr\"o\ss eren Bauteil Verwendung findet. Ihre Bauteilnummer taucht dann in der Koppel-Tabelle \enquote{BauteilBauteil} an keiner Stelle auf, die Speicherung solcher Bauteile ist im betrachteten Zusammenhang jedoch sinnlos.
          \end{itemize}
          Die beschriebenen Situationen kann das Datenbankmanagementsystem nicht verhindern. Sie m\"ussen durch die entsprechende Anwendungsprogrammierung unterbunden werden.
      \subsection{Der (1,*):(0,*) Rekursiv-Beziehungstyp}
        Beim (1,*):(0,*) Rekursiv-Beziehungstyp kann ein Objekt keine, eine oder mehrere Nachrichten senden, es muss aber stets mindestens eine Nachricht empfangen. Die folgende Abbildung vermittelt einen Eindruck von m\"oglichen Beziehungsstrukturen, ohne alle Varianten abzudecken.

        \begin{center}
          \scalebox{1}{
            \begin{tikzpicture}[]
              \node[circleA](A){Objekttyp A};
              \node[redspot](a1) at (1.2, 0.75){a1};
              \node[redspot](a2) at (0.0, 1.6){a2};
              \node[redspot](a3) at (-1.3, 0.5){a3};
              \node[redspot](a5) at (1.3, -0.4) {a5};
              \node[redspot](a6) at (0.6, -0.9) {a6};
              \node[redspot](a7) at (-0.2, -0.5) {a7};
              \node[redspot](a8) at (0.5, 1) {a8};
              \node[redspot](a9) at (0, 0.6) {a9};
              \node[redspot](a10) at (-0.7, 1) {a10};
              \node[redspot](a11) at (-0.3, -1.3) {a11};
              \node[redspot](a12) at (-1.0, -0.3) {a12};

      \path[->] (a1) edge[bend right=45] (a2)
                (a3) edge[bend right=45] (a1)
                (a5) edge (a6)
                (a5) edge[bend right=45] (a1)
                (a6) edge (a7)
                (a1) edge (a8)
                (a1) edge (a5)
                (a8) edge (a9)
                (a10) edge (a3)
                (a9) edge (a10)
                (a6) edge (a11)
                (a12) edge[loop right, out=270, in = 0, looseness=15] (a12);
            \end{tikzpicture}
          }
        \end{center}

        Der (1,*):(0,*) Rekursiv-Beziehungstyp l\"asst sich erst nach Umwandlung, mit Hilfe eines Koppel-Objekttyps \enquote{Sendung}, in das physische Modell \"uberf\"uhren. Der Objekttyp A ist mit dem Objekttyp \enquote{Sendung} durch einen (1,1):(0,*) und (1,1):(1,*) verbunden. Die folgende Abbildung zeigt das Schema der Umwandlung.
        \begin{center}
            \scalebox{.7}{
              \begin{tikzpicture}[node distance=1.5cm, every edge/.style={link}]
                \node[entity](A){A};
                  \node[attribute](a1)[left = of A]{\key{SA}} edge (A);
                \node[relationship](rel1)[right = of A]{};
                \node[auto,swap](l1) at (5.6,-1.4) {(1,*) empf\"angt Nachricht};
                \node[auto,swap](l2) at (5.6,1.4) {(0,*) sendet Nachricht};
                \path [draw, -] (A) |- ($(A.south) + (0.5,-0.5)$) -| (rel1);
                \path [draw, -] (A) |- ($(A.north) +(0.5, 0.5)$) -| (rel1);
              \end{tikzpicture}
            }
          \end{center}
          wird umgewandelt in:
          \begin{center}
            \scalebox{.7}{
              \begin{tikzpicture}[node distance=1.5cm, every edge/.style={link}]
                \node[entity](A){A};
                  \node[attribute](a1)[left = of A]{\key{SA}} edge (A);
                \node[relationship](rel1) at (4, 0){1};
                \node[entity](B)[right = of rel1]{Sendung};
                \node[relationship](rel2) at (4, -2){2};
                \node[auto,swap](l1) at (1, -1.5){(1,*)};
                \node[auto,swap](l2) at (6, -1.5){(1,1)};
                \path (A) edge node[auto,swap]{(0,*)} (rel1)
                  (rel1) edge node[auto,swap]{(1,1)} (B);
                \path[draw, -] (A.south) |- (rel2.west)
                  (rel2.east) -| (B.south);
              \end{tikzpicture}
            }
          \end{center}
\clearpage
        Nun haben wir bei den bin\"aren Beziehungstypen gesehen, dass sich ein (1,1):(1,*) Beziehungstyp im physischen Datenbankmodell nur als (1,1):(0,*) Beziehungstyp repr\"asentieren l\"asst. Damit kann der (1,*):(0,*) Rekursiv-Beziehungstyp nur gem\"a\ss{} der Transformationsregel T13 wie ein (0,*):(0,*) Rekursiv-Beziehungstyp dargestellt werden.
        \subsubsection{Beispiel - Literaturverweis}
          Betrachten wir als Beispiel Literaturverweise in Lehrb\"uchern. Jedes Lehrbuch verweist auf mindestens ein Vorg\"anger-Lehrbuch. Auf ein gerade erst erschienenes Lehrbuch kann noch nicht verwiesen werden. Handelt es sich um ein interessantes Lehrbuch, so wird im Laufe der Zeit von vielen Nachfolge-Lehrb\"uchern zitiert.
          \begin{center}
            \scalebox{.7}{
              \begin{tikzpicture}[node distance=1.5cm, every edge/.style={link}]
                \node[entity](A){Lehrbuch};
                  \node[attribute](a1)[left = of A]{\key{ISBN}} edge (A);
                  \node[attribute](a2)[above left = of A]{Titel} edge (A);
                \node[relationship](rel1)[right = of A]{};
                \node[auto,swap](l1) at (4,-1.4) {(0,*) wird zitiert von};
                \node[auto,swap](l2) at (4,1.4) {(1,*) verweist auf};
                \path [draw, -] (A) |- ($(A.south) + (0.5,-0.5)$) -| (rel1);
                \path [draw, -] (A) |- ($(A.north) +(0.5, 0.5)$) -| (rel1);
              \end{tikzpicture}
            }
          \end{center}
        \begin{small}
          Lehrbuch(\pk{ISBN}, Titel)

          LehrbuchLehrbuch(\pk{\fk{VorgaengerISBN + NachfolgerISBN}})
        \end{small}

          Jeder der beiden FK ist f\"ur sich nicht-unikal, so dass ein gegebenes Lehrbuch mehrmals als Vorg\"anger bzw. als Nachfolger in Erscheinung treten kann. Ist eine Lehrbuch ISBN kein einziges Mal Wert des FK \fk{VorgaengerISBN}, dann wurde dieses Lehrbuch noch nicht zitiert.

          Die Tabellentypbeschreibungen lassen folgende Situationen zu, die nicht der Semantik entsprechen.
          \begin{itemize}
            \item \textit{Vernachl\"assigung der Nichtoptionalit\"at:} Die nicht-optionale Beziehungstyprichtung\\ \enquote{Lehrbuch verweist auf Lehrbuch} wird als optionale Beziehungtyprichtung repr\"asentiert. Es ist somit m\"oglich, dass eine Lehrbuch ISBN nie als Wert des Fremdschl\"ussels \fk{NachfolgerISBN} auftaucht. Dann ist dieses Lehrbuch nicht als Nachfolger eines anderen Lehrbuchs ausgewiesen, d.h. es verweist - entgegen der Praxisregel - auf kein Vorg\"angerlehrbuch.
            \item \textit{Zyklen:} Ein Lehrbuch kann auf sich selbst als Vorg\"anger verweisen oder eine Objektekette kann sich schlie\ss en. Da Vorg\"anger-Verweise immer \enquote{in die Vergangenheit} zeigen, sind Zyklen eigentlich verboten.
          \end{itemize}
          Die beschriebenen \enquote{pathologischen} Situationen lassen sich nur durch die Anwendungsprogrammierung vermeiden.
      \subsection{Der (1,*):(1,*) Rekursiv-Beziehungstyp}
        Der (1,*):(1,*) Rekursiv-Beziehungstyp unterscheidet sich nur in einem Punkt, vom zuvor behandelten Rekursiv-Beziehungstyp (1,*):(0,*). Ein Objekt muss mindestens eine Nach\-richt senden. Diese Forderung hat jedoch entscheidende Konsequenzen f\"ur die m\"og\-lichen Beziehungsstrukturen, von denen in der folgenden Abbildung einige abgebildet sind.

        \begin{center}
          \scalebox{1}{
            \begin{tikzpicture}[]
              \node[circleA](A){Objekttyp A};
              \node[redspot](a1) at (1.2, 0.75){a1};
              \node[redspot](a2) at (0.0, 1.6){a2};
              \node[redspot](a3) at (-1.3, 0.5){a3};
              \node[redspot](a5) at (1.3, -0.4) {a5};
              \node[redspot](a6) at (0.6, -0.9) {a6};
              \node[redspot](a7) at (-0.2, -0.5) {a7};
              \node[redspot](a8) at (0.5, 1) {a8};
              \node[redspot](a9) at (0, 0.6) {a9};
              \node[redspot](a10) at (-0.7, 1) {a10};
              \node[redspot](a11) at (-0.3, -1.3) {a11};
              \node[redspot](a12) at (-1.0, -0.3) {a12};

      \path[->] (a1) edge[bend right=45] (a2)
                (a2) edge[bend right=45] (a3)
                (a3) edge[bend right=45] (a1)
                (a5) edge (a6)
                (a6) edge (a7)
                (a7) edge (a3)
                (a1) edge (a8)
                (a1) edge (a5)
                (a8) edge (a9)
                (a10) edge (a3)
                (a9) edge (a10)
                (a6) edge (a11)
                (a11) edge (a7)
                (a12) edge[loop right, out=270, in = 0, looseness=15] (a12);
            \end{tikzpicture}
          }
        \end{center}

        Auch der (1,*):(1,*) Rekursiv-Beziehungstyp l\"asst sich in das physische Modell, nur nach vorheriger Umwandlung in einen neuen Objekttyp, \"uberf\"uhren. Der Objekttyp A ist mit dem Koppel-Objekttyp \enquote{Sendung} durch zwei (1,1):(1,*) Beziehungstypen verbunden.
        \begin{center}
          \scalebox{.7}{
            \begin{tikzpicture}[node distance=1.5cm, every edge/.style={link}]
              \node[entity](A){A};
                \node[attribute](a1)[left = of A]{\key{SA}} edge (A);
              \node[relationship](rel1)[right = of A]{};
              \node[auto,swap](l1) at (5.6,-1.4) {(1,*) empf\"angt Nachricht};
              \node[auto,swap](l2) at (5.6,1.4) {(1,*) sendet Nachricht};
              \path [draw, -] (A) |- ($(A.south) + (0.5,-0.5)$) -| (rel1);
              \path [draw, -] (A) |- ($(A.north) +(0.5, 0.5)$) -| (rel1);
            \end{tikzpicture}
          }
        \end{center}
        wird umgewandelt in:
        \begin{center}
          \scalebox{.7}{
            \begin{tikzpicture}[node distance=1.5cm, every edge/.style={link}]
              \node[entity](A){A};
                \node[attribute](a1)[left = of A]{\key{SA}} edge (A);
              \node[relationship](rel1) at (4, 0){1};
              \node[entity](B)[right = of rel1]{Sendung};
              \node[relationship](rel2) at (4, -2){2};
              \node[auto,swap](l1) at (1, -1.5){(1,*)};
              \node[auto,swap](l2) at (6, -1.5){(1,1)};
              \path (A) edge node[auto,swap]{(1,*)} (rel1)
                (rel1) edge node[auto,swap]{(1,1)} (B);
              \path[draw, -] (A.south) |- (rel2.west)
                (rel2.east) -| (B.south);
            \end{tikzpicture}
          }
        \end{center}
        Da sich ein (1,1):(1,*) im physischen Modell nur als (1,1):(0,*) Beziehungstyp repr\"asentieren l\"asst, muss der (1,*):(1,*) Rekursiv-Beziehungstyp - unter Verlust von semantischen Informationen - gem\"a\ss{} der Transformationsregel T13 wie ein (0,*):(0,*) Rekursiv-Beziehungstyp dargestellt werden.
\clearpage
        \subsubsection{Beispiel - Pianisten}
          Als Beispiel betrachten wir eine Kunsthochschule, die Informationen \"uber Pianisten sammelt, die in der Ausbildung t\"atig sind. Wir nehmen der Einfachheit halber an, dass die Pianisten eindeutig durch ihren Namen unterschieden werden k\"onnen. Jeder dieser Pianisten hat wenigstens einen anderen Pianisten als Sch\"uler. Andererseits ist jeder Pianist Sch\"uler eines oder mehrerer anderer Pianisten (Autodidakten werden nicht ber\"ucksichtigt).
          \begin{center}
            \scalebox{.7}{
              \begin{tikzpicture}[node distance=1.5cm, every edge/.style={link}]
                \node[entity](A){Pianist};
                  \node[attribute](a1)[left = of A]{\key{Name}} edge (A);
                  \node[attribute](a2)[above left = of A]{Geburtsdatum} edge (A);
                \node[relationship](rel1)[right = of A]{};
                \node[auto,swap](l1) at (4,-1.4) {(1,*) hat Sch\"uler};
                \node[auto,swap](l2) at (4,1.4) {(1,*) ist Sch\"uler von};
                \path [draw, -] (A) |- ($(A.south) + (0.5,-0.5)$) -| (rel1);
                \path [draw, -] (A) |- ($(A.north) +(0.5, 0.5)$) -| (rel1);
              \end{tikzpicture}
            }
          \end{center}
          \begin{small}
            Pianist(\pk{Name}, Geburtsdatum)

            PianistPianist(\pk{\fk{LehrerName + SchuelerName}})
          \end{small}

          Jeder der beiden FK ist f\"ur sich nicht-unikal, so dass ein Pianist mehrmals als Lehrer bzw. Sch\"uler in Erscheinung treten kann.

          Die Tabellentypbeschreibungen erm\"oglichen Beziehungsstrukturen, die in der Praxis unzul\"assig sind, so beispielsweise:
          \begin{itemize}
            \item \textit{Vernachl\"assigung der Nichtoptionalit\"at:} Die beiden nicht-optionalen Beziehungstyprichtungen \enquote{Pianist hat als Sch\"uler Pianist} und \enquote{Pianist ist Sch\"uler von Pianist} werden als optionale Beziehungstyp-Richtungen repr\"asentiert. Es ist somit m\"oglich, dass ein Pianist keine Sch\"uler hat und dass ein Pianist nicht Sch\"uler eines anderen Pianisten ist.
            \item \textit{Zyklen:} Ein Pianist kann sein eigener Sch\"uler sein (Ein-Objekt-Zyklus) oder eine Sch\"u\-ler-Kette kann zu ihrem Ausgangspunkt zur\"uckkehren (Mehr-Objekte-Zyklus).
          \end{itemize}

          Diese unzul\"assigen Strukturen sind durch die Anwendungsprogrammierung zu verhindern.
    \section{Transformation von Kardinalit\"ats-Beschr\"ankungen}
      Bei den Beziehungstypen gibt es im ER-Modell die M\"oglichkeit einschr\"ankende Bedingungen f\"ur die Kardinalit\"at anzugeben, wenn diese durch die \enquote{Gesch\"aftsregeln} der Realit\"at vorgegeben sind. Nehmen wir z.B. an, dass ein Unternehmen Informationen \"uber die Dienstwagen und \"uber die (Sommer- und Winter-) R\"ader speichern will. Jedes Auto ist mit genau 5 R\"adern ausger\"ustet (inklusive Ersatzrad). Einige R\"ader werden als Reserve im Lager aufbewahrt. Dies entspricht einem (0,1):(1,*) Beziehungstyp, f\"ur den (1,*) nicht beliebige Werte annehmen kann, sondern nur den Wert 5.

      Wie wird diese Beschr\"ankung in der Tabellentypbeschreibung ber\"ucksichtigt? Leider gar nicht! Es gibt im physischen Modell keine M\"oglichkeit die Kardinalit\"at einzugrenzen. Das gilt f\"ur den bin\"aren wie auch f\"ur den rekursiven oder tern\"aren Beziehungstyp. F\"ur die Repr\"asentation von Kardinalit\"atsbeschr\"ankungen gilt also die negative Transformationsregel T14, welche im Folgenden dargestellt ist.
      \begin{center}
        \scalebox{.7}{
          \begin{tikzpicture}[node distance=1.5cm, every edge/.style={link}]
            \node[entity](A){Auto};
              \node[attribute](a1)[left = of A]{\key{PolKennzeichen}} edge (A);
              \node[attribute](a2)[above left = of A]{Marke} edge (A);
            \node[relationship](rel1)[right = of A]{zugeordnet} edge node[auto,swap] {(5,5)} (A);
            \node[entity](B)[right = of rel1]{Rad} edge node[auto,swap] {(0,1)} (rel1);
              \node[attribute](b1)[right = of B]{\key{Inventarnummer}} edge (B);
              \node[attribute](b2)[below right = of B]{Typ} edge (B);
          \end{tikzpicture}
        }
      \end{center}
        \subsubsection{Transformationsregel T14 Kardinalit\"ats-Beschr\"ankung}
          \tablefirsthead{%
          \hline
            \multicolumn{1}{|c}{\textbf{ER-Modell}} &
            \multicolumn{1}{|l}{\textbf{ }} &
            \multicolumn{1}{|c|}{\textbf{physisches Datenmodell}} \\
          \hline
          }
          \begin{supertabular}[h]{|p{7cm}|l|p{7cm}|}
            \footnotesize Kardinalit\"ats-Beschr\"ankung eines Be\-zieh\-ungs\-typs & $\Rightarrow $ & \footnotesize Eine Beschr\"ankung l\"asst sich nicht in der Typ\-be\-schrei\-bung repr\"asentieren.\\
            \hline
          \end{supertabular}

          Auch wenn sich die Kardinalit\"ats-Beschr\"ankungen nicht in das physische Modell transformieren lassen, sollte ihre Notierung im ER-Modell dennoch erfolgen, weil sie wichtige Gesch\"aftsregeln darstellen. Diese m\"ussen dann bei der Anwendungsprogrammierung ber\"uck\-sichtigt werden.
    \section{Transformation der Spezialisierung / Generalisierung}
      Die Transformation der Spezialisierung kann nicht in eigenst\"andige Regeln gefasst werden, da der Einzelfall zu betrachten ist. Es sollen hier nur allgemeine Hinweise f\"ur die \"Uberf\"uhrung in das physische Modell gegeben werden, welche endg\"ultig in Kombination mit den Transformationsregeln der bin\"aren Beziehungstypen erfolgt.
      \begin{center}
        \scalebox{.8}{
          \begin{tikzpicture}
            \node[entity](obj1){Person};
            \node[isa](isa) [below = of obj1]{ISA} edge node[auto] {(0,1) (1,1) disjunkt}(obj1);
            \node[entity](obj2)[below left = of isa]{Student};
            \node[entity](obj3)[below right = of isa]{WiMa};
            \path[draw, -] (isa.west) -| (obj2.north);
            \path[draw, -] (isa.east) -| (obj3.north);
          \end{tikzpicture}
        }
      \end{center}
      \begin{center}
        \scalebox{.8}{
          \begin{tikzpicture}[node distance=1.5cm, every edge/.style={link}]
            \node[entity](A){Person};
            \node[relationship](rel1)[below left = of A]{};
            \node[relationship](rel2)[below right = of A]{};
            \node[entity](B)[below = of rel1]{Student} edge node[auto,swap]{(1,1)} (rel1);
            \node[entity](C)[below = of rel2]{WiMa} edge node[auto,swap]{(1,1)} (rel2);
            \node[auto,swap](l1) at (3.1,-1.0) {(0,1)};
            \node[auto,swap](l2) at (-1.9,-1.0) {(0,1)};
            \path [draw, -] (A.west) -| (rel1.north);
            \path [draw, -] (A.east) -| (rel2.north);
          \end{tikzpicture}
        }
      \end{center}
      Die Transformation erfolgt mit T03 bzw. T04 und zus\"atzlichen programmtechnischen Ma\ss nahmen, um die Disjunktion zu gew\"ahrleisten.
