  \chapter{Entwicklung eines Datenmodells}
    \setcounter{page}{1}\kapitelnummer{chapter}
    \minitoc
\newpage

    In diesem Abschnitt erfolgt die Vermittlung der Kenntnisse dar\"uber, wie man aus einer gro\ss en Menge von Informationen und Anforderungen an eine neue Datenbank eine Datenstruktur entwirft. Im Zuge der Datenmodellierung soll ein exaktes und vollst\"andiges Modell des betrachteten Realit\"atsausschnitts erarbeitet werden, welches den Rahmen f\"ur die Entwicklung neuer oder die Erweiterung bestehender Anwendungssysteme bildet.

    Als Werkzeug f\"ur die Erstellung konzeptioneller Datenmodelle wird die Entity-Relation\-ship Modellierung (ER-Modellierung), zuerst in ihrer einfachsten Art und sp\"ater in einer erweiterten Fassung, verwendet. Es werden dabei alle vier Phasen des Datenmodellierungsprozesses anhand eines durchg\"angigen Beispiels beschrieben.

    Die ER-Modellierung stellt eine spezielle \enquote{Information-Engineering-Technik} dar, die zur Erstellung von Datenmodellen hoher Qualit\"at benutzt wird. Entworfen in den 70er Jahren von Peter Pin-Shan Chen wurde sie seither vielfach erweitert und verbessert.

    Ziel eines konzeptionellen Datenmodells ist es, die typm\"a\ss{}ige Struktur der Daten in der Datenbank ohne deren Inhalt zu beschreiben. Als Beispiel aus der realen Welt w\"are  die Ausstattung eines B\"uros mit M\"obeln zu nennen. Wer sp\"ater das B\"uro nutzt und mit welchem Inhalt die Schr\"anke gef\"ullt werden, ist f\"ur die Erstellung des Modells Bedeutungslos.

    Die Entwicklung eines Datenmodells teilt sich in die folgenden vier Phasen ein:
    \begin{enumerate}
      \item Klassifizierung von Objekten
      \item Festlegung relevanter Eigenschaften
      \item Bestimmung identifizierender Eigenschaften
      \item Beschreibung sachlogischer Zusammenh\"ange zwischen den einzelnen Objekten
    \end{enumerate}
    \section{Die Modellierungsinformationen}
			Die Struktur der Bundeswehr soll in einem ER-Modell dargestellt werden.
			Es wird jedoch nur ein Ausschnitt aus der Realit\"at betrachtet, um die
			\"Ubersichtlichkeit der Datenstrukturen zu gew\"ahrleisten. Im Folgenden
			werden die daf\"ur notwendigen Objekte festgelegt.
        \begin{enumerate}
          \item Die Bundeswehr besitzt zahlreiche Dienststellen an den unterschiedlichsten Standorten, welche sich untereinander \"uber- oder untergeordnet sind. Jede Dienststelle besteht aus Dienstposten. Diese wiederum werden mit Soldaten besetzt. Jeder Soldat empf\"angt bei Einstieg in die Bundeswehr seine pers\"onliche Ausr\"ustung und besitzt diese dann f\"ur die Dauer seines Dienstverh\"altnisses.
          \item Jeder Dienststelle der Bundeswehr ist eine Dienststellennummer zugeordnet, \"uber die diese identifiziert werden kann. Des Weiteren hat jede Dienststelle eine bestimmte Gr\"o\ss e sowie Bezeichnung, wie z. B. F\"uhrungsunterst\"utzungsschule der Bundeswehr.
          \item F\"ur die Standorte, an denen sich die Dienststellen befinden, m\"ussen in der Datenbank Postleitzahl (PLZ), Ort, Stra\ss e und die Hausnummer hinterlegt sein. Hierbei ist zu beachten, dass sich eine Dienststelle auch an mehreren Standorten befinden kann.
          \item Die den Dienststellen untergeordneten Dienstposten werden durch ihre Dienstposten\_ID und ein Beginn- und Enddatum charakterisiert. Als eine weitere Eigenschaft soll eine kurze Dienstpostenbeschreibung hinterlegt werden.
          \item Jeder in der Datenbank aufgef\"uhrte Soldat soll dort mit seiner Personanlnummer, einer Personenkennziffer, sowie Vor- und Nachnamen und Dienstgrad aufgef\"uhrt werden. Weiterhin kann der Anwender auch die aktuelle Adresse, bestehend aus PLZ, Wohnort, Stra\ss e und Hausnummer, aus der Datenbank heraussuchen.
          \item Das letzte Objekt in der Datenbank stellt die pers\"onliche Ausr\"ustung des Soldaten dar. Diese besitzt eine Bezeichnung, ist aus einem bestimmten Material gefertigt und hat eine Farbe. F\"ur eine bessere Zuordnung ist jeder Aur\"ustungsgegenstand mit einer Versorgungsnummer versehen.
        \end{enumerate}
    \section{Klassifizierung von Objekten}
      Das im vorigen Abschnitt vorgestellte Beispiel stellt nur einen kleinen Ausschnitt aus der Realit\"at dar. Komplexer gestaltete Datenbanken k\"onnen aus weit mehr Objekten bestehen.
      Um dabei nicht den \"Uberblick \"uber diese Flut von Objekten zu verlieren, werden diese in Klassen gruppiert. Diese Objektklassen enthalten dann die Objekte, die von ihrer Art her gleich sind und \"uber die die gleichen Informationen gesammelt werden. Damit wird eine Abstraktionsebene gebildet, die es erm\"oglicht, von den Besonderheiten der einzelnen Objekte abzusehen und nur das typische der gebildeten Objektklassen zu ber\"ucksichtigen.
      \subsection{Definitionen und Syntaxregeln}
				F\"ur die Darstellung der Objekttypen gelten folgende syntaktische
				Regeln:
				\begin {enumerate}
					\item Ein Objekttyp wird durch ein Rechteck dargestellt, in dessen
					Mitte der Objekttypname eingetragen wird.
					\begin{center}
						\scalebox{1}{
							\begin{tikzpicture}[node distance=1.5cm, every edge/.style={link}]
								\node[entity](e1){Soldat};
							\end{tikzpicture}
						}
					\end{center}
					\item Die Gr\"o\ss e und Position des Rechtecks sind bedeutungslos.
					\item Der Objekttypname steht im Singular und muss f\"ur
					das gesamte Datenmodell eindeutig sein.
        \end{enumerate}
        \begin{merke}
          \begin{itemize}
            \item \textbf{Objekt}: Ein Objekt (engl. Entity) ist ein Exemplar von Personen, Gegenst\"anden oder nichtmateriellen Dingen, \"uber das Informationen gespeichert wird, z. B. der konkrete Soldat Max Mustermann.
            \item \textbf{Objekttyp}: Ein Objekttyp (engl. Entity type) ist eine durch einen Objekttyp-namen eindeutig benannte Klasse von Objekten, \"uber die dieselben Informationen gespeichert und die prinzipiell in gleicher Weise verarbeitet werden, wie z. B. die benannte Klasse bzw. der Objekttyp SOLDAT.
          \end{itemize}
        \end{merke}

        Die Bildung von Objekttypen h\"angt entscheidend von den Anforderungen
        des jeweils zu modellierenden Gegenstandsbereiches ab. Aus der Sicht
        eines Gro\ss h\"andlers kann ein Unternehmen mit all seinen Bereichen
        als ein einziger Objekttyp gesehen werden. Dasselbe Unternehmen dagegen
        wird aus seiner eigenen Sicht detailliert mit seinen Bereichen,
        Abteilungen, Mitarbeitern, Werkshallen, Fahrzeugen usw. zu modellieren
        sein. Diese Modellierung ist ebenfalls nicht eine einmalige, in sich
        abgeschlossene T\"atigkeit, denn im Laufe der Zeit m\"ussen \"Anderungen
        der Realit\"at auch im Modell eingearbeitet werden.
      \subsection{Traditionelles Pendant}
        Die Informationsverarbeitung mit Hilfe elektronischer Datenverarbeitung
        hat hinsichtlich der Datenspeicherung nur wenig prinzipiell neue
        Methoden entwickelt. Fast alle Konzepte, in Bezug auf das
        Entity-Relationship-Modell, haben ihr Pendant in der traditionellen
        Informationsspeicherung. Zum besseren Verst\"andnis der eingef\"uhrten
        Begriffe wird auf diese Zusammenh\"ange an den entsprechenden Stellen
        hingewiesen.

        So entsprechen die Objekttypen den traditionellen Karteik\"asten und die
        Objekte eines Objekttyps den Karteikarten, die in einem Karteikasten
        eingeordnet sind. In der traditionellen Arbeitsweise w\"urde f\"ur
        \enquote{Axel Schweiss} eine Karteikarte angelegt werden und z. B. im
        Karteikasten \enquote{Soldat} abgelegt werden
\clearpage
    \section{Festlegung der relevanten Eigenschaften}
      Im ersten Schritt, der Klassifizierung von Objekten, wurden Objekte, die
      in gleicher Art und Weise verarbeitet werden, in Objekttypen
      zusammengefasst. Um die Verarbeitung dieser Objekte automatisiert
      durchf\"uhren zu k\"onnen, ist es Voraussetzung, dass jeder Objekttyp
      bestimmte Angaben speichert. Diese Angaben werden f\"ur die elektronische
      Verarbeitung entweder als Eingabeinformation oder als Ausgabeinformation
      ben\"otigt.
      
      Aus diesem Grund ist es notwendig, f\"ur jeden Objekttyp die relevanten
      Eigenschaften der Objekte, die in ihm zusammengefasst werden, anzugeben.
      Damit wird der durch den Objekttyp definierte Begriff auf einen Satz
      relevanter Eigenschaften reduziert. Die Festlegung dieser Eigenschaften
      ist aber nur bei genauer Kenntnis der Gesch\"afts- und
      Verarbeitungsprozesse des Auftraggebers m\"oglich. Ohne diese Kenntnisse
      befindet man sich in jedem Falle im Bereich von Spekulationen.
      \subsection{Definitionen und Syntaxregeln}
        \begin{merke}
          \begin{itemize}
            \item \textbf{Eigenschaft:} Eine Eigenschaft (engl. Attribute) ist
            die Benennung f\"ur ein relevantes Merkmal aller Objekte, die in
            einem Objekttyp zusammengefa\ss t werden, z. B. Soldaten haben die
            Eigenschaft Dienstgrad.
            \item \textbf{Eigenschaftswert:} Ein Eigenschaftswert (engl.
            Attribute value) ist eine spezielle Auspr\"agung, die eine
            Eigenschaft f\"ur ein konkretes Objekt annimmt, z. B. der Dienstgrad
            Hauptfeldwebel.
          \end{itemize}
          \end{merke}

        F\"ur die Darstellung der Eigenschaften gelten folgende Regeln:
        \begin {enumerate}
          \item Die Benennung der Eigenschaft wird als Blase an den Objekttyp
          angeh\"angt.

          \begin{center}
           \scalebox{1}{
            \begin{tikzpicture}[node distance=1.5cm, every edge/.style={link}]
              \node[entity](e1){Soldat};
                \node[attribute](a1)[right = of e1]{Dienstgrad} edge (e1);
            \end{tikzpicture}
           }
          \end{center}
          \item Die Reihenfolge der Eigenschaften ist bedeutungslos.
          \item Die Benennung der Eigenschaft steht im Singular und muss f\"ur
          den Objekttyp eindeutig sein.
        \end{enumerate}
        Auch an dieser Stelle wird eine Abstraktion der realen Welt
        vorgenommen, in dem statt der Eigenschaftswerte, die jedes einzelne
        Objekt besitzt, nur die Eigenschaften der Objekttypen angegeben werden.
        Um diesen Abstraktionsprozess korrekt durchf\"uhren zu k\"onnen, bedarf
        es der Einhaltung einiger Regeln:
        \begin{itemize}
          \item Es darf niemals ein Eigenschaftswert als Eigenschaftsname
					verwendet werden (beispielsweise anstatt 02.05.1985 die Bezeichnung
					\enquote{Geburtsdatum} oder statt \enquote{m\"annlich} die Bezeichnung
					\enquote{Geschlecht}).
          \item Die Bezeichnung eines Objekttyps sollte niemals im
					Eigenschaftsnamen auftauchen, da dieser immer nur im Kontext
					des Objekttyps gilt (z. B. Person mit der Eigenschaft Name, nicht
					\enquote{Personname}, sondern die Eigenschaft \enquote{Name}
					w\"ahlen).
          \item Bei komplexen Eigenschaften wie z. B. einer Adresse oder einer
					PK stellt sich immer wieder die Frage, ob diese zerlegt werden
					m\"ussen oder ob sie als atomar \footnote{atomare Information = eine
					einzige, nicht mehr teilbare Information} betrachtet werden. Die
					Antwort auf diese Frage ist abh\"angig von den einzelnen
					Verarbeitungsprozessen, denen diese Daten unterliegen. Muss
					beispielsweise die Information verf\"ugbar sein, von welchem
					Kreiswehrersatzamt ein Soldat betreut wird, so muss  die PK in ihre
					Bestandteile zerlegt werden. Ist diese Information irrelevant, kann
					die PK im ganzen als atomar betrachtet werden.
          \item Problematisch ist auch die Entscheidung, ob speicherw\"urdige
          Informationen als Eigenschaften oder als ein eigenst\"andiger
          Objekttyp betrachtet werden sollen. Um einen eigenst\"andigen
          Objekttyp handelt es sich immer dann, wenn er f\"ur das Unternehmen
          bedeutsame Objekte enth\"alt, die relevante individuelle Eigenschaften
          besitzen.
        \end{itemize}
      \subsection{Traditionelle Datenspeicherung}
        Die Eigenschaften eines Objekttyps A entsprechen den Feldern, die auf
        einer Karteikarte zur Aufnahme der relevanten Informationen angelegt
        werden. Durch die gew\"ahlten Eigenschaften wird also die einheitliche
        Struktur aller Karteikarten eines Kastens festgelegt.

        Welche Eigenschaften k\"onnen Sie, bezogen auf das Beispiel
        \enquote{Bundeswehr}, den identifizierten Objekttypen zuordnen?
        \subsubsection{L\"osungsvorschlag}
          \begin{itemize}
            \item Objekttyp \enquote{Dienststelle}: Dienststellennummer,
            Bezeichung
            \item Objekttyp \enquote{Standort}: PLZ, Ort, Stra\ss e, Hausnummer
            \item Objekttyp \enquote{Dienstposten}: Dienstposten\_ID,
            Beginndatum, Enddatum, Dienstpostenbeschreibung
            \item Objekttyp \enquote{Soldat}: Personalnummer, PK, Name, Vorname,
            Dienstgrad, PLZ, Ort, Stra\ss e, Hausnummer
            \item Objekttyp \enquote{Ausr\"ustung}: Versorgungsnummer, Material,
            Farbe
          \end{itemize}
    \section{Festlegung der Identifizierung}
      Ein Objekttyp stellt die Zusammenfassung mehrerer gleichartiger Objekte
      dar. Gleichartig bedeutet, dass diese Objekte die gleichen Eigenschaften
      aufweisen und die Verarbeitungsprozesse f\"ur all diese Objekte gleich
      sind. Die einzelnen Objekte eines Objekttyps m\"ussen aber voneinander
      unterscheidbar sein. Deshalb muss festgelegt werden, auf welche Weise ein
      Objekt innerhalb des Objekktyps identifiziert werden kann.

      \subsection{Identifizierungsvarianten}
        F\"ur die Identifizierung eines Objekts innerhalb eines Objekttyps
        stehen die konkreten Eigenschaftswerte des Objekts zur Verf\"ugung.

        Es werden drei Varianten zur Identifizierung von Objekten
        unterschieden.
        \subsubsection{Identifizierung eines Objekts durch eine einzelne
Eigenschaft}
          Es kann vorkommen, dass die Eigenschaftswerte einer Eigenschaft eines
          Objekttyps eindeutig ist. D. h. jeder Eigenschaftswert dieser
          Eigenschaft ist so geartet, dass jedes Objekt dieses Objekttyps einen
          unterschiedlichen Wert f\"ur diese Eigenschaft hat. Durch Angabe
          dieses Eigenschaftswertes ist dann das Objekt eindeutig identifiziert.

          Beispiel: Soldaten werden i. d. R. durch ihre PK eindeutig
          identifiziert

				\subsubsection{Identifizierung eines Objekts durch eine Kombination
mehrerer Eigenschaften}
          In einigen F\"allen ist keine der Eigenschaften eines Objekttyps
          geeignet, alleine als identifizierende Eigenschaft zu fungieren. Um
          dieses Problem zu l\"osen kann versucht werden, eine
          \underline{minimale} Kombination von Eigenschaften eines Objekttyps
          als Identifikationsmerkmal zu benutzen. Dies funktioniert dann, wenn
          die Kombination der Werte der entsprechenden Eigenschaften bei jedem
          Objekt eindeutig sind.

          Beispiel: Die Kombination aus PLZ und Ortsbezeichnung ist eindeutig,
					eine Eigenschaft alleine nicht.

				\subsubsection{Identifizierung eines Objekts durch eine organisatorische
Eigenschaft}
          Sollte es vorkommen, dass weder eine einzelne Eigenschaft, noch eine
					Kombination von Eigenschaften zur Identifikation der Objekte eines
					Objekttyps geeignet ist, kann eine k\"unstliche Eigenschaft
					eingef\"uhrt werden, bei der die Eindeutigkeit durch organisatorische
					Ma\ss nahmen gew\"ahrleistet wird.

					Ein Beispiel hierf\"ur w\"are eine Ort\_ID, die es m\"oglich
					macht, einen Ort eindeutig zu bestimmen ohne die Kombination aus PLZ
					und Ortsnamen heranziehen zu m\"ussen.

          Es ist aber auch m\"oglich, dass eine organisatorische Eigenschaft
          gew\"ahlt wird, da abzusehen ist, dass eine identifizierende
          Eigenschaft durch eine andere ersetzt werden soll. Die
          Umstrukturierung der Datenbank w\"are zu aufwendig und zu komplex.

          Ein Beispiel aus der Praxis ist die PK und die Personalnummer eines
          jeden Soldaten. Die Personalnummer wird in geraumer Zeit die PK
          ersetzen. Es ist aus diesem Grund von Vorteil eine organisatorische
          Eigenschaft, wie die Personen\_ID zu w\"ahlen, um eine
          Umstrukturierung zu vermeiden.

          Andere Beispiele f\"ur organisatorische Eigenschaften sind:
          Artikelnummer und LfdNr.

          Die Identifizierung der Objekte eines Objekttyps mittels einer
					\enquote{organisatorischen Eigenschaft} ist bei der automatisierten
					Datenverarbeitung eine beliebte Vorgehensweise. Sie wird h\"aufig
					selbst dann angewendet, wenn nat\"urliche identifizierende
					Eigenschaften vorhanden sind. Meist handelt es sich um eine laufende
					Nummer.

					Dies hat den Vorteil, dass kurze identifizierende Eigenschaftswerte
					entstehen. Der Nachteil ist, dass Werte wie z. B. eine laufende
					Nummer meist keinerlei Aussagekraft haben und somit Gefahren wie
					Verwechslung oder Fehleingabe entstehen.

					Die Festlegung der Identifizierungsform f\"ur einen Objekttyp muss mit
					gro\ss er Sorgfalt erfolgen. Relationale Datenbank-Managementsysteme
					f\"ur die wir unsere Modellierung durch\-f\"uh\-ren, lassen es
					n\"amlich nicht zu, dass zwei Objekte desselben Objekttyps in der
					Kombination ihrer identifizierenden Merkmale \"ubereinstimmen. Bleibt
					ein Restrisiko hinsichtlich der Unikalit\"at der identifizierenden
					Merkmale und treten dann bei der praktischen Datenbankarbeit
					tats\"achlich zwei Objekte mit \"ubereinstimmenden Werten ihrer
					identifizierenden Merkmale auf, lehnt das Datenbank-Managementsystem
					die Speicherung des zweiten Objekts ab.

					\begin{merke}
						Um ein Modell \"ubersichtlich und verst\"andlich zu halten, sollte
						konsequent immer nur eine der drei zur Verf\"ugung stehenden
						Methoden f\"ur die Identifizierung von Objekten genutzt werden!
					\end{merke}
			\subsection{Markierung der Identifizierenden Eigenschaft}
				F\"ur die Markierung der (teil)identifizierenden Eigenschaft gelten
				die folgenden syntaktischen Regeln (diese werden u. U. in Kombination
				mit anderen angewendet):
\clearpage
        \begin {enumerate}
          \item Eine identifizierende Eigenschaft (bzw. jede teilidentifizierende Eigenschaft) wird durch Unterstreichung kenntlich gemacht.

          \begin{center}
           \scalebox{1}{
            \begin{tikzpicture}[node distance=1.5cm, every edge/.style={link}]
              \node[entity](e1){Soldat};
							\node[attribute](a1)[right = of e1]{\key{Personalnummer}} edge
(e1);
            \end{tikzpicture}
           }
          \end{center}
          \item Die Position der (teil)identifizierenden Eigenschaft innerhalb der Liste der Eigenschaften ist bedeutungslos. Sie sollte jedoch aus Gr\"unden der \"Ubersichtlichkeit immer zu erst genannt werden.
        \end{enumerate}
      \subsection{Klassisches Pendant}
        Bei der traditionellen Informationsspeicherung mit Hilfe von Karteik\"asten entspricht der Identifizierung die Festlegung eines Kennbegriffs: \"Ublicherweise wird im Kopf einer Karteikarte f\"ur das betreffende Objekt ein Wert angegeben, der das Objekt innerhalb des Karteikastens identifiziert\footnote{Kennbegriff wird oft auch als \enquote{Reiter} bezeichnet}. Um eine bestimmt Karteikarte manuell schneller finden zu k\"onnen, werden die Karteikarten des Karteikastens nach diesem Begriff sortiert.

        Betrachten wir wieder das Beispiel \enquote{Bundeswehr}. Welche Objekteigenschaften k\"on\-nen Ihrer Meinung nach als identifizierende Merkmale verwendet werden?
        \subsubsection{L\"osungsvorschlag}
          \begin{itemize}
            \item Objekttyp \enquote{Dienststelle}: Dienststellennummer
            \item Objekttyp \enquote{Standort}: PLZ, Ort
            \item Objekttyp \enquote{Dienstposten}: Dienstposten\_ID
            \item Objekttyp \enquote{Soldat}: Personalnummer
            \item Objekttyp \enquote{Ausr\"ustung}: Versorgungsnummer
          \end{itemize}
          An dieser Stelle wird die Entscheidung getroffen, dass f\"ur die Fortf\"uhrung dieses Modells organisatorische Einheiten als identifizierende Eigenschaften eingef\"uhrt werden!
          \begin{itemize}
            \item Objekttyp \enquote{Dienststelle}: Dienststellen\_ID
            \item Objekttyp \enquote{Standort}: Ort\_ID
            \item Objekttyp \enquote{Dienstposten}: DP\_ID
            \item Objekttyp \enquote{Soldat}: Personen\_ID
            \item Objekttyp \enquote{Ausr\"ustung}: Ausruestungs\_ID
          \end{itemize}

    \section{Modellierungsformen}
          \begin{center}
           \scalebox{.6}{
            \begin{tikzpicture}[node distance=1.5cm, every edge/.style={link}]
              \node[entity](dienstposten){Dienstposten};
                \node[attribute](dpid)[above = of dienstposten]{\key{DP\_ID}} edge (dienstposten);
                \node[attribute](dienstpostenid)[above left = of dienstposten]{Dienstposten\_ID} edge (dienstposten);
                \node[attribute](begindatum)[above right = of dienstposten]{Beginndatum} edge (dienstposten);
                \node[attribute](enddatum)[below right = of dienstposten]{Enddatum} edge (dienstposten);
                \node[attribute](dienstpostenbeschreibung)[below left = of dienstposten]{Dienstpostenbeschreibung} edge (dienstposten);
              \node[entity](soldat)[below left = 6cm of dienstposten]{Soldat};
                \node[attribute](personenid)[above = 1.6cm of soldat]{\key{Personen\_ID}} edge (soldat);
                \node[attribute](pk)[above right = of soldat]{PK} edge (soldat);
                \node[attribute](name)[below right = of soldat]{Name} edge (soldat);
                \node[attribute](vorname)[below left = of soldat]{Vorname} edge (soldat);
                \node[attribute](dienstgrad)[above left = of soldat]{Dienstgrad} edge (soldat);
              \node[entity](ausruestung)[below = 6cm of soldat]{Ausr\"ustung};
                \node[attribute](versorgungsnummer)[above left = of ausruestung]{Versorgungsnummer} edge (ausruestung);
                \node[attribute](bezeichnung)[above = 1.7cm of ausruestung]{Bezeichnung} edge (ausruestung);
                \node[attribute](ausruestungsid)[above right = of ausruestung]{\key{Ausruestungs\_ID}} edge (ausruestung);
                \node[attribute](material)[below right = of ausruestung]{Material} edge (ausruestung);
                \node[attribute](farbe)[below left = of ausruestung]{Farbe} edge (ausruestung);
              \node[entity](dienststelle)[below right = 6cm of dienstposten]{Dienststelle};
                \node[attribute](dienststellennummer)[below left = of dienststelle]{\key{Dienststellen\_ID}} edge (dienststelle);
                \node[attribute](dienststellennummer)[above left = of dienststelle]{Dienststellennummer} edge (dienststelle);
                \node[attribute](bezeichung)[above right = of dienststelle]{Bezeichnung} edge (dienststelle);
                \node[attribute](groesse)[below right = of dienststelle]{Gr\"o\ss e} edge (dienststelle);
              \node[entity](standort)[below = 6cm of dienststelle]{Standort};
                \node[attribute](ortid)[above = of standort]{\key{Ort\_ID}} edge (standort);
                \node[attribute](plz)[above left = of standort]{PLZ} edge (standort);
                \node[attribute](ort)[above right = of standort]{Ort} edge (standort);
                \node[attribute](strasse)[below right = of standort]{Stra\ss e} edge (standort);
                \node[attribute](hausnummer)[below left = of standort]{Hausnummer} edge (standort);
            \end{tikzpicture}
           }
          \end{center}
        Hinweis: Das Objekt Soldat enth\"alt nicht alle Attribute. Es fehlen die Attribute Personalnummer, Plz, Ort, Stra\ss e und Hausnummer.
