      \subsection{\"Ubungsaufgabe IT-Helpdesk}
        Entwerfen Sie, basierend auf der folgenden Lage, ein ER-Modell inklusive der Beziehungen zwischen den Entit\"aten.

        F\"ur ein IT-Supportunternehmen soll eine Datenbank erschaffen werden, welche es erm\"oglicht, telefonische Supportanfragen von Kunden zu erfassen. Im Einzelnen m\"ussen die nachfolgend beschriebenen Zusammenh\"ange in der Datenbank abgebildet werden.

        \subsubsection{Vorgaben}
          \begin{itemize}
            \item Jeder Kunde, der den IT-Helpdesk anruft, muss mit Vorname, Nachname und Kundennummer
            gespeichert werden.
            \item Die Datenbank muss es erm\"oglichen, zu einem Kunden, mindestens eine oder mehrere
            Adressen zu speichern. Eine Adresse besteht aus Stra\ss{}e, Hausnummer, Postleitzahl (PLZ) sowie Ort und muss mehreren Kunden zugeordnet werden k\"onnen. Adressen zu denen keine Kunden mehr
            in der Datenbank existieren verbleiben noch f\"ur mindestens ein Jahr in der Datenbank, ehe
            sie gel\"oscht werden.
            \item Ein Kunde gibt beim IT-Helpdesk seine Kontaktdaten an. Diese Kontaktdaten bestehen
            meist aus Telefonnummer und E-Mail-Adresse. Die Telefonnummer muss nicht zwingend mit angegeben werden. Allerdings muss mind. eine Kontaktinformation gespeichert werden. Ein Kontaktdatensatz wird immer nur einem Kunden zugeordnet.
            \item F\"ur das Management des IT-Supportunternehmen  ist es wichtig zu wissen, welcher
            Mitarbeiter des Helpdesks mit welchem Kunden Kontakt hatte. Zu einem Mitarbeiter wird dessen
            Vorname, Nachname und seine Personalnummer gespeichert.
            \item Jedesmal wenn ein Kunde mit dem IT-Helpdesk einen Kontakt herstellt, muss der
            Mitarbeiter des Helpdesks die genaue Uhrzeit, das Datum, den Anlass und die Dauer der
            Dienstleistung notieren.
          \end{itemize}
          \subsubsection{Zusatzaufgabe}
            Das IT-Supportunternehmen hat angefragt, ob es m\"oglich ist, die Datenbank so zu ver\"andern,
            dass mehrere Mitarbeiter an einem Kontakt arbeiten k\"onnen. Jedesmal wenn ein Mitarbeiter an
            einem Kontakt arbeitet, m\"ussen die Uhrzeit und das Datum des Telefonats, sowie dessen Dauer
            gespeichert werden.

            Pr\"ufen Sie, ob diese M\"oglichkeit gegeben ist und falls Ja, passen Sie das Modell
            entsprechend an!
\clearpage