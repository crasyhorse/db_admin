\clearpage
      \subsection{\"Ubungsaufgabe Bank}
        Der Manager einer neu gegr\"undeten Bank beauftragt Sie mit der
        Erstellung eines Datenmodells f\"ur die Verwaltung der Bankgesch\"afte.
        In der vor kurzem erfolgten Besprechung wurden folgende Eckpunkte
        festgelegt:

        Meine Bank hat mehrere Kunden aber mindestens Einen sonst w\"urde meine
        Bank nicht existieren. Alle meine Kunden besitzen eine Kunden ID, einen
        Vorname, einen Nachnamen, ein Geburtsdatum und eine Rechnungsadresse.

        Kunden k\"onnen eines oder mehrere Konten besitzen. Jedes Konto hat
        jedoch genau einen Besitzer. Ein Konto ist entweder ein Sparbuch, ein
        Depot oder ein Girokonto. Andere Arten von Konten werden bei uns nicht
        gef\"uhrt und es gibt auch keine Mischformen dieser drei Kontoarten.
        Jedes Konto soll in unserer Datenbank mit einer IBAN (international bank
        account number) versehen werden. F\"ur die Sparb\"ucher und die
        Girokonten ist auch das aktuelle Guthaben zu speichern. Auf unsere
        Sparb\"ucher gibt es einen Habenzinssatz, f\"ur die Girokonten wird ein
        Sollzinssatz gef\"uhrt, der bei \"Uberziehung des Kontos zum Tragen
        kommt. Wer ein Girokonto besitzt, f\"ur den wird j\"ahrlich eine
        Kontof\"uhrungsgeb\"uhr f\"allig. Bei einem Depot muss eine
        Er\"offnungsgeb\"uhr gezahlt werden.

        Unsere Kunden k\"onnen einer anderen Person, die ebenfalls bei uns Kunde
        sein muss, h\"ochstens eine Vollmacht \"uber ein Konto geben. Ein Kunde
        kann mehrere Vollmachten \"uber verschiedene Konten besitzen.

        Ein wesentlicher Bestandteil der Datenbank ist unser Buchungssystem. Auf
        den Konten unserer Kunden erfolgen Buchungen (Einzahlungen,
        Auszahlungen, \"Uberweisungen, usw. NICHT SPEZIALISIEREN!!!), die mit
        einem Betrag und einem Buchungsdatum gespeichert werden m\"ussen. Jede
        Buchung ist immer genau einem Konto zuordenbar, w\"ahrend es f\"ur ein
        Konto mehrere Buchungen geben kann. Es muss auch der Fall
        ber\"ucksichtigt werden, dass z. B. auf einem neu er\"offneten Konto
        noch keine Buchung vorgenommen wurde.

        Damit unsere Kunden auch Gesch\"afte mit Kunden anderer Banken t\"atigen
        k\"onnen, m\"ussen von diesen fremden Personen der Vorname, der Nachname
        und eine IBAN gespeichert werden. Auch ben\"otigen wir den Namen und den
        BIC (bank identity code) der fremden Bank (Hinweis: Hier bietet es sich
        an, die Kunden in Eigenkunden unserer Bank und Fremdkunden zu
        unterteilen!). Es kommt auch vor, dass einer unserer Kunden Kunde bei
        einer anderen Bank ist und Geld von einem seiner Konten auf ein Anderes,
        bei einer anderen Bank, transferieren m\"ochte.

        Die Eigenkunden k\"onnen von mehreren Mitarbeitern unserer Bank betreut
        werden. Ein Mitarbeiter kann mehrere Kunden betreuen. Die Vorgesetzten
        der Filialleiter, die ebenfalls als Mitarbeiter gef\"uhrt werden,  haben
        keinen Kundenkontakt. Jeder Mitarbeiter hat genau einen Vorgesetzten,
        au\ss{}er mir selbst. Ein vorgesetzter Mitarbeiter hat aber mehrere
        Mitarbeiter, die er f\"uhrt. Meine Mitarbeiter sollen in der Datenbank
        mit Vorname, Nachname und einer Mitarbeiter ID gespeichert werden.

        Jeder Mitarbeiter der Bank arbeitet in einer Bankfiliale, au\ss{}er den
        Vorgesetzten der Filialleiter. In einer Bankfiliale arbeiten mindestens
        ein aber maximal zehn Mitarbeiter. Die Bankfiliale soll mit ihrer 
        Adresse in der Datenbank gespeichert werden.
\clearpage