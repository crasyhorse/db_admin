      \subsection{\"Ubungsaufgabe Autoh\"andler}
        Entwerfen Sie, basierend auf der folgenden Lage, ein ER-Modell, inklusive der Beziehungen zwischen
        den Entit\"{a}ten.

        Der Eigent\"umer eines namhaften Autohauses der Region beauftragt Sie eine Datenbank zu entwerfen.
        Bei einem ersten Gespr\"ach erfahren Sie, dass die Datenbank ben\"otigt wird, um das Autohaus besser zu
        verwalten. Des Weiteren wird die Grobgliederung der Datenbank festgehalten.
        \begin{itemize}
          \item Die Datenbank soll alle Angestellten des Autohauses auff\"uhren. Diese unterteilen sich in Verk\"aufer und Mechaniker. Verk\"aufer sind f\"ur die Betreuung der Kunden verantwortlich und f\"uhren  Autoverk\"aufe durch. Au\ss erdem nehmen die Verk\"aufer auch Auftr\"age f\"ur Reparaturen an. Mechaniker sind nur f\"ur die Durchf\"uhrung der Reparaturen zust\"andig.
          \item In der Datenbank sollen Vor- und Nachname, Geburtsdatum und die Adresse des jeweiligen Mitarbeiters
          hinterlegt werden k\"onnen. Jeder Verk\"aufer kann mehrere Kunden betreuen, einen Verkauf vornehmen oder eine Reparatur annehmen. Allerdings wird ein Kunde nur von genau einem Verk\"aufer betreut. Dies gilt auch im Bezug auf einen Autoverkauf und die Annahme einer Reparatur. Der Besitzer des Autohauses bittet Sie auch zu ber\"ucksichtigen, dass ein neu eingestellter Verk\"aufer noch nicht all diese T\"atigkeiten durchf\"uhren kann.
        \end{itemize}
        Bei einem zweiten Treffen mit dem Autohausbesitzer werden die Details der Datenbank besprochen:
        \begin{itemize}
          \item Kunden des Autohauses sollen mit Vor- und Nachnamen sowie ihrer Adresse im System erfasst werden. Es ist dabei zu bedenken, dass Personen, die ihr Auto nur an das Autohaus verkaufen, auch als Kunden erfasst werden, selbst wenn diese dann kein anderes Auto im Autohaus, bei Ihrem Auftraggeber, kaufen.
          \item Ihr Auftraggeber bittet Sie weiterhin, in der Datenbank seine gesamten Autos, inklusive der Autos
          seiner Kunden zu ber\"ucksichtigen. Autos k\"onnen schon einem Kunden geh\"oren oder werden an einen Kunden verkauft. Manche Kunden besitzen kein Auto z. B. Neukunden, andere haben zwei oder mehr Autos.
          \item Oft werden Autos zur Reparatur gebracht. Reparaturauftr\"age
          werden genau einem Auto zugeordnet. Manche Autos, z. B. Neuwagen haben
          noch keine Reparaturen, andere dagegen schon mehrere.
\clearpage
          \item Die Fahrzeuge werden mit der Automarke, dem Modell, dem
          Marktpreis und der Fahrleistung erfasst. Es ist zu erw\"ahnen, dass zu
          einem Kaufvertrag ein Datum und immer nur ein Auto geh\"oren. Hier
          kann davon ausgegangen werden, dass jedes Auto h\"ochstens einmal
          verkauft wird.
          \item Als Letztes soll festgehalten werden, dass ein Mechaniker eine oder mehrere Reparaturen durchf\"uhren kann. Zu jeder Reparatur m\"ussen deren Datum, der Rechnungspreis f\"ur die geleisteten Arbeiten und die Ersatzteile, sowie die ben\"otigten Arbeitsstunden gespeichert werden. Eine Reparatur wird auch nur von einem Mechaniker durchgef\"uhrt.
        \end{itemize}
\clearpage