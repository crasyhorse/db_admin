      \subsection{\"Ubungsaufgabe Universit\"at}
        Der Leiter der Abteilung f\"ur Verwaltungsangelegenheiten einer
        Universit\"at beauftragt Sie mit der Erstellung eines Datenmodells, um
        die Verwaltungsstruktur effizienter zu gestalten. In einer Besprechung
        wurden folgende Eckpunkte festgelegt:

        Meine Universit\"at gliedert sich in Fakult\"aten. Diese umfassen immer
        mindestens ein Institut. Ein Institut kann niemals zu mehreren
        Fakult\"aten gleichzeitig geh\"oren. F\"ur beide m\"ussen deren
        Bezeichungen gespeichert werden.

        In der Datenbank sollen drei unterschiedliche Personentypen eingetragen
        werden. Diese sind mit ihren Vornamen, Nachnamen, der Adresse und dem
        Geburtsdatum zu hinterlegen. Der erste Personentyp sind Professoren,
        welche genau ein Institut leiten. Umgekehrt kann ein Institut auch nur
        von genau einem Professor geleitet werden.

        Professoren halten Vorlesungen. Jeder Professor h\"alt mindestens eine
        Vorlesung. Der Fall, dass eine Vorlesung von mehreren Professoren
        gehalten wird, tritt nicht auf. Vorlesungen werden immer in
        H\"ors\"alen gehalten. In einem H\"orsaal k\"onnen mehrere Vorlesungen
        gehalten werden, jedoch wird eine Vorlesung immer nur in genau einem
        H\"orsaal abgehalten.

        Eine weitere Aufgabe der Professoren ist es, \"Ubungen bereitzustellen.
        Dies geschieht jedoch auf freiwilliger Basis, so dass nicht jeder
        Professor \"Ubungen f\"ur seine Studenten zur Verf\"ugung stellt. Eine
        \"Ubung wird immer von genau einem Professor erstellt.

        In bestimmten Zeitabst\"anden wird ein Professor zum \enquote{Dekan}
        gew\"ahlt. Der Dekan ist der Leiter einer Fakult\"at. Ein Professor kann
        nur h\"ochstens eine Dekanstelle besetzen und eine Fakult\"at wird immer
        von genau einem Dekan geleitet.

        Eine zweite Personengruppe in der Datenbank sind die Wissenschaftlichen
        Mitarbeiter (im Folgenden nur noch WiMa genannt). WiMas betreuen immer
        wieder \"Ubungen, haben aber auch andere Aufgaben. Eine \"Ubung kann nur
        von einem WiMa betreut werden. Zus\"atzlich zu den genannten Daten, die
        allgemein f\"ur einen Personentyp gespeichert werden, wird f\"ur WiMas
        und Professoren das jeweilige Gehalt in der Datenbank abgelegt.

        Der dritte Personentyp sind Studenten. Diese k\"onnen an \"Ubungen und
        Vorlesungen teilnehmen. Damit eine \"Ubung oder Vorlesung stattfindet,
        muss sich mindestens ein Student daf\"ur eingeschrieben haben. Studenten
        erhalten, neben den angesprochenen Personenparametern, noch zus\"atzlich
        eine Matrikelnummer.

        Die an der Universit\"at gehaltenen Vorlesungen und \"Ubungen sind mit
        einem Thema versehen. Die \"Ubungen unterteilen sich in Rechen- bzw.
        Labor\"ubungen und werden mit einer Aufgabennummer versehen.
        Rechen\"ubungen finden in H\"ors\"alen und Labor\"ubungen in Laboren
        statt. Nicht jeder H\"orsaal bzw. jedes Labor ist immer besetzt. Eine
        \"Ubung findet allerdings immer nur in einem Raum statt. F\"ur jeden
        H\"orsaal und jedes Labor muss die Anzahl der Sitzpl\"atze, die
        Raumnummer sowie die Geb\"audenummer gespeichert werden.
\clearpage