\clearpage
    \section{\"Ubungen - Unterabfragen}
      \begin{enumerate}
        \item Schreiben Sie eine Abfrage, die f\"ur alle Eigenkunden, die keinen
        Berater haben (die nicht in der Tabelle
        \identifier{EigenkundeMitarbeiter} enthalten sind), den Vor- und den
        Nachnamen anzeigt.
        \begin{itemize}
          \item L\"osen Sie die Aufgabe mit Hilfe des \languageorasql{EXISTS}-Operators!
          \item L\"osen Sie die Aufgabe mit Hilfe des \languageorasql{IN}-Operators!
        \end{itemize}
        \begin{center}
          \begin{small}
            \changefont{pcr}{m}{n}
            \tablefirsthead {
              \multicolumn{1}{l}{\textbf{VORNAME}} &
              \multicolumn{1}{l}{\textbf{NACHNAME}} \\
              \cmidrule(l){1-1}\cmidrule(l){2-2}
            }
            \tablehead{}
            \tabletail {
              \multicolumn{2}{l}{\textbf{16 Zeilen ausgew\"ahlt}} \\
            }
            \tablelasttail {
              \multicolumn{2}{l}{\textbf{16 Zeilen ausgew\"ahlt}} \\
            }
            \begin{msoraclesql}
              \begin{supertabular}{ll}
                Sebastian & Schr\"oder \\
                Udo & Schumacher \\
                Mia & Huber \\
                Simon & Witte \\
                Max & Bunzel \\
                Finn & Fischer \\
                Lara & Meierh\"ofer \\
                Jannis & Meier \\
              \end{supertabular}
            \end{msoraclesql}
          \end{small}
        \end{center}
        \item Erstellen Sie eine Abfrage, die ermittelt, ob es Mitarbeiter gibt
        (Vorname und Nachname), die keine Kundenberatung durchf\"uhren.
        Ausgenommen sind leitende Mitarbeiter (Mitarbeiter die in keiner
        Bankfiliale arbeiten) und Filialleiter.
        \begin{itemize}
          \item L\"osen Sie die Aufgabe mit Hilfe des \languageorasql{EXISTS}-Operators!
          \item L\"osen Sie die Aufgabe mit Hilfe des \languageorasql{IN}-Operators!
        \end{itemize}
        \begin{center}
          \begin{small}
            \changefont{pcr}{m}{n}
            \tablefirsthead {
              \multicolumn{1}{l}{\textbf{VORNAME}} &
              \multicolumn{1}{l}{\textbf{NACHNAME}} \\
              \cmidrule(l){1-1}\cmidrule(l){2-2}
            }
            \tablehead{}
            \tabletail {
              \multicolumn{2}{l}{\textbf{40 Zeilen ausgew\"ahlt}} \\
            }
            \tablelasttail {
              \multicolumn{2}{l}{\textbf{40 Zeilen ausgew\"ahlt}} \\
            }
            \begin{msoraclesql}
              \begin{supertabular}{ll}
                Amelie & Kr\"uger \\
                Anna & Schneider \\
                Chris & Simon \\
                Christian & Haas \\
                Elias & Sindermann \\
                Emilia & K\"ohler \\
                Emma & Kr\"uger \\
              \end{supertabular}
            \end{msoraclesql}
          \end{small}
        \end{center}
\clearpage
        \item Schreiben Sie eine Abfrage, die den h\"aufigsten Vornamen der
        Bankmitarbeiter anzeigt und wie oft dieser in der Tabelle
        \identifier{Mitarbeiter} vorkommt.
        \begin{center}
          \begin{small}
            \changefont{pcr}{m}{n}
            \tablefirsthead {
              \multicolumn{1}{l}{\textbf{VORNAME}} &
              \multicolumn{1}{r}{\textbf{ANZAHL}} \\
              \cmidrule(l){1-1}\cmidrule(r){2-2}
            }
            \tablehead{}
            \tabletail {
              \multicolumn{2}{l}{\textbf{1 Zeile ausgew\"ahlt}} \\
            }
            \tablelasttail {
              \multicolumn{2}{l}{\textbf{1 Zeile ausgew\"ahlt}} \\
            }
            \begin{msoraclesql}
              \begin{supertabular}{lr}
                Chris & 5 \\
              \end{supertabular}
            \end{msoraclesql}
          \end{small}
        \end{center}
        \item Schreiben Sie eine Abfrage, welche die drei Eigenkunden mit den
        niedrigsten Guthaben auf den Girokonten anzeigt.
        \begin{center}
          \begin{small}
            \changefont{pcr}{m}{n}
            \tablefirsthead {
              \multicolumn{1}{l}{\textbf{VORNAME}} &
              \multicolumn{1}{l}{\textbf{NACHNAME}} &
              \multicolumn{1}{r}{\textbf{GUTHABEN}} \\
              \cmidrule(l){1-1}\cmidrule(l){2-2}\cmidrule(r){3-3}
            }
            \tablehead{}
            \tabletail {
              \multicolumn{3}{l}{\textbf{3 Zeilen ausgew\"ahlt}} \\
            }
            \tablelasttail {
              \multicolumn{3}{l}{\textbf{3 Zeilen ausgew\"ahlt}} \\
            }
            \begin{msoraclesql}
              \begin{supertabular}{llr}
                Franz & Walther & -140505,1 \\
                Jan & Simon & -98218,6 \\
                Philipp & Hartmann & -69705,6 \\
              \end{supertabular}
            \end{msoraclesql}
          \end{small}
        \end{center}
        \item Ver\"andern Sie die Abfrage aus der vorangegangenen Aufgabe so,
        dass die drei Eigenkunden mit dem niedrigsten Guthaben (Girokonto +
        Sparbuch) angezeigt werden. Es m\"ussen auch diejenigen Kunden angezeigt
        werden, die nur ein Girokonto oder nur ein Sparbuch haben!
        \begin{center}
          \begin{small}
            \changefont{pcr}{m}{n}
            \tablefirsthead {
              \multicolumn{1}{l}{\textbf{VORNAME}} &
              \multicolumn{1}{l}{\textbf{NACHNAME}} &
              \multicolumn{1}{r}{\textbf{SUM(GUTHABEN)}} \\
              \cmidrule(l){1-1}\cmidrule(l){2-2}\cmidrule(r){3-3}
            }
            \tablehead{}
            \tabletail {
              \multicolumn{3}{l}{\textbf{3 Zeilen ausgew\"ahlt}} \\
            }
            \tablelasttail {
              \multicolumn{3}{l}{\textbf{3 Zeilen ausgew\"ahlt}} \\
            }
            \begin{msoraclesql}
              \begin{supertabular}{llr}
                Franz & Walther & -139154,4 \\
                Jan & Simon & -98218,6 \\
                Philipp & Hartmann & -69065,9 \\
              \end{supertabular}
            \end{msoraclesql}
          \end{small}
        \end{center}
        \item Schreiben Sie eine Abfrage, die alle Eigenkunden anzeigt, welche
        im Jahr 1985 keine Buchungen verursacht haben.
        \begin{center}
          \begin{small}
            \changefont{pcr}{m}{n}
            \tablefirsthead {
              \multicolumn{1}{l}{\textbf{VORNAME}} &
              \multicolumn{1}{l}{\textbf{NACHNAME}} \\
              \cmidrule(l){1-1}\cmidrule(l){2-2}
            }
            \tablehead{}
            \tabletail {
              \multicolumn{2}{l}{\textbf{285 Zeilen ausgew\"ahlt}} \\
            }
            \tablelasttail {
              \multicolumn{2}{l}{\textbf{285 Zeilen ausgew\"ahlt}} \\
            }
            \begin{msoraclesql}
              \begin{supertabular}{ll}
                Sarah & Bauer \\
                Sofia & Bauer \\
                Tom & Bauer \\
                Alina & Baumann \\
              \end{supertabular}
            \end{msoraclesql}
          \end{small}
        \end{center}
\clearpage
        \item Schreiben Sie eine Abfrage, die f\"ur jede Bankfiliale den
        Mitarbeiter mit dem h\"ochsten Gehalt ausgibt.
        \begin{center}
          \begin{small}
            \changefont{pcr}{m}{n}
            \tablefirsthead {
              \multicolumn{1}{l}{\textbf{BANKFILIALE}} &
              \multicolumn{1}{l}{\textbf{VORNAME}} &
              \multicolumn{1}{l}{\textbf{NACHNAME}} &
              \multicolumn{1}{r}{\textbf{GEHALT}} \\
              \cmidrule(l){1-1}\cmidrule(l){2-2}\cmidrule(l){3-3}\cmidrule(r){4-4}
            }
            \tablehead{}
            \tabletail {
              \multicolumn{4}{l}{\textbf{20 Zeilen ausgew\"ahlt}} \\
            }
            \tablelasttail {
              \multicolumn{4}{l}{\textbf{20 Zeilen ausgew\"ahlt}} \\
            }
            \begin{msoraclesql}
              \begin{supertabular}{lllr}
                Poststra\ss{}e 1 06449 Aschersleben & Dirk & Peters & 12000 \\
                Kirchstra\ss{}e 8 39444 Hecklingen & Leonie & Kaiser & 12000 \\
                Schmiedestra\ss{}e 3 39240 Sta\ss{}furt & Finn & K\"ohler & 12000 \\
                Am Dom 11 06449 Giersleben & Lena & Gro\ss{}e & 12000 \\
              \end{supertabular}
            \end{msoraclesql}
          \end{small}
        \end{center}
        \item Schreiben Sie eine Abfrage, die f\"ur jeden Wohnort
        (\identifier{Eigenkunde.Ort}) den Kunden anzeigt, der im Jahr 1987 das
        h\"ochste Einkommen hatte (Das Einkommen ist die Summe aller Betr\"age
        eines Kunden, in der Tabelle \identifier{Buchung}). Sortieren Sie die
        Abfrage nach den Wohnorten.
        \begin{center}
          \begin{small}
            \changefont{pcr}{m}{n}
            \tablefirsthead {
              \multicolumn{1}{l}{\textbf{ORT}} &
              \multicolumn{1}{l}{\textbf{VORNAME}} &
              \multicolumn{1}{l}{\textbf{NACHNAME}} &
              \multicolumn{1}{r}{\textbf{BETRAG}} \\
              \cmidrule(l){1-1}\cmidrule(l){2-2}\cmidrule(l){3-3}\cmidrule(r){4-4}
            }
            \tablehead{}
            \tabletail {
              \multicolumn{4}{l}{\textbf{30 Zeilen ausgew\"ahlt}} \\
            }
            \tablelasttail {
              \multicolumn{4}{l}{\textbf{30 Zeilen ausgew\"ahlt}} \\
            }
            \begin{msoraclesql}
              \begin{supertabular}{lllr}
                Alsleben & Peter & Koch & 57855,4 \\
                Aschersleben & Lara & D\"uhning & 2395,7 \\
                Barby & Chris & Beck & -6817,8 \\
              \end{supertabular}
            \end{msoraclesql}
          \end{small}
        \end{center}
        \item Erstellen Sie eine Abfrage, die die Ums\"atze der Bank
        (SUM(Buchung.Betrag)) f\"ur die Jahre 1985 bis einschlie\ss{}lich 1989
        als Pivottabelle anzeigt.
        \begin{center}
          \begin{small}
            \changefont{pcr}{m}{n}
            \tablefirsthead {
              \multicolumn{1}{r}{\textbf{'1985'}} &
              \multicolumn{1}{r}{\textbf{'1986'}} &
              \multicolumn{1}{r}{\textbf{'1987'}} &
              \multicolumn{1}{r}{\textbf{'1988'}} &
              \multicolumn{1}{r}{\textbf{'1989'}} \\
              \cmidrule(r){1-1}\cmidrule(r){2-2}\cmidrule(r){3-3}\cmidrule(r){4-4}\cmidrule(r){5-5}
            }
            \tablehead{}
            \tabletail {
              \multicolumn{5}{l}{\textbf{1 Zeile ausgew\"ahlt}} \\
            }
            \tablelasttail {
              \multicolumn{5}{l}{\textbf{1 Zeile ausgew\"ahlt}} \\
            }
            \begin{msoraclesql}
              \begin{supertabular}{rrrrr}
                559132,5 & 539497,2 & -2036841,3 & 1081361 & 1027003,1 \\
              \end{supertabular}
            \end{msoraclesql}
          \end{small}
        \end{center}
\clearpage
        \item Ver\"andern Sie die Abfrage aus der vorangegangenen Aufgabe so,
        dass die Betr\"age innerhalb der einzelnen Jahre nach Quartalen
        aufgeteilt werden.
        \begin{center}
          \begin{small}
            \changefont{pcr}{m}{n}
            \tablefirsthead {
              \multicolumn{1}{l}{\textbf{QUARTAL}} &
              \multicolumn{1}{r}{\textbf{'1985'}} &
              \multicolumn{1}{r}{\textbf{'1986'}} &
              \multicolumn{1}{r}{\textbf{'1987'}} &
              \multicolumn{1}{r}{\textbf{'1988'}} &
              \multicolumn{1}{r}{\textbf{'1989'}} \\
              \cmidrule(l){1-1}\cmidrule(r){2-2}\cmidrule(r){3-3}\cmidrule(r){4-4}\cmidrule(r){5-5}\cmidrule(r){6-6}
            }
            \tablehead{}
            \tabletail {
              \multicolumn{6}{l}{\textbf{4 Zeilen ausgew\"ahlt}} \\
            }
            \tablelasttail {
              \multicolumn{6}{l}{\textbf{4 Zeilen ausgew\"ahlt}} \\
            }
            \begin{msoraclesql}
              \begin{supertabular}{lrrrrr}
                1 & 32204,8 & 985,2 & 2981,1 & 176852 & 9777,1 \\
                3 & -11792,8 & -71935,3 & 191697,3 & 282848 & 681185,9 \\
                2 & 151841,1 & 53654,8 & -2174503,9 & 430097,2 & 223402,7 \\
                4 & 386879,4 & 556792,5 & -57015,8 & 191563,8 & 112637,4 \\
              \end{supertabular}
            \end{msoraclesql}
          \end{small}
        \end{center}
        \item Ver\"andern Sie die Abfrage aus der vorangegangenen Aufgabe so,
        dass eine Summenzeile, unterhalb der Pivottabelle angezeigt wird.
        \begin{center}
          \begin{small}
            \changefont{pcr}{m}{n}
            \tablefirsthead {
              \multicolumn{1}{l}{\textbf{QUARTAL}} &
              \multicolumn{1}{r}{\textbf{'1985'}} &
              \multicolumn{1}{r}{\textbf{'1986'}} &
              \multicolumn{1}{r}{\textbf{'1987'}} &
              \multicolumn{1}{r}{\textbf{'1988'}} &
              \multicolumn{1}{r}{\textbf{'1989'}} \\
              \cmidrule(l){1-1}\cmidrule(r){2-2}\cmidrule(r){3-3}\cmidrule(r){4-4}\cmidrule(r){5-5}\cmidrule(r){6-6}
            }
            \tablehead{}
            \tabletail {
              \multicolumn{6}{l}{\textbf{5 Zeilen ausgew\"ahlt}} \\
            }
            \tablelasttail {
              \multicolumn{6}{l}{\textbf{5 Zeilen ausgew\"ahlt}} \\
            }

            \begin{msoraclesql}
              \begin{supertabular}{lrrrrr}
                1 & 32204,8 & 985,2 & 2981,1 & 176852 & 9777,1 \\
                2 & 151841,1 & 53654,8 & -2174503,9 & 430097,2 & 223402,7 \\
                3 & -11792,8 & -71935,3 & 191697,3 & 282848 & 681185,9 \\
                4 & 386879,4 & 556792,5 & -57015,8 & 191563,8 & 112637,4 \\
                Summe & 559132,5 & 539497,2 & -2036841,3 & 1081361 & 1027003,1 \\
              \end{supertabular}
            \end{msoraclesql}
          \end{small}
        \end{center}
      \end{enumerate}
