\clearpage
    \section{L\"osungen - Gruppenfunktionen}
      \begin{enumerate}
        \item Schreiben Sie eine Abfrage, die das h\"ochste und das niedrigste
        Gehalt, das Durchschnittsgehalt und die Summe aller Geh\"alter ausgibt.
        Beschriften Sie die Spalten, wie es in der L\"osung zu sehen ist.
        \begin{msoraclesql}[\FALSE]
        \end{msoraclesql}
        \begin{lstlisting}[language=oracle_sql]
SELECT MAX(Gehalt) AS "Maximum", MIN(Gehalt) AS "Minimum",
       AVG(Gehalt) AS "Mittelwert", SUM(Gehalt) AS "Summe"
FROM   Mitarbeiter;
        \end{lstlisting}
        \item Ver\"andern Sie die Abfrage aus der vorangegangenen Abfrage so,
        dass die Informationen f\"ur jede einzelne Bankfiliale angezeigt werden.
        Sortieren sie das Ergebnis nach den IDs der Bankfilialen.
        \begin{msoraclesql}[\FALSE]
        \end{msoraclesql}
        \begin{lstlisting}[language=oracle_sql]
SELECT   Bankfiliale_ID, MAX(Gehalt) AS "Maximum", MIN(Gehalt) AS "Minimum",
         AVG(Gehalt) AS "Mittelwert", SUM(Gehalt) AS "Summme"
FROM     Mitarbeiter
WHERE    Bankfiliale_ID IS NOT NULL
GROUP BY Bankfiliale_ID
ORDER BY Bankfiliale_ID;
        \end{lstlisting}
        \item Schreiben Sie eine Abfrage, die die Anzahl der Mitarbeiter pro
        Bankfiliale ausgibt. Beschriften Sie die Spalten so, wie es in der
        L\"osung zu sehen ist und sortieren Sie das Ergebnis nach den IDs der
        Filialen.
        \begin{msoraclesql}[\FALSE]
        \end{msoraclesql}
        \begin{lstlisting}[language=oracle_sql]
SELECT   Bankfiliale_ID, COUNT(*) AS "Anzahl"
FROM     Mitarbeiter
GROUP BY Bankfiliale_ID
ORDER BY Bankfiliale_ID;
        \end{lstlisting}
        \item Schreiben Sie eine Abfrage, die f\"ur jeden Ort einzeln, die
        Anzahl der Eigenkunden z\"ahlt, die vor dem \enquote{01.01.1990} 18
        Jahre alt waren.
        \begin{oraclesql}[\FALSE]
        \end{oraclesql}
        \begin{lstlisting}[language=oracle_sql]
SELECT   Ort, COUNT(*) AS "Anzahl"
FROM     Eigenkunde ek
WHERE    Geburtsdatum + INTERVAL '18' YEAR <
           TO_DATE('01.01.1990', 'DD.MM.YYYY')
GROUP BY Ort;
        \end{lstlisting}
\clearpage
        \begin{mssql}[\FALSE]
        \end{mssql}
        \begin{lstlisting}[language=ms_sql]
SELECT   Ort, COUNT(*) AS "Anzahl"
FROM     Eigenkunde ek
WHERE    DATEADD(YEAR, 18, Geburtsdatum) <
           CONVERT(DATETIME2, '01.01.1990', 104)
GROUP BY Ort;
        \end{lstlisting}
        \item Erstellen Sie eine Abfrage, die f\"ur alle bankeigenen Kunden die
        Buchungen auf deren Girokonten z\"ahlt. Interessant sind nur Buchungen
        mit einem Betrag \textgreater 10.000 EUR. Sortieren Sie die Abfrage nach
        der Spalte \identifier{Konto\_ID}.
        \begin{msoraclesql}[\FALSE]
        \end{msoraclesql}
        \begin{lstlisting}[language=oracle_sql]
SELECT   ekk.Konto_ID, COUNT(*)
FROM     EigenkundeKonto ekk INNER JOIN Girokonto g
           ON (ekk.Konto_ID = g.Konto_ID)
         INNER JOIN Buchung b ON (g.Konto_ID = b.Konto_ID)
WHERE    b.Betrag > 10000
GROUP BY ekk.Konto_ID
ORDER BY 1;
        \end{lstlisting}
        \item Schreiben Sie eine Abfrage, die alle Mitarbeiter anzeigt, deren
        Gehalt um mehr als 4.000 EUR niedriger ist, als das Durchschnittsgehalt
        aller Mitarbeiter.
        \begin{msoraclesql}[\FALSE]
        \end{msoraclesql}
        \begin{lstlisting}[language=oracle_sql]
SELECT   m.Vorname, m.Nachname, m.Gehalt
FROM     Mitarbeiter m, Mitarbeiter v
GROUP BY m.Mitarbeiter_ID, m.Vorname, m.Nachname, m.Gehalt
HAVING   (m.Gehalt + 4000) < AVG(v.Gehalt);
        \end{lstlisting}
        \item Schreiben Sie eine Abfrage, die alle Mitarbeiter anzeigt, die
        h\"ochstens zwei Jahre \"alter sind, als der j\"ungste Mitarbeiter in
        deren Bankfiliale! \begin{oraclesql}[\FALSE]
        \end{oraclesql}
        \begin{lstlisting}[language=oracle_sql]
SELECT   m.Vorname, m.Nachname, m.Geburtsdatum,
			   MAX(a.Geburtsdatum) AS "JUENGSTER MITARBEITER"
FROM     Mitarbeiter m INNER JOIN Mitarbeiter a
           ON (m.Bankfiliale_ID = a.Bankfiliale_ID)
GROUP BY m.Mitarbeiter_ID, m.Vorname, m.Nachname, m.Geburtsdatum
HAVING   m.Geburtsdatum BETWEEN MAX(a.Geburtsdatum) - INTERVAL '2' YEAR AND
         MAX(a.Geburtsdatum);
        \end{lstlisting}
        \begin{mssql}[\FALSE]
        \end{mssql}
        \begin{lstlisting}[language=ms_sql]
SELECT   m.Vorname, m.Nachname, m.Geburtsdatum,
				 MAX(a.Geburtsdatum) AS "JUENGSTER MITARBEITER"
FROM     Mitarbeiter m INNER JOIN Mitarbeiter a
           ON (m.Bankfiliale_ID = a.Bankfiliale_ID)
GROUP BY m.Mitarbeiter_ID, m.Vorname, m.Nachname, m.Geburtsdatum
HAVING   m.Geburtsdatum BETWEEN DATEADD(YEAR, -2, MAX(a.Geburtsdatum)) AND
         MAX(a.Geburtsdatum);
        \end{lstlisting}
\clearpage
        \item Schreiben Sie eine Abfrage, die zu jedem Filialleiter, das Gehalt
        seines am schlechtesten bezahlten Mitarbeiters anzeigt. Sortieren Sie
        die Abfrage nach den Bankfilial-IDs der Filialleiter.
        \begin{msoraclesql}[\FALSE]
        \end{msoraclesql}
        \begin{lstlisting}[language=oracle_sql]
SELECT   v.Vorname, v.Nachname, v.Gehalt, 
         MIN(m.Gehalt) AS "Kleinstes Gehalt"
FROM     Mitarbeiter m INNER JOIN Mitarbeiter v
           ON (m.Vorgesetzter_ID = v.Mitarbeiter_ID)
WHERE    v.Bankfiliale_ID IS NOT NULL
GROUP BY v.Mitarbeiter_ID, v.Vorname, v.Nachname, v.Gehalt, v.Bankfiliale_ID
ORDER BY v.Bankfiliale_ID;
        \end{lstlisting}
      \end{enumerate}
\clearpage
