  \chapter{Gruppenfunktionen}
    \label{groupingfunctions}
    \setcounter{page}{1}\kapitelnummer{chapter}
    \minitoc
\newpage
      In vielen F\"allen ist es notwendig, die aus einer Abfrage resultierenden Datens\"atze nicht einzeln anzuzeigen, sondern sie nach bestimmten Kriterien zusammenzufassen. Dieser Vorgang wird als \enquote{gruppieren} bezeichnet und mittels der \GROUPBY-Klausel umgesetzt. Sie wird zwischen die beiden Klauseln \WHERE{} und \ORDERBY{} eingef\"ugt.
    \section{Die GROUP BY-Klausel}
      In einem ersten Beispiel werden aus der Tabelle \identifier{Mitarbeiter} die IDs aller Vorgesetzten angezeigt, so dass eine \enquote{Liste der Vorgesetzten} entsteht.
      \begin{lstlisting}[language=oracle_sql,caption={Die \enquote{Liste der Vorgesetzten}},label=sql05_01]
SELECT   Vorgesetzter_ID
FROM     Mitarbeiter
GROUP BY Vorgesetzter_ID
ORDER BY 1;
      \end{lstlisting}
      \begin{center}
        \begin{small}
          \changefont{pcr}{m}{n}
          \tablefirsthead {
            \multicolumn{1}{r}{\textbf{VORGESETZTER\_ID}} \\
            \cmidrule(r){1-1}
          }
          \tablehead{}
          \tabletail {
            \multicolumn{1}{l}{\textbf{28 Zeilen ausgew\"ahlt}} \\
          }
          \tablelasttail {
            \multicolumn{1}{l}{\textbf{28 Zeilen ausgew\"ahlt}} \\
          }
          \begin{msoraclesql}
            \begin{supertabular}{r}
              1 \\
              2 \\
              3 \\
              \dots \\
              27 \\
              NULL\\
              \\
            \end{supertabular}
          \end{msoraclesql}
        \end{small}
      \end{center}

      \bild{Gruppieren von Datens\"atzen}{group_by}{1}

       Mit Hilfe der \GROUPBY-Klausel werden die einzelnen IDs in Gruppen eingeteilt. Anschlie\ss{}end wird f\"ur jede Gruppe die ID genau einmal angezeigt.

       Statt einer einfachen Gruppierung, wie sie in \beispiel{sql05_01} erzeugt wurde, kann auch eine mehrfache Gruppierung erzeugt werden. Diese sind aber meist nur in Verbindung mit Aggregatfunktionen, die im folgenden Abschnitt behandelt werden, sinnvoll.
    \section{Die Aggregatfunktionen}
      Im Gegensatz zu den Single Row Functions, die sich immer nur auf eine Zeile auswirken und deshalb pro Zeile einmal ausgef\"uhrt werden m\"ussen, beziehen sich Aggregatfunktionen immer auf eine Gruppierung. Dies k\"onnen alle Werte einer Spalte oder mehrere getrennte Bereiche sein. Sinn und Zweck dieser Funktionen ist es, den Anwender dabei zu unterst\"utzen, vordefinierte Berechnungen durchzuf\"uhren. \tabelle{aggregatfunktionen} zeigt einen \"Uberblick, \"uber die wichtigsten Aggregatfunktionen.
      \begin{center}
        \tablecaption{Aggregatfunktionen}
        \label{aggregatfunktionen}
        \begin{small}
          \tablefirsthead{
            \multicolumn{1}{c}{\textbf{Aggregatfunktion}} &
            \multicolumn{1}{c}{\textbf{Bedeutung}} &
            \multicolumn{1}{c}{\textbf{Wertebereich}} \\
            \hline
          }
          \tabletail{
            \hline
          }
          \tablelasttail {
            \hline
          }
          \begin{supertabular}{|l|p{8.5cm}|p{2cm}|}
            AVG & Berechnet f\"ur den \"ubergebenen Bereich den Durchschnitt aller Werte. NULL-Werte werden bei der Berechnung nicht ber\"ucksichtigt. & Numerisch \\
            \hline
            COUNT & Z\"ahlt die zur Gruppierung geh\"orenden Datens\"atze. Als Funktionsargument kann ein beliebiger Ausdruck \"ubergeben werden. Wird ein Spaltenbezeichner verwendet, z\"ahlt die Funktion die Anzahl der Werte in dieser Spalte. NULL-Werte werden von dieser Funktion nicht ber\"ucksichtigt. & Universell \\
            \hline
            MAX & Liefert den gr\"o\ss ten Wert eines Bereiches zur\"uck. & Universell \\
            \hline
            MIN & Liefert den kleinsten Wert eines Bereiches zur\"uck. & Universell \\
            \hline
            SUM & Berechnet die Summe, f\"ur den \"ubergebenen Bereich. NULL-Werte werden bei der Berechnung nicht ber\"ucksichtigt. & Numerisch \\
          \end{supertabular}
        \end{small}
      \end{center}
      \beispiel{sql05_02} zeigt die Anwendung der Summen-Funktion \languageorasql{SUM}, um die Geh\"alter aller Mitarbeiter pro Filiale zu ermitteln.
      \begin{lstlisting}[language=oracle_sql,caption={Fehler in der Gruppierung},label=sql05_02]
SELECT Bankfiliale_ID, SUM(Gehalt)
FROM   Mitarbeiter;
      \end{lstlisting}
      Auch wenn auf den ersten Blick an dieser Abfrage nichts falsches zu bemerken ist, antwortet das DBMS mit einer Fehlermeldung, was daran liegt, dass die Funktion \languageorasql{SUM} automatisch eine Gruppierung bildet, in die die Spalte \identifier{Bankfiliale\_ID} nicht mit einbezogen wird.
      \begin{lstlisting}[language=oracle_sql,caption={Die Fehlermeldung in Oracle},label=sql05_03]
Fehler beim Start in Zeile 1 in Befehl:
SELECT Bankfiliale_ID, SUM(Gehalt)
FROM   Mitarbeiter
Fehler bei Befehlszeile:1 Spalte:7
Fehlerbericht:
SQL-Fehler: ORA-00937: keine Gruppenfunktion fuer Einzelgruppe
00937. 00000 -  "not a single-group group function"
      \end{lstlisting}
\clearpage
			\begin{lstlisting}[language=ms_sql,caption={Die Fehlermeldung in SQL Server},label=sql05_04]
Meldung 8120, Ebene 16, Status 1, Zeile 1
Die 'Bank.dbo.Bankfiliale_ID'-Spalte
ist in der Auswahlliste ungueltig, da sie nicht in einer
Aggregatfunktion und nicht in der GROUP BY-Klausel enthalten ist.
      \end{lstlisting}
      \begin{merke}
        Sobald eine Gruppenfunktion zum Einsatz kommt, m\"ussen alle in der \SELECT-Klausel gelisteten Attribute gruppiert werden. Dies kann durch die Anwendung weiterer Gruppenfunktionen oder durch die \GROUPBY-Klausel geschehen.
      \end{merke}
      Um diesen Fehler zu beheben, muss das Statement aus \beispiel{sql05_02} um eine \GROUPBY-Klausel erweitert werden.
      \begin{lstlisting}[language=oracle_sql,caption={Der korrekte Einsatz der \languageorasql{SUM}-Funktion},label=sql05_05]
SELECT   Bankfiliale_ID, SUM(Gehalt)
FROM     Mitarbeiter
GROUP BY Bankfiliale_ID;
      \end{lstlisting}
      \begin{center}
        \begin{small}
          \changefont{pcr}{m}{n}
          \tablefirsthead {
            \multicolumn{1}{r}{\textbf{BANKFILIALE\_ID}} &
            \multicolumn{1}{r}{\textbf{SUM(GEHALT)}} \\
            \cmidrule(r){1-1}\cmidrule(r){2-2}
          }
          \tablehead{}
          \tabletail {
            \multicolumn{2}{l}{\textbf{21 Zeilen ausgew\"ahlt}} \\
          }
          \tablelasttail {
            \multicolumn{2}{l}{\textbf{21 Zeilen ausgew\"ahlt}} \\
          }
          \begin{msoraclesql}
            \begin{supertabular}{rr}
              & 308000 \\
              1 & 20500 \\
              6 & 21000 \\
              11 & 23000 \\
              13 & 20000 \\
              2 & 17000 \\
              14 & 19500 \\
              20 & 19500 \\
           \end{supertabular}
          \end{msoraclesql}
        \end{small}
      \end{center}
        \subsection{Die Funktion COUNT}
          Wie in \tabelle{aggregatfunktionen} beschrieben, wird \languageorasql{COUNT} zum Z\"ahlen von Werten genutzt. In \beispiel{sql05_06} wird ermittelt, wie viele Mitarbeiter in der Tabelle \identifier{Mitarbeiter} gespeichert sind.
          \begin{lstlisting}[language=oracle_sql,caption={Das Z\"ahlen von Datens\"atzen},label=sql05_06]
SELECT COUNT(*) AS "Mitarbeiter"
FROM   Mitarbeiter;
          \end{lstlisting}
          \begin{center}
            \begin{small}
              \changefont{pcr}{m}{n}
              \tablefirsthead {
                \multicolumn{1}{r}{\textbf{MITARBEITER}} \\
                \cmidrule(r){1-1}
              }
              \tablehead{}
              \tabletail {
                \multicolumn{1}{l}{\textbf{1 Zeile ausgew\"ahlt}} \\
              }
              \tablelasttail {
                \multicolumn{1}{l}{\textbf{1 Zeile ausgew\"ahlt}} \\
              }
              \begin{msoraclesql}
                \begin{supertabular}{r}
                  100 \\
                \end{supertabular}
              \end{msoraclesql}
            \end{small}
          \end{center}
          \begin{merke}
            Der Stern * kann in der \languageorasql{COUNT}-Funktion als \enquote{Joker} genutzt werden. \languageorasql{COUNT(*)} z\"ahlt alle Zeilen einer Tabelle und hat somit den gleichen Effekt, wie das Z\"ahlen der Eintr\"age der Prim\"arschl\"usselspalte einer Tabelle.
          \end{merke}
          Ein weiteres Beispiel zeigt, dass die \languageorasql{COUNT}-Funktion NULL-Werte nicht ber\"ucksichtigt.
          \begin{lstlisting}[language=oracle_sql,caption={NULL-Werte werden nicht gez\"ahlt!},label=sql05_07]
SELECT COUNT(Vorgesetzter_ID)
FROM   Mitarbeiter;
          \end{lstlisting}
          \begin{center}
            \begin{small}
              \changefont{pcr}{m}{n}
              \tablefirsthead {
                \multicolumn{1}{r}{\textbf{COUNT(VORGESETZTER\_ID)}} \\
                \cmidrule(r){1-1}
              }
              \tablehead{}
              \tabletail {
                \multicolumn{1}{l}{\textbf{1 Zeile ausgew\"ahlt}} \\
              }
              \tablelasttail {
                \multicolumn{1}{l}{\textbf{1 Zeile ausgew\"ahlt}} \\
              }
              \begin{msoraclesql}
                \begin{supertabular}{r}
                  99 \\
                \end{supertabular}
              \end{msoraclesql}
            \end{small}
          \end{center}
          Da der Mitarbeiter Winter (\identifier{Mitarbeiter\_ID} = 1) keinen Vorgesetzten hat, ist bei ihm ein NULL-Wert in der Spalte \identifier{Vorgesetzter\_ID}, was dazu f\"uhrt, dass \languageorasql{COUNT} nur 99 Werte z\"ahlt.
      \subsection{Die Funktion SUM}
        Mit Hilfe von \languageorasql{SUM} kann die Summe aller Werte eines Bereichs gebildet werden. Ein Beispiel zu dieser Funktion ist in \beispiel{sql05_05} zu sehen. Im Gegensatz zur \languageorasql{COUNT}-Funktion, spielen NULL-Werte keine Rolle, in Bezug auf die \languageorasql{SUM}-Funktion. Ein NULL-Wert wird durch die \languageorasql{SUM}-Funktion einfach ignoriert und verf\"alscht das Ergebnis dadurch nicht.
      \subsection{Die Funktion AVG}
        Die Abk\"urzung \enquote{AVG} steht f\"ur das englische Wort \enquote{average} = Durchschnitt (arithmetisches Mittel). Die Funktion \languageorasql{AVG} berechnet den Durchschnitt der Werte einer Gruppierung.

        In \beispiel{sql05_08} wird die \languageorasql{AVG}-Funktion genutzt, um das Durchschnittsgehalt f\"ur jede Filiale zu berechnen.
        \begin{lstlisting}[language=oracle_sql,caption={Die AVG-Funktion},label=sql05_08]
SELECT   Bankfiliale_ID, AVG(Gehalt)
FROM     Mitarbeiter
GROUP BY Bankfiliale_ID;
        \end{lstlisting}
\clearpage
        \begin{center}
          \begin{small}
            \changefont{pcr}{m}{n}
            \tablefirsthead {
              \multicolumn{1}{r}{\textbf{BANKFILIALE\_ID}} &
              \multicolumn{1}{r}{\textbf{AVG(GEHALT)}} \\
              \cmidrule(r){1-1}\cmidrule(r){2-2}
            }
            \tablehead{}
            \tabletail {
              \multicolumn{2}{l}{\textbf{21 Zeilen ausgew\"ahlt}} \\
            }
            \tablelasttail {
              \multicolumn{2}{l}{\textbf{21 Zeilen ausgew\"ahlt}} \\
            }
            \begin{msoraclesql}
              \begin{supertabular}{rr}
                & 44000 \\
                1 & 4100 \\
                6 & 4200 \\
                11 & 4600 \\
                13 & 5000 \\
                2 & 3400 \\
                14 & 4875 \\
                20 & 4875 \\
                4 & 4100 \\
              \end{supertabular}
            \end{msoraclesql}
          \end{small}
        \end{center}
        Das n\"achste Beispiel erl\"autert den Zusammenhang zwischen \languageorasql{AVG} und NULL-Werten. Im Versuch soll die durchschnittliche Provision, die ein Mitarbeiter erh\"alt, berechnet werden. Wichtig f\"ur diesen Versuch ist, dass nur ein Teil der Mitarbeiter eine Provision erh\"alt.
        \begin{lstlisting}[language=oracle_sql,caption={AVG und NULL-Werte in Oracle},label=sql05_09]
SELECT COUNT(Provision) AS Anzahl, ROUND(AVG(Provision), 2) AS "AVG",
       COUNT(NVL(Provision, 0)) AS "Anzahl NVL",
       AVG(NVL(Provision, 0)) AS "AVG NVL"
FROM   Mitarbeiter;
        \end{lstlisting}
        \begin{center}
          \begin{small}
            \changefont{pcr}{m}{n}
            \tablefirsthead {
              \multicolumn{1}{r}{\textbf{ANZAHL}} &
              \multicolumn{1}{r}{\textbf{AVG}} &
              \multicolumn{1}{r}{\textbf{ANZAHL NVL}} &
              \multicolumn{1}{r}{\textbf{AVG NVL}} \\
              \cmidrule(r){1-1}\cmidrule(r){2-2}\cmidrule(r){3-3}\cmidrule(r){4-4}
            }
            \tablehead{}
            \tabletail {
              \multicolumn{4}{l}{\textbf{1 Zeile ausgew\"ahlt}} \\
            }
            \tablelasttail {
              \multicolumn{4}{l}{\textbf{1 Zeile ausgew\"ahlt}} \\
            }

            \begin{oraclesql}
              \begin{supertabular}{rrrr}
                33 & 22,58 & 100 & 7,45 \\
              \end{supertabular}
            \end{oraclesql}
          \end{small}
        \end{center}
        \begin{lstlisting}[language=ms_sql,caption={AVG und NULL-Werte im MS SQL Server},label=sql05_10]
SELECT COUNT(Provision) AS Anzahl, ROUND(AVG(Provision), 2) AS "AVG",
       COUNT(ISNULL(Provision, 0)) AS "Anzahl ISNULL",
       AVG(ISNULL(Provision, 0)) AS "AVG ISNULL"
FROM   Mitarbeiter;
        \end{lstlisting}
        \begin{center}
          \begin{small}
            \changefont{pcr}{m}{n}
            \tablefirsthead {
              \multicolumn{1}{l}{\textbf{ANZAHL}} &
              \multicolumn{1}{l}{\textbf{AVG}} &
              \multicolumn{1}{l}{\textbf{ANZAHL NVL}} &
              \multicolumn{1}{l}{\textbf{AVG NVL}} \\
              \cmidrule(l){1-1}\cmidrule(l){2-2}\cmidrule(l){3-3}\cmidrule(l){4-4}
            }
            \tablehead{}
            \tabletail {
              \multicolumn{4}{l}{\textbf{1 Zeile ausgew\"ahlt}} \\
            }
            \tablelasttail {
              \multicolumn{4}{l}{\textbf{1 Zeile ausgew\"ahlt}} \\
            }

            \begin{mssql}
              \begin{supertabular}{llll}
                33 & 22,58 & 100 & 7,45 \\
              \end{supertabular}
            \end{mssql}
          \end{small}
        \end{center}
        Je nach dem, ob NULL-Werte durch die \languageorasql{NVL}-Funktion/\languagemssql{ISNULL}-Funktion bereinigt werden oder nicht, wird ein unterschiedliches Ergebnis, durch die \languageorasql{AVG}-Funktion, errechnet.
      \subsection{Die Funktionen MIN und MAX}
        Die Funktionen \languageorasql{MIN} und \languageorasql{MAX} ermitteln den gr\"o\ss{}ten bzw. kleinsten Wert aus einer Menge und k\"onnen auf nahezu jeden Datentyp angewendet werden.  \beispiel{sql05_11} zeigt die Anwendung von \languageorasql{MAX} auf Spalten verschiedener Datentypen.
        \begin{lstlisting}[language=oracle_sql,caption={Anwendung von MAX auf verschiedene Datentypen},label=sql05_11]
SELECT MAX(Nachname), MAX(Geburtsdatum), MAX(PLZ), MAX(Provision)
FROM   Mitarbeiter;
        \end{lstlisting}
        \begin{center}
          \begin{small}
            \changefont{pcr}{m}{n}
            \tablefirsthead {
              \multicolumn{1}{l}{\textbf{MAX(NACHNAME)}} &
              \multicolumn{1}{l}{\textbf{MAX(GEBURTSDATUM)}} &
              \multicolumn{1}{l}{\textbf{MAX(PLZ)}} &
              \multicolumn{1}{r}{\textbf{MAX(PROVISION)}} \\
              \cmidrule(l){1-1}\cmidrule(l){2-2}\cmidrule(l){3-3}\cmidrule(r){4-4}
            }
            \tablehead{}
            \tabletail {
              \multicolumn{4}{l}{\textbf{1 Zeile ausgew\"ahlt}} \\
            }
            \tablelasttail {
              \multicolumn{4}{l}{\textbf{1 Zeile ausgew\"ahlt}} \\
            }

            \begin{msoraclesql}
              \begin{supertabular}{lllr}
                Zimmermann & 31.05.93 & 80995 & 30 \\
              \end{supertabular}
            \end{msoraclesql}
          \end{small}
        \end{center}
        Zu beachten ist, dass die angezeigten Daten in keiner Beziehung zueinander stehen. Es handelt sich um die Maximalwerte aus den einzelnen Spalten. Der Angestellte Zimmermann hat keinesfalls das Geburtsdatum 31.05.1993 und er bekommt auch keine Provision.

        Bez\"uglich NULL-Werte gibt es, sowohl in Oracle, als auch in MS SQL Server, keine Probleme mit \languageorasql{MIN} oder \languageorasql{MAX}, wie das folgende \beispiel{sql05_12} zeigt.
        \begin{lstlisting}[language=oracle_sql,caption={Die MAX-Funktion und NULL-Werte (Oracle)},label=sql05_12]
SELECT MAX(Provision), MAX(NVL(Provision,0))
FROM   Mitarbeiter;
        \end{lstlisting}
        \begin{center}
          \begin{small}
            \changefont{pcr}{m}{n}
            \tablefirsthead {
              \multicolumn{1}{r}{\textbf{MAX(PROVISION)}} &
              \multicolumn{1}{r}{\textbf{MAX(NVL(PROVISION,0))}} \\
              \cmidrule(r){1-1}\cmidrule(r){2-2}
            }
            \tablehead{}
            \tabletail {
              \multicolumn{2}{l}{\textbf{1 Zeile ausgew\"ahlt}} \\
            }
            \tablelasttail {
              \multicolumn{2}{l}{\textbf{1 Zeile ausgew\"ahlt}} \\
            }
            \begin{oraclesql}
              \begin{supertabular}{rr}
                30 & 30 \\
              \end{supertabular}
            \end{oraclesql}
          \end{small}
        \end{center}
        Das gleiche Beispiel kann in MS SQL Server, mit Hilfe der \languagemssql{ISNULL}-Funktion reproduziert werden.
        \begin{merke}
          Alle Eigenschaften, die f\"ur die \languageorasql{MAX}-Funktion gelten, gelten uneingeschr\"ankt auch f\"ur die \languageorasql{MIN}-Funktion.
        \end{merke}
      \subsection{Gruppierungen mit mehreren Ebenen}
        Wie bereits angek\"undigt, ist es m\"oglich, mit Hilfe der \GROUPBY-Klausel, eine Abfrage mehrfach zu gruppieren. Eine Mehrfachgruppierung ist immer dann notwendig, wenn innerhalb einer Gruppe weitere Gruppen gebildet werden m\"ussen. \beispiel{sql05_13} listet alle Kunden auf, die nach dem 01.01.1995 geboren wurden, gruppiert nach Ort und Strasse.
        \begin{lstlisting}[language=oracle_sql,caption={Eine Gruppierung mit mehreren Ebenen},label=sql05_13]
SELECT Ort, Strasse, COUNT(*) AS Anzahl
FROM   Kunde k INNER JOIN Eigenkunde ek
         ON (k.Kunden_ID = ek.Kunden_ID)
WHERE  Geburtsdatum > '01.01.1995'
GROUP BY Ort, Strasse
ORDER BY Ort;
        \end{lstlisting}
        \begin{center}
          \begin{small}
            \changefont{pcr}{m}{n}
            \tablefirsthead {
              \multicolumn{1}{l}{\textbf{ORT}} &
              \multicolumn{1}{l}{\textbf{STRASSE}} &
              \multicolumn{1}{r}{\textbf{ANZAHL}} \\
              \cmidrule(l){1-1}\cmidrule(l){2-2}\cmidrule(r){3-3}
            }
            \tablehead{}
            \tabletail {
              \multicolumn{3}{l}{\textbf{21 Zeilen ausgew\"ahlt}} \\
            }
            \tablelasttail {
              \multicolumn{3}{l}{\textbf{21 Zeilen ausgew\"ahlt}} \\
            }

            \begin{msoraclesql}
              \begin{supertabular}{llr}
                Aschersleben & Am Markt & 1 \\
                B\"ordeaue & Plutoweg & 1 \\
                B\"ordeaue & Okerstra\ss{}e & 1 \\
                \dots & \dots & \dots \\
                Hecklingen & Turmstra\ss{}e & 1 \\
                Hecklingen & Pestalozzistra\ss{}e & 1 \\
                Hecklingen & Seestra\ss{}e & 1 \\
                \dots & \dots & \dots \\
                Sta\ss{}furt & Wielandstra\ss{}e & 2 \\
              \end{supertabular}
            \end{msoraclesql}
          \end{small}
        \end{center}
    \section{Gruppierte Abfragen filtern}
      \subsection{Die WHERE-Klausel}
        Das die \WHERE-Klausel auch auf gruppierte Abfragen angewandt werden
        kann, ist bereits in \beispiel{sql05_13} zu sehen. Wesentlich dabei
        ist, dass sie vor dem Gruppieren abgearbeitet wird, d. h. es wird die
        Menge der Zeilen eingeschr\"ankt, die noch gruppiert werden muss. Hierzu
        ein Beispiel: Mit Hilfe einer Abfrage soll ermittelt werden, wer der
        j\"ungste Kunde ist.
        \begin{lstlisting}[language=oracle_sql,caption={Wer ist der j\"ungste Kunde},label=sql05_14]
SELECT MAX(Geburtsdatum)
FROM   Eigenkunde;
        \end{lstlisting}
        \begin{center}
          \begin{small}
            \changefont{pcr}{m}{n}
            \tablefirsthead {
              \multicolumn{1}{l}{\textbf{MAX(GEBURTSDATUM)}} \\
              \cmidrule(l){1-1}
            }
            \tablehead{}
            \tabletail {
              \multicolumn{1}{l}{\textbf{1 Zeile ausgew\"ahlt}} \\
            }
            \tablelasttail {
              \multicolumn{1}{l}{\textbf{1 Zeile ausgew\"ahlt}} \\
            }
            \begin{msoraclesql}
              \begin{supertabular}{l}
                07.04.97 \\
              \end{supertabular}
            \end{msoraclesql}
          \end{small}
        \end{center}
\clearpage
        Im Folgenden wird \beispiel{sql05_14} durch eine \WHERE-Klausel eingeschr\"ankt.
        \begin{lstlisting}[language=oracle_sql,caption={Der j\"ungest Kunde aus Alsleben},label=sql05_15]
SELECT MAX(Geburtsdatum)
FROM   Eigenkunde
WHERE  Ort LIKE 'Alsleben';
        \end{lstlisting}
        \begin{center}
          \begin{small}
            \changefont{pcr}{m}{n}
            \tablefirsthead {
              \multicolumn{1}{l}{\textbf{MAX(GEBURTSDATUM)}} \\
              \cmidrule(l){1-1}
            }
            \tablehead{}
            \tabletail {
              \multicolumn{1}{l}{\textbf{1 Zeile ausgew\"ahlt}} \\
            }
            \tablelasttail {
              \multicolumn{1}{l}{\textbf{1 Zeile ausgew\"ahlt}} \\
            }
            \begin{msoraclesql}
              \begin{supertabular}{l}
                21.05.93 \\
              \end{supertabular}
            \end{msoraclesql}
          \end{small}
        \end{center}
        Die angef\"ugte \WHERE-Klausel sorgt daf\"ur, dass die Gruppierung, die durch die \languageorasql{MAX}-Funktion entsteht, nur auf die Kunden angewandt wird, die in Alsleben wohnen.
        \begin{merke}
          Die \WHERE-Klausel wird immer vor dem Gruppieren abgearbeitet!
        \end{merke}
      \subsection{Die HAVING-Klausel}
        Gerade eben wurde gezeigt, dass mit Hilfe der \WHERE-Klausel eine Selektion vor der Gruppierung der Datens\"atze erreicht werden kann. Was aber ist, wenn eine Auswahl auf gruppierten Datens\"atzen erfolgen soll? Im folgenden Versuch soll f\"ur jede Bankfiliale das niedrigste Gehalt aufgelistet werden, aber nur dann, wenn es gr\"o\ss{}er als 1.500 EUR ist.
        \begin{lstlisting}[language=oracle_sql,caption={Ein Versuch\dots mit Oracle},label=sql05_16]
SELECT   Bankfiliale_ID, MIN(Gehalt)
FROM     Mitarbeiter
WHERE    MIN(Gehalt) > 1500
GROUP BY Bankfiliale_ID;

ORA-00934: Gruppenfunktion ist hier nicht zul\"assig
00934. 00000 -  "group function is not allowed here"
        \end{lstlisting}
        \begin{lstlisting}[language=ms_sql,caption={Der gleiche Versuch\dots mit MS SQL Server},label=sql05_17]
SELECT   Bankfiliale_ID, MIN(Gehalt)
FROM     Mitarbeiter
WHERE    MIN(Gehalt) > 1500
GROUP BY Bankfiliale_ID;

Meldung 147, Ebene 15, Status 1, Zeile 3
In der WHERE-Klausel darf kein Aggrgat auftreten, es sei denn, es befindet sich
in einer Unterabfrage, die in einer HAVING-Klausel oder einer Auswahlliste
enthalten ist und die Spalte, die aggregiert wird, ist ein \"au\ss{}erer
Verweise.
        \end{lstlisting}
        \beispiel{sql05_16} und \beispiel{sql05_17} zeigen, dass die Verarbeitung einer Aggregatfunktion in der \WHERE-Klausel nicht m\"oglich ist. Dies liegt daran, dass, wie bereits erw\"ahnt, die \WHERE-Klausel schon vor der Gruppierungsphase abgearbeitet wird.

        Um das gew\"unschte Ziel erreichen zu k\"onnen, muss eine neue Klausel, die \HAVING-Klausel, eingef\"uhrt werden. Sie erm\"oglicht es, Selektionen auf gruppierten Zeilen durchzuf\"uhren. \beispiel{sql05_16} muss korrekt lauten:
        \begin{lstlisting}[language=oracle_sql,caption={Die HAVING-Klausel},label=sql05_18]
SELECT   Bankfiliale_ID, MIN(Gehalt)
FROM     Mitarbeiter
GROUP BY Bankfiliale_ID
HAVING   MIN(Gehalt) > 1500;
        \end{lstlisting}
        \begin{center}
          \begin{small}
            \changefont{pcr}{m}{n}
            \tablefirsthead {
              \multicolumn{1}{r}{\textbf{BANKFILIALE\_ID}} &
              \multicolumn{1}{r}{\textbf{MIN(GEHALT)}} \\
              \cmidrule(r){1-1}\cmidrule(r){2-2}
            }
            \tablehead{}
            \tabletail {
              \multicolumn{2}{l}{\textbf{9 Zeilen ausgew\"ahlt}} \\
            }
            \tablelasttail {
              \multicolumn{2}{l}{\textbf{9 Zeilen ausgew\"ahlt}} \\
            }

            \begin{msoraclesql}
              \begin{supertabular}{rr}
                & 30000 \\
                1 & 2000 \\
                6 & 2000 \\
                11 & 2000 \\
                7 & 2000 \\
                18 & 2500 \\
                15 & 2000 \\
                16 & 3000 \\
                19 & 2000 \\
              \end{supertabular}
            \end{msoraclesql}
          \end{small}
        \end{center}
        Die \HAVING-Klausel eliminiert, nach dem Gruppieren, alle Datens\"atze, auf die die Bedingung zutrifft: \languageorasql{MIN(Gehalt) <= 1500}.
        \begin{merke}
          Die \HAVING-Klausel wird auf gruppierte Zeilen angewandt und kann deshalb nur in Verbindung mit der \GROUPBY-Klausel stehen.
        \end{merke}
\clearpage
    \section{Die Abarbeitungsreihenfolge des SELECT-Statements}
      Nachdem nun in den vorangegangenen Kapiteln alle standardisierten Klauseln des \SELECT-Kommandos behandelt wurden, stellt sich noch immer die Frage: \enquote{In welcher Reihenfolge werden die Klauseln des \SELECT-Kommandos abgearbeitet?}. Die korrekte Antwort lautet:
      \begin{enumerate}
        \item \FROM
        \item \WHERE
        \item \GROUPBY
        \item \HAVING
        \item \SELECT
        \item \ORDERBY
      \end{enumerate}
      Zuerst wird mit Hilfe der \FROM-Klausel ermittelt, auf welche Tabellen sich das \SELECT-Statement bezieht. Im zweiten Schritt filtert die \WHERE-Klausel alle Zeilen aus den Quelltabellen, die f\"ur das Statement nicht mehr relevant sind. Die beiden Klauseln \GROUPBY{} und \HAVING{} sorgen f\"ur Gruppierungen und das Filtern von gruppierten Zeilen. Zu guter letzt werden die \SELECT- und die \ORDERBY-Klausel abgearbeitet, so dass das Ergebnis auf dem Bildschirm ausgegeben werden kann.
