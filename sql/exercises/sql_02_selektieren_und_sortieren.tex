\clearpage
    \section{Übungen - Selektieren und Sortieren}
      \begin{enumerate}
        \item Erstellen Sie eine Abfrage, die die Konto\_ID und das aktuelle
        Guthaben des Girokontos der Bankkunden anzeigt, die weniger als 1000 EUR
        Guthaben besitzen.
        \begin{center}
          \begin{small}
            \changefont{pcr}{m}{n}
            \tablefirsthead {
              \multicolumn{1}{r}{\textbf{KONTO\_ID}} &
              \multicolumn{1}{r}{\textbf{GUTHABEN}} \\
              \cmidrule(r){1-1}\cmidrule(r){2-2}
            }
            \tablehead{}
            \tabletail {
              \multicolumn{2}{l}{\textbf{55 Zeilen ausgewählt}} \\
            }
            \tablelasttail {
              \multicolumn{2}{l}{\textbf{55 Zeilen ausgewählt}} \\
            }
            \begin{msoraclesql}
              \begin{supertabular}{rr}
                15 & -10496,4 \\
                16 & -54593,2 \\
                43 & -42144,1 \\
                48 & -140505,1 \\
                57 & -1088,4 \\
                59 & 760,1 \\
                83 & 336,2 \\
                87 & -9009,1 \\
                99 & -69705,6 \\
              \end{supertabular}
            \end{msoraclesql}
          \end{small}
        \end{center}
        \item Erstellen Sie eine Abfrage, die die Mitarbeiter\_ID und den
        Nachnamen der Mitarbeiter mit der Vorgesetzter\_ID \enquote{2} anzeigt.
        \begin{center}
          \begin{small}
            \changefont{pcr}{m}{n}
            \tablefirsthead {
              \multicolumn{1}{r}{\textbf{MITARBEITER\_ID}} &
              \multicolumn{1}{l}{\textbf{NACHNAME}} \\
              \cmidrule(r){1-1}\cmidrule(l){2-2}
            }
            \tablehead{}
            \tabletail {
              \multicolumn{2}{l}{\textbf{2 Zeilen ausgewählt}} \\
            }
            \tablelasttail {
              \multicolumn{2}{l}{\textbf{2 Zeilen ausgewählt}} \\
            }
            \begin{msoraclesql}
              \begin{supertabular}{rl}
                4 & Schwarz \\
                5 & Sindermann \\
              \end{supertabular}
            \end{msoraclesql}
          \end{small}
        \end{center}
        \item Erstellen Sie eine Abfrage, die die Konto\_ID und das aktuelle
        Guthaben des Girokontos der Bankkunden anzeigt, deren Guthaben nicht
        zwischen 1000 EUR und 1500 EUR liegt.
        \begin{center}
          \begin{small}
            \changefont{pcr}{m}{n}
            \tablefirsthead {
              \multicolumn{1}{r}{\textbf{KONTO\_ID}} &
              \multicolumn{1}{r}{\textbf{GUTHABEN}} \\
              \cmidrule(r){1-1}\cmidrule(r){2-2}
            }
            \tablehead{}
            \tabletail {
%               \multicolumn{2}{l}{\textbf{428 Zeilen ausgewählt}} \\
            }
            \tablelasttail {
              \multicolumn{2}{l}{\textbf{428 Zeilen ausgewählt}} \\
            }
            \begin{msoraclesql}
              \begin{supertabular}{rr}
                1 & 111316,9 \\
                2 & 96340,2 \\
                3 & 59633 \\
                5 & 98449 \\
                6 & 26130,7 \\
                7 & 23128,7 \\
                9 & 8857,6 \\
                10 & 68001,3 \\
              \end{supertabular}
            \end{msoraclesql}
          \end{small}
        \end{center}
\clearpage
        \item Lassen Sie sich die Kunden\_ID und das Geburtsdatum aller
        Eigenkunden anzeigen, die zwischen dem 20. Februar 1980 und dem 02.
        März 1988 geboren sind. Zusätzlich soll die Abfrage nach
        dem Geburtsdatum in aufsteigender Reihenfolge sortiert werden.
        \begin{center}
          \begin{small}
            \changefont{pcr}{m}{n}
            \tablefirsthead {
              \multicolumn{1}{r}{\textbf{KUNDEN\_ID}} &
              \multicolumn{1}{l}{\textbf{GEBURTSDATUM}} \\
              \cmidrule(r){1-1}\cmidrule(l){2-2}
            }
            \tablehead{}
            \tabletail {
              \multicolumn{2}{l}{\textbf{124 Zeilen ausgewählt}} \\
            }
            \tablelasttail {
              \multicolumn{2}{l}{\textbf{124 Zeilen ausgewählt}} \\
            }
            \begin{msoraclesql}
              \begin{supertabular}{rl}
                391 & 01.03.80 \\
                387 & 03.03.80 \\
                339 & 22.03.80 \\
                75 & 22.03.80 \\
                458 & 07.05.80 \\
                50 & 21.05.80 \\
              \end{supertabular}
            \end{msoraclesql}
          \end{small}
        \end{center}
        \item Zeigen Sie, in alphabetischer Reihenfolge, die Mitarbeiter\_ID und
        den Nachnamen der Mitarbeiter an, die die Vorgesetzter\_ID \enquote{5}
        oder \enquote{6} haben.
        \begin{center}
          \begin{small}
            \changefont{pcr}{m}{n}
            \tablefirsthead {
              \multicolumn{1}{r}{\textbf{MITARBEITER\_ID}} &
              \multicolumn{1}{l}{\textbf{NACHNAME}} \\
              \cmidrule(r){1-1}\cmidrule(l){2-2}
            }
            \tablehead{}
            \tabletail {
              \multicolumn{2}{l}{\textbf{10 Zeilen ausgewählt}} \\
            }
            \tablelasttail {
              \multicolumn{2}{l}{\textbf{10 Zeilen ausgewählt}} \\
            }
            \begin{msoraclesql}
              \begin{supertabular}{rl}
                17 & Becker \\
                16 & Berger \\
                20 & Große \\
                13 & Kaiser \\
                18 & Köhler \\
                14 & Lorenz \\
                22 & Rollert \\
              \end{supertabular}
            \end{msoraclesql}
          \end{small}
        \end{center}
        \item Erstellen Sie eine Abfrage, die den Nachnamen und die
        Bankfiliale\_ID der Mitarbeiter ausgibt, die die Vorgesetzten\_ID
        \enquote{5} oder \enquote{6} haben und deren Bankfiliale\_ID zwischen
        \enquote{10} und \enquote{20} ist. Die Spalten sollen mit
        \enquote{Mitarbeiter} und \enquote{Bankfiliale} benannt werden.
        \begin{center}
          \begin{small}
            \changefont{pcr}{m}{n}
            \tablefirsthead {
              \multicolumn{1}{l}{\textbf{MITARBEITER}} &
              \multicolumn{1}{r}{\textbf{BANKFILIALE}} \\
              \cmidrule(l){1-1}\cmidrule(l){2-2}
            }
            \tablehead{}
            \tabletail {
%               \multicolumn{2}{l}{\textbf{6 Zeilen ausgewählt}} \\
            }
            \tablelasttail {
              \multicolumn{2}{l}{\textbf{6 Zeilen ausgewählt}} \\
            }
            \begin{msoraclesql}
              \begin{supertabular}{lr}
                Becker & 10 \\
                Köhler & 11 \\
                Weber & 12 \\
                Große & 13 \\
                Walther & 14 \\
              \end{supertabular}
            \end{msoraclesql}
          \end{small}
        \end{center}
\clearpage
        \item Zeigen Sie die Mitarbeiter\_ID und den Nachnamen des Mitarbeiters
        an, der keinen Vorgesetzten hat.
        \begin{center}
          \begin{small}
            \changefont{pcr}{m}{n}
            \tablefirsthead {
              \multicolumn{1}{r}{\textbf{MITARBEITER\_ID}} &
              \multicolumn{1}{l}{\textbf{NACHNAME}} \\
              \cmidrule(r){1-1}\cmidrule(l){2-2}
            }
            \tablehead{}
            \tabletail {
              \multicolumn{2}{l}{\textbf{1 Zeile ausgewählt}} \\
            }
            \tablelasttail {
              \multicolumn{2}{l}{\textbf{1 Zeile ausgewählt}} \\
            }
            \begin{msoraclesql}
              \begin{supertabular}{rl}
                1 & Winter \\
              \end{supertabular}
            \end{msoraclesql}
          \end{small}
        \end{center}
        \item Zeigen Sie die Kunden\_ID und das Geburtsdatum derjenigen Eigenkunden an, die im Jahre 1980 geboren sind.
        \begin{center}
          \begin{small}
            \changefont{pcr}{m}{n}
            \tablefirsthead {
              \multicolumn{1}{r}{\textbf{KUNDEN\_ID}} &
              \multicolumn{1}{l}{\textbf{GEBURTSDATUM}} \\
              \cmidrule(r){1-1}\cmidrule(l){2-2}
            }
            \tablehead{}
            \tabletail {
              \multicolumn{2}{l}{\textbf{14 Zeilen ausgewählt}} \\
            }
            \tablelasttail {
              \multicolumn{2}{l}{\textbf{14 Zeilen ausgewählt}} \\
            }
            \begin{msoraclesql}
              \begin{supertabular}{rl}
                387 & 03.03.80 \\
                538 & 22.01.80 \\
                161 & 17.08.80 \\
                254 & 12.09.80 \\
              \end{supertabular}
            \end{msoraclesql}
          \end{small}
        \end{center}
        \item Erstellen Sie eine Abfrage, die den Nachnamen, das Gehalt und die
        Provision für alle Mitarbeiter anzeigt, die eine Provision erhalten.
        Sortieren Sie die Ausgabe in absteigender Reihenfolge nach dem Gehalt.
        \begin{center}
          \begin{small}
            \changefont{pcr}{m}{n}
            \tablefirsthead {
              \multicolumn{1}{l}{\textbf{NACHNAME}} &
              \multicolumn{1}{r}{\textbf{GEHALT}} &
              \multicolumn{1}{r}{\textbf{PROVISION}} \\
              \cmidrule(l){1-1}\cmidrule(r){2-2}\cmidrule(r){3-3}
            }
            \tablehead{}
            \tabletail {
              \multicolumn{3}{l}{\textbf{33 Zeilen ausgewählt}} \\
            }
            \tablelasttail {
              \multicolumn{3}{l}{\textbf{33 Zeilen ausgewählt}} \\
            }
            \begin{msoraclesql}
              \begin{supertabular}{lrr}
                Hartmann & 4000 & 30 \\
                Roth & 3500 & 20 \\
                Walther & 3500 & 20 \\
                Wagner & 3500 & 20 \\
                Zimmermann & 3500 & 30 \\
              \end{supertabular}
            \end{msoraclesql}
          \end{small}
        \end{center}
        \item Zeigen Sie die Nachnamen aller Mitarbeiter an, in deren Nachname
        an dritter Stelle ein \enquote{a} vorkommt.
        \begin{center}
          \begin{small}
            \changefont{pcr}{m}{n}
            \tablefirsthead {
              \multicolumn{1}{l}{\textbf{NACHNAME}} \\
              \cmidrule(l){1-1}
            }
            \tablehead{}
            \tabletail {
%               \multicolumn{1}{l}{\textbf{4 Zeilen ausgewählt}} \\
            }
            \tablelasttail {
              \multicolumn{1}{l}{\textbf{4 Zeilen ausgewählt}} \\
            }
            \begin{msoraclesql}
              \begin{supertabular}{l}
                Haas \\
                Haas \\
                Krause \\
                Krause \\
              \end{supertabular}
            \end{msoraclesql}
          \end{small}
        \end{center}
\clearpage
        \item Zeigen Sie die Nachnamen aller Mitarbeiter an, deren Nachname ein
        kleines \enquote{a} und ein kleines \enquote{e} enthält.
        \begin{center}
          \begin{small}
            \changefont{pcr}{m}{n}
            \tablefirsthead {
              \multicolumn{1}{l}{\textbf{NACHNAME}} \\
              \cmidrule(l){1-1}
            }
            \tablehead{}
            \tabletail {
            }
            \tablelasttail {
              \multicolumn{1}{l}{\textbf{24 Zeilen ausgewählt}} \\
            }
            \begin{msoraclesql}
              \begin{supertabular}{l}
                Sindermann \\
                Kaiser \\
                Zimmermann \\
                Walther \\
                Neumann \\
                Lehmann \\
              \end{supertabular}
            \end{msoraclesql}
          \end{small}
        \end{center}
      \end{enumerate}
