\clearpage
    \section{Übungen - Erstellen von Views}
      \begin{enumerate}
        \item Erstellen Sie die View \identifier{v\_Arbeitsort}. Diese muss
für jeden Mitarbeiter den Vorname, den Nachnamen, die Bankfiliale\_ID und den
Ort anzeigen, an dem sich die Filiale befindet.
        \begin{center}
          \begin{small}
            \changefont{pcr}{m}{n}
            \tablefirsthead {
              \multicolumn{1}{l}{\textbf{VORNAME}} &
              \multicolumn{1}{l}{\textbf{NACHNAME}} &
              \multicolumn{1}{r}{\textbf{BANKFILIALE\_ID}} &
              \multicolumn{1}{l}{\textbf{ORT}} \\
              \cmidrule(l){1-1}\cmidrule(l){2-2}\cmidrule(r){3-3}\cmidrule(l){4-4}
            }
            \tablehead{}
            \tabletail {
              \multicolumn{4}{l}{\textbf{93 Zeilen ausgewählt}} \\
            }
            \tablelasttail {
              \multicolumn{4}{l}{\textbf{93 Zeilen ausgewählt}} \\
            }
            \begin{msoraclesql}
              \begin{supertabular}{llrl}
                Marie & Kipp & 1 & Aschersleben \\
                Louis & Schmitz & 1 & Aschersleben \\
                Johannes & Lehmann & 1 & Aschersleben \\
                Dirk & Peters & 1 & Aschersleben \\
                Amelie & Krüger & 1 & Aschersleben \\
              \end{supertabular}
            \end{msoraclesql}
          \end{small}
        \end{center}
        \item Erstellen Sie die View \identifier{v\_Depotbesitzer}, die zu jedem Eigenkunden, der ein Depot besitzt, seinen Vor- und Nachnamen, die Strasse mit der Hausnummer, sowie PLZ und Ort anzeigt.
        \begin{center}
          \begin{small}
            \changefont{pcr}{m}{n}
            \tablefirsthead {
              \multicolumn{1}{l}{\textbf{VORNAME}} &
              \multicolumn{1}{l}{\textbf{NACHNAME}} &
              \multicolumn{1}{l}{\textbf{STRASSE}} &
              \multicolumn{1}{l}{\textbf{PLZ}} &
              \multicolumn{1}{l}{\textbf{ORT}} \\
              \cmidrule(l){1-1}\cmidrule(l){2-2}\cmidrule(l){3-3}\cmidrule(l){4-4}\cmidrule(l){5-5}
            }
            \tablehead{}
            \tabletail {
            }
            \tablelasttail {
              \multicolumn{5}{l}{\textbf{239 Zeilen ausgewählt}} \\
            }
            \begin{msoraclesql}
              \begin{supertabular}{lllll}
                Sophie & Junge & Plutoweg 3 & 39435 & Bördeaue \\
                Hanna & Beck & Beimsstraße 9 & 39439 & Güsten \\
                Sebastian & Peters & Steinigstraße 3 & 39240 & Staßfurt \\
                Tina & Berger & Bundschuhstraße 1 & 04177 & Leipzig \\
              \end{supertabular}
            \end{msoraclesql}
          \end{small}
        \end{center}
        \item Erstellen Sie die View \identifier{v\_Finanzberater}, die für alle Eigenkunden deren Vor- und Nachnamen anzeigt, sowie den Vor- und den Nachnamen ihres persönlichen Finanzberaters (Tabelle \identifier{EigenkundeMitarbeiter}).
        \begin{center}
          \begin{small}
            \changefont{pcr}{m}{n}
            \tablefirsthead {
              \multicolumn{1}{l}{\textbf{Vorname Kunde}} &
              \multicolumn{1}{l}{\textbf{Nachname Kunde}} &
              \multicolumn{1}{l}{\textbf{Vorname Berater}} &
              \multicolumn{1}{l}{\textbf{Nachname Berater}} \\
              \cmidrule(l){1-1}\cmidrule(l){2-2}\cmidrule(l){3-3}\cmidrule(l){4-4}
            }
            \tablehead{}
            \tabletail {
            }
            \tablelasttail {
              \multicolumn{4}{l}{\textbf{400 Zeilen ausgewählt}} \\
            }
            \begin{msoraclesql}
              \begin{supertabular}{llll}
                Mia & Keller & Lena & Herrmann \\
                Emilia & Keller & Louis & Wagner \\
                Finn & Junge & Leni & Friedrich \\
                Marie & Vogel & Finn & Wolf \\
                Rudi & Roggatz & Frank & Meierhöfer \\
                Leni & Koch & Frank & Hartmann \\
                Chris & Zimmermann & Clara & Walther \\
                Justin & Zimmermann & Leni & Friedrich \\
                Petra & Krause & Chris & Hartmann \\
                Clara & Rollert & Franz & Berger \\
              \end{supertabular}
            \end{msoraclesql}
          \end{small}
        \end{center}
\clearpage
        \item Erstellen Sie die View \identifier{v\_Unterstellungsverhaeltnis}, die zu jedem Mitarbeiter (Vorname, Nachname) den Vor- und den Nachnamen seines Vorgesetzten anzeigt. Wichtig ist, dass alle Mitarbeiter, auch Herr Max Winter, der keinen Vorgestzten hat, angezeigt werden.
        \begin{center}
          \begin{small}
            \changefont{pcr}{m}{n}
            \tablefirsthead {
              \multicolumn{1}{l}{\textbf{VORNAME}} &
              \multicolumn{1}{l}{\textbf{NACHNAME}} &
              \multicolumn{1}{l}{\textbf{VORNAME}} &
              \multicolumn{1}{l}{\textbf{NACHNAME}} \\
              \cmidrule(l){1-1}\cmidrule(l){2-2}\cmidrule(l){3-3}\cmidrule(l){4-4}
            }
            \tablehead{}
            \tabletail {
            }
            \tablelasttail {
              \multicolumn{4}{l}{\textbf{100 Zeilen ausgewählt}1} \\
            }
            \begin{msoraclesql}
              \begin{supertabular}{llll}
                Finn & Seifert & Max & Winter \\
                Sarah & Werner & Max & Winter \\
                Tim & Sindermann & Sarah & Werner \\
                Sebastian & Schwarz & Sarah & Werner \\
                Emily & Meier & Finn & Seifert \\
                Peter & Möller & Finn & Seifert \\
              \end{supertabular}
            \end{msoraclesql}
          \end{small}
        \end{center}
        \item Erstellen Sie die View \identifier{v\_Innendienstmitarbeiter}, die ermittelt, ob es Mitarbeiter gibt (Vorname und Nachname), die keine Kundenberatung durchführen. Ausgenommen sind leitende Mitarbeiter (Mitarbeiter die in keiner Bankfiliale arbeiten) und Filialleiter.
        \begin{center}
          \begin{small}
            \changefont{pcr}{m}{n}
            \tablefirsthead {
              \multicolumn{1}{l}{\textbf{VORNAME}} &
              \multicolumn{1}{l}{\textbf{NACHNAME}} \\
              \cmidrule(l){1-1}\cmidrule(l){2-2}
            }
            \tablehead{}
            \tabletail {
              \multicolumn{2}{l}{\textbf{40 Zeilen ausgewählt}} \\
            }
            \tablelasttail {
              \multicolumn{2}{l}{\textbf{40 Zeilen ausgewählt}} \\
            }
            \begin{msoraclesql}
              \begin{supertabular}{ll}
                Amelie & Krüger \\
                Anna & Schneider \\
                Chris & Simon \\
                Christian & Haas \\
                Elias & Sindermann \\
              \end{supertabular}
            \end{msoraclesql}
          \end{small}
        \end{center}
        \item Erstellen Sie die View \identifier{v\_Girokontoinhaber}, die alle Eigenkunden anzeigt, die nur ein Girokonto besitzen.
        \begin{center}
          \begin{small}
            \changefont{pcr}{m}{n}
            \tablefirsthead {
              \multicolumn{1}{l}{\textbf{VORNAME}} &
              \multicolumn{1}{l}{\textbf{NACHNAME}} \\
              \cmidrule(l){1-1}\cmidrule(l){2-2}
            }
            \tablehead{}
            \tabletail {
%               \multicolumn{2}{l}{\textbf{21 Zeilen ausgewählt}} \\
            }
            \tablelasttail {
              \multicolumn{2}{l}{\textbf{21 Zeilen ausgewählt}} \\
            }
            \begin{msoraclesql}
              \begin{supertabular}{ll}
                Amelie & Becker \\
                Amelie & Richter \\
                Chris & Walther \\
                Emilia & Keller \\
                Georg & Keller \\
                Johanna & Schäfer \\
                Justin & Zimmermann \\
              \end{supertabular}
            \end{msoraclesql}
          \end{small}
        \end{center}
      \end{enumerate}
