\clearpage
    \section{Übungen - Gruppenfunktionen}
      \begin{enumerate}
        \item Schreiben Sie eine Abfrage, die das höchste und das niedrigste
        Gehalt, das Durchschnittsgehalt und die Summe aller Gehälter ausgibt.
        Beschriften Sie die Spalten, wie es in der Lösung zu sehen ist.
        \begin{center}
          \begin{small}
            \changefont{pcr}{m}{n}
            \tablefirsthead {
              \multicolumn{1}{r}{\textbf{Maximum}} &
              \multicolumn{1}{r}{\textbf{Minimum}} &
              \multicolumn{1}{r}{\textbf{Mittelwert}} &
              \multicolumn{1}{r}{\textbf{Summe}} \\
              \cmidrule(r){1-1}\cmidrule(r){2-2}\cmidrule(r){3-3}\cmidrule(r){4-4}
            }
            \tablehead{}
            \tabletail {
              \multicolumn{4}{l}{\textbf{1 Zeile ausgewählt}} \\
            }
            \tablelasttail {
              \multicolumn{4}{l}{\textbf{1 Zeile ausgewählt}} \\
            }

            \begin{msoraclesql}
              \begin{supertabular}{rrrr}
                88000 & 1000 & 7255 & 725500 \\
              \end{supertabular}
            \end{msoraclesql}
          \end{small}
        \end{center}
        \item Verändern Sie die Abfrage aus der vorangegangenen Abfrage so,
        dass die Informationen für jede einzelne Bankfiliale angezeigt werden.
        Sortieren sie das Ergebnis nach den IDs der Bankfilialen.
        \begin{center}
          \begin{small}
            \changefont{pcr}{m}{n}
            \tablefirsthead {
              \multicolumn{1}{r}{\textbf{BANKFILIALE\_ID}} &
              \multicolumn{1}{r}{\textbf{Maximum}} &
              \multicolumn{1}{r}{\textbf{Minimum}} &
              \multicolumn{1}{r}{\textbf{Mittelwert}} &
              \multicolumn{1}{r}{\textbf{Summe}} \\
              \cmidrule(r){1-1}\cmidrule(r){2-2}\cmidrule(r){3-3}\cmidrule(r){4-4}\cmidrule(r){5-5}
            }
            \tablehead{}
            \tabletail {
              \multicolumn{5}{l}{\textbf{20 Zeilen ausgewählt}} \\
            }
            \tablelasttail {
              \multicolumn{5}{l}{\textbf{20 Zeilen ausgewählt}} \\
            }
            \begin{msoraclesql}
              \begin{supertabular}{rrrrr}
                1 & 12000 & 2000 & 4100 & 20500 \\
                2 & 12000 & 1000 & 3400 & 17000 \\
                3 & 12000 & 1000 & 3900 & 19500 \\
                4 & 12000 & 1500 & 4100 & 20500 \\
                5 & 12000 & 1000 & 4200 & 21000 \\
                6 & 12000 & 2000 & 4200 & 21000 \\
              \end{supertabular}
            \end{msoraclesql}
          \end{small}
        \end{center}
        \item Schreiben Sie eine Abfrage, die die Anzahl der Mitarbeiter pro
        Bankfiliale ausgibt. Beschriften Sie die Spalten so, wie es in der
        Lösung zu sehen ist und sortieren Sie das Ergebnis nach den IDs der
        Filialen.
        \begin{center}
          \begin{small}
            \changefont{pcr}{m}{n}
            \tablefirsthead {
              \multicolumn{1}{r}{\textbf{BANKFILIALE\_ID}} &
              \multicolumn{1}{r}{\textbf{Anzahl}} \\
              \cmidrule(r){1-1}\cmidrule(r){2-2}
            }
            \tablehead{}
            \tabletail {
              \multicolumn{2}{l}{\textbf{20 Zeilen ausgewählt}} \\
            }
            \tablelasttail {
              \multicolumn{2}{l}{\textbf{20 Zeilen ausgewählt}} \\
            }
            \begin{msoraclesql}
              \begin{supertabular}{rr}
                1 & 5 \\
                2 & 5 \\
                3 & 5 \\
                4 & 5 \\
                5 & 5 \\
                6 & 5 \\
             \end{supertabular}
            \end{msoraclesql}
          \end{small}
        \end{center}
\clearpage
        \item Schreiben Sie eine Abfrage, die für jeden Ort einzeln, die
        Anzahl der Eigenkunden zählt, die vor dem \enquote{01.01.1990} 18
        Jahre alt waren.
        \begin{center}
          \begin{small}
            \changefont{pcr}{m}{n}
            \tablefirsthead {
              \multicolumn{1}{l}{\textbf{ORT}} &
              \multicolumn{1}{r}{\textbf{Anzahl}} \\
              \cmidrule(l){1-1}\cmidrule(r){2-2}
            }
            \tablehead{}
            \tabletail {
            }
            \tablelasttail {
              \multicolumn{2}{l}{\textbf{15 Zeilen ausgewählt}} \\
            }
            \begin{msoraclesql}
              \begin{supertabular}{lr}
                Nienburg & 5 \\
                Calbe & 3 \\
                Hecklingen & 3 \\
                Dresden & 1 \\
                Berlin & 2 \\
                Schönebeck & 1 \\
                Leipzig & 1 \\
              \end{supertabular}
            \end{msoraclesql}
          \end{small}
        \end{center}
        \item Erstellen Sie eine Abfrage, die für alle bankeigenen Kunden die
        Buchungen auf deren Girokonten zählt. Interessant sind nur Buchungen
        mit einem Betrag \textgreater 10.000 EUR. Sortieren Sie die Abfrage nach
        der Spalte \identifier{Konto\_ID}.
        \begin{center}
          \begin{small}
            \changefont{pcr}{m}{n}
            \tablefirsthead {
              \multicolumn{1}{r}{\textbf{KONTO\_ID}} &
              \multicolumn{1}{r}{\textbf{COUNT(*)}} \\
              \cmidrule(r){1-1}\cmidrule(r){2-2}
            }
            \tablehead{}
            \tabletail {
              \multicolumn{2}{l}{\textbf{367 Zeilen ausgewählt}} \\
            }
            \tablelasttail {
              \multicolumn{2}{l}{\textbf{367 Zeilen ausgewählt}} \\
            }
            \begin{msoraclesql}
              \begin{supertabular}{rr}
                1 & 8 \\
                2 & 11 \\
                3 & 10 \\
                5 & 7 \\
                6 & 8 \\
                7 & 9 \\
                9 & 8 \\
              \end{supertabular}
            \end{msoraclesql}
          \end{small}
        \end{center}
        \item Schreiben Sie eine Abfrage, die alle Mitarbeiter anzeigt, deren
        Gehalt um mehr als 4.000 EUR niedriger ist, als das Durchschnittsgehalt
        aller Mitarbeiter.
        \begin{center}
          \begin{small}
            \changefont{pcr}{m}{n}
            \tablefirsthead {
              \multicolumn{1}{l}{\textbf{VORNAME}} &
              \multicolumn{1}{l}{\textbf{NACHNAME}} &
              \multicolumn{1}{r}{\textbf{GEHALT}} \\
              \cmidrule(l){1-1}\cmidrule(l){2-2}\cmidrule(r){3-3}
            }
            \tablehead{}
            \tabletail {
              \multicolumn{4}{l}{\textbf{59 Zeilen ausgewählt}} \\
            }
            \tablelasttail {
              \multicolumn{4}{l}{\textbf{59 Zeilen ausgewählt}} \\
            }
            \begin{msoraclesql}
              \begin{supertabular}{llrr}
                Louis & Wagner & 1500 \\
                Lukas & Weiß & 2000 \\
                Maja & Keller & 1000 \\
                Karolin & Klingner & 2000 \\
                Elias & Sindermann & 1000 \\
              \end{supertabular}
            \end{msoraclesql}
          \end{small}
        \end{center}
\clearpage
        \item Schreiben Sie eine Abfrage, die alle Mitarbeiter anzeigt, die
        höchstens zwei Jahre älter sind, als der jüngste Mitarbeiter in
        deren Bankfiliale! 
        \begin{center}
          \begin{small}
            \changefont{pcr}{m}{n}
            \tablefirsthead {
              \multicolumn{1}{l}{\textbf{VORNAME}} &
              \multicolumn{1}{l}{\textbf{NACHNAME}} &
              \multicolumn{1}{l}{\textbf{GEBURTSDATUM}} &
              \multicolumn{1}{l}{\textbf{Juengster Mitarbeiter}} \\
              \cmidrule(l){1-1}\cmidrule(l){2-2}\cmidrule(l){3-3}\cmidrule(l){4-4}
            }
            \tablehead{}
            \tabletail {
              \multicolumn{4}{l}{\textbf{32 Zeilen ausgewählt}} \\
            }
            \tablelasttail {
              \multicolumn{4}{l}{\textbf{32 Zeilen ausgewählt}} \\
            }
            \begin{msoraclesql}
              \begin{supertabular}{llll}
                Johannes & Lehmann & 1992-11-07 & 1992-11-07 \\
                Dirk & Peters & 1991-09-16 & 1992-11-07 \\
                Stefan & Beck & 1983-12-21 & 1984-11-16 \\
                Martin & Schacke & 1984-11-16 & 1984-11-16 \\
                Lukas & Weiß & 1989-03-23 & 1989-03-23 \\
                Alexander & Weber & 1987-11-05 & 1989-03-23 \\
                Anne & Zimmermann & 1991-01-28 & 1991-01-28 \\
              \end{supertabular}
            \end{msoraclesql}
          \end{small}
        \end{center}
        \item Schreiben Sie eine Abfrage, die zu jedem Filialleiter, das Gehalt
        seines am schlechtesten bezahlten Mitarbeiters anzeigt. Sortieren Sie
        die Abfrage nach den Bankfilial-IDs der Filialleiter.
        \begin{center}
          \begin{small}
            \changefont{pcr}{m}{n}
            \tablefirsthead {
              \multicolumn{1}{l}{\textbf{VORNAME}} &
              \multicolumn{1}{l}{\textbf{NACHNAME}} &
              \multicolumn{1}{r}{\textbf{GEHALT}} &
              \multicolumn{1}{r}{\textbf{Kleinstes Gehalt}} \\
              \cmidrule(l){1-1}\cmidrule(l){2-2}\cmidrule(r){3-3}\cmidrule(r){4-4}
            }
            \tablehead{}
            \tabletail {
            }
            \tablelasttail {
              \multicolumn{4}{l}{\textbf{20 Zeilen ausgewählt}} \\
            }
            \begin{msoraclesql}
              \begin{supertabular}{llrr}
                Dirk & Peters & 12000 & 2000 \\
                Louis & Winter & 12000 & 1000 \\
                Alexander & Weber & 12000 & 1000 \\
                Sophie & Schwarz & 12000 & 1500 \\
                Jessica & Weber & 12000 & 1000 \\
              \end{supertabular}
            \end{msoraclesql}
          \end{small}
        \end{center}
      \end{enumerate}