\documentclass[twoside,bibliography=totocnumbered,version=first,DIV14,BCOR=5mm]{scrbook}

% Paketdefinitionen
\usepackage{ifthen}
\usepackage{scrhack}
\usepackage{scrpage2}
\usepackage{array}
\usepackage{booktabs}
\usepackage{listings}
\usepackage[pdftex]{xcolor,graphicx}
\usepackage{tocloft,minitoc}
\usepackage[utf8]{inputenc}
\usepackage{ucs}
\usepackage[T1]{fontenc}
\usepackage[ngerman]{babel}
\usepackage{enumerate}
\usepackage{supertabular}
\usepackage{chappg}
\usepackage{setspace}
\usepackage{fixltx2e}
\usepackage[pdftex,extension=pdf,bookmarks,plainpages=false]{hyperref}
\usepackage{multirow}
\usepackage{tikz}
\usepackage{tikz-er2}
\usepackage[babel,german=quotes]{csquotes}
\usepackage{amssymb}
\usepackage{amsmath}
\usepackage{wrapfig}
\usepackage[backref=true, backend=bibtex, style=alphabetic]{biblatex}
\pdfcompresslevel=9





% Allgemeine Definitionen
\graphicspath{{../grafiken/}}
\DeclareGraphicsExtensions{.png,.pdf}
\KOMAoptions{fontsize=11pt,paper=a4,pagesize=pdftex}
\raggedbottom

\clubpenalty = 10000
\widowpenalty = 10000 \displaywidowpenalty = 10000

% Silbentrennung
\hyphenation{Ab-fra-ge An-wei-sung-en Arbeits-speicher Be-in-hal-ten Be-in-hal-tet Be-zieh-ungs-typ-rich-tung-en Da-ten Da-ten-bank Da-ten-bank-archi-tek-tur Feh-ler Hin-ter-grund PL/SQL Pro-zess Ser-ver Nut-zer State-ment State-ments Spei-cher-struk-tur-en SQL-State-ments Zu-griffs-rech-te}

\makeatletter
  \renewcommand{\@pnumwidth}{3.5em}
  \renewcommand{\@tocrmarg}{1em}
\makeatother

\onehalfspacing

\renewcommand{\headfont}{%
  \normalfont\sffamily\bfseries
}
\renewcommand{\pnumfont}{%
  \Large\bfseries\normalfont\rmfamily\slshape
}
\renewcommand*{\chapterpagestyle}{scrheadings}

\parskip1.5ex
\parindent0pt
\newcounter{lstcounter}
\setcounter{lstcounter}{1}

\newcounter{tabcounter}
\setcounter{tabcounter}{\value{table}}

\newcounter{figcounter}
\setcounter{figcounter}{\value{figure}}

% Definitionen fuer graphicx
\definecolor{lightblue}{rgb}{0,0.7,0.7}
\definecolor{darkblue}{rgb}{0,0,.5}
\definecolor{lightgreen}{rgb}{0,0.5,0.25}
\definecolor{darkgreen}{rgb}{0,0.75,0.5}
\definecolor{lightyellow}{rgb}{1,1,0.4}
\definecolor{darkmagenta}{rgb}{0.611,0.223,0.611}
\definecolor{mediumblue}{rgb}{0.243,0.447,0.764}
\definecolor{mediumred}{rgb}{0.6156,0.1529,0.1529}

% Definitionen fuer hyperref
\hypersetup{colorlinks=true, breaklinks=true, linkcolor=darkblue, menucolor=darkblue, urlcolor=darkblue}

\input{../einstellungen/plsql_sprachdefinition}
\input{../einstellungen/oracle_sql_sprachdefinition}
\lstdefinelanguage{ms_sql}{
morekeywords=[5]{ASC, ADD, ALL, ALTER, AND, AS, ATTACH, AUTHORIZATION, AVG,
BEGIN, BETWEEN, BULK, BY, CASCADE, CAST, CEILING, CHAR, CHARINDEX, CHECK,
CHECKPOINT, CLUSTERED, COLUMN, COMMIT, CONSTRAINT, CONTAINS, CONVERT, COUNT, CREATE, DATABASE, DATEADD,
DATEDIFF, DATENAME, DATEPART, DATETIME2, DAY, DEFAULT, DELETE, DENY, DESC,
DISABLE, DISTINCT, DROP, EXCEPT, EXEC, EXISTS, FILE, FILEGROWTH, FILEGROUP, FILENAME,
FILESTREAM, FLOOR, FOR, FROM, FULL, GETDATE, GO, GRANT, GROUP, HASH, HAVING,
HOUR, IF, IMMEDIATE, INCLUDE, INDEX, INNER, INSERT, INTERSECT, INTO, IS, ISNULL, JOIN,
KEY, LEFT, LEN, LIKE, LOG, LOG10, LOGIN, LOWER, MAX, MAXSIZE, MEMBER, MIN,
MINUTE, MODIFY, MONTH, NAME, NEWNAME, NONCLUSTERED, NOT, NUMERIC, NULL, PRIMARY, ON,
OPENROWSET, OPTION, OR, ORDER, OUTER, OVERRIDE, PIVOT, POWER, REBUILD,
RECONFIGURE, REORGANIZE, REVOKE, RIGHT, ROLE, ROLLBACK, ROUND, SIZE, SELECT,
SERVER, SET, SQRT, SUBSTRING, SUM, TABLE, TO, TOP, TRAN, TRANSACTION, TRUNCATE,
UNION, UNIQUE, UNPIVOT, UPDATE, UPPER, USE, USER, USING, VALUES, VARCHAR, VIEW,
WHERE, WINDOWS, WITH, YEAR},
  sensitive=true, 
  morestring=[b]', 
  morecomment=[l]{--}, 
  morecomment=[s]{/*}{*/},
  moredelim=[is][\color{mediumred}]{§}{§},       %% For Methods
  moredelim=[is][\color{blue}]{?}{?},            %% For DBCC
  keywordstyle=[5]\color{magenta}\bfseries, 
  commentstyle=\color{darkgray}\itshape,
  identifierstyle=, 
  stringstyle=\ttfamily, 
  basicstyle=\ttfamily\footnotesize,
  tabsize=2, showtabs=false, showspaces=false, showstringspaces=false,
  extendedchars=true,
  formfeed=\newpage,
  frame=single
}

\input{../einstellungen/sqlplus_sprachdefinition}
\input{../einstellungen/rman_sprachdefinition}
\input{../einstellungen/configfile_sprachdefinition}
\input{../einstellungen/expdp_impdp_sprachdefinition}
\input{../einstellungen/powershell_sprachdefinition}
\input{../einstellungen/terminal_sprachdefinition}
\input{../einstellungen/ebnf_sprachdefinition}

\lstset {
        tabsize=2,
        showtabs=false,
        showspaces=false,
        showstringspaces=false,
        extendedchars=true,
        formfeed=\newpage,
        frame=single,
        escapechar=\&,
        literate=%
        {ß}{{\ss}}1
        {Ä}{{\"A}}1
        {ä}{{\"a}}1
        {Ö}{{O}}1
        {ö}{{\"o}}1
        {Ü}{{\"U}}1
        {ü}{{\"u}}1
        }
% Definitionen fuer listings
% \lstset{
%         keywordstyle=\color{black},
%         commentstyle=\color{black},
%         identifierstyle=\color{black},
%         stringstyle=\color{black},
%         basicstyle=\color{black}\bfseries\ttfamily\footnotesize,
%         resetmargins=true
%         classoffset=0,
%           language=oracle_sql,
%         classoffset=1,
%           alsolanguage=sqlplus,
%         classoffset=2,
%           alsolanguage=ms_sql,
%         classoffset=3,
%           alsolanguage=plsql,
%         classoffset=4,
%           alsolanguage=rman,
%         classoffset=5,
%           alsolanguage=configfile,
%         classoffset=6,
%           alsolanguage=expdp_impdp,
%           emph={[8]TRUE,FALSE}, emphstyle={[8]\color{lightgreen}},
%           emph={[8]SQL}, emphstyle={[8]\color{black}},
%         classoffset=7,
%           alsolanguage=powershell}

% Schriftgroesse fuer Beschriftungen definieren
\setkomafont{caption}{\raggedright\scriptsize}
\renewcommand*{\figureformat}{Abb.~\thefigure\autodot}



% Neue Laengen definieren
\newlength{\bildhoehe}
\newlength{\vertspace}

% Definitionen neuer Kommandos
% Den boolschen Wert false als Kommando \FALSE darstellen
\newboolean{boolfalse}
\setboolean{boolfalse}{false}
\newcommand{\FALSE}{\boolean{boolfalse}}

% Den boolschen Wert true als Kommando \TRUE darstellen
\newboolean{booltrue}
\setboolean{booltrue}{true}
\newcommand{\TRUE}{\boolean{booltrue}}

%\boolparam wird ben\"otigt, da #1 im END-Bereich nicht verf\"ugbar
\newcommand*{\boolparam}{}

\newcommand{\bild}[3]{
  \begingroup
    \par
    \setcapindent*{-0em}
    \setcapwidth[o]{0.15\linewidth}
    \settoheight{\bildhoehe}{\includegraphics[scale=#3]{#2}}
    \addtolength{\bildhoehe}{-3em}
    \addvspace{\baselineskip}
  \begin{figure}[h!t]
    \begin{captionbeside}{#1}[o][\linewidth][4.3em]*
      \parbox[t][\bildhoehe][b]{0.85\linewidth}{
      \centering\includegraphics[scale=#3]{#2}}
    \end{captionbeside}
    \label{#2}
  \end{figure}
    \par
  \endgroup
}

\newsavebox{\litbox}

\newenvironment{literaturbuch}{
  \marginpar{\vspace{1em}\ifthispageodd{\hspace*{-4em}}{\hspace*{3em}}\includegraphics[scale=1]{buch}}
  \begin{lrbox}{\litbox}
    \begin{minipage}{.975\linewidth}
      \begin{small}
        \begin{itemize}
}{
        \end{itemize}
      \end{small}
    \end{minipage}
  \end{lrbox}
  \par\fbox{\usebox{\litbox}}\par
}

\newenvironment{merke}{
  \par
  \leaders\vbox to 2\baselineskip{%

  }\vskip2\baselineskip
  \marginpar{\vspace*{-1.5em}\ifthispageodd{\hspace*{1em}}{\hspace*{3em}}\includegraphics[scale=1]{gluehbirne}}
  \vspace{-1.5em}
  \begin{lrbox}           {\litbox}
    \begin{minipage}{.96\linewidth}
}{
    \end{minipage}
  \end{lrbox}
  \fbox{\colorbox{lightyellow}{\usebox{\litbox}}}
  \par
  \addvspace{\baselineskip}
}

\newenvironment{literaturinternet}{
  \par
  \leaders\vbox to 2\baselineskip{%

  }\vskip2\baselineskip
  \marginpar{\vspace*{-1.5em}\ifthispageodd{\hspace*{1em}}{\hspace*{3em}}\includegraphics[scale=1]{globus}}
  \vspace{-1.5em}
  \begin{lrbox}{\litbox}
    \begin{minipage}{.96\linewidth}
      \begin{small}
        \begin{itemize}
}{
        \end{itemize}
      \end{small}
    \end{minipage}
  \end{lrbox}
  \fbox{\usebox{\litbox}}
  \par
  \addvspace{\baselineskip}
}

%isTable=true, weil das Symbol meist vor Tabellen benutzt wird
\newboolean{isTable}
\setboolean{isTable}{true}
\newenvironment{oraclesql}[1][\boolean{isTable}]{
  \renewcommand*{\boolparam}{#1}
  \par
  \leaders\vbox to 2\baselineskip{%

  }\vskip2\baselineskip
  \marginpar{\vspace*{-1.5em}\ifthispageodd{\hspace*{1em}}{\hspace*{3em}}\includegraphics[scale=1]{oracle_11g}}
  \ifthenelse{\boolparam} {
    \vspace{-1.5em}
  } {
    \vspace{-3.5em}
  }
  \begin{lrbox}{\litbox}
    \begin{minipage}{.96\linewidth}
}{
    \end{minipage}
  \end{lrbox}
  \usebox{\litbox}
  \ifthenelse{\boolparam} {
  \par
  \addvspace{\baselineskip}
  } {
  }
}

\newenvironment{mssql}[1][\boolean{isTable}]{
  \renewcommand*{\boolparam}{#1}
  \par
  \leaders\vbox to 2\baselineskip{%

  }\vskip2\baselineskip
  \marginpar{\vspace*{-1.5em}\ifthispageodd{\hspace*{1em}}{\hspace*{3em}}\includegraphics[scale=1]{ms_sql}}
  \ifthenelse{\boolparam} {
    \vspace{-1.5em}
  } {
    \vspace{-3.5em}
  }
  \begin{lrbox}{\litbox}
    \begin{minipage}{.96\linewidth}
}{
    \end{minipage}
  \end{lrbox}
  \usebox{\litbox}
  \par
  \ifthenelse{\boolparam} {
  \par
  \addvspace{\baselineskip}
  } {
  }
}

\newenvironment{msoraclesql}[1][\boolean{isTable}]{
  \renewcommand*{\boolparam}{#1}
  \par
  \leaders\vbox to 2\baselineskip{%

  }\vskip2\baselineskip
  \marginpar{\vspace*{-1.5em}\ifthispageodd{\hspace*{1em}}{\hspace*{3em}}\includegraphics[scale=1]{ms_sql_oracle}}
  \ifthenelse{\boolparam} {
    \vspace{-1.5em}
  } {
    \vspace{-3.5em}
  }
  \begin{lrbox}{\litbox}
    \begin{minipage}{.96\linewidth}
}{
    \end{minipage}
  \end{lrbox}
  \usebox{\litbox}
  \par
  \ifthenelse{\boolparam} {
  \par
  \addvspace{\baselineskip}
  } {
  }
}

\newcommand{\kapitelnummer}[1]{
    \large\setlength{\vertspace}{-2.5em}
    \multiply\vertspace \value{#1}
    \rohead{
      \vspace{\vertspace}
      \large\colorbox{black}{\textcolor{white}{\thechapter\hspace{3mm}}}\hspace*{-5.8em}
    }
}

\newcommand{\identifier}[1]{\textsc{#1}}
\newcommand{\languageorasql}[1]{\lstinline[language=oracle_sql]{#1}}
\newcommand{\languagemssql}[1]{\lstinline[language=ms_sql]{#1}}
\newcommand{\languagerman}[1]{\lstinline[language=rman]{#1}}
\newcommand{\languageplsql}[1]{\lstinline[language=plsql]{#1}}
\newcommand{\languagesqlplus}[1]{\lstinline[language=sqlplus]{#1}}
\newcommand{\languageconfigfile}[1]{\lstinline[language=configfile]{#1}}
\newcommand{\languageexpdpimpdp}[1]{\lstinline[language=expdp_impdp]{#1}}
\newcommand{\languagepowershell}[1]{\lstinline[language=powershell]{#1}}
\newcommand{\oscommand}[1]{\texttt{#1}}
\newcommand{\privileg}[1]{\texttt{#1}}
\newcommand{\parameter}[1]{\MakeLowercase{\textsf{#1}}}

\newcommand{\pk}[1]{\underline{#1}}
\newcommand{\fk}[1]{$\Uparrow$#1$\Uparrow$}
\newcommand{\nn}[1]{#1 [NN]}
\newcommand{\un}[1]{#1 [UN]}

\newcommand{\SELECT}{\languageorasql{SELECT}}
\newcommand{\FROM}{\languageorasql{FROM}}
\newcommand{\WHERE}{\languageorasql{WHERE}}
\newcommand{\GROUPBY}{\languageorasql{GROUP BY}}
\newcommand{\HAVING}{\languageorasql{HAVING}}
\newcommand{\ORDERBY}{\languageorasql{ORDER BY}}
\newcommand{\CHECK}{\languageorasql{CHECK}}
\newcommand{\NOTNULL}{\languageorasql{NOT NULL}}
\newcommand{\UNIQUE}{\languageorasql{UNIQUE}}
\newcommand{\PRIMARYKEY}{\languageorasql{PRIMARY KEY}}
\newcommand{\FOREIGNKEY}{\languageorasql{FOREIGN KEY}}
\newcommand{\INSERT}{\languageorasql{INSERT}}
\newcommand{\UPDATE}{\languageorasql{UPDATE}}
\newcommand{\DELETE}{\languageorasql{DELETE}}
\newcommand{\COMMIT}{\languageorasql{COMMIT}}
\newcommand{\ROLLBACK}{\languageorasql{ROLLBACK}}
\newcommand{\GRANT}{\languageorasql{GRANT}}
\newcommand{\REVOKE}{\languageorasql{REVOKE}}
\newcommand{\DENY}{\languageorasql{DENY}}

\newcommand{\changefont}[3]{\fontfamily{#1} \fontseries{#2} \fontshape{#3} \selectfont}
\newcommand{\beispiel}[1]{\hyperref[#1]{Beispiel~\ref*{#1}}}
\newcommand{\abschnitt}[1]{\hyperref[#1]{Abschnitt~\ref*{#1}}}
\newcommand{\tabelle}[1]{\hyperref[#1]{Tabelle~\ref*{#1}}}
\newcommand{\abbildung}[1]{\hyperref[#1]{Abbildung~\ref*{#1}}}
\makeatletter
\newcommand{\myhref}[2]{\hyper@linkurl{#2}{#1}}
\makeatother


% Biblatex custom styles
\input{../einstellungen/biblatex_custom_styles}

% Definition der Titelseite
\title{Datenbankadministration\linebreak Microsoft SQL Server 2014}
\author{Florian Weidinger}
\date{Erstellung: 21.01.2014\\
\"Anderung: \today{}}

%%%%%%%%%%%%%%%%%%%%%%%%%%%%%%%%%%%%%%%%%%%%%%%%%%%%%%%%%%%%%%%%%%%%%%%%%%%%%%%%%%%
% Biblatex definitionen
\addbibresource{../bibliographie/bibliography.bib}

\defbibheading{tsqlreference}{\section*{Transact SQL Reference}}
\defbibheading{sqlserveradmin}{\section*{SQL Server Administration}}

%%%%%%%%%%%%%%%%%%%%%%%%%%%%%%%%%%%%%%%%%%%%%%%%%%%%%%%%%%%%%%%%%%%%%%%%%%%%%%%%%%%
\begin{document}
  \input{../einstellungen/tikzlibrary}
  % Minitocs fuer einzelne Kapitel einschalten
  \mtcselectlanguage{german}
  \makeatletter
    \renewcommand{\@pnumwidth}{1cm}
    \renewcommand{\@tocrmarg}{2cm}
  \makeatother
  \dominitoc

% Seitenlayout festlegen
  \headheight1cm

  \pagenumbering{bychapter}

  \setheadsepline{.4pt}
  \setfootsepline{.4pt}
  % Kopf- und Fusszeile definieren
  \clearscrheadfoot
  \lofoot{%
    \vspace{-2em}
    SQL Server 2014 Datenbankadministration
  }
  \rofoot{%
    \vspace{-2em}
    \pagemark
  }
  \lefoot{%
    \vspace{-2em}
    \pagemark\hspace{1cm}
  }
  \chead{\leftmark}

  \pagestyle{scrheadings}
 % Kopf- und Fusszeilen auch im Inhaltsverzeichnis benutzen
  \tocloftpagestyle{scrheadings}

% Zeilenabstand temporaer auf einzeilig zurueckschalten
\begin{spacing}{1.1}
  % Titelseite erzeugen
    \maketitle

  % Formatierung der Seitenzahlen im Inhaltsverzeichnis
  \addtocontents{toc}{\cftpagenumbersoff{part}}
  \renewcommand{\cftchappagefont}{\bfseries\small}
  \renewcommand{\cftsecpagefont}{\small}
  \renewcommand{\cftsubsecpagefont}{\small}
  \renewcommand{\cftchapfont}{\bfseries\small}
  \renewcommand{\cftsecfont}{\small}
  \renewcommand{\cftsubsecfont}{\small}
 
	% Inhaltsverzeichnis anzeigen
	\tableofcontents
\end{spacing}
  \part{Grundlagen des Datenbankdesigns}
    \chapter{Entwicklung eines Datenmodells}
    \setcounter{page}{1}\kapitelnummer{chapter}
    \minitoc
\newpage

    In diesem Abschnitt erfolgt die Vermittlung der Kenntnisse dar\"uber, wie man aus einer gro\ss en Menge von Informationen und Anforderungen an eine neue Datenbank eine Datenstruktur entwirft. Im Zuge der Datenmodellierung soll ein exaktes und vollst\"andiges Modell des betrachteten Realit\"atsausschnitts erarbeitet werden, welches den Rahmen f\"ur die Entwicklung neuer oder die Erweiterung bestehender Anwendungssysteme bildet.

    Als Werkzeug f\"ur die Erstellung konzeptioneller Datenmodelle wird die Entity-Relation\-ship Modellierung (ER-Modellierung), zuerst in ihrer einfachsten Art und sp\"ater in einer erweiterten Fassung, verwendet. Es werden dabei alle vier Phasen des Datenmodellierungsprozesses anhand eines durchg\"angigen Beispiels beschrieben.

    Die ER-Modellierung stellt eine spezielle \enquote{Information-Engineering-Technik} dar, die zur Erstellung von Datenmodellen hoher Qualit\"at benutzt wird. Entworfen in den 70er Jahren von Peter Pin-Shan Chen wurde sie seither vielfach erweitert und verbessert.

    Ziel eines konzeptionellen Datenmodells ist es, die typm\"a\ss{}ige Struktur der Daten in der Datenbank ohne deren Inhalt zu beschreiben. Als Beispiel aus der realen Welt w\"are  die Ausstattung eines B\"uros mit M\"obeln zu nennen. Wer sp\"ater das B\"uro nutzt und mit welchem Inhalt die Schr\"anke gef\"ullt werden, ist f\"ur die Erstellung des Modells Bedeutungslos.

    Die Entwicklung eines Datenmodells teilt sich in die folgenden vier Phasen ein:
    \begin{enumerate}
      \item Klassifizierung von Objekten
      \item Festlegung relevanter Eigenschaften
      \item Bestimmung identifizierender Eigenschaften
      \item Beschreibung sachlogischer Zusammenh\"ange zwischen den einzelnen Objekten
    \end{enumerate}
    \section{Die Modellierungsinformationen}
			Die Struktur der Bundeswehr soll in einem ER-Modell dargestellt werden.
			Es wird jedoch nur ein Ausschnitt aus der Realit\"at betrachtet, um die
			\"Ubersichtlichkeit der Datenstrukturen zu gew\"ahrleisten. Im Folgenden
			werden die daf\"ur notwendigen Objekte festgelegt.
        \begin{enumerate}
          \item Die Bundeswehr besitzt zahlreiche Dienststellen an den unterschiedlichsten Standorten, welche sich untereinander \"uber- oder untergeordnet sind. Jede Dienststelle besteht aus Dienstposten. Diese wiederum werden mit Soldaten besetzt. Jeder Soldat empf\"angt bei Einstieg in die Bundeswehr seine pers\"onliche Ausr\"ustung und besitzt diese dann f\"ur die Dauer seines Dienstverh\"altnisses.
          \item Jeder Dienststelle der Bundeswehr ist eine Dienststellennummer zugeordnet, \"uber die diese identifiziert werden kann. Des Weiteren hat jede Dienststelle eine bestimmte Gr\"o\ss e sowie Bezeichnung, wie z. B. F\"uhrungsunterst\"utzungsschule der Bundeswehr.
          \item F\"ur die Standorte, an denen sich die Dienststellen befinden, m\"ussen in der Datenbank Postleitzahl (PLZ), Ort, Stra\ss e und die Hausnummer hinterlegt sein. Hierbei ist zu beachten, dass sich eine Dienststelle auch an mehreren Standorten befinden kann.
          \item Die den Dienststellen untergeordneten Dienstposten werden durch ihre Dienstposten\_ID und ein Beginn- und Enddatum charakterisiert. Als eine weitere Eigenschaft soll eine kurze Dienstpostenbeschreibung hinterlegt werden.
          \item Jeder in der Datenbank aufgef\"uhrte Soldat soll dort mit seiner Personanlnummer, einer Personenkennziffer, sowie Vor- und Nachnamen und Dienstgrad aufgef\"uhrt werden. Weiterhin kann der Anwender auch die aktuelle Adresse, bestehend aus PLZ, Wohnort, Stra\ss e und Hausnummer, aus der Datenbank heraussuchen.
          \item Das letzte Objekt in der Datenbank stellt die pers\"onliche Ausr\"ustung des Soldaten dar. Diese besitzt eine Bezeichnung, ist aus einem bestimmten Material gefertigt und hat eine Farbe. F\"ur eine bessere Zuordnung ist jeder Aur\"ustungsgegenstand mit einer Versorgungsnummer versehen.
        \end{enumerate}
    \section{Klassifizierung von Objekten}
      Das im vorigen Abschnitt vorgestellte Beispiel stellt nur einen kleinen Ausschnitt aus der Realit\"at dar. Komplexer gestaltete Datenbanken k\"onnen aus weit mehr Objekten bestehen.
      Um dabei nicht den \"Uberblick \"uber diese Flut von Objekten zu verlieren, werden diese in Klassen gruppiert. Diese Objektklassen enthalten dann die Objekte, die von ihrer Art her gleich sind und \"uber die die gleichen Informationen gesammelt werden. Damit wird eine Abstraktionsebene gebildet, die es erm\"oglicht, von den Besonderheiten der einzelnen Objekte abzusehen und nur das typische der gebildeten Objektklassen zu ber\"ucksichtigen.
      \subsection{Definitionen und Syntaxregeln}
				F\"ur die Darstellung der Objekttypen gelten folgende syntaktische
				Regeln:
				\begin {enumerate}
					\item Ein Objekttyp wird durch ein Rechteck dargestellt, in dessen
					Mitte der Objekttypname eingetragen wird.
					\begin{center}
						\scalebox{1}{
							\begin{tikzpicture}[node distance=1.5cm, every edge/.style={link}]
								\node[entity](e1){Soldat};
							\end{tikzpicture}
						}
					\end{center}
					\item Die Gr\"o\ss e und Position des Rechtecks sind bedeutungslos.
					\item Der Objekttypname steht im Singular und muss f\"ur
					das gesamte Datenmodell eindeutig sein.
        \end{enumerate}
        \begin{merke}
          \begin{itemize}
            \item \textbf{Objekt}: Ein Objekt (engl. Entity) ist ein Exemplar von Personen, Gegenst\"anden oder nichtmateriellen Dingen, \"uber das Informationen gespeichert wird, z. B. der konkrete Soldat Max Mustermann.
            \item \textbf{Objekttyp}: Ein Objekttyp (engl. Entity type) ist eine durch einen Objekttyp-namen eindeutig benannte Klasse von Objekten, \"uber die dieselben Informationen gespeichert und die prinzipiell in gleicher Weise verarbeitet werden, wie z. B. die benannte Klasse bzw. der Objekttyp SOLDAT.
          \end{itemize}
        \end{merke}

        Die Bildung von Objekttypen h\"angt entscheidend von den Anforderungen
        des jeweils zu modellierenden Gegenstandsbereiches ab. Aus der Sicht
        eines Gro\ss h\"andlers kann ein Unternehmen mit all seinen Bereichen
        als ein einziger Objekttyp gesehen werden. Dasselbe Unternehmen dagegen
        wird aus seiner eigenen Sicht detailliert mit seinen Bereichen,
        Abteilungen, Mitarbeitern, Werkshallen, Fahrzeugen usw. zu modellieren
        sein. Diese Modellierung ist ebenfalls nicht eine einmalige, in sich
        abgeschlossene T\"atigkeit, denn im Laufe der Zeit m\"ussen \"Anderungen
        der Realit\"at auch im Modell eingearbeitet werden.
      \subsection{Traditionelles Pendant}
        Die Informationsverarbeitung mit Hilfe elektronischer Datenverarbeitung
        hat hinsichtlich der Datenspeicherung nur wenig prinzipiell neue
        Methoden entwickelt. Fast alle Konzepte, in Bezug auf das
        Entity-Relationship-Modell, haben ihr Pendant in der traditionellen
        Informationsspeicherung. Zum besseren Verst\"andnis der eingef\"uhrten
        Begriffe wird auf diese Zusammenh\"ange an den entsprechenden Stellen
        hingewiesen.

        So entsprechen die Objekttypen den traditionellen Karteik\"asten und die
        Objekte eines Objekttyps den Karteikarten, die in einem Karteikasten
        eingeordnet sind. In der traditionellen Arbeitsweise w\"urde f\"ur
        \enquote{Axel Schweiss} eine Karteikarte angelegt werden und z. B. im
        Karteikasten \enquote{Soldat} abgelegt werden
\clearpage
    \section{Festlegung der relevanten Eigenschaften}
      Im ersten Schritt, der Klassifizierung von Objekten, wurden Objekte, die
      in gleicher Art und Weise verarbeitet werden, in Objekttypen
      zusammengefasst. Um die Verarbeitung dieser Objekte automatisiert
      durchf\"uhren zu k\"onnen, ist es Voraussetzung, dass jeder Objekttyp
      bestimmte Angaben speichert. Diese Angaben werden f\"ur die elektronische
      Verarbeitung entweder als Eingabeinformation oder als Ausgabeinformation
      ben\"otigt.
      
      Aus diesem Grund ist es notwendig, f\"ur jeden Objekttyp die relevanten
      Eigenschaften der Objekte, die in ihm zusammengefasst werden, anzugeben.
      Damit wird der durch den Objekttyp definierte Begriff auf einen Satz
      relevanter Eigenschaften reduziert. Die Festlegung dieser Eigenschaften
      ist aber nur bei genauer Kenntnis der Gesch\"afts- und
      Verarbeitungsprozesse des Auftraggebers m\"oglich. Ohne diese Kenntnisse
      befindet man sich in jedem Falle im Bereich von Spekulationen.
      \subsection{Definitionen und Syntaxregeln}
        \begin{merke}
          \begin{itemize}
            \item \textbf{Eigenschaft:} Eine Eigenschaft (engl. Attribute) ist
            die Benennung f\"ur ein relevantes Merkmal aller Objekte, die in
            einem Objekttyp zusammengefa\ss t werden, z. B. Soldaten haben die
            Eigenschaft Dienstgrad.
            \item \textbf{Eigenschaftswert:} Ein Eigenschaftswert (engl.
            Attribute value) ist eine spezielle Auspr\"agung, die eine
            Eigenschaft f\"ur ein konkretes Objekt annimmt, z. B. der Dienstgrad
            Hauptfeldwebel.
          \end{itemize}
          \end{merke}

        F\"ur die Darstellung der Eigenschaften gelten folgende Regeln:
        \begin {enumerate}
          \item Die Benennung der Eigenschaft wird als Blase an den Objekttyp
          angeh\"angt.

          \begin{center}
           \scalebox{1}{
            \begin{tikzpicture}[node distance=1.5cm, every edge/.style={link}]
              \node[entity](e1){Soldat};
                \node[attribute](a1)[right = of e1]{Dienstgrad} edge (e1);
            \end{tikzpicture}
           }
          \end{center}
          \item Die Reihenfolge der Eigenschaften ist bedeutungslos.
          \item Die Benennung der Eigenschaft steht im Singular und muss f\"ur
          den Objekttyp eindeutig sein.
        \end{enumerate}
        Auch an dieser Stelle wird eine Abstraktion der realen Welt
        vorgenommen, in dem statt der Eigenschaftswerte, die jedes einzelne
        Objekt besitzt, nur die Eigenschaften der Objekttypen angegeben werden.
        Um diesen Abstraktionsprozess korrekt durchf\"uhren zu k\"onnen, bedarf
        es der Einhaltung einiger Regeln:
        \begin{itemize}
          \item Es darf niemals ein Eigenschaftswert als Eigenschaftsname
					verwendet werden (beispielsweise anstatt 02.05.1985 die Bezeichnung
					\enquote{Geburtsdatum} oder statt \enquote{m\"annlich} die Bezeichnung
					\enquote{Geschlecht}).
          \item Die Bezeichnung eines Objekttyps sollte niemals im
					Eigenschaftsnamen auftauchen, da dieser immer nur im Kontext
					des Objekttyps gilt (z. B. Person mit der Eigenschaft Name, nicht
					\enquote{Personname}, sondern die Eigenschaft \enquote{Name}
					w\"ahlen).
          \item Bei komplexen Eigenschaften wie z. B. einer Adresse oder einer
					PK stellt sich immer wieder die Frage, ob diese zerlegt werden
					m\"ussen oder ob sie als atomar \footnote{atomare Information = eine
					einzige, nicht mehr teilbare Information} betrachtet werden. Die
					Antwort auf diese Frage ist abh\"angig von den einzelnen
					Verarbeitungsprozessen, denen diese Daten unterliegen. Muss
					beispielsweise die Information verf\"ugbar sein, von welchem
					Kreiswehrersatzamt ein Soldat betreut wird, so muss  die PK in ihre
					Bestandteile zerlegt werden. Ist diese Information irrelevant, kann
					die PK im ganzen als atomar betrachtet werden.
          \item Problematisch ist auch die Entscheidung, ob speicherw\"urdige
          Informationen als Eigenschaften oder als ein eigenst\"andiger
          Objekttyp betrachtet werden sollen. Um einen eigenst\"andigen
          Objekttyp handelt es sich immer dann, wenn er f\"ur das Unternehmen
          bedeutsame Objekte enth\"alt, die relevante individuelle Eigenschaften
          besitzen.
        \end{itemize}
      \subsection{Traditionelle Datenspeicherung}
        Die Eigenschaften eines Objekttyps A entsprechen den Feldern, die auf
        einer Karteikarte zur Aufnahme der relevanten Informationen angelegt
        werden. Durch die gew\"ahlten Eigenschaften wird also die einheitliche
        Struktur aller Karteikarten eines Kastens festgelegt.

        Welche Eigenschaften k\"onnen Sie, bezogen auf das Beispiel
        \enquote{Bundeswehr}, den identifizierten Objekttypen zuordnen?
        \subsubsection{L\"osungsvorschlag}
          \begin{itemize}
            \item Objekttyp \enquote{Dienststelle}: Dienststellennummer,
            Bezeichung
            \item Objekttyp \enquote{Standort}: PLZ, Ort, Stra\ss e, Hausnummer
            \item Objekttyp \enquote{Dienstposten}: Dienstposten\_ID,
            Beginndatum, Enddatum, Dienstpostenbeschreibung
            \item Objekttyp \enquote{Soldat}: Personalnummer, PK, Name, Vorname,
            Dienstgrad, PLZ, Ort, Stra\ss e, Hausnummer
            \item Objekttyp \enquote{Ausr\"ustung}: Versorgungsnummer, Material,
            Farbe
          \end{itemize}
    \section{Festlegung der Identifizierung}
      Ein Objekttyp stellt die Zusammenfassung mehrerer gleichartiger Objekte
      dar. Gleichartig bedeutet, dass diese Objekte die gleichen Eigenschaften
      aufweisen und die Verarbeitungsprozesse f\"ur all diese Objekte gleich
      sind. Die einzelnen Objekte eines Objekttyps m\"ussen aber voneinander
      unterscheidbar sein. Deshalb muss festgelegt werden, auf welche Weise ein
      Objekt innerhalb des Objekktyps identifiziert werden kann.

      \subsection{Identifizierungsvarianten}
        F\"ur die Identifizierung eines Objekts innerhalb eines Objekttyps
        stehen die konkreten Eigenschaftswerte des Objekts zur Verf\"ugung.

        Es werden drei Varianten zur Identifizierung von Objekten
        unterschieden.
        \subsubsection{Identifizierung eines Objekts durch eine einzelne
Eigenschaft}
          Es kann vorkommen, dass die Eigenschaftswerte einer Eigenschaft eines
          Objekttyps eindeutig ist. D. h. jeder Eigenschaftswert dieser
          Eigenschaft ist so geartet, dass jedes Objekt dieses Objekttyps einen
          unterschiedlichen Wert f\"ur diese Eigenschaft hat. Durch Angabe
          dieses Eigenschaftswertes ist dann das Objekt eindeutig identifiziert.

          Beispiel: Soldaten werden i. d. R. durch ihre PK eindeutig
          identifiziert

				\subsubsection{Identifizierung eines Objekts durch eine Kombination
mehrerer Eigenschaften}
          In einigen F\"allen ist keine der Eigenschaften eines Objekttyps
          geeignet, alleine als identifizierende Eigenschaft zu fungieren. Um
          dieses Problem zu l\"osen kann versucht werden, eine
          \underline{minimale} Kombination von Eigenschaften eines Objekttyps
          als Identifikationsmerkmal zu benutzen. Dies funktioniert dann, wenn
          die Kombination der Werte der entsprechenden Eigenschaften bei jedem
          Objekt eindeutig sind.

          Beispiel: Die Kombination aus PLZ und Ortsbezeichnung ist eindeutig,
					eine Eigenschaft alleine nicht.

				\subsubsection{Identifizierung eines Objekts durch eine organisatorische
Eigenschaft}
          Sollte es vorkommen, dass weder eine einzelne Eigenschaft, noch eine
					Kombination von Eigenschaften zur Identifikation der Objekte eines
					Objekttyps geeignet ist, kann eine k\"unstliche Eigenschaft
					eingef\"uhrt werden, bei der die Eindeutigkeit durch organisatorische
					Ma\ss nahmen gew\"ahrleistet wird.

					Ein Beispiel hierf\"ur w\"are eine Ort\_ID, die es m\"oglich
					macht, einen Ort eindeutig zu bestimmen ohne die Kombination aus PLZ
					und Ortsnamen heranziehen zu m\"ussen.

          Es ist aber auch m\"oglich, dass eine organisatorische Eigenschaft
          gew\"ahlt wird, da abzusehen ist, dass eine identifizierende
          Eigenschaft durch eine andere ersetzt werden soll. Die
          Umstrukturierung der Datenbank w\"are zu aufwendig und zu komplex.

          Ein Beispiel aus der Praxis ist die PK und die Personalnummer eines
          jeden Soldaten. Die Personalnummer wird in geraumer Zeit die PK
          ersetzen. Es ist aus diesem Grund von Vorteil eine organisatorische
          Eigenschaft, wie die Personen\_ID zu w\"ahlen, um eine
          Umstrukturierung zu vermeiden.

          Andere Beispiele f\"ur organisatorische Eigenschaften sind:
          Artikelnummer und LfdNr.

          Die Identifizierung der Objekte eines Objekttyps mittels einer
					\enquote{organisatorischen Eigenschaft} ist bei der automatisierten
					Datenverarbeitung eine beliebte Vorgehensweise. Sie wird h\"aufig
					selbst dann angewendet, wenn nat\"urliche identifizierende
					Eigenschaften vorhanden sind. Meist handelt es sich um eine laufende
					Nummer.

					Dies hat den Vorteil, dass kurze identifizierende Eigenschaftswerte
					entstehen. Der Nachteil ist, dass Werte wie z. B. eine laufende
					Nummer meist keinerlei Aussagekraft haben und somit Gefahren wie
					Verwechslung oder Fehleingabe entstehen.

					Die Festlegung der Identifizierungsform f\"ur einen Objekttyp muss mit
					gro\ss er Sorgfalt erfolgen. Relationale Datenbank-Managementsysteme
					f\"ur die wir unsere Modellierung durch\-f\"uh\-ren, lassen es
					n\"amlich nicht zu, dass zwei Objekte desselben Objekttyps in der
					Kombination ihrer identifizierenden Merkmale \"ubereinstimmen. Bleibt
					ein Restrisiko hinsichtlich der Unikalit\"at der identifizierenden
					Merkmale und treten dann bei der praktischen Datenbankarbeit
					tats\"achlich zwei Objekte mit \"ubereinstimmenden Werten ihrer
					identifizierenden Merkmale auf, lehnt das Datenbank-Managementsystem
					die Speicherung des zweiten Objekts ab.

					\begin{merke}
						Um ein Modell \"ubersichtlich und verst\"andlich zu halten, sollte
						konsequent immer nur eine der drei zur Verf\"ugung stehenden
						Methoden f\"ur die Identifizierung von Objekten genutzt werden!
					\end{merke}
			\subsection{Markierung der Identifizierenden Eigenschaft}
				F\"ur die Markierung der (teil)identifizierenden Eigenschaft gelten
				die folgenden syntaktischen Regeln (diese werden u. U. in Kombination
				mit anderen angewendet):
\clearpage
        \begin {enumerate}
          \item Eine identifizierende Eigenschaft (bzw. jede teilidentifizierende Eigenschaft) wird durch Unterstreichung kenntlich gemacht.

          \begin{center}
           \scalebox{1}{
            \begin{tikzpicture}[node distance=1.5cm, every edge/.style={link}]
              \node[entity](e1){Soldat};
							\node[attribute](a1)[right = of e1]{\key{Personalnummer}} edge
(e1);
            \end{tikzpicture}
           }
          \end{center}
          \item Die Position der (teil)identifizierenden Eigenschaft innerhalb der Liste der Eigenschaften ist bedeutungslos. Sie sollte jedoch aus Gr\"unden der \"Ubersichtlichkeit immer zu erst genannt werden.
        \end{enumerate}
      \subsection{Klassisches Pendant}
        Bei der traditionellen Informationsspeicherung mit Hilfe von Karteik\"asten entspricht der Identifizierung die Festlegung eines Kennbegriffs: \"Ublicherweise wird im Kopf einer Karteikarte f\"ur das betreffende Objekt ein Wert angegeben, der das Objekt innerhalb des Karteikastens identifiziert\footnote{Kennbegriff wird oft auch als \enquote{Reiter} bezeichnet}. Um eine bestimmt Karteikarte manuell schneller finden zu k\"onnen, werden die Karteikarten des Karteikastens nach diesem Begriff sortiert.

        Betrachten wir wieder das Beispiel \enquote{Bundeswehr}. Welche Objekteigenschaften k\"on\-nen Ihrer Meinung nach als identifizierende Merkmale verwendet werden?
        \subsubsection{L\"osungsvorschlag}
          \begin{itemize}
            \item Objekttyp \enquote{Dienststelle}: Dienststellennummer
            \item Objekttyp \enquote{Standort}: PLZ, Ort
            \item Objekttyp \enquote{Dienstposten}: Dienstposten\_ID
            \item Objekttyp \enquote{Soldat}: Personalnummer
            \item Objekttyp \enquote{Ausr\"ustung}: Versorgungsnummer
          \end{itemize}
          An dieser Stelle wird die Entscheidung getroffen, dass f\"ur die Fortf\"uhrung dieses Modells organisatorische Einheiten als identifizierende Eigenschaften eingef\"uhrt werden!
          \begin{itemize}
            \item Objekttyp \enquote{Dienststelle}: Dienststellen\_ID
            \item Objekttyp \enquote{Standort}: Ort\_ID
            \item Objekttyp \enquote{Dienstposten}: DP\_ID
            \item Objekttyp \enquote{Soldat}: Personen\_ID
            \item Objekttyp \enquote{Ausr\"ustung}: Ausruestungs\_ID
          \end{itemize}

    \section{Modellierungsformen}
          \begin{center}
           \scalebox{.6}{
            \begin{tikzpicture}[node distance=1.5cm, every edge/.style={link}]
              \node[entity](dienstposten){Dienstposten};
                \node[attribute](dpid)[above = of dienstposten]{\key{DP\_ID}} edge (dienstposten);
                \node[attribute](dienstpostenid)[above left = of dienstposten]{Dienstposten\_ID} edge (dienstposten);
                \node[attribute](begindatum)[above right = of dienstposten]{Beginndatum} edge (dienstposten);
                \node[attribute](enddatum)[below right = of dienstposten]{Enddatum} edge (dienstposten);
                \node[attribute](dienstpostenbeschreibung)[below left = of dienstposten]{Dienstpostenbeschreibung} edge (dienstposten);
              \node[entity](soldat)[below left = 6cm of dienstposten]{Soldat};
                \node[attribute](personenid)[above = 1.6cm of soldat]{\key{Personen\_ID}} edge (soldat);
                \node[attribute](pk)[above right = of soldat]{PK} edge (soldat);
                \node[attribute](name)[below right = of soldat]{Name} edge (soldat);
                \node[attribute](vorname)[below left = of soldat]{Vorname} edge (soldat);
                \node[attribute](dienstgrad)[above left = of soldat]{Dienstgrad} edge (soldat);
              \node[entity](ausruestung)[below = 6cm of soldat]{Ausr\"ustung};
                \node[attribute](versorgungsnummer)[above left = of ausruestung]{Versorgungsnummer} edge (ausruestung);
                \node[attribute](bezeichnung)[above = 1.7cm of ausruestung]{Bezeichnung} edge (ausruestung);
                \node[attribute](ausruestungsid)[above right = of ausruestung]{\key{Ausruestungs\_ID}} edge (ausruestung);
                \node[attribute](material)[below right = of ausruestung]{Material} edge (ausruestung);
                \node[attribute](farbe)[below left = of ausruestung]{Farbe} edge (ausruestung);
              \node[entity](dienststelle)[below right = 6cm of dienstposten]{Dienststelle};
                \node[attribute](dienststellennummer)[below left = of dienststelle]{\key{Dienststellen\_ID}} edge (dienststelle);
                \node[attribute](dienststellennummer)[above left = of dienststelle]{Dienststellennummer} edge (dienststelle);
                \node[attribute](bezeichung)[above right = of dienststelle]{Bezeichnung} edge (dienststelle);
                \node[attribute](groesse)[below right = of dienststelle]{Gr\"o\ss e} edge (dienststelle);
              \node[entity](standort)[below = 6cm of dienststelle]{Standort};
                \node[attribute](ortid)[above = of standort]{\key{Ort\_ID}} edge (standort);
                \node[attribute](plz)[above left = of standort]{PLZ} edge (standort);
                \node[attribute](ort)[above right = of standort]{Ort} edge (standort);
                \node[attribute](strasse)[below right = of standort]{Stra\ss e} edge (standort);
                \node[attribute](hausnummer)[below left = of standort]{Hausnummer} edge (standort);
            \end{tikzpicture}
           }
          \end{center}
        Hinweis: Das Objekt Soldat enth\"alt nicht alle Attribute. Es fehlen die Attribute Personalnummer, Plz, Ort, Stra\ss e und Hausnummer.

    \chapter{Beschreibung sachlogischer Zusammenh\"ange zwischen Objekttypen}
    \setcounter{page}{1}\kapitelnummer{chapter}
    \minitoc
\newpage
    Bisher wurden im Rahmen der Datenmodellierung die speicherw\"urdigen Objekte mit ihren relevanten Eigenschaften lediglich isoliert beschrieben. In der Praxis stehen die interessanten Objekte jedoch in vielf\"altiger Weise miteinander in Zusammenhang: Dienststellen befinden sich an Orten, Soldaten arbeiten in Dienststellen usw. Auch Objekte desselben Typs k\"onnen miteinander in Zusammenhang stehen: Lehrer leiten Lehrer an, Schulen haben Partnerschaften mit anderen Schulen usw. Diese Zusammenh\"ange m\"ussen ebenfalls im Datenmodell dargestellt werden, denn sie bringen wesentliche Aspekte der betrachteten Realit\"at zum Ausdruck.

    Die sachlogischen Zusammenh\"ange zwischen den Objekten werden in drei Gruppen von Beziehungstypen unterteilt:
    \begin{itemize}
      \item \textbf{Bin\"are Beziehungstypen}: Beschreiben den Zusammenhang zwischen jeweils zwei Objekten, die verschiedenen Objekttypen angeh\"oren.
      \item \textbf{Beziehungstypen n-ten Grades}: Sie Beschreiben den Zusammenhang zwischen mehr als zwei Objekten, die verschiedenen Objekttypen angeh\"oren.
      \item \textbf{Rekursiv-Beziehungstypen}: Objekte, die aus demselben Objekttyp stammen, stehen in Zusammenhang.
    \end{itemize}

    Im Folgenden werden nur die bin\"aren oder auch dualen Beziehungstypen erl\"autert. Auf Beziehungstypen h\"oheren Grades wird im
    weiteren Verlauf nicht eingegangen, da diese in der Praxis kaum Relevanz besitzen. Die Behandlung der Rekursiven
    Beziehungstypen erfolgt im Kapitel 3.

      F\"ur den weiteren Verlauf sind die nachfolgenden zwei Definitionen zu unterscheiden:
      \begin{itemize}
        \item \textbf{Beziehung (engl. Relationship)}: Kennzeichnet den konkreten Zusammenhang zwischen zwei realen Objekten. Beispiel: \enquote{Max Mustermann} ist Lehrgangsteilnehmer im Lehrgang \enquote{Datenbank Administrator}
        \item \textbf{Beziehungstypen}: Beschreibt den typm\"a\ss igen Zusammenhang, der zwischen den Objekttypen besteht. Beispiel: Objekttypen sind \enquote{Soldat} \enquote{Lehrgang} mit dem Beziehungstyp \enquote{ist Lehrgangsteilnehmer in}
      \end{itemize}

      W\"ahrend in der realen Welt der Zusammenhang zwischen zwei konkreten Objekten \textbf{beobachtet} wird, \textbf{beschreibt} man in der Modellwelt das verallgemeinerte Wechselspiel zwischen zwei Objekttypen. Im mathematischen Sinne ist ein Beziehungstyp zwischen den Objekten A und B die Menge der Beziehungen zwischen jeweils einem Objekt aus dem Objekttyp A und einem Objekt aus dem Objekttyp B. Die f\"ur den Beziehungstyp formulierten Angaben m\"ussen somit f\"ur alle konkreten Beziehungen zwischen den betrachteten Objekttypen g\"ultig sein.
      \section{Benennung, Optionalit\"at und Kardinalit\"at}\label{naming_optionaliy_kardinality}
        Der sachlogische Zusammenhang zwischen den Objekten zweier Objekttypen, der durch einen Beziehungstyp beschrieben wird, besteht immer in \textbf{beiden Richtungen}: Soldaten stehen beispielsweise in einem Zusammenhang mit Dienststellen (sie arbeiten dort) und Dienststellen stehen im Zusammenhang mit Soldaten (sie besch\"aftigen sie). Jede der beiden Beziehungstyp-Richtungen wird durch 3 Angaben n\"aher bestimmt.
        \subsection{Benennung}
          Betrachtet man im Beispiel aus Kapitel 1.1 den Zusammenhang zwischen Soldat und Ausr\"ustung, k\"onnte man sich f\"ur folgende Dinge interessieren:
          \begin{itemize}
            \item Ein Soldat besitzt Ausr\"ustung.
            \item Ein Soldat empf\"angt seine Ausr\"ustung.
            \item Ein Soldat hat bestimmte Ausr\"ustungsgegenst\"ande mitzuf\"uhren.
          \end{itemize}
          Welcher Zusammenhang gespeichert werden soll, wird durch die Benennung zum Ausdruck gebracht.
          Hier ist es \enquote{besitzt}.
        \subsection{Optionalit\"at}
          Die Optionalit\"at kl\"art die Frage, ob jedes Objekt des Objekttyps A mit mindestens einem Objekt des Objekttyp B in Beziehung stehen muss? Je nach Antwort unterscheidet man zwei F\"alle:
          \begin{itemize}
            \item \textbf{Ja}: Die Beziehungstyp-Richtung wird als nichtoptional, also als \textbf{obligatorisch\footnote{obligare lat. = bindend, verpflichtend}} bezeichnet.
            \item \textbf{Nein}: Die Beziehungstyp-Richtung wird als optional, also \textbf{kann vorhanden sein} bezeichnet.
          \end{itemize}
        \subsection{Kardinalit\"at}
          F\"ur die Angabe der Kardinalit\"at einer Beziehungstyp-Richtung vom Objekttyp A zum Objekttyp B stellt man sich folgende Frage: \enquote{Kann ein Objekt des Objekttyps A mit mehreren Objekten des Objekttyps B in Beziehung stehen?} Es k\"onnen dabei zwei unterschiedliche F\"alle auftreten:
\clearpage
          \begin{itemize}
            \item \textbf{Ja}: Der Beziehungstyp-Richtung wird die Kardinalit\"at \textbf{N} zugeordnet. Dabei steht \textbf{N} f\"ur eine beliebige Zahl gr\"o\ss er oder gleich 0.
            \item \textbf{Nein}: Die Beziehungstyp-Richtung wird die Kardinalit\"at \textbf{1} zugeordnet, denn es gibt \textbf{h\"ochstens ein} Objekt des Objekttyps B zu dem Objekt aus Objekttyp A.
          \end{itemize}
        \subsection{Darstellung im Modell}
          Es gibt verschiedene Formen der Darstellung. Man kann jede Beziehungstyp-Richtung benennen, jedoch ist die Benennung meist nur f\"ur eine Richtung passend. Die Gegenrichtung m\"usste dann bei Bedarf umformuliert werden. Im Folgenden soll ein Beispiel die Zusammeh\"ange zwischen Benennung, Optionalit\"at und Kardinalit\"at zeigen. Gegeben seien die Objekttypen Soldat und Ausr\"ustung mit der angezeigten Beziehung.

          \begin{center}
            \scalebox{.7}{
              \begin{tikzpicture}[node distance=1.5cm, every edge/.style={link}]
                \node[entity](soldat){Soldat};
                  \node[attribute](personenid)[above = 1.6cm of soldat]{\key{Personen\_ID}} edge (soldat);
                  \node[attribute](pk)[above right = of soldat]{PK} edge (soldat);
                  \node[attribute](name)[below right = of soldat]{Name} edge (soldat);
                  \node[attribute](vorname)[below left = of soldat]{Vorname} edge (soldat);
                  \node[attribute](dienstgrad)[above left = of soldat]{Dienstgrad} edge (soldat);
                \node[relationship](besitzt)[right = 3cm of soldat]{besitzt} edge node[auto,swap] {n}(soldat);
                \node[entity](ausruestung)[right = 8cm of soldat]{Ausr\"ustung} edge node [auto,swap] {m}(besitzt);
                  \node[attribute](ausruestungsid)[right = of ausruestung]{\key{Ausruestungs\_ID}} edge (ausruestung);
                  \node[attribute](versorgungsnummer)[below right = of ausruestung]{Versorgungsnummer} edge (ausruestung);
                  \node[attribute](bezeichnung)[above = 1.7cm of ausruestung]{Bezeichnung} edge (ausruestung);
                  \node[attribute](material)[above left = of ausruestung]{Material} edge (ausruestung);
                  \node[attribute](farbe)[below left = of ausruestung]{Farbe} edge (ausruestung);
            \end{tikzpicture}
            }
          \end{center}

        Zu diesem Ausschnitt aus einem ER-Modell ergeben sich die drei folgenden Fragen:
        \begin{enumerate}
          \item Welcher Zusammenhang soll ausgedr\"uckt werden (\textbf{Benennung})?

          Durch die Benennung der beiden Objekttypen mit \enquote{Soldat} und \enquote{Ausr\"ustung} zeigt sich der Zusammenhang, dass ein Soldat Ausr\"ustung besitzt.
          \item Sind die beiden Objekttypen aneinader gebunden (\textbf{Optionalit\"at})?

          In diesem Beispiel ist es so, dass ein Soldat Ausr\"ustung besitzen kann aber nicht muss, z.B. vor der Einkleidung ist man schon Soldat, obwohl noch jegliche Ausr\"ustungsgegenst\"ande fehlen. Anders herum ist es m\"oglich, dass Ausr\"ustungsgegenst\"ande einem Soldaten
          zugeordnet wurden oder der Ausr\"ustungsgegenstand noch im Lager liegt.
          Antwort: Beide Richtungen sind optional.
          \item Wie stehen die beiden Objekttypen in Zusammenhang (\textbf{Kardinalit\"at})?

          \begin{itemize}
            \item Fragerichtung vom Objekttyp \enquote{Soldat}  zum Objekttyp \enquote{Ausr\"ustung}:
            \enquote{Kann ein Soldat mehrere Ausr\"ustungsgegenst\"ande besitzen?}, Antwort: Ja, daher N.
            \item Fragerichtung vom Objekttyp \enquote{Ausr\"ustung}  zu \enquote{Soldat}:

            \enquote{Kann ein Ausr\"ustungsgegenstand von mehreren Soldaten besessen werden?}
            Antwort: Ja, d. h. die Kardinalit\"at lautet M.
          \end{itemize}
        \end{enumerate}
        Ein Ausr\"ustungsgegenstand ist durch eine Versorgungsnummer
        gekennzeichnet. Somit ist es m\"oglich, unterschiedliche
        Ausr\"ustungsgegenst\"ande mit der gleichen Versorgnugnsnummer an die
        Soldaten auszugeben. Es entsteht eine \enquote{N:M} Kardinalit\"at (vgl.
        \ref{NotationKardinalitaet}).

        Bei der Festlegung der Kardinalit\"at ist au\ss erdem zu beachten,
        \"uber welchen Zeitraum hinweg die Angaben zu Beziehungen in der
        Datenbank aufgenommen werden sollen. Bei der Beziehung \enquote{Soldat
        arbeitet in Dienststelle}, ist die Kardinalit\"at auf 1 zu setzen, wenn
        der Soldat immer nur in einer Dienststelle arbeiten soll. Will man aber
        die Zuordnungsverh\"altnisse \"uber einen l\"angeren Zeitraum speichern,
        so ist die Kardinalit\"at auf N festzulegen, weil es dann vorkommen
        kann, dass ein Soldat mit mehreren Dienststellen in Verbindung gebracht
        werden muss, also eine Historie gespeichert wird.
      \section{Notationen f\"ur Kardinalit\"aten} \label{NotationKardinalitaet}
        F\"ur die Kardinalit\"at gibt es verschiedene Notationen, also
        einheitliche Schreibweisen. In dieser Unterlage wird die Chen-Notation
        kurz vorgestellt und die (Min,Max)-Notation eingehender behandelt.
        Sofern die Chen-Notation von Interesse ist, kann diese in vielen
       Fachb\"uchern leicht nachgelesen werden.
        \subsection{Die Chen-Notation}
          In der Chen-Notation gibt es im Wesentlichen drei verschiedene
          Beziehungstypen, dabei ist es unerheblich, ob die verwendeten
          Buchstaben gro\ss\ oder klein geschrieben werden. Die Werte geben die
          maximale Anzahl von beteiligten Objekten an.

          M\"ogliche Varianten (bin\"are Beziehungen):
          \begin{itemize}
            \item 1:1
            \item 1:n
            \item n:m
            \item n:m:k (tern\"are Beziehungen)
          \end{itemize}
          Welche Werte k\"onnen angenommen werden bzw. wof\"ur stehen die
          Buchstaben?
          \begin{itemize}
            \item n: beliebig viele (0,1,2...n)
            \item 1: h\"ochstens ein (0 oder 1)
            \item m oder k: entsprichen der n-Definition
          \end{itemize}
          Die Angabe von genaueren Werten ist in der Chen-Notation nicht
          vorgesehen.
        \subsection{(Min,Max)-Notation}
          Die (Min,Max)-Notation ist im Wesentlichen eine Konkretisierung der
          Angaben in der Chen-Notation, denn es werden nicht nur die maximalen,
          sondern auch die minimalen Werte angegeben. Folgende Abbildung zeigt
          die oben verwendete Beziehung in der (Min,Max)-Notation.
          \begin{center}
            \scalebox{.7}{
              \begin{tikzpicture}[node distance=1.5cm, every edge/.style={link}]
                \node[entity](soldat){Soldat};
                  \node[attribute](personenid)[above = 1.6cm of soldat]{\key{Personen\_ID}} edge (soldat);
                  \node[attribute](pk)[above right = of soldat]{PK} edge (soldat);
                  \node[attribute](name)[below right = of soldat]{Name} edge (soldat);
                  \node[attribute](vorname)[below left = of soldat]{Vorname} edge (soldat);
                  \node[attribute](dienstgrad)[above left = of soldat]{Dienstgrad} edge (soldat);
                \node[relationship](besitzt)[right = 3cm of soldat]{besitzt} edge node[auto,swap] {(0,*)}(soldat);
                \node[entity](ausruestung)[right = 8cm of soldat]{Ausr\"ustung} edge node [auto,swap] {(0,*)}(besitzt);
                  \node[attribute](ausruestungsid)[above right = of ausruestung]{\key{Ausruestungs\_ID}} edge (ausruestung);
                  \node[attribute](versorgungsnummer)[below right = of ausruestung]{Versorgungsnummer} edge (ausruestung);
                  \node[attribute](bezeichnung)[above = 1.7cm of ausruestung]{Bezeichnung} edge (ausruestung);
                  \node[attribute](material)[above left = of ausruestung]{Material} edge (ausruestung);
                  \node[attribute](farbe)[below left = of ausruestung]{Farbe} edge (ausruestung);
            \end{tikzpicture}
            }
          \end{center}

          Welche Werte k\"onnen angenommen werden bzw. wof\"ur stehen diese Werte? Bei der Min, Max-Notation gibt es eine Vielzahl von M\"oglichkeiten, diese hier aufzulisten, w\"are schier unm\"oglich. Daher einige h\"aufige Kombinationen.

          \tablefirsthead{%
            \multicolumn{1}{l}{\textbf{M\"oglichkeit}} &
            \multicolumn{1}{l}{\textbf{Bedeutung}} \\
          }
          \begin{supertabular}[ht]{lp{11cm}}
            (1,1) & genau 1 $\Longrightarrow$ mindestens 1 und h\"ochstens 1\\
            (1,*) & mindestens 1 $\Longrightarrow$ mindestens 1 und h\"ochstens beliebig viele\\
            (0,1) & h\"ochstens 1 $\Longrightarrow$ mindestens 0 und h\"ochstens 1\\
            (0,*) & kann haben $\Longrightarrow$ 0 oder beliebig viele\\
            \\
            (2,100) & mindestens 2 und h\"ochstens 100\\
            (4,*) & mindestens 4 und h\"ochstens beliebig viele\\
          \end{supertabular}

          Die beiden letzten Zeilen der obigen Auflistung sollen verdeutlichen, dass f\"ur die Min- und Max-Werte beliebige ganze Zahlen verwendet werden k\"onnen. Hier ist jedoch zu beachten, dass die Umsetzung bestimmter Kombinationen in den Kardinalit\"aten in einem relationalen Datenbanksystem nicht mehr mit der referentiellen Integrit\"at sichergestellt werden kann, sondern mit Elementen einer Programmiersprache auf Seiten der Anwendung oder der Datenbank. Sp\"ater dazu mehr.
          \subsection{Schreib-/Leseweise der Kardinalit\"aten}
          F\"ur die Syntax der Kardinalit\"aten gilt folgende Vorgehensweise.
          \begin{itemize}
            \item Die (Min,Max)-Notation: Man betrachtet zun\"achst ein Objekt a auf der Seite des Objekttyps A und schreibt die Kardinalit\"at auf die gleiche Seite des Beziehungstyps, an den Objekttyp A.
            \item (Chen)-Notation: Inverse Bezeichnung der Kardinalit\"aten wie bei der (Min,Max)-Notation. Die Kardinalit\"at des Objekts a auf der Seite des Objekttyps A wird auf die andere Seite des Beziehungstyps, also Objekttyp B, geschrieben.
            \item Anschlie\ss end in gleicher Weise am Objekttyp B f\"ur die jeweilige Notation.
          \end{itemize}
          \subsection{Referentielle Integrit\"at}
            Die Referentielle Integrit\"at stellt einen Satz aus zwei Regeln dar, der dazu dient, den korrekten Zusammenhang zwischen den Datens\"atzen zweier Tabellen zu regeln. Sie besagt:
            \begin{enumerate}
              \item Datens\"atze einer untergeordneten Entit\"at d\"urfen nur auf existierende Datens\"atze ihrer \"ubergeordneten Entit\"at verweisen.
              \item Datens\"atze aus einer \"ubergeordneten Entit\"at d\"urfen nur dann gel\"oscht werden, wenn es keine abh\"angigen Datens\"atze in einer untergeordneten Entit\"at mehr gibt.
            \end{enumerate}
            Hierzu ein Beispiel:
          \begin{center}
           \scalebox{.68}{
            \begin{tikzpicture}[node distance=1.5cm, every edge/.style={link}]
              \node[entity](dienststelle){Dienststelle};
                \node[attribute](dienststellennummer)[left = of dienststelle]{\key{Dienststellen\_ID}} edge (dienststelle);
                \node[attribute](dienststellennummer)[below left = of dienststelle]{Dienststellennummer} edge (dienststelle);
                \node[attribute](bezeichung)[above = of dienststelle]{Bezeichnung} edge (dienststelle);
                \node[attribute](groesse)[above left = of dienststelle]{Gr\"o\ss e} edge (dienststelle);
              \node[relationship](besetzt)[right = 3cm of dienststelle]{besetzt} edge node[auto,swap] {(1,*)}(soldat);
              \node[entity](dienstposten)[right = 3cm of besetzt]{Dienstposten} edge node[auto,swap] {(1,1)} (besetzt);
                \node[attribute](dpid)[above = of dienstposten]{\key{DP\_ID}} edge (dienstposten);
                \node[attribute](dienstpostenid)[above left = of dienstposten]{Dienstposten\_ID} edge (dienstposten);
                \node[attribute](beginndatum)[above right = of dienstposten]{Beginndatum} edge (dienstposten);
                \node[attribute](enddatum)[below right = of dienstposten]{Enddatum} edge (dienstposten);
                \node[attribute](aufgabenbeschreibung)[below left = of dienstposten]{Aufgabenbeschreibung} edge (dienstposten);
            \end{tikzpicture}
           }
          \end{center}
          In diesem Beispiel stehen die beiden Entit\"aten Dienststelle und Dienstposten in Zusammenhang. Die Entit\"at Dienststelle ist dabei der Entit\"at Dienstposten \"ubergeordnet. Wendet man die Regeln der Referentiellen Integrit\"at an, bedeutet dies:
          \begin{enumerate}
            \item Es darf keinen Dienstposten geben, der zu einer nicht existenten Dienststelle geh\"ort.
            \item Es darf keine Dienststelle gel\"oscht werden, zu der es noch Dienstposten gibt.
          \end{enumerate}
      \section{Redundante Beziehungstypen}
        L\"asst man den Vergleich von einem Objekttyp mit einer Dateninsel zu, kann man folgendes Bild aufbauen. Die Objekttypen werden als Dateninseln dargestellt und die Beziehungstypen bilden die Br\"ucken zwischen diesen Inseln. Mit Hilfe der Beziehungen, die ja konkrete Auspr\"agungen der Beziehungstypen darstellen, kann man nun eine Br\"uckenwanderung durchf\"uhren, indem man von den Eigenschaften eines Objektes zu den Eigenschaften des verkn\"upften Objektes gelangen kann.

        Nun k\"onnen die Br\"ucken aber so angelegt sein, dass es von Dateninsel A nach Dateninsel B zwei (oder mehr) verschiedene Wege gibt. Sind dann eine oder mehrere Br\"ucken \"uberfl\"ussig? Bei der Datenmodellierung spricht man in solchen F\"allen von \textbf{redundanten Beziehungstypen}. Das sind Beziehungstypen, die einen sachlogischen Zusammenhang zwischen zwei Objekttypen beschreiben, der bereits durch die Kombination anderer Beziehungstypen in gleicher Weise zum Ausdruck gebracht wird.

        Der Begriff der Redundanz spielt bei der Informationsspeicherung eine gro\ss e Rolle. Im praktischen Datenbankbetrieb wird zum Teil Redundanz erzeugt, um Suchprozesse innerhalb der Datenbank zu beschleunigen. In der Phase der Datenmodellierung sollte man Redundanzen vermeiden, denn diese f\"uhren u. a. zu folgenden Problemen:
        \begin{itemize}
          \item Mehrfache Eingabe derselben Informationen
          \item Unn\"otiger Speicherplatzbedarf
          \item Bei der \"Anderung der Informationen muss garantiert werden, dass alle Exemplare der redundant gespeicherten Information ge\"andert werden, weil sonst sog. inkonsistente Daten vorliegen.
        \end{itemize}

        Der Verdacht auf einen redundanten Beziehungstyp ergibt sich i.d.R. bei zyklischen Be\-zieh\-ungs\-typ-Struk\-turen. Kann man nun aber allein aus strukturellen Merkmalen des Datenmodells die Redundanz ableiten? Wenn dies so w\"are, k\"onnten automatisierte Optimierungsprozesse diese Redundanz wieder entfernen. Zur Verdeutlichung soll das in der folgenden Abbildung gezeigte Beispiel untersucht werden.

          \begin{center}
            \scalebox{.7}{
              \begin{tikzpicture}[node distance=1.5cm, every edge/.style={link}]
                \node[entity](soldat){Soldat};
                \node[relationship](besetzt)[right = of soldat]{besetzt} edge node[auto,swap] {(1,1)}(soldat);
                \node[entity](dienstposten)[right = of besetzt]{Dienstposten} edge node [auto,swap] {(1,1)}(besetzt);
                \node[relationship](gehoertzu)[below = of dienstposten]{geh\"ort zu} edge node [auto,swap] {(1,1)} (dienstposten);
                \node[entity](dienststelle)[below = of gehoertzu]{Dienststelle} edge node [auto,swap] {(1,*)}(gehoertzu);
            \end{tikzpicture}
            }
          \end{center}

          Ein Soldat besetzt genau einen Dienstposten und ein Dienstposten kann zu gleichen Zeit auch immer nur von einem Soldaten besetzt werden. Ein Dienstposten geh\"ort zu genau einer Dienststelle, wobei eine Dienststelle aus mindestens einem Dienstposten bestehen muss, um die Sinnhaftigkeit des Modells zu wahren.

          Nun kommt die \enquote{arbeitet f\"ur}-Beziehung hinzu, so dass auf
          Grund der entstehenden zyklischen Struktur zwei Wege vom Soldaten zur
          Dienststelle f\"uhren. Ist dieser Beziehungstyp nun redundant?
          Betrachten wir zwei verschiedene Interpretationen dieses
          Beziehungstyps.

          Ein Soldat arbeitet f\"ur genau eine Dienststelle. F\"ur eine
          Dienststelle arbeitet mindestens ein Soldat.
          \begin{center}
            \scalebox{.68}{
              \begin{tikzpicture}[node distance=1.5cm, every edge/.style={link}]
                \node[entity](soldat){Soldat};
                \node[relationship](besetzt)[right = of soldat]{besetzt} edge
                node[auto,swap] {(0,1)}(soldat);
                \node[entity](dienstposten)[right = of besetzt]{Dienstposten}
                edge node [auto,swap] {(1,1)}(besetzt);
                \node[relationship](gehoertzu)[below = of dienstposten]{geh\"ort
                zu} edge node [auto,swap] {(1,1)} (dienstposten);
                \node[entity](dienststelle)[below = of gehoertzu]{Dienststelle}
                edge node [auto,swap] {(1,*)}(gehoertzu);
                \node[relationship](arbeitetfuer)[below = of soldat]{arbeitet
                f\"ur} edge node [auto,swap] {(1,1)} (soldat); \draw[link]
                (dienststelle.west) -| node [pos=0.4,auto,xshift=4cm] {(1,*)}
                (arbeitetfuer.south);
            \end{tikzpicture}
            }
          \end{center}
          In diesem Falle w\"are der Beziehungstyp  \enquote{arbeitet f\"ur} redundant, denn aus der Tatsache, dass ein Soldat einen Dienstposten besetzt und der Dienstposten zu einer Dienststelle geh\"ort, folgt stets die Aussage, dass der Soldat f\"ur eine Dienststelle arbeitet.

          Ein Soldat leitet h\"ochstens eine Dienststelle und eine Dienststelle wird von genau einem Soldaten geleitet.
          \begin{center}
            \scalebox{.68}{
              \begin{tikzpicture}[node distance=1.5cm, every edge/.style={link}]
                \node[entity](soldat){Soldat};
                \node[relationship](besetzt)[right = of soldat]{besetzt} edge node[auto,swap] {(1,1)}(soldat);
                \node[entity](dienstposten)[right = of besetzt]{Dienstposten} edge node [auto,swap] {(1,1)}(besetzt);
                \node[relationship](gehoertzu)[below = of dienstposten]{geh\"ort zu} edge node [auto,swap] {(1,1)} (dienstposten);
                \node[entity](dienststelle)[below = of gehoertzu]{Dienststelle} edge node [auto,swap] {(1,*)}(gehoertzu);
                \node[relationship](arbeitetfuer)[below = of soldat]{leitet} edge node [auto,swap] {(0,1)} (soldat);
                \draw[link] (dienststelle.west) -| node [pos=0.4,auto,xshift=4cm] {(1,1)} (arbeitetfuer.south);
            \end{tikzpicture}
            }
          \end{center}
          Der Beziehungstyp ist jetzt nicht redundant, weil aus der Tatsache, dass ein Soldat einen Dienstposten besetzt und der Dienstposten zu einer Dienststelle geh\"ort, nicht in jedem Falle folgt, dass der Soldat die Dienststelle leitet.

          Das Beispiel zeigt, dass sich die Frage, ob ein Beziehungstyp redundant ist, nicht auf Grund der Struktur des Datenmodells beantworten l\"asst, sondern dass sie nur durch eine inhaltliche Betrachtung der Zusammenh\"ange entschieden werden kann.
      \section{Parallele Beziehungstypen}
        H\"aufig ist es bei der Sammlung von Informationen der Fall, dass unterschiedliche sachlogische Zusammenh\"ange zwischen zwei Objekttypen A und B zu ber\"ucksichtigen sind. Dies geschieht, in dem man mehrere Beziehungstypen zwischen A und B einf\"ugt. Diese werden dann als \textbf{parallele Beziehungstypen} bezeichnet.

        Sind nun Optionalit\"at und Kardinalit\"at der jeweiligen Beziehungstyp-Richtungen durch die beteiligten Objekttypen vorgegeben?

        Nehmen wir an Sie wollen wie in der folgenden Abbildung dargestellt, f\"ur eine Personengruppe und eine definierte Menge von Autos drei Beziehungstypen modellieren.

        Eine Person muss weder Eigent\"umer noch Halter noch Benutzer eines der betrachteten Autos sein, sie kann aber auch Eigent\"umer, Halter und Benutzer mehrerer Autos sein. Andererseits muss ein Auto mindestens einen, kann aber auch mehrere Eigent\"umer haben. Es kann keinen Halter haben, wenn es stillgelegt wurde, sonst aber h\"ochstens einen. Es kann im betrachteten Zeitraum von keinem, aber auch von mehreren Personen benutzt werden. Man sieht, das die Optionalit\"at und Kardinalit\"at nicht allein durch die beteiligten Objekttypen festgelegt sind, sondern, dass sie durch die spezielle Semantik des jeweiligen sachlogischen Zusammenhangs bestimmt werden.
        \begin{center}
          \scalebox{.8}{
            \begin{tikzpicture}[node distance=1.5cm, every edge/.style={link}]
              \node[entity](auto){Auto};
              \node[relationship](benutzt)[below left = 2.75cm of auto]{benutzt};
              \node[relationship](besitzt)[below = of auto]{besitzt} edge node [auto,swap] {(1,*)} (auto);
              \node[relationship](haelt)[below right = 3.2cm of auto]{h\"alt};
              \node[entity](person)[below = of besitzt]{Person} edge node [auto,swap] {(0,*)}(besitzt);
              \draw[link] (person.west) -| node [pos=0.4, auto, swap] {(0,*)} (benutzt);
              \draw[link] (person.east) -| node [pos=0.4, auto] {(0,*)} (haelt);
              \draw[link] (auto.west) -| node [pos=0.4, auto, swap] {(0,*)} (benutzt);
              \draw[link] (auto.east) -| node [pos=0.4, auto] {(0,1)} (haelt);
          \end{tikzpicture}
          }
        \end{center}

      \section{Mehrfachbeziehungen}
        Ein Objekttyp kann nicht nur mit einer, sondern mit beliebig vielen anderen Objekttypen in Beziehung stehen. Wenn man den Spezialfall \enquote{Parallele Beziehungstypen} ausklammert, so l\"asst sich folgendes Beispiel aufzeichnen.

          \begin{center}
            \scalebox{.7}{
              \begin{tikzpicture}[node distance=1.5cm, every edge/.style={link}]
                \node[entity](obj1){Objekt 1};
                \node[relationship](rel1)[below = of obj1]{} edge node[auto,swap] {(1,1)}(obj1);
                \node[entity](obj5)[below = of rel1]{Objekt 5} edge node [auto,swap] {(0,*)}(rel1);
                \node[relationship](rel2)[right = of obj5]{} edge node[auto,swap] {(1,1)}(obj5);
                \node[entity](obj2)[right = of rel2]{Objekt 2} edge node [auto,swap] {(0,*)}(rel2);
                \node[relationship](rel3)[below = of obj5]{} edge node[auto,swap] {(1,1)}(obj5);
                \node[entity](obj3)[below = of rel3]{Objekt 3} edge node [auto,swap] {(0,*)}(rel3);
                \node[relationship](rel4)[left = of obj5]{} edge node[auto,swap] {(1,1)}(obj5);
                \node[entity](obj4)[left = of rel4]{Objekt 4} edge node [auto,swap] {(0,*)}(rel4);
            \end{tikzpicture}
            }
          \end{center}

        Der Objekttyp 5 steht hier mit vier anderen Objekttypen in Beziehung.
        Auch die \"ubrigen Objekttypen k\"onnen mit anderen Objekttypen in
        Beziehung stehen. Eine evtl. Problematik ergibt sich erst durch die
        Transformation der Beziehungstypen, da bei diesem Prozess weitere
        Fremdschl\"ussel, in die dann entstandenen Tabellen aufgenommen werden
        m\"ussen. Die Transformation wird im Kapitel 4 ausf\"uhrlich behandelt.
      \section{Eigenschaften von Beziehungstypen}
        \label{attributes_of_entities}
        H\"aufig besteht die Notwendigkeit, die konkrete Beziehung, die zwei
        Objekte des betrachteten Gegenstandsbereichs eingehen, genauer zu
        spezifizieren. Betrachten wir dazu folgenden Fall:

        In der Bekleidungsstammkarte eines Soldaten werden Informationen
        dar\"uber gespeichert, welcher Soldat welchen Ausr\"ustungsgegenstand
        empfangen hat. Nun soll das Datum der Ausgabe des
        Ausr\"ustungsgegenstandes gespeichert werden - in unserem Beispiel durch
        das Attribut \enquote{Ausgabedatum}. Diese Eigenschaft kann aber weder
        dem Objekttyp \enquote{Soldat}, noch dem Objekttyp
        \enquote{Ausr\"ustung} zugeordnet werden. Es ist eine Eigenschaft des
        Beziehungstyps, der zwischen Soldat und Ausr\"ustung besteht. Folgende
        Abbildung stellt diese Situation dar.

          \begin{center}
            \scalebox{.7}{
              \begin{tikzpicture}[node distance=1.5cm, every edge/.style={link}]
                \node[entity](soldat){Soldat};
                \node[relationship](empfaengt)[right = of soldat]{empf\"angt} edge node[auto,swap] {(0,*)}(soldat);
                  \node[attribute](ausgabedatum)[below = of empfaengt]{Ausgabedatum} edge (empfaengt);
                \node[entity](ausruestung)[right = of empfaengt]{Ausr\"ustung} edge node [auto,swap] {(0,*)}(empfaengt);
            \end{tikzpicture}
            }
          \end{center}

      \section{Begriffe}
        An dieser Stelle soll eine Terminologie eingef\"uhrt werden, die beim Datenbank-Design \"ublich ist und in vielen F\"allen eine k\"urzere Sprechweise erm\"oglicht:
        \begin{itemize}
          \item \textbf{Schl\"ussel}: Die minimale Kombination von Eigenschaften / Attributen durch die die Objekte eines Objekttyps eindeutig identifiziert werden k\"onnen, wird als Schl\"us\-sel des Objekttyps bezeichnet. Ein Schl\"ussel kommt somit nicht doppelt vor. Der Eigenschaftswert des Schl\"ussels eines Objekttyps darf nicht leer sein.
          \item \textbf{Zusammengesetzter Schl\"ussel}: Ein Schl\"ussel, der sich aus mehreren Eigenschaften / Attributen zusammensetzt wird zusammengesetzter Schl\"ussel genannt. H\"aufig sind die verwendeten Eigenschaften Fremdschl\"ussel (siehe \ref{basics_definitions}).
          \item \textbf{Teilschl\"ussel}: Ein Teilschl\"ussel entsteht dadurch, dass man aus einem zusammengesetzten Schl\"ussel wenigstens ein teilidentifizierendes Element (Attribut) entfernt.
        \end{itemize}
        Eine weitere und feinere Unterteilung erfolgt im Kapitel \ref{basics_definitions} (Transformation).
\clearpage

        \section{\"Ubungen - Einfache ER-Modellierung}
            \subsection{\"Ubungsaufgabe Sportverein}
        Entwerfen Sie, basierend auf der folgenden Lage, ein ER-Modell inklusive der Beziehungen zwischen
        den Entit\"{a}ten.

        Ein Sportverein will zur besseren Verwaltung seiner eigenen Sportabteilungen, Trainer und
        Sportler eine Datenbank entwerfen. In der Datenbank soll ersichtlich werden, welcher Trainer
        welche Sportart trainiert und welcher Sportler in der jeweiligen Abteilung aktiv ist.

        Im Folgenden werden die angesprochenen Datenbankinhalte spezifiziert:
        \begin{itemize}
          \item Zu jeder Sportabteilung muss eine eindeutige ID und deren vereinsinterne
          Bezeichnung gespeichert werden.
          \item F\"ur jede Sportart ist eine ID und deren Bezeichnung wichtig. Eine Sportart wird in genau einer
          Abteilung durchgef\"uhrt, wobei in einer Abteilung mindestens eine Sportart durchgef\"uhrt wird.
          \item F\"ur den jeweiligen Trainer ist der Vor- und Nachname, das Geburtsdatum
          und das Eintrittsdatum in den Verein zu speichern. Ein Trainer trainiert mindestens eine Sportart.
          Eine Sportart wird von h\"ochstens einem Trainer trainiert, wobei es vorkommen kann, dass beim
          Ausscheiden eines Trainers aus dem Verein, eine Sportart kurzzeitig keinen Trainer hat.
          \item Jede Sportart wird von mindestens einem Sportler ausge\"ubt. Ein Sportler hingegen kann mehrere
          Sportarten aus\"uben, wobei es keine passiven Mitglieder/Sportler im Verein gibt. Bis auf die
          Trainernummer sind f\"ur den Sportler die selben Daten zu erheben, wie f\"ur die Trainer.
          \item Jede Sportabteilung hat mindestens einem Trainingsort (Adresse). Es kann sein, dass an
          einem Trainingsort mehrere Sportabteilungen trainieren. Bei der Adressspeicherung sind die
          Postleitzahl (PLZ), der Ort, die Stra\ss e und die Hausnummer relevant.
          \item Jeder Trainer und jeder Sportler wohnen bei genau einer Adresse. Es ist auch m\"oglich,
          dass mehrere Trainer oder Sportler an der selben Adresse wohnen.
        \end{itemize}
\clearpage

            \subsection{\"Ubungsaufgabe IT-Helpdesk}
        Entwerfen Sie, basierend auf der folgenden Lage, ein ER-Modell inklusive der Beziehungen zwischen den Entit\"aten.

        F\"ur ein IT-Supportunternehmen soll eine Datenbank erschaffen werden, welche es erm\"oglicht, telefonische Supportanfragen von Kunden zu erfassen. Im Einzelnen m\"ussen die nachfolgend beschriebenen Zusammenh\"ange in der Datenbank abgebildet werden.

        \subsubsection{Vorgaben}
          \begin{itemize}
            \item Jeder Kunde, der den IT-Helpdesk anruft, muss mit Vorname, Nachname und Kundennummer
            gespeichert werden.
            \item Die Datenbank muss es erm\"oglichen, zu einem Kunden, mindestens eine oder mehrere
            Adressen zu speichern. Eine Adresse besteht aus Stra\ss{}e, Hausnummer, Postleitzahl (PLZ) sowie Ort und muss mehreren Kunden zugeordnet werden k\"onnen. Adressen zu denen keine Kunden mehr
            in der Datenbank existieren verbleiben noch f\"ur mindestens ein Jahr in der Datenbank, ehe
            sie gel\"oscht werden.
            \item Ein Kunde gibt beim IT-Helpdesk seine Kontaktdaten an. Diese Kontaktdaten bestehen
            meist aus Telefonnummer und E-Mail-Adresse. Die Telefonnummer muss nicht zwingend mit angegeben werden. Allerdings muss mind. eine Kontaktinformation gespeichert werden. Ein Kontaktdatensatz wird immer nur einem Kunden zugeordnet.
            \item F\"ur das Management des IT-Supportunternehmen  ist es wichtig zu wissen, welcher
            Mitarbeiter des Helpdesks mit welchem Kunden Kontakt hatte. Zu einem Mitarbeiter wird dessen
            Vorname, Nachname und seine Personalnummer gespeichert.
            \item Jedesmal wenn ein Kunde mit dem IT-Helpdesk einen Kontakt herstellt, muss der
            Mitarbeiter des Helpdesks die genaue Uhrzeit, das Datum, den Anlass und die Dauer der
            Dienstleistung notieren.
          \end{itemize}
          \subsubsection{Zusatzaufgabe}
            Das IT-Supportunternehmen hat angefragt, ob es m\"oglich ist, die Datenbank so zu ver\"andern,
            dass mehrere Mitarbeiter an einem Kontakt arbeiten k\"onnen. Jedesmal wenn ein Mitarbeiter an
            einem Kontakt arbeitet, m\"ussen die Uhrzeit und das Datum des Telefonats, sowie dessen Dauer
            gespeichert werden.

            Pr\"ufen Sie, ob diese M\"oglichkeit gegeben ist und falls Ja, passen Sie das Modell
            entsprechend an!
\clearpage
      \input{../modellierung/uebungen/modellierung_unternehmensberatung_uebung}

        \section{L\"osungen - Einfache ER-Modellierung}
      \input{../modellierung/loesungen/modellierung_sportverein_loesung}
      \input{../modellierung/loesungen/modellierung_it-helpdesk_loesung}
      \input{../modellierung/loesungen/modellierung_unternehmensberatung_loesung}
    \chapter{Erweiterte ER Modellierung}
    \setcounter{page}{1}\kapitelnummer{chapter}
    \minitoc
\newpage
    In diesem Abschnitt wird das ER-Modell um die drei Eigenschaften Rekursion, Spezialisierung und Generalisierung erweitert, so dass eine umfangreichere und genauere Beschreibung des betrachteten Gegenstandbereichs m\"oglich ist.

    \section{Rekursive Beziehungen}
      H\"aufig ist es wichtig und notwendig den sachlogischen Zusammenhang zwischen zwei Objekten festzuhalten, die beide demselben Objekttyp angeh\"oren. So ist es f\"ur ein Unternehmen n\"utzlich zu wissen, welcher Mitarbeiter durch welche anderen Mitarbeiter - im Krankheits- oder Urlaubsfall - vertreten werden kann.

      \subsection{Definition eines rekursiven Beziehungstyps}
        \begin{merke}
          Ein rekursiver Beziehungstyp beschreibt den sachlogischen Zusammenhang zwischen Objekten, die dem gleichen Objekttyp angeh\"oren.
        \end{merke}

        F\"ur einen rekursiven Beziehungstyp sind dieselben Angaben erforderlich, wie f\"ur einen bin\"aren. Somit ist f\"ur einen solchen Beziehungstyp folgendes festzulegen (siehe \ref{naming_optionaliy_kardinality}):
        \begin{enumerate}
          \item eine aussagekr\"aftige Benennung
          \item eine Angabe zur Optionalit\"at
          \item eine Angabe zur Kardinalit\"at
        \end{enumerate}
        Bei der traditionellen Informationsspeicherung, mit Hilfe von Karteik\"artchen, wird der rekursive Beziehungstyp durch Querverweise, innerhalb des Karteikastens, realisiert. Im betrachteten Vertretungsbeispiel w\"urden auf der Karteikarte des Mitarbeiters die Personalnummern seiner m\"oglichen Vertreter notiert werden.

      \subsection{Syntaxregeln f\"ur rekursive Beziehungstypen}
        \label{syntaxofrecursiverelationtypes}
        \begin{itemize}
          \item Ein rekursiver Beziehungstyp zur Kennzeichnung eines sachlogischen Zusammenhangs zwischen den Objekten ein- und desselben Objekttyps A wird als eine Verbindungslinie dargestellt, die den Objekttyp A mit sich selbst verbindet. Diese Form der Verbindungslinie, die aus A austritt und zu A zur\"uckf\"uhrt, ist der Anlass f\"ur die Bezeichnung \enquote{rekursiv}\footnote{rekursiv = zur\"uckf\"uhrend}
          \item Die Benennung des Beziehungstyps steht in einer Raute, am Beziehungstyp (siehe \ref{naming_optionaliy_kardinality})
          \item Optionalit\"at und Kardinalit\"at der jeweiligen Beziehungstyp-Richtung werden in gleicher Weise angegeben, wie beim bin\"aren Beziehungstyp.
        \end{itemize}
        \begin{center}
          \scalebox{.7}{
            \begin{tikzpicture}[node distance=1.5cm, every edge/.style={link}]
              \node[entity](A){Soldat};
              \node[relationship](rel1)[right = of A]{Relation};
              \draw[link] (A.north) |- ($(A.north) +(0.5, 1)$) -| node [pos=0.4, auto, swap,yshift=0.75cm] {(0,1)} (rel1.north);
              \draw[link] (A.south) |- ($(A.south) + (0.5,-1)$) -| node [pos=0.4, auto, yshift=-0.75cm] {(0,*)} (rel1.south);
            \end{tikzpicture}
          }
        \end{center}
        Setzt man beim F\"uhrungsverh\"altnis der Soldaten voraus, dass ein Soldat von h\"ochstens einem anderen Soldaten gef\"uhrt wird und dass er selbst mehrere Soldaten f\"uhren kann, dann ergibt sich das folgende Datenmodell.
        \begin{center}
          \scalebox{.7}{
            \begin{tikzpicture}[node distance=1.5cm, every edge/.style={link}]
              \node[entity](A){Soldat};
              \node[relationship](rel1)[right = of A]{F\"uhrung};
              \draw[link] (A.north) |- ($(A.north) +(0.5, 1)$) -| node [pos=0.4, auto, swap,yshift=0.75cm] {(0,1) wird gef\"uhrt} (rel1.north);
              \draw[link] (A.south) |- ($(A.south) + (0.5,-1)$) -| node [pos=0.4, auto, yshift=-0.75cm] {(0,*) f\"uhrt} (rel1.south);
            \end{tikzpicture}
          }
        \end{center}
        Bei rekursiven Beziehungen ist eine genauere Beschreibung der Beziehungstyp-Richtung notwendig. Im o. a. Beispiel wird der Sachverhalt \enquote{F\"uhrung} durch \enquote{f\"uhrt} und \enquote{wird gef\"uhrt} n\"aher erl\"autert.

        Die Angaben sind folgenderma\ss en zu lesen:
        \begin{itemize}
          \item Ein Soldat \enquote{f\"uhrt} keinen oder mehrere andere Soldaten und
          \item ein Soldat \enquote{wird gef\"uhrt} von h\"ochstens einem anderen Soldaten
        \end{itemize}
        Betrachtet man bei diesem Beziehungstyp Kardinalit\"at und Optionalit\"at, stellt sich nach einiger \"Uberlegung die Frage, ob alle Kombinationen, welche bei den bin\"aren Beziehungstypen existieren, auch bei rekursive wiederzufinden sind.

        Die Besonderheit eines rekursiven Beziehungstyps gegen\"uber einem bin\"aren Beziehungstyp, der Objekttyp A und B miteinander verkn\"upft, besteht darin, dass die Objekttypen A und B identisch sind. Somit sind rekursive Beziehungstypen nur dann m\"oglich, wenn die Objekttypen A und B, die Mengen von Objekten repr\"asentieren, die dieselbe M\"achtigkeit besitzen k\"onnen.
\clearpage
        \begin{merke}
          Die M\"achtigkeit eines Objekttyps beschreibt die genaue Anzahl der in ihm zusammengefassten Objekte.
        \end{merke}
        An dieser Stelle soll dieser Sachverhalt nicht n\"aher erl\"autert, sondern nur das f\"ur die Modellierung relevante Ergebnis betrachtet werden.

        Die \tabelle{combinationsrecurisverelationtyps} \enquote{M\"ogliche Kombinationen von rekursiven Beziehungstypen} zeigt, in der\\ (Min,Max)-Notation, die m\"oglichen Kombinationen von Kardinalit\"at und Optionalit\"at und trifft eine Aussage, \"uber die Verwendungsm\"oglichkeit bei rekursiven Beziehungstypen.

        \tablefirsthead{%
          \multicolumn{1}{l}{\textbf{Beziehungstyp}} &
          \multicolumn{1}{l}{\textbf{Verwendung bei rekursiven Beziehungstypen}} \\
        }
        \tablecaption{M\"ogliche Kombinationen von rekursiven Beziehungstypen}
        \label{combinationsrecurisverelationtyps}
        \begin{supertabular}[h]{lp{5cm}}
          (1,1):(1,1) & m\"oglich\\
          (1,1):(0,1) & nicht m\"oglich\\
          (0,1):(0,1) & m\"oglich\\
          (1,1):(1,*) & nicht m\"oglich\\
          (0,1):(1,*) & nicht m\"oglich\\
          (1,1):(0,*) & m\"oglich\\
          (0,1):(0,*) & m\"oglich\\
          (1,*):(1,*) & m\"oglich\\
          (1,*):(0,*) & m\"oglich\\
          (0,*):(0,*) & m\"oglich\\
        \end{supertabular}

        Die sieben m\"oglichen Kombinationen werden, im sp\"ateren Verlauf
        dieser Unterlage (Kapitel 4 Transformation), im Zusammenhang mit ihrer
        Repr\"asentation im physischen Datenbankmodell, der Reihe nach
        untersucht und durch Beispiele veranschaulicht.
      \section{Anwendung rekursiver Beziehungstypen}
        Hier soll nun gezeigt werden, wie
        \tabelle{combinationsrecurisverelationtyps} hilft, in der Praxis Fehler
        zu vermeiden. Die n\"achste Abbildung zeigt zwei Entit\"aten (Version A
        und B), die das Verh\"altnis der Soldaten bzgl. F\"uhrung und gef\"uhrt
        werden darstellen sollen.
        
        \begin{center}
          \scalebox{.7}{
            \begin{tikzpicture}[node distance=1.5cm, every edge/.style={link}]
              \node[entity](A){Soldat};
              \node[relationship](rel1)[right = of A]{F\"uhrung};
              \draw[link] (A.north) |- ($(A.north) +(0.5, 1)$) -| node [pos=0.4, auto, swap,yshift=0.75cm] {(1,1) wird gef\"uhrt} (rel1.north);
              \draw[link] (A.south) |- ($(A.south) + (0.5,-1)$) -| node [pos=0.4, auto, yshift=-0.75cm] {(1,*) f\"uhrt} (rel1.south);
            \end{tikzpicture}
          }
        \end{center}
         \begin{center}
          \scalebox{.7}{
            \begin{tikzpicture}[node distance=1.5cm, every edge/.style={link}]
              \node[entity](A){Soldat};
              \node[relationship](rel1)[right = of A]{F\"uhrung};
              \draw[link] (A.north) |- ($(A.north) +(0.5, 1)$) -| node [pos=0.4, auto, swap,yshift=0.75cm] {(0,1) wird gef\"uhrt} (rel1.north);
              \draw[link] (A.south) |- ($(A.south) + (0.5,-1)$) -| node [pos=0.4, auto, yshift=-0.75cm] {(0,*) f\"uhrt} (rel1.south);
            \end{tikzpicture}
          }
        \end{center}
        
        Beide Versionen sollen bez\"uglich der n\"achst genannten Fragen
        untersucht werden:
        \begin{itemize}
          \item Was sagen die Versionen aus?
          \item Wird die Realit\"at richtig abgebildet?
          \item Kann man sie realisieren (gem\"a\ss\ \tabelle{combinationsrecurisverelationtyps})?
        \end{itemize}
        \subsection{Version A}
          \subsubsection{Was sagt die Version aus?}
            Ein Soldat f\"uhrt mindestens einen oder mehrere andere Soldaten und ein Soldat wird von genau einem anderen Soldaten gef\"uhrt.

            In dieser Abbildung der Realit\"at gibt es nur Vorgesetzte, da jeder Soldat immer mindestens einen anderen f\"uhrt.

          \subsubsection{Wird die Realit\"at richtig abgebildet?}
            Es gibt laut Version A nur solche Soldaten, die andere f\"uhren. Doch in der Realit\"at ist dies sicherlich nicht der Fall, da z.B. ein Soldat in der Grundausbildung nicht immer  einen anderen Soldaten f\"uhren wird. In der anderen Fragerichtung muss man ebenfalls alle Soldaten betrachten. Die Aussage, dass ein Soldat von genau einem anderen gef\"uhrt wird, betrachtet den Chef der Einheit nicht. Dieser sollte, in Bezug auf die eigene Einheit, keinen Vorgesetzten haben.

            Es zeigt sich, dass die Realit\"at mit sehr hoher Wahrscheinlichkeit nicht korrekt abgebildet wurde. Ausnahmen kann es nat\"urlich jederzeit geben.
          \subsubsection{Kann man sie realisieren?}
            \tabelle{combinationsrecurisverelationtyps} trifft, was die Realisierung in einem relationalem Datenbanksystem angeht, in Zeile 4 die klare Aussage, dass es nicht m\"oglich ist. Man kann anhand der Tabelle schon einen Hinweis auf evtl. Fehler in der Modellierung finden.
        \subsection{Version B}
          \subsubsection{Was sagt die Version aus?}
            Ein Soldat f\"uhrt null oder mehrere Untergebene und ein Soldat wird von null oder h\"ochstens einem anderen Soldaten gef\"uhrt. Dabei kann es sich entweder um den Chef oder einen Untergebenen handeln.
          \subsubsection{Wird die Realit\"at richtig abgebildet?}
            Version B stellt die M\"oglichkeit bereit, dass in der Tabelle Soldat sowohl der Chef als auch die Untergebenen aufgef\"uhrt werden k\"onnen, was sich darin widerspiegelt, dass ein Soldat keinen oder h\"ochstens einen Chef haben kann und ein Soldat keinen oder mehrere Soldaten f\"uhren kann. Dies entspricht, nach der allgemeinen Auffassung, der Realit\"at.
          \subsubsection{Kann man sie realisieren?}
            In \tabelle{combinationsrecurisverelationtyps} gibt in diesem Fall Zeile 7 die Auskunft, dass eine rekursive Beziehung mit diesen Kardinalit\"aten m\"oglich ist. Eine Best\"atigung, dass die Realit\"at mit gro\ss er Wahrscheinlichkeit richtig modelliert wurde. Eine hundertprozentige Aussage \"uber richtig oder falsch kann man in diesem Verfahren jedoch nicht treffen.
    \section{Spezialisierung}
      \label{specialization}
       Die Spezialisierung ist eine Vorgehensweise, die Objekttypen als Mengen
       betrachtet. Es kann f\"ur die ER-Modellierung unter Umst\"anden sinnvoll
       sein, eine Menge in mehrere Teilmengen zu zerlegen. Eine solche Zerlegung
       wird meist dann vorgenommen, wenn die Teilmengen eigene Attribute haben,
       die nur in den jeweiligen Teilmengen vorkommen. Auf diese Weise werden
       leere Felder in der Datenbank, sog. NULL-Werte vermieden, da m\"oglichst
       immer alle Attribute mit Werten bef\"ullt werden. Ein Beispiel hierzu:

       Die Menge aller Personen an einer Universtit\"at enth\"alt die beiden
       Teilmengen \enquote{Student} und \enquote{WiMa}\footnote{WiMa =
       Wissenschaftlicher Mitarbeiter}. Die Menge \enquote{Person} wird als
       \enquote{Supertyp}, der \"ubergeordnetet Objekttyp, bezeichnet und die
       beiden Teilmengen \enquote{Student} und \enquote{WiMa} als
       \enquote{Subtyp}, die untergeordneten Objekttypen.
\clearpage
       Ein Student besitzt eine Matrikelnummer, mit der er an der Universit\"at
       registriert wurde. Der WiMa besitzt diese nicht, bezieht aber ein Gehalt
       f\"ur seine wissenschaftlichen Studienarbeiten. Ein Attribut
       \enquote{Gehalt} macht f\"ur den Studenten keinen Sinn, da er keines
       empf\"angt und der WiMa ben\"otigt keine Matrikelnummer. Durch die
       Trennung von \enquote{Person} in \enquote{Student} und \enquote{WiMa}
       werden unn\"otige NULL-Werte in der Datenbank vermieden.

      Wie so vieles im Leben ist das Spezialisieren von Objekttypen jedoch
      nicht ganz so einfach, wie es im ersten Moment scheint. Bevor auf die
      insgesamt vier verschiedenen Variationen dieses Verfahrens eingegangen
      werden kann, werden noch einige Definitionen ben\"otigt.
      \subsection{Definitionen}
        \subsubsection{Disjunkt}
          Ein System von Subtypen und Supertypen heisst disjunkt, wenn die
          einzelnen Subtypen keine gemeinsamen Elemente haben, d. h. ein
          Datensatz kann nur in einem der Subtypen auftreten.
        \subsubsection{Vollst\"andige \"Uberdeckung}
          Ein System von Subtypen und Supertypen nennt man vollst\"andig, wenn
          der Supertyp keine eigenen Elemente enth\"alt, also jedes Element in
          einem der Subtypen enthalten ist.

          \begin{merke}
            Die beiden Begriffe \enquote{Disjunkt} und \enquote{Vollst\"andige
            \"Uberdeckung} werden als Konsistenzbedingungen bezeichnet.
            Konsistenzbedingungen ergeben sich entweder aus feststehenden
            Tatsachen oder sie m\"ussen durch den Designer definiert werden!
          \end{merke}

      \subsection{Kombinationen}
        Die Unter- und Obermengenbeziehungen lassen sich mit diesen Begriffen in
        vier verschiedene Systeme einteilen:
        \begin{enumerate}
          \item Vollst\"andige \"Uberdeckung und \"Uberlappung nicht zugelassen (= disjunkt, kein gemeinsames Element)
          \item Vollst\"andige \"Uberdeckung und \"Uberlappung zugelassen (= nicht disjunkt, gemeinsame Elemente m\"oglich)
          \item Nicht vollst\"andige \"Uberdeckung und \"Uberlappung nicht zugelassen (= disjunkt)
          \item Nicht vollst\"andige \"Uberdeckung und \"Uberlappung zugelassen (= nicht disjunkt)
        \end{enumerate}
        Es handelt sich bei der Spezialisierung um den Sonderfall eines 1:1
        Beziehungstyps mit besonderer Semantik. Es werden nun die oben
        angesprochenen F\"alle erl\"autert und anhand von Beispielen, die immer
        zwei Untermengen angeben, veranschaulicht. In der Realit\"at kann es
        nat\"urlich weitere spezialisierte Objekttypen geben.
      \subsection{Vollst\"andige \"Uberdeckung und Disjunkt}
        \label{complette_disjunkt}
        Wenn man die Objektmengen dieser 1. Beziehungsart grafisch darstellt,
        ergibt sich diese Abbildung:

        \begin{center}
          \scalebox{.7}{
            \begin{tikzpicture}[node distance=1.5cm]
              \node[auto,swap](C) at (2.25, -2){\huge Menge C};
              \node[circleA](A) at (0,0){Menge A};
              \node[circleB](B) at (4,0){Menge B};
            \end{tikzpicture}
          }
        \end{center}
        Die Objektmenge C besteht hier vollst\"andig aus den Objektmengen A und
        B, d. h. es gibt keine Elemente, die den beiden Mengen A und B nicht
        zuordenbar sind. Des Weiteren gibt es keine Schnittmenge zwischen A und
        B.

        $C := \{x \mid ((x \in A) \wedge (x \notin B)) \vee ((x \in B) \wedge (x \notin A))\}$

        Dieser Fall wird im ER-Modell folgenderma\ss en dargestellt.
        \begin{center}
          \scalebox{.7}{
            \begin{tikzpicture}[node distance=1.5cm, every edge/.style={link}]
              \node[entity](obj1){Supertyp};
              \node[isa](isa) [below = of obj1]{ISA} edge node[auto] {(0,1) (1,1) disjunkt}(obj1);
              \node[entity](obj2)[below left = of isa]{Subtyp 1};
              \node[entity](obj3)[below right = of isa]{Subtyp 2};
              \path[draw, -] (isa.west) -| (obj2.north);
              \path[draw, -] (isa.east) -| (obj3.north);
            \end{tikzpicture}
          }
        \end{center}
        \begin{merke}
          Spezifische oder lokale Attribute h\"angen an den Untermengen, dies sind hier die Subtypen 1 und 2.
        \end{merke}
          Der Supertyp (Person) umfasst alle Angestellten einer Universit\"at.
          Der Subtyp 1 (Student) beinhaltet alle Studenten dieser Universit\"at
          und der Subtyp 2 (WiMa) beinhaltet alle wissenschaftlichen Mitarbeiter
          (WiMa) einer Universit\"at. Es existieren im Supertyp \enquote{Person}
          nur Objekte, deren identifizierendes Attribut als Fremdschl\"ussel
          entweder im Subtyp \enquote{Student} oder im Subtyp \enquote{WiMa}
          vorkommt. Die drei folgenden Tabellen zeigen verschiedene
          Auspr\"agungen der Subtypen.
\clearpage
          \tablefirsthead{%
            \multicolumn{1}{l}{\textbf{Person\_ID}} &
            \multicolumn{1}{l}{\textbf{Vorname}} &
            \multicolumn{1}{l}{\textbf{Nachname}} &
            \multicolumn{1}{l}{\textbf{Geburtsdatum}}\\
          }
          \tablecaption{Person}
          \begin{supertabular}[h]{llll}
            1 & Heinz & Meier & 01.02.1990 \\
            2 & Michael & Schulz & 02.05.1980 \\
            3 & Frank & Bertling & 04.10.1981 \\
            4 & Hans & Fasshauer & 07.09.1991 \\
          \end{supertabular}

          \tablefirsthead{%
            \multicolumn{1}{l}{\textbf{Person\_ID}} &
            \multicolumn{1}{l}{\textbf{Matrikelnummer}} \\
          }
          \tablecaption{Student}
          \begin{supertabular}[h]{lp{5cm}}
            1 & 65420\\
            4 & 66530\\
          \end{supertabular}

          \tablefirsthead{%
            \multicolumn{1}{l}{\textbf{Person\_ID}} &
            \multicolumn{1}{l}{\textbf{Gehalt}} \\
          }

          \tablecaption{WiMa}
          \begin{supertabular}[h]{lp{5cm}}
            2 & 3000\\
            3 & 3150\\
          \end{supertabular}

          Zusammenfassend ergibt sich f\"ur den 1. Fall die folgende Abbildung:

          \begin{center}
            \scalebox{.7}{
              \begin{tikzpicture}
                \node[entity](obj1){Person};
                \node[isa](isa) [below = of obj1]{ISA} edge node[auto] {(0,1) (1,1) disjunkt}(obj1);
                \node[entity](obj2)[below left = of isa]{Student};
                \node[entity](obj3)[below right = of isa]{WiMa};
                \path[draw, -] (isa.west) -| (obj2.north);
                \path[draw, -] (isa.east) -| (obj3.north);
              \end{tikzpicture}
            }
          \end{center}
\vspace{\baselineskip}
          \begin{center}
            \scalebox{.6}{
              \begin{tikzpicture}[node distance=1.5cm]
                \node[auto,swap](C) at (2.25, -2.5){\huge Person};
                \node[circleA](A) at (0,0){Student};
                \node[circleB](B) at (5,0){WiMa};
              \end{tikzpicture}
            }
          \end{center}

          \begin{merke}
            Es m\"ussen zus\"atzliche programmtechnische Ma\ss nahmen getroffen werden, um sicherzustellen, dass jedes Objekt des Supertyps genau ein zugeh\"origes Objekt in Subtyp 1 oder Subtyp 2 besitzt.
          \end{merke}
      \subsection{Vollst\"andige \"Uberdeckung und nicht Disjunkt}
        Wenn man die Objektmengen dieser 2. Beziehungsart grafisch darstellt, ergibt sich die folgende Abbildung:

        \begin{center}
          \scalebox{.6}{
            \begin{tikzpicture}[node distance=1.5cm]
              \node[auto,swap](C) at (1.5,-0.5) {\huge Menge C};
              \node[circleA](A) at (0,2) {Menge A};
              \node[circleB](B) at (2,2) {Menge B};
            \end{tikzpicture}
          }
        \end{center}
        Hier besteht die Objektmenge C ebenfalls vollst\"andig aus den Objektmengen A und B, was widerum hei\ss{}t, dass es keine Elemente gibt, die den beiden Mengen A und B nicht zuordenbar sind.

        Der Unterschied zu Fall 1 ist, dass es hier eine Schnittmenge zwischen A und B gibt.

        Fall 2 wird im ER-Modell sehr \"ahnlich dargestellt, wie Fall 1, nur mit dem Unterschied, dass das Schl\"usselwort \enquote{Disjunkt} entf\"allt.
        \begin{center}
          \scalebox{.6}{
            \begin{tikzpicture}[node distance=1.5cm, every edge/.style={link}]
              \node[entity](obj1){Supertyp};
              \node[isa](isa) [below = of obj1]{ISA} edge node[auto] {(0,1) (1,1)}(obj1);
              \node[entity](obj2)[below left = of isa]{Subtyp 1};
              \node[entity](obj3)[below right = of isa]{Subtyp 2};
              \path[draw, -] (isa.west) -| (obj2.north);
              \path[draw, -] (isa.east) -| (obj3.north);
            \end{tikzpicture}
          }
        \end{center}

          Nun soll das im vorigen Abschnitt eingef\"uhre Beispiel herangezogen werden. Als \"Anderung soll ein WiMa an der selben Universit\"at auch noch studieren k\"onnen.
\vfil
          \tablefirsthead{%
            \multicolumn{1}{l}{\textbf{Person\_ID}} &
            \multicolumn{1}{l}{\textbf{Vorname}} &
            \multicolumn{1}{l}{\textbf{Nachname}} &
            \multicolumn{1}{l}{\textbf{Geburtsdatum}} \\
          }
          \tablecaption{Person}
          \begin{supertabular}[h]{llll}
            1 & Heinz & Meier & 01.02.1990 \\
            2 & Michael & Schulz & 02.05.1980 \\
            3 & Frank & Bertling & 04.10.1981 \\
            4 & Hans & Fasshauer & 07.09.1991 \\
            5 & Tobias & Schreiber & 20.07.1983 \\
          \end{supertabular}
\vfil
          \tablefirsthead{%
            \multicolumn{1}{l}{\textbf{Person\_ID}} &
            \multicolumn{1}{l}{\textbf{Matrikelnummer}} \\
          }
          \tablecaption{Student}
          \begin{supertabular}[h]{lp{5cm}}
            1 & 65420\\
            4 & 66530\\
            5 & 66460\\
          \end{supertabular}

          \tablefirsthead{%
            \multicolumn{1}{l}{\textbf{Person\_ID}} &
            \multicolumn{1}{l}{\textbf{Gehalt}} \\
          }
          \tablecaption{WiMa}
          \begin{supertabular}[h]{lp{5cm}}
            2 & 3000\\
            3 & 3150\\
            5 & 2900\\
          \end{supertabular}

          Die Person mit der Nummer 5 ist ein studierender WiMa, welcher ein Aufbaustudium an der selben Univesit\"at absolviert und geh\"ort sowohl dem Subtyp \enquote{Student} als auch dem Subtyp \enquote{WiMa} an. Als identifizierendes Attribut wird die Eigenschaft \enquote{Person\_ID} verwendet. Es ergibt sich f\"ur das Beispiel die folgende Abbildung:
          \begin{center}
            \scalebox{.7}{
              \begin{tikzpicture}
                \node[entity](obj1){Person};
                \node[isa](isa) [below = of obj1]{ISA} edge node[auto] {(0,1) (1,1)}(obj1);
                \node[entity](obj2)[below left = of isa]{Student};
                \node[entity](obj3)[below right = of isa]{WiMa};
                \path[draw, -] (isa.west) -| (obj2.north);
                \path[draw, -] (isa.east) -| (obj3.north);
              \end{tikzpicture}
            }
          \end{center}
\vspace{\baselineskip}
          \begin{center}
            \scalebox{.7}{
              \begin{tikzpicture}
                \node[circleA](A) at (0,2) {Student};
                \node[circleB](B) at (2,2) {WiMa};
              \end{tikzpicture}
            }
          \end{center}
      \subsection{Nicht vollst\"andige \"Uberdeckung und Disjunkt}
        In Fall Nummer 3 besteht die Obermenge C aus den beiden Teilmengen A und B, jedoch ist es m\"oglich, dass in der Menge C Elemente existieren, die nicht den Mengen A und B angeh\"oren. Da sich A und B disjunkt verhalten existiert keine Schnittmenge zwischen A und B.

        $C := \{x \mid (((x \in A) \wedge (x \notin B)) \vee ((x \in B) \wedge (x \notin A))) \vee ((x \notin A) \wedge (x \notin B))\}$

        Wenn man die Objektmengen dieser 3. Beziehungsart grafisch darstellt, ergibt sich folgende Abbildung:

        \begin{center}
          \scalebox{.6}{
            \begin{tikzpicture}[node distance=1.5cm]
              \node[circleC, scale=1.5](C) at (0,0) {\huge Menge C};
              \node[circleA, scale=0.75](A) at (-1.2,0) {Menge A};
              \node[circleB, scale=0.75](B) at (1.2,0) {Menge B};
            \end{tikzpicture}
          }
        \end{center}
\vspace{\baselineskip}
        \begin{center}
          \scalebox{.7}{
            \begin{tikzpicture}[node distance=1.5cm, every edge/.style={link}]
              \node[entity](obj1){Objekt 1};
              \node[isa](isa) [below = of obj1]{ISA} edge node[auto] {(0,1) (1,1) disjunkt}(obj1);
              \node[entity](obj2)[below left = of isa]{Objekt 2};
              \node[entity](obj3)[below right = of isa]{Objekt 3};
              \path[draw, -] (isa.west) -| (obj2.north);
              \path[draw, -] (isa.east) -| (obj3.north);
            \end{tikzpicture}
          }
        \end{center}
				Es wird wieder auf das Universit\"atsbeispiel zur\"uckgegriffen. Diesmal existieren in dem Supertyp keine Objekte, deren identifizierendes Attribut sowohl im Subtyp \enquote{Student} als auch im Subtyp \enquote{WiMa} vorkommt. Dies wird in den folgenden Tabellen gezeigt.

          \tablefirsthead{%
            \multicolumn{1}{l}{\textbf{Person\_ID}} &
            \multicolumn{1}{l}{\textbf{Vorname}} &
            \multicolumn{1}{l}{\textbf{Nachname}} &
            \multicolumn{1}{l}{\textbf{Geburtsdatum}}\\
          }
          \tablecaption{Person}
          \begin{supertabular}[h]{llll}
            1 & Heinz & Meier & 01.02.1990 \\
            2 & Michael & Schulz & 02.05.1980 \\
            3 & Frank & Bertling & 04.10.1981 \\
            4 & Hans & Fasshauer & 07.09.1991 \\
            5 & Tobias & Schreiber & 20.07.1983 \\
            6 & Martin & Speier & 21.08.1996 \\
          \end{supertabular}

          \tablefirsthead{%
            \multicolumn{1}{l}{\textbf{Person\_ID}} &
            \multicolumn{1}{l}{\textbf{Matrikelnummer}} \\
          }
          \tablecaption{Student}
          \begin{supertabular}[h]{lp{5cm}}
            1 & 65420\\
            4 & 66530\\
            5 & 66460\\
          \end{supertabular}
          \tablefirsthead{%
            \multicolumn{1}{l}{\textbf{Person\_ID}} &
            \multicolumn{1}{l}{\textbf{Gehalt}} \\
          }
          \tablecaption{WiMa}
          \begin{supertabular}[h]{lp{5cm}}
            2 & 3000\\
            3 & 3150\\
          \end{supertabular}

          Die Person mit der Person\_ID 6 kommt in den Subtypen nicht vor, da sie ein Praktikant ist und nur die Attribute des Supertyps in sich vereint.

          Es ergibt sich im Beispiel f\"ur den 3. Fall die folgende Abbildung:
          \begin{center}
            \scalebox{.6}{
              \begin{tikzpicture}
                \node[entity](obj1){Person};
                \node[isa](isa) [below = of obj1]{ISA} edge node[auto] {(0,1) (1,1) disjunkt}(obj1);
                \node[entity](obj2)[below left = of isa]{Student};
                \node[entity](obj3)[below right = of isa]{WiMa};
                \path[draw, -] (isa.west) -| (obj2.north);
                \path[draw, -] (isa.east) -| (obj3.north);
              \end{tikzpicture}
            }
          \end{center}
          \vspace{\baselineskip}
					\begin{center}
            \scalebox{.7}{
              \begin{tikzpicture}
                \node[circleC, scale=1.5](C) at (0,0) {\huge Menge C};
                \node[circleA, scale=0.75](A) at (-1.2,0) {Menge A};
                \node[circleB, scale=0.75](B) at (1.2,0) {Menge B};
              \end{tikzpicture}
            }
          \end{center}
      \subsection{Nicht vollst\"andige \"Uberdeckung und nicht Disjunkt}
        Wenn man die Objektmengen dieser 4. Beziehungsart grafisch darstellt, ergibt sich folgende Abbildung:
        \begin{center}
          \scalebox{.7}{
            \begin{tikzpicture}[node distance=1.5cm]
              \node[circleC, scale=1.5](C) at (0,0) {\huge Menge C};
              \node[circleA, scale=0.75](A) at (-1,0) {Menge A};
              \node[circleB, scale=0.75](B) at (1,0) {Menge B};
            \end{tikzpicture}
          }
        \end{center}
        In Fall Nummer 4 besteht die Obermenge C aus den beiden Teilmengen A und
        B, jedoch ist es m\"oglich, dass in der Menge C Elemente existieren, die
        nicht den Mengen A und B angeh\"oren. Da dieser Fall nicht disjunkt ist,
        ist eine Schnittmenge zwischen A und B vorhanden.
        \begin{center}
          \scalebox{.7}{
            \begin{tikzpicture}[node distance=1.5cm, every edge/.style={link}]
              \node[entity](obj1){Objekt 1};
              \node[isa](isa) [below = of obj1]{ISA} edge node[auto] {(0,1) (1,1)}(obj1);
              \node[entity](obj2)[below left = of isa]{Objekt 2};
              \node[entity](obj3)[below right = of isa]{Objekt 3};
              \path[draw, -] (isa.west) -| (obj2.north);
              \path[draw, -] (isa.east) -| (obj3.north);
            \end{tikzpicture}
          }
        \end{center}
        F\"ur den 4. Beziehungstyp wird erneut das Universit\"atsbeispiel herangezogen.
          \tablefirsthead{%
            \multicolumn{1}{l}{\textbf{Person\_ID}} &
            \multicolumn{1}{l}{\textbf{Klasse}} &
            \multicolumn{1}{l}{\textbf{Vorname}} &
            \multicolumn{1}{l}{\textbf{Nachname}} &
            \multicolumn{1}{l}{\textbf{Geburtsdatum}} \\
          }
          \tablecaption{Person}
          \begin{supertabular}[h]{lllll}
            1 & S & Heinz & Meier & 01.02.1990 \\
            2 & W & Michael & Schulz & 02.05.1980 \\
            3 & W & Frank & Bertling & 04.10.1981 \\
            4 & S & Hans & Fasshauer & 07.09.1991 \\
            5 & W & Tobias & Schreiber & 20.07.1983 \\
            6 & M & Kai & Sperling & 24.11.1980 \\
          \end{supertabular}

          \tablefirsthead{%
            \multicolumn{1}{l}{\textbf{Person\_ID}} &
            \multicolumn{1}{l}{\textbf{Matrikelnummer}} \\
          }
          \tablecaption{Student}
          \begin{supertabular}[h]{lp{5cm}}
            1 & 65420\\
            4 & 66530\\
            5 & 66460\\
          \end{supertabular}
          \tablefirsthead{%
            \multicolumn{1}{l}{\textbf{Person\_ID}} &
            \multicolumn{1}{l}{\textbf{Gehalt}} \\
          }
          \tablecaption{WiMa}
          \begin{supertabular}[h]{lp{5cm}}
            2 & 3000\\
            3 & 3150\\
            5 & 2900\\
          \end{supertabular}

          Hier wurde nun die Eigenschaft \enquote{Klasse} eingef\"uhrt. Diese wird ben\"otigt, um anzugeben, welches Objekt zu welcher Spezialisierung geh\"ort. Diese Eigenschaft wird auch als \enquote{diskriminierende Eigenschaft} bezeichnet.

          Die Person mit der Person\_ID 6 ist ein Mitarbeiter der Universit\"at und geh\"ort zu keinem der beiden Subtypen. Dieser Mitarbeiter vereint nur die Eigenschaften des Supertyps in sich. Als identifizierendes Attribut wird die Eigenschaft \enquote{Person\_ID} verwendet.

          Es ergibt sich f\"ur den 4. Fall im Beispiel die folgende Abbildung:

          \begin{center}
            \scalebox{.7}{
              \begin{tikzpicture}
                \node[entity](obj1){Person};
                \node[isa](isa) [below = of obj1]{ISA} edge node[auto] {(0,1) (1,1)}(obj1);
                \node[entity](obj2)[below left = of isa]{Student};
                \node[entity](obj3)[below right = of isa]{WiMa};
                \path[draw, -] (isa.west) -| (obj2.north);
                \path[draw, -] (isa.east) -| (obj3.north);
              \end{tikzpicture}
            }
          \end{center}

     \section{Generalisierung}
      Unter Generalisierung versteht man den Prozess der Gewinnung einer Obermenge aus mehreren \"ahnlichen Untermengen. Aus der gewonnenen Obermenge entsteht ein neuer Objekttyp, der durch diejenigen Eigenschaften beschrieben wird, die den \"ahnlichen Untermengen gemeinsam sind. Es handelt sich hierbei also um den Umkehrprozess zur Spezialisierung. Bei der Generalisierung handelt es sich um einen \enquote{Bottom-up-Ansatz}, zu dem die Spezialisierung den \enquote{Top-down-Ansatz} bildet. In der Praxis findet man oft die Kombination aus beiden Ans\"atzen.

      Gemeinsame Merkmale bzw. Eigenschaften von Objekttypen oder Subtypen werden identifiziert und zu einem einzigen Objekttyp (Obermenge) generalisiert.
      \subsection{Beispiel f\"ur die Generalisierung}
        Die Objekttypen \enquote{Student} und \enquote{WiMa} k\"onnen zum Objekttyp \enquote{Person} generalisiert werden. \enquote{Student} und \enquote{WiMa} sind jetzt Untermengen der generalisierten Obermenge \enquote{Person}. Bei der Generalisierung ist ebenfalls zu unterscheiden, ob bei den Mengen eine \"Uberdeckung und/oder eine \"Uberlappung m\"oglich sein soll (vgl. Fallunterscheidung bei Spezialisierung). Im ER-Modell sieht dieser Sachverhalt folgenderma\ss en aus:
        \begin{center}
          \scalebox{.7}{
            \begin{tikzpicture}
              \node[entity](obj1){Person};
              \node[isa90](isa) [below = of obj1]{ISA} edge node[auto] {(0,1) (1,1)}(obj1);
              \node[entity](obj2)[below left = of isa]{Student};
              \node[entity](obj3)[below right = of isa]{WiMa};
              \path[draw, -] (isa.west) -| (obj2.north);
              \path[draw, -] (isa.east) -| (obj3.north);
            \end{tikzpicture}
          }
        \end{center}
        Das Ergebnis der Generalisierung unterscheidet sich zur Spezialisierung nicht, es handelt sich wie oben beschrieben nur um zwei verschiedene Ans\"atze der Modellierung.

        F\"ur die Gerneralisierung gilt ebenfalls, dass es mehr als zwei Subtypen geben kann. In der grafischen Darstellung werden diese weiteren Subtypen an das Dreieck angeh\"angt. Im dargestellten Beispiel f\"ur die Gerneralisierung kann z.B. der Subtyp \enquote{Professor} angeh\"angt werden, um das Beispiel zu erweitern.

        W\"urde ein Objekttyp \enquote{Praktikant} in das Beispiel aufgenommen, ohne das ein eigener Subtyp f\"ur ihn geschaffen wird, m\"usste er im Objekttyp \enquote{Person} gespeichert werden, woraus folgt, dass das Beispiel keine vollst\"andige \"Uberdeckung mehr bietet. In der Mengendarstellung w\"are der \enquote{Praktikant} ausserhalb von \enquote{Student} und \enquote{WiMa} im Rechteck von \enquote{Person} zu finden.
\clearpage

        \section{\"Ubungen - Erweiterte ER-Modellierung}
            \subsection{\"Ubungsaufgabe Schwimmbad}
        Entwerfen Sie, basierend auf der folgenden Lage, ein ER-Modell, inklusive der Beziehungen zwischen den Entit\"aten.

        F\"ur ein gro\ss{}es Freizeitbad soll eine Datenbank f\"ur die Buchhaltung geschaffen werden. Im Einzelnen m\"ussen die nachfolgend beschriebenen Zusammenh\"ange in der Datenbank abgebildet werden.

        \subsubsection{Vorgaben}
          \begin{itemize}
            \item Ein Freizeitbad besitzt 15 verschiedene Kartentypen, von denen deren Bezeichnung, der
            Kartenpreis und eine eindeutige ID zu speichern sind. Die unterschiedlichen Kartentypen sind nachfolgend aufgeschl\"usselt:
              \begin{itemize}
                \item Stundenkarte (2 und 4 Stunden) f\"ur Kinder, Erwachsene, Studenten
                \item Tageskarte f\"ur Kinder, Erwachsene, Studenten
                \item Monatskarte f\"ur Kinder, Erwachsene, Studenten
                \item Saisonkarte f\"ur Kinder, Erwachsene, Studenten
              \end{itemize}
            \item F\"ur die Buchhaltung ist es wichtig, dass jede verkaufte Eintrittskarte gespeichert wird, so
            dass am Jahresende eine korrekte Steuererkl\"arung erstellt werden kann. Zu jeder Eintrittskarte wird
            eine Karten\_ID, ein Barcode und das Verkaufsdatum gespeichert.
            \item Jede Eintrittskarte ist von genau einem Kartentyp, wobei es vorkommen kann, dass von einem Kartentyp keine Eintrittskarte verkauft wird.
            \item Der Verkauf der Eintrittskarten erfolgt durch die Bademeister. Zu jedem Bademeister ist
            sein Name (Vorname, Nachname), seine Wohnadresse, das Datum der Teilnahme am letzten Rettungsschwimmerkurs und das Teilnahmedatum am letzten Erstehilfekurs, sowie eine eindeutige ID zu speichern.
            \item Wird eine Eintrittskarte verkauft, geschieht dies durch einen Bademeister. Es ist immer mindestens ein Bademeister mit dem Verkauf von Eintrittskarten besch\"aftigt.
\clearpage
            \item Die Bademeister k\"onnen noch andere Aufgaben (Aufsicht, Reinigungsarbeiten,
            Wartungsarbeiten, etc.), die in Arbeitspl\"anen zusammengestellt werden, ausf\"uhren. Jeder Bademeister muss im System festhalten, wann er welche Arbeit auf seinem aktuellen Arbeitsplan erledigt hat. Dabei ist es m\"oglich, dass eine Aufgabe von keinem Bademeister erledigt wird.
            \item F\"ur jede Aufgabe ist deren Kurzbezeichnung, eine Beschreibung und die
            Arbeitsdauer in Stunden, sowie eine eindeutige ID zu speichern.
          \end{itemize}

        \subsubsection{Zusatzaufgaben}
          \begin{enumerate} %{itemize}
            \item Nach der Fertigstellung des Modells verlangt der Kunde, dass der Preis eines
            Kartentyps abh\"angig von der jeweiligen Badesaison sein muss. F\"ur eine Saison wird deren
            Bezeichnung (z. B. Sommer 2012), das Anfangsdatum und das Enddatum gespeichert.

            Modellieren Sie dieses Szenario und entscheiden Sie selbst, welche Kardinalit\"aten f\"ur die
            Beziehung/Beziehungen notwendig sind!

            \item Unser Kunde, das Freizeitbad, ben\"otigt eine weitere \"Anderung am bestehenden System.
            Bei der Ermittlung der Vorgaben wurde vergessen, dass alle Monats- und Saisonkarten
            Namensbezogen (Vorname, Nachname, Adresse) und mit einem Foto versehen, verkauft werden.
            Au\ss{}erdem muss bei den Stundenkarten ein Zeitstempel gespeichert werden, so dass die
            Datenbank automatisch errechnen kann, wann die Person das Bad wieder verlassen, bzw. ob die
            Person bereits eine Nachzahlung leisten muss.

            \item In einer erneuten Anfrage bittet die Leitung des Freizeitbades darum, dass eine weitere
            \"Anderung eingearbeitet wird. Wenn einem Bademeister gek\"undigt wird, wird dieser nicht aus
            dem System gel\"oscht, sondern nur sein K\"undigungsdatum gespeichert.

            Setzen Sie diese \"Anderung so um, dass keine NULL-Werte im System erzeugt werden.
          \end{enumerate}%{itemize}
\clearpage

            \subsection{\"Ubungsaufgabe Kindergarten}
        Entwerfen Sie, basierend auf der folgenden Lage, ein ER-Modell, inklusive der Beziehungen zwischen den Entit\"{a}ten.

        Ein Kindergarten m\"ochte alle anfallenden Daten in einer Datenbank speichern.

        Er wird in Gruppen unterteilt. Zu jeder Gruppe ist deren Bezeichnung und der zugeh\"orige Raum zu speichern. Des Weiteren steht in jedem Gruppenraum ein Telefon, dessen Telefonnummer ebenfalls erfasst werden muss.

        In der Datenbank m\"ussen Angaben \"uber verschiedene Arten von Personen gespeichert werden. Zu jeder Person werden ihr Vorname, ihr Nachname und das Geschlecht gespeichert. Es gibt folgende Personenarten:
        \begin{itemize}
          \item Eltern: Jeder Elternteil (Vater und/oder Mutter) muss mit seiner Adresse erfasst werden. Es ist durchaus m\"oglich, dass Elternteile getrennt leben und daher unterschiedliche Adressen haben. Kein Elternteil arbeitet im Kindergarten.
          \item Kinder: Zu jedem Kind muss die Gruppenzugeh\"origkeit und sein Geburtsdatum gespeichert werden. In einer Kindergartengruppe ist immer mindestens ein Kind, h\"ochstens jedoch 20 Kinder. Ein Kind wird immer genau einer Gruppe zugeordnet.
          \item Weiterhin ist es wichtig zu wissen, welche Eltern (Vater und/oder Mutter) zu einem Kind geh\"oren und bei welchem Elternteil das Kind (eines oder mehrere) lebt. Es muss m\"oglich sein, F\"alle abzubilden wie z. B.: Die Eltern sind verheiratet/leben zusammen und das Kind lebt im Haushalt; die Eltern leben getrennt und das Kind lebt bei der Mutter (oder beim Vater); das Kind hat nur noch einen Elternteil und lebt in dessen Haushalt. (Hinweis: In diesem Ausschnitt der Realit\"at gibt es keine Waisenkinder!)
          \item Angestellte: Jeder Angestellte wird mit Sozialversicherungsnummer (SozVersNr) und Einstellungsdatum gespeichert.
        \end{itemize}

        Um den geforderten Realit\"atsausschnitt exakt abbilden zu k\"onnen,
        m\"ussen die Angestellten in zwei Gruppen unterteilt werden:

        \begin{itemize}
          \item Betreuer: Jeder Betreuer betreut genau eine Gruppe, wobei eine Gruppe immer von mindestens einem aber h\"ochstens von zwei Betreuern beaufsichtigt wird. Zu jedem Betreuer ist dessen Gehalt zu speichern.

          \item Praktikanten: Bei einem Praktikanten ist von vornherein bekannt, bis zu welchem Datum (Praktikumsende) sein Praktikum geht. Dieses Datum ist wichtig und muss gespeichert werden. Jeder Praktikant wird genau einem Betreuer zugeordnet, wobei ein Betreuer sich um h\"ochstens einen Praktikanten k\"ummert.
        \end{itemize}
\clearpage

      \clearpage
      \subsection{\"Ubungsaufgabe Bank}
        Der Manager einer neu gegr\"undeten Bank beauftragt Sie mit der
        Erstellung eines Datenmodells f\"ur die Verwaltung der Bankgesch\"afte.
        In der vor kurzem erfolgten Besprechung wurden folgende Eckpunkte
        festgelegt:

        Meine Bank hat mehrere Kunden aber mindestens Einen sonst w\"urde meine
        Bank nicht existieren. Alle meine Kunden besitzen eine Kunden ID, einen
        Vorname, einen Nachnamen, ein Geburtsdatum und eine Rechnungsadresse.

        Kunden k\"onnen eines oder mehrere Konten besitzen. Jedes Konto hat
        jedoch genau einen Besitzer. Ein Konto ist entweder ein Sparbuch, ein
        Depot oder ein Girokonto. Andere Arten von Konten werden bei uns nicht
        gef\"uhrt und es gibt auch keine Mischformen dieser drei Kontoarten.
        Jedes Konto soll in unserer Datenbank mit einer IBAN (international bank
        account number) versehen werden. F\"ur die Sparb\"ucher und die
        Girokonten ist auch das aktuelle Guthaben zu speichern. Auf unsere
        Sparb\"ucher gibt es einen Habenzinssatz, f\"ur die Girokonten wird ein
        Sollzinssatz gef\"uhrt, der bei \"Uberziehung des Kontos zum Tragen
        kommt. Wer ein Girokonto besitzt, f\"ur den wird j\"ahrlich eine
        Kontof\"uhrungsgeb\"uhr f\"allig. Bei einem Depot muss eine
        Er\"offnungsgeb\"uhr gezahlt werden.

        Unsere Kunden k\"onnen einer anderen Person, die ebenfalls bei uns Kunde
        sein muss, h\"ochstens eine Vollmacht \"uber ein Konto geben. Ein Kunde
        kann mehrere Vollmachten \"uber verschiedene Konten besitzen.

        Ein wesentlicher Bestandteil der Datenbank ist unser Buchungssystem. Auf
        den Konten unserer Kunden erfolgen Buchungen (Einzahlungen,
        Auszahlungen, \"Uberweisungen, usw. NICHT SPEZIALISIEREN!!!), die mit
        einem Betrag und einem Buchungsdatum gespeichert werden m\"ussen. Jede
        Buchung ist immer genau einem Konto zuordenbar, w\"ahrend es f\"ur ein
        Konto mehrere Buchungen geben kann. Es muss auch der Fall
        ber\"ucksichtigt werden, dass z. B. auf einem neu er\"offneten Konto
        noch keine Buchung vorgenommen wurde.

        Damit unsere Kunden auch Gesch\"afte mit Kunden anderer Banken t\"atigen
        k\"onnen, m\"ussen von diesen fremden Personen der Vorname, der Nachname
        und eine IBAN gespeichert werden. Auch ben\"otigen wir den Namen und den
        BIC (bank identity code) der fremden Bank (Hinweis: Hier bietet es sich
        an, die Kunden in Eigenkunden unserer Bank und Fremdkunden zu
        unterteilen!). Es kommt auch vor, dass einer unserer Kunden Kunde bei
        einer anderen Bank ist und Geld von einem seiner Konten auf ein Anderes,
        bei einer anderen Bank, transferieren m\"ochte.

        Die Eigenkunden k\"onnen von mehreren Mitarbeitern unserer Bank betreut
        werden. Ein Mitarbeiter kann mehrere Kunden betreuen. Die Vorgesetzten
        der Filialleiter, die ebenfalls als Mitarbeiter gef\"uhrt werden,  haben
        keinen Kundenkontakt. Jeder Mitarbeiter hat genau einen Vorgesetzten,
        au\ss{}er mir selbst. Ein vorgesetzter Mitarbeiter hat aber mehrere
        Mitarbeiter, die er f\"uhrt. Meine Mitarbeiter sollen in der Datenbank
        mit Vorname, Nachname und einer Mitarbeiter ID gespeichert werden.

        Jeder Mitarbeiter der Bank arbeitet in einer Bankfiliale, au\ss{}er den
        Vorgesetzten der Filialleiter. In einer Bankfiliale arbeiten mindestens
        ein aber maximal zehn Mitarbeiter. Die Bankfiliale soll mit ihrer 
        Adresse in der Datenbank gespeichert werden.
\clearpage
            \subsection{\"Ubungsaufgabe Autoh\"andler}
        Entwerfen Sie, basierend auf der folgenden Lage, ein ER-Modell, inklusive der Beziehungen zwischen
        den Entit\"{a}ten.

        Der Eigent\"umer eines namhaften Autohauses der Region beauftragt Sie eine Datenbank zu entwerfen.
        Bei einem ersten Gespr\"ach erfahren Sie, dass die Datenbank ben\"otigt wird, um das Autohaus besser zu
        verwalten. Des Weiteren wird die Grobgliederung der Datenbank festgehalten.
        \begin{itemize}
          \item Die Datenbank soll alle Angestellten des Autohauses auff\"uhren. Diese unterteilen sich in Verk\"aufer und Mechaniker. Verk\"aufer sind f\"ur die Betreuung der Kunden verantwortlich und f\"uhren  Autoverk\"aufe durch. Au\ss erdem nehmen die Verk\"aufer auch Auftr\"age f\"ur Reparaturen an. Mechaniker sind nur f\"ur die Durchf\"uhrung der Reparaturen zust\"andig.
          \item In der Datenbank sollen Vor- und Nachname, Geburtsdatum und die Adresse des jeweiligen Mitarbeiters
          hinterlegt werden k\"onnen. Jeder Verk\"aufer kann mehrere Kunden betreuen, einen Verkauf vornehmen oder eine Reparatur annehmen. Allerdings wird ein Kunde nur von genau einem Verk\"aufer betreut. Dies gilt auch im Bezug auf einen Autoverkauf und die Annahme einer Reparatur. Der Besitzer des Autohauses bittet Sie auch zu ber\"ucksichtigen, dass ein neu eingestellter Verk\"aufer noch nicht all diese T\"atigkeiten durchf\"uhren kann.
        \end{itemize}
        Bei einem zweiten Treffen mit dem Autohausbesitzer werden die Details der Datenbank besprochen:
        \begin{itemize}
          \item Kunden des Autohauses sollen mit Vor- und Nachnamen sowie ihrer Adresse im System erfasst werden. Es ist dabei zu bedenken, dass Personen, die ihr Auto nur an das Autohaus verkaufen, auch als Kunden erfasst werden, selbst wenn diese dann kein anderes Auto im Autohaus, bei Ihrem Auftraggeber, kaufen.
          \item Ihr Auftraggeber bittet Sie weiterhin, in der Datenbank seine gesamten Autos, inklusive der Autos
          seiner Kunden zu ber\"ucksichtigen. Autos k\"onnen schon einem Kunden geh\"oren oder werden an einen Kunden verkauft. Manche Kunden besitzen kein Auto z. B. Neukunden, andere haben zwei oder mehr Autos.
          \item Oft werden Autos zur Reparatur gebracht. Reparaturauftr\"age
          werden genau einem Auto zugeordnet. Manche Autos, z. B. Neuwagen haben
          noch keine Reparaturen, andere dagegen schon mehrere.
\clearpage
          \item Die Fahrzeuge werden mit der Automarke, dem Modell, dem
          Marktpreis und der Fahrleistung erfasst. Es ist zu erw\"ahnen, dass zu
          einem Kaufvertrag ein Datum und immer nur ein Auto geh\"oren. Hier
          kann davon ausgegangen werden, dass jedes Auto h\"ochstens einmal
          verkauft wird.
          \item Als Letztes soll festgehalten werden, dass ein Mechaniker eine oder mehrere Reparaturen durchf\"uhren kann. Zu jeder Reparatur m\"ussen deren Datum, der Rechnungspreis f\"ur die geleisteten Arbeiten und die Ersatzteile, sowie die ben\"otigten Arbeitsstunden gespeichert werden. Eine Reparatur wird auch nur von einem Mechaniker durchgef\"uhrt.
        \end{itemize}
\clearpage
            \subsection{\"Ubungsaufgabe Universit\"at}
        Der Leiter der Abteilung f\"ur Verwaltungsangelegenheiten einer
        Universit\"at beauftragt Sie mit der Erstellung eines Datenmodells, um
        die Verwaltungsstruktur effizienter zu gestalten. In einer Besprechung
        wurden folgende Eckpunkte festgelegt:

        Meine Universit\"at gliedert sich in Fakult\"aten. Diese umfassen immer
        mindestens ein Institut. Ein Institut kann niemals zu mehreren
        Fakult\"aten gleichzeitig geh\"oren. F\"ur beide m\"ussen deren
        Bezeichungen gespeichert werden.

        In der Datenbank sollen drei unterschiedliche Personentypen eingetragen
        werden. Diese sind mit ihren Vornamen, Nachnamen, der Adresse und dem
        Geburtsdatum zu hinterlegen. Der erste Personentyp sind Professoren,
        welche genau ein Institut leiten. Umgekehrt kann ein Institut auch nur
        von genau einem Professor geleitet werden.

        Professoren halten Vorlesungen. Jeder Professor h\"alt mindestens eine
        Vorlesung. Der Fall, dass eine Vorlesung von mehreren Professoren
        gehalten wird, tritt nicht auf. Vorlesungen werden immer in
        H\"ors\"alen gehalten. In einem H\"orsaal k\"onnen mehrere Vorlesungen
        gehalten werden, jedoch wird eine Vorlesung immer nur in genau einem
        H\"orsaal abgehalten.

        Eine weitere Aufgabe der Professoren ist es, \"Ubungen bereitzustellen.
        Dies geschieht jedoch auf freiwilliger Basis, so dass nicht jeder
        Professor \"Ubungen f\"ur seine Studenten zur Verf\"ugung stellt. Eine
        \"Ubung wird immer von genau einem Professor erstellt.

        In bestimmten Zeitabst\"anden wird ein Professor zum \enquote{Dekan}
        gew\"ahlt. Der Dekan ist der Leiter einer Fakult\"at. Ein Professor kann
        nur h\"ochstens eine Dekanstelle besetzen und eine Fakult\"at wird immer
        von genau einem Dekan geleitet.

        Eine zweite Personengruppe in der Datenbank sind die Wissenschaftlichen
        Mitarbeiter (im Folgenden nur noch WiMa genannt). WiMas betreuen immer
        wieder \"Ubungen, haben aber auch andere Aufgaben. Eine \"Ubung kann nur
        von einem WiMa betreut werden. Zus\"atzlich zu den genannten Daten, die
        allgemein f\"ur einen Personentyp gespeichert werden, wird f\"ur WiMas
        und Professoren das jeweilige Gehalt in der Datenbank abgelegt.

        Der dritte Personentyp sind Studenten. Diese k\"onnen an \"Ubungen und
        Vorlesungen teilnehmen. Damit eine \"Ubung oder Vorlesung stattfindet,
        muss sich mindestens ein Student daf\"ur eingeschrieben haben. Studenten
        erhalten, neben den angesprochenen Personenparametern, noch zus\"atzlich
        eine Matrikelnummer.

        Die an der Universit\"at gehaltenen Vorlesungen und \"Ubungen sind mit
        einem Thema versehen. Die \"Ubungen unterteilen sich in Rechen- bzw.
        Labor\"ubungen und werden mit einer Aufgabennummer versehen.
        Rechen\"ubungen finden in H\"ors\"alen und Labor\"ubungen in Laboren
        statt. Nicht jeder H\"orsaal bzw. jedes Labor ist immer besetzt. Eine
        \"Ubung findet allerdings immer nur in einem Raum statt. F\"ur jeden
        H\"orsaal und jedes Labor muss die Anzahl der Sitzpl\"atze, die
        Raumnummer sowie die Geb\"audenummer gespeichert werden.
\clearpage
            \subsection{\"Ubungsaufgabe Diensthundeschule}
        Der Kommandeur der Schule f\"ur Diensthundewesen der Bundeswehr (SDstHundeBw) bittet Sie ein Datenmodell zu entwerfen, um eine bereits vorhandene Patientenanwendung zu ersetzen. Patienten im Sinne dieser Anwendung sind Diensthunde. In einer Besprechung wurden folgende Eckpunkte festgelegt:

        Als Erstes m\"ussen alle relavanten Dienststellen mit ihrer Bezeichnung, der Dienststellennummer, der Adresse und der Telefonnummer eines Ansprechpartners in der Datenbank hinterlegt werden. Jede Dienststelle untersteht h\"ochstens einer anderen Dienststelle, eine Dienststelle hat keine oder mehrere Dienststellen unter sich.

        Zu einem Diensthund sind sein Name, seine Fellfarbe, das Geschlecht und das Kaufdatum wichtig.

        Zwischen den Dienststellen und den Diensthunden existieren zwei unterschiedliche Beziehungen. Es gibt Dienststellen, die Eigent\"umer der Diensthunde sind und andere Dienststellen, die Besitzer der Diensthunde sind. Dies kommt dadurch zu Stande, dass die Hunde nicht immer in der Dienststelle ihren Dienst verrichten, die der Eigent\"umer des Hundes ist. Jede hier erfasste Dienststelle besitzt bzw. ist Eigent\"umer von mindestens einem Hund. Ein Diensthund ist jedoch immer nur einer Dienststelle zugeordnet. Dies gilt f\"ur den Besitz eines Hundes und auch f\"ur das Eigentum an einem Hund.

        Die an der Klinik der SDstHundeBw befindlichen Diensthunde haben alle genau einen Status (z. B. dienstf\"ahig, eingeschr\"ankt dienstf\"ahig, usw.). Ein Status kann an mehrere Diensthunde vergeben werden. Des Weiteren bewohnt jeder Diensthund, w\"ahrend seines Aufenthaltes an der SDstHundeBw einen Zwinger. F\"ur das Personal in der Tierklinik ist es wichtig zu wissen, welcher Diensthund wann und wie oft welchen Zwinger bewohnt hat (es soll eine Historie der Zwingeraufenthalte der Hunde m\"oglich sein). Es muss mit eingeplant werden, dass nicht immer alle Zwinger von Hunden bewohnt werden. Ein Zwinger geh\"ort zu genau einer Zwingerart. Beispielsweise gibt es normale Zwinger und Quarant\"ane Zwinger. Von jeder Zwingerart gibt es mindestens einen Zwinger in der Tierklinik. Zu einem Zwinger werden der Ort, an dem er sich befindet und die Zwingernummer gespeichert. F\"ur die Zwingerart soll lediglich deren Bezeichnung angegeben werden.

        Ein Diensthund durchl\"auft im Laufe seines Lebens verschiedene Untersuchungen in der Tierklinik. Zu jeder Untersuchung ist das Untersuchungsdatum wichtig. Bei jedem Untersuchungstermin wird immer nur genau ein Hund behandelt.

        Durchgef\"uhrt werden die Untersuchungen von medizinischem Fachpersonal. Eine Untersuchung wird von genau einem Tierarzt durchgef\"uhrt, der w\"ahrend seiner Anwesenheit an der SDstHundeBw die verschiedenstens Untersuchungen durchf\"uhren kann.

        Zu jeder Untersuchung wird auch immer mindestens ein anderer Tierarzt oder eine Krankenschwester hinzugezogen. In der Datenbank soll f\"ur das medizinische Personal der Name
        und der Dienstgrad gespeichert werden.

        Es gibt vier verschiedene Arten von Untersuchungen, welche unterschieden werden m\"ussen:
        \begin{itemize}
          \item Die Ankaufuntersuchung: Jeder Hund wird vor seinem Ankauf durch die Bundeswehr gr\"undlich Untersucht.
          \item Die Behandlung: Ein Diensthund wird auch im Falle einer ganz normalen Erkrankung, w\"ahrend seiner Anwesenheit an der SDstHundeBw, in der Tierklinik behandelt.
          \item Die Nachuntersuchung: Nach seinem Ankauf durch die Bw wird ein Hund in seinen ersten zwei Dienstjahren mehrfach nachuntersucht.
          \item Das Ausmusterungsgutachten: Hat ein Diensthund ein gewisses Alter erreicht oder eine schwere Verletzung erlitten, wird er aus dem Dienst entlassen.
        \end{itemize}

        F\"ur die Ankaufuntersuchung sind Angaben wie die Gr\"o\ss{}e und das Gewicht des Hundes sowie sein R\"ontgenzahnalter wichtig. Zu einer Nachuntersuchung wird nur ein Befund in Textform gespeichert.

        Wird eine \enquote{normale} Behandlung an einem Diensthund durchgef\"uhrt, so besteht diese aus mindestens einer oder mehreren Behandlungspositionen (Operation, Medikamentengabe, usw.) die einzeln zu speichern sind. Dabei ist eine erfasste Behandlungsposition immer eindeutig einer Behandlung eines Diensthundes zuordenbar. Der behandelnde Arzt muss auch die M\"oglichkeit besitzen, zu einer Behandlungsposition eine kurze Notiz zu schreiben. Neben dieser Option ist zu jeder Behandlungsposition eine entsprechende Diagnose zu vermerken. Diese werden jedoch in einer separaten Liste verwaltet. So kann es auch vorkommen, dass dort Diagnosen aufgelistet sind, die bisher noch bei keinem Diensthund festgestellt wurden.

        F\"ur ein Ausmusterungsgutachten muss der Arzt einen kompletten Bericht im System hinterlegen k\"onnen.
        \section{L\"osungen - Erweiterte ER-Modellierung}
      \input{../modellierung/loesungen/modellierung_schwimmbad_loesung}
      \input{../modellierung/loesungen/modellierung_kindergarten_loesung}
      \input{../modellierung/loesungen/modellierung_bank_loesung}
      \input{../modellierung/loesungen/modellierung_autohaendler_loesung}
      \input{../modellierung/loesungen/modellierung_universitaet_loesung}
      \input{../modellierung/loesungen/modellierung_diensthundeschule_loesung}
  	\chapter{Transformation}
    \setcounter{page}{1}\kapitelnummer{chapter}
    \minitoc
\newpage
    Bei der Beschreibung der Realit\"at, mit Hilfe eines Entity-Relationship-Modells, bestand die Zielrichtung darin, Objekttypen und Beziehungstypen unabh\"angig vom sp\"ater einzusetzenden Datenbankmanagementsystem und somit auch unabh\"angig von einem speziellen Datenbankmodell zu beschreiben. Nun muss der Versuch unternommen werden, das durch die Modellierung entstandene Datenmodell m\"oglichst ohne semantische\footnote{Semantik = Bedeutungslehre, bezeichnen, anzeigen} Einbu\ss en mit den Strukturierungsmitteln der verf\"ugbaren Datenbankmanagementsysteme wiederzugeben. Dieser Vorgang wird als Transformation bezeichnet. Es ist die Umsetzung des konzeptionellen Datenmodells, welches nur die m\"oglichen Beziehungen zwischen den Objekttypen beschreibt, in das physische Datenmodell. Das physische Datenmodell ber\"ucksichtigt, wie die Beziehungen zwischen den Objekttypen auf Datenbankebene effektiv umgesetzt werden. Das physische Datenmodell wird in verschiedenen Quellen auch als relationales Modell
bezeichnet.

    Eine komplette Gleichsetzung ist je nach Interpretation nicht m\"oglich, hier soll jedoch nicht weiter unterschieden werden.

    Im Vorfeld sind diese beiden Fragen zu beantworten.
    \begin{enumerate}
      \item Welches Datenbankmodell soll der zu erstellenden Datenbank zugrunde liegen?
      \item Welche M\"oglichkeiten bietet das zu verwendende Datenbankmanagementsystem f\"ur die Gestaltung der Datenstrukturen?
    \end{enumerate}
    Hinsichtlich der ersten Frage hat heutzutage das relationale Datenbankmodell die meiste Relevanz. Was die zweite Fragestellung anbetrifft, so kommen hier nur die beiden Datenbankmanagementsysteme \textit{ORACLE} und \textit{Microsoft SQL Server} zum Einsatz. Bei beiden Systemen handelt es sich um relationale Datenbankmanagementsysteme (RDBMS), welche ihre Vor- und Nachteile besitzen, auf die im Unterricht kurz eingegangen werden soll.

    Dieses Kapitel schafft einheitliche Begriffe und stellt Regeln f\"ur die Transformation eines ER-Modells in ein relationales Modell auf.
\clearpage
    \section{Grundlagen}
      \subsection{Begriffsdefinitionen}
        \label{basics_definitions}
        Im vorangegangenen Kapitel wurde der Begriff \enquote{Schl\"ussel} erl\"autert, der an dieser Stelle aber noch weiter differenziert werden muss.
        \subsubsection{Identifikationsschl\"ussel (ID-Schl\"ussel)}
          Jedes Objekt einer Objektmenge muss eindeutig identifizierbar sein. Dies kann durch eine Eigenschaft/Attribut oder eine Kombination von Eigenschaften/Attributen gew\"ahrleistet werden. Beispielsweise ist ein Soldat der Bundeswehr eindeutig durch die Personalnummer identifizierbar. Der Name einer Person kann \textbf{kein} Identifikationsschl\"ussel sein, weil es sehr wahrscheinlich ist, dass es mehrere Personen mit dem gleichen Namen gibt (z. B. mehrere Meier).

          Der Identifikationsschl\"ussel muss folgende Kriterien erf\"ullen:
          \begin{itemize}
           \item Jedes Objekt muss eindeutig identifizierbar sein. Es d\"urfen nicht mehrere Objekte einen ID-Schl\"ussel mit dem gleichen Wert aufweisen.
           \item Jedem neuen Objekt muss augenblicklich ein Identifikationsschl\"ussel zugeteilt werden k\"onnen, da sonst keine Speicherung des Objektes erfolgen darf.
           \item Der ID-Schl\"usselwert eines Objektes darf sich w\"ahrend dessen Existenz nicht ver\"an\-dern.
          \end{itemize}
        \subsubsection{Prim\"arschl\"ussel}
          Der Prim\"arschl\"ussel wird h\"aufig mit dem Begriff \enquote{Identifikationsschl\"ussel} gleichgesetzt. Diese beiden Begriffe sind aber nicht geichbedeutend. Der Prim\"arschl\"ussel (PK - primary key) wird direkt in die Speicherorganisation einbezogen und ist somit dem physischen Datenmodell zugeordnet. Der ID-Schl\"ussel hingegen ist dem konzeptionellen Datenmodell zugeordnet. Ansonsten gelten f\"ur die Eigenschaftswerte eines PK dieselben Bedingungen, wie beim ID-Schl\"ussel beschrieben. Jeder transformierte Objekttyp kann nur einen PK haben. Da jedes Objekt eindeutig \"uber den PK identifiziert werden soll, ergibt sich automatisch die Bedingung, dass der Wert des PK einmalig (unikal) und nicht NULL (leer) sein muss. Daneben kann es aber auch weitere Eigenschaften mit eindeutigen Werten geben, die nicht zum PK geh\"oren.
\clearpage
        \subsubsection{Fremdschl\"ussel}
          Ein Fremdschl\"ussel (FK = Foreign Key) ist ein Attribut, welches zur Verkn\"upfung zweier Tabellen dient. Ein Fremdschl\"usselattribut wird in eine Tabelle eingef\"ugt, welche sich als untergeordnete Tabelle auf eine andere bezieht. Das Fremschl\"usselattribut bezieht sich dabei immer auf den Prim\"arschl\"ussel oder aber auf ein anderes, eindeutiges Attribut der \"ubergeordneten Tabelle. Genauso wie der Prim\"arschl\"ussel kann auch der Fremdschl\"ussel aus einer Kombination von Attributen bestehen. Im Gegensatz zum PK kann der FK NULL-Werte beinhalten.
        \subsubsection{NULL bzw. NOT NULL}
          Ein Eigenschaftswert kann in einer Datenbank verschiedene Einschr\"ankungen haben, um die Konsistenz der Daten zu gew\"ahrleisten. Eine dieser Einschr\"ankungen (engl. constraint) ist das NOT NULL constraint, welches im SQL Teil n\"aher erl\"autert wird. Es wird damit festgelegt, ob der Eigenschaftswert beim Anlegen eines Datensatzes vorhanden sein muss (NOT NULL, Abk. [NN]) oder ob er leer sein darf (NULL, Standard). Damit kann in der Datenbank die Forderung nach der Eigenschaft \enquote{eingabepflichtig} (Teil der Transformation) realisiert werden.
        \subsubsection{UNIQUE}
          Bei dem Begriff UNIQUE (Abk. [UN]) handelt es sich ebenfalls um ein constraint (Er\-l\"au\-terung\-en im SQL Teil). Mit diesem constraint kann die Forderung nach der \enquote{Unikalit\"at}, die sich durch die Transformation ergibt, erf\"ullt werden. Das UNIQUE (einmalig) constraint sorgt daf\"ur, dass die Eigenschaftswerte eines Objekttyps nicht doppelt vorkommen. Im RDBMS Oracle bildet hier der NULL-Wert eine Ausnahme, d.h. es k\"onnen mehrere NULL Werte innerhalb einer Spalte vorkommen. In Microsoft SQL Server ist dies nicht der Fall.
      \subsection{Kurzschreibweise der Tabellen}
        Den Aufbau einer Tabelle kann man mit folgender Kurzschreibweise darstellen:

        \centerline{Tabellenname(\pk{ID-Schl\"ussel}, Attribut$_1$, Attribut$_2$, Attribut$_3$, ..., Attribut$_n$)}

        Falls der PK aus zusammengesetzten Attributen besteht, werden alle erforderlichen Attribute unterstrichen:\\
        \hspace*{12mm}Tabellenname(\pk{Teil-ID-Schl\"ussel$_1$}, \pk{Teil-ID-Schl\"ussel$_2$}, Attribut$_1$, Attribut$_2$,\\
        \hspace*{40mm}Attribut$_3$, ..., Attribut$_n$)
    \section{Transformation von Objekttypen}
      Die Objekttypen des ER-Modells stellen das Haupt-Ordnungsprinzip der Datenmodellierung dar. Sie beschreiben die Klassen-Struktur, in die die speicherrelevanten Objekte der abzubildenden Realit\"at eingeordnet werden. Im physischen Datenbankmodell wird diese Klassen-Struktur durch einen Satz von Tabellen wiedergegeben.

      F\"ur jeden Objekttyp des ER-Modells wird eine Tabelle vereinbart. Dabei gilt die Transformationsregel T01.
      \subsubsection{Transformationsregel T01 (Objekttyp)}
      \tablefirsthead{%
        \hline
          \multicolumn{1}{|l}{\textbf{konzeptionelles Datenmodell}} &
          \multicolumn{1}{|l}{\textbf{ }} &
          \multicolumn{1}{|l|}{\textbf{physisches Datenmodell}} \\
        \hline
      }
      \begin{supertabular}[h]{|l|l|p{7.5cm}|}
        \hline
        \footnotesize Objekttypname & $\Rightarrow$ & \footnotesize Tabellen-Bezeichnung\\
        \hline
        \footnotesize (Teil-)Identifizierende Eigenschaft & $\Rightarrow$ & \footnotesize Prim\"arschl\"ussel-Attribut\\
        \hline
        \footnotesize Beschreibende Eigenschaft & $\Rightarrow$ & \footnotesize Spalten-Bezeichnung\\
        \hline
      \end{supertabular}

      \begin{center}
        \scalebox{.7}{
          \begin{tikzpicture}[node distance=1.5cm, every edge/.style={link}]
            \node[entity](obj1){Objekttypname};
              \node[attribute](teil1)[above left = of obj1]{\key{Teilidentifizierendes Attribut 1}} edge (obj1);
              \node[attribute](teil2)[above right = of obj1]{\key{Teilidentifizierendes Attribut 2}} edge (obj1);
              \node[attribute](beschr)[below = of obj1]{Beschreibendes Attribut} edge (obj1);
        \end{tikzpicture}
        }
      \end{center}
    \section{Transformation bin\"arer Beziehungstypen}
      In vorangegangenen Abschnitten wurde festgestellt, dass sich Beziehungstypen im physischen Datenbankmodell nur dadurch repr\"asentieren lassen, dass der Prim\"arschl\"ussel (PK) einer Tabelle \enquote{gedoppelt} und als Fremdschl\"ussel an \enquote{anderer Stelle} aufgenommen wird. Handelt es sich bei dieser \enquote{anderen Stelle} um eine andere Tabelle, wird ein bin\"arer Beziehungstyp dargestellt, der den sachlogischen Zusammenhang zwischen zwei Objekten aus verschiedenen Objekttypen beschreibt.

      Liegt diese \enquote{andere Stelle} dagegen in derselben Tabelle, aus der der PK stammt, wird ein rekursiver Beziehungstyp repr\"asentiert. Hier wird der sachlogische Zusammenhang zwischen zwei Objekten widergespiegelt, die demselben Objekttyp angeh\"oren. Nachfolgend werden die zehn m\"oglichen bin\"aren und die sieben rekursiven Beziehungstypen in (Min,Max)-Notation erl\"autert.
      \subsection{Der (1,1):(1,1) Beziehungstyp}
        Beim (1,1):(1,1) Beziehungstyp ist jedes Objekt des Objekttyps A mit genau einem Objekt des Objekttyps B verbunden und umgekehrt. Je ein A-Objekt und ein B-Objekt gehen eine feste Paarung ein, wie \abbildung{rel1111} zeigt. %oder auch in Abbildung \ref{fig:rel1111}.

        %\begin{figure}[!htb]
        \begin{center}
          \scalebox{1}{
            \begin{tikzpicture}
              \label{rel1111}
              \node[circleA](A) at (0,0){Objekttyp A};
                \node[redspot](a1) at (0, 1.3) {a1};
                \node[redspot](a2) at (0.5, 0.8){a2};
                \node[redspot](a3) at (0.1, 0.3){a3};
                \node[redspot](a4) at (-0.8, -0.4){a4};
                \node[redspot](a5) at (-0.2, -0.8){a5};
              \node[circleB](B) at (4.5, 0){Objekttyp B};
                \node[bluespot](b1) at (4.5, 1.3) {b1} edge (a1);
                \node[bluespot](b2) at (5, 0.8) {b2} edge (a2);
                \node[bluespot](b3) at (4.6, 0.3) {b3} edge (a3);
                \node[bluespot](b4) at (3.7, -0.4) {b4} edge (a4);
                \node[bluespot](b5) at (4.3, -0.8) {b5} edge (a5);
            \end{tikzpicture}
          }
        \end{center}
        (1,1):(1,1) Beziehungstypen sind bereits im ER-Modell kritisch zu betrachten, weil sie meist \"uber\-fl\"us\-sig sind. Ist n\"amlich jedes Objekt-A mit genau einem Objekt-B - und umgekehrt - verbunden, dann bildet sich eine derart feste Kopplung, dass sie als ein einziges, komplexes Objekt betrachtet werden k\"onnen. Die Umsetzung erfolgt, in dem alle Attribute von B in die Tabelle A eingef\"ugt werden. Die Kopplung wird dadurch erzwungen, dass der Schl\"ussel von B in der Tabelle A als eingabepflichtig (NOT NULL, NN) und als unikal (UNIQUE, UN) deklariert wird. Ein B-Objekt kann nun nicht losgel\"ost von \enquote{seinem} Objekt A gespeichert werden.

        \subsubsection{Beispiel (1,1):(1,1) Beziehungstyp}
          Betrachten wir zun\"achst einen \enquote{\"uberfl\"ussigen} (1,1):(1,1) Beziehungstyp: Mitarbeiter eines Unternehmens, von denen jeder genau einen Dienstausweis besitzt. Ein gegebener Dienstausweis ist nat\"urlich f\"ur genau einen Mitarbeiter ausgestellt.

          Da das Attribut \enquote{Ausweisnummer} eingabepflichtig ist, muss jeder gespeicherte Mitarbeiter einen Ausweis haben. Andererseits ist das Attribut unikal, so dass eine Ausweisnummer nur einem Mitarbeiter zugeordnet sein kann.

          Es gibt aber auch F\"alle, bei denen ein (1,1):(1,1) Beziehungstyp durchaus sinnvoll ist. Will man beispielsweise die Informationen \"uber Objekte in \"offentliche (z.B. Mitarbeiter-\\Offen(\pk{Personalnummer}, Name, TelNr, Abteilung)) und vertrauliche Daten (z.B. Mitar-\\beiterVS(\pk{Personalnummer}, Gehalt, Konfession)) unterteilen, kann man zwei Objekttypen A und B mit demselben Schl\"ussel \pk{S} ins Datenmodell aufnehmen. Dann wird sowohl der Schl\"ussel \pk{S} von A in B als unikaler eingabepflichtiger Fremdschl\"ussel vereinbart und umgekehrt, der Schl\"ussel \pk{S} von B wird in A als unikaler eingabepflichtiger FK deklariert. Die folgende Tabelle zeigt die entsprechende Transformationsregel T02.
\clearpage
        \subsubsection{Transformationsregel T02 f\"ur sinnvolle (1,1):(1,1) Beziehungen}
          \tablefirsthead{%
          \hline
            \multicolumn{1}{|c}{\textbf{konzeptionelles Datenmodell}} &
            \multicolumn{1}{|l}{\textbf{ }} &
            \multicolumn{1}{|c|}{\textbf{physisches Datenmodell}} \\
          \hline
          }
          \tabletail{%
          \hline
          }
          \begin{supertabular}[h]{|l|l|p{7.5cm}|}
            \footnotesize Objekttyp A mit Schl\"ussel \pk{S} & $\Rightarrow$ & \footnotesize Tabelle A mit PK \pk{S}\\
            \hline
            \footnotesize Objekttyp B mit Schl\"ussel \pk{S} & $\Rightarrow$ & \footnotesize Tabelle B mit PK \pk{S}\\
            \hline
            \footnotesize (1,1):(1,1) Beziehungstyp & $\Rightarrow$ & \footnotesize \pk{S} wird sowohl in A als auch in B als unikaler [UN], eingabepflichtiger [NN] Fremdschl\"ussel vereinbart (\fk{S})\\
            \hline
            \multicolumn{3}{|l|}{Alternative bei unterschiedlichen Schl\"usseln:} \\
            \hline
            \footnotesize Objekttyp A mit Schl\"ussel \pk{SA} & $\Rightarrow$ & \footnotesize Tabelle A mit PK \pk{SA}\\
            \hline
            \footnotesize Objekttyp B mit Schl\"ussel \pk{SB} & $\Rightarrow$ & \footnotesize Tabelle B mit PK \pk{SB}\\
            \hline
            \footnotesize (1,1):(1,1) Beziehungstyp & $\Rightarrow$ & \footnotesize Es wird \pk{SA} in B und \pk{SB} in A als unikaler [UN], eingabepflichtiger [NN] Fremdschl\"ussel eingef\"ugt (\fk{SA} und \fk{SB})\\
          \end{supertabular}

          Ein Fremdschl\"ussel wird in die Verweispfeile eingeschlossen - \fk{xyz}. Sollte der Fremd\-schl\"us\-sel seinerseits wieder Fremd\-schl\"us\-sel enthalten, werden f\"ur die inneren Fremd\-schl\"us\-sel die Verweispfeile weggelassen. Ist der Fremdschl\"ussel in der Tabelle f\"ur sich wiederum ein Prim\"arschl\"ussel, so wird der FK unterstrichen und fett gedruckt - \fk{\pk{xyz}}.

        \begin{center}
          \scalebox{.7}{
            \begin{tikzpicture}[node distance=1.5cm, every edge/.style={link}]
              \node[entity](A){A};
                \node[attribute](S)[left = of A]{\key{S}} edge (A);
              \node[relationship](rel1)[right = of A]{} edge node[auto,swap] {(1,1)}(A);
              \node[entity](B)[right = of rel1]{B} edge node[auto,swap] {(1,1)} (rel1);
                \node[attribute](S)[right = of B]{\key{S}} edge (B);
          \end{tikzpicture}
          }
        \end{center}

        \bild{}{transformationsregel_02_tabellen}{0.45}

          Bei diesem Beziehungstyp kann ein neues Objekt weder allein in A noch allein in B gespeichert werden. Das w\"urde der Forderung nach Nichtoptionalit\"at beider Beziehungstyprichtungen (siehe den jeweiligen Min-Wert) widersprechen. Deswegen muss vom Anwendungsprogramm im Rahmen einer Transaktion erreicht werden, dass zu einem neuen Schl\"usselwert je eine Zeile in die Tabellen A und B eingetragen wird.
          \begin{merke}
            Eine Transaktion ist eine Folge von Operationen, bei der sichergestellt wird, dass entweder alle Operationen fehlerfrei beendet werden oder das keine der Operationen ausgef\"uhrt wird.
          \end{merke}
        \subsubsection{Beispiel - Mitarbeiteradresse}
          Werden in einer Datenbank Adressen f\"ur mehrere Objekttypen gespeichert, z. B. Mitarbeiter und Kunden, kann es sinnvoll sein, einen eigenen Objekttyp \enquote{Adresse} zu erstellen.

					Da die Fremdschl\"ussel jeweils auf die andere Tabelle verweisen, muss es nach den Regeln der referentiellen Integrit\"at zu jedem FK genau einen Datensatz in der anderen Tabelle geben.

          \begin{center}
            \scalebox{.7}{
              \begin{tikzpicture}[node distance=1.2cm, every edge/.style={link}]
                \node[entity](A){Mitarbeiter};
                  \node[attribute](a1)[left = of A]{\key{Personalnummer}} edge (A);
                  \node[attribute](a2)[above left = of A]{Name} edge (A);
                  \node[attribute](a3)[below left = of A]{Telefon} edge (A);
                \node[relationship](rel1)[right = of A]{wohnt an} edge node[auto,swap] {(1,1)}(A);
                \node[entity](B)[right = of rel1]{Adresse} edge node[auto,swap] {(1,1)} (rel1);
                  \node[attribute](b1)[right = of B]{\key{Adresse\_ID}} edge (B);
                  \node[attribute](b2)[above right = of B]{Strasse} edge (B);
            \end{tikzpicture}
            }
          \end{center}
          \begin{small}
            Mitarbeiter(\pk{Personalnummer}, Name, Telefon, \un{\nn{\fk{Adresse\_ID}}})

            Adresse(\pk{Adresse\_ID}, Strasse, Hausnummer, PLZ, Ort, \un{\nn{\fk{Personalnummer}}})
          \end{small}
      \subsection{Der (1,1):(0,1) Beziehungstyp}
        Beim (1,1):(0,1) Beziehungstyp ist jedes Objekt des Objekttyps A mit genau einem Objekt des Objekttyps B verbunden. Ein Objekt aus B kann aber nur mit h\"ochstens einem Objekt aus A gekoppelt sein. Die Information(en) im A-Objekt k\"onnen als fakultative\footnote{wahlfreie, beliebige, optionale} erg\"anzende Angaben zum B-Objekt interpretiert werden. \abbildung{rel1101} verdeutlicht dies.

        \begin{center}
        \label{rel1101}
          \scalebox{1}{
            \begin{tikzpicture}
              \node[circleA](A) at (0,0){Objekttyp A};
                \node[redspot](a1) at (0, 1.3) {a1};
                \node[redspot](a2) at (0.5, 0.8){a2};
                \node[redspot](a3) at (0.1, 0.3){a3};
              \node[circleB](B) at (4.5, 0){Objekttyp B};
                \node[bluespot](b1) at (4.5, 1.3) {b1} edge (a1);
                \node[bluespot](b2) at (5, 0.8) {b2};
                \node[bluespot](b3) at (4.6, 0.3) {b3} edge (a3);
                \node[bluespot](b4) at (3.7, -0.4) {b4} edge (a2);
                \node[bluespot](b5) at (4.3, -0.8) {b5};
            \end{tikzpicture}
          }
        \end{center}

        Die Art der Transformation richtet sich nun danach, wie hoch der Anteil jener B-Objekte ist, f\"ur die erg\"anzende Angaben gemacht werden, die also in Beziehung zu einem A-Objekt stehen. Oder anders gefragt, ob der Objekttyp B wesentlich mehr Objekte enth\"alt als der Objekttyp A.
        \subsubsection{Fall 1:\#A unwesentlich kleiner als \#B}
          Gibt es bei einem (1,1):(0,1) Beziehungstyp nur unwesentlich mehr B-Objekte als A-Objekte, k\"onnen die Daten beider Objekttypen in einer Tabelle zusammengefasst werden. Die Eigenschaften von A werden dabei als nicht-eingabepflichtig deklariert.
\clearpage
          Der Schl\"ussel \pk{SA} von A wird in der Tabelle B jedoch als unikal deklariert. Ein A-Objekt, das damit nur zu einem einzigen B-Objekt geh\"oren kann, kann nicht losgel\"ost von \enquote{seinem} B-Objekt gespeichert werden. Bei den wenigen B-Objekten, die nicht mit einem A-Objekt verbunden sind, nehmen die A-Attribute den Wert NULL an.
        \subsubsection{Transformationsregel T03 f\"ur den (1,1):(0,1) Beziehungstyp (selten realisierte Optionalit\"at)}
          \tablefirsthead{%
          \hline
            \multicolumn{1}{|c}{\textbf{konzeptionelles Datenmodell}} &
            \multicolumn{1}{|l}{\textbf{ }} &
            \multicolumn{1}{|c|}{\textbf{physisches Datenmodell}} \\
          \hline
          }
          \begin{supertabular}[h]{|l|l|p{7.5cm}|}
            \footnotesize Objekttyp A mit Schl\"ussel \pk{SA} & $\Rightarrow$ & \footnotesize Alle Eigenschaften von A werden als nicht-eingabe-pflichtige Attribute in B aufgenommen. SA wird au\ss{}erdem als unikal [UN] deklariert\\
            \hline
            \footnotesize Objekttyp B mit Schl\"ussel \pk{SB} & $\Rightarrow$ & \footnotesize Tabelle B mit PK \pk{SB} \\
            \hline
            \footnotesize (1,1):(0,1) Beziehungstyp & $\Rightarrow$ & \footnotesize wird nicht gesondert dargestellt\\
          \end{supertabular}

          \begin{center}
            \scalebox{.7}{
              \begin{tikzpicture}[node distance=1.5cm, every edge/.style={link}]
                \node[entity](A){A};
                  \node[attribute](SA)[left = of A]{\key{SA}} edge (A);
                \node[relationship](rel1)[right = of A]{} edge node[auto,swap] {(1,1)}(A);
                \node[entity](B)[right = of rel1]{B} edge node[auto,swap] {(0,1)} (rel1);
                  \node[attribute](SB)[right = of B]{\key{SB}} edge (B);
            \end{tikzpicture}
            }
          \end{center}
        \subsubsection{Beispiel - Mitarbeiterhandy}
          Werden fast alle Mitarbeiter eines Unternehmens mit genau einem Handy ausgestattet, so sind die erg\"anzenden Angaben f\"ur das Handy bei nahezu allen Mitarbeitern erforderlich. Die entsprechende Transformation zeigt die folgende Abbildung.

          \begin{center}
            \scalebox{.7}{
              \begin{tikzpicture}[node distance=1.5cm, every edge/.style={link}]
                \node[entity](A){Handy};
                  \node[attribute](Inventarnummer)[left = of A]{\key{Inventarnummer}} edge (A);
                  \node[attribute](Mobilfunknummer)[above left = of A]{Mobilfunknummer} edge (A);
                \node[relationship](rel1)[right = of A]{hat} edge node[auto,swap] {(1,1)}(A);
                \node[entity](B)[right = of rel1]{Mitarbeiter} edge node[auto,swap] {(0,1)} (rel1);
                  \node[attribute](Personalnummer)[right = of B]{\key{Personalnummer}} edge (B);
                  \node[attribute](Name)[above right = of B]{Name} edge (B);
            \end{tikzpicture}
            }
          \end{center}
          \begin{small}
            Mitarbeiter(\pk{Personalnummer}, Name, \un{Handy-Inventarnummer}, Mobilfunknummer)
          \end{small}

          Die urspr\"ungliche Eigenschaftsbezeichnung \enquote{Inventarnummer}
          wurde durch den Zusatz\\ \enquote{Handy} in ihrem neuen Kontext
          \enquote{sprechender} gew\"ahlt. Die Handyspalten der Tabelle
          Mitarbeiter sind nicht-eingabepflichtig, d.h. sie nehmen f\"ur jene
          Mitarbeiter, die nicht mit einem Handy ausgestattet sind, den
          NULL-Wert an. Dadurch, dass das Attribut
          \enquote{Handy-Inventarnummer} als unikal deklariert ist, kann ein und
          dasselbe Handy nicht mehreren Mitarbeitern zugeordnet werden.
\clearpage
        \subsubsection{Fall 2: \#A ist wesentlich kleiner als \#B}
          Betrachten wir nun den Fall, dass es wesentlich mehr B-Objekte als A-Objekte gibt. W\"urde man jetzt die beiden Objekttypen in einer Tabelle vereinen, h\"atten die Attribute der A-Objekte in vielen Zeilen den NULL-Wert. Um dies zu vermeiden, werden zwei Tabellen angelegt: eine B-Tabelle f\"ur alle B-Objekte und eine A-Tabelle f\"ur die seltenen A-Objekte, die durch einen eingabepflichtigen Fremdschl\"ussel jeweils auf \enquote{ihr} B-Objekt verweisen. Die Kardinalit\"at \enquote{1} von (0,1) auf der B-Seite wird dadurch erzwungen, dass der Fremdschl\"ussel in A als unikal deklariert wird. Diese Umsetzung entspricht der Transformationsregel T04.
        \subsubsection{Transformationsregel T04 f\"ur den (1,1):(0,1) Beziehungstyp (oft realisierte Optionalit\"at)}
          \tablefirsthead{%
          \hline
            \multicolumn{1}{|c}{\textbf{konzeptionelles Datenmodell}} &
            \multicolumn{1}{|l}{\textbf{ }} &
            \multicolumn{1}{|c|}{\textbf{physisches Datenmodell}} \\
          \hline
          }
          \begin{supertabular}[h]{|l|l|p{7,25cm}|}
            \footnotesize Objekttyp A mit Schl\"ussel \pk{SA} & $\Rightarrow $ & \footnotesize Tabelle A mit PK \pk{SA}\\
            \hline
            \footnotesize Objekttyp B mit Schl\"ussel \pk{SB} & $\Rightarrow $ & \footnotesize Tabelle B mit PK \pk{SB}\\
            \hline
            \footnotesize (1,1):(0,1) Beziehungstyp & $\Rightarrow $ & \footnotesize \pk{SB} wird in A als ein\-ga\-be\-pflich\-tiger uni\-kaler Fremd\-schl\"ussel auf\-ge\-nommen (\fk{SB}[NN, UN])\\
          \end{supertabular}

          \begin{center}
            \scalebox{.7}{
              \begin{tikzpicture}[node distance=1.5cm, every edge/.style={link}]
                \node[entity](A){A};
                  \node[attribute](SA)[left = of A]{\key{SA}} edge (A);
                \node[relationship](rel1)[right = of A]{} edge node[auto,swap] {(1,1)}(A);
                \node[entity](B)[right = of rel1]{B} edge node[auto,swap] {(0,1)} (rel1);
                  \node[attribute](SB)[right = of B]{\key{SB}} edge (B);
            \end{tikzpicture}
            }
          \end{center}
        \subsubsection{Beispiel - Projektleiter}
          Soll beispielsweise festgehalten werden, welcher Mitarbeiter ein - und h\"ochstens ein - Projekt leitet, wobei f\"ur jedes Projekt genau ein Mitarbeiter verantwortlich ist, so wird es viele Mitarbeiter ohne Projektverantwortung geben. Daher ist in diesem Fall die Transformationsregel T04 anzuwenden.

          \begin{center}
            \scalebox{.7}{
              \begin{tikzpicture}[node distance=1.5cm, every edge/.style={link}]
                \node[entity](A){Projekt};
                  \node[attribute](a1)[left = of A]{\key{Projeknummer}} edge (A);
                  \node[attribute](a2)[above left = of A]{Bezeichnung} edge (A);
                \node[relationship](rel1)[right = of A]{leitet} edge node[auto,swap] {(1,1)}(A);
                \node[entity](B)[right = of rel1]{Mitarbeiter} edge node[auto,swap] {(0,1)} (rel1);
                  \node[attribute](Personalnummer)[right = of B]{\key{Personalnummer}} edge (B);
                  \node[attribute](Name)[above right = of B]{Name} edge (B);
            \end{tikzpicture}
            }
          \end{center}
          \begin{small}
            (A:) Projekt(\pk{Projeknummer}, Bezeichnung, \fk{Personalnummer} [NN, UN])

            (B:) Mitarbeiter(\pk{Personalnummer}, Name)
          \end{small}

          Da der Fremdschl\"ussel \fk{ Personalnummer } eingabepflichtig ist, muss jedes Projekt genau einen Projektleiter haben. Andererseits ist der Fremdschl\"ussel unikal, so dass ein Mitarbeiter nur f\"ur ein Projekt als Leiter ausgewiesen sein kann. Da es nicht m\"oglich ist sicherzustellen, dass jede Personalnummer eines Mitarbeiters auch tats\"achlich als Wert in der Fremdschl\"usselspalte auftritt, kann es Mitarbeiter geben, die mit keinem Projekt als Leiter verbunden sind.
      \subsection{Der (0,1):(0,1) Beziehungstyp}
        Beim (0,1):(0,1) Beziehungstyp ist jedes Objekt des Typs A mit keinem oder einem Objekt des Typs B verbunden und umgekehrt. Es gibt also A-Objekte ohne B-Partner und B-Objekte ohne A-Partner. Jedes der Objekte kann aber h\"ochstens einen Partner haben. Schematisch ist dies in der folgenden Abbildung dargestellt.

        \begin{center}
          \scalebox{1}{
            \begin{tikzpicture}
              \node[circleA](A) at (0,0){Objekttyp A};
                \node[redspot](a1) at (0, 1.3) {a1};
                \node[redspot](a2) at (0.5, 0.8){a2};
                \node[redspot](a3) at (0.1, 0.3){a3};
                \node[redspot](a4) at (-0.8, -0.4){a4};
                \node[redspot](a5) at (-0.2, -0.8){a5};
              \node[circleB](B) at (4.5, 0){Objekttyp B};
                \node[bluespot](b1) at (4.5, 1.3) {b1} edge (a1);
                \node[bluespot](b2) at (5, 0.8) {b2} edge (a3);
                \node[bluespot](b3) at (4.6, 0.3) {b3};
                \node[bluespot](b4) at (3.7, -0.4) {b4};
                \node[bluespot](b5) at (4.3, -0.8) {a4};
            \end{tikzpicture}
          }
        \end{center}
        Wir nehmen f\"ur die folgenden Betrachtungen - ohne Beschr\"ankung der Allgemeinheit - an, dass es h\"ochstens so viele B-Objekte gibt wie A-Objekte (\#A $\geq$ \#B, andernfalls m\"ussen die Objekttpypen einfach die Seiten tauschen). Da sowohl die A-Objekte als auch die B-Objekte unabh\"angig voneinander existieren k\"onnen, m\"ussen sie jeweils in einer eigenen Tabelle gespeichert werden.

        Bei den B-Objekten wird ein Verweis auf \enquote{ihren} A-Partner hinterlegt. Bei den \enquote{partnerlosen} B-Objekten hat dieser Verweis dann den NULL-Wert. Deshalb wird der Pri\-m\"ar\-schl\"us\-sel von A in der Tabelle B als nicht-eingabepflichtiger Fremdschl\"ussel aufgenommen. Fordert man f\"ur den Fremdschl\"ussel ausserdem die Unikalit\"at, kann auf ein A-Objekt nur von h\"ochstens einem B-Objekt aus verwiesen werden. Die Regel T05 zeigt diesen Sachverhalt.
        \subsubsection{Transformationsregel T05 f\"ur den (0,1):(0,1) Beziehungstyp}
          \tablefirsthead{%
          \hline
            \multicolumn{1}{|c}{\textbf{konzeptionelles Datenmodell}} &
            \multicolumn{1}{|l}{\textbf{ }} &
            \multicolumn{1}{|c|}{\textbf{physisches Datenmodell}} \\
          \hline
          }
          \begin{supertabular}[h]{|l|l|p{7,25cm}|}
            \footnotesize Objekttyp A mit Schl\"ussel \pk{SA} & $\Rightarrow $ & \footnotesize Tabelle A mit PK \pk{SA}\\
            \hline
            \footnotesize Objekttyp B mit Schl\"ussel \pk{SB} & $\Rightarrow $ & \footnotesize Tabelle B mit PK \pk{SB}\\
            \hline
            \footnotesize (0,1):(0,1) Beziehungstyp & $\Rightarrow $ & \footnotesize \pk{SA} wird in B als nicht-ein\-gabe\-pflichtiger uni\-kaler Fremd\-schl\"ussel auf\-ge\-nommen (\un{\fk{SA}})\\
          \end{supertabular}
          \begin{center}
            \scalebox{.7}{
              \begin{tikzpicture}[node distance=1.5cm, every edge/.style={link}]
                \node[entity](A){A};
                  \node[attribute](SA)[left = of A]{\key{SA}} edge (A);
                \node[relationship](rel1)[right = of A]{} edge node[auto,swap] {(0,1)}(A);
                \node[entity](B)[right = of rel1]{B} edge node[auto,swap] {(0,1)} (rel1);
                  \node[attribute](SB)[right = of B]{\key{SB}} edge (B);
            \end{tikzpicture}
            }
          \end{center}
          \bild{}{transformationsregel_05_tabellen}{0.35}

        \subsubsection{Beispiel - Ehe ohne Scheidung}
          In einem hier nicht n\"aher genannten Land wird die monogame Ehe praktiziert, jedoch mit dem Unterschied zu Deutschland, dass eine Scheidung rechtlich nicht vorgesehen ist. Die Eheschlie\ss{}ung zweier Personen lie\ss{}e sich somit als bin\"arer (0,1):(0,1)-Beziehungstyp darstellen, da jede Frau h\"ochstens einen Mann und jeder Mann h\"ochstens eine Frau heiraten kann.
          \begin{center}
            \scalebox{.7}{
              \begin{tikzpicture}[node distance=1.5cm, every edge/.style={link}]
                \node[entity](A){Frau};
                  \node[attribute](a1)[left = of A]{\key{Personen\_ID}} edge (A);
                  \node[attribute](a2)[above left = of A]{Name} edge (A);
                \node[relationship](rel1)[right = of A]{heiratet} edge node[auto,swap] {(0,1)}(A);
                \node[entity](B)[right = of rel1]{Mann} edge node[auto,swap] {(0,1)} (rel1);
                  \node[attribute](b1)[right = of B]{\key{Personen\_ID}} edge (B);
                  \node[attribute](b2)[above right = of B]{Name} edge (B);
            \end{tikzpicture}
            }
          \end{center}
          \begin{small}
            (A:) Frau(\pk{Personen\_ID}, Name)

            (B:) Mann(\pk{Personen\_ID}, Name, \un{\fk{ Ehegatten\_ID}})
          \end{small}

         Nicht verheiratete M\"anner haben, statt des Verweises auf die \fk{Personen\_ID} des Ehegatten, einen NULL-Wert. Wegen der Unikalit\"at des Fremdschl\"ussels kann eine Frau h\"ochs\-tens einen Mann heiraten.
      \subsection{Der (1,1):(0,*) Beziehungstyp}
        Beim (1,1):(0,*) Beziehungstyp ist jedes Objekt B mit keinem, einem oder
        mehreren Objekten des Objekttyps A verbunden. Ein A-Objekt ist dagegen
        immer mit genau einen B-Objekt gekoppelt. Die folgende Abbildung zeigt
        dies schematisch.

        \begin{center}
          \scalebox{0.97}{
            \begin{tikzpicture}
              \node[circleA](A) at (0,0){Objekttyp A};
                \node[redspot](a1) at (0, 1.3) {a1};
                \node[redspot](a2) at (0.5, 0.8){a2};
                \node[redspot](a3) at (0.1, 0.3){a3};
                \node[redspot](a4) at (-0.8, -0.4){a4};
                \node[redspot](a5) at (-0.2, -0.8){a5};
              \node[circleB](B) at (4.5, 0){Objekttyp B};
                \node[bluespot](b1) at (4.5, 1.3) {b1};
                \node[bluespot](b2) at (5, 0.8) {b2};
                \node[bluespot](b3) at (4.6, 0.3) {b3};
                \node[bluespot](b4) at (3.7, -0.4) {b4};
                \node[bluespot](b5) at (4.3, -0.8) {b5};
              \path (a1) edge (b1)
                (a2) edge (b2)
                (a3) edge (b3)
                (a4) edge (b3)
                (a5) edge (b4);
            \end{tikzpicture}
          }
        \end{center}
        Der Beziehungstyp wird im physischen Datenbankmodell durch je eine
        Tabelle f\"ur die A-Objekte und die B-Objekte repr\"asentiert, wobei der
        PK \pk{SB} von B als eingabepflichtiger Fremdschl\"ussel in A
        aufgenommen wird. Jedes A-Objekt muss dann auf genau ein B-Objekt
        verweisen. Da der Fremdschl\"ussel aber nicht als unikal vereinbart ist,
        k\"onnen mehrere A-Objekte mit demselben B-Objekt gekoppelt sein.
\clearpage
        \subsubsection{Transformationsregel T06 f\"ur den (1,1):(0,*) Beziehungstyp}
          \tablefirsthead{%
          \hline
            \multicolumn{1}{|c}{\textbf{konzeptionelles Datenmodell}} &
            \multicolumn{1}{|l}{\textbf{ }} &
            \multicolumn{1}{|c|}{\textbf{physisches Datenmodell}} \\
          \hline
          }
          \begin{supertabular}[h]{|l|l|p{7.5cm}|}
            \footnotesize Objekttyp A mit Schl\"ussel \pk{SA} & $\Rightarrow $ & \footnotesize Tabelle A mit PK \pk{SA}\\
            \hline
            \footnotesize Objekttyp B mit Schl\"ussel \pk{SB} & $\Rightarrow $ & \footnotesize Tabelle B mit PK \pk{SB}\\
            \hline
            \footnotesize (1,1):(0,*) Beziehungstyp & $\Rightarrow $ & \footnotesize \pk{SB} wird in A als ein\-ga\-be\-pflich\-tiger nicht-unikaler Fremd\-schl\"ussel auf\-ge\-nom\-men (\nn{\fk{SB}})\\
          \end{supertabular}
          \begin{center}
            \scalebox{.7}{
              \begin{tikzpicture}[node distance=1.5cm, every edge/.style={link}]
                \node[entity](A){A};
                  \node[attribute](a1)[left = of A]{\key{SA}} edge (A);
                \node[relationship](rel1)[right = of A]{} edge node[auto,swap] {(1,1)}(A);
                \node[entity](B)[right = of rel1]{B} edge node[auto,swap] {(0,*)} (rel1);
                  \node[attribute](b1)[right = of B]{\key{SB}} edge (B);
            \end{tikzpicture}
            }
          \end{center}
        \subsubsection{Beispiel - Gehaltsgruppen}
          Beispielsweise kann eine Gehaltsgruppe noch keinem, erst einem Mitarbeiter oder bereits mehreren Mitarbeitern zugeordnet sein. Andererseits wird jeder Mitarbeiter in genau eine Gehaltsgruppe eingeordnet.
          \begin{center}
            \scalebox{.7}{
              \begin{tikzpicture}[node distance=1.5cm, every edge/.style={link}]
                \node[entity](A){Mitarbeiter};
                  \node[attribute](a1)[left = of A]{\key{Personalnummer}} edge (A);
                  \node[attribute](a2)[above left = of A]{Name} edge (A);
                \node[relationship](rel1)[right = of A]{hat} edge node[auto,swap] {(1,1)}(A);
                \node[entity](B)[right = of rel1]{Gehaltsgruppe} edge node[auto,swap] {(0,*)} (rel1);
                  \node[attribute](b1)[right = of B]{\key{Gruppennummer}} edge (B);
                  \node[attribute](b2)[above right = of B]{Gehalt} edge (B);
            \end{tikzpicture}
            }
          \end{center}
          \begin{small}
            (B:) Gehaltsgruppe(\pk{Gruppennummer}, Gehalt)

            (A:) Mitarbeiter(\pk{Personalnummer}, Name, \nn{\fk{Gruppennummer}})
          \end{small}

          Da der Fremdschl\"ussel \fk{Gruppennummer} eingabepflichtig ist, muss jedem Mitarbeiter genau eine Gehaltsgruppe zugeordnet werden. Andererseits ist der Fremdschl\"ussel nichtunikal, so dass mehrere Mitarbeiter auf dieselbe Gehaltsgruppe verweisen k\"onnen. Da nicht zu fordern ist, dass jeder Wert des Prim\"arschl\"ussels \pk{Gruppennummer} auch tats\"achlich als Wert des Fremdschl\"ussels \fk{Gruppennummer} auftreten muss, kann es Gehaltsgruppen geben, auf die noch nicht verwiesen wird.
      \subsection{Der (0,1):(0,*) Beziehungstyp}
        Beim (0,1):(0,*) Beziehungstyp kann ein Objekt des Typs B mit keinem, einem oder mehreren A-Objekten gekoppelt sein. Ein A-Objekt kann aber zu h\"ochstens einem B-Objekt in Beziehung stehen.

        \begin{center}
          \scalebox{1}{
            \begin{tikzpicture}
              \node[circleA](A) at (0,0){Objekttyp A};
                \node[redspot](a1) at (0, 1.3) {a1};
                \node[redspot](a2) at (0.5, 0.8){a2};
                \node[redspot](a3) at (0.1, 0.3){a3};
                \node[redspot](a4) at (-0.8, -0.4){a4};
                \node[redspot](a5) at (-0.2, -0.8){a5};
              \node[circleB](B) at (4.5, 0){Objekttyp B};
                \node[bluespot](b1) at (4.5, 1.3) {b1};
                \node[bluespot](b2) at (5, 0.8) {b2};
                \node[bluespot](b3) at (4.6, 0.3) {b3};
                \node[bluespot](b4) at (3.7, -0.4) {b4};
                \node[bluespot](b5) at (4.3, -0.8) {b5};
              \path (a1) edge (b1)
                (a2) edge (b2)
                (a3) edge (b3)
                (a4) edge (b3);
            \end{tikzpicture}
          }
        \end{center}
        Die Transformation dieses Beziehungstyps erfolgt, indem der Prim\"arschl\"ussel von B als nicht-ein\-ga\-be\-pflich\-ti\-ger und nicht-unikaler Fremdschl\"ussel in A aufgenommen wird.
        \subsubsection{Transformationsregel T07 f\"ur den (0,1):(0,*) Beziehungstyp}
          \tablefirsthead{%
          \hline
            \multicolumn{1}{|c}{\textbf{konzeptionelles Datenmodell}} &
            \multicolumn{1}{|l}{\textbf{ }} &
            \multicolumn{1}{|c|}{\textbf{physisches Datenmodell}} \\
          \hline
          }
          \begin{supertabular}[h]{|l|l|p{7.5cm}|}
            \footnotesize Objekttyp A mit Schl\"ussel \pk{SA} & $\Rightarrow $ & \footnotesize Tabelle A mit PK \pk{SA}\\
            \hline
            \footnotesize Objekttyp B mit Schl\"ussel \pk{SB} & $\Rightarrow $ & \footnotesize Tabelle B mit PK \pk{SB}\\
            \hline
            \footnotesize (0,1):(0,*) Beziehungstyp & $\Rightarrow $ & \footnotesize \pk{SB} wird in A als nicht\-ein\-ga\-be\-pflich\-tiger nicht-uni\-kaler Fremd\-schl\"ussel auf\-ge\-nommen (\fk{SB})\\
          \end{supertabular}
          \begin{center}
            \scalebox{.7}{
              \begin{tikzpicture}[node distance=1.5cm, every edge/.style={link}]
                \node[entity](A){A};
                  \node[attribute](a1)[left = of A]{\key{SA}} edge (A);
                \node[relationship](rel1)[right = of A]{} edge node[auto,swap] {(0,1)}(A);
                \node[entity](B)[right = of rel1]{B} edge node[auto,swap] {(0,*)} (rel1);
                  \node[attribute](b1)[right = of B]{\key{SB}} edge (B);
            \end{tikzpicture}
            }
          \end{center}
        \subsubsection{Beispiel - Abteilungsmitarbeiter}
          Betrachten wir eine Abteilung, die noch keinen, schon einen oder bereits mehrere Mitarbeiter hat. Die meisten Mitarbeiter sind in eine Abteilung eingeordnet, einige wenige - mit zentralen Aufgaben - geh\"oren jedoch zu keiner Abteilung.
          \begin{center}
            \scalebox{.7}{
              \begin{tikzpicture}[node distance=1.5cm, every edge/.style={link}]
                \node[entity](A){Mitarbeiter};
                  \node[attribute](a1)[left = of A]{\key{Personalnummer}} edge (A);
                  \node[attribute](a2)[above left = of A]{Name} edge (A);
                \node[relationship](rel1)[right = of A]{} edge node[auto,swap] {(0,1)}(A);
                \node[entity](B)[right = of rel1]{Abteilung} edge node[auto,swap] {(0,*)} (rel1);
                  \node[attribute](b1)[right = of B]{\key{Abteilungsnummer}} edge (B);
                  \node[attribute](b2)[above right = of B]{Jahresbudget} edge (B);
                  \node[attribute](b3)[below right = of B]{Bezeichnung} edge (B);
            \end{tikzpicture}
            }
          \end{center}
          \begin{small}
            (B:) Abteilung(\pk{Abteilungsnummer}, Jahresbudget, Bezeichnung)

            (A:) Mitarbeiter(\pk{Personalnummer}, Name, \fk{Abteilungsnummer})
          \end{small}

          Bei den mit zentralen Aufgaben betrauten Mitarbeitern wird in der Fremdschl\"usselspalte \fk{Abteilungsnummer} ein NULL-Wert stehen. Da der Fremdschl\"ussel nicht-unikal ist, k\"onnen mehrere Mitarbeiter mit derselben Abteilung in Verbindung gebracht werden.

          Da es ohnehin nicht m\"oglich ist zufordern, dass jeder Wert eines Prim\"arschl\"ussels auch in der Spalte des Fremdschl\"ussels \fk{Abteilungsnummer} auftritt, kann es Abteilungen ohne Mitarbeiter geben.
      \subsection{Der (1,1):(1,*) Beziehungstyp}
        Der (1,1):(1,*) Beziehungstyp unterscheidet sich vom (1,1):(0,*) Beziehungstyp nur dadurch, dass jedes Objekt des Objekttyps B mit mindestens einem Objekt des Objekttyps A verbunden sein muss.
        \begin{center}
          \scalebox{1}{
            \begin{tikzpicture}
              \node[circleA](A) at (0,0){Objekttyp A};
                \node[redspot](a1) at (0, 1.3) {a1};
                \node[redspot](a2) at (0.5, 0.8){a2};
                \node[redspot](a3) at (0.1, 0.3){a3};
                \node[redspot](a4) at (-0.8, -0.4){a4};
                \node[redspot](a5) at (-0.2, -0.8){a5};
              \node[circleB](B) at (4.5, 0){Objekttyp B};
                \node[bluespot](b1) at (4.5, 1.3) {b1};
                \node[bluespot](b2) at (5, 0.8) {b2};
                \node[bluespot](b3) at (4.6, 0.3) {b3};
                \node[bluespot](b4) at (3.7, -0.4) {b4};
              \path (a1) edge (b1)
                (a2) edge (b2)
                (a3) edge (b3)
                (a4) edge (b3)
                (a5) edge (b4);
            \end{tikzpicture}
          }
        \end{center}
        In der Literatur zum physischen Datenbankmodell (auch relationales Modell) wird der Beziehungstyp (1,1):(1,*) gew\"ohnlich als \enquote{meistverbreitetste} Art von Beziehungstypen bezeichnet. Das ist aber falsch, denn dieser Beziehungstyp l\"asst sich im physikalischen Datenbankmodell gar nicht re\-pr\"as\-en\-tie\-ren. Der Grund daf\"ur liegt darin, dass es auf der Ebene der Tabellen-Typ\-be\-schrei\-bung\-en keine M\"oglichkeit gibt, die Nichtoptionalit\"at der Beziehungstyprichtung von B zu A zu repr\"asentieren.

        Betrachten wir die Situation im Einzelnen. Die letzten L\"osungen zur Transformation eines Beziehungstyps bestanden darin, den PK der B-Tabelle in die A-Tabelle als FK aufzunehmen. Soll nun jedes B-Objekt mit mindestens einem A-Objekt gekoppelt sein, so ist das gleichbedeutend mit der Forderung, das jeder Wert, den der PK B annimmt, wenigstens einmal als Wert des FK in A auftauchen muss. Diese Forderung hat nichts mit der referentiellen Integrit\"at zu tun. Diese fordert nur, dass jeder Wert des FK in A entweder der NULL-Wert sein oder aber als Wert des PK in B vorhanden sein muss. Die Umkehrrichtung, dass jeder Wert des PK auch als Wert des FK auftauchen muss, l\"asst sich im physischen Datenbankmodell nicht formulieren.

        Die Transformation des (1,1):(1,*) Beziehungstyps in das physische Datenbankmodell erfolgt deshalb auf die gleiche Weise, wie die des (1,1):(0,*) Beziehungstyps, gem\"a\ss\ T06. Dabei muss allerdings in Kauf genommen werden, dass wichtige semantische Informationen, die im konzeptionellen Datenmodell repr\"asentiert werden, verloren gehen. Diese Informationen k\"onnen nur im Rahmen der Anwendungsprogrammierung ber\"ucksichtigt werden.
        \subsubsection{Beispiel Kundenbestellung}
          Betrachten wir ein Unternehmen, in dem eine Gesch\"aftsregel besagt, dass ein Kunde erst dann gespeichert wird, wenn er die erste Bestellung vornimmt. Im Laufe der Zeit k\"onnen einem Kunden nat\"urlich mehrere Bestellungen zugeordnet werden. Jede Bestellung kommt von genau einem Kunden. Ein Kunde, f\"ur den keine Bestellung mehr besteht (weil er alle seine Bestellungen storniert hat), wird wieder gel\"oscht. Das Datenmodell muss nach der Transformationsregel T06, entsprechend dem folgenden Bild, umgesetzt werden.
          \begin{center}
            \scalebox{.7}{
              \begin{tikzpicture}[node distance=1.5cm, every edge/.style={link}]
                \node[entity](A){Bestellung};
                  \node[attribute](a1)[left = of A]{\key{Bestellnummer}} edge (A);
                  \node[attribute](a2)[above left = of A]{Datum} edge (A);
                \node[relationship](rel1)[right = of A]{} edge node[auto,swap] {(1,1)}(A);
                \node[entity](B)[right = of rel1]{Kunde} edge node[auto,swap] {(1,*)} (rel1);
                  \node[attribute](b1)[right = of B]{\key{Kundennummer}} edge (B);
                  \node[attribute](b2)[above right = of B]{Name} edge (B);
            \end{tikzpicture}
            }
          \end{center}
          \begin{small}
            (B:) Kunde(\pk{Kundennummer}, Name)

            (A:) Bestellung(\pk{Bestellnummer}, Datum, \nn{\fk{Kundennummer}})
          \end{small}

          Der FK \fk{Kundennummer} ist eingabepflichtig: Jede Bestellung wird also genau einem Kunden zugeordnet. Der FK ist nicht-unikal: Mehrere Bestellungen k\"onnen also auf denselben Kunden verweisen. Es wird aber nicht gesichert, dass jeder Kunde mit mindestens einer Bestellung verkn\"upft ist. Diese Gesch\"aftsregel kann nicht beim Datenbankentwurf \enquote{festgeschrieben} werden, sondern muss durch die Anwendungssoftware erzwungen werden.
      \subsection{Der (0,1):(1,*) Beziehungstyp}
        Der (0,1):(1,*) Beziehungstyp unterscheidet sich vom (1,1):(1,*) Beziehungstyp dadurch, dass nicht jedes Objekt des Objekttyps A mit einem Objekt des Objekttyps B gekoppelt sein muss. Es m\"ussen aber wieder alle Objekte aus B mindestens eine Beziehung zu Objekten aus A haben.
        \begin{center}
          \scalebox{1}{
            \begin{tikzpicture}
              \node[circleA](A) at (0,0){Objekttyp A};
                \node[redspot](a1) at (0, 1.3) {a1};
                \node[redspot](a2) at (0.5, 0.8){a2};
                \node[redspot](a3) at (0.1, 0.3){a3};
                \node[redspot](a4) at (-0.8, -0.4){a4};
                \node[redspot](a5) at (-0.2, -0.8){a5};
              \node[circleB](B) at (4.5, 0){Objekttyp B};
                \node[bluespot](b1) at (4.5, 1.3) {b1};
                \node[bluespot](b2) at (5, 0.8) {b2};
                \node[bluespot](b3) at (4.6, 0.3) {b3};
              \path (a1) edge (b1)
                (a2) edge (b2)
                (a3) edge (b3)
                (a4) edge (b3);
            \end{tikzpicture}
          }
        \end{center}
\clearpage
        Wie schon beim (1,1):(1,*) Beziehungstyp, so ist auch hier die Nichtoptionalit\"at der Beziehungstyprichtung von B zu A im physischen Datenbankmodell nicht zu erzwingen. Die Transformation des (0,1):(1,*) Beziehungstyps kann nur wie beim (0,1):(0,*) Beziehungstyp erfolgen, also gem\"a\ss\ T07. Es muss auch hier der Verlust von semantischen Informationen in Kauf genommen werden.
       \subsubsection{Beispiel - Exklusivangebotsbestellung}
          Betrachten wir als Beispiel einen Versandhandel, der Exklusivangebote f\"ur seine Kunden erstellt. Jeder Kunde kann zu einem Angebot h\"ochstens eine Bestellung abschicken, er muss es aber nicht tun (Kardinalit\"at (0,1)). In einer Bestellung kann der Kunde sich aber nicht nur auf ein Angebot beziehen, sondern auch auf mehrere (Kardinalit\"at (1,*)).
          \begin{center}
            \scalebox{.7}{
              \begin{tikzpicture}[node distance=1.5cm, every edge/.style={link}]
                \node[entity](A){Angebot};
                  \node[attribute](a1)[left = of A]{\key{Angebotsnummer}} edge (A);
                  \node[attribute](a2)[above left = of A]{Angebotsdatum} edge (A);
                \node[relationship](rel1)[right = of A]{enth\"alt} edge node[auto,swap] {(0,1)}(A);
                \node[entity](B)[right = of rel1]{Bestellung} edge node[auto,swap] {(1,*)} (rel1);
                  \node[attribute](b1)[right = of B]{\key{Bestellnummer}} edge (B);
                  \node[attribute](b2)[above right = of B]{Bestelldatum} edge (B);
            \end{tikzpicture}
            }
          \end{center}
          \begin{small}
            (B:) Bestellung(\pk{Bestellnummer}, Bestelldatum)

            (A:) Angebot(\pk{Angebotsnummer}, Angebotsdatum, \fk{Bestellnummer})
          \end{small}
          Der FK \fk{Bestellnummer} ist nicht-eingabepflichtig: Ein Angebot muss also nicht unbedingt eine Bestellung hervorrufen. Der FK ist nicht-unikal: Mehrere Angebote k\"onnen in einer Bestellung genutzt werden. Es l\"asst sich aber nicht erzwingen, dass jede Bestellung sich auf mindestens ein Angebot bezieht. Die Datenbank w\"urde also durchaus zulassen, dass ein Kunde eine Bestellung t\"atigt, ohne ein entsprechendes Angebot vorliegen zu haben. Will man diesen, in der Praxis nicht hinnehmbaren Fehler vermeiden, kann dies nur durch eine entsprechende Gestaltung der Anwendungssoftware erreicht werden.
      \subsection{Der (0,*):(0,*) Beziehungstyp}
        Beziehungstypen, die in beiden Richtungen die Kardinalit\"at \enquote{*} aufweisen, lassen sich nicht direkt im physischen Datenbankmodell darstellen. Der Grund daf\"ur liegt in der Tatsache, dass Attributwerte nur atomare Werte haben d\"urfen. Der Wert eines FK kann somit nur auf eine Zeile und nicht auf mehrere Zeilen verweisen.

        Beim (0,*):(0,*) Beziehungstyp kann ein Objekt des Objekttyps A jedoch nicht nur mit keinem oder einem, sondern auch mit mehreren Objekten des Objekttyps B verbunden sein. Ebenso wie ein B-Objekt mit keinem, einem oder mehreren A-Objekten gekoppelt sein kann. Folgende Abbildung zeigt daf\"ur ein Beispiel.
        \begin{center}
          \scalebox{1}{
            \begin{tikzpicture}
              \node[circleA](A) at (0,0){Objekttyp A};
                \node[redspot](a1) at (0, 1.3) {a1};
                \node[redspot](a2) at (0.5, 0.8){a2};
                \node[redspot](a3) at (0.1, 0.3){a3};
                \node[redspot](a4) at (-0.8, -0.4){a4};
                \node[redspot](a5) at (-0.2, -0.8){a5};
              \node[circleB](B) at (4.5, 0){Objekttyp B};
                \node[bluespot](b1) at (4.5, 1.3) {b1};
                \node[bluespot](b2) at (5, 0.8) {b2};
                \node[bluespot](b3) at (4.6, 0.3) {b3};
                \node[bluespot](b4) at (3.7, -0.4) {b4};
                \node[bluespot](b5) at (4.3, -0.8) {b5};
              \path (a1) edge (b1)
                (a1) edge (b2)
                (b3) edge (a3)
                (b3) edge (a4)
                (a5) edge (b4);
            \end{tikzpicture}
          }
        \end{center}
        Die Repr\"asentation ist im physischen Datenbankmodell nur dadurch m\"oglich, dass man einen neuen - rein technisch bedingten - Hilfsobjekttyp A/B einf\"uhrt. In mancher Literatur wird auch von einem Koppel-Objekttyp gesprochen. A/B wird mit den Objekttypen A und B jeweils durch einen (0,*):(1,1) Beziehungstyp verbunden.
          \begin{center}
          \scalebox{1}{
            \begin{tikzpicture}
              \node[circleA](A) at (0,0){Objekttyp A};
                \node[redspot](a1) at (0, 1.3) {a1};
                \node[redspot](a2) at (0.5, 0.8){a2};
                \node[redspot](a3) at (0.1, 0.3){a3};
                \node[redspot](a4) at (-0.8, -0.4){a4};
                \node[redspot](a5) at (-0.2, -0.8){a5};
              \node[circleC](C) at (4.5, 0){Hilfstyp AB};
                \node[yellowspot](c1) at (4.5, 1.3) {c1};
                \node[yellowspot](c2) at (5.0, 0.8) {c2};
                \node[yellowspot](c3) at (4.6, 0.3) {c3};
                \node[yellowspot](c4) at (3.7, -0.4) {c4};
                \node[yellowspot](c5) at (4.3, -0.8) {c5};
              \node[circleB](B) at (9, 0){Objekttyp B};
                \node[bluespot](b1) at (9.0, 1.3) {b1};
                \node[bluespot](b2) at (9.5, 0.8) {b2};
                \node[bluespot](b3) at (9.1, 0.3) {b3};
                \node[bluespot](b4) at (8.2, -0.4) {b4};
                \node[bluespot](b5) at (8.8, -0.8) {b5};
              \path (a1) edge (c1)
                    (a1) edge (c2)
                    (a3) edge (c3)
                    (a4) edge (c4)
                    (a5) edge (c5)
                    (c1) edge (b1)
                    (c2) edge (b2)
                    (c3) edge (b3)
                    (c4) edge (b3)
                    (c5) edge (b4);
            \end{tikzpicture}
          }
        \end{center}
        Die beiden (1,1):(0,*) Beziehungstypen werden gem\"a\ss\ T06 umgewandelt. Zusammenfassend gilt f\"ur den (0,*):(0,*) Beziehungstyp die Transformationsregel T08.
        \subsubsection{Transformationsregel T08 f\"ur den (0,*):(0,*) Beziehungstyp}
          \tablefirsthead{%
          \hline
            \multicolumn{1}{|c}{\textbf{ER-Modell}} &
            \multicolumn{1}{|l}{\textbf{ }} &
            \multicolumn{1}{|c|}{\textbf{physisches Datenmodell}} \\
          \hline
          }
          \begin{supertabular}[h]{|l|l|p{7.75cm}|}
            \footnotesize Objekttyp A mit Schl\"ussel \pk{SA} & $\Rightarrow $ & \footnotesize Tabelle A mit PK \pk{SA}\\
            \hline
            \footnotesize Objekttyp B mit Schl\"ussel \pk{SB} & $\Rightarrow $ & \footnotesize Tabelle B mit PK \pk{SB}\\
            \hline
            \footnotesize (0,*):(0,*) Beziehungstyp & $\Rightarrow $ & \footnotesize Hilfstabelle A/B. \pk{SA} und \pk{SB} werden als eingabepflichtige nicht-unikale Fremdschl\"ussel in A/B aufgenommen. Die Kombination der FK \fk{SA} und \fk{SB} als unikal vereinbart. Sie bildet den PK von A/B\\
          \end{supertabular}
          \begin{center}
            \scalebox{.7}{
              \begin{tikzpicture}[node distance=1.5cm, every edge/.style={link}]
                \node[entity](A){A};
                  \node[attribute](a1)[left = of A]{\key{SA}} edge (A);
                \node[relationship](rel1)[right = of A]{} edge node[auto,swap] {(0,*)}(A);
                \node[entity](B)[right = of rel1]{B} edge node[auto,swap] {(0,*)} (rel1);
                  \node[attribute](b1)[right = of B]{\key{SB}} edge (B);
            \end{tikzpicture}
            }
          \end{center}
          \bild{}{transformationsregel_08_tabellen}{0.85}
        \subsubsection{Beispiel - Projektmitarbeiter}
          Als Beispiel betrachten wir noch einmal die Mitarbeiter, die an keinem, einem oder mehreren Projekten beteiligt sein k\"onnen, wobei ein Projekt (noch) von keinem, einem oder auch von mehreren Mitarbeitern bearbeitet wird.
          \begin{center}
            \scalebox{.7}{
              \begin{tikzpicture}[node distance=1.5cm, every edge/.style={link}]
                \node[entity](A){Projekt};
                  \node[attribute](a1)[left = of A]{\key{Projeknummer}} edge (A);
                  \node[attribute](a2)[above left = of A]{Bezeichnung} edge (A);
                \node[relationship](rel1)[right = of A]{arbeitet} edge node[auto,swap] {(0,*)}(A);
                \node[entity](B)[right = of rel1]{Mitarbeiter} edge node[auto,swap] {(0,*)} (rel1);
                  \node[attribute](b1)[right = of B]{\key{Personalnummer}} edge (B);
                  \node[attribute](b2)[above right = of B]{Name} edge (B);
            \end{tikzpicture}
            }
          \end{center}
          \begin{small}
            (B:) Mitarbeiter(\pk{Personalnummer}, Name)

            (A:) Projekt(\pk{Projektnummer}, Bezeichnung)

            (A/B:) MitarbeiterProjekt(\fk{\pk{Personalnummer + Projektnummer}})
          \end{small}
          Eine Zeile der Hilfstabelle \enquote{MitarbeiterProjekt} verweist auf genau einen Mitarbeiter und auf genau ein Projekt. Da der FK \fk{Personalnummer} f\"ur sich alleine nichtunikal ist, kann es mehrere Zeilen der Hilfstabelle geben, die auf denselben Mitarbeiter verweisen. Die Nichtunikalit\"at des FK \fk{Projektnummer} erm\"oglicht es, dass mehrere Zeilen der Hilfstabelle mit demselben Projekt verkn\"upft sind. Erst die Kombination \fk{Personalnummer +  Projektnummer} ist als unikal vereinbart und bildet den PK der Hilfstabelle.
      \subsection{Der (1,*):(0,*) Beziehungstyp}
        Beim (1,*):(0,*) Beziehungstyp muss ein Objekt des Objekttyps A mit mindestens einem oder mehreren Objekten des Objekttyps B verbunden sein, ein B-Objekt jedoch mit keinem, einem oder aber mehreren A-Objekten.

        \begin{center}
          \scalebox{1}{
            \begin{tikzpicture}
              \node[circleA](A) at (0,0){Objekttyp A};
                \node[redspot](a1) at (0, 1.3) {a1};
                \node[redspot](a2) at (0.5, 0.8){a2};
                \node[redspot](a3) at (0.1, 0.3){a3};
                \node[redspot](a4) at (-0.8, -0.4){a4};
                \node[redspot](a5) at (-0.2, -0.8){a5};
              \node[circleB](B) at (4.5, 0){Objekttyp B};
                \node[bluespot](b1) at (4.5, 1.3) {b1};
                \node[bluespot](b2) at (5, 0.8) {b2};
                \node[bluespot](b3) at (4.6, 0.3) {b3};
                \node[bluespot](b4) at (3.7, -0.4) {b4};
                \node[bluespot](b5) at (4.3, -0.8) {b5};
                \node[bluespot](b6) at (5.2, -1) {b6};
              \path (a1) edge (b1)
                    (a1) edge (b2)
                    (b3) edge (a3)
                    (b3) edge (a4)
                    (a2) edge (b4)
                    (a5) edge (b5);
            \end{tikzpicture}
          }
        \end{center}

        Analog zum (0,*):(0,*) Beziehungstyp muss der (1,*):(0,*) Beziehungstyp vor seiner Transformation in das physische Datenbankmodell durch die Einf\"uhrung eines neuen Objekttyps A/B in einen (1,1):(0,*) Beziehungstyp und einen (1,*):(1,1) Beziehungstyp umgewandelt werden.
\clearpage
        \begin{merke}
          Es wurde bereits gezeigt, dass der (1,*):(1,1) Beziehungstyp im physischen Datenbankmodell lediglich als (0,*):(1,1) Beziehungstyp dargestellt werden kann. Die Transformation des (1,*):(0,*) Beziehungstyps erfolgt damit nach der Transformationsregel T08.
        \end{merke}
        \subsubsection{Beispiel - \"Arzte mit Operation(en)}
          Als Beispiel betrachten wir \"Arzte, die Operationen durchf\"uhren. Eine Operation ohne \"Arzte gibt es nicht. Mitunter wird eine Operation aber von mehreren \"Arzten ausgef\"uhrt.
          \begin{center}
            \scalebox{.65}{
              \begin{tikzpicture}[node distance=1.5cm, every edge/.style={link}]
                \node[entity](A){Operation};
                  \node[attribute](a1)[left = of A]{\key{Operationsnummer}} edge (A);
                  \node[attribute](a2)[above left = of A]{Bezeichnung} edge (A);
                \node[relationship](rel1)[right = of A]{f\"uhrt durch} edge node[auto,swap] {(1,*)}(A);
                \node[entity](B)[right = of rel1]{Arzt} edge node[auto,swap] {(0,*)} (rel1);
                  \node[attribute](b1)[right = of B]{\key{Personalnummer}} edge (B);
                  \node[attribute](b2)[above right = of B]{Name} edge (B);
            \end{tikzpicture}
            }
          \end{center}
          \begin{small}
            (B:) Arzt(\pk{Personalnummer}, Name)

            (A:) Operation(\pk{Operationsnummer}, Bezeichnung)

            (B/A:) ArztOperation(\fk{\pk{Personalnummer + Operationsnummer}})
          \end{small}
          Durch jede Zeile der Hilfstabelle \enquote{ArztOperation} wird ein Arzt mit einer Operation in Verbindung gebracht. Keiner der beiden Fremdschl\"ussel \fk{Personalnummer} beziehungsweise \fk{Operationsnummer} muss f\"ur sich genommen unikal sein. Damit k\"onnen mehrere Zeilen der Hilfstabelle auf denselben Arzt verweisen. Ebenso k\"onnen mehrere Zeilen dieselbe Operation betreffen.

          Es kann aber durch die Tabellentypbeschreibungen nicht erzwungen werden, dass jede Operationsnummer auch tats\"achlich als FK in der Hilfstabelle auftaucht. In der DB kann also die falsche Aussage gespeichert werden, dass einer Operation kein Arzt zugeordnet ist. Dieser Fehler kann wiederum nur software-technisch vermieden werden.
      \subsection{Der (1,*):(1,*) Beziehungstyp}
        Beim (1,*):(1,*) Beziehungstyp ist jedes Objekt des Objekttyps A mit einem oder mehreren Objekten des Objekttyps B verbunden, ebenso wie jedes B-Objekt mit einem oder mehreren A-Objekten gekoppelt ist.

        \begin{center}
          \scalebox{1}{
            \begin{tikzpicture}
              \node[circleA](A) at (0,0){Objekttyp A};
                \node[redspot](a1) at (0, 1.3) {a1};
                \node[redspot](a2) at (0.5, 0.8){a2};
                \node[redspot](a3) at (0.1, 0.3){a3};
                \node[redspot](a4) at (-0.8, -0.4){a4};
                \node[redspot](a5) at (-0.2, -0.8){a5};
              \node[circleB](B) at (4.5, 0){Objekttyp B};
                \node[bluespot](b1) at (4.5, 1.3) {b1};
                \node[bluespot](b2) at (5, 0.8) {b2};
                \node[bluespot](b3) at (4.6, 0.3) {b3};
                \node[bluespot](b4) at (3.7, -0.4) {b4};
                \node[bluespot](b5) at (4.3, -0.8) {b5};
              \path (a1) edge (b1)
                (a2) edge (b2)
                (a3) edge (b3)
                (a4) edge (b3)
                (a5) edge (b4)
                (a5) edge (b5);
            \end{tikzpicture}
          }
        \end{center}
        Dieser Beziehungstyp muss vor der Transformation in einen (1,*):(1,1) und einen\\ (1,1):(1,*) Beziehungstyp umgeformt werden. Beide neuen Beziehungstypen m\"ussen wiederum nur unter Semantikverlust in (0,*):(1,1) bzw. (1,1):(0,*) Beziehungstypen \"uber\-f\"uhrt werden. Die Transformation des (1,*):(1,*) Beziehungstyps erfolgt damit - so wie f\"ur den (1,*):(0,*) Beziehungstyp - nach der Transformationsregel T08.
        \subsubsection{Beispiel - B\"uchersachgruppe}
          Als Beispiel betrachten wir eine Bibliothek, in der die B\"ucher den
          einzelnen Sachgruppen zugeordnet werden. Ein Buch wird meist nur f\"ur
          eine Sachgruppe, mitunter aber auch zu mehreren Sachgruppen
          zugeordnet. Eine Sachgruppe wird erst dann eingef\"uhrt, wenn sie
          wenigstens ein Buch enth\"alt. Die Transformation sieht folgenderma\ss
          en aus.

          \begin{center}
            \scalebox{.7}{
              \begin{tikzpicture}[node distance=1.5cm, every edge/.style={link}]
                \node[entity](A){Buch};
                  \node[attribute](a1)[left = of A]{\key{ISBN}} edge (A);
                  \node[attribute](a2)[above left = of A]{Titel} edge (A);
                \node[relationship](rel1)[right = of A]{geh\"ort zu} edge node[auto,swap] {(1,*)}(A);
                \node[entity](B)[right = of rel1]{Sachgruppe} edge node[auto,swap] {(1,*)} (rel1);
                  \node[attribute](b1)[right = of B]{\key{Sachgruppencode}} edge (B);
                  \node[attribute](b2)[above right = of B]{Bezeichnung} edge (B);
            \end{tikzpicture}
            }
          \end{center}
          \begin{small}
            (A:) Buch(\pk{ISBN}, Titel)

            (B:) Sachgruppe(\pk{Sachgruppencode}, Bezeichnung)

            (A/B:) BuchSachgruppe(\fk{\pk{ISBN + Sachgruppencode}})
          \end{small}
          
          Eine Zeile der Hilfstabelle \enquote{Buch/Sachgruppe} ordnet ein Buch
          einer Sachgruppe zu. Da weder der FK \fk{ISBN} noch der FK
          \fk{Sachgruppencode} f\"ur sich unikal sind, k\"onnen mehrere Zeilen
          der Hilfstabelle auf dasselbe Buch oder auch auf dieselbe Sachgruppe
          verweisen. Die Tabellentypbeschreibungen sichern aber weder, dass
          einem Buch wenigstens eine Sachgruppe zugeordnet wird, noch k\"onnen
          sie garantieren, dass es keine \enquote{leere} Sachgruppe gibt. Diese
          Gesch\"aftsregeln m\"ussen von der Anwendungssoftware durchgesetzt
          werden.
      \subsection{Transformation von Beziehungsattributen}
        Bei der Transformation von Eigenschaften eines Beziehungstyps (Beziehungsattributen) kann folgende Verallgemeinerung angewandt werden:
        \begin{itemize}
          \item Das Beziehungsattribut \enquote{wandert} in die Tabelle auf der 1-Seite des Beziehungstyps.
          \item Gibt es auf beiden Seiten einen * in der Kardinalit\"at, dann \enquote{wandert/wandern} das/die Beziehungsattribut(e) in das Hilfsobjekt.
          \item Gibt es keine *-Seite, dann muss die angewandte Transformationsregel betrachtet und gepr\"uft werden, in welcher Tabelle das Beziehungsattribut am sinnvollsten ist. Normalerweise ist dies die Tabelle mit dem FK.
        \end{itemize}
        \subsubsection{Transformation von Beziehungsattributen bei einer m:n-Beziehung}
          \begin{center}
            \scalebox{.7}{
              \begin{tikzpicture}[node distance=1.5cm, every edge/.style={link}]
                \node[entity](A){Maschine};
                  \node[attribute](a1)[left = of A]{\key{Maschinennummer}} edge (A);
                  \node[attribute](a2)[above left = of A]{Bezeichnung} edge (A);
                \node[relationship](rel1)[right = of A]{stellt her} edge node[auto,swap] {(1,*)}(A);
                  \node[attribute](rel11)[above = of rel1]{Zeitstempel} edge (rel1);
                \node[entity](B)[right = of rel1]{Bauteil} edge node[auto,swap] {(1,*)} (rel1);
                  \node[attribute](b1)[right = of B]{\key{Teilenummer}} edge (B);
                  \node[attribute](b2)[above right = of B]{Bezeichnung} edge (B);
            \end{tikzpicture}
            }
          \end{center}
          \begin{small}
            (A:) Maschine(\pk{Maschinennummer}, Bezeichnung)

            (B:) Bauteil(\pk{Teilenummer}, Bezeichnung)

            (A/B:) Herstellung(\fk{\pk{Maschinennummer + Teilenummer}}, Zeitstempel)
          \end{small}
        \subsubsection{Transformation von Beziehungsattributen bei einer 1:n-Beziehung}
          \begin{center}
            \scalebox{.7}{
              \begin{tikzpicture}[node distance=1.5cm, every edge/.style={link}]
                \node[entity](A){Maschine};
                  \node[attribute](a1)[left = of A]{\key{Maschinennummer}} edge (A);
                  \node[attribute](a2)[above left = of A]{Bezeichnung} edge (A);
                \node[relationship](rel1)[right = of A]{stellt her} edge node[auto,swap] {(1,*)}(A);
                  \node[attribute](rel11)[above = of rel1]{Zeitstempel} edge (rel1);
                \node[entity](B)[right = of rel1]{Bauteil} edge node[auto,swap] {(1,1)} (rel1);
                  \node[attribute](b1)[right = of B]{\key{Teilenummer}} edge (B);
                  \node[attribute](b2)[above right = of B]{Bezeichnung} edge (B);
              \end{tikzpicture}
            }
          \end{center}
          \begin{small}
            (B): Maschine(\pk{Maschinennummer}, Bezeichnung)

            (A): Bauteil(\pk{Teilenummer}, Bezeichnung, \fk{Maschinennummer}, Zeitstempel)
          \end{small}
        \subsubsection{Transformation von Beziehungsattributen bei einer 1:1-Beziehung}
          Da bei Beziehungen des Typs 1:1 nur extrem selten Beziehungsattribute vorkommen, muss in solchen F\"allen eine Einzelfallbetrachtung durchgef\"uhrt werden. Daher wird an dieser Stelle auf ein Transformationsbeispiel verzichtet.
\clearpage
			\section{Transformation des Begleitbeispiels Bundeswehr}
        In diesem Abschnitt sollen die bis hier erlernten Dinge praktisch ge\"ubt werden. Aufgabe soll es sein, f\"ur das begleitende Beispiel \enquote{Bundeswehr}, ein ER-Modell zu konzipieren und dieses anschlie\ss end, mittels der in den vorigen Abschnitten kennengelernten Tranformationsregeln, in ein relationales Modell umzuandeln.

        Nachfolgendes ER-Modell kann als L\"osungsansatz f\"ur das Beispiel \enquote{Bundeswehr} gew\"ahlt werden.

					\begin{tabular}{>{\textbf\bgroup}p{5cm}<{\egroup}>{\small}p{10.1cm}}
						Dienstposten & (\pk{Dienstposten\_ID}, Beginndatum, Enddatum, Aufgabenbeschreibung) \\
						Soldat & (\pk{Personen\_ID}, PK, Vorname, Nachname, Dienstgrad) \\
						Ausr\"ustung & (\pk{Versorgungsnummer}, Bezeichnung, Verwendungszweck, Material, Farbe) \\
						Dieststelle & (\pk{Dienststellennummer}, Bezeichnung, Typ, Gr\"o\ss e) \\
						Standort & (\pk{Ort\_ID}, PLZ, Ort, Stra\ss e, Hausnummer) \\
						& Aufl\"osen des Beziehungstyps (besetzt) mit T04) \\
						& Aufl\"osen des Beziehungstyps (besteht aus) mit T06) \\
						& Aufl\"osen der Beziehungstypen (besitzt, befindet sich an) mit T08) \\
						& Aufl\"osen des Beziehungstyps (ist untergeordnet) mit T11) \\
						DienststelleStandort & (\fk{\pk{Dienststellennummer + Ort\_ID}}) \\
						SoldatAusr\"ustung & (\fk{\pk{Personen\_ID + Versorgungsnummer}}) \\
					\end{tabular}

        \begin{center}
            \scalebox{.5}{
              \begin{tikzpicture}[node distance=1.5cm, every edge/.style={link}]
                \node[entity](dienstposten){Dienstposten};
                  \node[attribute](dienstpostenid)[above left = of dienstposten]{\key{Dienstposten\_ID}} edge (dienstposten);
                  \node[attribute](begindatum)[above right = of dienstposten]{Beginndatum} edge (dienstposten);
                  \node[attribute](enddatum)[below right = of dienstposten]{Enddatum} edge (dienstposten);
                  \node[attribute](aufgabenbeschreibung)[below left = of dienstposten]{Aufgabenbeschreibung} edge (dienstposten);
                \node[relationship](besetzt)[left = 5.1cm of dienstposten]{besetzt} edge node[auto,swap] {(0,1)}(dienstposten);
                \node[entity](soldat)[below left = 7cm of dienstposten]{Soldat} edge node [auto,swap] {(1,1)}(besetzt);
                  \node[attribute](personenid)[above left = 1.6cm of soldat]{\key{Personen\_ID}} edge (soldat);
                  \node[attribute](pk)[above right = of soldat]{PK} edge (soldat);
                  \node[attribute](name)[below right = of soldat]{Nachname} edge (soldat);
                  \node[attribute](vorname)[below left = of soldat]{Vorname} edge (soldat);
                  \node[attribute](dienstgrad)[left = of soldat]{Dienstgrad} edge (soldat);
                \node[relationship](besitzt)[below = 3cm of soldat]{besitzt} edge node[auto,swap] {(0,*)}(soldat);
                \node[entity](ausruestung)[below = 8cm of soldat]{Ausr\"ustung} edge node[auto,swap] {(0,*)}(besitzt);
                  \node[attribute](versorgungsnummer)[above left = of ausruestung]{\key{Versorgungsnummer}} edge (ausruestung);
                  \node[attribute](bezeichnung)[left = of ausruestung]{Bezeichnung} edge (ausruestung);
                  \node[attribute](verwendungszweck)[above right = of ausruestung]{Verwendungszweck} edge (ausruestung);
                  \node[attribute](material)[below right = of ausruestung]{Material} edge (ausruestung);
                  \node[attribute](farbe)[below left = of ausruestung]{Farbe} edge (ausruestung);
                \node[relationship](bestehtaus)[right = 4.7cm of dienstposten]{besteht aus} edge node[auto,swap] {(1,1)}(dienstposten);

                \node[entity](dienststelle)[below right = 7cm of dienstposten]{Dienststelle} edge node [auto,swap] {(1,*)}(bestehtaus);

                  \node[attribute](dienststellennummer)[above left = of dienststelle]{\key{Dienststellennummer}} edge (dienststelle);
                  \node[attribute](bezeichung)[below left = of dienststelle]{Bezeichnung} edge (dienststelle);
                  \node[attribute](groesse)[below right = 0.9cm of dienststelle]{Gr\"o\ss e} edge (dienststelle);
                  \node[attribute](typ)[above right = 0.9cm of dienststelle]{Typ} edge (dienststelle);

                \node[relationship](istuntergeordnet)[right = 2cm of dienststelle]{ist untergeordnet};
                  \node[auto,swap](l1) at (11,-3.6) {(1,1)};
                  \node[auto,swap](l2) at (11,-8.7) {(0,*)};
                  \path [draw, -] (dienststelle) |- ($(dienststelle.south) +(1.5, -1.5)$) -| (istuntergeordnet);
                  \path [draw, -] (dienststelle) |- ($(dienststelle.north) +(1.5, 1.5)$) -| (istuntergeordnet);

                \node[relationship](befidetsichan)[below = 2cm of dienststelle]{befindet sich an} edge node[auto,swap] {(1,*)}(dienststelle);
                \node[entity](standort)[below = 8cm of dienststelle]{Standort} edge node[auto,swap] {(1,*)}(befidetsichan);
                  \node[attribute](ortid)[above left = of standort]{\key{Ort\_ID}} edge (standort);
                  \node[attribute](plz)[left = of standort]{PLZ} edge (standort);
                  \node[attribute](ort)[above right = of standort]{Ort} edge (standort);
                  \node[attribute](strasse)[below right = of standort]{Stra\ss e} edge (standort);
                  \node[attribute](hausnummer)[below left = of standort]{Hausnummer} edge (standort);
              \end{tikzpicture}
            }
            \end{center}
      Bei weiterer Betrachtung des ER-Modells sollte auffallen, dass die Anzahl an Ausr\"ustungsgegenst\"anden eines Soldaten nicht ber\"ucksichtigt wurde. In der Tabelle \enquote{SoldatAusr\"ustung} kann ein Attribut \enquote{Anzahl} aufgenommen werden, welches die jeweilige Anzahl an Ausr\"ustungsgegenst\"anden eines Soldaten enth\"alt. Die neue Tabelle sieht dann folgenderma\ss en aus:

      \begin{small}
        SoldatAusr\"ustung(\fk{\pk{Personen\_ID + Versorgungsnummer}}, Anzahl)
      \end{small}

      Dieser Fall muss sich dann mit einem Beziehungsattribut (vgl. Abschnitt \ref{attributes_of_entities}) im ER-Modell wiederfinden. Das angepasste ER-Modell sieht folgenderma\ss en aus:
      \begin{center}
        \scalebox{.5}{
        \begin{tikzpicture}[node distance=1.5cm, every edge/.style={link}]
          \node[entity](dienstposten){Dienstposten};
            \node[attribute](dienstpostenid)[above left = of dienstposten]{\key{Dienstposten\_ID}} edge (dienstposten);
            \node[attribute](begindatum)[above right = of dienstposten]{Begindatum} edge (dienstposten);
            \node[attribute](enddatum)[below right = of dienstposten]{Enddatum} edge (dienstposten);
            \node[attribute](aufgabenbeschreibung)[below left = of dienstposten]{Aufgabenbeschreibung} edge (dienstposten);
          \node[relationship](besetzt)[left = 5.1cm of dienstposten]{besetzt} edge node[auto,swap] {(0,1)}(dienstposten);
          \node[entity](soldat)[below left = 7cm of dienstposten]{Soldat} edge node [auto,swap] {(1,1)}(besetzt);
            \node[attribute](personenid)[above left = 1.6cm of soldat]{\key{Personen\_ID}} edge (soldat);
            \node[attribute](pk)[above right = of soldat]{PK} edge (soldat);
            \node[attribute](name)[below right = of soldat]{Name} edge (soldat);
            \node[attribute](vorname)[below left = of soldat]{Vorname} edge (soldat);
            \node[attribute](dienstgrad)[left = of soldat]{Dienstgrad} edge (soldat);
          \node[relationship](besitzt)[below = 3cm of soldat]{besitzt} edge node[auto,swap] {(0,*)}(soldat);
            \node[attribute](anzahl)[right  = of besitzt]{Anzahl} edge (besitzt);

          \node[entity](ausruestung)[below = 8cm of soldat]{Ausr\"ustung} edge node[auto,swap] {(0,*)}(besitzt);
            \node[attribute](versorgungsnummer)[above left = of ausruestung]{\key{Versorgungsnummer}} edge (ausruestung);
            \node[attribute](bezeichnung)[left = of ausruestung]{Bezeichnung} edge (ausruestung);
            \node[attribute](verwendungszweck)[above right = of ausruestung]{Verwendungszweck} edge (ausruestung);
            \node[attribute](material)[below right = of ausruestung]{Material} edge (ausruestung);
            \node[attribute](farbe)[below left = of ausruestung]{Farbe} edge (ausruestung);
          \node[relationship](bestehtaus)[right = 4.7cm of dienstposten]{besteht aus} edge node[auto,swap] {(1,1)}(dienstposten);

          \node[entity](dienststelle)[below right = 7cm of dienstposten]{Dienststelle} edge node [auto,swap] {(1,*)}(bestehtaus);

            \node[attribute](dienststellennummer)[above left = of dienststelle]{\key{Dienststellennummer}} edge (dienststelle);
            \node[attribute](bezeichung)[below left = of dienststelle]{Bezeichnung} edge (dienststelle);
            \node[attribute](groesse)[below right = 0.9cm of dienststelle]{Gr\"o\ss e} edge (dienststelle);
            \node[attribute](typ)[above right = 0.9cm of dienststelle]{Typ} edge (dienststelle);

          \node[relationship](istuntergeordnet)[right = 2cm of dienststelle]{ist untergeordnet};
            \node[auto,swap](l1) at (11,-3.6) {(0,1)};
            \node[auto,swap](l2) at (11,-8.7) {(0,*)};
            \path [draw, -] (dienststelle) |- ($(dienststelle.south) +(1.5, -1.5)$) -| (istuntergeordnet);
            \path [draw, -] (dienststelle) |- ($(dienststelle.north) +(1.5, 1.5)$) -| (istuntergeordnet);

          \node[relationship](befidetsichan)[below = 2cm of dienststelle]{befindet sich an} edge node[auto,swap] {(1,*)}(dienststelle);
          \node[entity](standort)[below = 8cm of dienststelle]{Standort} edge node[auto,swap] {(1,*)}(befidetsichan);
            \node[attribute](ortid)[above left = of standort]{\key{Ort\_ID}} edge (standort);
            \node[attribute](plz)[left = of standort]{PLZ} edge (standort);
            \node[attribute](ort)[above right = of standort]{Ort} edge (standort);
            \node[attribute](strasse)[below right = of standort]{Stra\ss e} edge (standort);
            \node[attribute](hausnummer)[below left = of standort]{Hausnummer} edge (standort);
        \end{tikzpicture}
        }
      \end{center}

    \section{Transformation rekursiver Beziehungstypen}
      \label{recursive_relatiaons}
      Da bei Rekursiv-Beziehungstypen die miteinander gekoppelten Objekte beide im selben Objekttyp liegen, k\"onnen sie nicht - wie bei den bin\"aren Beziehungstypen - durch ihre Zugeh\"origkeit zum jeweiligen Objekttyp unterschieden werden. Bei Rekursiv-Be\-ziehungs\-typen m\"ussen die Objekte hinsichtlich ihrer Rolle unterschieden werden. Deshalb soll folgende Sprachregelung eingef\"uhrt werden. Wir sprechen bei einem Rekursiv-Beziehungstyp allgemein von einem \enquote{Sender} (rechte Seite), der eine Nachricht an einen \enquote{Empf\"anger} (linke Seite) sendet. Als Beispiel daf\"ur ist in der folgenden Abbildung ein (0,1):(0,*) Rekursiv-Beziehungstyp dargestellt.

      \begin{center}
        \scalebox{.7}{
          \begin{tikzpicture}[node distance=1.5cm, every edge/.style={link}]
            \node[entity](A){A};
              \node[attribute](a1)[left = of A]{\key{SA}} edge (A);
            \node[relationship](rel1)[right = of A]{};
            \node[auto,swap](l1) at (5.6,-1.4) {(0,*) empf\"angt Nachricht};
            \node[auto,swap](l2) at (5.6,1.4) {(0,1) sendet Nachricht};
            \path [draw, -] (A) |- ($(A.south) + (0.5,-0.5)$) -| (rel1);
            \path [draw, -] (A) |- ($(A.north) +(0.5, 0.5)$) -| (rel1);
          \end{tikzpicture}
        }
      \end{center}

      An dieser Stelle sei auf den Abschnitt \ref{syntaxofrecursiverelationtypes} mit der \tabelle{combinationsrecurisverelationtyps} verwiesen. In der \tabelle{combinationsrecurisverelationtyps} sind die 7 m\"oglichen Rekursiv-Beziehungstypen aufgelistet.

      \subsection{Der (1,1):(1,1) Rekursiv-Beziehungstyp}
        Beim (1,1):(1,1) Beziehungstyp muss jedes Objekt des Objekttyps genau eine Nachricht an ein Objekt desselben Objekttyps senden. Umgekehrt muss jedes Objekt genau eine Nachricht von einem anderen Objekt empfangen. Die folgende Abbildung zeigt die vorliegenden Verh\"altnisse.

        \begin{center}
          \scalebox{1}{
            \begin{tikzpicture}[]
              \node[circleA](A){Objekttyp A};
              \node[redspot](a1) at (1.0, 0.5){a1};
              \node[redspot](a2) at (0.0, 1.7){a2};
              \node[redspot](a3) at (-1.0, 0.5){a3};
              \node[redspot](a4) at (-0.5, -0.4) {a4};

      \path[->] (a1) edge[bend right=45] (a2)
                (a2) edge[bend right=45] (a3)
                (a3) edge[bend right=45] (a1)
                (a4) edge[loop right, out=270, in = 0, looseness=18] (a4);
            \end{tikzpicture}
          }
        \end{center}

        Wie man sehen kann, k\"onnen ausschlie\ss lich Objekt-Zyklen gebildet werden ($a_1 \rightarrow a_2 \rightarrow a_3 \rightarrow a_1$). Im Minimalfall gibt es nur ein Objekt, welches dann einen Ein-Objekt-Zyklus bildet ($a_1 \rightarrow a_1$).

        Bei der Repr\"asentation von Rekursiv-Beziehungstypen im physischen Datenbankmodell muss zu\-n\"ach\-st jedes Objekt des Objekttyps A - unabh\"angig von den anderen Objekten - in einer Tabellenzeile gespeichert werden. Die Beziehung zwischen zwei Objekten $a_1$ und $a_2$ kann nun dadurch ausgedr\"uckt werden, dass in der Tabellenzeile von $a_2$ (beim Empf\"anger) auf das Objekt $a_1$ (auf den Sender) verwiesen wird. Der Verweis wird, wie schon bei den bin\"aren Beziehungstypen, durch Abspeichern des Indentifikators von $a_1$ realisiert. Dieser Identifikator ist aber der Wert, den der Prim\"arschl\"ussel von A im Falle des Objekts $a_1$ annimmt. Die Tabelle f\"ur den Objekttyp A muss somit den Schl\"ussel des Objekttyps A in doppelter Ausf\"uhrung aufnehmen:
\clearpage
        \begin{enumerate}
          \item als Prim\"arschl\"ussel, um eine Zeile der Tabelle eindeutig identifizieren zu k\"onnen;
          \item als Fremdschl\"ussel, um auf eine andere (oder auch auf dieselbe) Zeile der Tabelle, n\"amlich auf den Sender verweisen zu k\"onnen.
        \end{enumerate}
        Da die Spaltenbezeichnungen einer Tabelle voneinander verschieden sein m\"ussen, ist es erforderlich, f\"ur das Attribut (bzw. die Attribute) des PK im Falle ihrer \enquote{Wiedergeburt} als FK, andere Bezeichnungen zu vergeben.

        Im physischen Datenbankmodell l\"asst sich die Forderung, dass jedes Objekt genau eine Nachricht empfangen muss, einfach dadurch sichern, dass der FK, der auf den Sender verweist, als \textit{eingabepflichtig} deklariert wird, dass also jedes Objekt auf \enquote{sein} Senderobjekt verweisen muss. Dass ein und dasselbe Objekt nicht von mehreren Empf\"angern als ihr Sender angegeben werden kann, wird durch die Unikalit\"at des FK erreicht.
        \subsubsection{Transformationsregel T09 f\"ur den (1,1):(1,1) Rekursiv-Beziehungstyp}
          \tablefirsthead{%
          \hline
            \multicolumn{1}{|c}{\textbf{ER-Modell}} &
            \multicolumn{1}{|l}{\textbf{ }} &
            \multicolumn{1}{|c|}{\textbf{physisches Datenmodell}} \\
          \hline
          }
          \begin{supertabular}[h]{|l|l|p{7.5cm}|}
            \footnotesize Objekttyp A mit Schl\"ussel \pk{SA} & $\Rightarrow $ & \footnotesize Tabelle A mit PK \pk{SA}\\
            \hline
            \footnotesize (1,1):(1,1) Rekursiv-Beziehungstyp & $\Rightarrow $ & \footnotesize \pk{SA} wird in A mit einer anderen Attributbezeichnung \enquote{gedoppelt} und als eingabepflichtiger unikaler FK \fk{SA'} vereinbart, der auf den Sender verweist.
            (\fk{SA'} [NN, UN])\\
          \end{supertabular}

      \begin{center}
        \scalebox{.7}{
          \begin{tikzpicture}[node distance=1.5cm, every edge/.style={link}]
            \node[entity](A){A};
              \node[attribute](a1)[left = of A]{\key{SA}} edge (A);
            \node[relationship](rel1)[right = of A]{};
            \node[auto,swap](l1) at (5.6,-1.4) {(1,1) empf\"angt Nachricht};
            \node[auto,swap](l2) at (5.6,1.4) {(1,1) sendet Nachricht};
            \path [draw, -] (A) |- ($(A.south) + (0.5,-0.5)$) -| (rel1);
            \path [draw, -] (A) |- ($(A.north) +(0.5, 0.5)$) -| (rel1);
          \end{tikzpicture}
        }
      \end{center}

      \bild{}{transformationsregel_09_tabellen}{0.35}

        \subsubsection{Beispiel - Kreis einer Turnergruppe}
          Als Beispiel betrachten wir eine Turnergruppe, die bei einem Sportfest einen Kreis bilden soll. F\"ur jeden Turner wird festgelegt, wer sein rechter Nachbar ist.

          Von jedem Turner wird obligatorisch auf seinen (einzigen) rechten Nachbarn verwiesen. Durch die Unikalit\"at des FK wird au\ss erdem gesichert, dass ein Turner nur f\"ur einen anderen Turner rechter Nachbar sein kann. Da jedem Turner ein rechter Nachbar zugeordnet wird, muss auch jeder Turner rechter Nachbar genau eines anderen Turners sein. Damit ist die Kreisstruktur der Aufstellung der Turner erzwungen. Allerdings l\"asst sich durch die Tabellentypbeschreibung nicht ausschlie\ss en, dass ein Turner als sein eigener rechter Nachbar eingesetzt wird. (Ein-Objekt-Zyklus).

          Schwierig wird bei diesem Beziehungstyp die Speicherung der Daten so vorzunehmen, dass keine Integrit\"atsverletzung auftritt. Es sind entsprechende software-technische Ma\ss nahmen zu treffen, die hier nicht weiter erl\"autert werden sollen.
          \begin{center}
            \scalebox{.7}{
              \begin{tikzpicture}[node distance=1.5cm, every edge/.style={link}]
                \node[entity](A){Turner};
                  \node[attribute](a1)[left = of A]{\key{Mitgliedsnummer}} edge (A);
                  \node[attribute](a2)[above left= of A]{Name} edge (A);
                  \node[attribute](a3)[below left = of A]{rechter-Nachbar} edge (A);
                \node[relationship](rel1)[right = of A]{};
                \node[auto,swap](l1) at (5.6,-1.4) {(1,1) ist rechter Nachbar};
                \node[auto,swap](l2) at (5.6,1.4) {(1,1) hat rechten Nachbar};
                \path [draw, -] (A) |- ($(A.south) + (0.5,-0.5)$) -| (rel1);
                \path [draw, -] (A) |- ($(A.north) +(0.5, 0.5)$) -| (rel1);
              \end{tikzpicture}
            }
          \end{center}
          \begin{small}
            Turner(\pk{Mitgliedsnummer}, Name, \fk{rechter-Nachbar} [NN, UN])
          \end{small}
      \subsection{Der (0,1):(0,1) Rekursiv-Beziehungstyp}
        Beim (0,1):(0,1) Rekursiv-Beziehungstyp kann ein Objekt keine oder eine Nachricht an ein anderes Objekt (oder an sich selbst) senden. Andererseits kann ein Objekt keine oder eine Nachricht empfangen.
        \begin{center}
          \scalebox{1}{
            \begin{tikzpicture}[]
              \node[circleA](A){Objekttyp A};
              \node[redspot](a1) at (1.0, 0.5){a1};
              \node[redspot](a2) at (0, 1.7){a2};
              \node[redspot](a3) at (-1.0, 0.5){a3};
              \node[redspot](a4) at (-1, -0.6) {a4};
              \node[redspot](a5) at (1, -0.4) {a5};
              \node[redspot](a6) at (0.5, -0.8) {a6};
              \node[redspot](a7) at (0, -1.2) {a7};
              \node[redspot](a8) at (0, 0.85){a8};

      \path[->] (a1) edge[bend right=45] (a2)
                (a2) edge[bend right=45] (a3)
                (a3) edge[bend right=45] (a1)
                (a4) edge[loop right, out=270, in = 0, looseness=18] (a4)
                (a5) edge (a6)
                (a6) edge (a7);
            \end{tikzpicture}
          }
        \end{center}
        Beim (0,1):(0,1) Rekursiv-Beziehungstyp sind also - wie schon beim (1,1):(1,1) Rekursiv-Be\-zieh\-ungs\-typ - Objekt-Zyklen ($a_1 \rightarrow a_2 \rightarrow a_3 \rightarrow a_1$) m\"oglich, die im Minimalfall Ein-Objekt-Zyklen ($a_4 \rightarrow a_4$) sind. Dar\"uber hinaus kann es Objekt-Ketten ($a_5 \rightarrow a_6 \rightarrow a_7$) geben, die gegebenenfalls nur ein einziges Objekt enthalten ($a_8$). Ein solches Objekt ist dann weder Sender noch Empf\"anger.

        F\"ur die Umsetzung im physischen Modell, betrachten wir den Fall, dass die Objekte einen Verweis auf \enquote{ihren} Sender ben\"otigen. Dieser Sender kann der Empf\"anger selbst oder ein anderes A-Objekt sein. Da nicht jedes Objekt ein Empf\"anger ist, muss der FK, der auf den Sender verweist, als nicht-eingabepflichtig deklariert werden. Andererseits muss er unikal sein, damit ein $a_1$ Objekt nur von h\"ochstens einem Objekt $a_2$ als \enquote{sein} Sender ausgewiesen werden kann. Zusammengefasst ist dies die Transformationsregel T10.
\clearpage
        \subsubsection{Transformationsregel T10 f\"ur den (0,1):(0,1) Rekursiv-Beziehungstyp}
          \tablefirsthead{%
          \hline
            \multicolumn{1}{|c}{\textbf{ER-Modell}} &
            \multicolumn{1}{|l}{\textbf{ }} &
            \multicolumn{1}{|c|}{\textbf{physisches Datenmodell}} \\
          \hline
          }
          \begin{supertabular}[h]{|l|l|p{7.5cm}|}
            \footnotesize Objekttyp A mit Schl\"ussel \pk{SA} & $\Rightarrow $ & \footnotesize Tabelle A mit PK \pk{SA}\\
            \hline
            \footnotesize (0,1):(0,1) Rekursiv-Beziehungstyp & $\Rightarrow $ & \footnotesize \pk{SA} wird in A mit einer anderen Attributbezeichnung \enquote{gedoppelt} und als nicht-eingabepflichtiger unikaler FK \fk{SA'} vereinbart, der auf den Sender verweist. (\un{\fk{SA'}})\\
          \end{supertabular}
          \begin{center}
            \scalebox{.7}{
              \begin{tikzpicture}[node distance=1.5cm, every edge/.style={link}]
                \node[entity](A){A};
                  \node[attribute](a1)[left = of A]{\key{SA}} edge (A);
                \node[relationship](rel1)[right = of A]{};
                \node[auto,swap](l1) at (5.6,-1.4) {(0,1) empf\"angt Nachricht};
                \node[auto,swap](l2) at (5.6,1.4) {(0,1) sendet Nachricht};
                \path [draw, -] (A) |- ($(A.south) + (0.5,-0.5)$) -| (rel1);
                \path [draw, -] (A) |- ($(A.north) +(0.5, 0.5)$) -| (rel1);
              \end{tikzpicture}
            }
          \end{center}

          \bild{}{transformationsregel_10_tabellen}{0.35}

        \subsubsection{Beispiel - Ersatzbus}
          Nehmen wir als Beispiel an, dass in einem Busreiseunternehmen festgelegt wird, welcher Bus als Ersatz verwendet werden soll, wenn ein Bus defekt ist. Dabei soll ein Bus h\"ochstens f\"ur einen anderen als Ersatz dienen. F\"ur die wenigen neuen Busse mit geringer Ausfallwahrscheinlichkeit wird kein Ersatz vorgesehen und sie k\"onnen auch nicht als Ersatz dienen, da sie st\"andig im Einsatz sind. F\"ur andere Busse wird festgelegt, dass sie bei Sch\"aden schnell repariert werden; sie sind dann Ersatz f\"ur sich selbst (Ein-Objekt-Zyklen). Typischerweise wird f\"ur jeden Bus ein anderer als Ersatz festgelegt und jeder Bus kann als Ersatz f\"ur einen anderen Bus dienen (Objekte in einem Mehr-Objekte-Zyklus oder \enquote{innere Objekte} in einer Objekt-Kette). Es ist in seltenen F\"allen aber auch nicht auszuschlie\ss en, dass ein Bus zwar nicht als Ersatz dienen kann, aber einen Ersatz braucht (Anfangsglied einer Kette). Ebenso kann ein Bus in Reserve gehalten werden, also nur als Ersatz
dienen, ohne selbst einen Ersatz zu ben\"otigen (Endglied einer Kette).
          \begin{center}
            \scalebox{.7}{
              \begin{tikzpicture}[node distance=1.5cm, every edge/.style={link}]
                \node[entity](A){Bus};
                  \node[attribute](a1)[left = of A]{\key{PolKennzeichen}} edge (A);
                  \node[attribute](a2)[above left = of A]{Sitzplatzanzahl} edge (A);
                  \node[attribute](a3)[below left = of A]{ErsatzbusKennzeichen} edge (A);
                \node[relationship](rel1)[right = of A]{};
                \node[auto,swap](l1) at (5.6,-1.4) {(0,1) ist Ersatz f\"ur};
                \node[auto,swap](l2) at (5.6,1.4) {(0,1) hat als Ersatz};
                \path [draw, -] (A) |- ($(A.south) + (0.5,-0.5)$) -| (rel1);
                \path [draw, -] (A) |- ($(A.north) +(0.5, 0.5)$) -| (rel1);
              \end{tikzpicture}
            }
          \end{center}
          \begin{small}
            Bus(\pk{PolKennzeichen}, Sitzplatzanzahl, \un{\fk{ErsatzbusKennzeichen}})
          \end{small}
\clearpage
          Da der FK \fk{ErsatzbusKennzeichen} als nicht-eingabepflichtig deklariert ist, kann ein Bus auf keinen oder auf einen anderen Bus (evtl. auf sich selbst) als Ersatzbus verweisen. Da der FK unikal ist, kann auf einen Ersatzbus nur von \textit{einem} anderen Bus verwiesen werden. Andererseits ist nicht gefordert - und kann auch gar nicht gefordert werden -, dass jeder Wert des PK \pk{PolKennzeichen} auch tats\"achlich als FK auftritt. Somit kann es Busse geben, die nicht durch einen FK-Wert als Ersatzbusse ausgewiesen sind.
      \subsection{Der (1,1):(0,*) Rekursiv-Beziehungstyp}
        Beim (1,1):(0,*) Rekursiv-Beziehungstyp kann ein Objekt keine, eine oder mehrere Nachrichten senden, es muss aber stets genau eine Nachricht empfangen werden. Das entspricht den Bedingungen des (1,1):(1,1) Rekursiv-Beziehungstyps, zuz\"uglich der M\"oglichkeit eines Objekts, entweder gar nicht als Sender oder aber als Sender mehrerer Nachrichten aufzutreten.

        Da jedes Objekt des Objekttyps A genau eine Nachricht empfangen muss, m\"ussen auch \#A Nachrichten gesendet werden. F\"ur jedes Nicht-Sender-Objekt, muss somit ein anderes - gewisserma\ss en sein Vertreter - eine zus\"atzliche Nachricht senden.
        \begin{center}
          \scalebox{1}{
            \begin{tikzpicture}[]
              \node[circleA](A){Objekttyp A};
              \node[redspot](a1) at (1.0, 0.5){a1};
              \node[redspot](a2) at (0.0, 1.6){a2};
              \node[redspot](a3) at (-1.0, 0.5){a3};
              \node[redspot](a4) at (-1, -0.6) {a4};
              \node[redspot](a5) at (1, -0.4) {a5};
              \node[redspot](a6) at (0.5, -0.8) {a6};
              \node[redspot](a7) at (-0.2, -0.5) {a7};
              \node[redspot](a8) at (0.5, 1) {a8};
              \node[redspot](a9) at (0, 0.4){a9};

      \path[->] (a1) edge[bend right=45] (a2)
                (a2) edge[bend right=45] (a3)
                (a3) edge[bend right=45] (a1)
                (a4) edge[loop right, out=270, in = 0, looseness=18] (a4)
                (a8) edge[bend right=45] (a9)
                (a1) edge[bend left=45] (a5)

                (a5) edge (a6)
                (a5) edge (a7)
                (a1) edge (a8);
            \end{tikzpicture}
          }
        \end{center}
        Bei der Transformation, muss der Identifikator des Senders eingabepflichtig beim Empf\"anger hinterlegt werden. Wird der FK au\ss erdem als nicht-unikal deklariert, kann ein Objekt von mehreren Empf\"anger-Objekten als \enquote{ihr} Sender ausgewiesen werden. Folgend ist die Transformationsregel T11 dargestellt.
\clearpage
        \subsubsection{Transformationsregel T11 f\"ur den (1,1):(0,*) Rekursiv-Beziehungstyp}
          \tablefirsthead{%
          \hline
            \multicolumn{1}{|c}{\textbf{ER-Modell}} &
            \multicolumn{1}{|l}{\textbf{ }} &
            \multicolumn{1}{|c|}{\textbf{physisches Datenmodell}} \\
          \hline
          }
          \begin{supertabular}[h]{|l|l|p{7.5cm}|}
            \footnotesize Objekttyp A mit Schl\"ussel \pk{SA} & $\Rightarrow $ & \footnotesize Tabelle A mit PK \pk{SA}\\
            \hline
            \footnotesize (1,1):(0,*) Rekursiv-Beziehungstyp & $\Rightarrow $ & \footnotesize \pk{SA} wird in A mit anderen Attributbezeichnungen \enquote{gedoppelt} und als eingabepflichtiger nicht-unikaler FK \fk{SA'} vereinbart, der auf den Sender verweist. (\nn{\fk{SA'}})\\
          \end{supertabular}
          \begin{center}
            \scalebox{.7}{
              \begin{tikzpicture}[node distance=1.5cm, every edge/.style={link}]
                \node[entity](A){A};
                  \node[attribute](a1)[left = of A]{\key{SA}} edge (A);
                \node[relationship](rel1)[right = of A]{};
                \node[auto,swap](l1) at (5.6,-1.4) {(1,1) empf\"angt Nachricht};
                \node[auto,swap](l2) at (5.6,1.4) {(0,*) sendet Nachricht};
                \path [draw, -] (A) |- ($(A.south) + (0.5,-0.5)$) -| (rel1);
                \path [draw, -] (A) |- ($(A.north) +(0.5, 0.5)$) -| (rel1);
              \end{tikzpicture}
            }
          \end{center}
          \bild{}{transformationsregel_11_tabellen}{0.35}

        \subsubsection{Beispiel - Fotos der Vereinsmitglieder}
          Nehmen wir als Beispiel an, dass f\"ur die Festzeitschrift eines Vereins von allen Mitgliedern ein Foto ben\"otigt wird. Im Interesse der Kostend\"ampfung wird kein externer Fotograf hinzugezogen. Jedes Mitglied muss genau einmal fotografiert werden. Manche Mitglieder fotografieren gar nicht, andere m\"ussen daf\"ur umso mehr Fotos machen. Damit auch wirklich jeder Fotograf selbst auf einem Foto zu sehen ist, ist letztlich erforderlich, dass sich ein Fotograf entweder selbst fotografiert (Ein-Objekt-Zyklus) oder dass ihn jemand fotografiert, der selbst bereits fotografiert wurde (Mehr-Objekte-Zyklus). Die erforderliche Transformation ist in folgender Abbildung dargestellt.
          \begin{center}
            \scalebox{.7}{
              \begin{tikzpicture}[node distance=1.5cm, every edge/.style={link}]
                \node[entity](A){Mitglied};
                  \node[attribute](a1)[left = of A]{\key{Mitgliedsnummer}} edge (A);
                  \node[attribute](a2)[above left = of A]{Name} edge (A);
                  \node[attribute](a3)[below left = of A]{FotografenNr} edge (A);
                \node[relationship](rel1)[right = of A]{};
                \node[auto,swap](l1) at (5,-1.4) {(0,*) fotografiert};
                \node[auto,swap](l2) at (5.6,1.4) {(1,1) wird fotografiert von};
                \path [draw, -] (A) |- ($(A.south) + (0.5,-0.5)$) -| (rel1);
                \path [draw, -] (A) |- ($(A.north) +(0.5, 0.5)$) -| (rel1);
              \end{tikzpicture}
            }
          \end{center}
          \begin{small}
            Mitglied(\pk{Mitgliedsnummer}, Name, \nn{\fk{FotografenNr}})
          \end{small}
      \subsection{Der (0,1):(0,*) Rekursiv-Beziehungstyp}
        Der (0,1):(0,*) Rekursiv-Beziehungstyp unterscheidet sich vom (1,1):(0,*) Rekursiv-Be\-zieh\-ungs\-typ nur durch die Aufhebung der Forderung, dass jedes Objekt eine Nachricht empfangen muss. Die folgende Abbildung zeigt daf\"ur ein Beispiel.
        \begin{center}
          \scalebox{1}{
            \begin{tikzpicture}[]
              \node[circleA](A){Objekttyp A};
              \node[redspot](a1) at (1.2, 0.75){a1};
              \node[redspot](a2) at (0.0, 1.6){a2};
              \node[redspot](a3) at (-1.3, 0.5){a3};
              \node[redspot](a4) at (-1, -0.6) {a4};
              \node[redspot](a5) at (1.3, -0.4) {a5};
              \node[redspot](a6) at (0.6, -0.9) {a6};
              \node[redspot](a7) at (-0.2, -0.5) {a7};
              \node[redspot](a8) at (0.5, 1) {a8};
              \node[redspot](a9) at (0, 0.6) {a9};
              \node[redspot](a10) at (-0.7, 1) {a10};
              \node[redspot](a11) at (-0.2, -1.4){a11};


      \path[->] (a1) edge[bend right=45] (a2)
                (a2) edge[bend right=45] (a3)
                (a3) edge[bend right=45] (a1)
                (a4) edge[loop right, out=270, in = 0, looseness=18] (a4)
                (a5) edge (a6)
                (a5) edge (a7)
                (a1) edge (a8)
                (a9) edge (a10);
            \end{tikzpicture}
          }
        \end{center}
        Dieser Rekursiv-Beziehungstyp ist pr\"adestiniert f\"ur die Repr\"asentation von Monohierarchien, wie sie in der Praxis in Form von Organigrammen oder St\"ucklisten auftreten. Als Sonderfall eines \enquote{Baumes} ist eine Liste ($a_{9} \rightarrow a_{10}$) m\"oglich, die evtl. auch nur aus einem Element ($a_{11}$) bestehen kann.

        Bei der Transformation dieses Rekursiv-Beziehungstyps in das physische Datenbankmodell wird der FK als nicht-eingabepflichtig und nicht-unikal deklariert. Die entsprechende Transformation ist in T12 zusammengefasst.
        \subsubsection{Transformationsregel T12 f\"ur den (0,1):(0,*) Rekursiv-Beziehungstyp}
          \tablefirsthead{%
          \hline
            \multicolumn{1}{|c}{\textbf{ER-Modell}} &
            \multicolumn{1}{|l}{\textbf{ }} &
            \multicolumn{1}{|c|}{\textbf{physisches Datenmodell}} \\
          \hline
          }
          \begin{supertabular}[h]{|l|l|p{7.5cm}|}
            \footnotesize Objekttyp A mit Schl\"ussel \pk{SA} & $\Rightarrow $ & \footnotesize Tabelle A mit PK \pk{SA}\\
            \hline
            \footnotesize (0,1):(0,*) Rekursiv-Beziehungstyp & $\Rightarrow $ & \footnotesize \pk{SA} wird in A mit anderen Attributbezeichnungen \enquote{gedoppelt} und als nicht-eingabepflichtiger nicht-unikaler FK \fk{SA} vereinbart, der auf den Sender verweist. (\fk{SA'})\\
          \end{supertabular}
          \begin{center}
            \scalebox{.7}{
              \begin{tikzpicture}[node distance=1.5cm, every edge/.style={link}]
                \node[entity](A){A};
                  \node[attribute](a1)[left = of A]{\key{SA}} edge (A);
                \node[relationship](rel1)[right = of A]{};
                \node[auto,swap](l1) at (5.6,-1.4) {(0,1) empf\"angt Nachricht};
                \node[auto,swap](l2) at (5.6,1.4) {(0,*) sendet Nachricht};
                \path [draw, -] (A) |- ($(A.south) + (0.5,-0.5)$) -| (rel1);
                \path [draw, -] (A) |- ($(A.north) +(0.5, 0.5)$) -| (rel1);
              \end{tikzpicture}
            }
          \end{center}
          \bild{}{transformationsregel_12_tabellen}{0.35}
\clearpage
				\subsubsection{Beispiel - Unternehmenshierarchie}
          Betrachten wir als Beispiel die Leitungshierarchie im Unternehmen. Ein Mitarbeiter kann ohne Leitungsfunktion sein, er kann aber auch mehrere andere Mitarbeiter anleiten. Dies kann \"uber mehrere Hierarchiestufen erfolgen. Andererseits haben die meisten Mitarbeiter genau einen Chef. Mitarbeiter der obersten Leitungsebene haben keinen Chef. Die Transformation ist in der folgenden Abbildung dargestellt.
          \begin{center}
            \scalebox{.7}{
              \begin{tikzpicture}[node distance=1.5cm, every edge/.style={link}]
                \node[entity](A){Mitarbeiter};
                  \node[attribute](a1)[left = of A]{\key{Personalnummer}} edge (A);
                  \node[attribute](a2)[above left = of A]{Name} edge (A);
                  \node[attribute](a2)[below left = of A]{Chef-Personalnummer} edge (A);
                \node[relationship](rel1)[right = of A]{};
                \node[auto,swap](l1) at (5.6,-1.4) {(0,1) wird angeleitet von};
                \node[auto,swap](l2) at (5.6,1.4) {(0,*) leitet an};
                \path [draw, -] (A) |- ($(A.south) + (0.5,-0.5)$) -| (rel1);
                \path [draw, -] (A) |- ($(A.north) +(0.5, 0.5)$) -| (rel1);
              \end{tikzpicture}
            }
          \end{center}
          \begin{small}
            Mitarbeiter(\pk{Personalnummer}, Name, $\Uparrow$Chef-Personalnummer$\Uparrow $)
          \end{small}

          Der FK \fk{Chef-Personalnummer} ist nicht-eingabepflichtig, somit kann es Mitarbeiter ohne Chef geben. Da der FK nicht-unikal ist, k\"onnen mehrere Mitarbeiter auf denselben Chef verweisen. An diesem Beispiel sieht man, dass der (0,1):(0,*) Rekursiv-Beziehungstyp wesentlich mehr Objekt-Strukturen umfasst, als f\"ur die Repr\"asentation monohierarchischer Zusammenh\"ange ben\"otigt werden:

          \begin{itemize}
            \item \textit{Mehr-Objekt-Zyklen:} Der Mitarbeiter mit der Personalnummer \enquote{0815} kann Chef des Mitarbeiters \enquote{0816} sein. Dieser kann den Mitarbeiter \enquote{0817} anleiten, der wiederum Chef des Mitarbeiters \enquote{0815} ist.
            \item \textit{Ein-Objekt-Zyklen:} Ein Mitarbeiter kann als sein eigener Chef ausgewiesen werden, wenn n\"am\-lich in einem Datensatz die Werte von PK und FK identisch sind.
            \item \textit{Mehr-Objekt-Ketten:} Ein Mitarbeiter kann einen einzigen Mitarbeiter anleiten, der wiederum Chef eines einzigen Mitarbeiter ist.
            \item \textit{Ein-Objekt-Ketten:} Ein Mitarbeiter, der keinen Chef hat, leitet selbst auch niemanden an.
          \end{itemize}
          In der Praxis sind solche Situationen gew\"ohnlich verboten, sie k\"onnen aber durch die Tabellentypbeschreibung nicht verhindert werden. Die Anwendungssoftware muss sichern, dass die beschriebenen F\"alle nicht realisiert werden, sofern es nicht gewollt ist.
\clearpage
      \subsection{Der (0,*):(0,*) Rekursiv-Beziehungstyp}
        Der (0,*):(0,*) Rekursiv-Beziehungstyp stellt keinerlei einschr\"ankende Bedingungen an die Struktur, nach der die Objekte eines Objekttyps A miteinander in Beziehung stehen. Jedes Objekt kann keine, eine oder mehrere Nachrichten senden oder empfangen. Die nachstehende Abbildung vermittelt davon einen Eindruck, ohne dass sie alle M\"oglichkeiten ber\"ucksichtigen kann.

        Es lassen sich beliebig strukturierte Netzwerke darstellen. Insbesondere k\"onnen dies Polyhierarchien sein. In einer Polyhierarchie hat jedes Objekt keines, eines oder mehrere untergeordnete Objekte. Jedes Objekt kann keines, eines oder mehrere \"ubergeordnete Objekte besitzen. Als Sonderfall sind nat\"urlich auch Monohierarchien m\"oglich, diese entarten aber mitunter zur Objektkette, die evtl. auch nur aus einem Element bestehen kann. Die Objektkette kann in sich geschlossen sein und wird dann zum Objekt-Zyklus, der im Minimalfall ein Ein-Objekt-Zyklus ist.

        \begin{center}
          \scalebox{1}{
            \begin{tikzpicture}[]
              \node[circleA](A){Objekttyp A};
              \node[redspot](a1) at (1.2, 0.75){a1};
              \node[redspot](a2) at (0.0, 1.6){a2};
              \node[redspot](a3) at (-1.3, 0.5){a3};
              \node[redspot](a4) at (-1, -0.6) {a4};
              \node[redspot](a5) at (1.3, -0.4) {a5};
              \node[redspot](a6) at (0.6, -0.9) {a6};
              \node[redspot](a7) at (-0.2, -0.5) {a7};
              \node[redspot](a8) at (0.5, 1) {a8};
              \node[redspot](a9) at (0, 0.6) {a9};
              \node[redspot](a10) at (-0.7, 1) {a10};
              \node[redspot](a11) at (-0.2, -1.4){a11};
              \node[redspot](a12) at (0.4, -0.2){a12};

      \path[->] (a1) edge[bend right=45] (a2)
                (a2) edge[bend right=45] (a3)
                (a3) edge[bend right=45] (a1)
                (a5) edge[bend right=45] (a1)
                (a4) edge[loop right, out=270, in = 0, looseness=18] (a4)
                (a5) edge (a6)
                (a6) edge (a7)
                (a1) edge (a8)
                (a9) edge (a10)
                (a6) edge (a11);
            \end{tikzpicture}
          }
        \end{center}

        Der (0,*):(0,*) Rekursiv-Beziehungstyp l\"asst sich in das physische Datenbankmodell nur nach vorheriger Umwandlung in einen neuen Objekttyp \"uberf\"uhren. Diese Umwandlung erfolgt mit Hilfe eines Koppel-Objekttyps, der als \enquote{Sendung} bezeichnet werden soll. Er repr\"asentiert die Sender-Empf\"anger-Beziehung zwischen den Objekten des Objekttyps A. Der Objekttyp A ist mit dem Objekttyp \enquote{Sendung} durch zwei (1,1):(0,*) Beziehungen zu verbinden.
        \begin{center}
          \scalebox{.7}{
            \begin{tikzpicture}[node distance=1.5cm, every edge/.style={link}]
              \node[entity](A){A};
                \node[attribute](a1)[left = of A]{\key{SA}} edge (A);
              \node[relationship](rel1)[right = of A]{};
              \node[auto,swap](l1) at (5.6,-1.4) {(0,*) empf\"angt Nachricht};
              \node[auto,swap](l2) at (5.6,1.4) {(0,*) sendet Nachricht};
              \path [draw, -] (A) |- ($(A.south) + (0.5,-0.5)$) -| (rel1);
              \path [draw, -] (A) |- ($(A.north) +(0.5, 0.5)$) -| (rel1);
            \end{tikzpicture}
          }
        \end{center}
        wird umgewandelt in:
        \begin{center}
          \scalebox{.7}{
            \begin{tikzpicture}[node distance=1.5cm, every edge/.style={link}]
              \node[entity](A){A};
                \node[attribute](a1)[left = of A]{\key{SA}} edge (A);
              \node[relationship](rel1) at (4, 0){1};
              \node[entity](B)[right = of rel1]{Sendung};
              \node[relationship](rel2) at (4, -2){2};
              \node[auto,swap](l1) at (1, -1.5){(0,*)};
              \node[auto,swap](l2) at (6, -1.5){(1,1)};
              \path (A) edge node[auto,swap]{(0,*)} (rel1)
                (rel1) edge node[auto,swap]{(1,1)} (B);
              \path[draw, -] (A.south) |- (rel2.west)
                (rel2.east) -| (B.south);
            \end{tikzpicture}
          }
        \end{center}
        Stellt man den Objekttyp \enquote{Sendung} im physischen Datenbankmodell als Koppel-Tabelle A/A dar, so enth\"alt er den PK des Objekttyps A zweimal, als eingabepflichtige FK, die auf den Empf\"anger bzw. auf den Sender einer Nachricht verweisen. Beide FK sind f\"ur sich genommen nicht-unikal, so dass ein Objekt in mehreren Tabellenzeilen als Empf\"anger bzw. als Sender auftreten kann. Die Kombination dieser beiden FK stellt den PK der Koppel-Tabelle dar. Dieses Vorgehen ist in der Transformationsregel T13 dargestellt.
        \subsubsection{Transformationsregel T13 f\"ur den (0,*):(0,*) Rekursiv-Beziehungstyp}
          \tablefirsthead{%
          \hline
            \multicolumn{1}{|c}{\textbf{ER-Modell}} &
            \multicolumn{1}{|l}{\textbf{ }} &
            \multicolumn{1}{|c|}{\textbf{physisches Datenmodell}} \\
          \hline
          }
          \begin{supertabular}[h]{|l|l|p{7.5cm}|}
            \footnotesize Objekttyp A mit Schl\"ussel \pk{SA} & $\Rightarrow $ & \footnotesize Tabelle A mit PK \pk{SA}\\
            \hline
            \footnotesize (0,*):(0,*) Rekursiv-Beziehungstyp & $\Rightarrow $ & \footnotesize Koppel-Tabelle A/A, die \pk{SA} in zwei Exemplaren, als eingabepflichtige, nicht-unikale Fremdschl\"ussel enth\"alt: als Empf\"anger-Fremdschl\"ussel \fk{SA} und als Sender-Fremdschl\"ussel \fk{SA'}. Die Kombination beider Fremdschl\"ussel bildet den Prim\"arschl\"ussel von A/A.\\
          \end{supertabular}

          \bild{}{transformationsregel_13_tabellen}{0.35}

        \subsubsection{Beispiel - komplexes Bauteil}
          Zur Veranschaulichung der Transformationsregel T13 betrachten wir eine St\"uckliste, die den Aufbau eines komplexen Bauteils - z.B. eines Autos - aus kleineren Bauteilen beschreibt. Ein Bauteil kann elementar sein, kann sich also nicht mehr in kleinere Bauteile zerlegen lassen (z.B. Schraube). Andere Bauteile lassen sich, wie z.B. der Motor, in kleinere Einzelteile zerlegen. Des Weiteren ist es m\"oglich, dass ein gegebenes Bauteil nicht Bestandteil eines gr\"o\ss eren Bauteils ist, wenn es n\"amlich beispielsweise das komplette Auto darstellt. Es kann aber auch Bestandteil mehrerer gr\"o\ss erer Bauteile sein, wenn ein bestimmter Motor in mehrere Modelle eingebaut werden kann. Die erforderliche Transformation zeigt die folgende Abbildung.
          \begin{center}
            \scalebox{.7}{
              \begin{tikzpicture}[node distance=1.5cm, every edge/.style={link}]
                \node[entity](A){Bauteil};
                  \node[attribute](a1)[left = of A]{\key{Bauteilnummer}} edge (A);
                  \node[attribute](a1)[above left = of A]{Bezeichnung} edge (A);
                \node[relationship](rel1)[right = of A]{};
                \node[auto,swap](l1) at (5.6,-1.4) {(0,*) ist zerlegbar in};
                \node[auto,swap](l2) at (5.6,1.4) {(0,*) ist Bestandteil von};
                \path [draw, -] (A) |- ($(A.south) + (0.5,-0.5)$) -| (rel1);
                \path [draw, -] (A) |- ($(A.north) +(0.5, 0.5)$) -| (rel1);
              \end{tikzpicture}
            }
          \end{center}
          \begin{small}
            Bauteil(\pk{Bauteilenummer}, Bezeichnung)

            BauteilBauteil(\pk{\fk{Grossteilnummer + Kleinteilnummer}})
          \end{small}

          Jeder der beiden FK ist f\"ur sich nicht-unikal, so dass mehrere Bauteile jeweils als Gro\ss teil bzw. Kleinteil in der St\"uckliste auftreten k\"onnen. Es kann sein, dass eine bestimmte Bauteilnummer nicht als Wert des FK \fk{Grossteilnummer} auftritt. Dann ist dieses Bauteil nicht weiter zerlegbar. Ist die Bauteilnummer nie Wert des FK \fk{Kleinteilnummer}, dann geh\"ort dies zu keinem gr\"o\ss eren Bauteil.

          Die Tabellentypbeschreibungen lassen allerdings wiederum einige Situationen zu, welche man im angegebenen Praxisfall eigentlich ausschlie\ss en m\"ochte. Dazu geh\"oren beispielsweise:
          \begin{itemize}
            \item \textit{Mehr-Objekte-Zyklen:} Das Bauteil \enquote{1001} hat als eines seiner Bestandteile das Bauteil \enquote{1002}, f\"ur das wiederum als eines seiner Bestandteile das Bauteil \enquote{1003} angegeben ist. F\"ur das Bauteil \enquote{1003} ist aber angegeben, dass es das Bauteil \enquote{1001} als Bestandteil enth\"alt. Die Koppel-Tabelle \enquote{BauteilBauteil} enth\"alt dann folgende Zeilen.

            \tablefirsthead{%
            \hline
              \multicolumn{1}{|l}{\textbf{Grossteilnummer}} &
              \multicolumn{1}{|l|}{\textbf{Kleinteilnummer}} \\
            \hline
            }
            \begin{supertabular}[h]{|l|p{5cm}|}
              \hline
              .... & ....\\
              \hline
              1001 & 1002\\
              \hline
              1002 & 1003\\
              \hline
              1003 & 1001\\
              \hline
              ... & ...\\
              \hline
            \end{supertabular}
            \item \textit{Ein-Objekt-Zyklen:} Ein Bauteil kann als sein eigenes Bestandteil ausgewiesen werden, wenn seine Bauteilnummer in einer Zeile der Koppel-Tabelle als Wert beider FK angegeben wird.
            \item \textit{Mehr-Objekte-Ketten:} Ein Bauteil kann als Bestandteil nur ein einziges anderes Bauteil haben. Seine Nummer kommt dann in der Koppel-Tabelle \enquote{BauteilBauteil} nur einmal als Wert des FK \fk{Grossteilnummer} vor. Das ist eine unsinnige Konstruktion, denn ein Bauteil l\"asst sich entweder nicht zerlegen oder es ist mindestens in zwei Bestandteile zerlegbar.
            \item \textit{Ein-Objekt-Ketten:} Die Typbeschreibungen lassen die Speicherung eines Bauteils zu, das weder zerlegbar, noch Bestandteil eines gr\"o\ss eren Bauteils ist. Das w\"are beispielsweise eine Schraube, die in keinem weiteren gr\"o\ss eren Bauteil Verwendung findet. Ihre Bauteilnummer taucht dann in der Koppel-Tabelle \enquote{BauteilBauteil} an keiner Stelle auf, die Speicherung solcher Bauteile ist im betrachteten Zusammenhang jedoch sinnlos.
          \end{itemize}
          Die beschriebenen Situationen kann das Datenbankmanagementsystem nicht verhindern. Sie m\"ussen durch die entsprechende Anwendungsprogrammierung unterbunden werden.
      \subsection{Der (1,*):(0,*) Rekursiv-Beziehungstyp}
        Beim (1,*):(0,*) Rekursiv-Beziehungstyp kann ein Objekt keine, eine oder mehrere Nachrichten senden, es muss aber stets mindestens eine Nachricht empfangen. Die folgende Abbildung vermittelt einen Eindruck von m\"oglichen Beziehungsstrukturen, ohne alle Varianten abzudecken.

        \begin{center}
          \scalebox{1}{
            \begin{tikzpicture}[]
              \node[circleA](A){Objekttyp A};
              \node[redspot](a1) at (1.2, 0.75){a1};
              \node[redspot](a2) at (0.0, 1.6){a2};
              \node[redspot](a3) at (-1.3, 0.5){a3};
              \node[redspot](a5) at (1.3, -0.4) {a5};
              \node[redspot](a6) at (0.6, -0.9) {a6};
              \node[redspot](a7) at (-0.2, -0.5) {a7};
              \node[redspot](a8) at (0.5, 1) {a8};
              \node[redspot](a9) at (0, 0.6) {a9};
              \node[redspot](a10) at (-0.7, 1) {a10};
              \node[redspot](a11) at (-0.3, -1.3) {a11};
              \node[redspot](a12) at (-1.0, -0.3) {a12};

      \path[->] (a1) edge[bend right=45] (a2)
                (a3) edge[bend right=45] (a1)
                (a5) edge (a6)
                (a5) edge[bend right=45] (a1)
                (a6) edge (a7)
                (a1) edge (a8)
                (a1) edge (a5)
                (a8) edge (a9)
                (a10) edge (a3)
                (a9) edge (a10)
                (a6) edge (a11)
                (a12) edge[loop right, out=270, in = 0, looseness=15] (a12);
            \end{tikzpicture}
          }
        \end{center}

        Der (1,*):(0,*) Rekursiv-Beziehungstyp l\"asst sich erst nach Umwandlung, mit Hilfe eines Koppel-Objekttyps \enquote{Sendung}, in das physische Modell \"uberf\"uhren. Der Objekttyp A ist mit dem Objekttyp \enquote{Sendung} durch einen (1,1):(0,*) und (1,1):(1,*) verbunden. Die folgende Abbildung zeigt das Schema der Umwandlung.
        \begin{center}
            \scalebox{.7}{
              \begin{tikzpicture}[node distance=1.5cm, every edge/.style={link}]
                \node[entity](A){A};
                  \node[attribute](a1)[left = of A]{\key{SA}} edge (A);
                \node[relationship](rel1)[right = of A]{};
                \node[auto,swap](l1) at (5.6,-1.4) {(1,*) empf\"angt Nachricht};
                \node[auto,swap](l2) at (5.6,1.4) {(0,*) sendet Nachricht};
                \path [draw, -] (A) |- ($(A.south) + (0.5,-0.5)$) -| (rel1);
                \path [draw, -] (A) |- ($(A.north) +(0.5, 0.5)$) -| (rel1);
              \end{tikzpicture}
            }
          \end{center}
          wird umgewandelt in:
          \begin{center}
            \scalebox{.7}{
              \begin{tikzpicture}[node distance=1.5cm, every edge/.style={link}]
                \node[entity](A){A};
                  \node[attribute](a1)[left = of A]{\key{SA}} edge (A);
                \node[relationship](rel1) at (4, 0){1};
                \node[entity](B)[right = of rel1]{Sendung};
                \node[relationship](rel2) at (4, -2){2};
                \node[auto,swap](l1) at (1, -1.5){(1,*)};
                \node[auto,swap](l2) at (6, -1.5){(1,1)};
                \path (A) edge node[auto,swap]{(0,*)} (rel1)
                  (rel1) edge node[auto,swap]{(1,1)} (B);
                \path[draw, -] (A.south) |- (rel2.west)
                  (rel2.east) -| (B.south);
              \end{tikzpicture}
            }
          \end{center}
\clearpage
        Nun haben wir bei den bin\"aren Beziehungstypen gesehen, dass sich ein (1,1):(1,*) Beziehungstyp im physischen Datenbankmodell nur als (1,1):(0,*) Beziehungstyp repr\"asentieren l\"asst. Damit kann der (1,*):(0,*) Rekursiv-Beziehungstyp nur gem\"a\ss{} der Transformationsregel T13 wie ein (0,*):(0,*) Rekursiv-Beziehungstyp dargestellt werden.
        \subsubsection{Beispiel - Literaturverweis}
          Betrachten wir als Beispiel Literaturverweise in Lehrb\"uchern. Jedes Lehrbuch verweist auf mindestens ein Vorg\"anger-Lehrbuch. Auf ein gerade erst erschienenes Lehrbuch kann noch nicht verwiesen werden. Handelt es sich um ein interessantes Lehrbuch, so wird im Laufe der Zeit von vielen Nachfolge-Lehrb\"uchern zitiert.
          \begin{center}
            \scalebox{.7}{
              \begin{tikzpicture}[node distance=1.5cm, every edge/.style={link}]
                \node[entity](A){Lehrbuch};
                  \node[attribute](a1)[left = of A]{\key{ISBN}} edge (A);
                  \node[attribute](a2)[above left = of A]{Titel} edge (A);
                \node[relationship](rel1)[right = of A]{};
                \node[auto,swap](l1) at (4,-1.4) {(0,*) wird zitiert von};
                \node[auto,swap](l2) at (4,1.4) {(1,*) verweist auf};
                \path [draw, -] (A) |- ($(A.south) + (0.5,-0.5)$) -| (rel1);
                \path [draw, -] (A) |- ($(A.north) +(0.5, 0.5)$) -| (rel1);
              \end{tikzpicture}
            }
          \end{center}
        \begin{small}
          Lehrbuch(\pk{ISBN}, Titel)

          LehrbuchLehrbuch(\pk{\fk{VorgaengerISBN + NachfolgerISBN}})
        \end{small}

          Jeder der beiden FK ist f\"ur sich nicht-unikal, so dass ein gegebenes Lehrbuch mehrmals als Vorg\"anger bzw. als Nachfolger in Erscheinung treten kann. Ist eine Lehrbuch ISBN kein einziges Mal Wert des FK \fk{VorgaengerISBN}, dann wurde dieses Lehrbuch noch nicht zitiert.

          Die Tabellentypbeschreibungen lassen folgende Situationen zu, die nicht der Semantik entsprechen.
          \begin{itemize}
            \item \textit{Vernachl\"assigung der Nichtoptionalit\"at:} Die nicht-optionale Beziehungstyprichtung\\ \enquote{Lehrbuch verweist auf Lehrbuch} wird als optionale Beziehungtyprichtung repr\"asentiert. Es ist somit m\"oglich, dass eine Lehrbuch ISBN nie als Wert des Fremdschl\"ussels \fk{NachfolgerISBN} auftaucht. Dann ist dieses Lehrbuch nicht als Nachfolger eines anderen Lehrbuchs ausgewiesen, d.h. es verweist - entgegen der Praxisregel - auf kein Vorg\"angerlehrbuch.
            \item \textit{Zyklen:} Ein Lehrbuch kann auf sich selbst als Vorg\"anger verweisen oder eine Objektekette kann sich schlie\ss en. Da Vorg\"anger-Verweise immer \enquote{in die Vergangenheit} zeigen, sind Zyklen eigentlich verboten.
          \end{itemize}
          Die beschriebenen \enquote{pathologischen} Situationen lassen sich nur durch die Anwendungsprogrammierung vermeiden.
      \subsection{Der (1,*):(1,*) Rekursiv-Beziehungstyp}
        Der (1,*):(1,*) Rekursiv-Beziehungstyp unterscheidet sich nur in einem Punkt, vom zuvor behandelten Rekursiv-Beziehungstyp (1,*):(0,*). Ein Objekt muss mindestens eine Nach\-richt senden. Diese Forderung hat jedoch entscheidende Konsequenzen f\"ur die m\"og\-lichen Beziehungsstrukturen, von denen in der folgenden Abbildung einige abgebildet sind.

        \begin{center}
          \scalebox{1}{
            \begin{tikzpicture}[]
              \node[circleA](A){Objekttyp A};
              \node[redspot](a1) at (1.2, 0.75){a1};
              \node[redspot](a2) at (0.0, 1.6){a2};
              \node[redspot](a3) at (-1.3, 0.5){a3};
              \node[redspot](a5) at (1.3, -0.4) {a5};
              \node[redspot](a6) at (0.6, -0.9) {a6};
              \node[redspot](a7) at (-0.2, -0.5) {a7};
              \node[redspot](a8) at (0.5, 1) {a8};
              \node[redspot](a9) at (0, 0.6) {a9};
              \node[redspot](a10) at (-0.7, 1) {a10};
              \node[redspot](a11) at (-0.3, -1.3) {a11};
              \node[redspot](a12) at (-1.0, -0.3) {a12};

      \path[->] (a1) edge[bend right=45] (a2)
                (a2) edge[bend right=45] (a3)
                (a3) edge[bend right=45] (a1)
                (a5) edge (a6)
                (a6) edge (a7)
                (a7) edge (a3)
                (a1) edge (a8)
                (a1) edge (a5)
                (a8) edge (a9)
                (a10) edge (a3)
                (a9) edge (a10)
                (a6) edge (a11)
                (a11) edge (a7)
                (a12) edge[loop right, out=270, in = 0, looseness=15] (a12);
            \end{tikzpicture}
          }
        \end{center}

        Auch der (1,*):(1,*) Rekursiv-Beziehungstyp l\"asst sich in das physische Modell, nur nach vorheriger Umwandlung in einen neuen Objekttyp, \"uberf\"uhren. Der Objekttyp A ist mit dem Koppel-Objekttyp \enquote{Sendung} durch zwei (1,1):(1,*) Beziehungstypen verbunden.
        \begin{center}
          \scalebox{.7}{
            \begin{tikzpicture}[node distance=1.5cm, every edge/.style={link}]
              \node[entity](A){A};
                \node[attribute](a1)[left = of A]{\key{SA}} edge (A);
              \node[relationship](rel1)[right = of A]{};
              \node[auto,swap](l1) at (5.6,-1.4) {(1,*) empf\"angt Nachricht};
              \node[auto,swap](l2) at (5.6,1.4) {(1,*) sendet Nachricht};
              \path [draw, -] (A) |- ($(A.south) + (0.5,-0.5)$) -| (rel1);
              \path [draw, -] (A) |- ($(A.north) +(0.5, 0.5)$) -| (rel1);
            \end{tikzpicture}
          }
        \end{center}
        wird umgewandelt in:
        \begin{center}
          \scalebox{.7}{
            \begin{tikzpicture}[node distance=1.5cm, every edge/.style={link}]
              \node[entity](A){A};
                \node[attribute](a1)[left = of A]{\key{SA}} edge (A);
              \node[relationship](rel1) at (4, 0){1};
              \node[entity](B)[right = of rel1]{Sendung};
              \node[relationship](rel2) at (4, -2){2};
              \node[auto,swap](l1) at (1, -1.5){(1,*)};
              \node[auto,swap](l2) at (6, -1.5){(1,1)};
              \path (A) edge node[auto,swap]{(1,*)} (rel1)
                (rel1) edge node[auto,swap]{(1,1)} (B);
              \path[draw, -] (A.south) |- (rel2.west)
                (rel2.east) -| (B.south);
            \end{tikzpicture}
          }
        \end{center}
        Da sich ein (1,1):(1,*) im physischen Modell nur als (1,1):(0,*) Beziehungstyp repr\"asentieren l\"asst, muss der (1,*):(1,*) Rekursiv-Beziehungstyp - unter Verlust von semantischen Informationen - gem\"a\ss{} der Transformationsregel T13 wie ein (0,*):(0,*) Rekursiv-Beziehungstyp dargestellt werden.
\clearpage
        \subsubsection{Beispiel - Pianisten}
          Als Beispiel betrachten wir eine Kunsthochschule, die Informationen \"uber Pianisten sammelt, die in der Ausbildung t\"atig sind. Wir nehmen der Einfachheit halber an, dass die Pianisten eindeutig durch ihren Namen unterschieden werden k\"onnen. Jeder dieser Pianisten hat wenigstens einen anderen Pianisten als Sch\"uler. Andererseits ist jeder Pianist Sch\"uler eines oder mehrerer anderer Pianisten (Autodidakten werden nicht ber\"ucksichtigt).
          \begin{center}
            \scalebox{.7}{
              \begin{tikzpicture}[node distance=1.5cm, every edge/.style={link}]
                \node[entity](A){Pianist};
                  \node[attribute](a1)[left = of A]{\key{Name}} edge (A);
                  \node[attribute](a2)[above left = of A]{Geburtsdatum} edge (A);
                \node[relationship](rel1)[right = of A]{};
                \node[auto,swap](l1) at (4,-1.4) {(1,*) hat Sch\"uler};
                \node[auto,swap](l2) at (4,1.4) {(1,*) ist Sch\"uler von};
                \path [draw, -] (A) |- ($(A.south) + (0.5,-0.5)$) -| (rel1);
                \path [draw, -] (A) |- ($(A.north) +(0.5, 0.5)$) -| (rel1);
              \end{tikzpicture}
            }
          \end{center}
          \begin{small}
            Pianist(\pk{Name}, Geburtsdatum)

            PianistPianist(\pk{\fk{LehrerName + SchuelerName}})
          \end{small}

          Jeder der beiden FK ist f\"ur sich nicht-unikal, so dass ein Pianist mehrmals als Lehrer bzw. Sch\"uler in Erscheinung treten kann.

          Die Tabellentypbeschreibungen erm\"oglichen Beziehungsstrukturen, die in der Praxis unzul\"assig sind, so beispielsweise:
          \begin{itemize}
            \item \textit{Vernachl\"assigung der Nichtoptionalit\"at:} Die beiden nicht-optionalen Beziehungstyprichtungen \enquote{Pianist hat als Sch\"uler Pianist} und \enquote{Pianist ist Sch\"uler von Pianist} werden als optionale Beziehungstyp-Richtungen repr\"asentiert. Es ist somit m\"oglich, dass ein Pianist keine Sch\"uler hat und dass ein Pianist nicht Sch\"uler eines anderen Pianisten ist.
            \item \textit{Zyklen:} Ein Pianist kann sein eigener Sch\"uler sein (Ein-Objekt-Zyklus) oder eine Sch\"u\-ler-Kette kann zu ihrem Ausgangspunkt zur\"uckkehren (Mehr-Objekte-Zyklus).
          \end{itemize}

          Diese unzul\"assigen Strukturen sind durch die Anwendungsprogrammierung zu verhindern.
    \section{Transformation von Kardinalit\"ats-Beschr\"ankungen}
      Bei den Beziehungstypen gibt es im ER-Modell die M\"oglichkeit einschr\"ankende Bedingungen f\"ur die Kardinalit\"at anzugeben, wenn diese durch die \enquote{Gesch\"aftsregeln} der Realit\"at vorgegeben sind. Nehmen wir z.B. an, dass ein Unternehmen Informationen \"uber die Dienstwagen und \"uber die (Sommer- und Winter-) R\"ader speichern will. Jedes Auto ist mit genau 5 R\"adern ausger\"ustet (inklusive Ersatzrad). Einige R\"ader werden als Reserve im Lager aufbewahrt. Dies entspricht einem (0,1):(1,*) Beziehungstyp, f\"ur den (1,*) nicht beliebige Werte annehmen kann, sondern nur den Wert 5.

      Wie wird diese Beschr\"ankung in der Tabellentypbeschreibung ber\"ucksichtigt? Leider gar nicht! Es gibt im physischen Modell keine M\"oglichkeit die Kardinalit\"at einzugrenzen. Das gilt f\"ur den bin\"aren wie auch f\"ur den rekursiven oder tern\"aren Beziehungstyp. F\"ur die Repr\"asentation von Kardinalit\"atsbeschr\"ankungen gilt also die negative Transformationsregel T14, welche im Folgenden dargestellt ist.
      \begin{center}
        \scalebox{.7}{
          \begin{tikzpicture}[node distance=1.5cm, every edge/.style={link}]
            \node[entity](A){Auto};
              \node[attribute](a1)[left = of A]{\key{PolKennzeichen}} edge (A);
              \node[attribute](a2)[above left = of A]{Marke} edge (A);
            \node[relationship](rel1)[right = of A]{zugeordnet} edge node[auto,swap] {(5,5)} (A);
            \node[entity](B)[right = of rel1]{Rad} edge node[auto,swap] {(0,1)} (rel1);
              \node[attribute](b1)[right = of B]{\key{Inventarnummer}} edge (B);
              \node[attribute](b2)[below right = of B]{Typ} edge (B);
          \end{tikzpicture}
        }
      \end{center}
        \subsubsection{Transformationsregel T14 Kardinalit\"ats-Beschr\"ankung}
          \tablefirsthead{%
          \hline
            \multicolumn{1}{|c}{\textbf{ER-Modell}} &
            \multicolumn{1}{|l}{\textbf{ }} &
            \multicolumn{1}{|c|}{\textbf{physisches Datenmodell}} \\
          \hline
          }
          \begin{supertabular}[h]{|p{7cm}|l|p{7cm}|}
            \footnotesize Kardinalit\"ats-Beschr\"ankung eines Be\-zieh\-ungs\-typs & $\Rightarrow $ & \footnotesize Eine Beschr\"ankung l\"asst sich nicht in der Typ\-be\-schrei\-bung repr\"asentieren.\\
            \hline
          \end{supertabular}

          Auch wenn sich die Kardinalit\"ats-Beschr\"ankungen nicht in das physische Modell transformieren lassen, sollte ihre Notierung im ER-Modell dennoch erfolgen, weil sie wichtige Gesch\"aftsregeln darstellen. Diese m\"ussen dann bei der Anwendungsprogrammierung ber\"uck\-sichtigt werden.
    \section{Transformation der Spezialisierung / Generalisierung}
      Die Transformation der Spezialisierung kann nicht in eigenst\"andige Regeln gefasst werden, da der Einzelfall zu betrachten ist. Es sollen hier nur allgemeine Hinweise f\"ur die \"Uberf\"uhrung in das physische Modell gegeben werden, welche endg\"ultig in Kombination mit den Transformationsregeln der bin\"aren Beziehungstypen erfolgt.
      \begin{center}
        \scalebox{.8}{
          \begin{tikzpicture}
            \node[entity](obj1){Person};
            \node[isa](isa) [below = of obj1]{ISA} edge node[auto] {(0,1) (1,1) disjunkt}(obj1);
            \node[entity](obj2)[below left = of isa]{Student};
            \node[entity](obj3)[below right = of isa]{WiMa};
            \path[draw, -] (isa.west) -| (obj2.north);
            \path[draw, -] (isa.east) -| (obj3.north);
          \end{tikzpicture}
        }
      \end{center}
      \begin{center}
        \scalebox{.8}{
          \begin{tikzpicture}[node distance=1.5cm, every edge/.style={link}]
            \node[entity](A){Person};
            \node[relationship](rel1)[below left = of A]{};
            \node[relationship](rel2)[below right = of A]{};
            \node[entity](B)[below = of rel1]{Student} edge node[auto,swap]{(1,1)} (rel1);
            \node[entity](C)[below = of rel2]{WiMa} edge node[auto,swap]{(1,1)} (rel2);
            \node[auto,swap](l1) at (3.1,-1.0) {(0,1)};
            \node[auto,swap](l2) at (-1.9,-1.0) {(0,1)};
            \path [draw, -] (A.west) -| (rel1.north);
            \path [draw, -] (A.east) -| (rel2.north);
          \end{tikzpicture}
        }
      \end{center}
      Die Transformation erfolgt mit T03 bzw. T04 und zus\"atzlichen programmtechnischen Ma\ss nahmen, um die Disjunktion zu gew\"ahrleisten.


	\part{SQL}
  \input{../sql/01_einstieg_in_sql}
    \chapter{Selektieren und Sortieren}
    \setcounter{page}{1}\kapitelnummer{chapter}
    \minitoc
\newpage
    \section{Selektieren von Zeilen: Die WHERE-Klausel}
      Im vorangegangenen Kapitel wurde gezeigt, wie mit den beiden SQL-Klauseln \languageorasql{SELECT} und \languageorasql{FROM} der gesamte Inhalt einer Tabelle angezeigt werden kann. Zus\"atzlich zu diesen beiden Klauseln wird nun die optionale \languageorasql{WHERE}-Klausel eingef\"uhrt, die eine Selektion der Datens\"atze erm\"oglicht. Diese kann einen beliebig komplexen Ausdruck enthalten, der dann das \enquote{Auswahlkriterium} f\"ur die Datens\"atze darstellt. Die Syntax der \languageorasql{WHERE}-Klausel lautet wie folgt:
      \begin{lstlisting}[language=oracle_sql,caption={Die WHERE-Klausel},label=sql02_01]
WHERE <Ausdruck1> <Relationaler Operator> <Ausdruck2>
      \end{lstlisting}
      \begin{merke}
        Der Begriff \enquote{Ausdruck} steht in der Programmierung f\"ur ein auf einen Kontext bezogenes, auswertbares Gebilde. Bei \textit{Ausdruck1} und \textit{Ausdruck2} kann es sich beispielsweise um Spaltenbezeichner, Funktionsaufrufe, arithmetische Berechnungen, Konstanten usw. handeln.
      \end{merke}
      \beispiel{sql02_01} zeigt insgesamt drei Ausdr\"ucke:
      \begin{itemize}
          \item \textit{<Ausdruck1>}
          \item \textit{<Ausdruck2>}
          \item \textit{<Ausdruck1>} <Relationaler Operator> \textit{<Ausdruck2>}
      \end{itemize}
      Nicht nur \textit{Ausdruck1} und \textit{Ausdruck2} sind Ausdr\"ucke, sondern auch die Verbindung beider, mittels eines Operators, wird als Ausdruck betrachtet.
      \begin{merke}
        Ein Operator ist ein mit einer Semantik belegtes Zeichen, dass eine genau definierte Operation darstellt. Operatoren werden meist in Gruppen eingeteilt, z. B. arithmetische Operatoren (+, - , *, /), relationale Operatoren, logische Operatoren, usw.
      \end{merke}
      \tabelle{relopersql} listet die in Oracle und MS SQL Server vorhandenen relationalen Operatoren auf.
\clearpage
      \subsection{Relationale Operatoren}
        \begin{center}
          \tablecaption{Relationale Operatoren in Oracle und MS SQL Server}
          \label{relopersql}
          \begin{small}
            \tablefirsthead{
              \multicolumn{1}{c}{\textbf{(Operator)}} &
              \multicolumn{1}{c}{\textbf{(Bedeutung)}} \\
              \hline
            }
            \tablehead {
            \multicolumn{1}{c}{\textbf{(Operator)}} &
            \multicolumn{1}{c}{\textbf{(Bedeutung)}} \\
            \hline
            }
            \tabletail{
              \hline
            }
            \tablelasttail {
              \hline
            }
            \begin{supertabular}{lp{10cm}}
              =               & Gleichheit \\
              !=              & Ungleichheit \\
              \textless       & Kleiner als \\
              \textless=      & Kleiner oder gleich \\
              \textgreater    & Gr\"o\ss er als \\
              \textgreater=   & Gr\"o\ss er oder gleich \\
              LIKE            & \"Ahnlichkeit zweier Zeichenketten \\
              IN              & Der linke Ausdruck befindet sich in einer Liste von Werten, die der rechte Ausdruck erzeugt.\\
              IS NULL         & Der linke Ausdruck liefert den Wert NULL zur\"uck. \\
              BETWEEN A AND B & Der Wert des linken Ausdrucks liegt zwischen den Wertgrenzen A und B. Die Wertgrenzen A und B werden in das Intervall mit einbezogen.\\
            \end{supertabular}
          \end{small}
        \end{center}
        \subsubsection{Numerische Werte vergleichen}
          Der Vergleich von numerischen Werten gestaltet sich sowohl in Oracle als auch im MS SQL Server sehr einfach.
          \begin{lstlisting}[language=oracle_sql,caption={Gleichheit zweier numerischer Werte},label=sql02_02]
SELECT Vorname, Nachname
FROM   Mitarbeiter
WHERE  Mitarbeiter_ID = 5;
          \end{lstlisting}
          \begin{center}
            \begin{small}
              \changefont{pcr}{m}{n}
              \tablefirsthead{
                \multicolumn{1}{l}{\textbf{VORNAME}} &
                \multicolumn{1}{l}{\textbf{NACHNAME}} \\
                \cmidrule(l){1-1}\cmidrule(l){2-2}
              }
              \tablehead{}
              \tabletail{
                \multicolumn{2}{l}{\textbf{1 Zeile ausgew\"ahlt}} \\
              }
              \tablelasttail{
                \multicolumn{2}{l}{\textbf{1 Zeile ausgew\"ahlt}} \\
              }
              \begin{msoraclesql}
                \begin{supertabular}{ll}
                  Tim & Sindermann \\
                \end{supertabular}
              \end{msoraclesql}
            \end{small}
          \end{center}
          \begin{lstlisting}[language=oracle_sql,caption={Wert A gr\"o\ss er oder gleich Wert B},label=sql02_03]
SELECT Vorname, Nachname, Gehalt
FROM   Mitarbeiter
WHERE  Mitarbeiter_ID >= 50;
          \end{lstlisting}
\clearpage
          \begin{center}
            \begin{small}
              \changefont{pcr}{m}{n}
              \tablefirsthead{
                \multicolumn{1}{r}{\textbf{VORNAME}} &
                \multicolumn{1}{l}{\textbf{NACHNAME}} &
                \multicolumn{1}{l}{\textbf{GEHALT}} \\
                \cmidrule(l){1-1}\cmidrule(l){2-2}\cmidrule(r){3-3}
              }
              \tablehead{}
              \tabletail{
                \multicolumn{3}{l}{\textbf{51 Zeilen gew\"ahlt}} \\
              }
              \tablelasttail{
                \multicolumn{3}{l}{\textbf{51 Zeilen gew\"ahlt}} \\
              }
              \begin{msoraclesql}
                \begin{supertabular}{llr}
                Emilia &  K\"ohler & 2500 \\
                Karolin & Klingner & 2000 \\
                Chris & Roggatz & 3000 \\
                Christian & Haas & 2000 \\
                Jessica & Winkler & 2000 \\
                Anna & Keller & 2500 \\
                Johannes & Klingner & 2500 \\
                Emma & Kr\"uger & 3500 \\
                \end{supertabular}
              \end{msoraclesql}
            \end{small}
          \end{center}
          \begin{lstlisting}[language=oracle_sql,caption={Pr\"ufen eines Intevalls},label=sql02_04]
SELECT Mitarbeiter_ID, Vorname, Nachname
FROM   Mitarbeiter
WHERE  Mitarbeiter_ID BETWEEN 5 AND 9;
          \end{lstlisting}
          \begin{center}
            \begin{small}
              \changefont{pcr}{m}{n}
              \tablefirsthead{
                \multicolumn{1}{r}{\textbf{MITARBEITER\_ID}} &
                \multicolumn{1}{l}{\textbf{VORNAME}} &
                \multicolumn{1}{l}{\textbf{NACHNNAME}} \\
                \cmidrule(r){1-1}\cmidrule(l){2-2}\cmidrule(l){3-3}
              }
              \tablehead{}
              \tabletail{
                \multicolumn{3}{l}{\textbf{5 Zeilen gew\"ahlt}} \\
              }
              \tablelasttail{
                \multicolumn{3}{l}{\textbf{5 Zeilen gew\"ahlt}} \\
              }
              \begin{msoraclesql}
                \begin{supertabular}{rll}
                5 & Tim & Sindermann \\
                6 & Peter & M\"uller \\
                7 & Emily & Meier \\
                8 & Dirk & Peters \\
                9 & Louis & Winter \\
                \end{supertabular}
              \end{msoraclesql}
            \end{small}
          \end{center}
          \begin{lstlisting}[language=oracle_sql,caption={Alle Zeilen aus einer Wertemenge anzeigen},label=sql02_05]
SELECT Mitarbeiter_ID, Vorname, Nachname
FROM   Mitarbeiter
WHERE  Mitarbeiter_ID IN (5, 7, 9);
          \end{lstlisting}
          \begin{center}
            \begin{small}
              \changefont{pcr}{m}{n}
              \tablefirsthead{
              \multicolumn{1}{r}{\textbf{MITARBEITER\_ID}} &
              \multicolumn{1}{l}{\textbf{VORNAME}} &
              \multicolumn{1}{l}{\textbf{NACHNNAME}} \\
              \cmidrule(r){1-1}\cmidrule(l){2-2}\cmidrule(l){3-3}
              }
              \tablehead{}
              \tabletail{
                \multicolumn{3}{l}{\textbf{3 Zeilen gew\"ahlt}} \\
              }
              \tablelasttail{
                \multicolumn{3}{l}{\textbf{3 Zeilen gew\"ahlt}} \\
              }
              \begin{msoraclesql}
                \begin{supertabular}{rll}
                5 & Tim & Sindermann \\
                7 & Emily & Meier \\
                9 & Louis & Winter \\
                \end{supertabular}
              \end{msoraclesql}
            \end{small}
          \end{center}
\clearpage
        \subsubsection{Zeichenketten vergleichen}
          \label{stringdiff}
          Der Vergleich zweier Zeichenketten bringt, im Gegensatz zum Vergleich numerischer Werte, eine Schwierigkeit mit sich. Abh\"angig vom benutzten RDBMS\footnote{RDBMS = Relationales Datenbank Management System} werden Zeichenkettenvergleiche casesensitiv oder incasesensitiv durchgef\"uhrt. In Oracle beispielsweise ist \enquote{Oracle} ungleich \enquote{oracle} oder \enquote{ORACLE} ungleich \enquote{OrAcLe}. Der MS SQL Server hingegen verh\"alt sich nicht casesensitiv. F\"ur ihn sind alle vier Werte gleich.

          \begin{lstlisting}[language=oracle_sql,caption={Ein einfacher Zeichenkettenvergleich},label=sql02_06]
SELECT Mitarbeiter_ID, Vorname, Nachname
FROM   Mitarbeiter
WHERE  Nachname = 'Scholz';
          \end{lstlisting}
          \begin{center}
            \begin{small}
              \changefont{pcr}{m}{n}
              \tablefirsthead{
                \multicolumn{1}{r}{\textbf{MITARBEITER\_ID}} &
                \multicolumn{1}{l}{\textbf{VORNAME}} &
                \multicolumn{1}{l}{\textbf{NACHNAME}} \\
                \cmidrule(r){1-1}\cmidrule(l){2-2}\cmidrule(l){3-3}
              }
              \tablehead{}
              \tabletail{
              }
              \tablelasttail{
                \multicolumn{3}{l}{\textbf{1 Zeile ausgew\"ahlt}} \\
              }
              \begin{msoraclesql}
                \begin{supertabular}{rll}
                  96 & Johanna & Scholz \\
                \end{supertabular}
              \end{msoraclesql}
            \end{small}
          \end{center}
          Im n\"achsten Beispiel wird eine \"ahnliche \languageorasql{WHERE}-Klausel verwendet, wie in \beispiel{sql02_06}, sie f\"uhrt jedoch zu einem ganz anderen Ergebnis.
          \begin{merke}
            In SQL m\"ussen Zeichenketten in Hochkommas ' eingeschlossen werden! Diese d\"urfen nicht mit den Akzent-Zeichen verwechselt werden!
          \end{merke}
          \begin{lstlisting}[language=oracle_sql,caption={Ein einfacher Zeichenkettenvergleich},label=sql02_07]
SELECT Mitarbeiter_ID, Vorname, Nachname
FROM   Mitarbeiter
WHERE  Nachname = 'SCHOLZ';
          \end{lstlisting}
          \begin{center}
            \begin{small}
              \changefont{pcr}{m}{n}
              \begin{oraclesql}
Keine Zeilen ausgew\"ahlt!
              \end{oraclesql}
            \end{small}
          \end{center}
          Da die Oracle-Datenbank casesensitiv arbeitet, ist \enquote{SCHOLZ} ungleich \enquote{Scholz}. Somit werden keine Datens\"atze gefunden. Der MS SQL Server hat hier keine Schwierigkeiten. Ihn st\"ort die unterschiedliche Schreibweise der Zeichenketten nicht, weshalb er das gew\"unschte Ergebnis anzeigt.
\clearpage
          \begin{center}
            \begin{small}
              \changefont{pcr}{m}{n}
              \tablefirsthead{
                \multicolumn{1}{r}{\textbf{MITARBEITER\_ID}} &
                \multicolumn{1}{l}{\textbf{VORNAME}} &
                \multicolumn{1}{l}{\textbf{NACHNAME}} \\
                \cmidrule(r){1-1}\cmidrule(l){2-2}\cmidrule(l){3-3}
              }
              \tablehead{}
              \tabletail{
                \multicolumn{3}{l}{\textbf{1 Zeile ausgew\"ahlt}} \\
              }
              \tablelasttail{
                \multicolumn{3}{l}{\textbf{1 Zeile ausgew\"ahlt}} \\
              }
              \begin{mssql}
                \begin{supertabular}{rll}
                  96 & Johanna & Scholz \\
                \end{supertabular}
              \end{mssql}
            \end{small}
          \end{center}
        \subsubsection{Zeichenketten vergleichen mit LIKE}
          Ist es notwendig nach einem Zeichenmuster zu suchen, wie z. B. \textit{Alle Mitarbeiter, deren Nachname mit \enquote{Sch} beginnt}, so kann dies mit dem \languageorasql{LIKE}-Operator geschehen.
          \begin{lstlisting}[language=oracle_sql,caption={Zeichenkettensuche mit einem Suchmuster},label=sql02_08]
SELECT Mitarbeiter_ID, Vorname, Nachname
FROM   Mitarbeiter
WHERE  Nachname LIKE 'Sch%';
          \end{lstlisting}
          \begin{center}
            \begin{small}
              \changefont{pcr}{m}{n}
              \tablefirsthead{
                \multicolumn{1}{r}{\textbf{MITARBEITER\_ID}} &
                \multicolumn{1}{l}{\textbf{VORNAME}} &
                \multicolumn{1}{l}{\textbf{NACHNAME}} \\
                \cmidrule(r){1-1}\cmidrule(l){2-2}\cmidrule(l){3-3}
              }
              \tablehead{}
              \tabletail{
              }
              \tablelasttail{
                \multicolumn{1}{l}{\textbf{10 Zeilen ausgew\"ahlt}} \\
              }
              \begin{msoraclesql}
                \begin{supertabular}{rll}
                  4 & Sebastian & Schwarz \\
                  11 & Sophie & Schwarz \\
                  25 & Elias & Schreiber \\
                  29 & Louis & Schmitz \\
                  33 & Martin & Schacke \\
                  36 & Hans & Schumacher \\
                \end{supertabular}
              \end{msoraclesql}
            \end{small}
          \end{center}
          Der \languageorasql{LIKE}-Operator nutzt zwei Wildcards, um Suchmuster f\"ur Zeichenketten zu erstellen.
          \begin{center}
            \tablecaption{Wildcards des LIKE-Operators}
            \label{likewildcards}
            \begin{small}
              \tablefirsthead{
                \multicolumn{1}{c}{\textbf{(Wildcard)}} &
                \multicolumn{1}{c}{\textbf{(Bedeutung)}} \\
                \hline
              }
              \tabletail{
                \hline
              }
              \tablelasttail{
                \hline
              }
              \begin{supertabular}{lp{10cm}}
                \% & (Prozentzeichen) Null, eines oder beliebig viele Zeichen \\
                \_ & (Unterstrich) Genau ein Zeichen \\
              \end{supertabular}
            \end{small}
          \end{center}
          F\"ur \beispiel{sql02_08} bedeutet dies: \textit{Die ersten drei Zeichen des Suchmusters sind S, c und h. Nach dem h k\"onnen null, eines oder beliebig viele andere Zeichen stehen.} Im n\"achsten Beispiel wird die \textit{\_}-Wildcard benutzt, um alle Mitarbeiter zu suchen, deren Nachname an der dritten Stelle ein kleines g tr\"agt.
          \begin{lstlisting}[language=oracle_sql,caption={Zeichenkettensuche mit einem etwas komplexeren Suchmuster},label=sql02_09]
SELECT Mitarbeiter_ID, Vorname, Nachname
FROM   Mitarbeiter
WHERE  Nachname LIKE '__g%';
          \end{lstlisting}
\clearpage
          \begin{center}
            \begin{small}
              \changefont{pcr}{m}{n}
              \tablefirsthead{
                \multicolumn{1}{r}{\textbf{MITARBEITER\_ID}} &
                \multicolumn{1}{l}{\textbf{VORNAME}} &
                \multicolumn{1}{l}{\textbf{NACHNAME}} \\
                \cmidrule(r){1-1}\cmidrule(l){2-2}\cmidrule(l){3-3}
              }
              \tablehead{}
              \tabletail{
                \multicolumn{1}{l}{\textbf{4 Zeilen ausgew\"ahlt}} \\
              }
              \tablelasttail{
                \multicolumn{1}{l}{\textbf{4 Zeilen ausgew\"ahlt}} \\
              }
              \begin{msoraclesql}
                \begin{supertabular}{rll}
                37 & Louis & Wagner \\
                52 & Chris & Roggatz \\
                83 & Peter & Roggatz \\
                88 & Joachim & Wagner \\
                \end{supertabular}
              \end{msoraclesql}
            \end{small}
          \end{center}
          Die ersten beiden Zeichen des Suchmusters sind \textit{\_} Unterstriche, d. h. an der ersten und zweiten Stelle der gesuchten Zeichenketten \textbf{muss} sich jeweils genau ein Zeichen befinden. Das dritte Zeichen ist mit dem \textit{g} genau definiert. Anschlie\ss end k\"onnen wieder null, eines oder beliebig viele andere Zeichen stehen.
          \begin{merke}
            Der LIKE-Operator verwendet die beiden Wildcards \% und \_ . \% Steht f\"ur null, eines oder beliebig viele Zeichen. \_ steht f\"ur genau ein Zeichen.
          \end{merke}
        \subsubsection{Vergleiche mit NULL-Werten}
          Sowohl Oracle, als auch der MS SQL Server kennen beide den Operator \languageorasql{IS NULL}. Mit seiner Hilfe k\"onnen Spalten auf NULL-Werte hin \"uberpr\"uft werden. Sollen z. B. alle Mitarbeiter, die keinen Vorgesetzten haben, angezeigt werden, wird ein Vergleich mit dem \languageorasql{IS NULL}-Operator angestellt.
          \begin{lstlisting}[language=oracle_sql,caption={Der IS NULL Operator},label=sql02_10]
SELECT Mitarbeiter_ID, Vorname, Nachname, Vorgesetzter_ID
FROM   Mitarbeiter
WHERE  Vorgesetzter_ID IS NULL;
          \end{lstlisting}
          \begin{center}
            \begin{small}
              \changefont{pcr}{m}{n}
              \tablefirsthead{
                \multicolumn{1}{r}{\textbf{MITARBEITER\_ID}} &
                \multicolumn{1}{l}{\textbf{VORNAME}} &
                \multicolumn{1}{l}{\textbf{NACHNAME}} &
                \multicolumn{1}{r}{\textbf{VORGESETZTER\_ID}} \\
                \cmidrule(r){1-1}\cmidrule(l){2-2}\cmidrule(l){3-3}\cmidrule(r){4-4}
              }
              \tablehead{}
              \tabletail{
                \multicolumn{4}{l}{\textbf{1 Zeile ausgew\"ahlt}} \\
              }
              \tablelasttail{
                \multicolumn{4}{l}{\textbf{1 Zeile ausgew\"ahlt}} \\
              }
              \begin{msoraclesql}
                \begin{supertabular}{rllr}
                1 & Max & Winter & \\
                \end{supertabular}
              \end{msoraclesql}
            \end{small}
          \end{center}
          Da es in diesem Beispiel keinen wesentlichen Unterschied bei der Ergebnisanzeige zwischen Oracle und SQL Server gibt (SQL Server zeigt das Wort NULL f\"ur NULL-Werte und alle Spaltenwerte linksb\"undig an), wurde hier auf ein getrenntes Abdrucken der Ergebnisse verzichtet.

          Das Gegenst\"uck zum \languageorasql{IS NULL}-Operator, ist der \languageorasql{IS NOT NULL}-Operator.
\clearpage
          \begin{merke}
            Wird ein Ausdruck, mit Hilfe des Gleichheitsoperators (=), mit dem Wert NULL verglichen, ist das Ergebnis des Vergleichs immer NULL!
          \end{merke}
      \subsection{Logische Verkn\"upfung von Ausdr\"ucken}
        In vielen F\"allen ist es notwendig komplexe Ausdr\"ucke zu formulieren, indem mehrere Ausdr\"ucke miteinander verkn\"upft werden. Eine solche Verkn\"upfung geschieht unter Zuhilfenahme der logischen Operatoren \textit{AND}, \textit{OR} und \textit{NOT}.
        \subsubsection{Logische Verkn\"upfungen mit AND}
          Der logische Operator \textit{AND} verkn\"upft zwei Bedingungen miteinander und liefert ein wahres Ergebnis, sobald beide Ausdr\"ucke ein wahres Ergebnis haben. Die Logiktabelle \tabelle{logikand} zeigt die m\"oglichen Ergebnisse einer AND-Verkn\"upfung.
					\vspace{\baselineskip}
          \begin{center}
            \tablecaption{Der logische Operator AND}
            \label{logikand}
            \tablefirsthead{
              \multicolumn{1}{l}{\textbf{Aussagen}} &
              \multicolumn{1}{c}{\textbf{(Wahr)}} &
              \multicolumn{1}{c}{\textbf{(Falsch)}} \\
              \hline
            }
            \tablehead{}
            \tabletail{
              \hline
            }
            \tablelasttail{
              \hline
            }
            \begin{supertabular}{|l|c|c|}
              Wahr & w & f \\
              \hline
              Falsch & f & f \\
            \end{supertabular}
          \end{center}
          In \beispiel{sql02_11} wird gezeigt, wie der \textit{AND}-Operator dazu genutzt werden kann, um zwei Bedingungen miteinander zu verkn\"upfen. Es sollen alle Mitarbeiter angezeigt werden, deren Gehalt unter 1.500 EUR liegt und die in der Bankfiliale Nummer zwei arbeiten.
          \begin{lstlisting}[language=oracle_sql,caption={Der AND Operator},label=sql02_11]
SELECT Vorname, Nachname, Gehalt, Bankfiliale_ID
FROM   Mitarbeiter
WHERE  Gehalt < 1500
  AND  Bankfiliale_ID = 2;
          \end{lstlisting}
          \begin{center}
            \begin{small}
              \changefont{pcr}{m}{n}
              \tablefirsthead{
                \multicolumn{1}{l}{\textbf{VORNAME}} &
                \multicolumn{1}{l}{\textbf{NACHNAME}} &
                \multicolumn{1}{r}{\textbf{GEHALT}} &
                \multicolumn{1}{r}{\textbf{BANKFILIALE\_ID}}\\
                \cmidrule(l){1-1}\cmidrule(l){2-2}\cmidrule(r){3-3}\cmidrule(r){4-4}
              }
              \tablehead{}
              \tabletail{
                \multicolumn{1}{l}{\textbf{2 Zeilen ausgew\"ahlt}} \\
              }
              \tablelasttail{
                \multicolumn{1}{l}{\textbf{2 Zeilen ausgew\"ahlt}} \\
              }
              \begin{msoraclesql}
                \begin{supertabular}{llrr}
                  Martin & Schacke & 1000 & 2 \\
                  Oliver & Wolf & 1000 & 2 \\
                \end{supertabular}
              \end{msoraclesql}
            \end{small}
          \end{center}
        \subsubsection{Logische Verkn\"upfungen mit OR}
          Der logische Operator \textit{OR} liefert, im Unterschied zu \textit{AND}, ein wahres Ergebnis, sobald mindestens einer der beiden Ausdr\"ucke ein wahres Ergebnis hat.
\clearpage
          \begin{center}
            \tablecaption{Der logische Operator OR}
            \label{logikor}
            \tablefirsthead{
              \multicolumn{1}{l}{\textbf{Aussagen}} &
              \multicolumn{1}{c}{\textbf{(Wahr)}} &
              \multicolumn{1}{c}{\textbf{(Falsch)}} \\
              \hline
            }
            \tablehead{}
            \tabletail{
              \hline
            }
            \tablelasttail{
              \hline
            }
            \begin{supertabular}{|l|c|c|}
              Wahr & w & w \\
              \hline
              Falsch & w & f \\
            \end{supertabular}
          \end{center}
          Wird in \beispiel{sql02_11} der Operator \textit{AND} durch ein \textit{OR} ersetzt, ver\"andert sich die Ergebnismenge. Es werden jetzt alle Mitarbeiter angezeigt, die entweder ein Gehalt unter 1.500 EUR haben oder die in Bankfilialie Nummer zwei arbeiten.
          \begin{lstlisting}[language=oracle_sql,caption={Der OR Operator},label=sql02_12]
SELECT Vorname, Nachname, Gehalt, Bankfiliale_ID
FROM   Mitarbeiter
WHERE  Gehalt < 1500
   OR  Bankfiliale_ID = 2;
          \end{lstlisting}
          \begin{center}
            \begin{small}
              \changefont{pcr}{m}{n}
              \tablefirsthead{
                \multicolumn{1}{l}{\textbf{VORNAME}} &
                \multicolumn{1}{l}{\textbf{NACHNAME}} &
                \multicolumn{1}{r}{\textbf{GEHALT}} &
                \multicolumn{1}{r}{\textbf{BANKFILIALE\_ID}}\\
                \cmidrule(l){1-1}\cmidrule(l){2-2}\cmidrule(l){3-3}\cmidrule(l){4-4}
              }
              \tablehead{}
              \tabletail{
                \multicolumn{1}{l}{\textbf{9 Zeilen ausgew\"ahlt}} \\
              }
              \tablelasttail{
                \multicolumn{1}{l}{\textbf{9 Zeilen ausgew\"ahlt}} \\
              }
              \begin{msoraclesql}
                \begin{supertabular}{llrr}
                  Louis & Winter & 12000 & 2 \\
                  Stefan & Beck & 1500 & 2 \\
                  Martin & Schacke & 1000 & 2 \\
                  Max & Oswald & 1500 & 2 \\
                  Oliver & Wolf & 1000 & 2 \\
                  Hans & Schumacher & 1000 & 3 \\
                  Maja & Keller & 1000 & 5 \\
                  Elias & Sindermann & 1000 & 8 \\
                  Jonas & Meier & 1000 & 12 \\
                \end{supertabular}
              \end{msoraclesql}
            \end{small}
          \end{center}
        \subsubsection{Aussagen mit NOT umkehren}
          Die Bedeutung des Operators \textit{NOT} ist sehr einfach zu umschreiben. Er kehrt ein Ergebnis um. Aus einem wahren Ergebnis wird ein falsches und umgekehrt. Dieser Effekt ist auch mit \languageorasql{IS NULL} und \languageorasql{IS NOT NULL} zu sehen. In \beispiel{sql02_13} werden alle Mitarbeiter angezeigt, deren Gehalt kleiner als 1.500 EUR ist und die nicht in der Bankfiliale Nummer zwei arbeiten.
          \begin{lstlisting}[language=oracle_sql,caption={Der NOT Operator},label=sql02_13]
SELECT Vorname, Nachname, Gehalt, Bankfiliale_ID
FROM   Mitarbeiter
WHERE  Gehalt < 1500
  AND  NOT Bankfiliale_ID = 2;
          \end{lstlisting}
\clearpage
          \begin{center}
            \begin{small}
              \changefont{pcr}{m}{n}
              \tablefirsthead{
                \multicolumn{1}{l}{\textbf{VORNAME}} &
                \multicolumn{1}{l}{\textbf{NACHNAME}} &
                \multicolumn{1}{r}{\textbf{GEHALT}} &
                \multicolumn{1}{r}{\textbf{BANKFILIALE\_ID}}\\
                \cmidrule(l){1-1}\cmidrule(l){2-2}\cmidrule(r){3-3}\cmidrule(r){4-4}
              }
              \tablehead{}
              \tabletail{
                \multicolumn{1}{l}{\textbf{4 Zeilen ausgew\"ahlt}} \\
              }
              \tablelasttail{
                \multicolumn{1}{l}{\textbf{4 Zeilen ausgew\"ahlt}} \\
              }
              \begin{msoraclesql}
                \begin{supertabular}{llrr}
                Hans & Schumacher & 1000 & 3 \\
                Maja & Keller & 1000 & 5 \\
                Elias & Sindermann & 1000 & 8 \\
                Jonas & Meier & 1000 & 12 \\
                \end{supertabular}
              \end{msoraclesql}
            \end{small}
          \end{center}
          \begin{merke}
            Die Klammern ( und ) haben Einfluss auf die Bedeutung von Ausdr\"ucken. Werden mehrere logische Operatoren kombiniert, kann so die Lesbarkeit von Ausdr\"ucken verbessert oder deren Bedeutung ver\"andert werden.
          \end{merke}
    \section{Festlegen einer Sortierung}
      In allen vorangegangenen Beispielen war die Reihenfolge der Ausgabe der Datens\"atze unbestimmt. Sowohl Oracle als auch Microsoft SQL Server geben die Datens\"atze immer in der Reihenfolge aus, in der sie in der Quelltabelle vorliegen. Soll eine sortierte Ausgabe erfolgen, muss dies mit Hilfe der in \beispiel{sql01_01} gezeigten \languageorasql{ORDER BY}-Klausel geschehen. Dazu muss diese, mit Sortierbegriffen versehen, am Ende des SQL-Statements angegeben werden.
      \subsection{Die ORDER BY Klausel}
        \begin{lstlisting}[language=oracle_sql,caption={Die ORDER BY Klausel},label=sql02_14]
ORDER BY <Sortierbegriff 1> [asc|desc],
         <Sortierbegriff 2> [asc|desc],
         <Sortierbegriff n> [asc|desc] ...
        \end{lstlisting}
        Als Sortierbegriffe k\"onnen Spaltenbezeichner, Spaltenaliasnamen, berechnete Ausdr\"ucke und auch Spaltenpositionsangaben, bezogen auf die Reihenfolge der Spaltennamen in der SELECT-Liste, genutzt werden. \beispiel{sql02_15} und \beispiel{sql02_16} zeigen die Anwendung der \languageorasql{ORDER BY}-Klausel.
        \begin{merke}
          Werden mehrere Sortierbegriffe angegeben, wird die Sortierung von links nach rechts durchgef\"uhrt. Das bedeutet, dass zuerst nach dem \"au\ss erst linken Sortierbegriff sortiert wird und anschlie\ss end wird, innerhalb dieser Sortierung, jeder weitere Sortierbegriff angewandt. Die Sortierungen werden also ineinander geschachtelt.
        \end{merke}
        \begin{lstlisting}[language=oracle_sql,caption={Die ORDER BY Klausel mit Spaltenbezeichnern},label=sql02_15]
SELECT   Vorname, Nachname, Gehalt, Bankfiliale_ID
FROM     Mitarbeiter
WHERE    Gehalt <= 1500
ORDER BY Gehalt;
        \end{lstlisting}
        \begin{center}
          \begin{small}
            \changefont{pcr}{m}{n}
            \tablefirsthead{
              \multicolumn{1}{l}{\textbf{VORNAME}} &
              \multicolumn{1}{l}{\textbf{NACHNAME}} &
              \multicolumn{1}{r}{\textbf{GEHALT}} &
              \multicolumn{1}{r}{\textbf{BANKFILIALE\_ID}}\\
              \cmidrule(l){1-1}\cmidrule(l){2-2}\cmidrule(l){3-3}\cmidrule(l){4-4}
            }
            \tablehead{}
            \tabletail{
              \multicolumn{1}{l}{\textbf{18 Zeilen ausgew\"ahlt}} \\
            }
            \tablelasttail{
              \multicolumn{1}{l}{\textbf{18 Zeilen ausgew\"ahlt}} \\
            }
            \begin{msoraclesql}
              \begin{supertabular}{llrr}
                Oliver & Wolf & 1000 & 2 \\
                Hans & Schumacher & 1000 & 3 \\
                Maja & Wolf & 1000 & 5 \\
                Elias & Sindermann & 1000 & 8 \\
                Jonas & Meier & 1000 & 12 \\
                Martin & Schacke & 1000 & 2 \\
                Max & Oswald & 1500 & 2 \\
                Stefan & Beck & 1500 & 2\\
              \end{supertabular}
            \end{msoraclesql}
          \end{small}
        \end{center}
        \begin{lstlisting}[language=oracle_sql,caption={Die ORDER BY Klausel mit Positionsangaben},label=sql02_16]
SELECT   Vorname, Nachname, Gehalt, Bankfiliale_ID
FROM     Mitarbeiter
WHERE    Gehalt <= 1500
ORDER BY 3, 2;
        \end{lstlisting}
        \begin{center}
          \begin{small}
            \changefont{pcr}{m}{n}
            \tablefirsthead{
              \multicolumn{1}{l}{\textbf{VORNAME}} &
              \multicolumn{1}{l}{\textbf{NACHNAME}} &
              \multicolumn{1}{r}{\textbf{GEHALT}} &
              \multicolumn{1}{r}{\textbf{BANKFILIALE\_ID}}\\
              \cmidrule(l){1-1}\cmidrule(l){2-2}\cmidrule(l){3-3}\cmidrule(l){4-4}
            }
            \tablehead{}
            \tabletail{
%               \multicolumn{1}{l}{\textbf{18 Zeilen ausgew\"ahlt}} \\
            }
            \tablelasttail{
              \multicolumn{1}{l}{\textbf{18 Zeilen ausgew\"ahlt}} \\
            }
            \begin{msoraclesql}
              \begin{supertabular}{llrr}
                Maja & Keller & 1000 & 5 \\
                Jonas & Meier & 1000 & 12 \\
                Martin & Schacke & 1000 & 2 \\
                Hans & Schumacher & 1000 & 3 \\
                Elias & Sindermann & 1000 & 8 \\
                Oliver & Wolf & 1000 & 2 \\
                Stefan & Beck & 1500 & 2 \\
                Georg & D\"uhning & 1500 & 20 \\
                \end{supertabular}
            \end{msoraclesql}
          \end{small}
        \end{center}
        \begin{merke}
          Bei der Benutzung von Positionsangaben (siehe \beispiel{sql02_16}) muss darauf geachtet werden, dass sich diese auf die Reihenfolge der Spaltenbezeichner in der SELECT-Liste beziehen. Wird die SELECT-Liste sp\"ater ver\"andert, m\"ussen unter Umst\"anden auch die Positionsangaben angepasst werden.
        \end{merke}
      \subsection{Auf- und absteigendes Sortieren}
      Zu jedem Suchbegriff k\"onnen die beiden K\"urzel \languageorasql{ASC} und \languageorasql{DESC} mit angegeben werden. \languageorasql{ASC}\footnote{engl. Ascending = aufsteigend} bewirkt aufsteigende Sortierung (Standard) und \languageorasql{DESC}\footnote{engl. Descending = absteigend} absteigende Sortierung. \beispiel{sql02_17} zeigt, wie sich das Ergebnis durch die absteigende Sortierung der Spalte \identifier{gehalt} ver\"andert.
        \begin{lstlisting}[language=oracle_sql,caption={Die ORDER BY Klausel mit absteigender Sortierung},label=sql02_17]
SELECT   Vorname, Nachname, Gehalt, Bankfiliale_ID
FROM     Mitarbeiter
WHERE    Gehalt <= 1500
ORDER BY Gehalt DESC, 2 ASC;
        \end{lstlisting}
        \begin{center}
          \begin{small}
            \changefont{pcr}{m}{n}
            \tablefirsthead{
              \multicolumn{1}{l}{\textbf{VORNAME}} &
              \multicolumn{1}{l}{\textbf{NACHNAME}} &
              \multicolumn{1}{r}{\textbf{GEHALT}} &
              \multicolumn{1}{r}{\textbf{BANKFILIALE\_ID}}\\
              \cmidrule(l){1-1}\cmidrule(l){2-2}\cmidrule(l){3-3}\cmidrule(l){4-4}
            }
            \tablehead{}
            \tabletail{
              \multicolumn{1}{l}{\textbf{18 Zeilen ausgew\"ahlt}} \\
            }
            \tablelasttail{
              \multicolumn{1}{l}{\textbf{18 Zeilen ausgew\"ahlt}} \\
            }
            \begin{msoraclesql}
              \begin{supertabular}{llrr}
                Stefen & Beck & 1500 & 2 \\
                Georg & D\"uhning & 1500 & 20 \\
                Tom & Fischer & 1500 & 17 \\
                Jannis & Friedrich & 1500 & 14 \\
                Maximilian & Hahn & 1500 & 13 \\
                Lena & Hermann & 1500 & 4 \\
                Anne & Huber & 1500 & 10 \\
                \end{supertabular}
            \end{msoraclesql}
          \end{small}
        \end{center}
        \begin{merke}
          Eine in der \languageorasql{ORDER BY}-Klausel als Sortierbegriff genutzte Spalte, muss nicht in der SELECT-Liste der Abfrage vorhanden sein.
        \end{merke}

    \input{../sql/uebungen/sql_02_selektieren_und_sortieren_uebungen}
    \input{../sql/loesungen/sql_02_selektieren_und_sortieren_loesungen}
  \input{../sql/03_funktionen}
    \input{../sql/uebungen/sql_03_funktionen_uebungen}
    \input{../sql/loesungen/sql_03_funktionen_loesungen}
    \chapter{Erweiterte Datenselektion}
    \setcounter{page}{1}\kapitelnummer{chapter}
    \minitoc
\newpage
      Werden zwei Relationen R1 und R2 in einer Abfrage miteinander verkn\"upft, entsteht ein kartesisches Kreuzprodukt. Die Anzahl der Zeilen in diesem Produkt entspricht $R_1 * R_2$. Es bildet die Grundlage f\"ur eine Join-Operation, bei der aus einem Kreuzprodukt, mit Hilfe eines Selektionsausdruckes, gezielt die nicht ben\"otigten Zeilen eliminiert werden.
    \section{Der Inner Join}
      Beim Inner Join werden, im Ergebnis der Abfrage, nur die Zeilen angezeigt, die der Join-Bedingung gen\"ugen.
      \subsection{Die ON-Klausel}
        Die \languageorasql{ON}-Klausel stellt die flexibelste und am H\"aufigsten genutzte M\"oglichkeit dar, um zwei Tabellen, in einer Join-Operation, miteinander zu verkn\"upfen. Daf\"ur werden zwei Spaltenbezeichner und ein Operator ben\"otigt. \beispiel{sql04_01} zeigt einen Inner Join zwischen den beiden Tabellen \identifier{Kunde} und \identifier{Eigenkunde}.
        \begin{lstlisting}[language=oracle_sql,caption={Ein Join zwischen Kunde und Eigenkunde},label=sql04_01]
SELECT Vorname, Nachname, PLZ, Ort
FROM   Kunde INNER JOIN Eigenkunde
         ON (Kunde.Kunden_ID = Eigenkunde.Kunden_ID);
        \end{lstlisting}
        \begin{center}
          \begin{small}
            \changefont{pcr}{m}{n}
            \tablefirsthead {
              \multicolumn{1}{l}{\textbf{VORNAME}} &
              \multicolumn{1}{l}{\textbf{NACHNAME}} &
              \multicolumn{1}{l}{\textbf{PLZ}} &
              \multicolumn{1}{l}{\textbf{ORT}} \\
              \cmidrule(l){1-1}\cmidrule(l){2-2}\cmidrule(l){3-3}\cmidrule(l){4-4}
            }
            \tablehead{}
            \tabletail {
              \multicolumn{4}{l}{\textbf{400 Zeilen ausgew\"ahlt}} \\
            }
            \tablelasttail {
              \multicolumn{4}{l}{\textbf{400 Zeilen ausgew\"ahlt}} \\
            }
            \begin{msoraclesql}
              \begin{supertabular}{llll}
                Sophie & Junge & 39435 & B\"ordeaue \\
                Hanna & Beck & 39439 & G\"usten \\
                Noah & Bunzel & 39435 & Egeln \\
                Sebastian & Peters & 39240 & Sta\ss{}furt \\
                Leni & Braun & 06425 & Alsleben \\
                Jannis & Schreiber & 06406 & Bernburg \\
                Noah & Rollert & 39435 & Wolmirsleben \\
                Amelie & Becker & 06425 & Pl\"otzkau \\
                Christian & Keller & 06449 & Giersleben \\
              \end{supertabular}
            \end{msoraclesql}
          \end{small}
        \end{center}
        Im Vergleich zu allen Beispielen, die in den vorangegangenen Kapiteln zu sehen waren, \"andert sich in \beispiel{sql04_01} nur die \languageorasql{FROM}-Klausel. Hier werden zwei Tabellen, \identifier{Kunde} und \identifier{Eigenkunde}, getrennt durch die beiden Schl\"usselworte \languageorasql{INNER JOIN} angegeben. Diese Syntax stammt aus dem SQL-99-Standard und ist selbsterkl\"arend.
\clearpage
        In der \languageorasql{ON}-Klausel werden die beiden Spalten angegeben, mit deren Hilfe die Verkn\"upfung zwischen den Tabellen hergestellt wird. Wichtig f\"ur diese beiden Spalten ist, dass sie beide miteinander vergleichbare Werte enthalten. Eine Namensgleichheit beider Spalten ist jedoch nicht notwendig.
        \begin{merke}
          Da beide Spalten den Bezeichner \identifier{Kunden\_ID} haben, ist es notwendig die Spaltenbezeichner voll zu qualifizieren. Ein voll qualifizierter Spaltenbezeichner wird immer in der Form \identifier{Tabellenbezeichner}.\identifier{Spaltenbezeichner} angegeben.
        \end{merke}
        Es wird empfohlen Spaltenbezeichner immer zu qualifizieren, da dies der Datenbank das Auffinden der Spalten erleichtert und somit die Perfomance des SQL-Statements steigt. Ohne die Qualifizierung der Spaltenbezeichner in der \languageorasql{ON}-Klausel antworten sowohl Oracle, als auch der MS SQL Server mit einer Fehlermeldung.
        \begin{lstlisting}[language=oracle_sql,caption={Eine fehlerhafte ON-Klausel in Oracle},label=sql04_02]
SELECT Vorname, Nachname, PLZ, Ort
FROM   Kunde INNER JOIN Eigenkunde
         ON (Kunden_ID = Kunden_ID);

Fehler bei Befehlszeile:3 Spalte:24
Fehlerbericht:
SQL-Fehler: ORA-00918: column ambiguously defined
00918. 00000 -  "column ambiguously defined"
*Cause:
*Action:
        \end{lstlisting}
        \begin{lstlisting}[language=ms_sql,caption={Eine fehlerhafte ON-Klausel in MS SQL Server},label=sql04_03]
SELECT Vorname, Nachname, PLZ, Ort
FROM   Kunde INNER JOIN Eigenkunde
         ON (Kunden_ID = Kunden_ID);

Meldung 209, Ebene 16, Status 1, Zeile 3
Mehrdeutiger Spaltenname 'Kunden_ID'.
Meldung 209, Ebene 16, Status 1, Zeile 3
Mehrdeutiger Spaltenname 'Kunden_ID'.
        \end{lstlisting}
      \subsection{Tabellenaliasnamen}
        Genau wie bei Spaltenbezeichnern existiert auch f\"ur Tabellenbezeichner die M\"oglichkeit, Aliasnamen festzulegen. Der Vorteil solcher Tabellenaliasnamen liegt darin, dass die L\"ange eines SQL-Statements, durch die Vergabe von sehr kurzen Aliasnamen, stark reduziert werden kann. \beispiel{sql04_04} produziert das gleiche Ergebnis, wie \beispiel{sql04_01}, nutzt jedoch Aliasnamen f\"ur die beiden Tabellen.

        \begin{lstlisting}[language=oracle_sql,caption={Die Benutzung von Tabellenaliasnamen},label=sql04_04]
SELECT Vorname, Nachname, PLZ, Ort
FROM   Kunde k INNER JOIN Eigenkunde ek
         ON (k.Kunden_ID = ek.Kunden_ID);
        \end{lstlisting}
        \begin{center}
          \begin{small}
            \changefont{pcr}{m}{n}
            \tablefirsthead {
              \multicolumn{1}{l}{\textbf{VORNAME}} &
              \multicolumn{1}{l}{\textbf{NACHNAME}} &
              \multicolumn{1}{l}{\textbf{PLZ}} &
              \multicolumn{1}{l}{\textbf{ORT}} \\
              \cmidrule(l){1-1}\cmidrule(l){2-2}\cmidrule(l){3-3}\cmidrule(l){4-4}
            }
            \tablehead{}
            \tabletail {
              \multicolumn{4}{l}{\textbf{400 Zeilen ausgew\"ahlt}} \\
            }
            \tablelasttail {
              \multicolumn{4}{l}{\textbf{400 Zeilen ausgew\"ahlt}} \\
            }
            \begin{msoraclesql}
              \begin{supertabular}{llll}
                Sophie & Junge & 39435 & B\"ordeaue \\
                Hanna & Beck & 39439 & G\"usten \\
                Noah & Bunzel & 39435 & Egeln \\
                Sebastian & Peters & 39240 & Sta\ss{}furt \\
              \end{supertabular}
            \end{msoraclesql}
          \end{small}
        \end{center}
        \begin{merke}
          Tabellenaliasnamen gelten nur innerhalb eines Statements und beeinflussen die Struktur der Datenbank nicht. Wird ein Tabellenaliasname vergeben, so muss er im gesamten SQL-Statement genutzt werden!
        \end{merke}
        Die bereits bekannten Klauseln \languageorasql{WHERE} und \languageorasql{ORDER BY} k\"onnen auch in einer Join-Abfrage genutzt werden. In \beispiel{sql04_05} wird das Ergebnis auf die Kunden mit Wohnort \enquote{Egeln} reduziert und eine aufsteigende Sortierung nach dem Feld \identifier{Nachname} eingerichtet.
        \begin{lstlisting}[language=oracle_sql,caption={Join mit einschr\"ankender WHERE-Klausel und Sortierung},label=sql04_05]
SELECT   Vorname, Nachname, PLZ, Ort
FROM     Kunde k INNER JOIN Eigenkunde ek
           ON (k.Kunden_ID = ek.Kunden_ID)
WHERE    Ort LIKE 'Egeln'
ORDER BY Nachname;
        \end{lstlisting}
        \begin{center}
          \begin{small}
            \changefont{pcr}{m}{n}
            \tablefirsthead {
              \multicolumn{1}{l}{\textbf{VORNAME}} &
              \multicolumn{1}{l}{\textbf{NACHNAME}} &
              \multicolumn{1}{l}{\textbf{PLZ}} &
              \multicolumn{1}{l}{\textbf{ORT}} \\
              \cmidrule(l){1-1}\cmidrule(l){2-2}\cmidrule(l){3-3}\cmidrule(l){4-4}
            }
            \tablehead{}
            \tabletail {
              \multicolumn{4}{l}{\textbf{18 Zeilen ausgew\"ahlt}} \\
            }
            \tablelasttail {
              \multicolumn{4}{l}{\textbf{18 Zeilen ausgew\"ahlt}} \\
            }
            \begin{msoraclesql}
              \begin{supertabular}{llll}
                Alina & Braun & 39435 & Egeln \\
                Noah & Bunzel & 39435 & Egeln \\
                Hanna & Bunzel & 39435 & Egeln \\
                Paul & Koch & 39435 & Egeln \\
                \end{supertabular}
            \end{msoraclesql}
          \end{small}
        \end{center}
      \subsection{Die USING-Klausel (Nur Oracle)}
        Die \languageorasql{USING}-Klausel stellt eine weitere M\"oglichkeit dar, eine Join-Operation durchzuf\"uhren. Sie ist eine Kurzschreibweise f\"ur \languageorasql{ON R1.Spalte = R2.Spalte}.
\clearpage
        \begin{lstlisting}[language=oracle_sql,caption={Die USING-Klausel},label=sql04_06]
SELECT   Vorname, Nachname, PLZ, Ort
FROM     Kunde k INNER JOIN Eigenkunde ek
           USING(Kunden_ID)
WHERE    Ort LIKE 'Egeln'
ORDER BY Nachname;
        \end{lstlisting}
        \begin{center}
          \begin{small}
            \changefont{pcr}{m}{n}
            \tablefirsthead {
              \multicolumn{1}{l}{\textbf{VORNAME}} &
              \multicolumn{1}{l}{\textbf{NACHNAME}} &
              \multicolumn{1}{l}{\textbf{PLZ}} &
              \multicolumn{1}{l}{\textbf{ORT}} \\
              \cmidrule(l){1-1}\cmidrule(l){2-2}\cmidrule(l){3-3}\cmidrule(l){4-4}
            }
            \tablehead{}
            \tabletail {
              \multicolumn{4}{l}{\textbf{18 Zeilen ausgew\"ahlt}} \\
            }
            \tablelasttail {
              \multicolumn{4}{l}{\textbf{18 Zeilen ausgew\"ahlt}} \\
            }
            \begin{oraclesql}
              \begin{supertabular}{llll}
                Alina & Braun & 39435 & Egeln \\
                Noah & Bunzel & 39435 & Egeln \\
                Hanna & Bunzel & 39435 & Egeln \\
                Paul & Koch & 39435 & Egeln \\
                Karolin & Lange & 39435 & Egeln \\
                Marie & Lehmann & 39435 & Egeln \\
              \end{supertabular}
            \end{oraclesql}
          \end{small}
        \end{center}
        Die Nutzung der \languageorasql{USING}-Klausel unterliegt auch einigen Einschr\"ankungen.
        \begin{itemize}
          \item Die in der \languageorasql{USING}-Klausel genutzte Spalte darf nicht qualifiziert werden.
          \item Die in der \languageorasql{USING}-Klausel genutzte Spalte muss in den beiden, an der Join-Operation teilnehmenden Tabellen den gleichen Namen tragen.
        \end{itemize}
        \begin{lstlisting}[language=oracle_sql,caption={Fehlerhafte Nutzung der USING-Klausel in Oracle},label=sql04_07]
SELECT   Vorname, Nachname, PLZ, Ort
FROM     Kunde k INNER JOIN Eigenkunde ek USING(Kunden_ID)
WHERE    k.Kunden_ID = 200
ORDER BY Nachname;

Fehler bei Befehlszeile:4 Spalte:10
Fehlerbericht:
SQL-Fehler: ORA-00904: "D"."KUNDEN_ID": invalid identifier
00904. 00000 -  "%s: invalid identifier"
*Cause:
*Action:
        \end{lstlisting}
        \subsection{Der Natural-Join (Nur Oracle)}
          Die Natural-Join-Syntax stellt die dritte Variante zur Realisierung von Inner Joins dar.
          \begin{lstlisting}[language=oracle_sql,caption={Die Natural-Join-Syntax},label=sql04_08]
SELECT   Vorname, Nachname, PLZ, Ort
FROM     Kunde k NATURAL JOIN Eigenkunde ek
WHERE    Ort LIKE 'Egeln'
ORDER BY Nachname;
          \end{lstlisting}
          \begin{center}
            \begin{small}
              \changefont{pcr}{m}{n}
              \tablefirsthead {
                \multicolumn{1}{l}{\textbf{VORNAME}} &
                \multicolumn{1}{l}{\textbf{NACHNAME}} &
                \multicolumn{1}{l}{\textbf{PLZ}} &
                \multicolumn{1}{l}{\textbf{ORT}} \\
                \cmidrule(l){1-1}\cmidrule(l){2-2}\cmidrule(l){3-3}\cmidrule(l){4-4}
              }
              \tablehead{}
              \tabletail {
                \multicolumn{4}{l}{\textbf{18 Zeilen ausgew\"ahlt}} \\
              }
              \tablelasttail {
              \multicolumn{4}{l}{\textbf{18 Zeilen ausgew\"ahlt}} \\
              }
              \begin{oraclesql}
                \begin{supertabular}{llll}
                  Alina & Braun & 39435 & Egeln \\
                  Noah & Bunzel & 39435 & Egeln \\
                  Hanna & Bunzel & 39435 & Egeln \\
                \end{supertabular}
              \end{oraclesql}
            \end{small}
          \end{center}
          \beispiel{sql04_08} zeigt, das bei dieser Syntax sowohl die \languageorasql{ON}-Klausel, als auch die \languageorasql{USING}-Klausel \"uberfl\"ussig sind. Dies r\"uhrt daher, dass Oracle automatisch die Spalten in den beiden Tabellen sucht, die den gleichen Namen und den gleichen Datentyp aufweisen. Es werden dabei so vielen Spalten einbezogen wie m\"oglich.
          \begin{merke}
            Da Oracle immer alle Spalten mit gleichem Namen und gleichem Datentyp in den Natural-Join einbezieht, sollte diese Syntax mit bedacht genutzt werden!
          \end{merke}
      \subsection{Die Theta-Style Syntax}
        \begin{merke}
          Sowohl in Oracle, als auch in MS SQL Server kann die Theta-Style-Syntax nur noch zur Realisierung von Inner Joins genutzt werden. Bis auf wenige Ausnahmen ist daher die ANSI-Style-Syntax, mit dem Schl\"usselwort \languageorasql{INNER JOIN}, vorzuziehen!
        \end{merke}
        Die Theta-Style-Syntax stellt die Urvariante der Join-Syntax dar, die auch schon vor dem SQL-99-Standard existierte. Bei dieser Form der Syntax wird in der \languageorasql{FROM}-Klausel nur eine kommaseparierte Liste von Tabellen angegeben, w\"ahrend die Verkn\"upfungsbedingung in der \languageorasql{WHERE}-Klausel formuliert wird.
        \begin{lstlisting}[language=oracle_sql,caption={Ein Inner Join mit Theta-Style-Syntax},label=sql04_09]
SELECT   Vorname, Nachname, PLZ, Ort
FROM     Kunde k, Eigenkunde ek
WHERE    k.Kunden_ID = ek.Kunden_ID
  AND    Ort LIKE 'Egeln'
ORDER BY Nachname;
        \end{lstlisting}
        \begin{center}
          \begin{small}
            \changefont{pcr}{m}{n}
            \tablefirsthead {
              \multicolumn{1}{l}{\textbf{VORNAME}} &
              \multicolumn{1}{l}{\textbf{NACHNAME}} &
              \multicolumn{1}{l}{\textbf{PLZ}} &
              \multicolumn{1}{l}{\textbf{ORT}} \\
              \cmidrule(l){1-1}\cmidrule(l){2-2}\cmidrule(l){3-3}\cmidrule(l){4-4}
            }
            \tablehead{}
            \tabletail {
              \multicolumn{4}{l}{\textbf{18 Zeilen ausgew\"ahlt}} \\
            }
            \tablelasttail {}
            \begin{msoraclesql}
              \begin{supertabular}{llll}
                Alina & Braun & 39435 & Egeln \\
                Noah & Bunzel & 39435 & Egeln \\
                Hanna & Bunzel & 39435 & Egeln \\
              \end{supertabular}
            \end{msoraclesql}
          \end{small}
        \end{center}
      \subsection{Mehr als zwei Tabellen verkn\"upfen}
        Bei komplexeren Abfragen ist es oft notwendig, auf die Daten von mehr als nur zwei Tabellen zu\"urckzugreifen. Dies kann mit allen bisher gezeigten Syntax-Varianten geschehen.
        \begin{merke}
          Da der MS SQL Server sowohl die \languageorasql{USING}-Klausel, als auch die \languageorasql{NATURAL JOIN}-Klausel nicht kennt, kann dieser nur die ANSI-Style-Syntax und den Theta-Style nutzen!
        \end{merke}
        \begin{lstlisting}[language=oracle_sql,caption={Vier Tabellen, verbunden durch Inner Joins},label=sql04_10]
SELECT   Vorname, Nachname, IBAN
FROM     Kunde k INNER JOIN Eigenkunde ek
           ON (k.Kunden_ID = ek.Kunden_ID)
         INNER JOIN EigenkundeKonto ekk
           ON (ek.Kunden_ID = ekk.Kunden_ID)
         INNER JOIN Konto ko
           ON (ekk.Konto_ID = ko.Konto_ID)
WHERE    Ort LIKE 'Egeln'
ORDER BY Nachname, IBAN;
        \end{lstlisting}
        \begin{center}
          \begin{small}
            \changefont{pcr}{m}{n}
            \tablefirsthead {
              \multicolumn{1}{l}{\textbf{VORNAME}} &
              \multicolumn{1}{l}{\textbf{NACHNAME}} &
              \multicolumn{1}{l}{\textbf{IBAN}} \\
              \cmidrule(l){1-1}\cmidrule(l){2-2}\cmidrule(l){3-3}
            }
            \tablehead{}
            \tabletail {
              \multicolumn{3}{l}{\textbf{46 Zeilen ausgew\"ahlt}} \\
            }
            \tablelasttail {
            \multicolumn{3}{l}{\textbf{46 Zeilen ausgew\"ahlt}} \\
            }
            \begin{msoraclesql}
              \begin{supertabular}{lll}
                Alina & Braun & DE2327682878309669110 \\
                Alina & Braun & DE23582034208834002588 \\
                Hanna & Bunzel & DE23343859500956216053 \\
                Noah & Bunzel & DE23419162344850780394 \\
                Noah & Bunzel & DE23506210719641227144 \\
              \end{supertabular}
            \end{msoraclesql}
          \end{small}
        \end{center}
        \begin{merke}
          F\"ur die Ausf\"uhrung des SQL-Statements ist die Reihenfolge, in der die Tabellen miteinander verbunden werden, nicht wichtig.
        \end{merke}
        Das gleiche Ergebnis l\"asst sich auch mit der Theta-Style-Syntax erzielen, wie \beispiel{sql04_11} zeigt.
        \begin{lstlisting}[language=oracle_sql,caption={Ein komplexer Join in der Theta-Style-Syntax},label=sql04_11]
SELECT   Vorname, Nachname, IBAN
FROM     Kunde k, Eigenkunde ek, EigenkundeKonto ekk, Konto ko
WHERE    k.Kunden_ID = ek.Kunden_ID
  AND    ek.Kunden_ID = ekk.Kunden_ID
  AND    ekk.Konto_ID = ko.Konto_ID
  AND    Ort LIKE 'Egeln'
ORDER BY Nachname, IBAN;
        \end{lstlisting}
        \begin{center}
          \begin{small}
            \changefont{pcr}{m}{n}
            \tablefirsthead {
              \multicolumn{1}{l}{\textbf{VORNAME}} &
              \multicolumn{1}{l}{\textbf{NACHNAME}} &
              \multicolumn{1}{l}{\textbf{IBAN}} \\
              \cmidrule(l){1-1}\cmidrule(l){2-2}\cmidrule(l){3-3}
            }
            \tablehead{}
            \tabletail {
              \multicolumn{3}{l}{\textbf{46 Zeilen ausgew\"ahlt}} \\
            }
            \tablelasttail {
              \multicolumn{3}{l}{\textbf{46 Zeilen ausgew\"ahlt}} \\
            }
            \begin{msoraclesql}
              \begin{supertabular}{lll}
                Alina & Braun & DE2327682878309669110 \\
                Alina & Braun & DE23582034208834002588 \\
                Hanna & Bunzel & DE23343859500956216053 \\
                Noah & Bunzel & DE23419162344850780394 \\
                Noah & Bunzel & DE23506210719641227144 \\
                Hanna & Bunzel & DE23916870475976982996 \\
                Paul & Koch & DE23337659559291799957 \\
                Paul & Koch & DE23747825550493162192 \\
                Karolin & Lange & DE2338135354878273969 \\
                Karolin & Lange & DE23657965268917709598 \\
                Marie & Lehmann & DE23311656553298147754 \\
              \end{supertabular}
            \end{msoraclesql}
          \end{small}
        \end{center}
    \section{Outer Joins}
      W\"ahrend bei den Inner Joins nur solche Zeilen im Ergebnis angezeigt werden, die der Join-Bedingung gen\"ugen, ist dieses Verhalten bei den Outer-Joins anders, da auch Datens\"atze sichtbar werden, die der Join-Bedingung nicht entsprechen.
      \begin{merke}
        Wie bereits erw\"ahnt, k\"onnen Outer-Joins nicht mehr, mit Hilfe der Theta-Style-Syntax, dargestellt werden!
      \end{merke}
      \subsection{Left- und Right-Outer-Join}
        Bei Left- bzw. Right-Outer-Joins wird eine der beiden teilnehmenden Tabellen vollst\"andig angzeigt. Die Schl\"usselworte \languageorasql{LEFT} und \languageorasql{RIGHT} geben dabei an, welche der beiden Seiten komplett angezeigt werden soll.
        \subsubsection{Der Left-Outer-Join}
          Beim Left-Outer-Join wird die Tabelle, die auf der linken Seite der Join-Klausel steht vollst\"andig angezeigt. Von der Tabelle auf der rechten Seite werden nur solche Datens\"atze angezeigt, die der Join-Bedingung gen\"ugen.

          \bild{Left-Outer-Join}{left_outer_join}{1}

          \beispiel{sql04_12} zeigt einen Left-Outer-Join zwischen den beiden Tabellen \identifier{Mitarbeiter} und \identifier{Bankfiliale}. Die Auswirkungen des Left-Outer-Joins zeigen sich im Ergebnis nur in den letzten sieben Zeilen. Dort werden Mitarbeiter angezeigt, die in keiner Bankfiliale arbeiten und somit nicht der Join-Bedingung gen\"ugen.
          \begin{lstlisting}[language=oracle_sql,caption={Ein Left-Outer-Join in Oracle},label=sql04_12]
SELECT   Vorname, Nachname, b.PLZ, b.Ort
FROM     Mitarbeiter m LEFT OUTER JOIN Bankfiliale b
           ON (m.Bankfiliale_ID = b.Bankfiliale_ID);
          \end{lstlisting}
          \begin{center}
            \begin{small}
              \changefont{pcr}{m}{n}
              \tablefirsthead {
                \multicolumn{1}{l}{\textbf{VORNAME}} &
                \multicolumn{1}{l}{\textbf{NACHNAME}} &
                \multicolumn{1}{l}{\textbf{B.PLZ}} &
                \multicolumn{1}{l}{\textbf{B.ORT}} \\
                \cmidrule(l){1-1}\cmidrule(l){2-2}\cmidrule(l){3-3}\cmidrule(l){4-4}
              }
              \tablehead{}
              \tabletail {
                \multicolumn{4}{l}{\textbf{100 Zeilen ausgew\"ahlt}} \\
              }
              \tablelasttail {
                \multicolumn{4}{l}{\textbf{100 Zeilen ausgew\"ahlt}} \\
              }
              \begin{oraclesql}
                \begin{supertabular}{llll}
                  Amelie & Kr\"uger & 06449 & Aschersleben \\
                  Marie & Kipp & 06449 & Aschersleben \\
                  \dots & \dots & \dots & \dots \\
                  Emily & Meier &  &  \\
                  Peter & M\"uller &  &  \\
                  Tim & Sindermann &  &  \\
                  Sebastian & Schwarz &  &  \\
                  Finn & Seifert &  &  \\
                  Sarah & Werner &  &  \\
                  Max & Winter &  &  \\
                \end{supertabular}
              \end{oraclesql}
            \end{small}
          \end{center}
          \begin{merke}
          Da in \beispiel{sql04_12} keine Sortierung vorgegeben wurde zeigt Oracle die Zeilen mit den NULL-Werten, in den Spalten \identifier{plz} und \identifier{ort}, automatisch ganz zuletzt an! Dieses Verhalten kann mit dem \languageorasql{NULLS FIRST}-Schl\"usselwort, in der \languageorasql{ORDER BY}-Klausel ge\"andert werden.
          \end{merke}
          \begin{lstlisting}[language=oracle_sql,caption={NULL-Werte nach oben sortieren, NULLS FIRST},label=sql04_13]
SELECT   Vorname, Nachname, b.PLZ, b.Ort
FROM     Mitarbeiter m LEFT OUTER JOIN Bankfiliale b
           ON (m.Bankfiliale_ID = b.Bankfiliale_ID)
ORDER BY PLZ NULLS FIRST;
          \end{lstlisting}
\clearpage
          \begin{center}
            \begin{small}
              \changefont{pcr}{m}{n}
              \tablefirsthead {
                \multicolumn{1}{l}{\textbf{VORNAME}} &
                \multicolumn{1}{l}{\textbf{NACHNAME}} &
                \multicolumn{1}{l}{\textbf{B.PLZ}} &
                \multicolumn{1}{l}{\textbf{B.ORT}} \\
                \cmidrule(l){1-1}\cmidrule(l){2-2}\cmidrule(l){3-3}\cmidrule(l){4-4}
              }
              \tablehead{}
              \tabletail {
%                 \multicolumn{4}{l}{\textbf{100 Zeilen ausgew\"ahlt}} \\
              }
              \tablelasttail {
                \multicolumn{4}{l}{\textbf{100 Zeilen ausgew\"ahlt}} \\
              }
              \begin{oraclesql}
                \begin{supertabular}{llll}
                  Emily & Meier &  &  \\
                  Peter & M\"oller &  &  \\
                  Tim & Sindermann &  &  \\
                  Sebastian & Schwarz &  &  \\
                  Max & Winter &  &  \\
                  Sarah & Werner &  &  \\
                  Finn & Seifert &  &  \\
                  Sophie & Schwarz & 06406 & Bernburg \\
                \end{supertabular}
              \end{oraclesql}
            \end{small}
          \end{center}
          Der MS SQL Server unterst\"utzt die gleiche Syntax wie Oracle, kennt jedoch das \languageorasql{NULLS FIRST}-Schl\"usselwort nicht, da er NULL-Werte bei Angabe einer \languageorasql{ORDER BY}-Klausel automatisch oben anzeigt.
          \begin{lstlisting}[language=ms_sql,caption={Der Left-Outer-Join im MS SQL Server},label=sql04_14]
SELECT   Vorname, Nachname, b.PLZ, b.Ort
FROM     Mitarbeiter m LEFT OUTER JOIN Bankfiliale b
           ON (m.Bankfiliale_ID = b.Bankfiliale_ID)
ORDER BY PLZ;
          \end{lstlisting}
          \begin{center}
            \begin{small}
              \changefont{pcr}{m}{n}
              \tablefirsthead {
                \multicolumn{1}{l}{\textbf{VORNAME}} &
                \multicolumn{1}{l}{\textbf{NACHNAME}} &
                \multicolumn{1}{l}{\textbf{PLZ}} &
                \multicolumn{1}{l}{\textbf{ORT}} \\
                \cmidrule(l){1-1}\cmidrule(l){2-2}\cmidrule(l){3-3}\cmidrule(l){4-4}
              }
              \tablehead{}
              \tabletail {
                \multicolumn{4}{l}{\textbf{100 Zeilen ausgew\"ahlt}} \\
              }
              \tablelasttail {
                \multicolumn{4}{l}{\textbf{100 Zeilen ausgew\"ahlt}} \\
              }
              \begin{mssql}
                \begin{supertabular}{llll}
                  Emily & Meier & NULL & NULL \\
                  Peter & M\"oller & NULL  & NULL \\
                  Tim & Sindermann & NULL & NULL \\
                  Sebastian & Schwarz & NULL & NULL \\
                  Max & Winter & NULL & NULL \\
                  Sarah & Werner & NULL & NULL \\
                  Finn & Seifert & NULL & NULL \\
                  Sophie & Schwarz & 06406 & Bernburg \\
                \end{supertabular}
              \end{mssql}
            \end{small}
          \end{center}
        \subsubsection{Der Right-Outer-Join}
          Der Right-Outer-Join ist das Komplement zum Left-Outer-Join. Er zeigt alle Datens\"atze der Tabelle an, die sich auf der rechten Seite befindet. Aus der Tabelle auf der linken Join-Seite werden wiederum nur jene Zeilen angezeigt, die der Join-Bedingung gen\"ugen.
          \bild{Der Right-Outer-Join}{right_outer_join}{1}
          \begin{lstlisting}[language=oracle_sql,caption={Ein Right-Outer-Join in Oracle},label=sql04_15]
SELECT   Vorname, Nachname, b.PLZ, b.Ort
FROM     Mitarbeiter m RIGHT OUTER JOIN Bankfiliale b
           ON (m.Bankfiliale_ID = b.Bankfiliale_ID)
ORDER BY Nachname NULLS FIRST, PLZ;
          \end{lstlisting}
          \begin{center}
            \begin{small}
              \changefont{pcr}{m}{n}
              \tablefirsthead {
                \multicolumn{1}{l}{\textbf{VORNAME}} &
                \multicolumn{1}{l}{\textbf{NACHNAME}} &
                \multicolumn{1}{l}{\textbf{B.PLZ}} &
                \multicolumn{1}{l}{\textbf{B.ORT}} \\
                \cmidrule(l){1-1}\cmidrule(l){2-2}\cmidrule(l){3-3}\cmidrule(l){4-4}
              }
              \tablehead{}
              \tabletail {
%                 \multicolumn{4}{l}{\textbf{94 Zeilen ausgew\"ahlt}} \\
              }
              \tablelasttail {
                \multicolumn{4}{l}{\textbf{94 Zeilen ausgew\"ahlt}} \\
              }
              \begin{oraclesql}
                \begin{supertabular}{llll}
                &  & 06425 & Alsleben \\
                Finn & Bauer & 06425 & Pl\"otzkau \\
                Leonie & Bauer & 39444 & Hecklingen \\
                \end{supertabular}
              \end{oraclesql}
            \end{small}
          \end{center}
          Der erste Datensatz aus \beispiel{sql04_15} zeigt, dass es eine Bankfiliale gibt, in der noch keine Mitarbeiter arbeiten. Das gleiche Beispiel l\"asst sich auch  in MS SQL Server abarbeiten.
          \begin{lstlisting}[language=ms_sql,caption={Der gleiche Right-Outer-Join in MS SQL Server},label=sql04_16]
SELECT   Vorname, Nachname, b.PLZ, b.Ort
FROM     Mitarbeiter m RIGHT OUTER JOIN Bankfiliale b
           ON (m.Bankfiliale_ID = b.Bankfiliale_ID)
ORDER BY PLZ, Nachname;
          \end{lstlisting}
          \begin{center}
            \begin{small}
              \changefont{pcr}{m}{n}
              \tablefirsthead {
                \multicolumn{1}{l}{\textbf{VORNAME}} &
                \multicolumn{1}{l}{\textbf{NACHNAME}} &
                \multicolumn{1}{l}{\textbf{PLZ}} &
                \multicolumn{1}{l}{\textbf{ORT}} \\
                \cmidrule(l){1-1}\cmidrule(l){2-2}\cmidrule(l){3-3}\cmidrule(l){4-4}
              }
              \tablehead{}
              \tabletail {
%                 \multicolumn{4}{l}{\textbf{94 Zeilen ausgew\"ahlt}} \\
              }
              \tablelasttail {
                \multicolumn{4}{l}{\textbf{94 Zeilen ausgew\"ahlt}} \\
              }
              \begin{mssql}
                \begin{supertabular}{llll}
									NULL & NULL & 06425 & Alsleben \\
									Finn & Bauer & 06425 & Pl\"otzkau \\
									Leonie & Bauer & 39444 & Hecklingen \\
                \end{supertabular}
              \end{mssql}
            \end{small}
          \end{center}
      \subsection{Der Full Outer Join}
        Der Full-Outer-Join stellt die logische Erg\"anzung zu Left-Outer-Join und Right-Outer-Join dar. Er verk\"upft zwei Tabellen miteinander und zeigt auf beiden Seiten jeweils alle Tabellenzeilen an. Er ist in beiden DBMS, Oracle und MS SQL Server bekannt und syntaktisch gleich.
        \begin{lstlisting}[language=oracle_sql,caption={Ein Full-Outer-Join in Oracle},label=sql04_17]
SELECT   Vorname, Nachname, b.PLZ, b.Ort
FROM     Mitarbeiter m FULL OUTER JOIN Bankfiliale b
           ON (m.Bankfiliale_ID = b.Bankfiliale_ID)
ORDER BY PLZ NULLS FIRST, Nachname NULLS FIRST;
        \end{lstlisting}
\clearpage
				\begin{center}
          \begin{small}
            \changefont{pcr}{m}{n}
            \tablefirsthead {
              \multicolumn{1}{l}{\textbf{VORNAME}} &
              \multicolumn{1}{l}{\textbf{NACHNAME}} &
              \multicolumn{1}{l}{\textbf{B.PLZ}} &
              \multicolumn{1}{l}{\textbf{B.ORT}} \\
              \cmidrule(l){1-1}\cmidrule(l){2-2}\cmidrule(l){3-3}\cmidrule(l){4-4}
            }
            \tablehead{}
            \tabletail {
              \multicolumn{4}{l}{\textbf{101 Zeilen ausgew\"ahlt}} \\
            }
            \tablelasttail {
              \multicolumn{4}{l}{\textbf{101 Zeilen ausgew\"ahlt}} \\
            }
            \begin{oraclesql}
              \begin{supertabular}{llll}
                Emily & Meier &  &  \\
                Peter & M\"oller &  &  \\
                Sebastian & Schwarz &  &  \\
                Finn & Seifert &  &  \\
                \dots & \dots & \dots & \dots \\
                Anne & Zimmermann & 06406 & Bernburg \\
                Franz & Berger & 06408 & Ilberstedt \\
                \dots & \dots & \dots & \dots \\
                &  & 06425 & Alsleben \\
                Finn & Bauer & 06425 & Pl\"otzkau \\
              \end{supertabular}
            \end{oraclesql}
          \end{small}
        \end{center}
    \section{Spezielle Joins}
      \subsection{Der Self-Join}
        Ein Self-Join ist eine besondere Form des Inner Join. Er kommt immer dann zum Einsatz wenn der Primary Key einer Tabelle auf einen Foreign Key in der gleichen Tabelle zeigt, also bei rekursiven Beziehungstypen. Ein solcher rekursiver Beziehungstyp existiert in der Tabelle \identifier{Mitarbeiter}. Er stellt das Vorgesetztenverh\"altnis zwischen den Mitarbeitern dar.

        Wenn als Ergebnis einer Abfrage zu jedem Mitarbeiter sein Vorgesetzter angezeigt werden soll, so geht dies nur mittels Self-Join. Die folgende Tabelle zeigt das Ergebnis eines solchen Self-Joins.
        \begin{center}
          \begin{small}
            \changefont{pcr}{m}{n}
            \tablefirsthead {
              \multicolumn{1}{r}{\textbf{M\#}} &
              \multicolumn{1}{l}{\textbf{MVORNAME}} &
              \multicolumn{1}{l}{\textbf{MNACHNAME}} &
              \multicolumn{1}{r}{\textbf{V\#}} &
              \multicolumn{1}{l}{\textbf{VVORNAME}} &
              \multicolumn{1}{l}{\textbf{VNACHNAME}} \\
              \cmidrule(r){1-1}\cmidrule(l){2-2}\cmidrule(l){3-3}\cmidrule(r){4-4}\cmidrule(l){5-5}\cmidrule(l){6-6}
            }
            \tablehead{}
            \tabletail {
              \multicolumn{6}{l}{\textbf{99 Zeilen ausgew\"ahlt}} \\
            }
            \tablelasttail {}
            \begin{msoraclesql}
              \begin{supertabular}{rllrll}
                2 & Sarah & Werner & 1 & Max & Winter \\
                3 & Finn & Seifert & 1 & Max & Winter \\
                4 & Sebastian & Schwarz & 2 & Sarah & Werner \\
                5 & Tim & Sindermann & 2 & Sarah & Werner \\
                6 & Peter & M\"oller & 3 & Finn & Seifert \\
                7 & Emily & Meier & 3 & Finn & Seifert \\
                8 & Dirk & Peters & 4 & Sebastian & Schwarz \\
                9 & Louis & Winter & 4 & Sebastian & Schwarz \\
              \end{supertabular}
            \end{msoraclesql}
          \end{small}
        \end{center}
\clearpage
        \subsubsection{Die Quelltabelle aufspalten}
          Grunds\"atzlich ist die Aufgabe einer Join-Operation zwei Tabellen zu einer Ergebnisrelation zu verkn\"upfen. Im besonderen Falle eines rekursiven Beziehungstyps existiert jedoch nur eine Tabelle. Wie kann der Join stattfinden? Die Antwort auf diese Frage liegt in der Nutzung von Tabellenaliasnamen.
          \begin{merke}
            Durch die Vergabe von Tabellenaliasnamen kann mehrfach auf ein und die selbe Tabelle, innerhalb eines SQL-Statements, zugegriffen werden!
          \end{merke}
          \bild{Gespaltene Pers\"onlichkeit - Eine Tabelle, zwei Aliase}{tabellenaliasnamen_im_selfjoin}{1}
          \abbildung{tabellenaliasnamen_im_selfjoin} zeigt, das f\"ur die Tabelle \identifier{Mitarbeiter} zwei Tabellenaliasnamen vergeben werden, n\"amlich \enquote{A} f\"ur Angestellter und \enquote{V} f\"ur Vorgesetzter. In SQL ausgedr\"uckt bedeutet dies:
          \begin{lstlisting}[language=oracle_sql,caption={Eine Tabelle - zwei Aliasnamen},label=sql04_18]
SELECT m.*
FROM   Mitarbeiter m INNER JOIN Mitarbeiter v
...
          \end{lstlisting}
        \subsubsection{Die richtige Join-Bedingung finden}
          Die eigentliche Leistung, bei der Erstellung eines Self-Join, liegt darin, die korrekte Join-Bedingung zu finden. Fest steht, dass die beiden Spalten \identifier{Mitarbeiter\_ID} und \identifier{Vorgesetzter\_ID} am Join beteiligt sein werden, aber es gibt insgesamt vier verschiedene M\"oglichkeiten, diese beiden Spalten zu kombinieren:
          \begin{itemize}
            \item \languageorasql{ON (m.Mitabeiter_ID = v.Mitarbeiter_ID)}
            \item \languageorasql{ON (m.Mitarbeiter_ID = v.Vorgesetzter_ID)}
            \item \languageorasql{ON (m.Vorgesetzter_ID = v.Mitarbeiter_ID)}
            \item \languageorasql{ON (m.Vorgesetzter_ID = v.Vorgesetzter_ID)}
          \end{itemize}
          Nun gilt es herauszufinden, welche die richtige Variante ist. Dies geht am Einfachsten, in dem man sich Beispieldaten schafft.
\clearpage
          \begin{center}
            \begin{small}
              \changefont{pcr}{m}{n}
              \tablefirsthead {
                \multicolumn{1}{r}{\textbf{MIT\_M\#}} &
                \multicolumn{1}{l}{\textbf{MIT\_NACHNAME}} &
                \multicolumn{1}{r}{\textbf{MIT\_V\#}} &
                \multicolumn{1}{r}{\textbf{VOR\_M\#}} &
                \multicolumn{1}{l}{\textbf{VOR\_NACHNAME}} &
                \multicolumn{1}{r}{\textbf{VOR\_V\#}} \\
                \cmidrule(r){1-1}\cmidrule(l){2-2}\cmidrule(r){3-3}\cmidrule(r){4-4}\cmidrule(l){5-5}\cmidrule(r){6-6}
              }
              \tablehead{}
              \tabletail {}
              \tablelasttail {}
              \begin{msoraclesql}
                \begin{supertabular}{rlrrlr}
                  3 & Seifert & 1 & 1 & Winter &  \\
                  2 & Werner & 1 & 1 & Winter &  \\
                  5 & Sindermann & 2 & 2 & Werner & 1 \\
                  4 & Schwarz & 2 & 2 & Werner & 1 \\
                  7 & Meier & 3 & 3 & Seifert & 1 \\
                  6 & M\"oller & 3 & 3 & Seifert & 1 \\
                  12 & Weber & 4 & 4 & Schwarz & 2 \\
                  11 & Schwarz & 4 & 4 & Schwarz & 2 \\
                \end{supertabular}
              \end{msoraclesql}
            \end{small}
          \end{center}
          Betrachtet man nun diese vier Join-Bedingungen, im Zusammenhang mit den Beispieldaten, lassen sich zwei davon direkt ausschlie\ss{}en.
          \begin{itemize}
            \item \languageorasql{ON (m.Mitabeiter_ID = v.Mitarbeiter_ID)}
            \item \languageorasql{ON (m.Vorgesetzter_ID = v.Vorgesetzter_ID)}
          \end{itemize}
          Die Bedingung \languageorasql{ON (m.Mitabeiter_ID = v.Mitarbeiter_ID)} verkn\"upft den Mitarbeiter aus der \enquote{Tabelle A} mit dem gleichen Mitarbeiter aus der \enquote{Tabelle V}. Das bedeutet, dass alle Mitarbeiter mit sich selbst verkn\"upft werden, aber nicht mit Ihrem Vorgesetzten.

          Die zweite Bedingung \languageorasql{ON (m.Vorgesetzter_ID = v.Vorgesetzter_ID)} erzeugt \enquote{logisches Chaos}. Der Mitarbeiter Seifert liefert hierzu ein gutes Beispiel:
          \begin{center}
            \begin{small}
              \changefont{pcr}{m}{n}
              \tablefirsthead {
                \multicolumn{1}{r}{\textbf{MIT\_M\#}} &
                \multicolumn{1}{l}{\textbf{MIT\_NACHNAME}} &
                \multicolumn{1}{r}{\textbf{MIT\_V\#}} &
                \multicolumn{1}{r}{\textbf{VOR\_M\#}} &
                \multicolumn{1}{l}{\textbf{VOR\_NACHNAME}} &
                \multicolumn{1}{r}{\textbf{VOR\_V\#}} \\
                \cmidrule(r){1-1}\cmidrule(l){2-2}\cmidrule(r){3-3}\cmidrule(r){4-4}\cmidrule(l){5-5}\cmidrule(r){6-6}
              }
              \tablehead{}
              \tabletail {}
              \tablelasttail {}
              \begin{msoraclesql}
                \begin{supertabular}{rlrrlr}
                  3 & Seifert & 1 & 1 & Werner & 1 \\
                  3 & Seifert & 1 & 1 & Seifert & 1 \\
                \end{supertabular}
              \end{msoraclesql}
            \end{small}
          \end{center}
          Es zeigt sich, dass der Mitarbeiter Seifert mit sich selbst und mit seinem Kollegen Werner verkn\"upft wird. Beide haben eines gemeinsam: Sie haben den gleichen Vorgesetzten. Somit verbleiben nur noch zwei Bedingungen:
          \begin{itemize}
            \item \languageorasql{ON (m.Mitarbeiter_ID = v.Vorgesetzter_ID)}
            \item \languageorasql{ON (m.Vorgesetzter_ID = v.Mitarbeiter_ID)}
          \end{itemize}
          Die verbliebenen Bedingungen liefern beide ein sinnvolles Ergebnis. Verwendet man die erste \languageorasql{ON (m.Mitarbeiter_ID = v.Vorgesetzter_ID)} zeigt sich folgendes Ergebnis:
          \begin{center}
            \begin{small}
              \changefont{pcr}{m}{n}
              \tablefirsthead {
                \multicolumn{1}{r}{\textbf{MIT\_M\#}} &
                \multicolumn{1}{l}{\textbf{MIT\_NACHNAME}} &
                \multicolumn{1}{r}{\textbf{MIT\_V\#}} &
                \multicolumn{1}{r}{\textbf{VOR\_M\#}} &
                \multicolumn{1}{l}{\textbf{VOR\_NACHNAME}} &
                \multicolumn{1}{r}{\textbf{VOR\_V\#}} \\
                \cmidrule(r){1-1}\cmidrule(l){2-2}\cmidrule(r){3-3}\cmidrule(r){4-4}\cmidrule(l){5-5}\cmidrule(r){6-6}
              }
              \tablehead{}
              \tabletail {}
              \tablelasttail {}
              \begin{msoraclesql}
                \begin{supertabular}{rlrrlr}
                  3 & Seifert & 1 & 6 & M\"oller & 3 \\
                  3 & Seifert & 1 & 7 & Meier & 3 \\
                \end{supertabular}
              \end{msoraclesql}
            \end{small}
          \end{center}
          Bei beiden Mitarbeitern, M\"oller und Meier, steht in der Spalte \identifier{Vorgesetzter\_ID} der Wert 3. Daraus folgt, beide haben den Mitarbeiter Nummer drei als Vorgesetzten. Mitarbeiter Nummer drei ist Seifert. Mit Hilfe dieser Join-Bedingung werden zu jedem Vorgesetzten die Untergebenen angezeigt. Gesucht ist aber etwas anderes:

          Zu jedem Angestellten soll der Vorgesetzte angezeigt werden. Die aktuelle Join-Bedingung zeigt die Informationen also nur aus der falschen Sichtweise an.

          Was bleibt, ist nur noch die Bedingung \languageorasql{ON (m.Vorgesetzter_ID = v.Mitarbeiter_ID)}. Diese zeigt das korrekte, gew\"unschte Ergebnis an.
          \begin{center}
            \begin{small}
              \changefont{pcr}{m}{n}
              \tablefirsthead {
                \multicolumn{1}{r}{\textbf{MIT\_M\#}} &
                \multicolumn{1}{l}{\textbf{MIT\_NACHNAME}} &
                \multicolumn{1}{r}{\textbf{MIT\_V\#}} &
                \multicolumn{1}{r}{\textbf{VOR\_M\#}} &
                \multicolumn{1}{l}{\textbf{VOR\_NACHNAME}} &
                \multicolumn{1}{r}{\textbf{VOR\_V\#}} \\
                \cmidrule(r){1-1}\cmidrule(l){2-2}\cmidrule(r){3-3}\cmidrule(r){4-4}\cmidrule(l){5-5}\cmidrule(r){6-6}
              }
              \tablehead{}
              \tabletail {}
              \tablelasttail {}
              \begin{msoraclesql}
                \begin{supertabular}{rlrrlr}
                  3 & Seifert & 1 & 1 & Winter &  \\
                \end{supertabular}
              \end{msoraclesql}
            \end{small}
          \end{center}
          Das komplette SQL-Statement zu dieser Problemstellung lautet:
          \begin{lstlisting}[language=oracle_sql,caption={Ein Self-Join},label=sql04_19]
SELECT m.Mitarbeiter_ID AS MIT_M#, m.Vorname AS MIT_Vorname,
       m.Nachname AS MIT_Nachname, m.Vorgesetzter_ID AS MIT_V#,
       v.Mitarbeiter_ID AS VOR_M#, v.Vorname AS VOR_Vorname,
       v.Nachname AS VOR_Nachname, v.Vorgesetzter_ID AS VOR_V#
FROM   Mitarbeiter m INNER JOIN Mitarbeiter v
       ON (m.Vorgesetzter_ID = v.Mitarbeiter_ID);
          \end{lstlisting}
      \subsection{Non-Equi-Joins}
        Kurzgesagt ist ein Non-Equi-Join ein Join, der nicht den Gleichheitsoperator (=) verwendet, sondern einen beliebigen anderen. Meist ist dies dann der \languageorasql{BETWEEN}-Operator. Da diese Art von Join in der Praxis jedoch \"au\ss{}erst selten ist, soll an dieser Stelle nicht weiter darauf eingegangen werden.
    \section{Mengenoperationen}
      In den vorangegangenen Abschnitten wurde gezeigt, wie zwei Tabellen durch eine Join-Operation miteinander verkn\"upft werden k\"onnen. Dies bedingt immer, dass in beiden Tabellen eine Spalte vorhanden ist, die als Join-Attribut genutzt werden kann. Zus\"atzlich dazu, gibt es noch eine weitere Methode Datens\"atze unterschiedlicher Tabellen miteinander zu verkn\"upfen, die \textit{SET-Operatoren}.
\clearpage
      SET-Operatoren erm\"oglichen es, die Operationen der Mengenlehre in einer Datenbank durchzuf\"uhren. \tabelle{setoperators} zeigt die Operationen und die dazu geh\"orenden  Operatoren:
      \begin{center}
        \tablecaption{Die SET-Operatoren}
        \label{setoperators}
        \begin{small}
          \tablefirsthead{
            \multicolumn{1}{c}{\textbf{Mengenoperation}} &
            \multicolumn{1}{c}{\textbf{SET-Operator}} &
            \multicolumn{1}{c}{\textbf{Erl\"auterung}} \\
            \hline
          }
          \tabletail{
            \hline
          }
          \tablelasttail {
            \hline
          }
          \begin{supertabular}{|l|c|p{8cm}|}
          Vereinigung & UNION & Zeigt die Vereinigungsmenge der beiden beteiligten Tabellen an. Duplikatzeilen werden vor der Anzeige eliminiert. \\
          \hline
          Vollst\"andige Vereinigung & UNION ALL & Zeigt die Vereinigungsmenge der beiden beteiligten Tabellen an. Duplikatzeilen werden vor der Anzeige nicht eliminiert. \\
          \hline
          Differenz (Oracle) & MINUS  & Zeigt nur die Datens\"atze an, die in der linken der beiden Tabellen vorkommen und keine Entsprechung in der rechten Tabelle haben.\\
          \hline
          Differenz (MS SQL Server) & EXCEPT  & Zeigt nur die Datens\"atze an, die in der linken der beiden Tabellen vorkommen und keine Entsprechung in der rechten Tabelle haben.\\
          \hline
          Durchschnitt & INTERSECT & Zeigt nur die Schnittmenge beider Tabellen an.\\
          \end{supertabular}
        \end{small}
      \end{center}
      \subsection{Voraussetzungen zur Nutzung der SET-Operatoren}
        Um Mengenoperationen, auf zwei Relationen \identifier{R} und \identifier{S}, anwenden zu k\"onnen, m\"ussen beide miteinander kompatibel sein. Diese Form der Kompatibilit\"at wird \textit{Typenkompatibilit\"at} oder auch \textit{Vereinigungsvertr\"aglichkeit} genannt. Damit zwei Tabellen zueinander Typenkompatibel sind, m\"ussen folgende Bedingungen gegeben sein:
        \begin{itemize}
          \item \identifier{R} und \identifier{S} m\"ussen die gleiche Anzahl Attribute aufweisen.
          \item Der Wertebereich/Datentyp der Attribute von \identifier{R} und \identifier{S} muss identisch sein.
        \end{itemize}
        Das bedeutet zum einen, dass nur solche Abfragen mit Hilfe von SET-Operatoren kombiniert werden k\"onnen, die die gleiche Anzahl Spalten in der \languageorasql{SELECT}-Klausel haben. Zum anderen m\"ussen die verkn\"upften Spalten den gleichen Datentyp aufweisen.
      \subsection{Die SET-Operatoren}
        \subsubsection{UNION und UNION ALL}
          Der \languageorasql{UNION ALL}-Operator verbindet die Ergebnisse zweier \languageorasql{SELECT}-Statements (Vereinigungsmenge). Sollte es Datens\"atze geben, die in beiden Abfragen ausgew\"ahlt werden (redundante Zeilen), werden diese angezeigt.

          Der \languageorasql{UNION}-Operator verbindet, genau wie der \languageorasql{UNION ALL}-Operator, die Ergebnisse zweier SQL-Statements. Der Unterschied zwischen beiden liegt darin, dass der \languageorasql{UNION}-Operator redundante Zeilen ausschlie\ss{}t.

          \bild{Vereinigungs\-menge mit UNION ALL}{union_all}{1}

          In einem einfachen Beispiel zum \languageorasql{UNION ALL}-Operator sollen alle Orte angezeigt werden, in denen Kunden oder Mitarbeiter leben. Um diese Aufgabe zu l\"osen, m\"ussen zwei Abfragen ausgef\"uhrt werden.
          \begin{lstlisting}[language=oracle_sql,caption={Orte, an denen Kunden leben},label=sql04_20]
SELECT Ort
FROM   Eigenkunde;
          \end{lstlisting}
          \begin{lstlisting}[language=oracle_sql,caption={Orte, an denen Mitarbeiter leben},label=sql04_21]
SELECT Ort
FROM   Mitarbeiter;
          \end{lstlisting}
          Zur L\"osung der Aufgabe, m\"ussen die Ergebnisse beider Abfragen kombiniert werden. Dies wird im ersten Anlauf durch den \languageorasql{UNION ALL}-Operator erledigt.
          \begin{lstlisting}[language=oracle_sql,caption={Orte, an denen Kunden oder Mitarbeiter leben},label=sql04_22]
SELECT Ort
FROM   Mitarbeiter
UNION ALL
SELECT Ort
FROM   Eigenkunde;
          \end{lstlisting}
          \begin{center}
            \begin{small}
              \changefont{pcr}{m}{n}
              \tablefirsthead {
                \multicolumn{1}{l}{\textbf{ORT}} \\
                \cmidrule(l){1-1}
              }
              \tablehead{}
              \tabletail {}
              \tablelasttail {
                \multicolumn{1}{l}{\textbf{500 Zeilen ausgew\"ahlt}} \\
              }
              \begin{msoraclesql}
                \begin{supertabular}{l}
                  Aschersleben \\
                  B\"ordeaue \\
                  Borne \\
                  Sch\"onebeck \\
                  Alsleben \\
                  Hamburg \\
                  Borne \\
                  Egeln \\
                  Sch\"onebeck \\
                \end{supertabular}
              \end{msoraclesql}
            \end{small}
          \end{center}
          An einigen Orten, wie z. B. Aschersleben, Borne, Egeln und Sch\"onebeck, ist zu erkennen, dass der \languageorasql{UNION ALL}-Operator keine redundanten Zeilen ausblendet. Soll das Ergebnis reduziet werden, so dass jeder Ort genau einmal angezeigt wird, kommt der \languageorasql{UNION}-Operator zum Einsatz.
          \begin{lstlisting}[language=oracle_sql,caption={Orte, an denen Kunden oder Mitarbeiter leben (reduziert)},label=sql04_23]
SELECT Ort
FROM   Mitarbeiter
UNION
SELECT Ort
FROM   Eigenkunde;
          \end{lstlisting}
          \begin{center}
            \begin{small}
              \changefont{pcr}{m}{n}
              \tablefirsthead {
                \multicolumn{1}{l}{\textbf{ORT}} \\
                \cmidrule(l){1-1}
              }
              \tablehead{}
              \tabletail {
                \multicolumn{1}{l}{\textbf{30 Zeilen ausgew\"ahlt}} \\
              }
              \tablelasttail {
                \multicolumn{1}{l}{\textbf{30 Zeilen ausgew\"ahlt}} \\
              }
              \begin{msoraclesql}
                \begin{supertabular}{l}
                  Alsleben \\
                  Aschersleben \\
                  Barby \\
                  Berlin \\
                  Bernburg \\
                  Borne \\
                  B\"ordeaue \\
                  Calbe \\
                \end{supertabular}
              \end{msoraclesql}
            \end{small}
          \end{center}
          Durch die Anwendung des \languageorasql{UNION}-Operators, statt des \languageorasql{UNION ALL}-Operators verk\"urzt sich das Ergebnis von 500 Zeilen auf 30.
          \begin{merke}
            Der \languageorasql{UNION ALL}-Operator sollte nur dann zum Einsatz kommen, wenn dies zwingend notwendig ist!
          \end{merke}
          In einem weiteren Beispiel soll gezeigt werden, wie Datens\"atze aus unterschiedlichen Tabellen im Ergebnis gekennzeichnet werden k\"onnen. In einer Abfrage sollen alle Mitarbeiter und alle Kunden mit den Attributen \identifier{Vorname}, \identifier{Nachname}, \identifier{PLZ} und \identifier{Ort} angezeit werden. F\"ur die Kunden muss in einer extra Spalte der Buchtabe \enquote{K} und f\"ur alle Mitarbeiter der Buchstabe \enquote{M} angezeigt werden.
          \begin{lstlisting}[language=oracle_sql,caption={Spalten mit konstanten Werten und UNION},label=sql04_24]
SELECT 'M' AS Personentyp, Vorname, Nachname, PLZ, Ort
FROM   Mitarbeiter
UNION
SELECT 'K', Vorname, Nachname, PLZ, Ort
FROM   Kunde k INNER JOIN Eigenkunde ek
       ON (k.Kunden_ID = ek.Kunden_ID);
          \end{lstlisting}
          \begin{center}
            \begin{small}
              \changefont{pcr}{m}{n}
              \tablefirsthead {
                \multicolumn{1}{l}{\textbf{PERSONENTYP}} &
                \multicolumn{1}{l}{\textbf{VORNAME}} &
                \multicolumn{1}{l}{\textbf{NACHNAME}} &
                \multicolumn{1}{l}{\textbf{PLZ}} &
                \multicolumn{1}{l}{\textbf{ORT}} \\
                \cmidrule(l){1-1}\cmidrule(l){2-2}\cmidrule(l){3-3}\cmidrule(l){4-4}\cmidrule(l){5-5}
              }
              \tablehead{}
              \tabletail {
                \multicolumn{5}{l}{\textbf{500 Zeilen ausgew\"ahlt}} \\
              }
              \tablelasttail {
                \multicolumn{5}{l}{\textbf{500 Zeilen ausgew\"ahlt}} \\
              }
              \begin{msoraclesql}
                \begin{supertabular}{lllll}
                  K & Alexander & Huber & 22043 & Hamburg \\
                  K & Alexander & Lorenz & 06408 & Ilberstedt \\
                  K & Alina & Baumann & 07545 & Gera \\
                  K & Alina & Braun & 39435 & Egeln \\
                  \dots & \dots & \dots & \dots & \dots \\
                  M & Alexander & Weber & 06449 & Aschersleben \\
                  M & Amelie & Kr\"uger & 03042 & Cottbus \\
                  M & Anna & Keller & 39104 & Magdeburg \\
                  M & Anna & Schneider & 06449 & Giersleben \\
                \end{supertabular}
              \end{msoraclesql}
            \end{small}
          \end{center}
          In der ersten Abfrage wird eine Spalte, mit Aliasnamen \identifier{Personentyp} eingef\"ugt. Sie bezieht ihren Wert nicht aus einer Tabelle, sondern sie enth\"alt einfach nur den Buchstaben \enquote{K} f\"ur Kunde. Die gleiche Spalte muss nun auch in der zweiten Abfrage eingef\"uhrt werden, da beide Abfragen, wie bereits erw\"ahnt, die gleiche Anzahl Spalten, mit den gleichen Datentypen haben m\"ussen. In der zweiten Abfrage kann jedoch der Aliasname entfallen, da dieser nur in der ersten Abfrage registriert/genutzt wird.
        \subsubsection{INTERSECT}
          Mit Hilfe des \languageorasql{INTERSECT}-Operators kann der Durchschnitt zweier Ergebnisse angezeigt werden. Das bedeutet, es werden nur die Zeilen angezeigt, die in beiden Relationen, R und S, gleicherma\ss{}en vorkommen.

          Um die Wirkungsweise dieses Operators zu demonstrieren, wird \beispiel{sql04_23} abgewandelt. Der \languageorasql{UNION}-Operator wird durch den \languageorasql{INTERSECT}-Operator ausgetauscht.

          \bild{Schnittmenge mit INTERSECT}{intersect}{1}

          \begin{lstlisting}[language=oracle_sql,caption={Orte, an denen sowohl Kunden als auch Mitarbeiter leben},label=sql04_25]
SELECT Ort
FROM   Mitarbeiter
INTERSECT
SELECT Ort
FROM   Eigenkunde;
          \end{lstlisting}
\clearpage
          \begin{center}
            \begin{small}
              \changefont{pcr}{m}{n}
              \tablefirsthead {
                \multicolumn{1}{l}{\textbf{ORT}} \\
                \cmidrule(l){1-1}
              }
              \tablehead{}
              \tabletail {
                \multicolumn{1}{l}{\textbf{25 Zeilen ausgew\"ahlt}} \\
              }
              \tablelasttail {
                \multicolumn{1}{l}{\textbf{25 Zeilen ausgew\"ahlt}} \\
              }
              \begin{msoraclesql}
                \begin{supertabular}{l}
                  Alsleben \\
                  Aschersleben \\
                  Bernburg \\
                  Borne \\
                  B\"ordeaue \\
                \end{supertabular}
              \end{msoraclesql}
            \end{small}
          \end{center}
          Das Ergebnis dieser Abfrage liefert nur noch die Orte, an denen sowohl Kunden als auch Mitarbeiter leben.
        \subsubsection{MINUS / EXCEPT}
          Dieser Operator zeigt den Inhalt der linken Relation, ohne den Inhalt der Rechten an. Korrekt ausgedr\"uckt bedeutet dies: $ t1 MINUS t2 = t1 \setminus t2$. F\"ur SQL Server muss anstatt \languageorasql{MINUS} der Operator \languagemssql{EXCEPT} genutz werden.

          \bild{Der MINUS / EXCEPT Operator}{minus}{1}

          F\"ur das kommende Beispiel werden die beiden Tabellen \identifier{Mitarbeiter} und \identifier{Eigenkunde} vertauscht. Der \languageorasql{INTERSECT}-Operator wird gegen den \languageorasql{MINUS}-Operator ausgewechselt.
          \begin{lstlisting}[language=oracle_sql,caption={Orte, an denen nur Kunden, aber keine Mitarbeiter leben},label=sql04_26]
SELECT Ort
FROM   Eigenkunde
MINUS
SELECT Ort
FROM   Mitarbeiter;
          \end{lstlisting}
          \begin{center}
            \begin{small}
              \changefont{pcr}{m}{n}
              \tablefirsthead {
                \multicolumn{1}{l}{\textbf{ORT}} \\
                \cmidrule(l){1-1}
              }
              \tablehead{}
              \tabletail {}
              \tablelasttail {
                \multicolumn{1}{l}{\textbf{5 Zeilen ausgew\"ahlt}} \\
              }
              \begin{oraclesql}
                \begin{supertabular}{l}
                  Barby \\
                  Berlin \\
                  Leipzig \\
                  Sta\ss{}furt \\
                  Wolmirsleben \\
                \end{supertabular}
              \end{oraclesql}
            \end{small}
          \end{center}
          Es gibt 30 verschiedene Orte, an denen Kunden leben und 25 verschiedene Orte, an denen Mitarbeiter leben. In 5 Orten leben nur Kunden, aber keine Mitarbeiter. Der \languageorasql{MINUS}-Operation bzw. der \languagemssql{EXCEPT}-Operator ist dabei behilflich, diese Orte herauszufiltern.

    \input{../sql/uebungen/sql_04_erweiterte_datenselektion_uebungen}
    \input{../sql/loesungen/sql_04_erweiterte_datenselektion_loesungen}
  \input{../sql/05_gruppenfunktionen}
    \clearpage
    \section{\"Ubungen - Gruppenfunktionen}
      \begin{enumerate}
        \item Schreiben Sie eine Abfrage, die das h\"ochste und das niedrigste
        Gehalt, das Durchschnittsgehalt und die Summe aller Geh\"alter ausgibt.
        Beschriften Sie die Spalten, wie es in der L\"osung zu sehen ist.
        \begin{center}
          \begin{small}
            \changefont{pcr}{m}{n}
            \tablefirsthead {
              \multicolumn{1}{r}{\textbf{Maximum}} &
              \multicolumn{1}{r}{\textbf{Minimum}} &
              \multicolumn{1}{r}{\textbf{Mittelwert}} &
              \multicolumn{1}{r}{\textbf{Summe}} \\
              \cmidrule(r){1-1}\cmidrule(r){2-2}\cmidrule(r){3-3}\cmidrule(r){4-4}
            }
            \tablehead{}
            \tabletail {
              \multicolumn{4}{l}{\textbf{1 Zeile ausgew\"ahlt}} \\
            }
            \tablelasttail {
              \multicolumn{4}{l}{\textbf{1 Zeile ausgew\"ahlt}} \\
            }

            \begin{msoraclesql}
              \begin{supertabular}{rrrr}
                88000 & 1000 & 7255 & 725500 \\
              \end{supertabular}
            \end{msoraclesql}
          \end{small}
        \end{center}
        \item Ver\"andern Sie die Abfrage aus der vorangegangenen Abfrage so,
        dass die Informationen f\"ur jede einzelne Bankfiliale angezeigt werden.
        Sortieren sie das Ergebnis nach den IDs der Bankfilialen.
        \begin{center}
          \begin{small}
            \changefont{pcr}{m}{n}
            \tablefirsthead {
              \multicolumn{1}{r}{\textbf{BANKFILIALE\_ID}} &
              \multicolumn{1}{r}{\textbf{Maximum}} &
              \multicolumn{1}{r}{\textbf{Minimum}} &
              \multicolumn{1}{r}{\textbf{Mittelwert}} &
              \multicolumn{1}{r}{\textbf{Summe}} \\
              \cmidrule(r){1-1}\cmidrule(r){2-2}\cmidrule(r){3-3}\cmidrule(r){4-4}\cmidrule(r){5-5}
            }
            \tablehead{}
            \tabletail {
              \multicolumn{5}{l}{\textbf{20 Zeilen ausgew\"ahlt}} \\
            }
            \tablelasttail {
              \multicolumn{5}{l}{\textbf{20 Zeilen ausgew\"ahlt}} \\
            }
            \begin{msoraclesql}
              \begin{supertabular}{rrrrr}
                1 & 12000 & 2000 & 4100 & 20500 \\
                2 & 12000 & 1000 & 3400 & 17000 \\
                3 & 12000 & 1000 & 3900 & 19500 \\
                4 & 12000 & 1500 & 4100 & 20500 \\
                5 & 12000 & 1000 & 4200 & 21000 \\
                6 & 12000 & 2000 & 4200 & 21000 \\
              \end{supertabular}
            \end{msoraclesql}
          \end{small}
        \end{center}
        \item Schreiben Sie eine Abfrage, die die Anzahl der Mitarbeiter pro
        Bankfiliale ausgibt. Beschriften Sie die Spalten so, wie es in der
        L\"osung zu sehen ist und sortieren Sie das Ergebnis nach den IDs der
        Filialen.
        \begin{center}
          \begin{small}
            \changefont{pcr}{m}{n}
            \tablefirsthead {
              \multicolumn{1}{r}{\textbf{BANKFILIALE\_ID}} &
              \multicolumn{1}{r}{\textbf{Anzahl}} \\
              \cmidrule(r){1-1}\cmidrule(r){2-2}
            }
            \tablehead{}
            \tabletail {
              \multicolumn{2}{l}{\textbf{20 Zeilen ausgew\"ahlt}} \\
            }
            \tablelasttail {
              \multicolumn{2}{l}{\textbf{20 Zeilen ausgew\"ahlt}} \\
            }
            \begin{msoraclesql}
              \begin{supertabular}{rr}
                1 & 5 \\
                2 & 5 \\
                3 & 5 \\
                4 & 5 \\
                5 & 5 \\
                6 & 5 \\
             \end{supertabular}
            \end{msoraclesql}
          \end{small}
        \end{center}
\clearpage
        \item Schreiben Sie eine Abfrage, die f\"ur jeden Ort einzeln, die
        Anzahl der Eigenkunden z\"ahlt, die vor dem \enquote{01.01.1990} 18
        Jahre alt waren.
        \begin{center}
          \begin{small}
            \changefont{pcr}{m}{n}
            \tablefirsthead {
              \multicolumn{1}{l}{\textbf{ORT}} &
              \multicolumn{1}{r}{\textbf{Anzahl}} \\
              \cmidrule(l){1-1}\cmidrule(r){2-2}
            }
            \tablehead{}
            \tabletail {
            }
            \tablelasttail {
              \multicolumn{2}{l}{\textbf{15 Zeilen ausgew\"ahlt}} \\
            }
            \begin{msoraclesql}
              \begin{supertabular}{lr}
                Nienburg & 5 \\
                Calbe & 3 \\
                Hecklingen & 3 \\
                Dresden & 1 \\
                Berlin & 2 \\
                Sch\"onebeck & 1 \\
                Leipzig & 1 \\
              \end{supertabular}
            \end{msoraclesql}
          \end{small}
        \end{center}
        \item Erstellen Sie eine Abfrage, die f\"ur alle bankeigenen Kunden die
        Buchungen auf deren Girokonten z\"ahlt. Interessant sind nur Buchungen
        mit einem Betrag \textgreater 10.000 EUR. Sortieren Sie die Abfrage nach
        der Spalte \identifier{Konto\_ID}.
        \begin{center}
          \begin{small}
            \changefont{pcr}{m}{n}
            \tablefirsthead {
              \multicolumn{1}{r}{\textbf{KONTO\_ID}} &
              \multicolumn{1}{r}{\textbf{COUNT(*)}} \\
              \cmidrule(r){1-1}\cmidrule(r){2-2}
            }
            \tablehead{}
            \tabletail {
              \multicolumn{2}{l}{\textbf{367 Zeilen ausgew\"ahlt}} \\
            }
            \tablelasttail {
              \multicolumn{2}{l}{\textbf{367 Zeilen ausgew\"ahlt}} \\
            }
            \begin{msoraclesql}
              \begin{supertabular}{rr}
                1 & 8 \\
                2 & 11 \\
                3 & 10 \\
                5 & 7 \\
                6 & 8 \\
                7 & 9 \\
                9 & 8 \\
              \end{supertabular}
            \end{msoraclesql}
          \end{small}
        \end{center}
        \item Schreiben Sie eine Abfrage, die alle Mitarbeiter anzeigt, deren
        Gehalt um mehr als 4.000 EUR niedriger ist, als das Durchschnittsgehalt
        aller Mitarbeiter.
        \begin{center}
          \begin{small}
            \changefont{pcr}{m}{n}
            \tablefirsthead {
              \multicolumn{1}{l}{\textbf{VORNAME}} &
              \multicolumn{1}{l}{\textbf{NACHNAME}} &
              \multicolumn{1}{r}{\textbf{GEHALT}} \\
              \cmidrule(l){1-1}\cmidrule(l){2-2}\cmidrule(r){3-3}
            }
            \tablehead{}
            \tabletail {
              \multicolumn{4}{l}{\textbf{59 Zeilen ausgew\"ahlt}} \\
            }
            \tablelasttail {
              \multicolumn{4}{l}{\textbf{59 Zeilen ausgew\"ahlt}} \\
            }
            \begin{msoraclesql}
              \begin{supertabular}{llrr}
                Louis & Wagner & 1500 \\
                Lukas & Wei\ss{} & 2000 \\
                Maja & Keller & 1000 \\
                Karolin & Klingner & 2000 \\
                Elias & Sindermann & 1000 \\
              \end{supertabular}
            \end{msoraclesql}
          \end{small}
        \end{center}
\clearpage
        \item Schreiben Sie eine Abfrage, die alle Mitarbeiter anzeigt, die
        h\"ochstens zwei Jahre \"alter sind, als der j\"ungste Mitarbeiter in
        deren Bankfiliale! 
        \begin{center}
          \begin{small}
            \changefont{pcr}{m}{n}
            \tablefirsthead {
              \multicolumn{1}{l}{\textbf{VORNAME}} &
              \multicolumn{1}{l}{\textbf{NACHNAME}} &
              \multicolumn{1}{l}{\textbf{GEBURTSDATUM}} &
              \multicolumn{1}{l}{\textbf{Juengster Mitarbeiter}} \\
              \cmidrule(l){1-1}\cmidrule(l){2-2}\cmidrule(l){3-3}\cmidrule(l){4-4}
            }
            \tablehead{}
            \tabletail {
              \multicolumn{4}{l}{\textbf{32 Zeilen ausgew\"ahlt}} \\
            }
            \tablelasttail {
              \multicolumn{4}{l}{\textbf{32 Zeilen ausgew\"ahlt}} \\
            }
            \begin{msoraclesql}
              \begin{supertabular}{llll}
                Johannes & Lehmann & 1992-11-07 & 1992-11-07 \\
                Dirk & Peters & 1991-09-16 & 1992-11-07 \\
                Stefan & Beck & 1983-12-21 & 1984-11-16 \\
                Martin & Schacke & 1984-11-16 & 1984-11-16 \\
                Lukas & Wei\ss{} & 1989-03-23 & 1989-03-23 \\
                Alexander & Weber & 1987-11-05 & 1989-03-23 \\
                Anne & Zimmermann & 1991-01-28 & 1991-01-28 \\
              \end{supertabular}
            \end{msoraclesql}
          \end{small}
        \end{center}
        \item Schreiben Sie eine Abfrage, die zu jedem Filialleiter, das Gehalt
        seines am schlechtesten bezahlten Mitarbeiters anzeigt. Sortieren Sie
        die Abfrage nach den Bankfilial-IDs der Filialleiter.
        \begin{center}
          \begin{small}
            \changefont{pcr}{m}{n}
            \tablefirsthead {
              \multicolumn{1}{l}{\textbf{VORNAME}} &
              \multicolumn{1}{l}{\textbf{NACHNAME}} &
              \multicolumn{1}{r}{\textbf{GEHALT}} &
              \multicolumn{1}{r}{\textbf{Kleinstes Gehalt}} \\
              \cmidrule(l){1-1}\cmidrule(l){2-2}\cmidrule(r){3-3}\cmidrule(r){4-4}
            }
            \tablehead{}
            \tabletail {
            }
            \tablelasttail {
              \multicolumn{4}{l}{\textbf{20 Zeilen ausgew\"ahlt}} \\
            }
            \begin{msoraclesql}
              \begin{supertabular}{llrr}
                Dirk & Peters & 12000 & 2000 \\
                Louis & Winter & 12000 & 1000 \\
                Alexander & Weber & 12000 & 1000 \\
                Sophie & Schwarz & 12000 & 1500 \\
                Jessica & Weber & 12000 & 1000 \\
              \end{supertabular}
            \end{msoraclesql}
          \end{small}
        \end{center}
      \end{enumerate}
    \input{../sql/loesungen/sql_05_gruppenfunktionen_loesungen}
    \chapter{Unterabfragen (Subqueries)}
  \label{subqueries}
    \setcounter{page}{1}\kapitelnummer{chapter}
    \minitoc
\newpage
    \section{Grunds\"atzliches zu Unterabfragen}
      \subsection{Was sind Unterabfragen?}\label{whataresubqueries}
        Unterabfragen sind Abfragen, die in eine andere Abfrage, die Hauptabfrage oder \enquote{Mainquery}, eingebettet werden. Dies kann an mehreren Stellen geschehen.
        \begin{itemize}
          \item \SELECT-Klausel
          \item \FROM-Klausel (Inlineview)
          \item \WHERE-Klausel
          \item \HAVING-Klausel
        \end{itemize}
        \bild{Unterabfragen}{abfrage}{0.8}
        F\"ur Unterabfragen gibt es die unterschiedlichsten Bezeichnungen.
        \begin{itemize}
          \item Subquery
          \item Inner query
          \item Nested query
        \end{itemize}
      \subsection{Wann sind Unterabfragen notwendig?}
        Mit Hilfe von SQL k\"onnen zwei verschiedene Arten von Problemstellungen gel\"ost werden:
        \begin{itemize}
          \item Einschrittige Problemstellungen
          \item Mehrschrittige Problemstellungen
        \end{itemize}
        Unter einer einschrittigen Problemstellung versteht man die Art von Fragestellung, die mit einer einzigen Abfrage (einem einzigen Arbeitsschritt) gel\"ost werden kann, so wie dies in den vorangegangenen Kapiteln der Fall war.

        Mehrschrittige Problemstellungen erfordern, wie der Name es sagt, mehrere Abfragen, die aufeinander aufbauen (die eine Abfrage ben\"otigt das Ergebnis der anderen), um zu einer L\"osung zu kommen. Eine solche Problemstellung k\"onnte z. B. so lauten: \enquote{Wie hoch ist das Gehalt des Vorgesetzten der Mitarbeiterin \textit{Lena Gro\ss{}e}?}
\clearpage
        Diese Frage l\"asst sich in zwei Fragen teilen:
        \begin{enumerate}
          \item Wer ist der Vorgesetzte von Lena Gro\ss{}e?
          \item Wie hoch ist dessen Gehalt?
        \end{enumerate}
        Die Antworten zu beiden Fragen lassen sich sehr einfach als SQL-Statements formulieren.
        \begin{lstlisting}[language=oracle_sql,caption={Wer ist der Vorgesetzte von Lena Gro\ss{}e},label=sql06_01]
SELECT Vorgesetzter_ID
FROM   Mitarbeiter
WHERE  Vorname LIKE 'Lena' AND  Nachname LIKE 'Große';
        \end{lstlisting}
        \begin{center}
          \begin{small}
            \changefont{pcr}{m}{n}
            \tablefirsthead {
              \multicolumn{1}{r}{\textbf{VORGESETZTER\_ID}} \\
              \cmidrule(r){1-1}
            }
            \tablehead{}
            \tabletail {
              \multicolumn{1}{l}{\textbf{1 Zeile ausgew\"ahlt}} \\
            }
            \tablelasttail {
              \multicolumn{1}{l}{\textbf{1 Zeile ausgew\"ahlt}} \\
            }
            \begin{msoraclesql}
              \begin{supertabular}{r}
                6 \\
              \end{supertabular}
            \end{msoraclesql}
          \end{small}
        \end{center}
        \begin{lstlisting}[language=oracle_sql,caption={Wie hoch ist dessen Gehalt},label=sql06_02]
SELECT Gehalt
FROM   Mitarbeiter
WHERE  Mitarbeiter_ID = 6;
        \end{lstlisting}
        \begin{center}
          \begin{small}
            \changefont{pcr}{m}{n}
            \tablefirsthead {
              \multicolumn{1}{r}{\textbf{GEHALT}} \\
              \cmidrule(r){1-1}
            }
            \tablehead{}
            \tabletail {
              \multicolumn{1}{l}{\textbf{1 Zeile ausgew\"ahlt}} \\
            }
            \tablelasttail {
              \multicolumn{1}{l}{\textbf{1 Zeile ausgew\"ahlt}} \\
            }
            \begin{msoraclesql}
              \begin{supertabular}{r}
                30000 \\
              \end{supertabular}
            \end{msoraclesql}
          \end{small}
        \end{center}
        Mit Hilfe der beiden Abfragen wurde die Antwort ermittelt: \enquote{Der Vorgesetzte von Lena Gro\ss{}e hat ein Gehalt von 30.000 EUR.}. Durch das Kombinieren beider Queries, l\"asst sich diese Aufgabe viel eleganter l\"osen. \beispiel{sql06_03} zeigt einen m\"oglichen L\"osungsansatz.
        \begin{lstlisting}[language=oracle_sql,caption={Wie hoch ist das Gehalt des Vorgesetzten der Mitarbeiterin \textit{Lena Gro\ss{}e}?},label=sql06_03]
SELECT Gehalt
FROM   Mitarbeiter
WHERE  Mitarbeiter_ID = (SELECT Vorgesetzter_ID
                         FROM   Mitarbeiter
                         WHERE  Vorname LIKE 'Lena'
                           AND  Nachname LIKE 'Grosse');
        \end{lstlisting}
        \begin{center}
          \begin{small}
            \changefont{pcr}{m}{n}
            \tablefirsthead {
              \multicolumn{1}{r}{\textbf{GEHALT}} \\
              \cmidrule(r){1-1}
            }
            \tablehead{}
            \tabletail {
              \multicolumn{1}{l}{\textbf{1 Zeile ausgew\"ahlt}} \\
            }
            \tablelasttail {
              \multicolumn{1}{l}{\textbf{1 Zeile ausgew\"ahlt}} \\
            }
            \begin{msoraclesql}
              \begin{supertabular}{r}
                30000 \\
              \end{supertabular}
            \end{msoraclesql}
          \end{small}
        \end{center}
\clearpage
        \begin{merke}
          Das DBMS arbeitet bei einer solchen Auswahlabfrage immer zuerst die Unterabfrage(n) ab!
        \end{merke}
      \subsection{Regeln f\"ur Unterabfragen}
        F\"ur die Anwendung von Unterabfragen gelten die folgenden Grunds\"atze:
        \begin{itemize}
          \item Unterabfragen stehen immer in Klammern!
          \item Es k\"onnen alle ihnen bisher bekannten Operatoren eingesetzt werden!
          \item Unterabfragen \textbf{sollten immer ohne \ORDERBY-Klausel} erstellt werden!
        \end{itemize}
        Die Aussage, dass Unterabfragen immer ohne \ORDERBY{} verwendet werden sollten, r\"uhrt daher, dass falls eine Sortierung in der Hauptabfrage stattfindet, zuerst in der Unterabfrage sortiert wird und anschlie\ss end nochmals in der Hauptabfrage. Dies f\"uhrt zu unn\"otiger Sortierarbeit, die die Datenbank belastet.
        \begin{merke}
          In MS SQL Server darf eine Unterabfrage kein \ORDERBY{} enthalten. Das DBMS antwort sonst mit einer Fehlermeldung (Meldung 1033, Ebene 15).
        \end{merke}
      \subsection{Arten von Unterabfragen}
        Grunds\"atzlich gibt es vier unterschiedliche Arten von Unterabfragen:
        \begin{itemize}
          \item Skalare Unterabfragen: Eine solche Abfrage liefert exakt einen Wert zur\"uck.
          \item Einspaltige Unterabfragen: Dieser Abfragetyp liefert mehrere Werte aus einer Spalte zur\"uck.
          \item Mehrspaltige Unterabfragen: Mit einer solchen Abfrage werden Werte mehrerer Spalten zur\"uckgeliefert.
          \item Korrelierte Unterabfragen: Ihre Ausf\"uhrung ist von der Hauptabfrage abh\"angig.
        \end{itemize}
\clearpage
    \section{Skalare Unterabfragen (Scalar Subqueries)}
      \begin{merke}
        \textbf{Skalar:}
        \vspace{1em}

        Gr\"o\ss{}e aus der Mathematik, die durch die Angabe eines einzelnen Wertes genau definiert werden kann.
        \vspace{1em}

        Beispiele f\"ur Skalare sind: Gehalt, Provision, \dots
      \end{merke}
      Skalare Unterabfragen zeichnen sich dadurch aus, dass sie genau einen einzigen Wert zur\"uckliefern.   Dies wird mit Hilfe einer entsprechenden \WHERE-Klausel innerhalb der Unterabfrage erreicht. Ergibt die Abfrage kein Ergebnis, wird NULL zur\"uckgliefert. Ein erstes Beispiel f\"ur diese Art von Unterabfrage war in \beispiel{sql06_03} zu sehen. Es wird nur das \identifier{Gehalt} eines einzigen Angestellten angezeigt.
      \subsection{Wo k\"onnen skalare Unterabfragen stehen?}
        Skalare Unterabfragen k\"onnen in allen in \abschnitt{whataresubqueries} erw\"ahnten Klauseln stehen.
        \subsubsection{Skalare Unterabfragen in der SELECT-Klausel}
          Skalare Unterabfragen sind die einzigen, die in der \SELECT-Klausel eines SQL-Statements stehen d\"urfen. Sie k\"onnen beispielsweise dazu dienen, um einen Outer-Join zu vermeiden, meist ist jedoch die Join-Variante sehr viel performanter. Aus diesem Grund sollten skalare Unterabfragen in der \SELECT-Klausel absolut vermieden werden.
          \begin{merke}
            Skalare Unterabfragen in der \SELECT-Klausel sollten unter allen Umst\"anden vermieden werden!
          \end{merke}
        \subsubsection{Skalare Unterabfragen in der WHERE-Klausel}
          Die \WHERE-Klausel ist der Ort, an dem skalare Unterabfragen am h\"aufigsten anzutreffen sind. Sie dienen zur Berechnung von Werten, mit deren Hilfe das Resultat der Hauptabfrage eingeschr\"ankt wird (siehe \beispiel{sql06_03}).
        \subsubsection{Skalare Unterabfragen in der Having-Klausel}
        Hier gelten die gleichen Grunds\"atze, wie in der \WHERE-Klausel. Der einzige Unterschied ist, das hier ein Aggregat mit dem Resultat einer skalaren Unterabfrage verglichen werden kann.
      \subsection{Fehlerquellen in skalaren Unterabfragen}
        Die h\"aufigste Fehlerquelle, im Umgang mit skalaren Unterabfragen, ist eine falsche \WHERE-Klausel. Schr\"ankt sie das Ergebnis der Unterabfrage nicht gen\"ugend ein, wird mehr als ein Datensatz/Wert zur\"uckgeliefert und die Datenbank antwortet mit einer Fehlermeldung.
          \begin{lstlisting}[language=oracle_sql,caption={Mehr als eine Zeile: Fehlermeldung in Oracle},label=sql06_04]
ORA-01427: Unterabfrage fuer eine Zeile liefert mehr als eine Zeile
          \end{lstlisting}
          \begin{lstlisting}[language=ms_sql,caption={Mehr als eine Zeile: Fehlermeldung in SQL Server},label=sql06_05]
Meldung 512, Ebene 16, Status 1, Zeile 1
Die Unterabfrage hat mehr als einen Wert zurueckgegeben. Das ist nicht
zulaessig, wenn die Unterabfrage auf =, !=, <, <=, > oder >= folgt oder
als Ausdruck verwendet wird.
          \end{lstlisting}
          Hier ein Beispiel zu diesen Fehlermeldungen: Es soll das Geburtsdatum des Vorgesetzten der Mitarbeiterin \enquote{Gro\ss{}e} ermittelt werden.
          \begin{lstlisting}[language=oracle_sql,caption={Eine Single Row Unterabfrage mit Problemen!},label=sql06_06]
SELECT Geburtsdatum
FROM   Mitarbeiter
WHERE  Mitarbeiter_ID = (SELECT Vorgesetzter_ID
                         FROM   Mitarbeiter
                         WHERE  LOWER(Nachname) LIKE 'große');
          \end{lstlisting}
          Das Problem bei dieser Abfrage ist, dass die Tabelle \identifier{Mitarbeiter} zwei Angestellte mit dem Namen \enquote{Gro\ss{}e} enth\"alt. Das bedeutet, die Unterabfrage liefert mehr als einen Wert zur\"uck, so dass der Vergleich mit einem Single Row Operator scheitert.
          \begin{merke}
            Als Single Row Operatoren werden relationale Operatoren bezeichnet, die einen Wert auf ihrer linken Seite mit genau einem Wert auf ihrer rechten Seite vergleichen k\"onnen. Hierzu z\"ahlen: = >= <= < > != LIKE
          \end{merke}
    \section{Einspaltige Unterabfragen}
      Diese Kategorie der Unterabfragen unterschiedet sich von den skalaren dahingehend, dass sie eine einspaltige Liste von mehreren Werten (Vektor) zur\"uckliefern und das sie nicht in der \SELECT-Klausel eines SQL-Statements vorkommen d\"urfen.
      \subsection{Einspaltige Unterabfragen in WHERE- und HAVING-Klausel}
        \languageorasql{IN} (bekannt aus \abschnitt{relopersql}) ist der einzige Operator, der auf seiner rechten Seite nicht nur einen einzelnen Wert, sondern eine ganze Wertemenge verarbeiten kann. Dies kann eine konstante Menge sein, so wie dies bisher der Fall war, aber es kann auch eine, durch eine Query dynamisch generierte Menge sein. \beispiel{sql06_07} zeigt den Einsatz des \languageorasql{IN}-Operators. Es muss eine Liste aller Kunden ermittelt werden, die vor dem \enquote{01.01.1980} ein Konto bei der Bank er\"offnet haben.
        \begin{lstlisting}[language=oracle_sql,caption={\languageorasql{IN} mit Unterabfrage},label=sql06_07]
SELECT Vorname, Nachname
FROM   Kunde
WHERE  Kunden_ID IN (SELECT Kunden_ID
                     FROM   EigenkundeKonto
                     WHERE  Eroeffnungsdatum < TO_DATE('01.01.1980'));
        \end{lstlisting}
        \begin{center}
          \begin{small}
            \changefont{pcr}{m}{n}
            \tablefirsthead {
              \multicolumn{1}{l}{\textbf{VORNAME}} &
              \multicolumn{1}{l}{\textbf{NACHNAME}} \\
              \cmidrule(l){1-1}\cmidrule(l){2-2}
            }
            \tablehead{}
            \tabletail {
              \multicolumn{2}{l}{\textbf{28 Zeilen ausgew\"ahlt}} \\
            }
            \tablelasttail {
              \multicolumn{2}{l}{\textbf{28 Zeilen ausgew\"ahlt}} \\
            }
            \begin{msoraclesql}
              \begin{supertabular}{ll}
									Jan & Wei\ss{} \\
									Petra & Berger \\
									Karolin & Lange \\
									Tom & Hartmann \\
								\end{supertabular}
            \end{msoraclesql}
          \end{small}
        \end{center}
        Auf die gleiche Art und Weise, wie in \beispiel{sql06_07} gezeigt, k\"onnen einspaltige Unterabfragen auch in einer \HAVING-Klausel eingesetzt werden, was jedoch nur sehr selten vorkommt.
      \subsection{Existenzpr\"ufungen}
				\subsubsection{Der EXISTS-Operator}
          Der Name \textit{Existenzpr\"ufung} sagt ohne Umschweife aus, worum es geht. Mit Hilfe des Operators \languageorasql{EXISTS} kann die Existenz bestimmter Daten gepr\"uft werden. \beispiel{sql06_08} zeigt auf worum es sich hierbei handelt. Es soll eine Liste der Bankfilialen ermittelt werden, in denen Mitarbeiter eingesetzt sind.
          \begin{lstlisting}[language=oracle_sql,caption={Der \languageorasql{EXISTS}-Operator},label=sql06_08]
SELECT Strasse, Hausnummer, PLZ, Ort
FROM   Bankfiliale b
WHERE  EXISTS (SELECT 1
               FROM   Mitarbeiter m
               WHERE  b.Bankfiliale_ID = m.Bankfiliale_ID);
          \end{lstlisting}
          \begin{center}
            \begin{small}
              \changefont{pcr}{m}{n}
              \tablefirsthead {
                \multicolumn{1}{l}{\textbf{STRASSE}} &
                \multicolumn{1}{l}{\textbf{HAUSNUMMER}} &
                \multicolumn{1}{l}{\textbf{PLZ}} &
                \multicolumn{1}{l}{\textbf{ORT}} \\
                \cmidrule(l){1-1}\cmidrule(l){2-2}\cmidrule(l){3-3}\cmidrule(l){4-4}
              }
              \tablehead{}
              \tabletail {
              }
              \tablelasttail {
                \multicolumn{4}{l}{\textbf{20 Zeilen ausgew\"ahlt}} \\
              }
              \begin{msoraclesql}
                \begin{supertabular}{llll}
                  Poststra\ss{}e & 1 & 06449 & Aschersleben \\
                  Markt & 5 & 06449 & Aschersleben \\
                  Goethestra\ss{}e & 4 & 39240 & Calbe \\
                  Lessingstra\ss{}e & 1 & 06406 & Bernburg \\
                  Schillerstra\ss{}e & 7 & 39240 & Barby \\
                \end{supertabular}
              \end{msoraclesql}
            \end{small}
          \end{center}
          Das Ergebnis dieser Auswahlabfrage sind alle Bankfilialen, in denen Mitarbeiter arbeiten. Es verbleibt eine Filiale ohne Mitarbeiter.

          Was geschieht in dieser Abfrage nun in welcher Reihenfolge?
          \begin{enumerate}
            \item Die \FROM-Klausel der Hauptabfrage wird ausgewehrtet und die erforderlichen Daten werden ermittelt.
            \item Die \FROM-Klausel der Unterabfrage wird ausgewehrtet und die erforderlichen Daten werden ermittelt.
            \item Die \WHERE-Klausel der Unterabfrage wird ausgef\"uhrt. Der Join zwischen \identifier{Bankfiliale} und \identifier{Mitarbeiter} wird gebildet.
            \item Die \WHERE-Klausel der Hauptabfrage wird ausgef\"uhrt.
            \item Die \SELECT-Klausel der Hauptabfrage liefert die ben\"otigten Daten.
          \end{enumerate}
          Das Besondere an dieser Form der Abfrage ist die \WHERE-Klausel der Unterabfrage. Dort wird die Tabelle \identifier{Bankfiliale} (Hauptabfrage) mit der Tabelle \identifier{Mitarbeiter} (Unterabfrage) verkn\"upft. Die Unterabfrage kann somit auf die Datens\"atze der Hauptabfrage zugreifen.
          \begin{merke}
            Werden die Tabellen einer Unterabfrage mit einer Tabelle der Hauptabfrage verkn\"upft, spricht man von einer \enquote{korrelierten Unterabfrage}.
          \end{merke}
\clearpage
          F\"ur die Ausf\"uhrung des gesamten Statements bedeutet dies, das die Unterabfrage nicht nur einmal, sondern mehrfach ausgef\"uhrt werden muss. Genauer gesagt wird die Unterabfrage f\"ur jede Zeile der Hauptabfrage einmal ausgef\"uhrt. Bezogen auf \beispiel{sql06_08} bedeutet dies, dass die Unterabfrage 21 mal ausgef\"uhrt wird, da die Tabelle Bankfiliale 21 Datens\"atze hat. Die Mehrfachausf\"uhrung der Unterabfrage ist notwendig, da f\"ur jede Bankfiliale einzeln gepr\"uft werden muss, ob es dort Mitarbeiter gibt oder nicht.

          Eine weitere Besonderheit dieser Art von Abfrage ist die \SELECT-Klausel der Unterabfrage. Dort stehen keine Spaltenbezeichner und auch kein *. Statt dessen wird hier ein Literal, eine 1 (eins) verwendet. Der Hintergrund hierf\"ur ist, das die \SELECT-Klausel der Unterabfrage f\"ur die Ausf\"uhrung des gesamten Statements keine Bedeutung hat. Es wird nur gepr\"uft, ob f\"ur jeden Datensatz der Hauptabfrage ein Datensatz in der Unterabfrage existiert. Das bedeutet, dass sobald die Unterabfrage eine Zeile zur\"uckliefert die Bedingung erf\"ullt ist und der Datensatz der Hauptabfrage angezeigt wird.
        \subsubsection{Der NOT EXISTS-Operator}
          Der \languageorasql{NOT EXISTS}-Operator stellt das Pendant zum \languageorasql{EXISTS}-Operator dar. M\"ussen beispielsweise alle Filialen ermittelt werden, in denen keine Mitarbeiter arbeiten kommt \languageorasql{NOT EXISTS} zum Einsatz.
          \begin{lstlisting}[language=oracle_sql,caption={Der \languageorasql{NOT EXISTS}-Operator},label=sql06_09]
SELECT Strasse, Hausnummer, PLZ, Ort
FROM   Bankfiliale b
WHERE  NOT EXISTS (SELECT 1
                   FROM   Mitarbeiter m
                   WHERE  b.Bankfiliale_ID = m.Bankfiliale_ID);
          \end{lstlisting}
          \begin{center}
            \begin{small}
              \changefont{pcr}{m}{n}
              \tablefirsthead {
                \multicolumn{1}{l}{\textbf{STRASSE}} &
                \multicolumn{1}{l}{\textbf{HAUSNUMMER}} &
                \multicolumn{1}{l}{\textbf{PLZ}} &
                \multicolumn{1}{l}{\textbf{ORT}} \\
                \cmidrule(l){1-1}\cmidrule(l){2-2}\cmidrule(l){3-3}\cmidrule(l){4-4}
              }
              \tablehead{}
              \tabletail {
                \multicolumn{4}{l}{\textbf{1 Zeile ausgew\"ahlt}} \\
              }
              \tablelasttail {
                \multicolumn{4}{l}{\textbf{1 Zeile ausgew\"ahlt}} \\
              }

              \begin{msoraclesql}
                \begin{supertabular}{llll}
                  Kurze Gasse & 47 & 06425 & Alsleben \\
                \end{supertabular}
              \end{msoraclesql}
            \end{small}
          \end{center}
    \section{Inlineviews / Derived Tables}
      In \abschnitt{whataresubqueries} wurde bereits erw\"ahnt, dass eine Unterabfrage auch in der \FROM-Klausel eines SQL-Statements stehen kann.
      \begin{merke}
        Eine Unterabfrage in der \FROM-Klausel wird in Oracle als \enquote{Inlineview} und in MS SQL Server als \enquote{Derived Table} bezeichnet.
      \end{merke}
      \beispiel{sql06_10} zeigt ein SQL-Statement, welches eine Inlineview nutzt.
      \begin{lstlisting}[language=oracle_sql,caption={Eine Inlineview},label=sql06_10]
SELECT Vorname, Nachname, MinGehalt
FROM   (SELECT   Bankfiliale_ID, MIN(Gehalt) MinGehalt
        FROM     Mitarbeiter
        GROUP BY Bankfiliale_ID) m1
        INNER JOIN Mitarbeiter m
        ON (m1.Bankfiliale_ID = m.Bankfiliale_ID)
WHERE   m.Gehalt = m1.MinGehalt;
      \end{lstlisting}
      \begin{center}
        \begin{small}
          \changefont{pcr}{m}{n}
          \tablefirsthead {
            \multicolumn{1}{l}{\textbf{VORNAME}} &
            \multicolumn{1}{l}{\textbf{NACHNAME}} &
            \multicolumn{1}{r}{\textbf{MINGEHALT}} \\
            \cmidrule(l){1-1}\cmidrule(l){2-2}\cmidrule(r){3-3}
          }
          \tablehead{}
          \tabletail {
          }
          \tablelasttail {
            \multicolumn{3}{l}{\textbf{29 Zeilen ausgew\"ahlt}} \\
          }
          \begin{msoraclesql}
            \begin{supertabular}{llr}
              Johannes & Lehmann & 2000 \\
              Louis & Schmitz & 2000 \\
              Marie & Kipp & 2000 \\
              Martin & Schacke & 1000 \\
              Oliver & Wolf & 1000 \\
              Hans & Schumacher & 1000 \\
              Lena & Herrmann & 1500 \\
            \end{supertabular}
          \end{msoraclesql}
        \end{small}
      \end{center}
      In \beispiel{sql06_10} wird die Inlineview dazu benutzt, um das kleinste Gehalt je Abteilung zu berechnen. Mit Hilfe des Joins wird sie mit der Tabelle \identifier{Mitarbeiter} verkn\"upft, so dass die Attribute \identifier{Vorname} und \identifier{Nachname} angezeigt werden k\"onnen, ohne in Konflikt mit der \GROUPBY-Klausel zu kommen.
      \begin{merke}
        Inlineviews bieten eine gute M\"oglichkeit, um gruppierte und ungruppierte Informationen in einer Abfrage gemeinsam anzeigen zu k\"onnen.
      \end{merke}
    \section{Top N Analysen}
      Die Top N Analyse ist ein Verfahren, bei dem Datens\"atze in ein Ranking eingeordnet werden. Hiermit werden Fragestellungen gekl\"art wie z. B.:
      \begin{itemize}
        \item Die 3 reichsten Kunden anzeigen
        \item Die 5 Mitarbeiter mit den h\"ochsten Geh\"altern auflisten
        \item Die beiden gr\"o\ss{}ten Schuldner der Bank ermitteln
      \end{itemize}
      Beide Datenbankmanagementsysteme beherrschen diese Technik, gehen dabei aber unterschiedliche Wege.
      \subsection{Die Top N Analyse in Oracle}
        Die Top N Analyse funktioniert in Oracle mit Hilfe einer sortierten Inlineview und einer Pseudospalte Namens \identifier{Rownum}.
        \subsubsection{Die Pseudospalte Rownum}
          Mit der Bezeichnung \enquote{Pseudospalte} ist gemeint, dass die \identifier{Rownum} keine tats\"achlich vorhandene Spalte ist, obwohl sie in jeder Abfrage verwendet werden kann. Sie bietet die M\"oglichkeit, die Ergebniszeilen einer Abfrage fortlaufend zu nummerieren (1, 2, 3, \dots, N). Zu beachten ist dabei, dass eine Zeile in einer Oracle-Datenbank keine feste Nummerierung hat. Diese wird erst im Ergebnis einer Abfrage zugeordnet.
          \begin{lstlisting}[language=oracle_sql,caption={Ein einfaches Beispiel f\"ur die Rownum},label=sql06_11]
SELECT   Rownum, Vorname, Nachname
FROM     Mitarbeiter
WHERE    Ort LIKE 'Aschersleben';
          \end{lstlisting}
          \begin{center}
            \begin{small}
              \changefont{pcr}{m}{n}
              \tablefirsthead {
                \multicolumn{1}{r}{\textbf{ROWNUM}} &
                \multicolumn{1}{l}{\textbf{VORNAME}} &
                \multicolumn{1}{l}{\textbf{NACHNAME}} \\
                \cmidrule(r){1-1}\cmidrule(l){2-2}\cmidrule(l){3-3}
              }
              \tablehead{}
              \tabletail {
              }
              \tablelasttail {
                \multicolumn{3}{l}{\textbf{3 Zeilen ausgew\"ahlt}} \\
              }

              \begin{oraclesql}
                \begin{supertabular}{rll}
                  1 & Max & Winter \\
                  2 & Alexander & Weber \\
                  3 & Leni & D\"uhning \\
                \end{supertabular}
              \end{oraclesql}
            \end{small}
          \end{center}
          \begin{merke}
            Eine Tabellenzeile hat keine feste Nummerierung. Die Rownum wird w\"ahrend der Abarbeitung einer Abfrage zugewiesen.
          \end{merke}
          Eine weitere, entscheidende Tatsache ist, dass die \identifier{Rownum} erst nach der Abarbeitung der \WHERE-Klausel zugeordnet wird, aber noch bevor Gruppierungen oder Sortierungen ausgef\"uhrt werden. Aus diesem Grund, wird die Abfrage in \beispiel{sql06_12} ein falsches Ergebnis liefern, da die Sortierung h\"atte zuerst stattfinden m\"ussen. Hier werden h\"ochstwahrscheinlich nicht die beiden gr\"o\ss{}ten Guthaben, sondern zwei beliebige Guthaben angezeigt. Welche Zeilen gelistet werden h\"angt davon ab, welche die Abfrage zuerst ermittelt.
          \begin{lstlisting}[language=oracle_sql,caption={Falsche Anwendung der Rownum-Pseudospalte},label=sql06_12]
SELECT   Konto_ID, Guthaben
FROM     Girokonto
WHERE    Rownum < 3
ORDER BY Guthaben DESC;
          \end{lstlisting}
          \begin{center}
            \begin{small}
              \changefont{pcr}{m}{n}
              \tablefirsthead {
                \multicolumn{1}{r}{\textbf{KONTO\_ID}} &
                \multicolumn{1}{r}{\textbf{GUTHABEN}} \\
                \cmidrule(r){1-1}\cmidrule(r){2-2}
              }
              \tablehead{}
              \tabletail {
                \multicolumn{2}{l}{\textbf{2 Zeilen ausgew\"ahlt}} \\
              }
              \tablelasttail {
                \multicolumn{2}{l}{\textbf{2 Zeilen ausgew\"ahlt}} \\
              }

              \begin{oraclesql}
                \begin{supertabular}{rr}
                  1 & 111316,9 \\
                  2 & 96340,2 \\
                \end{supertabular}
              \end{oraclesql}
            \end{small}
          \end{center}
          \begin{merke}
            Die Rownum wird erst nach Abarbeitung der \WHERE-Klausel, aber noch vor allen Gruppierungen und Sortierungen hinzugef\"ugt.
          \end{merke}
          Ein dritter \enquote{Stolperstein}, in Zusammenhang mit der \identifier{Rownum} ist, dass die \identifier{Rownum} erst inkrementiert wird, wenn sie zugewiesen wurde. Das soll hei\ss{}en, dass die \WHERE-Klausel in \beispiel{sql06_13} ebenfalls fehlschl\"agt, da nach allen Rownums gr\"o\ss{}er eins gefragt wird, ohne das Rownum eins jemals zugewiesen worden w\"are (ohne 1 keine 2).
          \begin{lstlisting}[language=oracle_sql,caption={Erneut eine falsche Anwendung der Rownum},label=sql06_13]
SELECT   Konto_ID, Guthaben
FROM     Girokonto
WHERE    Rownum > 3
ORDER BY Guthaben DESC;
          \end{lstlisting}
          \begin{center}
            \begin{small}
              \changefont{pcr}{m}{n}
              \tablefirsthead {
                \multicolumn{1}{r}{\textbf{KONTO\_ID}} &
                \multicolumn{1}{r}{\textbf{GUTHABEN}} \\
                \cmidrule(r){1-1}\cmidrule(r){2-2}
              }
              \tablehead{}
              \tabletail {
                \multicolumn{2}{l}{\textbf{0 Zeilen ausgew\"ahlt}} \\
              }
              \tablelasttail {
                \multicolumn{2}{l}{\textbf{0 Zeilen ausgew\"ahlt}} \\
              }

              \begin{oraclesql}
                \begin{supertabular}{rr}

                \end{supertabular}
              \end{oraclesql}
            \end{small}
          \end{center}
          Die L\"osung f\"ur diese Probleme besteht nun darin,
          \begin{enumerate}
            \item dass niemals einer der beiden Operatoren \textgreater{} oder \textgreater = in Zusammenhang mit der \identifier{Rownum} verwendet werden sollte und
            \item dass die Abfrage aus \beispiel{sql06_12} in eine Inlineview geschachtelt wird.
          \end{enumerate}
        \subsubsection{Durchf\"uhrung der Top N Analyse}
          Die korrekte Form der Top N Analyse sieht in Oracle wie folgt aus:
          \begin{lstlisting}[language=oracle_sql,caption={Eine korrekt funktionierende Top N Analyse in Oracle},label=sql06_14]
SELECT *
FROM   (SELECT   Konto_ID, Guthaben
        FROM     Girokonto
        ORDER BY Guthaben DESC)
WHERE    Rownum < 3;
        \end{lstlisting}
\clearpage
        \begin{center}
          \begin{small}
            \changefont{pcr}{m}{n}
            \tablefirsthead {
              \multicolumn{1}{r}{\textbf{KONTO\_ID}} &
              \multicolumn{1}{r}{\textbf{GUTHABEN}} \\
              \cmidrule(r){1-1}\cmidrule(r){2-2}
            }
            \tablehead{}
            \tabletail {
              \multicolumn{2}{l}{\textbf{2 Zeilen ausgew\"ahlt}} \\
            }
            \tablelasttail {
              \multicolumn{2}{l}{\textbf{2 Zeilen ausgew\"ahlt}} \\
            }
            \begin{oraclesql}
              \begin{supertabular}{rr}
                362 & 147670,3 \\
                198 & 147264 \\
              \end{supertabular}
            \end{oraclesql}
          \end{small}
        \end{center}
        Im Gegensatz zu \beispiel{sql06_12} werden hier wirklich die beiden gr\"o\ss{}ten Geh\"alter angezeigt. Warum dies so ist, kann durch die Abarbeitungsreihenfolge der Abfrage aus \beispiel{sql06_14} erkl\"art werden.
        \begin{enumerate}
          \item \FROM-Klausel der Inlineview
          \item \SELECT- und \ORDERBY-Klausel der Inlineview
          \item \FROM-Klausel der Hauptabfrage
          \item Zuweisung der \identifier{Rownum}
          \item Ausf\"uhrung der \WHERE-Klausel der Hauptabfrage
          \item \SELECT-Klausel der Hauptabfrage
        \end{enumerate}
        In \beispiel{sql06_14} wird also zuerst nummeriert und dann selektiert.
      \subsection{Die Top N Analyse in MS SQL Server}
        In Microsoft SQL Server existiert eigens der Operator \languagemssql{TOP} zur Durchf\"uhrung von Top N Analysen. Er wird in der \SELECT-Klausel eingesetzt und legt fest, wie viele Zeilen angezeigt werden.
        \begin{lstlisting}[language=ms_sql,caption={Top N Analyse in MS SQL Server},label=sql06_15]
SELECT   TOP (2) Konto_ID, Guthaben
FROM     Girokonto
ORDER BY Guthaben DESC;
        \end{lstlisting}
        \begin{center}
          \begin{small}
            \changefont{pcr}{m}{n}
            \tablefirsthead {
              \multicolumn{1}{l}{\textbf{KONTO\_ID}} &
              \multicolumn{1}{l}{\textbf{GUTHABEN}} \\
              \cmidrule(l){1-1}\cmidrule(l){2-2}
            }
            \tablehead{}
            \tabletail {
              \multicolumn{2}{l}{\textbf{2 Zeilen ausgew\"ahlt}} \\
            }
            \tablelasttail {
              \multicolumn{2}{l}{\textbf{2 Zeilen ausgew\"ahlt}} \\
            }
            \begin{mssql}
              \begin{supertabular}{ll}
                362 & 147670,3 \\
                198 & 147264 \\
              \end{supertabular}
            \end{mssql}
          \end{small}
        \end{center}
        Durch die Angabe von \languagemssql{TOP (2)} werden nur die ersten zwei Zeilen der Ergebnismenge angezeigt.
    \section{Pivot-Tabellen}
      Mit MS SQL Server 2005 bzw. Oracle 11g R1 wurden der \languageorasql{PIVOT} und der \languageorasql{UNPIVOT}-Operator eingef\"uhrt. Diese erm\"oglichen die einfache Erstellung von Pivottabellen.
      \begin{merke}
        In einer Pivottabelle werden Daten, die im Zeilenformat vorliegen, im Spaltenformat angezeigt oder umgekehrt. Das \enquote{Drehen} der Daten wird als \enquote{Pivoting} bezeichnet, woraus sich der Name f\"ur diese Tabellen ableitet.
      \end{merke}
      \begin{itemize}
        \item \textbf{PIVOT:} Dreht Daten die zeilenweise vorliegen so, dass eine spaltenweise Darstellung m\"oglich ist.
        \item \textbf{UNPIVOT:} Dreht Daten die spaltenweise vorliegen so, dass eine zeilenweise Darstellung m\"oglich ist.
      \end{itemize}
      \subsection{Der PIVOT-Operator (Oracle)}
        Die M\"oglichkeiten, die der \languageorasql{PIVOT}-Operator bietet, werden anhand des folgenden Beispiels verdeutlicht. F\"ur die Filialen 1 bis 3 sollen die jeweils kleinsten Geh\"alter angezeigt werden.
        \begin{lstlisting}[language=oracle_sql,caption={Die niedrigsten Geh\"alter in den Filialen 1 bis 3},label=sql06_16]
SELECT   Bankfiliale_ID, MIN(Gehalt)
FROM     Mitarbeiter
WHERE    Bankfiliale_ID IN (1, 2, 3)
GROUP BY Bankfiliale_ID;
        \end{lstlisting}
        \begin{center}
          \begin{small}
            \changefont{pcr}{m}{n}
            \tablefirsthead {
              \multicolumn{1}{r}{\textbf{BANKFILIALE\_ID}} &
              \multicolumn{1}{r}{\textbf{MIN(GEHALT)}} \\
              \cmidrule(r){1-1}\cmidrule(r){2-2}
            }
            \tablehead{}
            \tabletail {
            }
            \tablelasttail {
              \multicolumn{2}{l}{\textbf{3 Zeilen ausgew\"ahlt}} \\
            }
            \begin{oraclesql}
              \begin{supertabular}{rr}
                1 & 2000 \\
                2 & 1000 \\
                3 & 1000 \\
              \end{supertabular}
            \end{oraclesql}
          \end{small}
        \end{center}
        In \beispiel{sql06_16} werden die gew\"unschten Zahlen ermittelt. Die Darstellung der Geh\"alter erfolgt zeilenweise. Sollen die gleichen Zahlen spaltenweise dargestellt werden, wird der \languageorasql{PIVOT}-Operator ben\"otigt. \beispiel{sql06_17} zeigt dessen Einsatz.
        \begin{lstlisting}[language=oracle_sql,caption={Das Ergebnis als Pivottabelle},label=sql06_17]
SELECT *
FROM   (SELECT Gehalt, Bankfiliale_ID
        FROM   Mitarbeiter)
PIVOT  (MIN(Gehalt) AS Gehalt FOR Bankfiliale_ID IN (1, 2, 3));
        \end{lstlisting}
        \begin{center}
          \begin{small}
            \changefont{pcr}{m}{n}
            \tablefirsthead {
              \multicolumn{1}{r}{\textbf{1\_GEHALT}} &
              \multicolumn{1}{r}{\textbf{2\_GEHALT}} &
              \multicolumn{1}{r}{\textbf{3\_GEHALT}} \\
              \cmidrule(r){1-1}\cmidrule(r){2-2}\cmidrule(r){3-3}
            }
            \tablehead{}
            \tabletail {
              \multicolumn{3}{l}{\textbf{1 Zeile ausgew\"ahlt}} \\
            }
            \tablelasttail {
              \multicolumn{3}{l}{\textbf{1 Zeile ausgew\"ahlt}} \\
            }

            \begin{oraclesql}
              \begin{supertabular}{rrr}
                2000 & 1000 & 1000 \\
              \end{supertabular}
            \end{oraclesql}
          \end{small}
        \end{center}
        \subsubsection{Die Syntax des PIVOT-Operators}
          Da das SQL-Statement aus \beispiel{sql06_17} auf den ersten Blick sehr komplex wirkt, ist es notwendig, es an dieser Stelle im Detail zu betrachten.

          F\"ur die Ausf\"uhrung des Pivotings wird in \beispiel{sql06_17} eine Inlineview verwendet.
          \begin{lstlisting}[language=oracle_sql,caption={Die Inlineview},label=sql06_18]
(SELECT Gehalt, Bankfiliale_ID
 FROM   Mitarbeiter)
          \end{lstlisting}
          Diese Inlineview legt fest, welche Spalten im Endergebnis der Abfrage zu sehen sein werden. Sie kann beliebig komplex sein. Der \languageorasql{PIVOT}-Operator verarbeitet im zweiten Schritt die Spalten dieser View weiter.
          \begin{lstlisting}[language=oracle_sql,caption={Der \languageorasql{PIVOT}-Operator},label=sql06_19]
PIVOT  (MIN(Gehalt) AS Gehalt FOR Bankfiliale_ID IN (1, 2, 3));
          \end{lstlisting}
          Die Bedeutung dieses Operators ist:
          \begin{itemize}
            \item Gruppiere nach der Spalte \identifier{Bankfiliale\_ID}.
            \item Zeige \languageorasql{MIN(Gehalt)} f\"ur \languageorasql{Bankfiliale_ID = 1}.
            \item Zeige \languageorasql{MIN(Gehalt)} f\"ur \languageorasql{Bankfiliale_ID = 2}.
            \item Zeige \languageorasql{MIN(Gehalt)} f\"ur \languageorasql{Bankfiliale_ID = 3}.
            \item Benutzte den Alias \enquote{Gehalt} f\"ur den Ausdruck \languageorasql{MIN(Gehalt)}.
          \end{itemize}
        \subsubsection{Spaltenaliase in der FOR-Klausel}
          Die Spaltenbezeichnungen im Ergebnis von \beispiel{sql06_17} werden
          gebildet, in dem der Name der aggregierten Spalte (hier
          \identifier{Gehalt}) mit den Werten der \languageorasql{FOR}-Klausel
          kombiniert werden. Dadurch entstehen die Namen \identifier{1\_GEHALT},
          \identifier{2\_GEHALT} und \identifier{3\_GEHALT}. Auch an dieser
          Stelle sind Aliasnamen m\"oglich.
\clearpage
          \begin{lstlisting}[language=oracle_sql,caption={Die \languageorasql{FOR}-Klausel mit Aliasnamen},label=sql06_20]
SELECT *
FROM   (SELECT Gehalt, Bankfiliale_ID
        FROM   Mitarbeiter)
PIVOT  (MIN(Gehalt) AS Gehalt FOR Bankfiliale_ID
       IN (1 AS "Filiale 1", 2 AS "Filiale 2", 3 AS "Filiale 3"));
          \end{lstlisting}
          \begin{center}
            \begin{small}
              \changefont{pcr}{m}{n}
              \tablefirsthead {
                \multicolumn{1}{r}{\textbf{Filiale 1\_GEHALT}} &
                \multicolumn{1}{r}{\textbf{Filiale 2\_GEHALT}} &
                \multicolumn{1}{r}{\textbf{Filiale 3\_GEHALT}} \\
                \cmidrule(r){1-1}\cmidrule(r){2-2}\cmidrule(r){3-3}
              }
              \tablehead{}
              \tabletail {
                \multicolumn{3}{l}{\textbf{1 Zeile ausgew\"ahlt}} \\
              }
              \tablelasttail {
                \multicolumn{3}{l}{\textbf{1 Zeile ausgew\"ahlt}} \\
              }
              \begin{oraclesql}
                \begin{supertabular}{rrr}
                  2000 & 1000 & 1000 \\
                \end{supertabular}
              \end{oraclesql}
            \end{small}
          \end{center}
        \subsubsection{Zus\"atzliche Spalten zum Pivoting}
          In einer Pivot-Abfrage k\"onnen noch weitere Spalten enthalten sein, die nicht aggregiert oder in der \languageorasql{FOR}-Klausel genutzt werden. Diese Spalten werden als zus\"atzliche Gruppierungsmerkmale genutzt.
          \begin{merke}
            Oracle f\"uhrt eine implizite Gruppierung der Ergebnismenge durch. Diese basiert auf allen nicht gruppierten Spalten, inklusive der Spalten, die in der \languageorasql{FOR}-Klausel genutzt werden.
          \end{merke}
          In \beispiel{sql06_21} wird im ersten Schritt nach dem Geburtsjahr, von 1987 bis 1989 gruppiert. Da diese Spalte in der \languageorasql{FOR}-Klausel verwendet wird, wird diese Information in Spaltenform dargestellt.

          Die Spalte Ort hingegen, wird in Zeilenform angezeigt, da sie nicht in der \languageorasql{FOR}-Klausel angegeben wurde.
          \begin{merke}
            Ob eine Information in Spalten- oder Zeilenform dargestellt wird, h\"angt davon ab, ob die betreffende Spalte in der \languageorasql{FOR}-Klausel gelistet wurde oder nicht.
          \end{merke}
          \begin{lstlisting}[language=oracle_sql,caption={Zus\"atzliche Gruppierungen in einer Pivot-Abfrage},label=sql06_21]
SELECT *
FROM   (SELECT Gehalt, TO_CHAR(Geburtsdatum, 'YYYY') AS Geburtsdatum, Ort
        FROM   Mitarbeiter)
PIVOT  (MIN(Gehalt) AS Gehalt FOR Geburtsdatum IN ('1987', '1988', '1989'));
          \end{lstlisting}
\clearpage
          \begin{center}
            \begin{small}
              \changefont{pcr}{m}{n}
              \tablefirsthead {
                \multicolumn{1}{l}{\textbf{ORT}} &
                \multicolumn{1}{r}{\textbf{'1987'\_GEHALT}} &
                \multicolumn{1}{r}{\textbf{'1988'\_GEHALT}} &
                \multicolumn{1}{r}{\textbf{'1989'\_GEHALT}} \\
                \cmidrule(l){1-1}\cmidrule(r){2-2}\cmidrule(r){3-3}\cmidrule(r){4-4}
              }
              \tablehead{}
              \tabletail {
                \multicolumn{4}{l}{\textbf{25 Zeilen ausgew\"ahlt}} \\
              }
              \tablelasttail {
                \multicolumn{4}{l}{\textbf{25 Zeilen ausgew\"ahlt}} \\
              }
              \begin{oraclesql}
                \begin{supertabular}{lrrr}
                  Calbe &  &  &  \\
                  Pl\"otzkau & 2500 &  &  \\
                  Nienburg &  &  &  \\
                  Bernburg &  &  &  \\
                  Dresden &  &  &  \\
                  Hecklingen &  &  & 3000 \\
                  Borne &  &  & 30000 \\
                  Sch\"onebeck &  &  &  \\
                  Giersleben &  &  &  \\
                  Gera &  &  & 3500 \\
                  M\"unchen &  &  &  \\
                  Egeln &  &  &  \\
                  G\"usten &  &  &  \\
                  Seeland &  &  & 2500 \\
                  Ilberstedt &  &  &  \\
                  B\"ordeaue &  &  &  \\
                  Hamburg & 12000 &  &  \\
                  Alsleben &  &  &  \\
                  Schwerin &  &  &  \\
                  Dessau & 2500 &  &  \\
                  K\"onnern &  &  &  \\
                  Cottbus &  &  &  \\
                  Potsdam & 3500 & 2000 & 2000 \\
                  Aschersleben & 12000 & 88000 & 2000 \\
                  Magdeburg &  &  & 3000 \\
                \end{supertabular}
              \end{oraclesql}
            \end{small}
          \end{center}
          Die vorangegangenen Beispiele stellen nur einen Einstieg in das Thema \enquote{Pivottabellen} dar. Tats\"achlich ist der \languageorasql{PIVOT}-Operator noch weitaus m\"achtiger.
      \subsection{Der PIVOT-Operator (MS SQL Server)}
        Die M\"oglichkeiten, welche der \languageorasql{PIVOT}-Operator bietet, werden anhand des folgenden Beispiels verdeutlicht. F\"ur die Bankfilialen 1 bis 3 sollen die jeweils kleinsten Geh\"alter angezeigt werden.
        \begin{lstlisting}[language=ms_sql,caption={Die niedrigsten Geh\"alter in den Filialen 1 bis 3},label=sql06_22]
SELECT   Bankfiliale_ID, MIN(Gehalt)
FROM     Mitarbeiter
WHERE    Bankfiliale_ID IN (1, 2, 3)
GROUP BY Bankfiliale_ID;
        \end{lstlisting}
\clearpage
        \begin{center}
          \begin{small}
            \changefont{pcr}{m}{n}
            \tablefirsthead {
              \multicolumn{1}{l}{\textbf{Bankfiliale\_ID}} &
              \multicolumn{1}{l}{\textbf{(Kein Spaltenname)}} \\
              \cmidrule(l){1-1}\cmidrule(l){2-2}
            }
            \tablehead{}
            \tabletail {
              \multicolumn{2}{l}{\textbf{3 Zeilen ausgew\"ahlt}} \\
            }
            \tablelasttail {
              \multicolumn{2}{l}{\textbf{3 Zeilen ausgew\"ahlt}} \\
            }
            \begin{mssql}
              \begin{supertabular}{ll}
                1 & 2000 \\
                2 & 1000 \\
                3 & 1000 \\
              \end{supertabular}
            \end{mssql}
          \end{small}
        \end{center}
        In \beispiel{sql06_22} werden die gew\"unschten Zahlen ermittelt. Die Darstellung der Geh\"alter erfolgt zeilenweise. Sollen die gleichen Zahlen spaltenweise dargestellt werden, wird der \languageorasql{PIVOT}-Operator ben\"otigt. \beispiel{sql06_23} zeigt dessen Einsatz.
        \begin{lstlisting}[language=ms_sql,caption={Das Ergebnis als Pivottabelle},label=sql06_23]
SELECT *
FROM   (SELECT Gehalt, Bankfiliale_ID
        FROM   Mitarbeiter) AS Sourcetable
PIVOT  (MIN(Gehalt)
        FOR Bankfiliale_ID IN ([1], [2], [3])
        ) AS Pivottable;
        \end{lstlisting}
        \begin{center}
          \begin{small}
            \changefont{pcr}{m}{n}
            \tablefirsthead {
              \multicolumn{1}{l}{\textbf{1}} &
              \multicolumn{1}{l}{\textbf{2}} &
              \multicolumn{1}{l}{\textbf{3}} \\
              \cmidrule(l){1-1}\cmidrule(l){2-2}\cmidrule(l){3-3}
            }
            \tablehead{}
            \tabletail {
              \multicolumn{3}{l}{\textbf{1 Zeile ausgew\"ahlt}} \\
            }
            \tablelasttail {
              \multicolumn{3}{l}{\textbf{1 Zeile ausgew\"ahlt}} \\
            }

            \begin{mssql}
              \begin{supertabular}{lll}
                2000 & 1000 & 1000 \\
              \end{supertabular}
            \end{mssql}
          \end{small}
        \end{center}
        \subsubsection{Die Syntax des PIVOT-Operators}
          Da das SQL-Statement aus \beispiel{sql06_23} auf den ersten Blick sehr komplex wirkt, ist es notwendig, es an dieser Stelle im Detail zu betrachten.

          F\"ur die Ausf\"uhrung des Pivotings wird in \beispiel{sql06_23} eine Inlineview verwendet.
          \begin{lstlisting}[language=ms_sql,caption={Die Inlineview},label=sql06_24]
(SELECT Gehalt, Bankfiliale_ID
 FROM   Mitarbeiter) AS Sourcetable
          \end{lstlisting}
          Diese Inlineview legt fest, welche Spalten im Endergebnis der Abfrage zu sehen sein werden. Sie kann beliebig komplex sein. Der \languageorasql{PIVOT}-Operator verarbeitet im zweiten Schritt die Spalten dieser View weiter.
          \begin{lstlisting}[language=ms_sql,caption={Der \languageorasql{PIVOT}-Operator},label=sql06_25]
PIVOT  (MIN(Gehalt) FOR Bankfiliale_ID IN ([1], [2], [3])) AS Pivottable;
          \end{lstlisting}
\clearpage
          Die Bedeutung dieses Operators ist:
          \begin{itemize}
            \item Gruppiere nach der Spalte \identifier{Bankfiliale\_ID}.
            \item Zeige \languagemssql{MIN(Gehalt)} f\"ur \languagemssql{Bankfiliale_ID = 1}.
            \item Zeige \languagemssql{MIN(Gehalt)} f\"ur \languagemssql{Bankfiliale_ID = 2}.
            \item Zeige \languagemssql{MIN(Gehalt)} f\"ur \languagemssql{Bankfiliale_ID = 3}.
          \end{itemize}
          F\"ur eine Pivotabfrage gelten in MS SQL Server folgende Syntaxregeln:
          \begin{itemize}
            \item F\"ur die Quell-View muss zwingend ein Aliasname vergeben werden. In \beispiel{sql06_23} ist dies \enquote{Sourcetable}
            \item F\"ur die Pivottabelle muss zwingend ein Aliasname vergeben
werden. In \beispiel{sql06_23} ist dies \enquote{Pivottable}
						\item Es d\"urfen in der Pivottabelle keine
Aliasnamen vergeben werden.
            \item Die Werte in der \languagemssql{FOR}-Klausel m\"ussen in eckigen Klammern stehen.
          \end{itemize}
        \subsubsection{Zus\"atzliche Spalten zum Pivoting}
          In einer Pivot-Abfrage k\"onnen noch weitere Spalten enthalten sein, die nicht aggregiert oder in der \languagemssql{FOR}-Klausel genutzt werden. Diese Spalten werden als zus\"atzliche Gruppierungsmerkmale genutzt.
          \begin{merke}
            MS SQL Server f\"uhrt eine implizite Gruppierung der Ergebnismenge durch. Diese basiert auf allen nicht gruppierten Spalten, inklusive der Spalten, die in der \languagemssql{FOR}-Klausel genutzt werden.
          \end{merke}
          In \beispiel{sql06_26} wird im ersten Schritt nach dem Geburtsjahr, von 1987 bis 1989 gruppiert. Da diese Spalte in der \languagemssql{FOR}-Klausel verwendet wird, wird diese Information in Spaltenform dargestellt.

          Die Spalte Ort hingegen, wird in Zeilenform angezeigt, da sie nicht in der \languagemssql{FOR}-Klausel angegeben wurde.
          \begin{merke}
            Ob eine Information in Spalten- oder Zeilenform dargestellt wird, h\"angt davon ab, ob die betreffende Spalte in der \languagemssql{FOR}-Klausel gelistet wurde oder nicht.
          \end{merke}
\clearpage
          \begin{lstlisting}[language=ms_sql,caption={Zus\"atzliche Gruppierungen in einer Pivot-Abfrage},label=sql06_26]
SELECT *
FROM   (SELECT Gehalt, DATEPART(YEAR, Geburtsdatum) AS Geburtsdatum, Ort
        FROM   Mitarbeiter) AS Sourcetable
PIVOT  (MIN(Gehalt)
        FOR Geburtsdatum IN ([1987], [1988], [1989])) AS Pivottable;
          \end{lstlisting}
          \begin{center}
            \begin{small}
              \changefont{pcr}{m}{n}
              \tablefirsthead {
                \multicolumn{1}{l}{\textbf{Ort}} &
                \multicolumn{1}{l}{\textbf{1987}} &
                \multicolumn{1}{l}{\textbf{1988}} &
                \multicolumn{1}{l}{\textbf{1989}} \\
                \cmidrule(l){1-1}\cmidrule(l){2-2}\cmidrule(l){3-3}\cmidrule(l){4-4}
              }
              \tablehead{}
              \tabletail {
                \multicolumn{4}{l}{\textbf{25 Zeilen ausgew\"ahlt}} \\
              }
              \tablelasttail {
                \multicolumn{4}{l}{\textbf{25 Zeilen ausgew\"ahlt}} \\
              }
              \begin{mssql}
                \begin{supertabular}{llll}
                  Calbe &  &  &  \\
                  Pl\"otzkau & 2500 &  &  \\
                  Nienburg &  &  &  \\
                  Bernburg &  &  &  \\
                  Dresden &  &  &  \\
                  Hecklingen &  &  & 3000 \\
                  Borne &  &  & 30000 \\
                  Sch\"onebeck &  &  &  \\
                  Giersleben &  &  &  \\
                  Gera &  &  & 3500 \\
                  M\"unchen &  &  &  \\
                  Egeln &  &  &  \\
                  G\"usten &  &  &  \\
                  Seeland &  &  & 2500 \\
                  Ilberstedt &  &  &  \\
                  B\"ordeaue &  &  &  \\
                  Hamburg & 12000 &  &  \\
                  Alsleben &  &  &  \\
                  Schwerin &  &  &  \\
                  Dessau & 2500 &  &  \\
                  K\"onnern &  &  &  \\
                  Cottbus &  &  &  \\
                  Potsdam & 3500 & 2000 & 2000 \\
                  Aschersleben & 12000 & 88000 & 2000 \\
                  Magdeburg &  &  & 3000 \\
                \end{supertabular}
              \end{mssql}
            \end{small}
          \end{center}
          Die vorangegangenen Beispiele stellen nur einen Einstieg in das Thema \enquote{Pivottabellen} dar. Tats\"achlich ist der \languageorasql{PIVOT}-Operator noch weitaus m\"achtiger.

    \clearpage
    \section{\"Ubungen - Unterabfragen}
      \begin{enumerate}
        \item Schreiben Sie eine Abfrage, die f\"ur alle Eigenkunden, die keinen
        Berater haben (die nicht in der Tabelle
        \identifier{EigenkundeMitarbeiter} enthalten sind), den Vor- und den
        Nachnamen anzeigt.
        \begin{itemize}
          \item L\"osen Sie die Aufgabe mit Hilfe des \languageorasql{EXISTS}-Operators!
          \item L\"osen Sie die Aufgabe mit Hilfe des \languageorasql{IN}-Operators!
        \end{itemize}
        \begin{center}
          \begin{small}
            \changefont{pcr}{m}{n}
            \tablefirsthead {
              \multicolumn{1}{l}{\textbf{VORNAME}} &
              \multicolumn{1}{l}{\textbf{NACHNAME}} \\
              \cmidrule(l){1-1}\cmidrule(l){2-2}
            }
            \tablehead{}
            \tabletail {
              \multicolumn{2}{l}{\textbf{16 Zeilen ausgew\"ahlt}} \\
            }
            \tablelasttail {
              \multicolumn{2}{l}{\textbf{16 Zeilen ausgew\"ahlt}} \\
            }
            \begin{msoraclesql}
              \begin{supertabular}{ll}
                Sebastian & Schr\"oder \\
                Udo & Schumacher \\
                Mia & Huber \\
                Simon & Witte \\
                Max & Bunzel \\
                Finn & Fischer \\
                Lara & Meierh\"ofer \\
                Jannis & Meier \\
              \end{supertabular}
            \end{msoraclesql}
          \end{small}
        \end{center}
        \item Erstellen Sie eine Abfrage, die ermittelt, ob es Mitarbeiter gibt
        (Vorname und Nachname), die keine Kundenberatung durchf\"uhren.
        Ausgenommen sind leitende Mitarbeiter (Mitarbeiter die in keiner
        Bankfiliale arbeiten) und Filialleiter.
        \begin{itemize}
          \item L\"osen Sie die Aufgabe mit Hilfe des \languageorasql{EXISTS}-Operators!
          \item L\"osen Sie die Aufgabe mit Hilfe des \languageorasql{IN}-Operators!
        \end{itemize}
        \begin{center}
          \begin{small}
            \changefont{pcr}{m}{n}
            \tablefirsthead {
              \multicolumn{1}{l}{\textbf{VORNAME}} &
              \multicolumn{1}{l}{\textbf{NACHNAME}} \\
              \cmidrule(l){1-1}\cmidrule(l){2-2}
            }
            \tablehead{}
            \tabletail {
              \multicolumn{2}{l}{\textbf{40 Zeilen ausgew\"ahlt}} \\
            }
            \tablelasttail {
              \multicolumn{2}{l}{\textbf{40 Zeilen ausgew\"ahlt}} \\
            }
            \begin{msoraclesql}
              \begin{supertabular}{ll}
                Amelie & Kr\"uger \\
                Anna & Schneider \\
                Chris & Simon \\
                Christian & Haas \\
                Elias & Sindermann \\
                Emilia & K\"ohler \\
                Emma & Kr\"uger \\
              \end{supertabular}
            \end{msoraclesql}
          \end{small}
        \end{center}
\clearpage
        \item Schreiben Sie eine Abfrage, die den h\"aufigsten Vornamen der
        Bankmitarbeiter anzeigt und wie oft dieser in der Tabelle
        \identifier{Mitarbeiter} vorkommt.
        \begin{center}
          \begin{small}
            \changefont{pcr}{m}{n}
            \tablefirsthead {
              \multicolumn{1}{l}{\textbf{VORNAME}} &
              \multicolumn{1}{r}{\textbf{ANZAHL}} \\
              \cmidrule(l){1-1}\cmidrule(r){2-2}
            }
            \tablehead{}
            \tabletail {
              \multicolumn{2}{l}{\textbf{1 Zeile ausgew\"ahlt}} \\
            }
            \tablelasttail {
              \multicolumn{2}{l}{\textbf{1 Zeile ausgew\"ahlt}} \\
            }
            \begin{msoraclesql}
              \begin{supertabular}{lr}
                Chris & 5 \\
              \end{supertabular}
            \end{msoraclesql}
          \end{small}
        \end{center}
        \item Schreiben Sie eine Abfrage, welche die drei Eigenkunden mit den
        niedrigsten Guthaben auf den Girokonten anzeigt.
        \begin{center}
          \begin{small}
            \changefont{pcr}{m}{n}
            \tablefirsthead {
              \multicolumn{1}{l}{\textbf{VORNAME}} &
              \multicolumn{1}{l}{\textbf{NACHNAME}} &
              \multicolumn{1}{r}{\textbf{GUTHABEN}} \\
              \cmidrule(l){1-1}\cmidrule(l){2-2}\cmidrule(r){3-3}
            }
            \tablehead{}
            \tabletail {
              \multicolumn{3}{l}{\textbf{3 Zeilen ausgew\"ahlt}} \\
            }
            \tablelasttail {
              \multicolumn{3}{l}{\textbf{3 Zeilen ausgew\"ahlt}} \\
            }
            \begin{msoraclesql}
              \begin{supertabular}{llr}
                Franz & Walther & -140505,1 \\
                Jan & Simon & -98218,6 \\
                Philipp & Hartmann & -69705,6 \\
              \end{supertabular}
            \end{msoraclesql}
          \end{small}
        \end{center}
        \item Ver\"andern Sie die Abfrage aus der vorangegangenen Aufgabe so,
        dass die drei Eigenkunden mit dem niedrigsten Guthaben (Girokonto +
        Sparbuch) angezeigt werden. Es m\"ussen auch diejenigen Kunden angezeigt
        werden, die nur ein Girokonto oder nur ein Sparbuch haben!
        \begin{center}
          \begin{small}
            \changefont{pcr}{m}{n}
            \tablefirsthead {
              \multicolumn{1}{l}{\textbf{VORNAME}} &
              \multicolumn{1}{l}{\textbf{NACHNAME}} &
              \multicolumn{1}{r}{\textbf{SUM(GUTHABEN)}} \\
              \cmidrule(l){1-1}\cmidrule(l){2-2}\cmidrule(r){3-3}
            }
            \tablehead{}
            \tabletail {
              \multicolumn{3}{l}{\textbf{3 Zeilen ausgew\"ahlt}} \\
            }
            \tablelasttail {
              \multicolumn{3}{l}{\textbf{3 Zeilen ausgew\"ahlt}} \\
            }
            \begin{msoraclesql}
              \begin{supertabular}{llr}
                Franz & Walther & -139154,4 \\
                Jan & Simon & -98218,6 \\
                Philipp & Hartmann & -69065,9 \\
              \end{supertabular}
            \end{msoraclesql}
          \end{small}
        \end{center}
        \item Schreiben Sie eine Abfrage, die alle Eigenkunden anzeigt, welche
        im Jahr 1985 keine Buchungen verursacht haben.
        \begin{center}
          \begin{small}
            \changefont{pcr}{m}{n}
            \tablefirsthead {
              \multicolumn{1}{l}{\textbf{VORNAME}} &
              \multicolumn{1}{l}{\textbf{NACHNAME}} \\
              \cmidrule(l){1-1}\cmidrule(l){2-2}
            }
            \tablehead{}
            \tabletail {
              \multicolumn{2}{l}{\textbf{285 Zeilen ausgew\"ahlt}} \\
            }
            \tablelasttail {
              \multicolumn{2}{l}{\textbf{285 Zeilen ausgew\"ahlt}} \\
            }
            \begin{msoraclesql}
              \begin{supertabular}{ll}
                Sarah & Bauer \\
                Sofia & Bauer \\
                Tom & Bauer \\
                Alina & Baumann \\
              \end{supertabular}
            \end{msoraclesql}
          \end{small}
        \end{center}
\clearpage
        \item Schreiben Sie eine Abfrage, die f\"ur jede Bankfiliale den
        Mitarbeiter mit dem h\"ochsten Gehalt ausgibt.
        \begin{center}
          \begin{small}
            \changefont{pcr}{m}{n}
            \tablefirsthead {
              \multicolumn{1}{l}{\textbf{BANKFILIALE}} &
              \multicolumn{1}{l}{\textbf{VORNAME}} &
              \multicolumn{1}{l}{\textbf{NACHNAME}} &
              \multicolumn{1}{r}{\textbf{GEHALT}} \\
              \cmidrule(l){1-1}\cmidrule(l){2-2}\cmidrule(l){3-3}\cmidrule(r){4-4}
            }
            \tablehead{}
            \tabletail {
              \multicolumn{4}{l}{\textbf{20 Zeilen ausgew\"ahlt}} \\
            }
            \tablelasttail {
              \multicolumn{4}{l}{\textbf{20 Zeilen ausgew\"ahlt}} \\
            }
            \begin{msoraclesql}
              \begin{supertabular}{lllr}
                Poststra\ss{}e 1 06449 Aschersleben & Dirk & Peters & 12000 \\
                Kirchstra\ss{}e 8 39444 Hecklingen & Leonie & Kaiser & 12000 \\
                Schmiedestra\ss{}e 3 39240 Sta\ss{}furt & Finn & K\"ohler & 12000 \\
                Am Dom 11 06449 Giersleben & Lena & Gro\ss{}e & 12000 \\
              \end{supertabular}
            \end{msoraclesql}
          \end{small}
        \end{center}
        \item Schreiben Sie eine Abfrage, die f\"ur jeden Wohnort
        (\identifier{Eigenkunde.Ort}) den Kunden anzeigt, der im Jahr 1987 das
        h\"ochste Einkommen hatte (Das Einkommen ist die Summe aller Betr\"age
        eines Kunden, in der Tabelle \identifier{Buchung}). Sortieren Sie die
        Abfrage nach den Wohnorten.
        \begin{center}
          \begin{small}
            \changefont{pcr}{m}{n}
            \tablefirsthead {
              \multicolumn{1}{l}{\textbf{ORT}} &
              \multicolumn{1}{l}{\textbf{VORNAME}} &
              \multicolumn{1}{l}{\textbf{NACHNAME}} &
              \multicolumn{1}{r}{\textbf{BETRAG}} \\
              \cmidrule(l){1-1}\cmidrule(l){2-2}\cmidrule(l){3-3}\cmidrule(r){4-4}
            }
            \tablehead{}
            \tabletail {
              \multicolumn{4}{l}{\textbf{30 Zeilen ausgew\"ahlt}} \\
            }
            \tablelasttail {
              \multicolumn{4}{l}{\textbf{30 Zeilen ausgew\"ahlt}} \\
            }
            \begin{msoraclesql}
              \begin{supertabular}{lllr}
                Alsleben & Peter & Koch & 57855,4 \\
                Aschersleben & Lara & D\"uhning & 2395,7 \\
                Barby & Chris & Beck & -6817,8 \\
              \end{supertabular}
            \end{msoraclesql}
          \end{small}
        \end{center}
        \item Erstellen Sie eine Abfrage, die die Ums\"atze der Bank
        (SUM(Buchung.Betrag)) f\"ur die Jahre 1985 bis einschlie\ss{}lich 1989
        als Pivottabelle anzeigt.
        \begin{center}
          \begin{small}
            \changefont{pcr}{m}{n}
            \tablefirsthead {
              \multicolumn{1}{r}{\textbf{'1985'}} &
              \multicolumn{1}{r}{\textbf{'1986'}} &
              \multicolumn{1}{r}{\textbf{'1987'}} &
              \multicolumn{1}{r}{\textbf{'1988'}} &
              \multicolumn{1}{r}{\textbf{'1989'}} \\
              \cmidrule(r){1-1}\cmidrule(r){2-2}\cmidrule(r){3-3}\cmidrule(r){4-4}\cmidrule(r){5-5}
            }
            \tablehead{}
            \tabletail {
              \multicolumn{5}{l}{\textbf{1 Zeile ausgew\"ahlt}} \\
            }
            \tablelasttail {
              \multicolumn{5}{l}{\textbf{1 Zeile ausgew\"ahlt}} \\
            }
            \begin{msoraclesql}
              \begin{supertabular}{rrrrr}
                559132,5 & 539497,2 & -2036841,3 & 1081361 & 1027003,1 \\
              \end{supertabular}
            \end{msoraclesql}
          \end{small}
        \end{center}
\clearpage
        \item Ver\"andern Sie die Abfrage aus der vorangegangenen Aufgabe so,
        dass die Betr\"age innerhalb der einzelnen Jahre nach Quartalen
        aufgeteilt werden.
        \begin{center}
          \begin{small}
            \changefont{pcr}{m}{n}
            \tablefirsthead {
              \multicolumn{1}{l}{\textbf{QUARTAL}} &
              \multicolumn{1}{r}{\textbf{'1985'}} &
              \multicolumn{1}{r}{\textbf{'1986'}} &
              \multicolumn{1}{r}{\textbf{'1987'}} &
              \multicolumn{1}{r}{\textbf{'1988'}} &
              \multicolumn{1}{r}{\textbf{'1989'}} \\
              \cmidrule(l){1-1}\cmidrule(r){2-2}\cmidrule(r){3-3}\cmidrule(r){4-4}\cmidrule(r){5-5}\cmidrule(r){6-6}
            }
            \tablehead{}
            \tabletail {
              \multicolumn{6}{l}{\textbf{4 Zeilen ausgew\"ahlt}} \\
            }
            \tablelasttail {
              \multicolumn{6}{l}{\textbf{4 Zeilen ausgew\"ahlt}} \\
            }
            \begin{msoraclesql}
              \begin{supertabular}{lrrrrr}
                1 & 32204,8 & 985,2 & 2981,1 & 176852 & 9777,1 \\
                3 & -11792,8 & -71935,3 & 191697,3 & 282848 & 681185,9 \\
                2 & 151841,1 & 53654,8 & -2174503,9 & 430097,2 & 223402,7 \\
                4 & 386879,4 & 556792,5 & -57015,8 & 191563,8 & 112637,4 \\
              \end{supertabular}
            \end{msoraclesql}
          \end{small}
        \end{center}
        \item Ver\"andern Sie die Abfrage aus der vorangegangenen Aufgabe so,
        dass eine Summenzeile, unterhalb der Pivottabelle angezeigt wird.
        \begin{center}
          \begin{small}
            \changefont{pcr}{m}{n}
            \tablefirsthead {
              \multicolumn{1}{l}{\textbf{QUARTAL}} &
              \multicolumn{1}{r}{\textbf{'1985'}} &
              \multicolumn{1}{r}{\textbf{'1986'}} &
              \multicolumn{1}{r}{\textbf{'1987'}} &
              \multicolumn{1}{r}{\textbf{'1988'}} &
              \multicolumn{1}{r}{\textbf{'1989'}} \\
              \cmidrule(l){1-1}\cmidrule(r){2-2}\cmidrule(r){3-3}\cmidrule(r){4-4}\cmidrule(r){5-5}\cmidrule(r){6-6}
            }
            \tablehead{}
            \tabletail {
              \multicolumn{6}{l}{\textbf{5 Zeilen ausgew\"ahlt}} \\
            }
            \tablelasttail {
              \multicolumn{6}{l}{\textbf{5 Zeilen ausgew\"ahlt}} \\
            }

            \begin{msoraclesql}
              \begin{supertabular}{lrrrrr}
                1 & 32204,8 & 985,2 & 2981,1 & 176852 & 9777,1 \\
                2 & 151841,1 & 53654,8 & -2174503,9 & 430097,2 & 223402,7 \\
                3 & -11792,8 & -71935,3 & 191697,3 & 282848 & 681185,9 \\
                4 & 386879,4 & 556792,5 & -57015,8 & 191563,8 & 112637,4 \\
                Summe & 559132,5 & 539497,2 & -2036841,3 & 1081361 & 1027003,1 \\
              \end{supertabular}
            \end{msoraclesql}
          \end{small}
        \end{center}
      \end{enumerate}

    \input{../sql/loesungen/sql_06_unterabfragen_loesungen}
    \chapter{Data Manipulation Language (DML)}
    \setcounter{page}{1}\kapitelnummer{chapter}
    \minitoc
\newpage
      In den vergangenen Kapiteln wurde bisher nur der Teil von SQL beschrieben,
      der als sog. \enquote{Query language} bezeichnet wird. Hier wird jetzt
      gezeigt, wie vorhandene Daten manipuliert werden k\"onnen. Der daf\"ur
      zust\"andige Teil von SQL hei\ss{}t: \enquote{Data Manipulation Language}
      oder kurz \enquote{DML}.

      Gem\"a\ss\ SQL-Standard besteht DML aus drei Befehlen:
      \begin{itemize}
        \item \INSERT: Daten einf\"ugen.
        \item \UPDATE: Daten \"andern.
        \item \DELETE: Daten l\"oschen.
      \end{itemize}
    \section{Die DML-Anweisungen}
      \subsection{Datens\"atze einf\"ugen - Die INSERT-Anweisung}
        Mit Hilfe der \INSERT-Anweisung werden neue Datens\"atze an eine Tabelle
        angef\"ugt. Die Syntax f\"ur ein einfaches \INSERT lautet:
        \begin{lstlisting}[language=oracle_sql,caption={Die INSERT Anweisungen},label=sql07_01]
INSERT INTO <Tabelle> (<Spalte 1>, <Spalte 2>, ..., <Spalte n>)
VALUES (<Wert 1>, <Wert 2>, ..., <Wert n>);
        \end{lstlisting}
        \begin{center}
          \tablecaption{Die INSERT-Anweisung}
          \label{insertsyntax}
          \begin{small}
            \tablefirsthead{
              \multicolumn{1}{c}{\textbf{Ausdruck}} &
              \multicolumn{1}{c}{\textbf{Bedeutung}} \\
              \hline
            }
            \tablehead{
              \multicolumn{1}{c}{\textbf{Ausdruck}} &
              \multicolumn{1}{c}{\textbf{Bedeutung}} \\
              \hline
            }
            \tabletail{
              \hline
            }
            \tablelasttail{
              \hline
            }
            \begin{supertabular}{|l|p{10.8cm}|}
              INSERT INTO <Tabelle> & An dieser Stelle steht der Name der
              Tabelle oder View, in die der Datensatz eingef\"ugt werden soll.
              \\
              \hline
              <Spalte 1>, <Spalte 2>, ... & Dies ist die Spaltenliste. Hier
              k\"onnen alle Spalten angegeben werden, in die Daten eingef\"ugt
              werden. Die Spaltenliste ist optional. \\
              \hline
              VALUES <Wert 1>, ... & Dies ist die Werteliste. Hier werden alle
              Werte aufgef\"uhrt, die in <Tabelle> eingef\"ugt werden sollen.
              Statt einem festen Wert, kann an jeder Stelle auch ein Ausdruck
              stehen, der einen Wert erzeugt (z. B. eine Funktion). \\
            \end{supertabular}
          \end{small}
        \end{center}
        \beispiel{sql07_02} demonstriert die einfachste Form eines
        \INSERT-Statements: Es wird eine neue Zeile in die Tabelle
        \identifier{Bankfiliale} eingef\"ugt.
        \begin{lstlisting}[language=oracle_sql,caption={Ein einfaches INSERT},label=sql07_02]
INSERT INTO Bankfiliale (Bankfiliale_ID, Strasse, Hausnummer, PLZ, Ort)
VALUES (22, 'Rosenweg', '14a', '06425', 'Ploetzkau');
        \end{lstlisting}
\clearpage
        In obigem Beispiel wird der Wert \enquote{22} in die Spalte
        \identifier{Bankfiliale\_ID}, der Wert \enquote{Rosenweg} in die Spalte
        \identifier{Strasse} eingef\"ugt. Die restlichen drei Werte werden in
        die Spalten \identifier{Hausnummer}, \identifier{PLZ} und
        \identifier{Ort} geschrieben.
        Die Spaltenliste der \INSERT-Anweisung muss die einzelnen Spalten keineswegs in der Reihenfolge enthalten, wie sie in der Tabelle enthalten sind.
        \begin{lstlisting}[language=oracle_sql,caption={Ein einfaches INSERT},label=sql07_03]
INSERT INTO Bankfiliale (Strasse, Hausnummer, PLZ, Ort, Bankfiliale_ID)
VALUES ('Rosenweg', '14a', '06425', 'Ploetzkau', 22);
        \end{lstlisting}
        \begin{merke}
          In der Spaltenliste m\"ussen die Spalten nicht in der Reihenfolge
          aufgef\"uhrt werden, wie sie in der Tabelle vorkommen. Die Reihenfolge
          in der Spaltenliste ist beliebig!
        \end{merke}
        \vspace{1em}
        Wie in \tabelle{insertsyntax} bereits beschrieben, ist die Spaltenliste
        hinter den \languageorasql{INSERT INTO}-Schl\"ussel\-w\"ortern optional.
        Daraus folgt, dass sich \beispiel{sql07_02} auch so schreiben l\"asst:
        \begin{lstlisting}[language=oracle_sql,caption={Ein einfaches INSERT ohne Spaltenliste},label=sql07_04]
INSERT INTO Bankfiliale
VALUES (22, 'Rosenweg', '14a', '06425', 'Ploetzkau');
        \end{lstlisting}
        Das \INSERT-Statement in \beispiel{sql07_04} wird vom DBMS so
        interpretiert, dass der erste Wert in die erste Spalte, der zweite Wert
        in die zweite Spalte, der dritte Wert in die dritte Spalte, usw.
        eingef\"ugt wird.
        \subsubsection{Die INSERT-Anweisung und NULL-Werte}
          Soll mit einer \INSERT-Anweisung ein NULL-Wert in eine Tabellenspalte
          eingef\"ugt werden, geschieht dies mit Hilfe des Schl\"usselwortes
          \languageorasql{NULL}. In \beispiel{sql07_05} wird eine neue Zeile in
          die Tabelle \identifier{Bank} eingef\"ugt. W\"ahrend des
          Einf\"ugevorgangs ist der Wert f\"ur die Spalte \identifier{Rating}
          noch nicht bekannt. Die Zeile soll nun ohne diesen Wert eingef\"ugt
          werden.
        \begin{lstlisting}[language=oracle_sql,caption={Ein einfaches \INSERT{} mit \languageorasql{NULL}-Werten},label=sql07_05]
INSERT INTO Bank
VALUES (21, 'KRDCU21SES', 'Lokki Bank of Cyprus', 'Steuerparadies', '42',
        '01067', 'Berlin', NULL);
        \end{lstlisting}
        \begin{merke}
          Mit Hilfe des Schl\"usselwortes \languageorasql{NULL} kann ein
          \languageorasql{NULL}-Wert in eine Tabellenspalte eingef\"ugt werden.
        \end{merke}
        \subsubsection{Standardwerte}
          \label{defaultvalues}
          Standardwerte werden meist dann genutzt, wenn in eine Spalte h\"aufig
          der gleiche Wert eingef\"ugt werden muss. Sie m\"ussen bei der
          Erstellung einer Tabelle mit definiert werden. Ein Beispiel hierf\"ur
          k\"onnte die Spalte \identifier{Buchungsdatum} der Tabelle
          \identifier{Buchung} sein. Wird eine neue Buchung erfasst, muss immer
          das aktuelle Tagesdatum eingetragen werden. Diese kann durch die
          Funktion \languageorasql{SYSDATE} (Oracle) bzw.
          \languagemssql{GETDATE} (SQL Server) erzeugt werden.

          \beispiel{sql07_06} zeigt, wie in die Spalte
          \identifier{Buchungsdatum} das aktuelle Datum, als Standardwert
          eingef\"ugt wird.
          \begin{lstlisting}[language=oracle_sql,caption={Einf\"ugen eines Standardwertes},label=sql07_06]
INSERT INTO Buchung (Buchungs_ID, Betrag, Buchungsdatum, Konto_ID,
                     Transaktions_ID)
VALUES (500000, 300.20, DEFAULT, 1, 666666);
          \end{lstlisting}
          \begin{merke}
            Soll in eine Tabellenspalte deren Standardwert eingef\"ugt werden,
            muss das Schl\"usselwort \languageorasql{DEFAULT} benutzt werden.
          \end{merke}
        \subsubsection{Die INSERT-Anweisung und Unterabfragen}
          Die \INSERT-Anweisung ist in der Lage eine Unterabfrage zu verwenden,
          um den Inhalt einer Tabelle in eine andere Tabelle einzuf\"ugen. Dies
          kann z. B. das Kopieren eines Datensatzes in eine Tabelle gleicher
          Struktur sein oder das Abfragen einzelner Attribute, um diese f\"ur
          die Berechnung neuer Werte zu nutzen. Die Syntax f\"ur die
          \INSERT-Anweisung mit Unterabfrage lautet:
          \begin{lstlisting}[language=oracle_sql,caption={Die \INSERT-Anweisung mit Unterabfrage},label=sql07_07]
INSERT INTO <Tabelle> (<Spalte 1>, <Spalte 2>, ..., <Spalte n>)
<Unterabfrage>;
          \end{lstlisting}
          Das \INSERT-Statement kann eine beliebig komplexe Unterabfrage, wie in
          \abschnitt{subqueries} beschrieben, verwenden. \beispiel{sql07_08}
          zeigt, wie ein Datensatz aus der Tabelle \identifier{Mitarbeiter} in
          die strukturgleiche Tabelle \identifier{Ausgeschieden} kopiert wird.
          \begin{lstlisting}[language=oracle_sql,caption={Die \INSERT-Anweisung
      mit Unterabfrage},label=sql07_08]
-- Erstellen der Tabelle in Oracle
CREATE TABLE Ausgeschieden
AS
  SELECT *
  FROM   Mitarbeiter
  WHERE  1 = 2;
          \end{lstlisting}
\clearpage
          \begin{lstlisting}[language=oracle_sql]
-- Erstellen der Tabelle in SQL Server
SELECT *
INTO   Ausgeschieden
FROM   Mitarbeiter
WHERE  1 = 2;

-- Die INSERT-Anweisung für Oracle und SQL Server
INSERT INTO Ausgeschieden
SELECT *
FROM   Mitarbeiter
WHERE  Mitarbeiter_ID = 70;
          \end{lstlisting}
          Der Datensatz des Mitarbeiters Nummer 70 wird in die Tabelle
          \identifier{Ausgeschieden} kopiert.
      \subsection{Datens\"atze \"andern - Die UPDATE-Anweisung}
        Die \UPDATE-Anweisung repr\"asentiert den Teil von DML der es
        erm\"oglicht, bestehende Datens\"atze zu ver\"andern. Die Syntax von
        \UPDATE{} lautet:
        \begin{lstlisting}[language=oracle_sql,caption={Die Syntax des \UPDATE-Kommandos},label=sql07_09]
UPDATE <Tabelle>
SET    <Spalte 1> = <Wert>,
       <Spalte 2> = <Wert>,
       ...
       <Spalte n> = <Wert>
[WHERE  <Where-Klausel>];
        \end{lstlisting}

        \begin{center}
          \tablecaption{Die UPDATE-Anweisung}
          \label{updatesyntax}
          \begin{small}
            \tablefirsthead{
              \multicolumn{1}{c}{\textbf{Ausdruck}} &
              \multicolumn{1}{c}{\textbf{Bedeutung}} \\
              \hline
            }
            \tablehead{
              \multicolumn{1}{c}{\textbf{Ausdruck}} &
              \multicolumn{1}{c}{\textbf{Bedeutung}} \\
              \hline
            }
            \tabletail{
              \hline
            }
            \tablelasttail{\hline}
            \begin{supertabular}{|l|p{10.8cm}|}
              UPDATE <Tabelle> & An dieser Stelle steht der Name der Tabelle
              oder View, in der ein Datensatz ver\"andert werden soll. \\
              \hline
              SET <Spalte 1> = <Wert> & Die SET-Anweisung gibt die Spalten an,
              deren aktueller Wert durch den neuen Wert <Wert> ersetzt werden
              soll. Hier k\"onnen mehrere \enquote{Spalte = Wert}-Paare, durch
              Komma getrennt, stehen.\\
              \hline
              WHERE <Where-Klausel> & Optionale WHERE-Klausel, die den Umfang
              der Datens\"atze, die ge\"andert werden sollen, einschr\"ankt. \\
            \end{supertabular}
          \end{small}
        \end{center}
        Eine genauso einfache, wie auch gef\"ahrliche Form der
        \UPDATE-Anweisung, ist in \beispiel{sql07_10} zu sehen.
        \begin{lstlisting}[language=oracle_sql,caption={Ein gef\"ahrliches \UPDATE},label=sql07_10]
UPDATE Mitarbeiter
SET    Gehalt = Gehalt * 1.035;
        \end{lstlisting}
        Die Gefahr bei dieser \UPDATE-Anweisung besteht darin, das die Angabe
        einer einschr\"ankenden \WHERE-Klausel fehlt. Das DBMS wird in diesem
        Falle alle Datens\"atze der Tabelle \identifier{Mitarbeiter} ver\"andern
        und nicht nur eine bestimmte Gruppe.

        Soll nur das Gehalt des Mitarbeiters \textit{Max Winter} ge\"andert
        werden, muss das \UPDATE-Statement um eine \WHERE-Klausel erweitert
        werden:
        \begin{lstlisting}[language=oracle_sql,caption={Ein korrektes \UPDATE},label=sql07_11]
UPDATE Mitarbeiter
SET    Gehalt = Gehalt * 1.035
WHERE  Mitarbeiter_ID = 1;
        \end{lstlisting}
        Wie in \tabelle{updatesyntax} zu sehen ist, k\"onnen auch mehrere
        Spalten eines Datensatzes gleichzeitig ge\"andert werden. In
        \beispiel{sql07_12} wird die Mitarbeiterin \enquote{Lena Hermann}
        (Mitarbeiter\_ID 40) von Filiale 4 nach Filiale 8 versetzt und
        gleichzeitig wird ihre Provision von 20 \% auf 30 \% erh\"oht.
        \begin{lstlisting}[language=oracle_sql,caption={Ein korrektes \UPDATE{} mehrerer Spalten},label=sql07_12]
UPDATE Mitarbeiter
SET    Bankfiliale_ID = 8,
       Provision = 30
WHERE  Mitarbeiter_ID = 40;
        \end{lstlisting}
        Wo Licht ist, da ist aber immer auch Schatten. Wenn bei einem
        Mitarbeiter die Provision erh\"oht wird, muss sie bei einem anderen
        gek\"urzt oder gestrichen werden. Der Mitarbeiter \enquote{Lukas
        Wei\ss{}} hat im vergangenen Gesch\"aftsjahr ein sehr schlechtes
        Ergebnis erziehlt, weshalb ihm die Provision gestrichen wird. Dies
        geschieht, in dem die Spalte \identifier{Provision} mit einem
        \languageorasql{NULL}-Wert gef\"ullt wird.
        \begin{lstlisting}[language=oracle_sql,caption={Da geht sie hin, die Provision},label=sql07_13]
UPDATE Mitarbeiter
SET    Provision = NULL
WHERE  Mitarbeiter_ID = 38;
        \end{lstlisting}
        Nicht nur NULL-Werte, auch Standardwerte k\"onnen innerhalb eines
        \UPDATE-Statements genutzt werden.
        \begin{lstlisting}[language=oracle_sql,caption={Ein \UPDATE{} mit Standardwert},label=sql07_14]
UPDATE Mitarbeiter
SET    Gehalt = DEFAULT
WHERE  Mitarbeiter_ID = 82;
        \end{lstlisting}
        \beispiel{sql07_14} geht davon aus, dass f\"ur die Spalte
        \identifier{Gehalt} ein Standardwert von \enquote{1500} festgelegt
        worden ist.
        \subsubsection{UPDATE mit Unterabfrage}
          Wie bei der \INSERT-Anweisung, kann auch bei der \UPDATE-Anweisung
          eine Unterabfrage genutzt werden. Diese kann an zwei Stellen stehen:
          In der \languageorasql{SET}-Klausel und in der \WHERE-Klausel. Hierzu
          einige Beispiele.

          Das Gehalt des Mitarbeiters \enquote{Jannis Friedrich} soll ge\"andert
          werden. Das neue Gehalt muss 20 \% des Gehalts seines unmittelbaren
          Vorgesetzten betragen.
          \begin{lstlisting}[language=oracle_sql,caption={\UPDATE{} mit Unterabfrage},label=sql07_15]
UPDATE Mitarbeiter
SET    Gehalt = (SELECT v.Gehalt
                 FROM   Mitarbeiter m INNER JOIN Mitarbeiter v
                          ON (m.Vorgesetzter_ID = v.Mitarbeiter_ID)
                 WHERE  m.Mitarbeiter_ID = 79) * 0.20
WHERE  Mitarbeiter_ID = 79;
          \end{lstlisting}
          Mit Hilfe der folgenden \SELECT-Anweisung kann die Korrektheit des
          \UPDATE-Statements aus \beispiel{sql07_15} nachgewiesen werden.
          \begin{lstlisting}[language=oracle_sql,caption={Der Beweis},label=sql07_16]
SELECT Mitarbeiter_ID, Vorname, Nachname, Gehalt
FROM   Mitarbeiter
WHERE  Mitarbeiter_ID IN (79, 21);
          \end{lstlisting}
          In \beispiel{sql07_15} wird die Mitarbeiter\_ID 79 an zwei Stellen
          angegeben. Durch eine Ver\"anderung des \UPDATE-Statements kann dies
          auf eine Angabe reduziert werden.
          \begin{lstlisting}[language=oracle_sql,caption={\UPDATE{} mit
korrelierter Unterabfrage in Oracle},label=sql07_17]
UPDATE Mitarbeiter m1
SET    Gehalt = (SELECT v.Gehalt
                 FROM   Mitarbeiter m INNER JOIN Mitarbeiter v
                          ON (m.Vorgesetzter_ID = v.Mitarbeiter_ID)
                 WHERE  m.Mitarbeiter_ID = m1.Mitarbeiter_ID)
WHERE  m1.Mitarbeiter_ID = 79;
          \end{lstlisting}

          In der \UPDATE-Klausel wird ein Alias f\"ur die Tabelle
          \identifier{Mitarbeiter} festgelegt. Diesen Alias benutzt die
          Unterabfrage, um auf die Werte des \"au\ss{}eren Statements, des
          \UPDATE-Statements, zuzugreifen. Dadurch gen\"ugt es, wenn die
          Mitarbeiter\_ID nur einmal gesetzt wird. Im MS SQL Server muss der
          Alias f\"ur die Tabelle \identifier{Mitarbeiter} \"uber eine
          \FROM-Klausel definiert werden, so dass sich \beispiel{sql07_17} wie
          folgt \"andert:
\clearpage
          \begin{lstlisting}[language=ms_sql,caption={\UPDATE{} mit
korrelierter Unterabfrage im MS SQL Server},label=sql07_171]
UPDATE m1
SET    Gehalt = (SELECT v.Gehalt
								 FROM   Mitarbeiter m INNER JOIN Mitarbeiter v
													ON (m.Vorgesetzter_ID = v.Mitarbeiter_ID)
								 WHERE  m.Mitarbeiter_ID = m1.Mitarbeiter_ID)
FROM Mitarbeiter m1
WHERE  m1.Mitarbeiter_ID = 79;
          \end{lstlisting}

          \enquote{Was des einen Freud ist, ist des andern Leid}. Dieser
          Grundsatz trifft auch bei der Gehaltserh\"ohung f\"ur Herrn Friedrich
          zu. Da er nun 400 EUR mehr Gehalt bekommt, m\"ussen bei den anderen
          Angestellten dementsprechende Einsparungen vorgenommen  werden. F\"ur
          alle Mitarbeiter der Filiale 14, mit Ausnahme von Herrn Friedrich,
          muss das Gehalt um 2~\% gek\"urzt werden.
          \begin{lstlisting}[language=oracle_sql,caption={Gehaltsk\"urzung f\"ur eine ganze Filiale},label=sql07_18]
UPDATE Mitarbeiter
SET    Gehalt = Gehalt * 0.98
WHERE  Mitarbeiter_ID IN (SELECT Mitarbeiter_ID
                          FROM   Mitarbeiter
                          WHERE  Bankfiliale_ID = 14
                            AND  Mitarbeiter_ID <> 79);
          \end{lstlisting}
      \subsection{Datens\"atze l\"oschen - Die DELETE-Anweisung}
        Die dritte und letzte der DML-Anweisungen, ist die \DELETE-Anweisung.
        Sie erm\"oglicht es, Datens\"atze zu l\"oschen. Die Syntax der
        \DELETE-Anweisung lautet wie folgt:
        \begin{lstlisting}[language=oracle_sql,caption={Die \DELETE-Anweisung},label=sql07_19]
DELETE FROM <Tabelle>
WHERE <Where-Klausel>;
        \end{lstlisting}
        \begin{center}
          \tablecaption{Die DELETE-Anweisung}
          \label{deletesyntax}
          \begin{small}
            \tablefirsthead{
              \multicolumn{1}{c}{\textbf{Ausdruck}} &
              \multicolumn{1}{c}{\textbf{Bedeutung}} \\
              \hline
            }
            \tablehead{
              \multicolumn{1}{c}{\textbf{Ausdruck}} &
              \multicolumn{1}{c}{\textbf{Bedeutung}} \\
              \hline
            }
            \tabletail{
              \hline
            }
            \begin{supertabular}{|l|p{10.8cm}|}
              DELETE <Tabelle> & An dieser Stelle steht der Name der Tabelle
              oder View, aus der Datens\"atze gel\"oscht werden sollen. \\
              \hline
              WHERE <Where-Klausel> & Optionale WHERE-Klausel, die den Umfang
              der Datens\"atze begrenzt, die gel\"oscht werden sollen. \\
            \end{supertabular}
          \end{small}
        \end{center}
        \"Ahnlich wie bei der \UPDATE-Anweisung, gibt es auch bei der
        \DELETE-Anweisung eine kleine Falle. \beispiel{sql07_20} zeigt, wie man
        mit einer sehr einfachen \DELETE-Anweisung in gro\ss{}e Schwierigkeiten
        geraten kann.
        \begin{lstlisting}[language=oracle_sql,caption={Eine t\"odliche \DELETE-Anweisung},label=sql07_20]
DELETE FROM Buchung;
        \end{lstlisting}
        Die Auswirkungen der \DELETE-Anweisung aus \beispiel{sql07_20} sind
        einfach und kurz erkl\"art. Es werden alle Datens\"atze aus der Tabelle
        \identifier{Buchung} gel\"oscht. Des R\"atsels L\"osung ist die gleiche
        wie beim \UPDATE-Statement: Es fehlt die einschr\"ankende
        \WHERE-Klausel. Um beispielsweise nur eine einzelne Buchung zu l\"oschen
        ist folgende Modifikation notwendig:
        \begin{lstlisting}[language=oracle_sql,caption={Schon viel besser!!!},label=sql07_21]
DELETE FROM Buchung
WHERE  Transaktions_ID = 345;
        \end{lstlisting}
        \subsubsection{DELETE mit Unterabfrage}
         Auch in der \DELETE-Anweisung kann eine Unterabfrage genutzt werden.
         Hierzu ein einfaches Beispiel: Da die Bankfiliale, in der
         Poststra\ss{}e, in Aschersleben aufgel\"ost wird, m\"ussen leider auch
         die dort besch\"aftigten Mitarbeiter wieder dem Arbeitsmarkt zur
         Verf\"ugung gestellt werden.
          \begin{lstlisting}[language=oracle_sql,caption={\DELETE{} mit Unterabfrage},label=sql07_22]
DELETE FROM Mitarbeiter
WHERE  Bankfiliale_ID = (SELECT Bankfiliale_ID
                         FROM   Bankfiliale
                         WHERE  LOWER(Strasse) LIKE 'poststraße'
                           AND  PLZ = '06449');
          \end{lstlisting}
    \section{Das Transaktionskonzept - COMMIT und ROLLBACK}
      Die Datenbankmanagementsysteme Oracle und SQL Server sind beides
      transaktionsbasierte DBMS. Das bedeutet, dass alle DML-Anweisungen
      innerhalb einer Transaktion ablaufen. Die Frage die sich dabei stellt ist:
      \enquote{Was ist eine Transaktion?} Der Begriff Transaktion ist dem
      sp\"atlateinischen \enquote{transagere = \"Uberf\"uhren, \"Ubertragen}
      entliehen und den meisten Leuten aus dem Finanzbereich bekannt. Man denke
      einfach an die \"Uberweisung eines Betrags von Konto A auf Konto B. Der
      vereinfachte Ablauf einer solchen Finanztransaktion k\"onnte wie folgt
      aussehen:
      \begin{enumerate}
        \item Kontoinhaber A f\"ullt einen \"Uberweisungstr\"ager aus. Damit
        beginnt die Transaktion.
        \item Die Bank des Kontoinhabers A zieht den \"Uberweisungsbetrag von
        seinem Konto ab und \"ubermittelt die Informationen bez\"uglich der
        \"Uberweisung an Bank B.
        \item Bank B schreibt den Betrag auf dem Konto von Kontoinhaber B gut.
        \item Der Vorgang wird in einem Journal protokolliert. Damit ist die
        \"Uberweisung abgeschlossen.
      \end{enumerate}
      Warum aber der Begriff der Transaktion? Die Antwort auf diese Frage
      h\"angt eng mit der Antwort auf eine andere Frage zusammen: \enquote{Was
      w\"are wenn, nach der Abbuchung von Konto A der Vorgang unterbrochen
      w\"urde?} In so einem Falle ist das gewohnte Verhalten, das alle bisher
      gemachten Schritte wieder r\"uckg\"angig gemacht werden, d. h. der
      abgebuchte Betrag muss wieder auf das Konto von A zur\"uckgebucht werden.
      W\"urde dies nicht geschehen, w\"are das Geld von A verschwunden.

      Das R\"uckg\"angigmachen aller bisher gemachten Aktionen ist aber nur dann
      m\"oglich, wenn
      \begin{itemize}
        \item genau bekannt ist, welche Aktionen zusammengeh\"oren und
        \item  in welcher Reihenfolge sie stattgefunden haben.
      \end{itemize}
      Deshalb werden alle Aktionen in einer gr\"o\ss eren Einheit, der
      Transaktion, zusammengefa\ss t. Es muss also im Ernstfall nur ermittelt
      werden, zu welcher Transaktion die letzte Aktion geh\"orte um alle
      Vorg\"angeraktionen ermitteln zu k\"onnen.
      \begin{merke}
        Definition \textit{Transaktion}: Eine Transaktionen ist eine logische
        Arbeitseinheit, die einen oder mehrere Arbeitsschritte enth\"alt.
        Transaktionen sind in sich geschlossene Einheiten. Die Ergebnisse aller
        Arbeitsschritte einer Transaktion k\"onnen entweder \"ubernommen oder
        r\"uckg\"angig gemacht werden.
      \end{merke}
      Dieses Konzept l\"asst sich auch auf Datenbanken \"ubertragen. Werden
      mehrere zusammengeh\"orende SQL-Anweisungen ausgef\"uhrt, muss auch
      gew\"ahrleistet werden, dass entweder alle erfolgreich beendet werden oder
      aber alle r\"uckg\"angig gemacht werden.
      \subsection{Beginn und Ende einer Transaktion}
        \subsubsection{Wann beginnt eine Transaktion?}
          In Oracle startet eine Transaktion:
          \begin{itemize}
            \item Implizit bei jedem ersten DML-Statement.
            \item Explizit durch die Anweisung \languageorasql{SET TRANSACTION}.
          \end{itemize}
          In MS SQL Server startet eine Transaktion:
          \begin{itemize}
            \item Wenn der implizite Transaktionsmodus aktiviert wurde, bei
            jedem ersten DML-State\-ment.
            \item Explizit durch die Anweisung \languagemssql{BEGIN TRANSACTION}
          \end{itemize}
          \begin{merke}
            Das Standardverhalten in SQL Server ist, dass jedes einzelne
            DML-Statement als eigene Transaktion abgehandelt wird. Zur
            Ak\-ti\-vie\-rung des impliziten Transaktionsmodus muss die
            SQL-Anweisung \languagemssql{SET IMPLICIT_TRANSACTIONS ON} abgesetzt
            werden.
          \end{merke}
        \subsubsection{Wann endet eine Transaktion?}
          Eine Transaktion kann an zwei verschiedenen Punkten enden:
          \begin{itemize}
            \item Sie wird erfolgreich abgeschlossen.
            \item Sie wird manuell r\"uckg\"angig gemacht.
          \end{itemize}
      \subsection{Eine Transaktion erfolgreich abschlie\ss en}
        \subsubsection{Das COMMIT-Kommando}
          Wenn alle Statements einer Transaktion erfolgreich verlaufen sind, muss
          die Transaktion beendet werden, um die gemachten \"Anderungen dauerhaft
          in der Datenbank zu speichern. Dies geschieht in Oracle mit Hilfe des
          Kommandos \COMMIT. Wird eine Transaktion nicht mit \COMMIT{} abgeschlossen, werden
          automatisch alle unbestätigten \"Anderungen r\"uckg\"angig gemacht.
          \beispiel{sql07_23} ff. zeigen dieses Verhalten.
          \begin{lstlisting}[language=oracle_sql,caption={Eine Transaktion wird abgebrochen},label=sql07_23]
SELECT Bank_ID, BIC, Name
FROM   Bank
WHERE  Bank_ID >= 18;
          \end{lstlisting}
          \begin{center}
            \begin{small}
              \changefont{pcr}{m}{n}
              \tablefirsthead {
                \multicolumn{1}{r}{\textbf{BANK\_ID}} &
                \multicolumn{1}{l}{\textbf{BIC}} &
                \multicolumn{1}{l}{\textbf{NAME}} \\
                \cmidrule(r){1-1}\cmidrule(l){2-2}\cmidrule(l){3-3}
              }
              \tablehead{}
              \tabletail {
                \multicolumn{3}{l}{\textbf{3 Zeilen ausgew\"ahlt}} \\
              }
              \tablelasttail {
                \multicolumn{3}{l}{\textbf{3 Zeilen ausgew\"ahlt}} \\
              }
              \begin{msoraclesql}
                \begin{supertabular}{rll}
                  18 & BVXYDE21SES & Bank der Landwirte \\
                  19 & BGIODE21SES & Austrailian Bank Association \\
                  20 & DFGHDE21SES & South Africa Bank \\
                \end{supertabular}
              \end{msoraclesql}
            \end{small}
          \end{center}
          \begin{lstlisting}[language=oracle_sql,label=sql07_24]
INSERT INTO Bank
VALUES      (21, 'NOSDEL21SES', 'Lokki Bank of Cyprus',
             'Strasse der Europaeischen Union', '3', '00000', 'Pleitingen',
             'D--');

SELECT Bank_ID, BIC, Name
FROM   Bank
WHERE  Bank_ID >= 18;
        \end{lstlisting}
\clearpage
          \begin{center}
            \begin{small}
              \changefont{pcr}{m}{n}
              \tablefirsthead {
                \multicolumn{1}{r}{\textbf{BANK\_ID}} &
                \multicolumn{1}{l}{\textbf{BIC}} &
                \multicolumn{1}{l}{\textbf{NAME}} \\
                \cmidrule(r){1-1}\cmidrule(l){2-2}\cmidrule(l){3-3}
              }
              \tablehead{}
              \tabletail {
                \multicolumn{3}{l}{\textbf{4 Zeilen ausgew\"ahlt}} \\
              }
              \tablelasttail {
                \multicolumn{3}{l}{\textbf{4 Zeilen ausgew\"ahlt}} \\
              }
              \begin{msoraclesql}
                \begin{supertabular}{rll}
                  18 & BVXYDE21SES & Bank der Landwirte \\
                  19 & BGIODE21SES & Austrailian Bank Association \\
                  20 & DFGHDE21SES & South Africa Bank \\
                  21 & NOSDEL21SES & Lokki Bank of Cyprus  \\
                \end{supertabular}
              \end{msoraclesql}
            \end{small}
          \end{center}
          \begin{lstlisting}[language=oracle_sql,label=sql07_25]
-- An dieser Stelle findet ein Absturz der Client-Anwendung statt
-- und die Anwendung wird neu gestartet.

SELECT Bank_ID, BIC, Name
FROM   Bank
WHERE  Bank_ID >= 18;
          \end{lstlisting}
          \begin{center}
            \begin{small}
              \changefont{pcr}{m}{n}
              \tablefirsthead {
                \multicolumn{1}{r}{\textbf{BANK\_ID}} &
                \multicolumn{1}{l}{\textbf{BIC}} &
                \multicolumn{1}{l}{\textbf{NAME}} \\
                \cmidrule(r){1-1}\cmidrule(l){2-2}\cmidrule(l){3-3}
              }
              \tablehead{}
              \tabletail {
                \multicolumn{3}{l}{\textbf{3 Zeilen ausgew\"ahlt}} \\
              }
              \tablelasttail {
                \multicolumn{3}{l}{\textbf{3 Zeilen ausgew\"ahlt}} \\
              }
              \begin{msoraclesql}
                \begin{supertabular}{rll}
                  18 & BVXYDE21SES & Bank der Landwirte \\
                  19 & BGIODE21SES & Austrailian Bank Association \\
                  20 & DFGHDE21SES & South Africa Bank \\
                \end{supertabular}
              \end{msoraclesql}
            \end{small}
          \end{center}
          Weil vor dem Absturz der Client-Anwendung die Transaktion nicht mit
          \COMMIT{} abgeschlossen wurde, ist die gemachte \"Anderung wieder
          verschwunden. Das gleiche Szenario nun noch einmal, aber mit \COMMIT{}
          am Ende.
          \begin{lstlisting}[language=oracle_sql,label=sql07_26]
INSERT INTO Bank
VALUES      (21, 'NOSDEL21SES', 'Lokki Bank of Cyprus',
             'Strasse der Europaeischen Union', '3', '00000', 'Pleitingen',
             'D--');

SELECT Bank_ID, BIC, Name
FROM   Bank
WHERE  Bank_ID >= 18;

COMMIT;
          \end{lstlisting}
          \begin{center}
            \begin{small}
              \changefont{pcr}{m}{n}
              \tablefirsthead {
                \multicolumn{1}{r}{\textbf{BANK\_ID}} &
                \multicolumn{1}{l}{\textbf{BIC}} &
                \multicolumn{1}{l}{\textbf{NAME}} \\
                \cmidrule(r){1-1}\cmidrule(l){2-2}\cmidrule(l){3-3}
              }
              \tablehead{}
              \tabletail {
  %               \multicolumn{3}{l}{\textbf{4 Zeilen ausgew\"ahlt}} \\
              }
              \tablelasttail {
                \multicolumn{3}{l}{\textbf{4 Zeilen ausgew\"ahlt}} \\
              }
              \begin{msoraclesql}
                \begin{supertabular}{rll}
                  18 & BVXYDE21SES & Bank der Landwirte \\
                  19 & BGIODE21SES & Austrailian Bank Association \\
                  20 & DFGHDE21SES & South Africa Bank \\
                  21 & NOSDEL21SES & Lokki Bank of Cyprus  \\
                \end{supertabular}
              \end{msoraclesql}
            \end{small}
          \end{center}
\clearpage
         \begin{lstlisting}[language=oracle_sql,label=sql07_27]
-- An dieser Stelle wird die Client-Anwendung beendet und
-- neugestartet.

SELECT Bank_ID, BIC, Name
FROM   Bank
WHERE  Bank_ID >= 18;
          \end{lstlisting}
          \begin{center}
            \begin{small}
              \changefont{pcr}{m}{n}
              \tablefirsthead {
                \multicolumn{1}{r}{\textbf{BANK\_ID}} &
                \multicolumn{1}{l}{\textbf{BIC}} &
                \multicolumn{1}{l}{\textbf{NAME}} \\
                \cmidrule(r){1-1}\cmidrule(l){2-2}\cmidrule(l){3-3}
              }
              \tablehead{}
              \tabletail {
              }
              \tablelasttail {
                \multicolumn{3}{l}{\textbf{4 Zeilen ausgew\"ahlt}} \\
              }
              \begin{msoraclesql}
                \begin{supertabular}{rll}
                  18 & BVXYDE21SES & Bank der Landwirte \\
                  19 & BGIODE21SES & Austrailian Bank Association \\
                  20 & DFGHDE21SES & South Africa Bank \\
                  21 & NOSDEL21SES & Lokki Bank of Cyprus  \\
                \end{supertabular}
              \end{msoraclesql}
            \end{small}
          \end{center}
          \begin{merke}
            Die \COMMIT-Anweisung persistiert\footnote{persistent = dauerhaft}
  					die Aktionen einer Transaktion in der Datenbank. Ohne \COMMIT{} werden alle
  					\"Anderungen wieder zur\"uckgerollt.
          \end{merke}
        \subsubsection{COMMIT in Microsoft SQL Server}
          In Microsoft SQL Server muss dem \COMMIT-Kommando noch das
          Schlüsselwort \languagemssql{TRANSACTION} (oder
          \languagemssql{TRAN}) hinzugefügt werden. Dies beendet sowohl
          implizite als auch explizite Transaktionen.
        \begin{lstlisting}[language=ms_sql,caption={Eine implizite Transaktion
        committen},label=sql07_27a]
SET IMPLICIT_TRANSACTIONS ON
INSERT INTO Bank
VALUES      (21, 'NOSDEL21SES', 'Lokki Bank of Cyprus',
             'Strasse der Europaeischen Union', '3', '00000', 'Pleitingen',
             'D--');

SELECT Bank_ID, BIC, Name
FROM   Bank
WHERE  Bank_ID >= 18;

COMMIT TRAN;
        \end{lstlisting}
\clearpage
        \begin{lstlisting}[language=ms_sql,caption={Eine explizite Transaktion
        committen},label=sql07_27b]
BEGIN TRANSACTION
INSERT INTO Bank
VALUES      (21, 'NOSDEL21SES', 'Lokki Bank of Cyprus',
             'Strasse der Europaeischen Union', '3', '00000', 'Pleitingen',
             'D--');

SELECT Bank_ID, BIC, Name
FROM   Bank
WHERE  Bank_ID >= 18;
COMMIT TRAN;
        \end{lstlisting}
          
      \subsection{Eine Transaktion r\"uckg\"angig machen}
        \subsubsection{Das ROLLBACK-Kommando}
          Das Kommando \ROLLBACK{} stellt das Gegenst\"uck zu \COMMIT{} dar.
          Sollen die Aktionen einer Transaktion nicht dauerhaft in der Datenbank
          gespeichert werden, k\"onnen sie mit \ROLLBACK{} zur\"uckgerollt
          (r\"uckg\"angig gemacht) werden.
          \begin{lstlisting}[language=oracle_sql,caption={Eine Transaktion wird
        abgebrochen},label=sql07_28] SELECT Bank_ID, BIC, Name
FROM   Bank
WHERE  Bank_ID >= 18;
          \end{lstlisting}
          \begin{center}
            \begin{small}
              \changefont{pcr}{m}{n}
              \tablefirsthead {
                \multicolumn{1}{r}{\textbf{BANK\_ID}} &
                \multicolumn{1}{l}{\textbf{BIC}} &
                \multicolumn{1}{l}{\textbf{NAME}} \\
                \cmidrule(r){1-1}\cmidrule(l){2-2}\cmidrule(l){3-3}
              }
              \tablehead{}
              \tabletail {
                \multicolumn{3}{l}{\textbf{4 Zeilen ausgew\"ahlt}} \\
              }
              \tablelasttail {
                \multicolumn{3}{l}{\textbf{4 Zeilen ausgew\"ahlt}} \\
              }
              \begin{msoraclesql}
                \begin{supertabular}{rll}
                  18 & BVXYDE21SES & Bank der Landwirte \\
                  19 & BGIODE21SES & Austrailian Bank Association \\
                  20 & DFGHDE21SES & South Africa Bank \\
                  21 & NOSDEL21SES & Lokki Bank of Cyprus \\
                \end{supertabular}
              \end{msoraclesql}
            \end{small}
          \end{center}
          \begin{lstlisting}[language=oracle_sql,label=sql07_29]
DELETE FROM Bank
WHERE  Bank_ID = 21;

SELECT Bank_ID, BIC, Name
FROM   Bank
WHERE  Bank_ID >= 18;
          \end{lstlisting}
          \begin{center}
            \begin{small}
              \changefont{pcr}{m}{n}
              \tablefirsthead {
                \multicolumn{1}{r}{\textbf{BANK\_ID}} &
                \multicolumn{1}{l}{\textbf{BIC}} &
                \multicolumn{1}{l}{\textbf{NAME}} \\
                \cmidrule(r){1-1}\cmidrule(l){2-2}\cmidrule(l){3-3}
              }
              \tablehead{}
              \tabletail {
                \multicolumn{3}{l}{\textbf{3 Zeilen ausgew\"ahlt}} \\
              }
              \tablelasttail {
                \multicolumn{3}{l}{\textbf{3 Zeilen ausgew\"ahlt}} \\
              }
              \begin{msoraclesql}
                \begin{supertabular}{rll}
                  18 & BVXYDE21SES & Bank der Landwirte \\
                  19 & BGIODE21SES & Austrailian Bank Association \\
                  20 & DFGHDE21SES & South Africa Bank \\
                \end{supertabular}
              \end{msoraclesql}
            \end{small}
          \end{center}
          \begin{lstlisting}[language=oracle_sql,label=sql07_30]
ROLLBACK;

SELECT Bank_ID, BIC, Name
FROM   Bank
WHERE  Bank_ID >= 18;
          \end{lstlisting}
          \begin{center}
            \begin{small}
              \changefont{pcr}{m}{n}
              \tablefirsthead {
                \multicolumn{1}{r}{\textbf{BANK\_ID}} &
                \multicolumn{1}{l}{\textbf{BIC}} &
                \multicolumn{1}{l}{\textbf{NAME}} \\
                \cmidrule(r){1-1}\cmidrule(l){2-2}\cmidrule(l){3-3}
              }
              \tablehead{}
              \tabletail {
                \multicolumn{3}{l}{\textbf{4 Zeilen ausgew\"ahlt}} \\
              }
              \tablelasttail {
                \multicolumn{3}{l}{\textbf{4 Zeilen ausgew\"ahlt}} \\
              }
              \begin{msoraclesql}
                \begin{supertabular}{rll}
                  18 & BVXYDE21SES & Bank der Landwirte \\
                  19 & BGIODE21SES & Austrailian Bank Association \\
                  20 & DFGHDE21SES & South Africa Bank \\
                  21 & NOSDEL21SES & Lokki Bank of Cyprus \\
                \end{supertabular}
              \end{msoraclesql}
            \end{small}
          \end{center}
          \begin{merke}
            Die Anweisung \ROLLBACK{} rollt alle Aktionen einer Transaktion
            zur\"uck und beendet sie.
          \end{merke}
        \subsubsection{ROLLBACK in Microsoft SQL Server}
          Genau wie das \COMMIT-Kommando, muss auch das \ROLLBACK-Kommando um
          das Schlüsselwort \languagemssql{TRANSACTION} ergänzt werden.

    \chapter{Data Definition Language}
    \setcounter{page}{1}\kapitelnummer{chapter}
    \minitoc
\newpage
      Die Data definition language ist der Teil von SQL, der es erm\"oglicht, Objekte in der Datenbank zu erstellen und zu verwalten. DDL besteht im wesentlichen aus den vier Befehlen:
      \begin{itemize}
        \item \textbf{CREATE}: Erstellen von Objekten
        \item \textbf{ALTER}: \"Andern von Objekten
        \item \textbf{DROP}: Objekte l\"oschen
        \item \textbf{TRUNCATE}: Leeren von Tabellen.
      \end{itemize}
      Der Begriff des \enquote{Objekts} bezieht sich, je nach DBMS, auf die Unterschiedlichsten Dinge:
      \begin{itemize}
        \item Tabellen
        \item Views
        \item Indizes
        \item Sequenzen
        \item PL/SQL oder T-SQL Prozeduren und Funktionen
      \end{itemize}
      \dots und vieles mehr. Welche M\"oglichkeiten dem Anwender bei der Erstellung eines Objekts geboten werden, ist stark abh\"angig vom jeweiligen DBMS.
    \section{Tabellen erstellen und verwalten}
      \subsection{Namenskonventionen und Einschr\"ankungen}
        Bevor n\"aher auf die Namenskonventionen f\"ur Objekte eingegangen wird, m\"ussen an dieser Stelle zuerst einige Fachbegriffe gekl\"art werden.
        \begin{itemize}
          \item \textbf{Bezeichner}: Namen f\"ur Objekte (Tabellen, Spalten, Views, usw.) hei\ss{}en im Fachjargon Bezeichner.
          \item \textbf{Umschlossene Bezeichner}: Sind Bezeichner, die in Anf\"uhrungszeichen \char34{} eingeschlossen sind
          \item \textbf{Reservierte W\"orter}: Begriffe die in SQL eine bestimmte Bedeutung haben, z. B. \SELECT , \WHERE , usw.
          \item \textbf{Namensraum}: Logische Einteilung f\"ur Objektnamen. Bezeichner m\"ussen innerhalb eines Namensraumes eindeutig sein.
        \end{itemize}
        \tabelle{createtablerestrictions} listet die wichtigsten
        Einschr\"ankungen auf, die f\"ur Bezeichner in beiden DBMS gelten.
        \begin{center}
          \tablecaption{Einschr\"ankungen f\"ur Bezeichner}
          \label{createtablerestrictions}
          \begin{small}
            \tablefirsthead{
              \multicolumn{1}{c}{\textbf{}} &
              \multicolumn{1}{c}{\textbf{\includegraphics[scale=1]{oracle_11g}}} &
              \multicolumn{1}{c}{\textbf{\includegraphics[scale=1]{ms_sql}}} \\
              \hline
            }
            \tablehead{
              \multicolumn{1}{c}{\textbf{Einschr\"ankung}} &
              \multicolumn{1}{c}{\textbf{\includegraphics[scale=1]{oracle_11g}}} &
              \multicolumn{1}{c}{\textbf{\includegraphics[scale=1]{ms_sql}}} \\
              \hline
            }
            \tabletail{
              \hline
            }
            \tablelasttail{
              \hline
            }
            \begin{supertabular}{|l|p{5.5cm}|p{5.5cm}|}
              \textbf{Bezeichnerl\"ange} & 30 & 128\\
              \hline
              \textbf{Reservierte W\"orter} & Bezeichner k\"onnen keine reservierten W\"orter sein, es sei denn, sie sind in Anf\"uhrungszeichen \char34{} eingeschlossen. & Bezeichner k\"onnen keine reservierten W\"orter sein, es sei denn, sie sind in Anf\"uhrungszeichen \char34{} eingeschlossen. \\
              \hline
              \textbf{Namensgebung} & Wenn Bezeichner nicht in Anf\"uhrungszeichen (\char34{}) einschlossen sind, m\"ussen diese mit einem Buchstaben beginnen. F\"ur umschlossene Bezeichner gilt dies nicht. & Wenn Bezeichner nicht in Anf\"uhrungszeichen (\char34{}) oder ([]) einschlossen sind, m\"ussen diese mit einem Buchstaben, \_, @ oder \# beginnen. F\"ur umschlossene Bezeichner gilt dies nicht. \\
              \hline
              \textbf{G\"ultige Zeichen} & Nicht umschlossene Bezeichner k\"onnen nur aus den Buchstaben a-z und A-Z, den Ziffern 0-9, sowie \_, \$ und \# bestehen. F\"ur umschlossene Bezeichner gilt, dass dort alle Zeichen, auch Leerzeichen vorkommen k\"onnen. & Nicht umschlossene Bezeichner k\"onnen nur aus den Buchstaben a-z und A-Z, den Ziffern 0-9, sowie @, \$, \_ und \# bestehen. F\"ur umschlossene Bezeichner gilt, dass dort alle Zeichen, auch Leerzeichen vorkommen k\"onnen. \\
              \hline
              \textbf{Namensgleichheit} & Zwei Datenbankobjekte im gleichen Namensraum m\"ussen unterschiedliche Namen haben.  & Bezeichner m\"ussen innerhalb eines Schemas eindeutig sein.\\
              \hline
              \textbf{Casesensitivit\"at} & Nicht umschlossene Bezeichner sind nicht Casesensitiv. Bezeichner die mit (\char34{}) oder ([]) umschlossen sind, sind Casesensitiv. & Bezeichner die mit (\char34{}) oder ([]) umschlossen sind, sind nicht Casesensitiv. \\
            \end{supertabular}
          \end{small}
        \end{center}
        \begin{merke}
          Damit umschlossene Bezeichner in SQL Server 2008 R2 genutzt werden k\"onnen, muss die Option \textit{QUOTED\_IDENTIFIER} den Wert \textit{ON} haben. Dieser kann n\"otigenfalls mit \languagemssql{SET QUOTED_IDENTIFIER ON} gesetzt werden.
        \end{merke}
        Die folgenden Internetliteraturhinweise liefern weitere Informationen.
        \begin{literaturinternet}
          \item \cite{i27561}
          \item \cite{ms187879}
        \end{literaturinternet}
\clearpage
      \subsection{CREATE TABLE - Tabellen erstellen}
        Sowohl in Oracle als auch in SQL Server werden Tabellen mit Hilfe des
Kommandos \languageorasql{CREATE TABLE} erstellt. Die grundlegende,
SQL-Standardkonforme Syntax f\"ur \languageorasql{CREATE TABLE} lautet:
        \begin{lstlisting}[language=oracle_sql,caption={Die Syntax der CREATE TABLE-Anweisung},label=sql08_01]
CREATE TABLE <tabellen_name> (
  <spaltenbezeichner 1> <datentyp>,
  <spaltenbezeichner 2> <datentyp>,
  ...,
  <spaltenbezeichner n> <datentyp>
);
        \end{lstlisting}
        \begin{center}
          \tablecaption{Die CREATE TABLE-Anweisung}
          \label{createtablesyntax}
          \begin{small}
            \tablefirsthead{
              \multicolumn{1}{c}{\textbf{Ausdruck}} &
              \multicolumn{1}{c}{\textbf{Bedeutung}} \\
              \hline
            }
            \tablehead{
              \multicolumn{1}{c}{\textbf{Ausdruck}} &
              \multicolumn{1}{c}{\textbf{Bedeutung}} \\
              \hline
            }
            \tabletail{
              \hline
            }
            \begin{supertabular}{|l|p{9.45cm}|}
              CREATE TABLE <Tabellenname> & Diese Klausel leitet das Erstellen der Tabelle ein. F\"ur den Tabellennamen gelten die in \tabelle{createtablerestrictions} angegebenen Beschr\"ankungen. \\
              \hline
              <Spaltenbezeichner> <Datentyp> & Jede Tabellenspalte wird durch einen Bezeichner/Name und einen Datentyp repr\"asentiert. Mit Hilfe des Namens kann die Spalte sp\"ater angesprochen werden und der Datentyp legt den Wertebereich der Spalte fest. Je nach DBMS gelten auch hier unterschiedliche Einschr\"ankungen. \\
            \end{supertabular}
          \end{small}
        \end{center}
        \beispiel{sql08_02} zeigt ein einfaches \languageorasql{CREATE TABLE}-Statement.
        \begin{lstlisting}[language=oracle_sql,caption={Eine einfache CREATE TABLE-Anweisung in Oracle},label=sql08_02]
CREATE TABLE Aktie (
  Aktie_ID  NUMBER,
  Name      VARCHAR2(25),
  WKN       NUMBER,
  ISIN      VARCHAR2(12)
);
        \end{lstlisting}
        \begin{lstlisting}[language=ms_sql,caption={Das gleiche in MS SQL Server},label=sql08_03]
CREATE TABLE Aktie (
  Aktie_ID  NUMERIC,
  Name      VARCHAR(25),
  WKN       NUMERIC,
  ISIN      VARCHAR(12)
);
        \end{lstlisting}
        Es wird eine Tabelle namens \identifier{Aktie}, mit den Spalten \identifier{Aktie\_ID}, \identifier{Name}, \identifier{WKN} und \identifier{ISIN} angelegt.

        Zur besseren Umsetzung der Beispiele in den folgenden Abschnitten, werden nun einige Datens\"atze in die Tabelle \identifier{Aktie} eingef\"ugt.
        \begin{lstlisting}[language=oracle_sql,caption={Beispieldatens\"atze},label=sql08_04]
INSERT INTO Aktie
VALUES (1, 'Henker Co KG', 1236547, 'DE0006800002');

INSERT INTO Aktie
VALUES (2, 'AD and D', 43116589, 'DE0002300023');

COMMIT;
        \end{lstlisting}
      \subsection{CREATE TABLE AS... (CTAS)}
        Die Abk\"urzung \enquote{CTAS} steht f\"ur \languageorasql{CREATE TABLE AS} und meint ein \languageorasql{CREATE TABLE} mit Unterabfrage. Mit Hilfe von CTAS k\"onnen bestehende Tabellen teilweise oder ganz kopiert werden. \beispiel{sql08_05} zeigt, wie in Oracle eine vollst\"andige Kopie der Tabelle \identifier{Aktie} angefertigt wird.
        \begin{lstlisting}[language=oracle_sql,caption={Oracle - CREATE TABLE AS (CTAS)},label=sql08_05]
CREATE TABLE Aktie_Kopie
AS
  SELECT *
  FROM   Aktie;
        \end{lstlisting}
          Es wird eine Tabelle namens \identifier{Aktie\_Kopie} erstellt. Diese erh\"alt die komplette Struktur und den gesamten Inhalt der Tabelle \identifier{Aktie}.

          In Microsoft SQL Server kennt das \languagemssql{CREATE TABLE}-Statement keine M\"oglichkeit, eine Unterabfrage zu nutzen. Hier muss stattdessen das \languagemssql{SELECT INTO}-Statement genutzt werden.
        \begin{lstlisting}[language=ms_sql,caption={MS SQL Server - SELECT INTO},label=sql08_06]
SELECT *
INTO   Aktie_Kopie
FROM   Aktie;
        \end{lstlisting}
        Die Auswirkungen bleiben die gleichen, wie unter Oracle mit CTAS.
        \begin{merke}
          Microsoft SQL Server kennt das \languagemssql{CREATE TABLE AS}-Statement nicht. Es muss stattdessen das \languagemssql{SELECT INTO}-Statement genutzt werden. Die Auswirkungen von \languagemssql{CREATE TABLE AS} und \languagemssql{SELECT INTO} sind gleich.
        \end{merke}
      \subsection{ALTER TABLE - Tabellen ver\"andern}
        Mit Hilfe der \languageorasql{ALTER TABLE}-Anweisung k\"onnen bestehende Tabellendefinition ver\"andert werden. Dies betrifft z. B.:
        \begin{itemize}
          \item Das Hinzuf\"ugen neuer Spalten zu einer Tabelle.
          \item Das L\"oschen von Spalten.
          \item Das Umbenennen von Spalten.
          \item Das \"Andern des Datentyps einer Spalte.
          \item \"Andern der Gr\"o\ss e einer Spalte.
          \item Das Hinzuf\"ugen, \"andern und l\"oschen eines Standardwerts.
          \item Das Hinzuf\"ugen und L\"oschen von Constraints (siehe \abschnitt{constraints1})
        \end{itemize}
        \subsubsection{Eine neue Spalte an eine Tabelle anf\"ugen}
          In beiden DBMS gibt es, zum Hinzuf\"ugen einer Spalte zu einer Tabelle, die \lstinline{ADD}-Klausel des \languageorasql{ALTER TABLE}-Kommandos. In \beispiel{sql08_07}, wird der Tabelle \identifier{Aktie} eine neue Spalte namens \identifier{Herkunft} hinzugef\"ugt.
          \begin{lstlisting}[language=oracle_sql,caption={Oracle - Tabellenspalte hinzuf\"ugen},label=sql08_07]
ALTER TABLE Aktie
ADD Herkunft VARCHAR2(25);
          \end{lstlisting}
          In SQL Server unterscheidet sich dieses Statement nur durch den Datentyp.
          \begin{lstlisting}[language=ms_sql,caption={MS SQL Server - Tabellenspalte hinzuf\"ugen},label=sql08_08]
ALTER TABLE Aktie
ADD Herkunft VARCHAR(25);
          \end{lstlisting}
          \begin{merke}
            Wird eine neue Spalte an eine Tabelle angef\"ugt, haben alle Zellen dieser Spalte den Wert NULL, es sei den, es wird ein Standardwert f\"ur diese Spalte definert. In diesem Falle f\"ullt Oracle die Spalte mit dem Standardwert auf. SQL Server tut dies nicht.
          \end{merke}
           \beispiel{sql08_09} und \beispiel{sql08_10} zeigen, wie sich Oracle und MS SQL Server verhalten, wenn eine neue Spalte, mit einem Standardwert, hinzugef\"ugt wird. Die Spalte \identifier{Herkunft} wird mit dem Standardwert \enquote{Deutschland} an die Tabelle \identifier{Aktie} angef\"ugt. In Oracle werden dann automatisch alle bereits vorhandenen Zeilen mit dem neuen Standardwert aufgef\"ullt. In SQL Server wird dies nicht der Fall sein.
\clearpage
          \begin{lstlisting}[language=oracle_sql,caption={Tabellenspalte mit Standardwert hinzuf\"ugen in Oracle},label=sql08_09]
ALTER TABLE Aktie
ADD Herkunft VARCHAR2(25) DEFAULT 'Deutschland';

SELECT *
FROM   Aktie;
          \end{lstlisting}
          \begin{center}
            \begin{small}
              \changefont{pcr}{m}{n}
              \tablefirsthead {
                \multicolumn{1}{r}{\textbf{AKTIE\_ID}} &
                \multicolumn{1}{l}{\textbf{NAME}} &
                \multicolumn{1}{l}{\textbf{WKN}} &
                \multicolumn{1}{l}{\textbf{ISIN}} &
                \multicolumn{1}{l}{\textbf{HERKUNFT}} \\
                \cmidrule(r){1-1}\cmidrule(l){2-2}\cmidrule(l){3-3}\cmidrule(l){4-4}\cmidrule(l){5-5}
              }
              \tablehead{}
              \tabletail {
                \multicolumn{5}{l}{\textbf{2 Zeilen ausgew\"ahlt}} \\
              }
              \tablelasttail {
                \multicolumn{5}{l}{\textbf{2 Zeilen ausgew\"ahlt}} \\
              }
              \begin{oraclesql}
                \begin{supertabular}{rllll}
                  1 & Henker Co KG & 1236547 & DE0006800002 & Deutschland \\
                  2 & AD and D & 43116589 & DE0002300023 & Deutschland \\
                \end{supertabular}
              \end{oraclesql}
            \end{small}
          \end{center}

          Wird das gleiche Experiment in MS SQL Server durchgef\"uhrt, zeigt sich das die Spalte \identifier{Herkunft} nicht automatisch aufgef\"ullt wird.
          \begin{lstlisting}[language=ms_sql,caption={Tabellenspalte mit Standardwert hinzuf\"ugen in SQL Server},label=sql08_10]
ALTER TABLE Aktie
ADD Herkunft VARCHAR(25) DEFAULT 'Deutschland';

SELECT *
FROM   Aktie;
          \end{lstlisting}
          \begin{center}
            \begin{small}
              \changefont{pcr}{m}{n}
              \tablefirsthead {
                \multicolumn{1}{l}{\textbf{AKTIE\_ID}} &
                \multicolumn{1}{l}{\textbf{NAME}} &
                \multicolumn{1}{l}{\textbf{WKN}} &
                \multicolumn{1}{l}{\textbf{ISIN}} &
                \multicolumn{1}{l}{\textbf{HERKUNFT}} \\
                \cmidrule(l){1-1}\cmidrule(l){2-2}\cmidrule(l){3-3}\cmidrule(l){4-4}\cmidrule(l){5-5}
              }
              \tablehead{}
              \tabletail {
                \multicolumn{5}{l}{\textbf{2 Zeilen ausgew\"ahlt}} \\
              }
              \tablelasttail {
                \multicolumn{5}{l}{\textbf{2 Zeilen ausgew\"ahlt}} \\
              }
              \begin{oraclesql}
                \begin{supertabular}{lllll}
                  1 & Henker Co KG & 1236547 & DE0006800002 &  NULL \\
                  2 & AD and D & 43116589 & DE0002300023 & NULL \\
                \end{supertabular}
              \end{oraclesql}
            \end{small}
          \end{center}
        \subsubsection{Spalten vergr\"o\ss{}ern und verkleinern}
          Es besteht die M\"oglichkeit, die Definition einer Spalte nachtr\"aglich zu ver\"andern. Dabei k\"onnen verschiedene Dinge, wie z. B. der Spaltendatentyp oder der Standardwert einer Spalte ge\"andert werden. Um eine solche \"Anderung durchzuf\"uhren, kennt das \languageorasql{ALTER TABLE}-Kommando unter Oracle die \languageorasql{MODIFY}-Klausel und unter SQL Server die \languagemssql{ALTER COLUMN}-Klausel. Hierzu einige Beispiele.

          In \beispiel{sql08_11} wird die Breite der Spalte \identifier{Herkunft} in der Tabelle \identifier{Aktie} ver\"andert.
          \begin{lstlisting}[language=oracle_sql,caption={Anpassen der Spaltenl\"ange in Oracle},label=sql08_11]
ALTER TABLE Aktie
MODIFY Herkunft VARCHAR2(30);
          \end{lstlisting}
\clearpage
          Eine Vergr\"o\ss erung stellt prinzipiell niemals ein Problem dar. Schwieriger wird es hingegen, wenn eine Spalte verkleinert werden muss. In Oracle geht das nur dann, wenn die Inhalte der Spalte kleiner sind als die neue Spaltengr\"o\ss{}e. Anderenfalls antwortet Oracle mit der in \beispiel{sql08_12} sichtbaren Fehlermeldung:
          \begin{lstlisting}[language=oracle_sql,caption={Fehlermeldung beim verkleinern einer Spalte in Oracle},label=sql08_12]
MODIFY Herkunft VARCHAR2(15)
       *
FEHLER in Zeile 2:
ORA-01441: Spaltenlaenge kann nicht vermindert werden, weil ein Wert zu gross
ist
          \end{lstlisting}
          \begin{merke}
            Eine Tabellenspalte kann in Oracle nur auf die Gr\"o\ss{}e des gr\"o\ss{}ten darin enthaltenden Werts verkleinert werden. In SQL Server kann eine Tabellenspalte auch mit Inhalt verkleinert werden.
          \end{merke}
          Bei SQL Server muss lediglich die \languageorasql{MODIFY}-Klausel durch die \languagemssql{ALTER COLUMN}-Klausel ersetzt werden.
          \begin{lstlisting}[language=ms_sql,caption={Anpassen der Spaltenl\"ange in SQL Server},label=sql08_13]
ALTER TABLE Aktie
ALTER COLUMN Herkunft VARCHAR(30);
          \end{lstlisting}
        \subsubsection{\"Andern des Datentyps}
          Mit Hilfe der \languageorasql{MODIFY}-Klausel kann nicht nur die Gr\"o\ss e einer Spalte ver\"andert werden, sondern auch der Datentyp. In \beispiel{sql08_14} wird der Datentyp der Spalte \identifier{WKN} von \languageorasql{NUMBER} auf \languageorasql{VARCHAR2}, bzw. von \languagemssql{NUMERIC} auf \languagemssql{VARCHAR} ge\"andert.
          \begin{lstlisting}[language=oracle_sql,caption={\"Andern des Datentyps},label=sql08_14]
ALTER TABLE Aktie
MODIFY WKN VARCHAR2(10);
          \end{lstlisting}
          In SQL Server sieht das \"Andern des Datentyps einer Spalte sehr \"ahnlich aus.
          \begin{lstlisting}[language=ms_sql,caption={\"Andern des Datentyps},label=sql08_15]
ALTER TABLE Aktie
ALTER COLUMN WKN VARCHAR(10);
          \end{lstlisting}
          \begin{merke}
            Eine Tabellenspalte muss in Oracle leer sein, damit ihr Datentyp ver\"andert werden kann. In SQL Server kann der Datentyp einer Spalte auch mit Inhalt ver\"andert werden.
          \end{merke}
\clearpage
        \subsubsection{Einen Defaultvalue hinzuf\"ugen}
          Eine weitere Aktion die mit \languageorasql{MODIFY} bzw. \languagemssql{ALTER COLUMN} m\"oglich ist, ist das Hinzuf\"ugen, \"andern oder entfernen eines Standardwertes bei einer Tabellenspalte. In \beispiel{sql08_16} wird in Oracle der Standardwert der Spalte \identifier{Herkunft} von \enquote{Deutschland} auf \enquote{USA} ge\"andert.
          \begin{lstlisting}[language=oracle_sql,caption={Einen Standardwert \"andern},label=sql08_16]
ALTER TABLE Aktie
MODIFY Herkunft DEFAULT 'USA';
          \end{lstlisting}
          Mit der gleichen Anweisung kann der Standardwert eine Spalte, unter Oracle, nicht nur ge\"andert sondern auch hinzugef\"ugt werden. Das L\"oschen des Standardwertes geschieht, indem NULL als Standardwert zugewiesen wird.
          \begin{lstlisting}[language=oracle_sql,caption={Standardwert hinzuf\"ugen},label=sql08_17]
ALTER TABLE Aktie
MODIFY Herkunft DEFAULT NULL;
          \end{lstlisting}
          Bei SQL Server ist das L\"oschen eines Standardwerts anders als bei
          Oracle. In SQL Server wird ein Standardwert als sogenanntes
          Constraint\footnote{constraint engl. = Einschr\"ankung} gehandhabt.
          Deshalb wird diese Aktion zu einem sp\"ateren Zeitpunkt in
          \abschnitt{sqlserverdefaultconstraint} behandelt.
          \begin{merke}
            Wird in Oracle mit Hilfe \languageorasql{ADD}-Klausel eine Spalte
            mit Standardwert hinzuge\"ugt, werden alle NULL-Werte in der
            gleichen Spalte mit dem Standardwert aufgef\"ullt. Wird die Spalte
            dagegen mit der \languageorasql{MODIFY}-Klausel, nachtr\"aglich mit
            einem Default-Wert ausgestattet, bleiben alle NULL-Werte erhalten!
          \end{merke}

				\subsubsection{Tabellenspalten umbenennen}
          Es ist m\"oglich, bestehende Spalten umzubenennen. Daf\"ur wird in
          Oracle die \languageorasql{RENAME COLUMN}-Klausel des
          \languageorasql{ALTER TABLE}-Kommandos verwendet. In Microsoft SQL
          Server gibt es hierf\"ur eine gespeicherte Hilfsprozedur, welche das
          Umbenennen \"ubernimmt. Der neue Spaltenname muss innerhalb der
          Tabelle eindeutig sein und es d\"urfen keine anderen Operationen
          zusammen mit dem Umbenennen geschehen.
          \begin{lstlisting}[language=oracle_sql,caption={Tabellenspalte umbenennen in Oracle},label=sql08_18]
ALTER TABLE Aktie
RENAME COLUMN Name TO Bezeichnung;
          \end{lstlisting}
          \begin{lstlisting}[language=ms_sql,caption={Tabellenspalte umbenennen in SQL Server},label=sql08_19, emphstyle={[9]\color{red}},emph={[9]sp_rename}]
EXEC sp_rename 'Aktie.Name', 'Bezeichnung', 'COLUMN'
          \end{lstlisting}
          \begin{merke}
            F\"ur das Umbenennen von Objekten ist in SQL Server die gespeicherte
            Hilfsprozedur \languagemssql{sp\_rename} zust\"andig.
          \end{merke}
          Zu beachten ist, dass das Umbenennen einer Spalte Auswirkungen auf
          abh\"angige Objekte wie z. B. Views oder Trigger haben kann und
          deshalb mit gr\"o\ss ter Vorsicht durchzuf\"uhren ist.
        \subsubsection{Tabellenspalten l\"oschen}
          Tabellenspalten, die nicht mehr ben\"otigt werden, k\"onnen jeder
          Zeit gel\"oscht werden. Auf diese einfache Art und Weise kann
          Speicherplatz zu weiteren Nutzung freigegeben werden. Allgemein gilt
          als Einschr\"ankung beim L\"oschen einer Tabellenspalte:
          \begin{itemize}
            \item Die letzte Spalte in einer Tabelle kann nicht gel\"oscht
            werden. Es muss dann die gesamte Tabelle gel\"oscht werden.
          \end{itemize}
          F\"ur Oracle gilt zus\"atzlich:
          \begin{itemize}
            \item Ein normaler Nutzer kann keine Spalten aus einer Tabelle
            l\"oschen, die dem Nutzer sys geh\"ort.
          \end{itemize}
          \beispiel{sql08_19} zeigt das L\"oschen einer Tabellenspalte in Oracle und SQL Server.
          \begin{lstlisting}[language=oracle_sql, caption={Tabellenspalte
          l\"oschen},label=sql08_20]
ALTER TABLE Aktie
DROP COLUMN WKN;
          \end{lstlisting}
      \subsection{DROP TABLE - Tabellen l\"oschen}
        Eine nicht mehr ben\"otigte Tabelle, wird in Oracle und SQL Server mit dem \languageorasql{DROP TABLE}-Kommando gel\"oscht.
        \begin{itemize}
          \item Alle verkn\"upften Indizes und Trigger werden mitgel\"oscht.
          \item Alle abh\"angigen Views bleiben bestehen und werden ung\"ultig.
        \end{itemize}
        Das folgende Beispiel l\"oscht die Tabelle \identifier{Aktie}.
        \begin{lstlisting}[language=oracle_sql,caption={Eine Tabelle l\"oschen},label=sql08_21]
DROP TABLE Aktie;
        \end{lstlisting}
      \subsection{TRUNCATE TABLE - Tabellen leeren}
        Eine Tabelle kann mit \languageorasql{TRUNCATE TABLE} geleert werden. Die Tabelle selbst bleibt dabei erhalten. Um eine Tabelle zu leeren, gibt es drei M\"oglichkeiten:
        \begin{itemize}
          \item Das DML-Statement \languageorasql{DELETE}
          \item Das L\"oschen der Tabelle mit \languageorasql{DROP TABLE} und neu erstellen mit \languageorasql{CREATE TABLE}
          \item Das DDL-Statement \languageorasql{TRUNCATE}
        \end{itemize}
        \subsubsection{Den Tabelleninhalt mit DELETE l\"oschen}
          Es k\"onnen alle Zeilen einer Tabelle mit dem DML-Kommando \languageorasql{DELETE} gel\"oscht werden.
          \begin{lstlisting}[language=oracle_sql,caption={Zeilen mit DELETE l\"oschen},label=sql08_22]
DELETE FROM Aktie;
          \end{lstlisting}
          Bei einer gro\ss en Tabelle werden hierf\"ur sehr viele
          Systemressourcen ben\"otigt (CPU, RAM, usw.). Des Weiteren kann es
          passieren, dass beim L\"oschen von Zeilen, Trigger ausgel\"ost werden.
          \begin{merke}
            Der Speicherplatz, der durch die Tabelle vor dem L\"oschen belegt
            wurde, bleibt bei der Verwendung von \languageorasql{DELETE} belegt.
          \end{merke}
          Einziger Vorteil ist, dass mit der \languageorasql{DELETE}-Klausel die
          Zeilen ausgew\"ahlt werden k\"onnen, die gel\"oscht werden sollen.
        \subsubsection{Die Tabelle l\"oschen und neu erstellen}
          \label{dropandrecreatetable}
          Eine Tabelle kann gel\"oscht und mit \languageorasql{CREATE TABLE} neu
          erstellt werden. Dabei gehen alle mit dieser Tabelle verbundenen
          Indizes, Integrit\"ats Constraints und Trigger verloren und alle von
          der Tabelle abh\"angigen Objekte werden ung\"ultig.
        \subsubsection{Eine Tabelle mit TRUNCATE leeren}
          Um alle Zeilen einer Tabelle zu l\"oschen kann das \languageorasql{TRUNCATE}-Statement verwendet werden.
          \begin{lstlisting}[language=oracle_sql, caption={Zeilen mit TRUNCATE
          abschneiden},label=sql08_23]
TRUNCATE TABLE Aktie;
          \end{lstlisting}
\clearpage
          \begin{merke}
            In Oracle produziert das \languageorasql{TRUNCATE}-Statement, wie
            alle DDL-Statements, automatisch ein \languageorasql{COMMIT}, d. h.
            es kann nicht r\"uckg\"angig gemacht werden. In SQL Server ist das
            Zur\"uckrollen eines \languagemssql{TRUNCATE}-Statements m\"oglich.
          \end{merke}

          \begin{merke}
            Der Speicherplatz, der durch die Tabelle vor dem L\"oschen belegt
            wurde, wird bei der Verwendung von \languageorasql{TRUNCATE},
            freigegeben.
          \end{merke}
    \section{Views erstellen verwalten}
      \subsection{Was sind Views?}
        Bei der t\"aglichen Arbeit mit einer Datenbank treten h\"aufig immer wiederkehrende \SELECT-Statements auf. Dies kann z. B. deshalb sein, weil ein Nutzer immer wieder die gleiche Sicht (gleiche Spalten, gleiche Filterbedingung) auf die Daten einer Tabelle ben\"otigt.
        \begin{merke}
          Eine View ist eine genau definierte Sicht auf eine bestimmte Datenmenge.
        \end{merke}
      \subsection{Views erstellen}
        Views werden mit dem \languageorasql{CREATE VIEW}-Kommando erstellt. Die Syntax f\"ur \languageorasql{CREATE VIEW} sieht wie folgt aus:
        \begin{lstlisting}[language=oracle_sql,caption={Die Syntax von CREATE VIEW},label=sql08_24]
CREATE VIEW <View_name>
(<Spalten_alias 1, Spalten_alias 2, ..., Spalten_alias n)
AS
  <Auswahlabfrage>;
        \end{lstlisting}
        \begin{center}
          \tablecaption{Die CREATE VIEW-Anweisung}
          \label{createviewsyntax}
          \begin{small}
            \tablefirsthead{
              \multicolumn{1}{c}{\textbf{Ausdruck}} &
              \multicolumn{1}{c}{\textbf{Bedeutung}} \\
              \hline
            }
            \tablehead{
              \multicolumn{1}{c}{\textbf{Ausdruck}} &
              \multicolumn{1}{c}{\textbf{Bedeutung}} \\
              \hline
            }
            \tabletail{
              \hline
            }
            \tablelasttail{
              \hline
            }
            \begin{supertabular}{|l|p{9.45cm}|}
              CREATE VIEW <View\_name> & Diese Klausel leitet das Erstellen der View ein. F\"ur den Viewname gelten die in \tabelle{createtablerestrictions} angegebenen Beschr\"ankungen. \\
              \hline
              <Spalten\_alias> & F\"ur jeden Spaltenbezeichner, der in der Auswahlabfrage genutzt wird, kann an dieser Stelle ein Aliasname festgelegt werden.\\
              \hline
              <Auswahlabfrage> & Dies ist das SELECT-Statement. \\
            \end{supertabular}
          \end{small}
        \end{center}
        Ein einfaches Beispiel f\"ur das Erstellen einer View ist in \beispiel{sql08_24} zu sehen.
        \begin{lstlisting}[language=oracle_sql,caption={Eine einfache View},label=sql08_25]
CREATE VIEW v_Kunde
AS
  SELECT Vorname, Nachname
  FROM   Kunde;
        \end{lstlisting}
        Was an dieser Stelle passiert ist, dass das DBMS das \SELECT-Statement verarbeitet und unter dem Namen \identifier{v\_Kunden} abspeichert. Anschlie\ss end kann, wie in \beispiel{sql08_26}, mit SQL auf die View zugegriffen werden.
        \begin{lstlisting}[language=oracle_sql,caption={Zugriff auf eine View},label=sql08_26]
SELECT Vorname, Nachname
FROM   v_Kunde;
        \end{lstlisting}
        \begin{center}
          \begin{small}
            \changefont{pcr}{m}{n}
            \tablefirsthead {
              \multicolumn{1}{l}{\textbf{VORNAME}} &
              \multicolumn{1}{l}{\textbf{NACHNAME}} \\
              \cmidrule(l){1-1}\cmidrule(l){2-2}
            }
            \tablehead{}
            \tabletail {
              \multicolumn{2}{l}{\textbf{561 Zeilen ausgew\"ahlt}} \\
            }
            \tablelasttail {
              \multicolumn{2}{l}{\textbf{561 Zeilen ausgew\"ahlt}} \\
            }
            \begin{msoraclesql}
              \begin{supertabular}{ll}
                Niklas & Schneider \\
                Mia & Keller \\
                Lilli & Beck \\
                Emilia & Keller \\
                Finn & Junge \\
                Marie & Vogel \\
                Rudi & Roggatz \\
                Leni & Koch \\
                Chris & Zimmermann \\
                Justin & Gabriel \\
                Sebastian & Schr\"oder \\
              \end{supertabular}
            \end{msoraclesql}
          \end{small}
        \end{center}
        Auch wenn in der \SELECT-Klausel der Auswahlabfrage der * verwendet wird, wird im Hintergrund folgendes Statement, als View gespeichert:
        \begin{lstlisting}[language=oracle_sql,caption={Was tats\"achlich gespeichert wird},label=sql08_27]
-- So wird die View erstellt
CREATE VIEW v_Kunde
AS
  SELECT *
  FROM   Kunde;

-- Das wird gespeichert
SELECT Kunden_ID, Vorname, Nachname
FROM   Kunde;
        \end{lstlisting}
\clearpage
        Diese Tatsache ist nicht ganz unwichtig, wie folgendes Szenario beweist:
        \begin{lstlisting}[language=oracle_sql,caption={Eine Szenario mit T\"ucke},label=sql08_28]
CREATE VIEW v_Aktie
AS
  SELECT *
  FROM   Aktie;

ALTER TABLE Aktie
ADD Herkunft VARCHAR2(30);

SELECT *
FROM   v_Aktie;
        \end{lstlisting}
        \begin{center}
          \begin{small}
            \changefont{pcr}{m}{n}
            \tablefirsthead {
              \multicolumn{1}{r}{\textbf{AKTIE\_ID}} &
              \multicolumn{1}{l}{\textbf{NAME}} &
              \multicolumn{1}{l}{\textbf{WKN}} &
              \multicolumn{1}{l}{\textbf{ISIN}} \\
              \cmidrule(r){1-1}\cmidrule(l){2-2}\cmidrule(l){3-3}\cmidrule(l){4-4}
            }
            \tablehead{}
            \tabletail {
              \multicolumn{5}{l}{\textbf{2 Zeilen ausgew\"ahlt}} \\
            }
            \tablelasttail {
              \multicolumn{5}{l}{\textbf{2 Zeilen ausgew\"ahlt}} \\
            }
            \begin{msoraclesql}
              \begin{supertabular}{rllll}
                1 & Henker Co KG & 1236547 & DE0006800002  \\
                2 & AD and D & 43116589 & DE0002300023  \\
              \end{supertabular}
            \end{msoraclesql}
          \end{small}
        \end{center}
        Da bei der Erstellung der View \identifier{v\_Aktie}, der *, in die einzelnen Spalten der Tabelle \identifier{Aktie} aufgel\"ost wurde, ist die neu hinzugef\"ugte Spalte \identifier{Herkunft}, in der View \identifier{v\_Aktie} noch nicht zu sehen. Hierzu m\"usste die Viewdefinition ge\"andert bzw. die View neu erstellt werden.
        \begin{merke}
          Wird in der Auswahlabfrage einer View das * Symbol verwendet, wird dieses interpretiert. D. h. es wird ersetzt durch die tats\"achliche Spaltenliste der Quelltabelle. \"Anderungen an der Struktur der Tabelle werden somit von der View nicht erkannt.
        \end{merke}
        Wie in \tabelle{createviewsyntax} bereits erkl\"art, kann bei der Erstellung einer View auch eine Liste mit Spaltenaliasnamen angegeben werden. Dies ist in den folgenden F\"allen immer notwendig:
        \begin{itemize}
          \item Wenn in der View ein berechneter Ausdruck vorhanden ist
          \item Wenn in der View mehrere Tabellen mit einem Join verbunden sind und Spalten mit gleichem Namen ausgegeben werden m\"ussen.
        \end{itemize}
        \beispiel{sql08_29} zeigt eine View mit Spaltenaliasliste.
\clearpage
        \begin{lstlisting}[language=oracle_sql,caption={Eine einfache View mit Spaltenaliasliste},label=sql08_29]
CREATE VIEW v_Kunde
(Vorname, Nachname, Lebensalter)
AS
SELECT k.Vorname, k.Nachname,
       ROUND(MONTHS_BETWEEN(SYSDATE, Geburtsdatum) / 12, 0)
FROM   Kunde k INNER JOIN Eigenkunde ek
       ON (k.Kunden_ID = ek.Kunden_ID);

SELECT *
FROM   v_Kunde;
        \end{lstlisting}
        \begin{center}
          \begin{small}
            \changefont{pcr}{m}{n}
            \tablefirsthead {
              \multicolumn{1}{l}{\textbf{VORNAME}} &
              \multicolumn{1}{l}{\textbf{NACHNAME}} &
              \multicolumn{1}{r}{\textbf{LEBENSALTER}} \\
              \cmidrule(l){1-1}\cmidrule(l){2-2}\cmidrule(r){3-3}
            }
            \tablehead{}
            \tabletail {
              \multicolumn{3}{l}{\textbf{400 Zeilen ausgew\"ahlt}} \\
            }
            \tablelasttail {
              \multicolumn{3}{l}{\textbf{400 Zeilen ausgew\"ahlt}} \\
            }

            \begin{oraclesql}
              \begin{supertabular}{llr}
                Mia & Keller & 41 \\
                Emilia & Keller & 23 \\
                Finn & Junge & 37 \\
                Marie & Vogel & 42 \\
                Rudi & Roggatz & 26 \\
                Leni & Koch & 38 \\
                Chris & Zimmermann & 23 \\
                Sebastian & Schr\"oder & 24 \\
                Justin & Zimmermann & 34 \\
                Petra & Krause & 34 \\
                Clara & Rollert & 23 \\
                Gustav & Witte & 23 \\
              \end{supertabular}
            \end{oraclesql}
          \end{small}
        \end{center}
        Wie gut zu erkennen ist, ersetzen die Spaltenaliase die tats\"achlichen Spaltennamen in \identifier{v\_Kunde}. Die gleiche Auswirkung w\"are auch mit dem folgenden Statement zu erreichen:
        \begin{lstlisting}[language=oracle_sql,caption={Eine einfache View mit Spaltenaliasen},label=sql08_30]
CREATE VIEW v_Kunde
AS
SELECT k.Vorname, k.Nachname,
       ROUND(MONTHS_BETWEEN(SYSDATE, Geburtsdatum) / 12, 0) AS Lebensalter
FROM   Kunde k INNER JOIN Eigenkunde ek
       ON (k.Kunden_ID = ek.Kunden_ID);
        \end{lstlisting}
        \begin{merke}
          Wird eine Spaltenaliasliste genutzt, muss diese genauso viele Aliasnamen umfassen, wie die \SELECT-Liste der Auswahlabfrage Spaltennamen zur\"uckgibt.
        \end{merke}
        Hierzu ein kleines Beispiel. Im folgenden \languageorasql{CREATE VIEW}-Statement werden zu wenige Spaltenaliase angegeben. Oracle und auch SQL Server antworten prompt mit einer Fehlermeldung.
        \begin{lstlisting}[language=oracle_sql,caption={Eine einfache View mit fehlerhafter Spaltenaliasliste in Oracle},label=sql08_31]
CREATE VIEW v_Kunde
(Vorname, Nachname)
AS
SELECT k.Vorname, k.Nachname,
       ROUND(MONTHS_BETWEEN(SYSDATE, Geburtsdatum) / 12, 0)
FROM   Kunde k INNER JOIN Eigenkunde ek
       ON (k.Kunden_ID = ek.Kunden_ID);

(Vorname, Nachname)
 *
FEHLER in Zeile 2:
ORA-01730: Ung&\"u&ltige Anzahl an Spaltennamen angegeben
        \end{lstlisting}
        SQL Server antwortet wie folgt:
        \begin{lstlisting}[language=ms_sql,caption={Eine einfache View mit fehlerhafter Spaltenaliasliste in SQL Server},label=sql08_32]
CREATE VIEW v_Kunde
(Vorname, Nachname)
AS
SELECT k.Vorname, k.Nachname, DATEDIFF(YEAR, getDate(), Geburtsdatum)
FROM   Kunde k INNER JOIN Eigenkunde ek
       ON (k.Kunden_ID = ek.Kunden_ID);

Meldung 8158, Ebene 16, Status 1, Prozedur v_Kunde, Zeile 4
'v_Kunde' besitzt mehr Spalten, als in der Spaltenliste angegeben sind.
        \end{lstlisting}
        Bereits weiter oben in diesem Abschnitt wurde erl\"autert, dass eine View, in der ein berechneter Ausdruck vorkommt, zwingend mit Spaltenaliasen versehen werden muss. \beispiel{sql08_33} beweist dies:
        \begin{lstlisting}[language=oracle_sql,caption={Eine View mit einer berechneten Spalte in Oracle},label=sql08_33]
CREATE VIEW v_Kunde
(Vorname, Nachname)
AS
SELECT k.Vorname, k.Nachname,
       ROUND(MONTHS_BETWEEN(SYSDATE, Geburtsdatum) / 12, 0)
FROM   Kunde k INNER JOIN Eigenkunde ek
       ON (k.Kunden_ID = ek.Kunden_ID);

SELECT k.Vorname, k.Nachname,
ROUND(MONTHS_BETWEEN(SYSDATE, Geburtsdatum) / 12, 0)
                                   *
FEHLER in Zeile 3:
ORA-00998: Dieser Ausdruck braucht einen Spalten-Alias
        \end{lstlisting}
\clearpage
        Auch SQL Server hat hiermit Probleme:
        \begin{lstlisting}[language=ms_sql,caption={Eine View mit einer berechneten Spalte in SQL Server},label=sql08_34]
CREATE VIEW v_Kunde (Vorname, Nachname)
AS
SELECT k.Vorname, k.Nachname, DATEDIFF(YEAR, getDate(), Geburtsdatum)
FROM   Kunde k INNER JOIN Eigenkunde ek ON (k.Kunden_ID = ek.Kunden_ID);

Meldung 4511, Ebene 16, Status 1, Prozedur v_Kunde, Zeile 3
Fehler beim Ausf&\"u&hren von CREATE VIEW oder CREATE FUNCTION, da
f&\"u&r die 3-Spalte kein Spaltenname angegeben wurde.
        \end{lstlisting}
        Dieses Problem kann durch eine Spaltenaliasliste oder durch die direkte Vergabe eines Spaltenalias gel\"ost werden.
        \begin{merke}
          Bei SQL Server gibt es noch die Einschr\"ankung, dass die Auswahlabfrage einer View keine \ORDERBY-Klausel enthalten darf.
        \end{merke}
      \subsection{Views und DML}
        Views k\"onnen  auch f\"ur die Ausf\"uhrung von DML-Statements verwendet werden. Dabei gibt es jedoch einige Einschr\"ankungen und Regeln die zu beachten sind.
        \begin{center}
          \tablecaption{Regeln f\"ur DML-Operationen auf Views}
          \label{rulesdmlviews}
          \begin{small}
            \tablefirsthead{
              \multicolumn{1}{c}{} &
              \multicolumn{1}{c}{\textbf{\includegraphics[scale=1]{oracle_11g}}} &
              \multicolumn{1}{c}{\textbf{\includegraphics[scale=1]{ms_sql}}} &
              \multicolumn{1}{c}{\textbf{\includegraphics[scale=1]{oracle_11g}}} &
              \multicolumn{1}{c}{\textbf{\includegraphics[scale=1]{ms_sql}}} &
              \multicolumn{1}{c}{\textbf{\includegraphics[scale=1]{oracle_11g}}} &
              \multicolumn{1}{c}{\textbf{\includegraphics[scale=1]{ms_sql}}} \\
              \multicolumn{1}{c}{Einschr\"ankung} &
              \multicolumn{2}{c}{INSERT} &
              \multicolumn{2}{c}{UPDATE} &
              \multicolumn{2}{c}{DELETE} \\
              \hline
            }
            \tabletail{
              \hline
            }
            \tablelasttail {
              \hline
            }
            \begin{supertabular}{|p{8.5cm}|c|c|c|c|c|c|}
              Es d\"urfen keine Aggregatfunktionen (COUNT, SUM, MAX, MIN, AVG) in der View genutzt werden. & X & X & X & X & X & X \\
              \hline
              Die View darf keine GROUP BY-Klausel enthalten. & X & X & X & X & X & X \\
              \hline
              Die View darf das DISTINCT-Schl\"usselwort nicht benutzen. & X & X & X & X & X & X\\
              \hline
              Die View darf keine berechneten Ausdr\"ucke aufweisen. & X & X & X & X & & \\
              \hline
              Die View darf keine Pseudospalten enthalten & X & & X & & X & \\
              \hline
              Die View darf keinen Join enthalten. & X & X & & & X & X \\
              \hline
              Alle mit NOT NULL markierten Spalten der Basistabelle m\"ussen im INSERT-Statement ber\"ucksichtigt werden. & X & X & & & & \\
              \hline
              Der Einf\"uge- bzw. \"Anderungsvorgang muss, falls eine CHECK-Option in der View vorhanden ist (siehe \abschnitt{CHECK}), den Vorgaben der WHERE-Klausel der Abfrage gen\"ugen. & X & X & X & X & & \\
              \hline
              Die View darf keine READ ONLY-Option (siehe \abschnitt{READONLY}) enthalten. & X & & X & & &\\
            \end{supertabular}
          \end{small}
        \end{center}
        \subsubsection{Die Einschr\"ankung WITH CHECK OPTION}
          \label{CHECK}
          Bei der Erstellung einer View kann eine zus\"atzliche Einschr\"ankung mit angegeben werden, die \languageorasql{CHECK OPTION}. Diese schr\"ankt den Nutzer dahingehend ein, dass nur noch solche Datens\"atze ge\"andert werden k\"onnen, die auch in der View zu sehen sind.
          \begin{lstlisting}[language=oracle_sql,caption={Ein Experiment mit den CHECK OPTION},label=sql08_35]
CREATE VIEW v_Mitarbeiter
AS
  SELECT *
  FROM   Mitarbeiter
  WHERE  Bankfiliale_ID = 5;

INSERT INTO v_Mitarbeiter
VALUES (666, 'Florian', 'Weidinger', 12, 8,
        TO_DATE('01.03.1988', 'DD.MM.YYYY'),
        '38B546C1-CDF-36A7B97', 1500, 'Abendrot Gase',
        '13', '39444', 'Hecklingen', 20);

1 Zeile eingef&\"u&gt.

ROLLBACK;
          \end{lstlisting}
          Obwohl die \WHERE-Klausel der View \identifier{v\_Mitarbeiter} die Anzeige auf die Bankfiliale mit der ID f\"unf einschr\"ankt, kann trotzdem ein Datensatz in die Bankfiliale Nummer acht eingef\"ugt werden.

          Um die DML-M\"oglichkeiten der View \identifier{v\_Mitarbeiter} einzuschr\"anken, wird im n\"achsten Beispiel die \languageorasql{CHECK}-Option angewendet.
          \begin{lstlisting}[language=oracle_sql,caption={Ein Experiment mit der CHECK OPTION in Oracle},label=sql08_36]
CREATE VIEW v_Mitarbeiter
AS
  SELECT *
  FROM   Mitarbeiter
  WHERE  Bankfiliale_ID = 5
WITH CHECK OPTION;

INSERT INTO v_Mitarbeiter
VALUES (666, 'Florian', 'Weidinger', 12, 8,
        TO_DATE('01.03.1988', 'DD.MM.YYYY'),
        '38B546C1-CDF-36A7B97', 1500, 'Abendrot Gase',
        '13', '39444', 'Hecklingen', 20);

INSERT INTO v_Mitarbeiter
         *
FEHLER in Zeile 1:
ORA-01402: Verletzung der where-Klausel einer View with check option
          \end{lstlisting}
          Da jetzt die \languageorasql{CHECK}-Option genutzt wurde, reagiert das DBMS mit einer Fehlermeldung auf DML-Statements, die sich auf \identifier{v\_Mitarbeiter} beziehen und nicht der \WHERE-Klausel der View entsprechen.
          \begin{lstlisting}[language=ms_sql,caption={Ein Experiment mit der CHECK OPTION in SQL Server},label=sql08_37]
CREATE VIEW v_Mitarbeiter
AS
  SELECT *
  FROM   Mitarbeiter
  WHERE  Bankfiliale_ID = 5
WITH CHECK OPTION;

INSERT INTO v_Mitarbeiter
VALUES (666, 'Florian', 'Weidinger', 12, 8,
        CONVERT(DATETIME2, '01.03.1988', 104),
        '38B546C1-CDF-36A7B97', 1500, 'Abendrot Gase',
        '13', '39444', 'Hecklingen', 20);

Meldung 550, Ebene 16, Status 1, Zeile 1
Fehler beim Einf&\"u&gen oder Aktualisieren, da die Zielsicht WITH CHECK OPTION
angibt oder sich auf eine Sicht erstreckt, die WITH CHECK OPTION angibt,
und mindestens eine Ergebniszeile nicht der CHECK OPTION-Einschr&\"a&nkung
entsprach.
          \end{lstlisting}
        \subsubsection{Die Einschr\"ankung WITH READ ONLY - Oracle}
          \label{READONLY}
          Die \languageorasql{READ ONLY}-Option f\"ur Views erm\"oglicht es, einem Nutzer den Schreibzugriff auf eine View zu verbieten. Die View kann somit nur noch lesend genutzt werden.
          \begin{lstlisting}[language=oracle_sql,caption={Eine View mit mit READ ONLY Option erstellen},label=sql08_38]
CREATE VIEW v_Mitarbeiter
AS
  SELECT *
  FROM   Mitarbeiter
WITH READ ONLY;
          \end{lstlisting}
          Versucht ein Nutzer trotzdem mit einem DML-Statement auf die View zuzugreifen, wird er mit einer Fehlermeldung abgewiesen.
\clearpage
          \begin{lstlisting}[language=oracle_sql,caption={Daten in eine READ ONLY View einf\"ugen schl\"agt fehl},label=sql08_39]
INSERT INTO v_Mitarbeiter
VALUES (666, 'Florian', 'Weidinger', 12, 8,
        TO_DATE('01.03.1988', 'DD.MM.YYYY'),
        '38B546C1-CDF-36A7B97', 1500, 'Abendrot Gase',
        '13', '39444', 'Hecklingen', 20);

INSERT INTO v_Mitarbeiter
*
FEHLER in Zeile 1:
ORA-42399: cannot perform a DML operation on a read-only view
          \end{lstlisting}
          Um diese Option wieder von der View zu nehmen, muss die View neu erstellt werden (siehe \abschnitt{alterview})
      \subsection{Views \"andern}
        \label{alterview}
        M\"ussen an einer View Ver\"anderungen vorgenommen werden, bedeutet dies immer, dass die View neu erstellt werden muss. Oracle und SQL Server kennen hierzu unterschiedliche Wege:
        \begin{itemize}
          \item In Oracle wird die \languageorasql{CREATE VIEW}-Klausel erweitert: \languageorasql{CREATE OR REPLACE VIEW}.
          \item SQL Server benutzt hierf\"ur die \languagemssql{ALTER VIEW}-Anweisung.
        \end{itemize}
        Die beiden Beispiele \beispiel{sql08_40} und \beispiel{sql08_41} zeigen, wie in Oracle und SQL Server eine Viewdefinition ge\"andert werden kann.
        \begin{lstlisting}[language=oracle_sql,caption={Eine View \"andern in Oracle},label=sql08_40]
-- Zuerst wird die View erstellt
CREATE VIEW v_Mitarbeiter
AS
  SELECT *
  FROM   Mitarbeiter
WITH READ ONLY;

-- Dann wird sie geaendert
CREATE OR REPLACE VIEW v_Mitarbeiter
AS
  SELECT *
  FROM   Mitarbeiter;
        \end{lstlisting}
\clearpage
        Und nun SQL Server.
        \begin{lstlisting}[language=ms_sql,caption={Eine View \"andern in SQL
Server},label=sql08_41]
-- Zuerst wird die View erstellt
CREATE VIEW v_Mitarbeiter
AS
  SELECT *
  FROM   Mitarbeiter;

-- Dann wird sie ge&\"a&ndert
ALTER VIEW v_Mitarbeiter
AS
  SELECT *
  FROM   Mitarbeiter
  WHERE  Bankfiliale_ID = 5;
        \end{lstlisting}
      \subsection{Views l\"oschen}
        Zum L\"oschen von Views gibt es das Kommando \languageorasql{DROP VIEW}.
        \begin{lstlisting}[language=ms_sql,caption={Eine View l\"oschen},label=sql08_42]
DROP VIEW viw_countries;
        \end{lstlisting}

    \input{../sql/uebungen/sql_08_data_definition_language_uebungen}
    \input{../sql/loesungen/sql_08_data_definition_language_loesungen}
    \chapter{Constraints}
    \setcounter{page}{1}\kapitelnummer{chapter}
    \minitoc
\newpage
    \section{Was sind Constraints}
      \label{constraints1}
      Der englische Begriff \enquote{Constraint} bedeutet \"ubersetzt soviel wie: \enquote{Einschr\"ankung} oder \enquote{Zwang}. Constraints werden in Datenbankmanagementsystemen verwendet, um genau definierte Richtlinien f\"ur die Erfassung und die Verwaltung der Daten zu schaffen. Sie sorgen z. B. daf\"ur, dass manche Spalten immer zwingend einen Wert ungleich NULL haben m\"ussen oder das sie nur eindeutige Werte aufnehmen k\"onnen. Es ist auch m\"oglich einen genauen Wertebereich f\"ur eine Spalte zu definieren oder Werte aus Spalten anderer Tabellen zu referenzieren. Oracle und MS SQL Server kennen f\"unf Constraints f\"ur das relationale Datenmodell:
      \begin{itemize}
        \item \textbf{CHECK}: Definiert einen exakten Wertebereich f\"ur eine Spalte.
        \item \textbf{NOT NULL}: Definiert eine Spalte so, dass sie zwingend immer einen Wert ungleich NULL enthalten muss.
        \item \textbf{UNIQUE}: Legt fest, dass die Werte einer Spalte oder einer Kombination von Spalten eindeutig sein m\"ussen.
        \item \textbf{PRIMARY KEY}: Hat die Aufgabe, ein eindeutiges Identifikationsmerkmal f\"ur jede Zeile einer Tabelle darzustellen. Er ist eine Kombination aus dem \NOTNULL- und dem \UNIQUE-Constraint und kann sich ebenfalls auf eine Kombination von Spalten beziehen.
        \item \textbf{FOREIGN KEY}: Referenziert eine Spalte einer anderen Tabelle, die mit einem \UNIQUE- oder \PRIMARYKEY-Constraint versehen sein muss und stellt somit die referentielle Integrit\"at (siehe \abschnitt{refint}) der Datenbank sicher.
      \end{itemize}
      Zus\"atzlich zu diesen f\"unf kennt Oracle noch das
      \enquote{REF}-Constraint, das jedoch nur im Rahmen der objektorientierten
      Anteile von Oracle Bedeutung hat und hier keine weitere Erw\"ahnung
      findet. SQL Server kennt zus\"atzlich noch ein weiteres Constraint: das
      \languagemssql{DEFAULT}-Constraint.
    \section{Die Constraints}
      Constraints k\"onnen mit Hilfe der beiden Kommandos \languageorasql{CREATE
      TABLE} und \languageorasql{ALTER TABLE} angelegt werden. Sie werden durch
      einen Bezeichner und ihren Typ repr\"asentiert. Die Bezeichner von
      Constraints unterliegen ebenfalls den in \tabelle{createtablerestrictions}
      beschriebenen Regeln.
\clearpage
      \begin{merke}
        Wird f\"ur ein Constraint kein Name festgelegt, legt Oracle automatisch
        einen Namen nach dem Schema \enquote{SYS\_Cn} fest, wobei n eine
        sechstellige Zufallszahl darstellt  z. B. SYS\_C168349. SQL Server
        verwendet ein Namensschema mit dem Aufbau
        \enquote{typ\_\_tabelle\_\_spalte\_\_n} wobei n eine eindeutige
        hexadezimal Nummer darstellt, z. B.
        PK\_\_mitarbeiter\_\_mitarbeiter\_id\_\_4B561A78.
      \end{merke}
      In einem \languageorasql{CREATE TABLE}-Kommando k\"onnen Constraints als \enquote{Inline Constraint} und als \enquote{Out Of Line Constraint} angelegt werden.
      \begin{lstlisting}[language=oracle_sql,caption={Constraints erstellen},label=sql09_01]
CREATE TABLE <Tabellenname>(
  <Spalte 1> <Datentyp> CONSTRAINT <Inline Constraint Name> <Constraint Typ>,
  <Spalte 2> <Datentyp> CONSTRAINT <Inline Constraint Name> <Constraint Typ>,
  ...
  <Spalte n> <Datentyp> CONSTRAINT <Inline Constraint Name> <Constraint Typ>,
  CONSTRAINT <Out Of Line Constraint Name> <Constraint Typ> <Spalte 1, Spalte n>
  CONSTRAINT <Out Of Line Constraint Name> <Constraint Typ> <Spalte>
);
      \end{lstlisting}
      \begin{merke}
        Wird ein Constraint direkt mit der Definition einer Spalte angelegt, wird es als Inline Constraint bezeichnet und bezieht sich auf die Spalte mit der es definiert wurde. Wird ein Constraint im Anschluss an die Spaltendefinitionen angelegt, wird es als Out Of Line Constraint bezeichnet und kann sich auf mehrere Spalten beziehen.
      \end{merke}
      \subsection{Das CHECK-Constraint}
        Das \CHECK-Constraint hat die Aufgabe einen genauen Wertebereich f\"ur
        eine Spalte festzulegen. Beispielsweise w\"are ein \CHECK-Constraint
        auf der Spalte \identifier{Gehalt} der Tabelle \identifier{Mitarbeiter}
        sinnvoll, das definiert, dass Geh\"alter niemals negativ und niemals
        \"uber 90000 EUR sein k\"onnen.

        In \beispiel{sql09_02} wird gezeigt, wie die oben genannte
        Einschr\"ankung f\"ur die \identifier{Gehalt}-Spalte der Tabelle
        \identifier{Mitarbeiter} als Out Of Line Constraint angelegt wird.
        \begin{lstlisting}[language=oracle_sql,caption={Ein \CHECK-Constraint als Out Of Line Constraint},label=sql09_02]
ALTER TABLE Mitarbeiter
ADD CONSTRAINT gehalt_ck CHECK (Gehalt > 0 AND Gehalt <= 90000);
        \end{lstlisting}
        Um ein \CHECK-Constraint als Inline Constraint anzulegen, muss es direkt bei der Tabellenerstellung mit angelegt werden. \beispiel{sql09_03} zeigt das gleiche Constraint nocheinmal, aber als Inline Constraint.
        \begin{lstlisting}[language=oracle_sql,caption={Ein \CHECK-Constraint als Inline Constraint},label=sql09_03]
CREATE TABLE Mitarbeiter (
...
  Gehalt         NUMBER(12,2)
    CONSTRAINT gehalt_ck (Gehalt> 0 AND Gehalt <= 90000),
...
);
        \end{lstlisting}
        \begin{merke}
          In welchem Format ein \CHECK-Constraint angelegt wird, ob als Inline oder als Out Of Line Constraint, spielt keine Rolle. Beide Formen sind m\"oglich. Der Unterschied besteht darin, das sich ein Inline Constraint nur auf die Spalte beziehen kann, mit deren Definition es angelegt wurde. Ein Out Of Line Constraint kann sich auf alle Spalten der Tabelle beziehen, mit der zusammen es angelegt wurde.
        \end{merke}
        Um die Auswirkungen des obigen Merksatzes zu zeigen, wird das \identifier{gehalt\_ck}-Constraint ein wenig modifiziert. Es muss jetzt auch die Spalte \identifier{Provision} mit einbezogen werden. Das Gesamtgehalt eines Mitarbeiters darf 90.000 EUR nicht \"uberschreiten, die Provision mit eingerechnet.
        \begin{lstlisting}[language=oracle_sql,caption={Ein komplexes \CHECK-Constraint},label=sql09_04]
ALTER TABLE Mitarbeiter
ADD CONSTRAINT gehalt_ck CHECK (Gehalt > 0
               AND (Gehalt + (Gehalt * Provision / 100))  <= 90000);
        \end{lstlisting}
      \subsection{Das NOT NULL-Constraint}
        Das \NOTNULL-Constraint ist daf\"ur zust\"andig sicherzustellen, dass beim Einf\"ugen oder \"Andern einer Tabellenzeile bestimmte Spalten immer einen Wert haben m\"ussen.
        \begin{merke}
          Das \NOTNULL-Constraint stellt eine Ausnahme zu allen anderen Constraints dar, denn es kann nur als Inline Constraint angelegt werden.
        \end{merke}
        \beispiel{sql09_05} zeigt, wie ein \NOTNULL-Constraint angelegt wird.
        \begin{lstlisting}[language=oracle_sql,caption={Ein \NOTNULL-Constraint anlegen in Oracle},label=sql09_05]
ALTER TABLE Mitarbeiter
MODIFY Gehalt CONSTRAINT gehalt_nn NOT NULL;
        \end{lstlisting}
        Um ein solches Constraint wieder r\"uckg\"angig zu machen, kann die folgende Kurzform verwendet werden:
        \begin{lstlisting}[language=oracle_sql,caption={Das Gegenteil von \NOTNULL},label=sql09_06]
ALTER TABLE Mitarbeiter
MODIFY Gehalt NULL;
        \end{lstlisting}
        In den meisten DBMS wird ein \NOTNULL-Constraint intern als
\CHECK-Constraint umgesetzt, weshalb \beispiel{sql09_05} und
\beispiel{sql09_07} gleichbedeutend sind.
        \begin{lstlisting}[language=oracle_sql,caption={Die alternative Form eines \NOTNULL-Constraints in Oracle},label=sql09_07]
ALTER TABLE Mitarbeiter
ADD CONSTRAINT gehalt_nn CHECK (Gehalt IS NOT NULL);
        \end{lstlisting}
        In beiden F\"allen wird intern ein \CHECK-Constraint, nach dem in \beispiel{sql09_07} gezeigten Schema, angelegt. Auch in SQL Server ist dies der Fall. Im Gegensatz zu Oracle, muss bei SQL Server immer der Datentyp der Spalte mit angegeben werden, wenn eine Spalte ein \NOTNULL-Constraint erh\"alt.
        \begin{lstlisting}[language=ms_sql,caption={Ein \NOTNULL{} Constraint
anlegen in SQL Server},label=sql09_08]
ALTER TABLE Mitarbeiter
ALTER COLUMN Gehalt NUMERIC(12,2) NOT NULL;
        \end{lstlisting}
        \begin{lstlisting}[language=ms_sql,caption={Die alternative Form eines
\NOTNULL{} Constraints in SQL Server},label=sql09_09]
ALTER TABLE Mitarbeiter
ADD CONSTRAINT gehalt_nn CHECK Gehalt IS NOT NULL;
        \end{lstlisting}
        \begin{merke}
          Um in SQL Server eine Spalte mit einem \NOTNULL-Constraint zu belegen, muss der Datentyp der Spalte mit angegeben werden, auch wenn dieser sich nicht \"andern soll!
        \end{merke}
      \subsection{Das UNIQUE-Constraint}
        Das \UNIQUE-Constraint hat die Aufgabe, daf\"ur Sorge zu tragen, dass alle Werte, die in eine Tabellenspalte eingetragen werden, eindeutig sind.
        \begin{merke}
          In Oracle sind NULL-Werte eindeutig. Das hei\ss{}t, in einer mit einem
          \UNIQUE-Constraint belegten Spalte k\"onnen beliebig viele NULL-Werte
          vorkommen. In SQL Server sind NULL-Werte nicht eindeutig. Somit kann
          in SQL Server nur ein NULL-Wert pro Tabellenspalte vorkommen, wenn
          die Spalte mit einem \UNIQUE-Constraint belegt ist.
        \end{merke}
        \beispiel{sql09_10} zeigt, wie in Oracle und SQL Server ein
        \UNIQUE-Constraint auf die Spalte \identifier{SozVersNr} der Tabelle
        \identifier{Mitarbeiter} gelegt wird.
\clearpage
        \begin{lstlisting}[language=oracle_sql,caption={Ein UNIQUE-Constraint anlegen},label=sql09_10]
ALTER TABLE Mitarbeiter
ADD CONSTRAINT sozversnr_uk UNIQUE (SozVersNr);
        \end{lstlisting}
        Wie bereits beim \CHECK-Constraint gezeigt, kann auch ein \UNIQUE-Constraint als Inline Constraint erstellt werden. \beispiel{sql09_11} zeigt diesen Vorgang. Die Syntax ist in Oracle und SQL Server gleich.
        \begin{lstlisting}[language=oracle_sql,caption={Ein \UNIQUE-Constraint als Inline Constraint anlegen},label=sql09_11]
CREATE TABLE Mitarbeiter (
...
  SozVersNr       VARCHAR2(20)
    CONSTRAINT sozversnr_uk UNIQUE,
...
);
        \end{lstlisting}
        Oftmals gen\"ugt es nicht, wenn der Wert einer Spalte eindeutig ist. Es kann auch sein, dass die Kombination mehrerer Werte aus mehreren Spalten eindeutig sein muss. In so einem Fall kann ein \UNIQUE-Constraint auch auf eine Kombination mehrerer Spalten gelegt werden, wie \beispiel{sql09_12} zeigt.
        \begin{lstlisting}[language=oracle_sql,caption={Ein kombiniertes UNIQUE-Constraint anlegen},label=sql09_12]
ALTER TABLE Mitarbeiter
ADD CONSTRAINT mitarbeiter_uk UNIQUE (Vorname, Nachname, SozVersNr);
        \end{lstlisting}
      \subsection{Das PRIMARY KEY-Constraint}
        Das \PRIMARYKEY-Constraint hat eine ganz besondere Aufgabe. Es ist daf\"ur zust\"andig, ein Attribut oder eine Gruppe von Attributen einer Tabelle als eindeutig zu kennzeichnen, um so ein Identifikationsmerkmal f\"ur jede Tabellenzeile einer Tabelle zu schaffen.

        Die Nutzung von Prim\"arschl\"usseln ist notwendig, da es eine wesentliche Leistung eines relationalen Datenbankmanagementsystems ist, die Datenkonsistenz zu gew\"ahrleisten und hierzu geh\"ort auch das Vermeiden von redundanten Datens\"atzen.
        \begin{merke}
          Der Unterschied zwischen einem \UNIQUE-Constraint und einem \PRIMARYKEY-Constraint ist, dass ein \PRIMARYKEY-Constraint keine NULL-Werte zul\"asst. Ein \PRIMARYKEY-Constraint ist eine Mischung aus einem \NOTNULL- und einem \UNIQUE-Constraint.
        \end{merke}
        Da eine relationale Datenbank nicht ohne \PRIMARYKEY-Constraints
        auskommt, werden diese meist schon bei der Erstellung einer Tabelle
        angelegt.
\clearpage
        \begin{lstlisting}[language=oracle_sql,caption={Ein PRIMARY KEY-Constraint als Inline Constraint anlegen},label=sql09_13]
CREATE TABLE Mitarbeiter (
  Mitarbeiter_ID      NUMBER CONSTRAINT mitarbeiter_pk PRIMARY KEY,
...
);
        \end{lstlisting}
        Genau wie bei einem \UNIQUE-Constraint, kann es notwendig sein, einen Prim\"aschl\"ussel nicht nur auf eine Spalte, sondern auf eine Gruppe von Spalten zu legen. Dies ist meist in schwachen Entit\"aten der Fall, da hier die Kombination zweier Prim\"arschl\"ussel aus den beiden \"au\ss eren Entit\"aten als Prim\"arschl\"ussel genutzt wird.
        \begin{lstlisting}[language=oracle_sql,caption={Ein PRIMARY KEY-Constraint als Out Of Line Constraint auf mehrere Spalten anlegen},label=sql09_14]
CREATE TABLE Mitarbeiter (
  Mitarbeiter_ID      NUMBER,
  Vorname             VARCHAR2(30),
  Nachname            VARCHAR2(35),
...
  CONSTRAINT mitarbeiter_pk
    PRIMARY KEY (Mitarbeiter_ID, Vorname, Nachname)
...
);
        \end{lstlisting}
      \subsection{Das FOREIGN KEY-Constraint}
        \label{refint}
        In einem RDBMS steht \"ublicherweise keine Entit\"at \enquote{einzeln im Raum}. Sie steht immer in Zusammenhang mit anderen Entit\"aten. Diese Zusammenh\"ange werden durch Foreign Key-Constraints dargestellt und \"uberwacht.
        \begin{merke}
          Der Zusammenhang, in dem die Entit\"aten eines RDBMS stehen, wird als \enquote{Referentielle Integrit\"at} bezeichnet.
        \end{merke}
        Ein Beispiel hierf\"ur stellen die beiden Tabellen \identifier{Mitarbeiter} und \identifier{Bankfiliale} bereit. Sie stehen, durch die Spalte \identifier{Bankfiliale\_ID}, die in beiden Relationen vorkommt, in Zusammenhang zu einander. Dieser Zusammenhang besteht darin, dass jeder Mitarbeiter genau einer Bankfiliale zugeordnet ist. Das hei\ss{}t,  in die Spalte \identifier{Bankfiliale\_ID} der Tabelle \identifier{Mitarbeiter} werden die Prim\"arschl\"usselwerte der Tabelle \identifier{Bankfiliale} eingetragen, um so den Zusammenhang herzustellen.
        \beispiel{sql09_15} zeigt, wie ein Fremdschl\"usselconstraint angelegt wird.
\clearpage
        \begin{lstlisting}[language=oracle_sql,caption={Ein Foreign Key-Constraint als Out Of Line Constraint anlegen},label=sql09_15]
ALTER TABLE Mitarbeiter
ADD CONSTRAINT mitarbeiter_filiale_fk
FOREIGN KEY (Bankfiliale_ID)
REFERENCES Bankfiliale(Bankfiliale_ID);
        \end{lstlisting}
        Die Definition eines Fremdschl\"ussels als Out Of Line Constraint hat zwei Teile:
        \begin{itemize}
          \item Die \languageorasql{FOREIGN KEY}-Klausel: Sie legt fest, welche Spalte die referenzierende Spalte ist.
          \item die \languageorasql{REFERENCES}-Klausel: Sie legt fest, welche Spalte referenziert wird. Bei dieser Spalte muss es sich um eine Prim\"arschl\"ussel- oder \UNIQUE-Spalte handeln.
        \end{itemize}
        \begin{merke}
          Wird ein Fremdschl\"ussel als Inline Constraint bei der Erstellung der Tabelle miterstellt, entf\"allt die \languageorasql{FOREIGN KEY}-Klausel.
        \end{merke}
       \begin{merke}
          Es gibt zwei Situationen, die in einer relationalen Datenbank keines Falls auftreten d\"urfen:
          \begin{itemize}
            \item Ein referenzierter Prim\"arschl\"usselwert wird gel\"oscht. Beispiel: Eine Bankfiliale, in der sich noch Mitarbeiter befinden, wird aus der Tabelle \identifier{Bankfiliale} gel\"oscht. Dies w\"urde Datens\"atze in der Tabelle \identifier{Mitarbeiter} zur\"ucklassen, die sich auf eine Filiale beziehen, die gar nicht mehr existiert.
            \item In eine Fremdschl\"usselspalte wird ein Wert eingetragen, der in der referenzierten Prim\"arschl\"usselspalte nicht vorkommt. Beispiel: Ein Mitarbeiter wird in die Bankfiliale mit der ID 300 aufgenommen, welche gar nicht existiert. Auch hier w\"urde sich ein Angestellter auf eine Abteilung beziehen, welche es nicht gibt.
        \end{itemize}
        In beiden F\"allen w\"are die Referentielle Integrit\"at der Datenbank verletzt, was zu Informationsverlust bzw. fehlerhafter Information f\"uhrt. Dies zu vermeiden ist die Aufgabe des Foreign Key-Constraints.
        \end{merke}
        \begin{lstlisting}[language=oracle_sql,caption={Ein Foreign Key-Constraint als Inline Constraint anlegen},label=sql09_16]
CREATE TABLE Mitarbeiter (
...
  Bankfiliale_ID      NUMBER
  CONSTRAINT mitarbeiter_filiale_fk
    REFERENCES Bankfiliale(Bankfiliale_ID)
...
);
        \end{lstlisting}
        Der SQL-Standard kennt zwei Erweiterungen zum \languageorasql{FOREIGN KEY}-Constraint. Dies sind die Klauseln \languageorasql{ON DELETE CASCADE} und \languageorasql{ON DELETE SET NULL}.
        \begin{itemize}
          \item \languageorasql{ON DELETE CASCADE}: Wird ein referenzierter Wert gel\"oscht, werden automatisch alle referenzierenden Werte mitgel\"oscht. Beispiel: Wird eine Filiale aus der Tabelle \identifier{Bankfiliale} gel\"oscht, werden automatisch auch alle Mitarbeiter gel\"oscht, welche sich in dieser Filiale befinden.
          \item \languageorasql{ON DELETE SET NULL}: Wird ein referenzierter Wert gel\"oscht, werden automatisch alle referenzierenden Werte auf NULL gesetzt. Beispiel: Wird eine Filiale aus der Tabelle \identifier{Bankfiliale} gel\"oscht, wird die \identifier{Bankfiliale\_ID} eines jeden Angestellen automatisch auf NULL gesetzt.
        \end{itemize}
        Beide Zus\"atze k\"onnen sehr n\"utzlich sein, bergen jedoch auch gro\ss e Risiken in sich. Wird die \languageorasql{ON DELETE CASCADE}-Klausel zu unvorsichtig angewandt, kann es passieren, das Daten gel\"oscht werden, die gar nicht gel\"oscht werden d\"urfen.

        Die \languageorasql{ON DELETE SET NULL} ist nicht so radikal, wie \languageorasql{ON DELETE CASCADE}, aber auch sie ist nicht ganz ungef\"ahrlich. Wird ein referenzierter Wert gel\"oscht, werden alle referenzierenden Werte kaskadierend auf NULL gesetzt. Das hat zur Folge, das pl\"otzlich Datens\"atze bestehen, die keinen Bezug mehr zu anderen Datens\"atzen haben.
        \begin{merke}
          Sowohl bei der \languageorasql{ON DELETE CASCADE}- als auch bei der \languageorasql{ON DELETE SET NULL}-Klausel muss mit \"au\ss erster Vorsicht gearbeitet werden.
        \end{merke}
        \beispiel{sql09_17} und \beispiel{sql09_18} zeigen, wie diese Klauseln angewandt werden.
        \begin{lstlisting}[language=oracle_sql,caption={Ein Foreign Key-Constraint mit ON DELETE CASCADE-Klausel},label=sql09_17]
ALTER TABLE Mitarbeiter
ADD CONSTRAINT mitarbeiter_filiale_fk FOREIGN KEY (Bankfiliale_ID))
  REFERENCES Bankfiliale(Bankfiliale_ID)
  ON DELETE CASCADE;
        \end{lstlisting}

        \begin{lstlisting}[language=oracle_sql,caption={Ein Foreign Key-Constraint mit ON DELETE SET NULL-Klausel},label=sql09_18]
ALTER TABLE Mitarbeiter
ADD CONSTRAINT mitarbeiter_filiale_fk FOREIGN KEY (Bankfiliale_ID))
  REFERENCES Bankfiliale(Bankfiliale_ID)
  ON DELETE SET NULL;
        \end{lstlisting}
      \subsection{Das SQL Server DEFAULT-Constraint}
        In Microsoft SQL Server werden Standardwerte als Constraints an eine
        Spalte angefügt. Das Anfügen eines Default-Constraints an eine Spalte
        während der Tabellenerstellung funktioniert genauso wie in Oracle.
          \begin{lstlisting}[language=ms_sql,caption={Erstellen
          einer Tabelle mit einem Standardwert},label=sql09_18a]
CREATE TABLE
  Aktie ( Aktie_ID  NUMERIC,
  Name      VARCHAR(25),
  Herkunft  VARCHAR(25) DEFAULT 'USA',
  WKN       NUMERIC,
  ISIN      VARCHAR(12)
);
          \end{lstlisting}
          Der Unterschied zwischen Oracle und MS SQL Server zeigt sich aber,
          wenn ein Default-Constraint nachträglich hinzugefügt werden soll.
          \begin{lstlisting}[language=ms_sql,caption={Tabellenspalte mit
          Standardwert hinzuf\"ugen in SQL Server},label=sql09_18b] 
ALTER TABLE Aktie
ADD CONSTRAINT herkunft_dv
DEFAULT 'USA'
FOR Herkunft;
          \end{lstlisting}
          Anders als in Oracle muss für den SQL Server die \languagemssql{ADD
          CONSTRAINT}-Klausel benutzt werden.
    \section{Constraints umbenennen und l\"oschen}
      \subsection{Constraints umbenennen}
        Sowohl in Oracle als auch in SQL Server ist es m\"oglich ein Constraint umzubenennen.
        \begin{lstlisting}[language=oracle_sql,caption={Ein Constraint umbenennen in Oracle},label=sql09_19]
ALTER TABLE Mitarbeiter
RENAME CONSTRAINT gehalt_ck TO gehalt_provision_ck;
        \end{lstlisting}
        \begin{lstlisting}[language=ms_sql,caption={Ein Constraint umbenennen in SQL Server},label=sql09_20,emphstyle={[9]\color{red}},emph={[9]sp_rename}]
EXEC sp_rename 'gehalt_ck', 'gehalt_provision_ck', 'OBJECT';
        \end{lstlisting}
      \subsection{Constraints l\"oschen}
        Soll ein bestehendes Constraint wieder entfernt werden, muss in Oracle und SQL Server die \languageorasql{DROP CONSTRAINT}-Klausel des \languageorasql{ALTER TABLE}-Kommandos genutzt werden.
        \begin{lstlisting}[language=oracle_sql,caption={Ein Constraint l\"oschen},label=sql09_21]
ALTER TABLE Mitarbeiter
DROP CONSTRAINT mitarbeiter_filiale_fk;
        \end{lstlisting}
        Dies l\"a\ss t sich auf alle f\"unf Constraintarten anwenden.

        Enth\"alt eine zu l\"oschende Tabelle Prim\"ar\-schl\"ussel- oder
        Unique\-Constraints, welche durch Fremd\-schl\"ussel anderer Tabellen
        referenziert werden, muss in Oracle zus\"atzlich die Klausel
        \languageorasql{CASCADE CONSTRAINTS} verwendet werden. Dadurch werden
        die Fremdschl\"ussel der anderern Objekte entfernt. In SQL Server
        m\"ussen zuerst die referenzierenden Foreign Key Constraints gel\"oscht
        werden, ehe die Tabelle gel\"oscht werden kann.
        \begin{lstlisting}[language=oracle_sql, caption={Eine Tabelle mit
        Fremdschl\"usselbeziehungen l\"oschen},label=sql09_22]
DROP TABLE Mitarbeiter CASCADE CONSTRAINTS;
        \end{lstlisting}
      \subsection{Standardwerte in SQL Server l\"oschen}
        \label{sqlserverdefaultconstraint}
        Was Standardwerte sind, ist bereits aus dem vorhergehenden Kapitel bekannt. Wie sie in Oracle und in SQL Server angelegt werden ist ebenfalls bekannt. Was bisher noch nicht gezeigt wurde, ist, wie sie in SQL Server wieder gel\"oscht werden. Da in SQL Server ein Standardwert wie ein Constraint behandelt wird, muss auch die \languagemssql{DROP CONSTRAINT}-Klausel des \languagemssql{ALTER TABLE}-Statements verwendet werden, um einen Standardwert zu l\"oschen.
        \begin{lstlisting}[language=ms_sql,caption={Einen Standardwert in SQL Server l\"oschen},label=sql09_23]
ALTER TABLE Aktie
DROP CONSTRAINT herkunft_dv;
        \end{lstlisting}


	\part{Datenbankadministration}
    \chapter{Einf\"uhrung in die Oracle Datenbankarchitektur}
    \setcounter{page}{1}\kapitelnummer{chapter}
    \minitoc
\newpage
    \section{Oracle und die Client-Server-Architektur}
      Oracle-Datenbanken sind dazu geschaffen, um riesige Datenmengen zu verwalten und diese in einer Multi-User-Umgebung einer gro\ss{}en Nutzeranzahl zur Verf\"ugung zu stellen. Eine solche Umgebung kann auf unterschiedliche Arten realisiert werden. Die einfachste davon ist eine  Client-Server-Architektur, bestehend aus:
      \begin{itemize}
        \item Client
        \item Netzwerk
        \item Datenbankserver
      \end{itemize}

      \bild{Oracle Datenbank\-architektur}{datenbankarchitektur1}{0.8}

      \subsection{Der Client}
        \subsubsection{Nutzerprozess}
          Eine Client-Anwendung, die sich mit einer Oracle-Datenbank verbindet, wird als \enquote{Nutzerprozess} bezeichnet. Beispiele f\"ur solche Nutzerprozesse sind:
          \begin{itemize}
            \item SQL*Plus
            \item SQL*Developer
            \item J*Developer
            \item Enterprise Manager Console
          \end{itemize}
        \subsubsection{Connection}
          Eine Connection ist eine physikalische Verbindung zwischen einem Client und dem Datenbankserver. Sind Client und Datenbankserver eins, wird die Connection durch einen Interprozesskommunikationsmechanismus (IPC) erzeugt. Bei zwei unterschiedlichen Rechnern, wird die Connection \"uber ein Netzwerkprotokoll, wie beispielsweise TCP/IP realisiert.
          \begin{merke}
            Eine Connection stellt die Grundlage f\"ur eine Verbindung mit einer Oracle-Datenbank dar.
          \end{merke}
        \subsubsection{Session}
          W\"ahrend die Connection eine physikalische Verbindung zwischen dem Client und dem Datenbankserver darstellt, ist eine Session eine Kommunikationsverbindung zwischen dem Nutzerprozess und der Datenbank. Der Aufbau einer Session erfolgt, sobald sich ein Benutzer bei der Datenbank authentifiziert hat.

          \begin{merke}
            Eine Session stellt einen Zugang zur Datenbank dar.
          \end{merke}

          Oracle ist in der Lage, mehrere Sessions \"uber ein und die selbe physikalische Connection zu \"offnen. Zum Beispiel kann sich ein Nutzer mit Name/Passwort \enquote{SCOTT/TIGER} beliebig oft von einem Rechner aus, an der Datenbank anmelden.

          \bild{Oracle Datenbank\-architektur}{datenbankarchitektur2}{0.8}

      \subsection{Der Datenbankserver}
        \subsubsection{Instanz und Datenbank}
          Ein Oracle-Datenbankserver besteht aus zwei gro\ss{}en Teilen:
          \begin{itemize}
            \item \textbf{Instanz}: Die Instanz ist das Herzst\"uck einer Oracle-Datenbank. Es handelt sich dabei um eine Menge von Arbeitsspeicherstrukturen und Prozessen, die ein schnelles und effizientes Arbeiten mit den Daten erm\"oglichen. Sie gliedern sich grob in drei Teile: System Global Area, Program Global Area und Hintergrundprozesse.
% \clearpage
            \item \textbf{Datenbank}: Dies ist die Menge aller Dateien, aus denen der Oracleserver besteht. Es sind sowohl die Installationsdateien der Oracle-Software, als auch die Datendateien gemeint, welche die Nutzdaten enthalten, sowie alle weiteren Dateien.
          \end{itemize}

          \begin{merke}
            Die Begriffe Instanz und Datenbank, werden oft als Synonyme verwendet. Tats\"achlich handelt es sich aber um zwei sehr unterschiedliche Dinge.
          \end{merke}
          W\"ahrend eine Instanz immer nur auf genau eine Datenbank zugreifen kann, kann eine Datenbank gleichzeitig von mehreren Instanzen gemountet\footnote{engl. to mount = anschlie\ss{}en} werden.

          \bild{Oracle Datenbank\-architektur}{datenbankarchitektur3}{0.8}

        \subsubsection{Serverprozess}
          Ein Serverprozess stellt das serverseitige Pendant zu einem Nutzerprozess dar. Er ist die eigentliche \enquote{Arbeitsmaschine}, nimmt die Anforderungen eines Nutzerprozesses entgegen und verarbeitet diese innerhalb der Instanz.

          \begin{merke}
            Ein Nutzerprozess ben\"otigt immer einen Serverprozess um arbeitsf\"ahig zu sein. Dadurch wird gew\"ahrleistet, dass keine Clientanwendung direkt in der Datenbank arbeitet und das somit auch bestimmte Regeln gewahrt bleiben.
          \end{merke}

          Um besser erkl\"aren zu k\"onnen, was ein Serverprozess ist, soll als bildliches Beispiel ein Ober in einem Restaurant dienen. Kein Gast geht in einem Lokal in die K\"uche und bereitet sich sein Essen selbst zu. Statt dessen wird er bei einem Ober (dem Serverprozess) eine Bestellung aufgeben, welcher diese dann an den Koch weiterreicht.

          Bei ihm  angekommen, wird die Bestellung zubereitet bzw. im Sinne einer Datenbank werden die geforderten Datens\"atze zusammengestellt. Anschlie\ss{}end hat der Ober noch die Aufgabe, das Essen / das Ergebnis dem Gast zu servieren.

          Der einzige Haken an diesem Beispiel ist, dass der Serverprozess Ober und K\"uchenpersonal in Personalunion ist. Er nimmt Anforderungen entgegen, verarbeitet diese innerhalb der Instanz und reicht das Ergebnis an den Nutzerprozess zur\"uck.

          \bild{Oracle Datenbank\-architektur}{datenbankarchitektur4}{1}

     \section{Die System Global Area (SGA)}
        Die System Global Area (kurz SGA) ist eine Speicherstruktur mit variabler Gr\"o\ss{}e innerhalb einer Instanz. Alle von den Serverprozessen ausgef\"uhrten Arbeitsschritte ben\"otigen in irgendeiner Form die SGA. Sie enth\"alt Nutz- und Metadaten, sowie andere Kontrollinformationen zu einer bestimmten Datenbank. Die Erstellung der SGA erfolgt automatisch beim Instanzstart. Durch das Herunterfahren einer Instanz wird die SGA zerst\"ort.

        \bild{Oracle Datenbank\-architektur}{datenbankarchitektur5}{1}

        Die SGA teilt sich in verschiedene Regionen, mit unterschiedlichem Inhalt.

      \subsection{Die Fixed SGA}
        Ein sehr kleiner Teil der SGA wird als \enquote{Fixed SGA} bezeichnet. Hierbei handelt es sich um einen betriebssystemabh\"angigen, unver\"anderlichen Teil der SGA, der als eine Art \enquote{Einstiegspunkt} von allen Serverprozessen genutzt wird. Er enth\"alt die Speicheradressen anderer Strukturen innerhalb der SGA. Der Datenbankadministrator hat keinerlei Kontrolle \"uber diesen Teil der SGA.

      \subsection{Der Database Buffer Cache}
        Der Database Buffer Cache ist f\"ur die Zwischenspeicherung der zuletzt benutzten Oraclebl\"ocke zust\"andig. Da eine theoretische Chance besteht, dass ein Block der bereits einmal ben\"otigt wurde, auch noch weitere Male ben\"otigt wird, kann so die Anzahl der Zugriffe auf den Datentr\"ager reduziert und die Arbeitsgeschwindigkeit erh\"oht werden.

        \bild{Oracle Datenbank\-architektur}{datenbankarchitektur6}{0.8}

        Neue oder ge\"anderte Daten werden nicht sofort auf den Datentr\"ager geschrieben. Um die Anzahl der Festplattenzugriffe zu reduzieren und somit die Performance der Datenbank zu steigern, werden die Daten im Database Buffer Cache gesammelt. Abh\"angig von bestimmten Kriterien, wird der Inhalt des Caches dann auf den Datentr\"ager geschrieben.
      \subsection{Der Shared Pool}
        Der Teil der SGA, der \enquote{Shared Pool} genannt wird, ist f\"ur die
        Speicherung von Informationen bez\"uglich ausgef\"uhrter SQL-Statements
        zust\"andig. Er gliedert sich seit Oracle 11g in drei Bereiche:
        \begin{itemize}
          \item Library Cache
          \item Data Dictionary Cache
          \item Result Cache
        \end{itemize}
        \bild{Oracle Datenbank\-architektur}{datenbankarchitektur7}{0.8}
        \subsubsection{Der Library Cache}
          Der Library Cache enth\"alt Informationen \"uber alle ausgef\"uhrten
          SQL- und PL/SQL-Statements. Setzt ein Nutzer ein SQL-Statement ab,
          werden Informationen dar\"uber in Form eines Ausf\"uhrungsplanes im
          Library Cache abgelegt. Wird exakt das gleiche Statement von einem
          anderen Nutzer ausgef\"uhrt, k\"onnen die vorhanden Informationen im
          Library Cache wiederverwendet werden, was die
          Ausf\"uhrungsgeschwindigkeit wesentlich erh\"oht.

          Oracle nutzt f\"ur jedes SQL-Statement soviel Speicherplatz im Shared
          Pool wie notwendig. Ist kein Platz mehr f\"ur weitere Informationen
          vorhanden, wird nach einem modifizierten LRU-Algorithmus wieder
          Speicher freigegeben.

          \begin{merke}
            Die Abk\"urzung LRU steht f\"ur \enquote{Least Recently Used}, was
            soviel bedeutet wie: \enquote{Am wenigsten ben\"otigt}. Ein
            LRU-Algorithmus hat die Aufgabe die am wenigsten ben\"otigten
            Informationen aus einer Speicherstruktur zu entfernen, um so
            Speicherplatz freizugeben.
          \end{merke}
        \subsubsection{Der Dictionary Cache}
          Das Data Dictionary ist eine Sammlung von Datenbanktabellen und Views, die Metadaten \"uber die gesamte Datenbank enthalten. Oracle muss st\"andig im laufenden Betrieb auf dieses Data Dictionary zugreifen. Um h\"aufige Datentr\"agerzugriffe zu vermeiden, werden ben\"otigte Informationen aus dem Data Dictionary im Data Dictionary Cache zwischengespeichert.
        \subsubsection{Der Result Cache}
          Beim Result Cache handelt es sich um ein neues Feature der Oracle Version 11g. Dieser Cache speichert keine Datenbl\"ocke, sondern Ergebniszeilen. Der Vorteil dieser Vorgehensweise liegt auf der Hand.

          Wird ein SQL-Statement h\"aufig ausgef\"uhrt, ohne das sich die Datenbasis \"andert, kann das Ergebnis direkt aus dem Result Cache \"ubermittelt werden. Sinnvoll genutzt werden kann der Result Cache immer dann, wenn:
          \begin{itemize}
            \item SQL-Abfragen sehr rechenintensiv sind,
            \item das Ergebnis einer SQL-Abfrage nahezu unver\"anderlich ist,
            \item sehr viel Arbeitsspeicher vorhanden ist.
          \end{itemize}
      \subsection{Der Redo Log Buffer}
        F\"ur jede \"Anderung, die an einer Oracle-Datenbank durchgef\"uhrt wird, wird ein sogenannter Redo Record erzeugt. Ein Redo Record protokolliert jeweils eine \"Anderung und macht diese somit nachvollziehbar. Durch Redo Records wird Datensicherheit dahingehend gew\"ahrleistet, dass im Falle eines Crashes alle protokollierten \"Anderungen wieder in die Datenbank eingepflegt werden k\"onnen.

        \bild{Oracle Datenbank\-architektur}{datenbankarchitektur8}{1}

        Im Redo Log Buffer werden die zuletzt erzeugten Redo Records zwischengespeichert. Er ist als Ringpuffer (beim Erreichen des letzten Eintrags wird der erste Eintrag wieder \"uber\-schrie\-ben) organisiert. In bestimmten Zeitabst\"anden oder wenn andere Bedinungen erf\"ullt sind, wird der Inhalt des Redo Log Buffers in die Redo Log Dateien geschrieben.

    \section{Die Program Global Area (PGA)}
      W\"ahrend die SGA f\"ur alle Nutzer relevante Informationen beinhaltet, ist die PGA ein Speicherbereich, der sessionabh\"angige Informationen enth\"alt. Das hei\ss t jeder Nutzer hat seine eigene PGA. Eine PGA wird beim Starten eines Serverprozesses angelegt und geh\"ort zu genau einem bestimmten Serverprozess.

      \bild{Die Program Global Area}{pga}{0.8}

      \abbildung{pga} zeigt den Aufbau einer PGA. Sie besteht im Wesentlichen aus einem Bereich namens \enquote{Stackspace}. Dieser beinhaltet Informationen \"uber die Nutzersession, verschiedene Variablen der Nutzersession, private Informationen zur Abarbeitung von SQL-Statements und eine Workarea. Die Workarea ist ein Bereich, in dem Hash- und Sortieroperationen durchgef\"uhrt werden.

      \begin{literaturinternet}
        \item \cite{CNCPT1237}
      \end{literaturinternet}
      \section{Memory Management}
        \label{memorymanagement}
        Unter dem Begriff Memory Management werden alle Aufgaben und Einstellungen zusammengefasst, die sich um die Dimensionierung der Speicherkomponenten von SGA und PGA drehen. Bis zur Version Oracle 9i konnte das Memory Management nur manuell durch den Administrator durchgef\"uhrt werden. Diese Technik wurde als \enquote{Manual Shared Memory Management} bezeichnet. Der Administrator musste die Gr\"o\ss{}en f\"ur alle SGA-Komponenten manuell eingeben und, falls notwendig, sie den aktuellen Gegebenheiten anpassen. Dies erm\"oglichte dem Admin zwar eine sehr genaue Kontrolle \"uber die SGA und alle PGAs, jedoch konnten durch schlechte Einstellungen auch viele Fehler gemacht und die Performance der Instanz stark gesenkt werden.

        Mit Oracle 10g wurde dann das \enquote{Automatic Shared Memory Management} eingef\"uhrt, wodurch ein automatisches Tuning der SGA-Komponenten und der PGAs erfolgte. Bei diesem neuen Verfahren musste der DBA lediglich noch zwei Speichergr\"o\ss{}en angeben: Eine Speichermenge f\"ur alle SGA-Komponenten und einen Speicherpool f\"ur alle PGAs. Oracle 10g vergr\"o\ss{}erte/verkleinerte dann die Speicherbereiche der SGA automatisch, so dass diese immer den aktuellen Erfordernissen entsprachen und die Instanz eine optimale Performance erreichen konnte.

        Aus dem Speicherpool der PGAs wurden dann die einzelnen PGAs der Serverprozesse erstellt. Dadurch das auch hier nur noch ein einzelner Wert angegeben werden musste, konnte Oracle auch das automatische Anpassen der PGA-Gr\"o\ss{}en \"ubernehmen, so dass jeder Serverprozess die f\"ur ihn optimale Speichermenge zur Verf\"ugung hatte.

        Mit Oracle 11g kam nun das \enquote{Automatic Memory Management}, was die Verwaltung von SGA- und PGA-Speicher noch weiter vereinfacht und ein besseres Tuning der Speicherstrukturen erlaubt. Der DBA muss nur einen einzigen Wert angeben, n\"amlich die Gesamtmenge an Arbeitspeicher, die f\"ur SGA und PGAs verf\"ugbar sein soll. Oracle 11g \"ubernimmt die gesamte Verwaltung aller Speicherkomponenten der SGA und PGAs. Dadurch das nun nur noch ein Limit f\"ur beide existiert, teilen sich SGA und PGA den gleichen Speicher, was eine h\"ohere Flexibilit\"at bei der Gr\"o\ss{}enanpassung von SGA und PGA bedeutet.

        \tabelle{tabmemorymanagement} fasst noch einmal alle Arten des Memory Management und deren Features zusammen.

        \begin{center}
          \tablecaption{Oracle Shared Memory Management}
          \label{tabmemorymanagement}
          \begin{small}
            \tablefirsthead{
              \multicolumn{1}{c}{\textbf{Oracle}} &
              \multicolumn{1}{c}{\textbf{Memory Management}} &
              \multicolumn{1}{c}{\textbf{Auto. Tuning}} &
              \multicolumn{1}{c}{\textbf{Einstellungen}}\\
              \hline
            }
            \tabletail{
              \hline
            }
            \tablelasttail{
              \hline
            }
            \begin{supertabular}{|p{2.1cm}|p{4.15cm}|p{4.5cm}|p{3.6cm}|}
              bis Oracle 9i & \raggedright Manual Shared Memory Management & PGA-Gr\"o\ss{}e & Alle Komponenten der SGA einzeln \\
              \hline
              ab Oracle 10g & \raggedright Automatic Shared Memory Management & Gr\"o\ss{}e aller Komponenten der SGA und die Speichergr\"o\ss{}en der PGAs & SGA-Zielgr\"o\ss{}e und ag\-gre\-gier\-ter Speicher aller PGAs\\
              \hline
              ab Oracle 11g & \raggedright Automatic Memory Management & Gr\"o\ss{}e aller Komponenten der SGA und die Speichergr\"o\ss{}en der PGAs & Speichermenge der gesamten Instanz (SGA + PGA)\\
            \end{supertabular}
          \end{small}
        \end{center}

      \section{\"Uberblick \"uber die Struktur einer Oracle Datenbank}
        \subsection{Datendateien}
          Jede Oracle Datenbank besteht aus einer oder mehreren Datendateien. Sie enthalten alle Nutz- und Metadaten, die dort in Form von logischen Datenstrukturen, wie z. B. Tabellen und Indizes gespeichert werden.
          \begin{merke}
            \begin{itemize}
              \item Eine Datendatei kann nur zu einer Datenbank geh\"oren.
              \item Das verwendete Dateisystem begrenzt die Anzahl der Datendateien, die angelegt werden k\"onnen.
            \end{itemize}
          \end{merke}
        \subsection{Die Parameterdatei}
          Beim Start einer Oracle Datenbank muss eine Vielzahl von Parametern gesetzt werden. Diese Parameter werden in so genannten Parameterdateien (PFile oder SPFile) zusammengefasst.

          Es gibt zwei unterschiedliche Arten von Parameterdateien:

          \begin{itemize}
            \item Die Parameterdatei (PFile): Sie ist eine statische Sammlung von Parametern in einer Textdatei. Es k\"onnen keine dynamischen \"Anderungen vorgenommen werden. Jede \"Anderung an den Initialisierungsparametern der Datenbank muss auch an der Parameterdatei vorgenommen werden.
            \item Die Server-Parameterdatei (SPFile): Hierbei handelt es sich um eine dynamische Parameterdatei, in Form einer Bin\"ardatei. Mit Hilfe einer Server-Parameterdatei ist es m\"oglich, \"Anderungen an einem Initialisierungsparameter in einem einzigen Schritt vorzunehmen. Diese \"Anderungen k\"onnen dann wahlweise nur an der Instanz, nur in der Server-Parameterdatei oder aber an Instanz und Server-Parameterdatei gleichzeitigt get\"atigt werden.
          \end{itemize}
          \bild{Oracle Datenbank\-architektur}{datenbankarchitektur9}{0.8}

          \begin{merke}
            Eine Server-Parameterdatei darf niemals mit einem Texteditor ver\"andert werden, da Oracle eine solche Datei sofort als korrupt erkennt und nicht mehr benutzen kann.
          \end{merke}

          Oracle empfiehlt die Nutzung von Server-Parameterdateien. Es gibt sie seit Oracle 9i. W\"ahrend in Oracle 9i standardm\"assig noch Parameterdateien genutzt wurden, wird seit Oracle 10g automatisch jede Datenbank, die mit dem Database Configuration Assistant angelegt wird, mit einer Server-Parameterdatei erstellt.
       \subsection{Kontrolldateien}
        Die Kontrolldatei ist ein wesentlicher Bestandteil einer jeden Oracle-Datenbank. Sie enth\"alt alle  Informationen \"uber den Aufbau der jeweiligen Datenbank. Dies sind beispielsweise:
          \begin{itemize}
            \item Datenbankname
            \item Namen und Speicherorte aller Datendateien
            \item Zeitstempel der Datenbankerstellung
            \item Informationen \"uber Backups
          \end{itemize}

          \bild{Oracle Datenbank\-architektur}{datenbankarchitektur10}{0.8}

          Aus Gr\"unden der Ausfallsicherheit wird die Kontrolldatei gespiegelt, d. h. sie wird mehrfach mit gleichem Inhalt gespeichert. Der Verlust aller Kontrolldateien ist ein besonders kritischer Fall, da die Datenbank nur mit gro\ss{}em Aufwand wieder ge\"offnet werden kann.

          Gebraucht wird die Kontrolldatei an verschiedenen Stellen w\"ahrend des Datenbankbetriebs. Beispielsweise beim Start einer Oracleinstanz. Die Kontrolldatei stellt dann die Informationen \"uber Namen und Speicherorte der Datendateien und anderer wichtiger Dateien bereit, die f\"ur den Startvorgang ge\"offnet werden m\"ussen.

          Wird das physische Layout der Datenbank ge\"andert, z. B. dadurch, dass eine neue Datendatei hinzugef\"ugt wird, wird diese \"Anderung sofort in der Kontrolldatei vermerkt, damit diese immer den aktuellen Stand der Datenbank wiedergibt.
        \subsection{Redo Log Dateien}
          \subsubsection{Aufbau und Funktion}
            F\"ur den Betrieb einer Oracle Datenbank wird ein Set aus \enquote{Redo Log Gruppen}, umgangssprachlich als \enquote{Redo Logs} bezeichnet, ben\"otigt. Eine Redo Log Gruppe besteht aus einer oder mehreren Redo Log Dateien, die auch als Redo Log Member bezeichnet werden.

            Die prim\"are Funktion der Redo Logs ist das Aufzeichnen aller \"Anderungen, die an den Daten vorgenommen wurden (Nutz- und Metadaten).

            \bild{Redo Log Gruppen und Member}{redo_log_gruppen}{0.8}

            Die Redo Log Member nehmen den Inhalt des Redo Log Buffers auf. Im Falle eines Recovery benutzt die Datenbank die Redo Logs, um alle \"Anderungen der Nutzer wieder in die Datenbank einzuarbeiten und sie so auf den letzten Stand vor dem Crash zu bringen.


          \subsubsection{Spiegelung der Redo Logs}
            Um die Redo Logs vor Ausf\"allen zu sch\"utzen, hat eine Oracle Datenbank die M\"og\-lich\-keit sie zu spiegeln und die einzelnen Kopien auf mehrere verschiedene Datentr\"ager zu verteilen.

            In \abbildung{redo_log_verteilung} wird angenommen, dass es zwei Redo Log Gruppen (Gruppe 1 und Gruppe 2) mit je drei Membern (Member a, b und c) gibt. Die Member sind auf zwei Datentr\"agern (/u02 und /u03) verteilt gespeichert.

            \bild{Verteilung und Spiegelung von Redo Log
            Membern}{redo_log_verteilung}{0.8}

          \subsubsection{Empfehlungen}
            Grunds\"atzlich gelten die folgenden Empfehlungen f\"ur Redo Log Dateien/-Gruppen:
            \begin{itemize}
              \item Jede Redo Log Gruppe sollte mindestens zwei Member haben (besser drei).
              \item Es sollten niemals alle Member einer Redo Log Gruppe auf dem gleichen Datentr\"ager liegen.
              \item Alle Redo Log Gruppen sollten immer die gleich Anzahl Member haben und damit symmetrisch sein.
              \item Die Gr\"o\ss{}e der Redo Log Member sollte so angepasst werden, dass kein Platz auf den Sicherungsmedien verschwendet wird.
            \end{itemize}
\clearpage
            \begin{merke}
              Zu beachten ist, dass die Redo Logs nur gegen System- oder Medienfehler sch\"utzen k\"onnen (z.~B. im Falle eines Stromausfalls), nicht aber gegen Fehleingaben eines Nutzers.
            \end{merke}
        \subsection{Archivierte Log Dateien}
          Um eine Oracle Datenbank, im Falle eines Fehlers, wiederherstellen zu k\"onnen, werden die in den Redo Log Dateien gespeicherten Informationen ben\"otigt. Da die Redo Logs jedoch zyklisch \"uberschrieben werden, stehen diese Informationen nur f\"ur eine begrenzte Zeitspanne zur Verf\"ugung. Um diese wichtigen Informationen f\"ur einen l\"angeren (theoretisch unbegrenzten) Zeitraum verf\"ugbar machen zu k\"onnen, bietet Oracle die M\"oglichkeit, Kopien der Redo Logs anzulegen. Diese Kopien werden dann als \enquote{Archive Logs} oder \enquote{Archivierte Log Dateien} bezeichnet.
        \subsection{Die Alert Log Datei und Trace-Dateien}
          \bild{Oracle Datenbank\-architektur}{datenbankarchitektur11}{1}
          \subsubsection{Trace-Dateien}
            Eine Oracle-Datenbank kennt zwei unterschiedliche Arten von Trace-Dateien. Einerseits sind dies Trace-Dateien, die auf Anforderung des Administrators erzeugt werden. Diese enthalten meist Performance- oder Diagnose-Informationen die zur Fehlerbehebung bzw. zum Tuning der Datenbank eingesetzt werden k\"onnen.

            Andererseits entstehen die meisten Trace-Dateien aufgrund von Fehlern bei den Hintergrundprozessen. Jeder Hintergrundprozess hat seine eigene Trace-Datei in die er Informationen schreibt.

						Die Inhalte einer Trace-Datei sind teils f\"ur den Datenbankadministrator und teils f\"ur den Oracle-Support bestimmt.
\clearpage
            \begin{lstlisting}[caption={Eine Tracedatei des Log Writers},
            label=orcl_lgwr_12331,emph={[9]ORACLE_HOME},emphstyle={[9]\color{black}},language=terminal]
/u01/app/oracle/diag/rdbms/orcl/orcl/trace/orcl_lgwr_12331.trc 
Oracle Database 11g Enterprise Edition Release 11.2.0.1.0 - Production
With the Partitioning, OLAP and Data Mining options
ORACLE_HOME = /u01/app/oracle/product/11.2.0
System name:    Linux
Node name:      FEA11-119WD01
Release:        3.10.5.1-100.fc19
Version:        #1 SMP Mon Jul 21 02:06:29 EDT 2011
Machine:        i686
Instance name: ORCL
Redo thread mounted by this instance: 1
Oracle process number: 6
Unix process pid: 12331, image: oracle@FEA11-119WD01 (LGWR)
            \end{lstlisting}
					\subsubsection{Die Alert Log Datei}
						Die \enquote{Alert Log Datei} nimmt eine Sonderstellung unter den Trace-Dateien ein. Sie ist ein chronologisch sortiertes Protokoll, in dem alle in der Datenbank auftretenden Ereignisse und Fehler protokolliert werden. Man k\"onnte sie auch als das \enquote{Tagebuch} der Datenbank bezeichnen.

						Seit Oracle 11g R1 existiert die Alert Log Datei in zwei Versionen, als Text-Datei und als XML-Datei. Beide geh\"oren zum ebenfalls neu erschienenen \enquote{Automatic Diagnostic Repository}, kurz ADR. Das ADR ist im Wesentlichen ein Verzeichnis, dass von Oracle genutzt wird, um die verschiedensten Diagnose-Informationen geordnet abzulegen. Mit Hilfe dieser Dateien und des ADRCI (Automatic Diagnostic Repository Commandline Interface) k\"onnen sogenannte Incident Packages geschn\"urt und an den Oracle Support verschickt werden.

						Die beiden Alert Log Dateien befinden sich in Unterverzeichnissen des ADR:
						\begin{itemize}
							\item \textbf{Textdatei}: Sie hei\ss{}t \textbf{alert\_\textless{}SID\textgreater{}.ora}. Die Abk\"urzung \enquote{SID} steht f\"ur \enquote{System Identifier}, den Namen der Datenbankinstanz. Bei einer Instanz mit Namen \enquote{ORCL} hie\ss{}e die Alert Log Datei dann: \textbf{alert\_ORCL.ora}. Sie wird im Verzeichnis \texttt{\$ADR\_BASE/diag/product\_type/product\_id/instance\_id/trace} erzeugt.
							\item \textbf{XML-Datei}: Der Name der XML-Datei lautet \textbf{log.xml}. Diese wird im Verzeichnis \texttt{\$ADR\_BASE/diag/product\_type/product\_id/instance\_id/alert} angelegt.
						\end{itemize}

						\begin{literaturinternet}
						  \item \cite{CNCPT005}
						\end{literaturinternet}
\clearpage
          \begin{lstlisting}[caption={Die Alert Log
          Textdatei},label=alert_orcl.log,language=terminal]
Fri Aug 14 13:33:43 2011 Process J000 died, see its trace file
Fri Aug 14 13:33:43 2011
kkjcre1p: unable to spawn jobq slave process
Fri Aug 14 13:33:43 2011
Errors in file /u01/app/oracle/admin/ORCL/bdump/orcl_cjq0_23189.trc:
Fri Aug 14 13:35:33 2011
Errors in file /u01/app/oracle/admin/ORCL/bdump/orcl_j000_18573.trc:
ORA-00600: Interner Fehlercode, Argumente: [keltnfy-ldmInit], [46]
Fri Aug 14 13:35:34 2011
Errors in file /u01/app/oracle/admin/ORCL/bdump/orcl_j000_18573.trc:
ORA-00600: Interner Fehlercode, Argumente: [keltnfy-ldmInit], [46]
Fri Aug 14 13:35:35 2011
Process J000 died, see its trace file
Fri Aug 14 13:35:35 2011
kkjcre1p: unable to spawn jobq slave process
Fri Aug 14 13:35:35 2011
Errors in file /u01/app/oracle/admin/ORCL/bdump/orcl_cjq0_23189.trc:
          \end{lstlisting}
      \section{Die Oracle Hintergrundprozesse}
        \begin{merke}
          Ein Prozess ist ein Thread oder ein \"ahnlicher Mechanismus eines Betriebssystems, der einen oder mehrere Arbeitsschritte durchf\"uhren kann (einige Betriebssysteme verwenden daf\"ur die Bezeichnungen \enquote{Job} oder \enquote{Task}). Jeder Prozess hat seinen eigenen Speicherbereich im Arbeitsspeicher.
        \end{merke}

        Die Aufteilung der Arbeit in einzelne Prozesse geschieht, um die Datenbank Multi-user-f\"ahig zu machen. Durch diese Aufteilung k\"onnen sich (theoretisch) beliebig viele Nutzer mit einer Oracle-Datenbank verbinden, da jeder Nutzer seinen eigenen Serverprozess bekommt.

        Eine Oracleinstanz kann viele verschiedene Hintergrundprozesse haben. Es sind jedoch nicht immer alle Hintergrundprozesse aktiv. Im Folgenden werden die wichtigsten Hintergrundprozesse beschrieben.
        \subsection{Der Database Writer (DBWn)}

          Der Database Writer ist f\"ur das R\"uck\"ubertragen aller
          modifizierten Bl\"ocke aus dem Database Buffer Cache in die
          Datendateien verantwortlich. Auch wenn meist ein einziger DBWn-Prozess
          f\"ur eine Datenbank ausreichend ist, k\"onnen mehrere gestartet
          werden, um die Performance des Systems zu erh\"ohen.

          \bild{Oracle Datenbank\-architektur}{datenbankarchitektur12}{1}

          Da durch Schreibzugriffe die Anzahl der freien Bl\"ocke im Database Buffer Cache abnimmt, ist es die Aufgabe des Database Writers, diese Anzahl nie unter einen bestimmten Schwellenwert fallen zu lassen. Wird der Schwellenwert dennoch unterschritten, m\"ussen die Nutzerprozesse auf den Database Writer warten, bis der Schwellenwert wieder \"uberstiegen wird.
          \begin{merke}
            Der Database Writer schreibt, wenn eine der folgenden Bedingungen erf\"ullt ist:
            \begin{itemize}
              \item Wenn sich zu viele modifzierte Bl\"ocke im Database Buffer Cache befinden (Schwellenwert: weniger als 3 \% freie Bl\"ocke).
              \item Wenn ein Serverprozess zu lange nach freien Bl\"ocken suchen muss.
              \item Beim Ausl\"osen eines Checkpoints.
              \item Wenn die Instanz heruntergefahren wird (au\ss er bei einem shutdown abort).
              \item Wenn ein Tablespace Offline oder Read Only gesetzt wird.
              \item Wenn eine Speicherstruktur, wie z. B. eine Tabelle gel\"oscht wird.
            \end{itemize}
          \end{merke}
          \subsubsection{Aufbau und Organisation des Database Buffer Caches}
            Der Database Buffer Cache ist in zwei Listen aufgeteilt:
            \begin{itemize}
              \item \textbf{Least recently used list (LRU)}
              \item \textbf{Buffer Checkpoint Queue}
            \end{itemize}
						Die Buffer Checkpoint Queue ist eine Liste der ge\"anderten Bl\"ocke (dirty Blocks) des Buffer Caches, die noch nicht auf den Datentr\"ager zur\"uckgesichert wurden. Diese werden in aufsteigender Reihenfolge, nach ihrer Low Redo Byte Adress (low RBA) gespeichert.
\clearpage
            Die LRU List enth\"alt drei Blockarten:
            \begin{itemize}
              \item freie Bl\"ocke (\textbf{clean blocks})

              Ein Block gilt als clean, wenn er bisher noch nicht verwendet wurde oder aber, wenn sein Inhalt nach einer Modifikation auf die Festplatte zur\"uck\"ubertragen wurde (d. h. das Blockabbild im Buffer Cache und der Originalblock auf der Festplatte haben den gleichen Inhalt).
              \item Bl\"ocke die aktuell in Verwendung sind (\textbf{pinned blocks})
              \item modifizierte Bl\"ocke (\textbf{dirty blocks})

              Wird ein Block im Buffer Cache ver\"andert, dann \"andert sich sein Status auf dirty (d. h. Das Blockabbild im Buffer Cache hat einen anderen Inhalt als der Originalblock auf der Festplatte). Bl\"ocke mit diesem Status m\"ussen erst noch auf die Festplatte \"ubertragen werden.
            \end{itemize}

            \bild{Aufbau des Database Buffer Caches}{buffer_cache1}{1.24}

            Jeder Block in der LRU hat einen \enquote{Touchcounter}, einen Z\"ahler, der die Anzahl der Zugriffe auf den Datenblock z\"ahlt. Er wird f\"ur die Verwaltung der Bl\"ocke in der LRU ben\"otigt.
          \subsubsection{Verwaltung des Database Buffer Caches}
            Sucht ein Serverprozess einen bestimmten Block im Database Buffer Cache, beginnt er mit seiner Suche am MRU\footnote{MRU = Most Recently Used}-Ende der LRU\footnote{LRU = Least Recently Used}-Liste, da die Wahrscheinlichkeit, dass sich der gesuchte Block dort befindet h\"oher ist, als die, dass er sich am LRU-Ende befindet. Der Serverprozess durchsucht die LRU-Liste sequenziell, Block f\"ur Block.

            \bild{Suche nach freien Bl\"ocken im Buffer Cache}{buffer_cache2}{1.4}

            Wird der gesuchte Block gefunden (cache hit), kann er verarbeitet werden. Wird er nicht gefunden (cache miss), muss der Block in den Database Buffer Cache geladen werden.

            Beim Laden eines Blockes in den Database Buffer Cache, wird er auf halber H\"ohe des Coldspot eingeschrieben. Um einen neuen Block in den Buffer Cache \"ubertragen zu k\"onnen, muss der Serverprozess vorher nach freien Bl\"ocken suchen. Dabei beginnt er seine Suche am LRU-Ende der LRU-Liste. Trifft er dabei auf dirty blocks, notiert er die Blockadressen dieser Bl\"ocke in der Checkpoint Queue, so dass der Database Writer diese in die Datendateien \"ubertragen kann.

            Wurden gen\"ugend freie Bl\"ocke gefunden, k\"onnen die neuen Bl\"ocke in den Buffer Cache \"uber\-tra\-gen werden. Werden keine freien B\"ocke gefunden, muss der Serverprozess den Database Writer damit beauftragen, die Checkpoint Queue abzuarbeiten, um Platz im Database Buffer Cache zu schaffen.

            \bild{Einf\"ugen neuer Bl\"ocke im Buffer Cache}{buffer_cache3}{1.2}

            Der Touchcounter z\"ahlt die Zugriffe auf einen Datenblock. Wird eine bestimmte Anzahl Zugriffe erreicht, wird der Block relativ von seiner aktuellen Position aus, um 50 \% nach oben verschoben. So kann der Block aus dem Coldspot an das obere Ende der LRU-Liste,   den Hotspot, wandern. Dabei verdr\"angt er andere Bl\"ocke nach unten. Wird auf einen Block nur selten zugegriffen, erreicht er irgendwann den Coldspot und bekommt fr\"uher oder sp\"ater den Status clean.
            \begin{merke}
              Die Serverprozesse sind daf\"ur zust\"andig, angeforderte Bl\"ocke in den Database Buffer Cache zu \"ubertragen und diese dort zu lesen bzw. nach den Anforderungen der Nutzer entsprechend zu modifizieren. Der Database Writer ist als einziger daf\"ur zust\"andig modifizierte Bl\"ocke auf die Festplatte zur\"uckzu\"ubertragen.
            \end{merke}
        \subsection{Der Log Writer (LGWr)}
          Der Log Writer Prozess, der oft auch als Redo Thread bezeichnet wird, ist zust\"andig f\"ur das Management der Redo Logs. Er schreibt die Redo Log Eintr\"age (Redo Records) im Redo Log Buffer in die Redo Log Dateien. Der Redo Log Buffer ist zyklisch aufgebaut. Wenn der Log Writer alle Eintr\"age aus dem Buffer in die Dateien geschrieben hat, k\"onnen die alten Werte im Redo Log Buffer \"uberschrieben werden.
          \bild{Oracle Datenbank\-architektur}{datenbankarchitektur13}{1}

          \begin{merke}
            Der Log Writer tritt in Aktion, wenn die folgenden Bedingungen erf\"ullt sind:
            \begin{itemize}
              \item Periodisch alle 3 Sekunden (Timeout)
              \item Wenn sich der Redo Log Buffer bis zu einem Drittel gef\"ullt hat (Standardgr\"o\ss{}e 512 KB)
              \item Wenn im Redo Log Buffer mehr als 1 MB Redo Informationen enthalten sind
              \item Bevor der Database Writer schreibt
              \item Wenn ein Nutzer eine Transaktion mit \languageorasql{COMMIT} beendet
            \end{itemize}
          \end{merke}
          Der Log Writer Prozess schreibt synchron in alle Member der aktiven Log Gruppe. Ist einer defekt oder nicht verf\"ugbar, setzt der Log Writer das Schreiben in den anderen Dateien fort und gibt eine Fehlermeldung in seiner Trace-Datei und der Alert Log Datei aus.
\clearpage
          \begin{merke}
            Es sollten immer so viele Redo Log Gruppen vorhanden sein, dass der Log Writer zu jeder Zeit eine freie Gruppe finden kann, ohne warten zu m\"ussen.
          \end{merke}
          Der Log Writer und der Database Writer sind zwei Prozesse, die von einander abh\"angig sind. Bevor der Database Writer einen dirty block in die Datendateien schreiben darf, m\"ussen vorher alle mit diesem Block verbundenen Redo Log Eintr\"age durch den Log Writer in die Redo Logs \"ubertragen worden sein. Dies wird durch das \textit{write-ahead Protokoll} gew\"ahrleistet.
          \subsubsection{Log Switch}
            Wenn der Log Writer eine Gruppe zu 100 \% gef\"ullt hat wechselt er in die n\"achste freie Gruppe. Dieser Vorgang des Wechselns einer Redo Log Gruppe hei\ss t Log Switch. Aus der Forderung, dass der Log Writer immer eine freie Gruppe ben\"otigt, in die er wechseln kann, resultiert die Forderung, dass mindestens zwei Redo Log Gruppen vorhanden sein m\"ussen. Findet der Log Writer keine freie Gruppe zum Wechseln, bleibt die Datenbank stehen.

            Der Log Writer benutzt alle vorhandenen Gruppen der Reihe nach. Ist er bei der letzten Gruppe angekommen, versucht er die erste Gruppe erneut zu nutzen, d. h. die Redo Log Gruppen werden im Kreis immer wieder genutzt. Der Inhalt einer bereits bef\"ullten Gruppe geht bei erneuter Nutzung verloren. Um einen solchen Verlust zu vermeiden, kann ein weiterer Hintergrundprozess, der Archiver eingeschaltet werden.
            \begin{merke}
              Als Faustregel gilt: Es sollte ca. alle 20 Minuten ein Log Switch stattfinden. Dies kann durch die Gr\"o\ss{}e der Redo Log Dateien beeinflu\ss{}t werden.
            \end{merke}
          \subsubsection{Die Log Sequence Number (LSN)}
            Wenn der Log Writer beginnt, eine Redo Log Gruppe zu benutzten, ordnet er ihr eine Log Sequence Number zu. Die Log Sequence Number ist eine fortlaufende Nummer, anhand derer die Reihenfolge der Nutzung der Redo Log Gruppe erkannt werden kann. Wenn der Log Writer z. B. die Gruppe 1 zu erst benutzt, erh\"alt diese Gruppe die LSN 1. Nach einem Log Switch auf Gruppe 2 erh\"alt Gruppe 2 die LSN 2. Bei einem weiteren Log Switch hin zu Gruppe 1 erh\"alt Gruppe 1 die LSN 3 usw.
          \subsubsection{Status einer Redo Log Gruppe}
            Jede Redo Log Gruppe hat einen bestimmten Status. Der Status gibt an, ob die Gruppe gerade benutzt wird, bzw. ob sie f\"ur die Nutzung frei ist. Der Status kann sein:
            \begin{itemize}
              \item \textbf{Current}: Die Redo Log Gruppe, die aktuell durch den
              Log Writer benutzt wird hat den Status Current.
              \item \textbf{Active}: Eine Redo Log Gruppe, die f\"ur ein
              Instance-Recovery ben\"otigt wird hat den Status Active. Eine Redo
              Log Gruppe wird solange f\"ur ein Instance-Recovery ben\"otigt,
              bis der Database Writer alle betreffenden Oracle-Bl\"ocke aus dem
              Database Buffer Cache in die Datendateien geschrieben hat. Hat der
              DBWn seine Arbeit vollendet, sind die Informationen dieser Redo
              Log Gruppe nicht mehr f\"ur ein Instance-Recovery von N\"oten und
              der Status der Redo Log Gruppe wechselt zu \textit{Inactive}.
              \item \textbf{Inactive}: Redo Log Gruppen, die nicht mehr f\"ur
              ein Instance-Recovery ben\"otigt werden haben diesen Status.
            \end{itemize}

          \bild{Status einer Redo Log Grup\-pe}{redolog_status}{0.75}
        \subsection{Der Checkpoint Prozess (CKPT)}
          \bild{Oracle Datenbank\-architektur}{datenbankarchitektur14}{1}
          Der Checkpoint Prozess, kurz CKPT, geh\"ort seit Oracle 8 zum Kreise der Prozesse, die zwingend notwendig sind f\"ur den Betrieb einer Oracle-Datenbank. Sein Name leitet sich von einem Ereignis ab, das in regelm\"a\ss{}igen Intervallen auftritt, dem Checkpoint.
          \subsubsection{Checkpoints}
            Die offizielle Definition eines Checkpoints lautet:
            \begin{merke}
              Ein Checkpoint ist ein Ereignis, das durch andere Ereignisse ausgel\"ost wird (Log Switch, Manuell).
            \end{merke}
            Checkpoints haben die Aufgabe eine Datenbank konsistent zu halten. Dies geschieht, in dem bei jedem Auftreten eines Checkpoints die ge\"anderten Daten aus dem Database Buffer Cache auf den Datentr\"ager geschrieben werden, so wie dies bereits beschrieben wurde. Die haupts\"achliche Last eines Checkpoints tragen somit der DB Writer und der Log Writer.

            Nach dem Abschluss eines Checkpoints muss aber noch Verwaltungsarbeit durchgf\"uhrt werden. Die Datenbank muss sich \enquote{notieren}, dass, bzw. wann der Checkpoint beendet wurde. Dies geschieht, in dem der Checkpoint Prozess (CKPT), am Ende eines Checkpoints die System Change Number, kurz SCN, in die Kontrolldateien und die Header aller Datendateien schreibt.
          \subsubsection{Die System Change Number (SCN)}
            \begin{merke}
              Die System Change Number (SCN) ist eine fortlaufende Nummer, anhand derer das Alter einer Oracledatenbank bestimmt werden kann. Sie wird durch die verschiedensten Nutzeraktionen und durch die Arbeit der Hintergrundprozesse inkrementiert.
            \end{merke}
            Die aktuelle SCN kann mit der Prozedur \identifier{get\_system\_change\_number}, aus dem PL/SQL-Paket \identifier{dbms\_flashback} herausgefunden werden. Sie ist in der Kontrolldatei eingetragen und wird auch als \enquote{Referenz-SCN} bezeichnet. Die SCN des letzten Checkpoint ist im Feld \identifier{checkpoint\_change\#} in der V\$-View \identifier{v\$database} zu finden.

            \begin{literaturinternet}
              \item \cite{p665}
            \end{literaturinternet}
        \subsection{Der System Monitor (SMON)}
          \bild{Oracle Datenbank\-architektur}{datenbankarchitektur15}{0.8}

          Der SMON kann als eine Art \enquote{Reinigungskraft der Datenbank auf Betriebssystemebene} verstanden werden. Zu seinen Aufgaben z\"ahlt:
          \begin{itemize}
            \item \textbf{Instance-Recovery}: Wird die Datenbank nach einem Crash hochgefahren, muss der SMON die inkonsistente Datenbank in einen konsistenten Zustand \"uberf\"uhren.
            \item \textbf{Aufr\"aumarbeiten}: Der SMON muss immer wieder Aufr\"aumarbeiten in verschiedenen Systemtabellen der Datenbank (z. B. \identifier{OBJ\$}) durchf\"uhren.
            \item \textbf{Undo Segmente schrumpfen}: Der SMON schrumpft Undo Segmente automatisch auf ihre optimale Gr\"o\ss{}e.
            \item \textbf{Abschalten von Undo Segmenten}: Legt der DB-Admin fest, dass ein Undo Segment abgeschaltet werden soll, so f\"uhrt der SMON diese T\"atigkeit aus.
          \end{itemize}
        \subsection{Der Prozess Monitor (PMON)}
          Der Prozess Monitor stellt die Erg\"anzung zum System Monitor dar. Er ist die \enquote{Reinigungskraft auf Datenbankebene}. Zu seinen Aufgaben z\"ahlt:

          \begin{itemize}
            \item \textbf{Serverprozesse aufr\"aumen}: Wird eine Session abnormal getrennt bleibt der zugeh\"orige Serverprozess als sogenannter \enquote{Zombieprozess} stehen. Die Aufgabe des Prozessmonitors ist es, solche Zombies zu beenden.
            \item \textbf{Sperren l\"osen}
            \item \textbf{Zur\"uckrollen von Transaktionen}: Alle Transkationen die vor einem Instanz-Crash nicht beendet wurden, m\"ussen durch den PMON zur\"uckgerollt werden.
            \item \textbf{Hintergrundprozesse \"uberwachen}: PMON ist daf\"ur zust\"andig die anderen Hintergrundprozesse zu \"uberwachen und diese bei Bedarf neuzustarten.
            \item \textbf{Dynamic Service Registration}
          \end{itemize}

          \bild{Oracle Datenbank\-architektur}{datenbankarchitektur16}{0.9}

        \subsection{Der Archiver (ARCn)}
          Der Archiver Prozess ist daf\"ur zust\"andig, bef\"ullte Redo Log Dateien, nach einem \enquote{Log switch}, an einem definierten Speicherort zu sichern. Eine Redo Log Gruppe muss bei aktivierter Archivierung erst archiviert werden, bevor sie durch den Log Writer erneut genutzt werden kann. Ist die Archivierung nicht aktiviert, kann eine Redo Log Gruppe sofort wieder verwendet werden.

          Sind alle Member einer Redo Log Gruppe besch\"adigt oder nicht mehr vorhanden, kann der Archiver diese Gruppe nicht mehr archivieren. Dies f\"uhrt in dem Moment zum Stillstand der Datenbank, in dem der Log Writer versucht, diese Gruppe erneut zu nutzen.

          \bild{Oracle Datenbank\-architektur}{datenbankarchitektur17}{0.9}

          \begin{literaturinternet}
            \item \cite{CNCPT020}
          \end{literaturinternet}

          \begin{merke}
            Eine Redo Log Gruppe ist in zwei F\"allen f\"ur die Nutzung frei:
            \begin{itemize}
              \item Direkt nach ihrer Erstellung
              \item Wenn die Gruppe unbenutzt ist und wenn die Archivierung beendet wurde.
            \end{itemize}
          \end{merke}

  \chapter{Installieren einer Oracle 11g Release 2 Datenbank}
    \setcounter{page}{1}\kapitelnummer{chapter}
    \minitoc
\newpage
    \section{Aufgaben und Verantwortungsbereich eines DBA}
      Der Verantwortungsbereich eines Datenbankadministrators kann die folgenden Aufgaben einschlie\ss en:
      \begin{itemize}
        \item Installation und Upgrade des Datenbankservers und der Nutzeranwendungen
        \item Planen und \"Uberwachen des Speicherbedarfs der Datenbank auf dem Datentr\"ager
        \item Erstellen von Storage Strukturen in der Datenbank (z. B. Tabellen und Indizes)
        \item Nutzerkonten verwalten und \"uberwachen der Sicherheit
        \item Performance Monitoring und Tuning
        \item Entwickeln von Backup- und Recoverystrategien
        \item Verwalten von archivierten Daten
        \item Durchf\"uhren von Backup und Recovery
      \end{itemize}
      \subsection{Planen und durchf\"uhren einer Oracle Installation}
        \subsubsection{Lesen der Releasenotes}
          Die Releasenotes sind mit Sicherheit der am wenigsten beachtete Teil einer Software Dokumentation. So wenig Beachtung sie jedoch finden, um so wichtiger sind sie tats\"achlich. Sie enthalten wichtige Neuerungen zur Vorg\"angerversion und andere Hinweise, die bei einem Update auf eine neue Version, der Auswahl der notwendigen Hardware und anderen Preinstallation Tasks von gro\ss er Bedeutung sein k\"onnen.

          \begin{literaturinternet}
            \item \cite{e23557}
            \item \cite{e23558}
          \end{literaturinternet}

        \subsubsection{Evaluieren der vorhandenen Hardware}
          Vor der Installation sollte der DBA pr\"ufen, wie die vorhandene
          Hardware bestm\"oglich durch die Oracle-Datenbank und die ben\"otigten
          Anwendungsprogramme genutzt werden kann. Diese \"Uberpr\"ufung sollte
          die folgenden \"Uberlegungen einschlie\ss en:
          \begin{itemize}
            \item Wie viele Datentr\"ager sind f\"ur die Datenbank verf\"ugbar?
            \item Wie viel Arbeitsspeicher ist f\"ur die Oracle Instanz verf\"ugbar?
          \end{itemize}

          \begin{literaturinternet}
            \item \cite{BABFDGHJ}
            \item \cite{i1011417}
          \end{literaturinternet}

        \subsubsection{Planen der Datenbank}
          Zum Planen einer Datenbank geh\"oren die folgenden Aufgaben:
          \begin{itemize}
            \item Planen der logischen Speicherstrukturen der Datenbank
            \item Pr\"ufen des Datenbankdesigns
            \item Entwickeln von Backupstrategien
          \end{itemize}
          Die Planung der logischen Speicherstrukturen der Datenbank ist wichtig, um deren Einfl\"usse auf die Systemperformance erkennen zu k\"onnen. Beispielsweise ist es entscheidend, ob in einer Datenbank gro\ss e Objekte, wie z. B. Bilder oder ISO-Images gespeichert werden oder nur normale Text/Zahlen-Werte. Auch das Datenbankdesign ist entscheidend f\"ur die Performance der Datenbank. Meist hat der DBA zwar keinen direkten Einfluss\ auf das Design, er kann aber Hinweise und Verbesserungsvorschl\"age geben.
      \subsection{Download und Installation von Patches}
        Um die Datenbank sicherer zu machen und um Bugs zu beheben, muss der DBA von Zeit zu Zeit Patches oder Patchsets installieren. Ein Patch (auch \enquote{single interim patch} genannt) behebt ein einzelnes spezielles Problem und ist nicht bei jeder Installation notwendig. Ein Patchset ist eine Sammlung von Patches, die so erstellt wurde, das sie f\"ur jeden Kunden passt. Ein Patchset hat eine Releasenummer. Wurde beispielsweise die Oracle Version 11.2.0.0 installiert, dann hat das erste Patchset die Versionsnummer 11.2.0.1.
    \section{OFA - Die Optimal Flexible Architecture}
      \subsection{\"Uberblick \"uber die OFA}
        Die Optimal Flexible Architecture ist eine Menge von Namenskonventionen und Konfigurationsrichtlinien die zuverl\"assige Oracle Installationen mit minimalem Wartungsaufwand sicherstellen sollen. Der OFA-Standard wurde entwickelt um:
        \begin{itemize}
          \item eine gro\ss e Menge komplexer Anwendungen zusammen mit deren Nutzdaten auf einem Datentr\"ager zu speichern und dabei Bottlenecks und schlechte Performance zu vermeiden,
          \item administrative Routineaufgaben zu erleichtern, wie z. B. Backups,
          \item das Wechseln zwischen verschiedenen Oracle Instanzen zu erleichtern,
          \item es dem Administrator zu erm\"oglichen angemessen auf das Wachstum einer Datenbank reagieren zu k\"onnen,
          \item die Fragmentierung von Teilen der Datenbank m\"oglichst gering zu halten.
        \end{itemize}
        Die OFA kann als eine Art \enquote{Sammlung von guten Manieren} bei der Erstellung von Verzeichnissen f\"ur Oracle angesehen werden. Der Oracle Universal Installer platziert automatisch alle Oracle Komponenten in Verzeichnissen, die gem\"a\ss{} OFA erstellt wurden. Auch wenn die Nutzung einer OFA kein Muss ist, wird dies in jedem Falle von Oracle empfohlen.

        Der Oracle Universal Installer (OUI) trennt die Datenbanksoftware von den Nutzdaten. Er legt die Software im Oracle Homeverzeichnis \oscommand{\$ORACLE\_HOME} ab. Die Nutzdaten werden in \oscommand{\$ORACLE\_BASE/oradata} abgelegt. \oscommand{\$ORACLE\_BASE} und \oscommand{\$ORACLE\_HOME} sind Umgebungsvariablen die Verzeichispfade enthalten.

        Der Vorteil an dieser Methode ist, wenn eine neue Version der Datenbanksoftware installiert wird, hat diese ein neues Oracle Homeverzeichnis. Sie kann auf Zuverl\"assigkeit getestet werden und die alte Version kann nach erfolgreichen Tests der neuen Version gel\"oscht werden, ohne dass dabei Probleme entstehen.
      \subsection{Charakteristika einer OFA-Konformen Installation}
        Eine OFA-Konforme Installation  besitzt die folgenden Charakterristika:
        \begin{itemize}
          \item Unabh\"angige Unterverzeichnisse

            Dateien unterschiedlicher Arten werden in unterschiedliche Unterverzeichnisse gelegt. Somit ist eine Dateiart wenig bis gar nicht betroffen, wenn eine Andere ver\"andert oder gel\"oscht wird.
          \item Konsequente Namenskonventionen f\"ur die Dateien der Datenbank

            Die Dateien der Oracle-Datenbank k\"onnen einfach anhand der Verzeichnisstruktur von anderen Dateien unterschieden werden. Datenbankdateien unterschiedlicher Oracle Versionen k\"onnen auf die gleiche Art und Weise sehr einfach von einander unterschieden werden.
          \item Integrit\"at der Oracle-Home-Verzeichnisse

            Oracle-Home-Verzeichnisse k\"onnen hinzugef\"ugt, verschoben oder gel\"oscht werden, ohne das die betroffene Software dadurch besch\"adigt werden w\"urde.
\clearpage
					\item Metadaten unterschiedlicher Datenbanken von einander trennen

            Die strikte Trennung administrativer Daten von einer Datenbank gegen\"uber allen anderen erleichtert administrative Arbeiten deutlich und schafft eine \"ubersichtliche und zuverl\"a{}ssige Struktur.
          \item Strikte Trennung unterschiedlicher Arten von Nutzdaten

            Nutzdaten unterschiedlicher Arten k\"onnen von einander getrennt werden, um eine bessere Performance zu schaffen
          \item Verteilung von I/O-Last auf alle verf\"ugbaren Datentr\"ager
        \end{itemize}
      \subsection{Die OFA-Konventionen}
        Die OFA schreibt folgende Dateiendungen f\"ur Datenbankdateien vor:
        \begin{itemize}
          \item *.ctl Kontrolldatei
          \item *.dbf Datendatei
          \item *.log Redo Log Datei
        \end{itemize}
        \subsubsection{Das Oracle Base-Verzeichnis}
          Die Umgebungsvariable \oscommand{\$ORACLE\_BASE} enth\"alt den Pfad des Oracle Basisverzeichnisses. Es stellt die Wurzel des Oracleverzeichnisbaumes dar. Der OUI setzt f\"ur \oscommand{\$ORACLE\_BASE} automatisch den Wert

          \oscommand{/u01/app/oracle}

        \subsubsection{Das Oracle Homeverzeichnis}
          Die Umgebungsvariable \oscommand{\$ORACLE\_HOME} enth\"alt das Oracle Homeverzeichnis. Es ist der Speicherort f\"ur die Oracle Datenbanksoftware. \oscommand{\$ORACLE\_HOME} ist ein Unterverzeichnis von \oscommand{\$ORACLE\_BASE}. Zum Beispiel:

          \oscommand{/u01/app/oracle/product/11.2.0/orcl}

          \begin{literaturinternet}
            \item \cite{BABHAIIJ}
          \end{literaturinternet}

    \section{Durchf\"uhren einer Oracle Installation auf Linux}
      In dieser Sektion wird die Installation einer Oracle 11g R2 Datenbank auf einem Oracle Enterprise Linux 6 System beschrieben. Alle Voreinstellungen die am Betriebssystem gemacht werden m\"ussen, sind bereits get\"atigt worden. An dieser Stelle wird nur noch die reine Softwareinstallation beschrieben.
      \subsection{Der Installationsbeginn}
        Das Starten der Installation erfolgt durch den Aufruf: \oscommand{/media/CDROM/runInstaller}. Die Pfadangabe \oscommand{/media/CDROM} kann von System zu System variieren. Hierbei handelt es sich um den Pfad, an dem das CD/DVD-Laufwerk oder die ISO-Datei gemountet wurde. \oscommand{runInstaller} ist das Kommando zum Starten des Oracle Universal Installer, kurz OUI.
        \subsubsection{Schritt 1- Sicherheitsupdates}
          Der OUI ist ein grafisches Werkzeug zur Installation der Oracle-Datenbank-Software. Der Setup-Vorgang umfasst insgesamt neun Schritte.

          \bild{Schritt 1 - Sicherheitsupdates}{oui_step_1}{0.45}
\clearpage
          In Schritt 1  der Installation wird das Fenster f\"ur die Konfiguration von Sicherheitsupdates angezeigt. Hier kann der Administrator seine E-Mail-Adresse eingeben, um \"uber Sicherheitsupdates informiert zu werden. Zus\"atzlich kann er auch sein \enquote{My Oracle Support} Kennwort angeben, so dass Sicherheitsupdates direkt bezogen werden k\"onnen.

          Sollte der Administrator in diesem Schritt \enquote{vergessen haben}, seine E-Mail-Adresse anzugeben, wird er prompt auf diesen \enquote{Fehler} hingewiesen.

          Ein Klick auf den \enquote{Ja}-Button l\"ost das Problem.

          \bild{E-Mail-Adresse vergessen?}{oui_step_1_no_mail_address}{0.5}
        \subsubsection{Schritt 2 - Installationsoptionen}
          In diesem Installationsschritt wird abgefragt, ob:
          \begin{itemize}
            \item Die Software installiert und eine neue Datenbank angelegt werden soll,
            \item nur die Software installiert werden soll oder
            \item ob ein Upgrade einer bestehenden Datenbank durchgef\"uhrt werden soll.
          \end{itemize}
          \bild{Schritt 2 - Installationsoptionen}{oui_step_2}{0.35}
        \subsubsection{Schritt 3 - Grid-Installation Optionen}
          Dieser Schritt ist seit Oracle 11g neu. Die Version 11 der Oracle-Datenbank beinhaltet eine Software, die als \enquote{Grid Infrastructure} bezeichnet wird. Diese wird jedoch nur dann ben\"otigt, wenn eine Real Application Cluster Installation durchgef\"uhrt werden soll.
          \bild{Schritt 3 - Grid-Optionen}{oui_step_3}{0.38}
        \subsubsection{Schritt 4 - Produktsprachen}
          Oracle 11g R2 wird standardm\"a\ss{}ig in den Sprachen Deutsch und Englisch installiert. Zus\"atzlich stehen weitere 45 Sprachen zur Verf\"ugung.
          \bild{Schritt 4 - Produktsprachen}{oui_step_4}{0.38}
        \subsubsection{Schritt 5 - Datenbank Edition}
          Die Oracle-Datenbank existiert in verschiedenen Editionen. Es gibt die:
          \begin{itemize}
            \item Enterprise Edition
            \item Standard Edition
            \item Standard Edition One
            \item Personal Edition (nur unter Windows)
          \end{itemize}
          \bild{Schritt 5 - Datenbank Edition}{oui_step_5}{0.42}
          Der Unterschied zwischen diesen Editionen liegt in deren Funktionsumfang und in den Lizenzkosten. Die Enterprise Edition ist das umfangreichste Paket. Sie enth\"alt alle Features und sie bietet die M\"oglichkeit \enquote{Datenbank Optionen} hinzu zu installieren, um die F\"ahigkeiten der Datenbank noch mehr zu erweitern.

          Die Standard Edition hat einen eingeschr\"ankten Funktionsumfang. Sie ist nur noch f\"ur  mittlere Unternehmen geeignet, da Features fehlen, die im Umgang mit gro\ss{}en Datenmengen absolut unverzichtbar sind. F\"ur diese Edition ist es auch nicht m\"oglich, Datenbank Optionen hinzuzuf\"ugen.

          Die Standard Edition One ist das kleinste Softwarepaket. Sie hat einen sehr stark eingeschr\"ankten Funktionsumfang und ist somit nur noch f\"ur kleine Unternehmen oder Abteilungen gedacht.

          Sollte die Enterprise Edition ausgew\"ahlt worden sein, so k\"onnen \"uber den Button \enquote{Optionen w\"ahlen} weitere Datenbankoptionen hinzugef\"ugt werden.
          \bild{Schritt 5 - Datenbank Optionen hinzuf\"ugen}{oui_step_5_database_options}{0.42}
          \begin{merke}
            Einige Datenbankoptionen m\"ussen eigenst\"andig lizensiert werden, wodurch zus\"atzliche Lizenzkosten entstehen!
          \end{merke}
        \subsubsection{Schritt 6 - Installationsspeicherort}
          Um diesen Schritt erfolgreich zu beenden, m\"ussen zwei Angaben gemacht werden:
          \begin{itemize}
            \item \textbf{Oracle Base}: Das Oracle Base-Verzeichnis ist die Wurzel des Installationspfades. Alle weiteren Angaben beziehen sich standardm\"a\ss{}ig auf dieses Verzeichnis.
            \item \textbf{Softwareverzeichnis}: Dies ist das \oscommand{\$ORACLE\_HOME}-Verzeichnis. Hierhinein wird die Oracle-Software installiert. Bei diesem Verzeichnis handelt es sich um ein Unterverzeichnis von \oscommand{\$ORACLE\_BASE}.
          \end{itemize}
          \bild{Schritt 6 - Installationsspeicherort}{oui_step_6}{0.39}
        \subsubsection{Schritt 7 - Bestandsverzeichnis erstellen (nur Linux)}
          Beim Oracle Bestandsverzeichnis handelt es sich um ein Verzeichnis, in dem alle installierten Oracle-Produkte Bestandsdateien anlegen. Es wird bei der ersten Installation eines beliebigen Oracle-Produktes angelegt. Alle Produkte erstellen dort Unterverzeichnisse f\"ur ihre eigenen Bestandsdaten.
          \bild{Schritt 7 - Bestandsverzeichnis erstellen}{oui_step_7}{0.45}
        \subsubsection{Schritt 8 - Berechtigte Betriebssystemgruppen}
          Bei der Installation der Oracle-Datenbank werden zwei Betriebssystemgruppen angelegt, welche f\"ur die \enquote{Betriebssystemauthentifizierung} relevant sind: OSDBA und OSOPER. Mitglieder der Gruppe OSDBA haben innerhalb der Datenbank alle Berechtigungen. Wie der Name der Gruppe sagt, sollten nur DBAs Mitglied sein.

          Alle Mitglieder der Betriebssystemgruppe OSOPER haben eingeschr\"ankte administrative Rechte. F\"ur die t\"agliche Arbeit eines Oracle-DBAs ist diese Gruppe faktisch irrelevant.
\clearpage
          \bild{Schritt 8 - Berechtigte Betriebssystemgruppen}{oui_step_8}{0.39}
        \subsubsection{Schritt 9 - Voraussetzungen pr\"ufen}
          In diesem Schritt werden alle Bedingungen gepr\"uft, die betriebssystemseitig vor der Installation erf\"ullt sein m\"ussen.
          \bild{Schritt 9 - Voraussetzungen pr\"ufen}{oui_step_9}{0.39}
        \subsubsection{Schritt 10 - \"Uberblick}
          Bevor der Nutzer mit einem Klick auf den Button \enquote{Fertig stellen} die Installation startet wird ein \"Uberblick \"uber alle gew\"ahlten Einstellungen gegeben. Erstmalig wird hier auch die M\"oglichkeit geboten eine Antwortdatei mit allen get\"atigten Einstellungen erstellen zu lassen. Diese kann an sp\"aterer Stelle f\"ur eine automatisierte Installation genutzt werden.
          \bild{Schritt 10 - \"Uberblick}{oui_step_10}{0.45}
        \subsubsection{Schritt 11 - Produkt installieren}
          Hier erfolgt nun die eigentliche Installation.

          Unter Linux/Unix ist es notwendig zwei Aufgaben als Benutzer \enquote{root} durchzuf\"uhren, da diese lokale Administratorrechte erfordern. Beide Aufgaben liegen in Form von Shell-Skripten vor. Um diese Skripte ausf\"uhren zu k\"onnen, muss ein neues Terminalfenster ge\"offnet und die Identit\"at des Nutzers root angenommen werden.

          Nach dem Ausf\"uhren beider Skripte kann das neue Terminalfenster wieder geschlossen und das Fenster \enquote{Konfigurationsskripte ausf\"uhren} mit einem Klick auf \enquote{OK} geschlossen werden.

          \bild{Schritt 11 - Produkt installieren}{oui_step_11}{0.45}

          \bild{Schritt 11 - Das Skript \enquote{orainstRoot.sh}}{oui_step_11_orainstRoot}{0.45}
\clearpage
          \bild{Schritt 11 - Das Skript \enquote{root.sh}}{oui_step_11_root}{0.42}
        \subsubsection{Schritt 12 - Beenden}
          Zu guter Letzt erh\"alt man noch die Best\"atigung, dass die Installation erfolgreich verlaufen ist. Mit einem Klick auf \enquote{Schlie\ss{}en} kann der Oracle Universal Installer nun geschlossen werden.
          \bild{Schritt 12 - Beenden}{oui_step_12}{0.42}
    \section{Software installieren/deinstallieren}
      W\"ahrend der Installation der Oracle-Software wird auch eine Kopie des Oracle Universal Installers installiert. Diese befindet sich im Oraclesoftwareverzeichnis, Unterordner \oscommand{oui/bin}. Mit Hilfe dieses OUI kann weitere Software nachinstalliert bzw. k\"onnen Komponenten deinstalliert werden.

      Unter Microsoft Windows kann der OUI aus dem Startmen\"u ge\"offnet werden. F\"ur Linux-Systeme muss die Kommandozeile genutzt werden.

      \oscommand{/u01/app/oracle/product/11.2.0/ORCL/oui/bin/runInstaller}

      Nach dem Start erfolgt die \"ubliche Begr\"u\ss{}ung.
      \bild{Willkommen im OUI}{local_oui_welcome}{0.45}

      Hier erfolgt nun die Auswahl, was als n\"achstes geschehen soll. Mit einem Klick auf den Button \enquote{Installierte Produkte} kann das Oracle Bestandsverzeichnis abgefragt werden.

      Durch anhaken/markieren einzelner Komponenten und anklicken der Schaltfl\"ache \enquote{Entfernen} k\"onnen Teile der Software deinstalliert werden.

      Zur\"uck im Willkommensfenster kann mit einem Klick auf die Schaltfl\"ache \enquote{Weiter}, die Installation einer weiteren Oracle-Instanz begonnen werden. Die Installationsschritte bleiben dabei die gleichen, wie sie soeben beschrieben wurden.
\clearpage
      \bild{Auflistung aller installierten Produkte}{local_oui_orainventory}{0.45}

    \chapter{Erstellen einer Oracle Datenbank}
    \setcounter{page}{1}\kapitelnummer{chapter}
    \minitoc
\newpage
    \section{\"Uberlegungen vor der Erstellung}
      Eine Oracle Datenbank kann auf zwei verschiedene Arten erstellt werden:
      \begin{itemize}
        \item \textbf{Database Configuration Assistant (DBCA)}: Der DBCA kann w\"ahrend der Installation durch den OUI oder sp\"ater von Hand gestartet werden. Er stellt ein grafisches Interface f\"ur die Erstellung von Datenbanken zur Verf\"ugung.
        \item \textbf{CREATE DATABASE}: Das SQL-Kommando \languageorasql{CREATE DATABASE} bietet die M\"oglichkeit zur manuellen Erstellung einer Datenbank. Wird eine Datenbank auf diesem Weg erstellt, anstatt mit dem DBCA, m\"ussen zus\"atzliche Schritte ausgef\"uhrt werden, bis die Datenbank voll funktionsf\"ahig ist.
      \end{itemize}
      Bevor aber eine dieser beiden M\"oglichkeiten genutzt werden kann, muss zuerst einiges an Vorarbeit geleistet werden. Die folgenden \"Uberlegungen sollen dabei helfen, eine Oracle Datenbank planvoll zu erstellen.
      \subsection{Kapazit\"atenplanung}
        F\"ur jedes Datenbankobjekt, wie z. B. eine Tabelle, kann anhand der verwendeten Datentypen, der Spaltenanzahl und der gesch\"atzten maximal Anzahl Zeilen eine durchschnittliche Gr\"o\ss e berechnet werden. Um den Speicherbedarf einer Datenbank realistisch einsch\"atzen zu k\"onnen sollte dies f\"ur jedes Objekt geschehen. Folgendes Beispiel verdeutlicht die Kapazit\"atsplanung einer einzelnen Tabelle.
          \begin{verbatim}
Tabelle: tblKunden
Spalten:
  1.)    KundenID   NUMBER(4)           3 Byte
  2.)    Name       VARCHAR2(30)       30 Byte
  3.)    Vorname    VARCHAR2(25)       25 Byte
  4.)    Strasse    VARCHAR2(50)       50 Byte
  5.)    PLZ        VARCHAR2(5)         5 Byte
  6.)    Ort        VARCHAR2(30)       30 Byte
Voraussichtliche Anzahl Zeilen   : 300
          \end{verbatim}
          Die Gesamtgr\"o\ss e der Tabelle \identifier{tblKunden} wird somit wie folgt berechnet:

          \centerline{$300*(3+30+25+50+5+30)=42.900$ Byte}

      \subsection{Planen des physischen Datenbanklayouts}
        Unter dem physischen Layout einer Oracle Datenbank versteht man die Aufteilung der selben in einzelne Dateien und die Verteilung dieser Dateien auf verschiedene Datentr\"ager. Der Einsatz von Mirroring- und Striping-Mechanismen wird ebenfalls zum physischen Datenbanklayout gez\"ahlt. Eine vern\"unftige Planung kann sich entscheidend auf Performance und Ausfallsicherheit der Datenbank auswirken.
        \subsubsection{Oracle Managed Files (OMF) und Automatic Storage Management (ASM)}
          OMF und ASM sind zwei Mechanismen, die den Administrator bei der Verwaltung einer Datenbank unterst\"utzen sollen. Beide sind aber nicht zwingend notwendig f\"ur den Betrieb.
        \subsubsection{Ausw\"ahlen des globalen Datenbanknamens}
          Der globale Datenbankname ist der Name der Datenbank, der sie eindeutig identifiziert, z. B. \enquote{ORCL}.
        \subsubsection{Initialisierungsparameter}
          Die Initialisierungsparameter beeinflussen das Verhalten einer Oracle-Instanz. Eine ung\"unstige Konfiguration kann sich negativ auf die Performance der Datenbank auswirken.
        \subsubsection{Der Datenbankzeichensatz}
          Je nach dem welcher Zeichensatz f\"ur die Datenbank ausgew\"ahlt wurde kann diese verschiedene Sprachen darstellen und andere nicht. Dieser Punkt gewinnt im multinationalen Betrieb eine erh\"ohte Bedeutung. F\"ur die Auswahl des richtigen Datenbankzeichensatzes sind die folgenden Punkte entscheidend:
          \begin{itemize}
            \item Welche Sprachen sollen in der Datenbank dargestellt werden (jetzt und in Zukunft)?
            \item Ist der Zeichensatz im verwendeten Betriebssystem verf\"ugbar?
            \item Welchen Zeichensatz nutzen die Clients?
          \end{itemize}

          \begin{literaturinternet}
            \item \cite{NLSPG002}
          \end{literaturinternet}

        \subsubsection{Ausw\"ahlen der korrekten Zeitzone(n)}
          Die gew\"ahlte Zeitzone ist dann entscheidend, wenn Clients in unterschiedlichen Zeitzonen stehen. Sie ist daf\"ur zust\"andig, eine korrekte Umrechnung der Datum/Uhrzeit-Angaben des Clients in die Zeitzone des Servers sicherzustellen.

          \begin{literaturinternet}
            \item \cite{i1006705}
          \end{literaturinternet}

        \subsubsection{Festlegen der Oracle Datenblockgr\"o\ss e}
          Die Standard Oracle Blockgr\"o\ss e betr\"agt 8 Kb. Abh\"angig vom verwendeten Dateisystem und dem Verwendungszweck der Datenbank muss dieser Wert angepasst werden, um die Datenbank performant zu machen.
        \subsubsection{Absch\"atzen der passenden Gr\"o\ss e f\"ur den Sysaux-Tablespace}
          Der \identifier{Sysaux}-Tablespace wird automatisch bei der Datenbankerstellung generiert. Er beinhaltet Nutzdaten f\"ur verschiedene Oracle Features wie z. B. Oracle Text, Ultra Search, Log Miner, Oracle Spatial und andere. Erstellt wird dieser Tablespace abh\"angig von den gew\"ahlten Oracle Features, in der richtigen Gr\"o\ss e. Der Administrator muss sich darum k\"ummern, dass dieser Tablespace auf einem Datentr\"ager mit gen\"ugend freiem Speicher erstellt wird.
        \subsubsection{Einplanen von Standardtablespaces f\"ur Nutzer}
          Wenn Nutzer neue Datenbankobjekte anlegen, werden diese im \enquote{Defaulttablespace} des jeweiligen Nutzers abgelegt. Wurde einem Nutzer kein Defaulttablespace zugewiesen oder existiert kein Tablespace, au\ss{}er dem \identifier{Sysaux}-Tablespace, wird automatisch dieser verwendet. SYSTEM stellt jedoch das Herzst\"uck einer Oracle Datenbank dar und sollte deshalb niemals als Defaulttablespace dienen.
    \section{Der Database Configuration Assistant (DBCA)}
      Der DBCA stellt ein grafisches Werkzeug dar, mit dessen Hilfe folgende T\"atigkeiten durchgef\"uhrt werden k\"onnen:
      \begin{itemize}
        \item Eine Datenbank erstellen
        \item Konfigurieren einer Datenbank
        \item Datenbanken l\"oschen
        \item Datenbanktemplates erstellen und verwalten
      \end{itemize}
      Gestartet wird der DBCA unter Windows mit Hilfe des Startmen\"us. Unter Linux muss, in einem Terminalfenster, der gesamte Pfad zur Datei \oscommand{dbca}, also \oscommand{\$ORACLE\_HOME/bin/dbca} angegeben werden. Dies kann jedoch nur dann funktionieren, wenn bereits die Umgebungsvariable \oscommand{\$ORACLE\_HOME} gesetzt wurde. Andernfalls muss der Pfad manuell angegeben werden: \oscommand{/u01/app/oracle/product/11.2.0/<SID>/bin/dbca} (<SID> stellt hier einen Platzhalter f\"ur die SID der Datenbank dar.). Wie gewohnt, wird der Nutzer zu aller erstbegr\"u\ss{}t.

      \bild{Willkommen im DBCA}{dbca_welcome}{3}

      Nach einem Klick auf den Button \enquote{Weiter} muss zuerst die gew\"unschte T\"atigkeit ausgew\"ahlt werden.

      \bild{Schritt 1 - Vorg\"ange}{dbca_step_1}{3}
      \subsection{Eine Datenbank mit dem DBCA erstellen}
        \label{db_mit_dbca_erstellen}
        F\"ur die Erstellung einer Datenbank bietet der DBCA zwei Varianten:
        \begin{itemize}
          \item Erstellen einer benutzerdefinierten Datenbank
          \item Erstellen einer Datenbank mittels einer Vorlage
        \end{itemize}
        Die erste Option erlaubt es, eine Datenbank komplett selbst zu
        gestalten, jedoch mit dem Nachteil, dass hier sehr viel Arbeit zu
        leisten ist. Die Erstellung einer Datenbank aus einer Vorlage hingegen
        bietet nahezu genauso viele M\"oglichkeiten, entlastet aber den
        Ersteller deutlich.
        \subsubsection{Schritt 2 - Datenbankvorlagen}
        \bild{Schritt 2 - Datenbankvorlagen}{dbca_step_2}{2.5}
          Standardm\"a\ss{}ig werden zwei Vorlagen durch den DBCA angeboten:
          \begin{itemize}
            \item Allgemeiner Gebrauch oder Transaktionsverarbeitung
            \item Data Warehouse
          \end{itemize}
          Diese beiden unterscheiden sich haupts\"achlich in den gespeicherten Initialisierungsparametern, mit denen die fertige Datenbank gestartet wird. Welche der beiden Vorlagen genutzt wird h\"angt davon ab, welchen Einsatzzweck die Datenbank haben soll.

          Ein Datawarehouse, auch als Decision Support System bezeichnet, ist eine Datenbank, die oft eine sehr gro\ss{}e Datenmenge h\"alt und durch aufwendige SQL-Anfragen analysiert wird. Das hei\ss{}t, dass f\"ur eine solche Datenbank eine Optimierung der Lesevorg\"ange im Vordergrund steht. Schreibvorg\"ange haben hier lediglich eine untergeordnete Rolle, da sie selten erfolgen und meist gro\ss{}e Datenmengen am St\"uck, in die Datenbank importiert werden.

          In einer OLTP - Oline Transaction Processing - Datenbank ist das Verh\"altnis zwischen Lese- und Schreibzugriffen ganz anders. Daraus folgt, dass die Datenbank f\"ur beide Vorg\"ange so optimal wie m\"oglich eingestellt werden sollte (Kompromissl\"osung).

          Ein Klick auf den Button \enquote{Details zeigen\dots} zeigt die Einstellungen der Vorlagen.

          \bild{Schritt 2 - Vorlagen-Details}{dbca_step_2_template_details}{3}
        \subsubsection{Schritt 3 - Datenbank-ID}
          Nach einem Klick auf \enquote{Weiter} muss in Schritt 3 der Datenbank ein Name gegeben werden. Zu diesem Thema gibt es jedoch einiges zu sagen.

          Eine Oracle-Datenbank hat, \"ahnlich wie ein Mensch, einen Vornamen und einen Nachnamen. Korrekt ausgedr\"uckt hat eine Oracle-Datenbank einen Datenbanknamen und eine Datenbankdom\"ane. Der Datenbankname kann maximal acht Zeichen umfassen und sollte daf\"ur genutzt werden, um den Einsatzzweck der Datenbank wieder zu spiegeln. Datenbanknamen wie \identifier{ORCL} oder \identifier{ORA11G} sind maximal in Trainingsumgebungen sinnvoll. In einer Produktivumgebung sollten Namen wie \identifier{dss} f\"ur ein Decision Support System oder \identifier{pers} f\"ur eine Personalverwaltung genutzt werden. Evtl. ist es auch n\"utzlich, dem Datenbanknamen eine Versionsnummer anzuh\"angen, um im Falle einer Migration die alte und die neue Datenbank am Namen unterscheiden zu k\"onnen.

          Die Datenbankdom\"ane ist der zweite Namensbestandteil. Sie kann dazu genutzt werden, um dem Datenbanknamen zus\"atzliche Informationen hinzuzuf\"ugen. Wenn beispielsweise zu einer produktiven Datenbank ein Testsystem hinzugef\"ugt werden soll, k\"onnten die Dom\"anen \identifier{prod.oracle.local} und \identifier{test.oracle.local} dazu genutzt werden, um beide Systeme zu unterscheiden. F\"ugt man beide Informationen zusammen, den Datenbanknamen und die Datenbankdom\"ane, so erh\"alt man den \enquote{Globalen Datenbanknamen}.
          \begin{merke}
            Der Globale Datenbankname besteht aus dem Datenbanknamen und der Datenbankdom\"ane. Er wird im Format \oscommand{db\_name.db\_domain} angegeben.
          \end{merke}
          In der Welt der Oracle-Datenbanken hat jedoch nicht nur die Datenbank einen Namen, sondern auch die zu ihr geh\"orende Instanz. Der Instanzname wird im Betriebssystem, in einer Umgebungsvariablen namens \identifier{ORACLE\_SID} hinterlegt. Die Abk\"urzung SID (gesprochen S-ID) steht f\"ur System Identifier. Welche L\"ange die SID haben darf ist Betriebssystemspezifisch, jedoch sind auf nahezu allen Platformen acht Zeichen ohne Probleme m\"oglich. Da die Instanz keine Dom\"ane als Erg\"anzung hat, kann hier eine Unterscheidung, z. B. zwischen Test- und Produktivsystem nur im Instanznamen selbst erfolgen, beispielsweise \enquote{ppers} oder \enquote{tpers}.
          \begin{merke}
            Der Instanznamen muss nicht gleich dem Datenbanknamen sein. Es empfiehlt sich jedoch beide Namen gleich zu w\"ahlen, um Datenbank und Instanz zusammen finden zu k\"onnen.
          \end{merke}
          \bild{Schritt 3 - Datenbank-ID}{dbca_step_3}{3}
        \subsubsection{Schritt 4 - Verwaltungsoptionen}
          Dieser Schritt gliedert sich in zwei Registerkarten. Auf dem Register \enquote{Enterprise Manager} kann gew\"ahlt werden, wie die Datenbank verwaltet werden soll.
          \begin{itemize}
            \item \textbf{Enterprise Manager Grid Control}: Grid Control ist eine zentrale Verwaltungskonsole f\"ur verschiedenste Oracle-Produkte, wie z. B. die Datenbank oder den Application Server. Sie kann f\"ur die Verwaltung beliebig vieler Systeme genutzt werden.
            \item \textbf{Enterprise Manager Database Control}: Hierbei handelt es sich um eine lokale Verwaltungskonsole f\"ur nur eine Datenbank.
            \item \textbf{Kein Enterprise Manager}: Es ist nicht zwingend notwendig f\"ur die Verwaltung einer Oracle-Datenbank den Enterprise Manager einzusetzen. Soll lediglich SQL*Plus genutzt werden oder sind Produkte von Drittanbietern im Einsatz, kann der Enterprise Manager einfach weggelassen werden.
          \end{itemize}
          \bild{Schritt 4 - Verwaltungs\-optionen}{dbca_step_4}{2.7}
          \begin{merke}
            Die Option \enquote{Enterprise Manager konfigurieren} kann seit Oracle 11g erst genutzt werden, wenn ein Listener (Siehe \enquote{Konfigurieren der Oracle Netzwerkumgebung}) konfiguriert und eine Datenbank erstellt wurde.
          \end{merke}
          Die Registerkarte \enquote{Automatische Wartungs-Tasks} bietet dem Administrator die M\"oglichkeit, automatische Wartungs-Tasks der Datenbank zu aktivieren. Dabei handelt es sich um Jobs, wie das Sammeln von Performance-Statistiken oder das Generieren von Berichten.
          \bild{Schritt 4 - Verwaltungs\-optionen}{dbca_step_4_automatic_maintenance_tasks}{3}
        \subsubsection{Schritt 5 - Datenbank-ID-Daten}
          In Schritt 5 m\"ussen Passw\"orter f\"ur die beiden Datenbankbenutzer \identifier{SYS} und \identifier{SYSTEM} festgelegt werden. Der Nutzer \identifier{SYS} ist ein Account mit uneingeschr\"ankten Rechten. Er ist von zentraler Bedeutung in einer Oracle-Datenbank. Diese Tatsache sollte sich in der Komplexit\"at seines Passwortes reflektieren.

          Das Benutzerkonto \identifier{SYSTEM} hat ebenfalls administrative Berechtigungen, ist jedoch st\"arker eingeschr\"ankt, als \identifier{SYS}. F\"ur diesen Nutzer sollte ebenso ein sicheres Passwort gew\"ahlt werden, da er das Recht hat, den gesamten Datenbankinhalt zu exportieren.

          \bild{Schritt 5 - Datenbank-ID-Daten}{dbca_step_5}{2.7}
        \subsubsection{Schritt 6 - Speicherort von Datenbankdateien}
          Eine Oracle Datenbank besteht aus einer Vielzahl unterschiedlicher Dateien. Damit diese erstellt werden k\"onnen, muss ein Speicherort vorgegeben werden. Der DBCA bietet dazu die beiden grunds\"atzlichen M\"oglichkeiten:
          \begin{itemize}
            \item Speichern der Dateien in einem Dateisystem
            \item Nutzung der Oracle-Eigenen Technologie \enquote{Automatic Storage Management}
          \end{itemize}
          Egal welche der beiden Methoden gew\"ahlt wurde, es muss nun noch die Auswahl des genauen Speicherortes getroffen werden. Es kann der Speicherort aus der Vorlage genutzt, ein eigenes Verzeichnis f\"ur alle Dateien angegeben oder die Techonologie \enquote{Oracle Managed Files}, die an sp\"aterer Stelle noch besprochen wird, genutzt werden.

          Falls keine dieser drei M\"oglichkeiten die richtige ist, kann am Ende des Assistenten der Speicherort einer jeden einzelnen Datei ge\"andert werden.
          \bild{Schritt 6 - Speicherort von Daten\-bank\-dateien}{dbca_step_6}{3}

          Um herauszufinden, was hinter der Angabe \oscommand{\textbraceleft{}ORACLE\_BASE\textbraceright{}/oradata} steckt, kann der Button \enquote{Variablen f\"ur Dateispeicherort} angeklickt werden.
          \bild{Schritt 6 - Variablen f\"ur Dateispeicherort}{dbca_step_6_file_locations}{4}
        \subsubsection{Schritt 7 - Recovery-Konfiguration}
          Dieser Dialog bietet zwei Optionen:
          \begin{itemize}
            \item \textbf{Flash Recovery-Bereich angeben}: Hierbei handelt es sich um ein von der Datenbank \"uberwachtes Verzeichnis, welches als Puffer f\"ur Backups dient.
            \item \textbf{Archivierung aktivieren}: Diese Option aktiviert den Archiver-Hintergrundprozess, der daf\"ur sorgt, dass automatisch Kopien der benutzten Redo Log Dateien angelegt werden.
          \end{itemize}
          \bild{Schritt 7 - Recovery-Konfiguration}{dbca_step_7}{2.5}
        \subsubsection{Schritt 8 - Datenbankinhalt}
          Hier ist es m\"oglich, die Datenbank mit Beispielinhalten f\"ur Test-
          und Schulungszwecke zu f\"ullen. Produktivsysteme sollten niemals die
          Beispielschemata enthalten, da diese bekannt sind und eine
          zus\"atzliche Angriffsfl\"ache darstellen.

          Auf der zweiten Registerkarte \enquote{Benutzerdefinierte Skripts}
          k\"onnen eigene SQL-Skripte angegeben werden, die der DBCA automatisch
          nach Erstellung der Datenbank ausf\"uhrt.
          \bild{Schritt 8 - Daten\-bank\-inhalt}{dbca_step_8}{2.8}
        \subsubsection{Schritt 9 - Initialisierungsparameter}
          Dies ist der wohl umfangreichste Dialog des DBCA. Auf der
          Registerkarte \enquote{Speicher} muss die Entscheidung getroffen
          werden, wie der Arbeitsspeicher der Instanz verwaltet werden soll.
          Folgende Methoden stehen zur Verf\"ugung:
          \begin{itemize}
            \item \textbf{Automatic Memory Management}: Bei dieser Option wird der Instanz eine Speichermenge zugeteilt, die diese dann vollst\"andig autark verwaltet. Es wird keine Unterscheidung in SGA und PGA gemacht. Der Standard sind 40 \% der gesamten Arbeitsspeichermenge des Servers.
            \bild{Schritt 9 - Automatic Memory Management}{dbca_step_9_automatic_memory_management}{3}
            \item \textbf{Automatic Shared Memory Management}: Dies ist der Vorg\"anger zum Automatic Memory Management. Hier werden der Instanz zwei getrennte Werte f\"ur die Speichermenge der SGA und der aggregierten PGAs gegeben. Beide Speicherbereiche werden unabh\"angig von einander von der Instanz verwaltet.
            \bild{Schritt 9 - Automatic Shared Memory Management}{dbca_step_9_automatic_shared_memory_management}{0.35}
\clearpage
            \item \textbf{Manual Shared Memory Management}: Mit dieser Variante
            muss f\"ur jeden Speicherpool in der SGA und f\"ur die aggregierten
            PGAs ein eigener Wert angegeben werden. Auch wenn der DBA hier die
            gr\"o\ss{}ten Einflussm\"oglichkeiten hat, so sollte doch eine der
            beiden automatischen Verwaltungsoptionen genutzt werden, da diese
            sich immer wieder den aktuellen Gegebenheiten anpassen.
            \bild{Schritt 9 - Manual Shared Memory Management}{dbca_step_9_manual_shared_memory_management}{0.3}
          \end{itemize}
          Nach der Einstellung der Speicherverwaltung kann nur auf der zweiten
          Registerkarte, mit dem Namen \enquote{Skalierung}, die Option
          \enquote{Prozesse} ge\"andert werden. Dieser Wert ist aus zwei
          Gr\"unden interessant:
          \begin{itemize}
            \item lizenzrechtlich: F\"ur jeden Client muss eine
            Cient-Access-Licence vorliegen.
            \item Serverauslastung: Damit der Server nicht \"uberlastet wird.
          \end{itemize}
          Der zweite Wert, die \enquote{Blockgr\"o\ss{}e} kann nicht ge\"andert
          werden, wenn die Datenbank aus einer Vorlage heraus erstellt wird. Nur
          bei einer benutzerdefinierten DB ist dies m\"oglich.
          \bild{Schritt 9 - Skalierung}{dbca_step_9_scalability}{2.8}
          Auf der dritten Registerkarte k\"onnen zwei Zeichens\"atze f\"ur die
          Datenbank ausgew\"ahlt werden, der Datenbankzeichensatz und der
          l\"anderspezifische Zeichensatz.
          \begin{merke}
            Ein Zeichensatz besteht, wie sein Name besagt, aus einer Menge von Zeichen (Buchstaben,               Ziffern, Sonderzeichen). Jedem Zeichen wird ein numerischer Code zugeordnet. Dieser wird vom Computer ben\"otigt, um die Zeichen verarbeiten zu k\"onnen.
          \end{merke}
          \bild{Schritt 9 - Zeichens\"atze}{dbca_step_9_character_sets}{2.5}
          Der Datenbankzeichensatz wird f\"ur die folgenden Aufgaben verwendet:
          \begin{itemize}
            \item Speichern von Daten in den Datentypen \identifier{CHAR}, \identifier{VARCHAR2}, \identifier{CLOB} und \identifier{LONG},
            \item Speichern von Objektbezeichnern, Spaltennamen und Variablen\-be\-zeich\-nern,
            \item Speichern von SQL und PL/SQL-Quellcode.
          \end{itemize}
          Bevor der Datenbankzeichensatz ausgew\"ahlt wird, sollten die folgenden \"Uberlegungen angestellt werden:
          \begin{itemize}
            \item Welche Landessprachen muss die Datenbank jetzt und in zukunft unterst\"utzen?
            \item Ist der gew\"unschte Zeichensatz auch auf dem Betriebssystem des Datenbankservers verf\"ugbar und welche Zeichens\"atze nutzen die Clients?
            \item Kommen die genutzten Anwendungen mit dem Zeichensatz zurecht?
            \item Gibt es Performance-Probleme oder andere Einschr\"ankungen bei der Nutzung dieses Zeichensatzes?
          \end{itemize}
          \begin{merke}
            Der Datenbankzeichensatz kann nach der Datenbankerstellung nur unter einer Bedingung ge\"andert werden: Der neue Zeichensatz muss eine strikte Obermenge des Aktuellen sein. Oft ist dies nicht der Fall, weshalb der Datenbankzeichensatz, auf Empfehlung von Oracle, immer \identifier{AL32UTF8} sein sollte, da dieser am umfassendsten ist.
          \end{merke}
          Der l\"anderspezifische Zeichensatz dient dazu, um Unicode-Zeichen in einer Datenbank zu speichern, die keinen Unicode-Datenbankzeichensatz nutzt. Nur die Datentypen \identifier{NCHAR}, \identifier{NVARCHAR2} und \identifier{NCLOB} unterst\"utzen diesen alternativen Zeichensatz.
          Auf der Registerkarte \enquote{Verbindungsmodus} kann die Art und Weise gew\"ahlt werden, wie sich Clients standardm\"assig mit der Datenbank verbinden sollen. Die beiden Modi \enquote{Dedizierter Server} und \enquote{Shared Server} werden sp\"ater im Skript noch n\"aher erl\"autert.
        \subsubsection{Schritt 10 - Datenbankspeicherung}
          In diesem vorletzten Schritt k\"onnen die Speicherorte aller Datenbankdateien ge\"andert werden. Des Weiteren ist es m\"oglich, verschiedene Optionen f\"ur einzelne Dateiarten zu \"andern.
          \bild{Schritt 10 - Datenbankspeicher}{dbca_step_10}{3.2}
        \subsubsection{Schritt 11 - Optionen f\"ur das Erstellen}
          In Schritt 11 von 11 bleibt nur die Auswahl, was mit den soeben get\"atigten Einstellungen geschehen soll. Soll damit eine neue Datenbank erstellt, ein neues Template kreiert oder ein SQL-Skript generiert werden. Diese Optionen sind kombinierbar.

          An dieser Stelle muss der Nutzer auf den \enquote{Beenden}-Button klicken, um eine Zusammenfassung des Erstellvorgangs angezeigt zu bekommen.
\clearpage
          \bild{Schritt 11 - Optionen f\"ur das Erstellen}{dbca_step_11}{2.5}
					\bild{Schritt 12 - Best\"atigung}{dbca_step_12_confirmation}{2.5}
          Nach einem Klick auf \enquote{OK} bleibt nur noch abzuwarten, bis die Datenbank fertig ist.
          \bild{Schritt 13 - Daten\-bank\-erstellung}{dbca_step_13_database_creation}{2.7}
\clearpage
				\subsection{Konfigurieren einer Datenbank mit dem DBCA}
          Der DBCA bietet die M\"oglichkeit, Konfigurationseinstellungen von Datenbankoptionen zu \"andern. Folgende Optionen stehen zur Verf\"ugung:
          \begin{itemize}
            \item Einbinden von installierten Komponenten in die Datenbank
            \item \"Andern des Verbindungsmodus
          \end{itemize}
          \subsubsection{Schritt 1 - Starten des DBCA}
%             \bild{Schritt 1 - Vorg\"ange}{dbca_configure_step_1}{2.8}
            Nach dem Start des DBCA muss die Option \enquote{Datenbankoptionen konfigurieren} ausgew\"ahlt werden.
          \subsubsection{Schritt 2 - Datenbank }
            Im zweiten Schritt wird eine Auswahl aller installierten Datenbanken geboten. Hier ist die Richtige auszuw\"ahlen.
            \bild{Schritt 2 - Datenbank}{dbca_configure_step_2}{2.8}

            Nachdem die Datenbank ausgew\"ahlt wurde, werden deren Einstellungen eingelesen.
            \bild{Schritt 2 - Datenbank}{dbca_configure_step_2b}{2.8}
          \subsubsection{Schritt 3 - Verwaltungsoptionen}
            Wie unter Schritt 4 der Datenbankerstellung beschrieben kann hier der Enterprise Manager f\"ur die Datenbank konfiguriert werden.
            \bild{Schritt 3 - Verwaltungs\-optionen}{dbca_configure_step_3}{0.375}
          \subsubsection{Schritt 4 - Datenbankinhalt}
            Hier k\"onnen verschiedene Datenbankoptionen hinzugef\"ugt oder entfernt werden. Dies ist nur m\"oglich, wenn es sich um eine benutzerdefinierte Datenbank handelt.
            \bild{Schritt 4 - Datenbankinhalt}{dbca_configure_step_4}{0.375}
          \subsubsection{Schritt 5 - Verbindungsmodus}
            Im letzten Schritt ist das \"Andern des Verbindungsmodus m\"oglich.
\clearpage
            \bild{Schritt 5 - Verbindungsmodus}{dbca_configure_step_5}{0.375}

    \chapter{Verwalten einer Oracle Instanz}
    \setcounter{page}{1}
    \kapitelnummer{chapter}
    \minitoc
\newpage
    Eine Oracle-Umgebung besteht immer aus zwei Teilen: Datenbank und Instanz.
    Beim Start-Up, dem \enquote{Hochfahren der Datenbank}, wird die Instanz als
    Shared-Memory-Block im Arbeitsspeicher erzeugt. Dieser wird in die SGA und
    die PGA unterteilt. Des Weiteren werden verschiedene Hintergrundprozesse
    gestartet, die die Arbeit in der SGA verrichten.
    \section{Das SQL*Plus-Tool}
      Das SQL*Plus-Tool (Aussprache: sequel plus) ist ein textbasiertes, interaktives Tool, welches haupts\"achlich f\"ur administrative Aufgaben gedacht ist. Es wird automatisch bei jedem Oracle-Datenbankserver mit installiert. Ebenso kann es mit Hilfe der Oracle Client Tools auf einem Rechner ohne Datenbankserver installiert werden.
      \bild{Das SQL*Plus-Tool}{sql_plus}{1}
      Das SQL*Plus-Tool kennt drei verschiedene Arten von Befehlen:
      \begin{itemize}
        \item SQL-Kommandos
        \item PL/SQL-Kommandos
        \item SQL*Plus-Befehle
      \end{itemize}
      W\"ahrend SQL- und PL/SQL-Befehle durch die Datenbank verarbeitet werden,
      bleiben SQL*Plus-Kommandos lokal. Sie dienen nur zur Formatierung der
      Anzeige in SQL*Plus.
      \subsection{Die erste Anmeldung}
        Das Anmelden an einer Datenbank mit Hilfe des SQL*Plus-Tools verl\"auft
        in drei Schritten:
        \begin{enumerate}
          \item \"Offnen eines Terminalfensters
          \item Ausw\"ahlen der gew\"unschten Instanz
          \item Starten von SQL*Plus
        \end{enumerate}
        Um eine Instanz ausw\"ahlen zu k\"onnen, liefert Oracle das Shell-Skript
        \oscommand{oraenv} aus.
        \bild{Das oraenv Shell-Skript}{oraenv}{1}
        \begin{merke}
          Beachten Sie den Punkt im Aufruf: \oscommand{. oraenv}
        \end{merke}
        Der Start des SQL*Plus-Tools erfolgt mittels der Kommandozeile: \oscommand{sqlplus / as sysdba}. Die Bedeutung des Zusatzes \oscommand{ / as sysdba} wird an sp\"aterer Stelle noch behandelt.
      \subsection{Die wichtigsten SQL*Plus-Befehle}
        \begin{center}
          \tablecaption{Die wichtigsten SQL*Plus-Befehle}
          \label{importantsqlpluscommands}
          \begin{small}
            \tablefirsthead{
              \multicolumn{1}{c}{\textbf{Befehl}} &
              \multicolumn{1}{c}{\textbf{Beispiel}} &
              \multicolumn{1}{c}{\textbf{Erl\"auterung}} \\
              \hline
            }
            \tablehead{
              \multicolumn{1}{c}{\textbf{Befehl}} &
              \multicolumn{1}{c}{\textbf{Beispiel}} &
              \multicolumn{1}{c}{\textbf{Erl\"auterung}} \\
              \hline
            }
            \tabletail{
              \hline
            }
            \tablelasttail {
              \hline
            }
            \begin{supertabular}{|p{4.25cm}|p{4.25cm}|p{6.25cm}|}
              \oscommand{show user} & \oscommand{show user} & Zeigt den aktuellen Benutzernamen an. \\
              \hline
              \oscommand{conn[ect]} & \oscommand{conn hr/hr} & \"Offnet eine Session mit dem angegebenen Nutzernamen/Passwort. \\
              \hline
              \oscommand{disconn[ect]} & \oscommand{disconn} & Beendet die aktuelle Session. \\
              \hline
              \oscommand{exit} & \oscommand{exit} & Beendet die aktuelle Session und schlie\ss{}t das SQL*Plus-Tool. \\
              \hline
              \oscommand{ho[st]} & \oscommand{ho} & Verl\"asst SQL*Plus und wechselt in einen Terminal. Das SQL*Plus-Tool bleibt im Hintergrund ge\"offnet. Durch die Eingabe von \oscommand{exit} kann zur\"uckgewechselt werden. \\
              \hline
              \oscommand{desc[ribe]} & \oscommand{desc employees} & Zeigt die Definition einer Tabelle an. \\
              \hline
              \oscommand{ed[it]} & \oscommand{ed} & \"Offnet das letzte SQL-Statement in einem Editor. Das Statement kann ge\"andert und erneut ausgef\"uhrt werden. \\
              \hline
              \oscommand{col xx format  aNN} & \oscommand{col mail format a20} & Breite einer Tabellenspalte der Typen \identifier{CHAR}, \identifier{VARCHAR2} oder \identifier{DATE} begrenzen. \oscommand{xx} steht f\"ur einen Spaltenbezeichner und \oscommand{NN} steht f\"ur die Breite. \\
              \hline
              \oscommand{col xx format NN} & \oscommand{col salary 999999} & Begrenzt die Breite einer Spalte des Typs \identifier{NUMBER}. Jede \oscommand{9} steht f\"ur eine Stelle, d. h. \oscommand{999} erzeugt eine dreistellige Spalte. \\
              \hline
              \oscommand{set linesize NN} & \oscommand{set linesize 300} & Zeilenl\"ange auf \oscommand{NN} Zeichen begrenzen. \\
              \hline
              \oscommand{set long NN} & \oscommand{set long 4000} & Begrenzt die Breite einer Spalte des Typs \identifier{LONG} auf \oscommand{NN} Zeichen. \\
              \hline
              \oscommand{set pagesize NN} & \oscommand{set pagesize 50} & Seitenh\"ohe auf \oscommand{NN} Zeilen begrenzen. Die H\"ohe gibt an, nach wie vielen Zeilen die Spalten\"uberschriften wiederholt werden. \\
              \hline
              \oscommand{set serveroutput on\textbar{}off} & \oscommand{set serveroutput on} & F\"ur PL/SQL-Bl\"ocke wird die Bild\-schirm\-ausgabe ein- oder ausgeschaltet.\\
              \hline
              \oscommand{startup} & startup & Startet eine Oracle-Instanz. \\
              \hline
              \oscommand{shutdown} & shutdown & Schlie\ss{}t eine Oracle-Instanz. \\
              \hline
              \oscommand{[l]ist} & l & Letztes SQL-Kommando anzeigen. \\
              \hline
              \oscommand{[r]erun} & r & Letztes SQL-Kommando wiederholen. \\
           \end{supertabular}
          \end{small}
        \end{center}
    \section{Der Start-Up-Prozess}
      Unter dem Begriff Start-Up versteht Oracle das \enquote{Hochfahren der Datenbank}. Da die Datenbank jedoch nur eine Sammlung von Dateien ist, ist somit die Ausdrucksweise \enquote{Hochfahren der Datenbank} inkorrekt. Richtiger Weise muss es hei\ss{}en \enquote{Erstellen der Instanz}, da beim Start-Up eine Instanz erstellt und mit ihrer Datenbank verbunden wird.

      Der Start-Up-Prozess verl\"auft in drei Schritten, die Start-Up-Phasen genannt werden. Jede Phase hat einen eigenen Namen, einen bestimmten Zweck und ist f\"ur unterschiedliche, meist administrative T\"atigkeiten notwendig. Die Namen der Start-Up-Phasen lauten:
      \begin{itemize}
        \item NOMOUNT
        \item MOUNT
        \item OPEN
      \end{itemize}
      \subsection{Die NOMOUNT-Phase}
        Nach dem ersten Schritt des Start-Ups befindet sich die Datenbank in der NOMOUNT-Phase. Das bedeutet, dass die Instanz erstellt, aber noch nicht an die Datenbank angeschlossen wurde. Es existiert eine Instanz ohne Datenbank.
        \begin{merke}
          Um den Start-Up einer Instanz durchf\"uhren zu k\"onnen, m\"ussen die beiden Umgebungsvariablen \oscommand{ORACLE\_SID} und \oscommand{ORACLE\_HOME} gesetzt sein. Dies geschieht beim Ausf\"uhren des Shell-Skriptes: \oscommand{. oraenv}.
        \end{merke}
        Das Starten der Instanz geschieht mit dem SQL*Plus-Befehl \languagesqlplus{startup}. Diesem Kommando k\"onnen die Zus\"atze \languagesqlplus{nomount}, \languagesqlplus{mount} oder \languagesqlplus{open} mitgegeben werden, um die jeweils gew\"unschte Start-Up-Phase zu erreichen.
        \begin{lstlisting}[caption={Einen Start-Up bis zur NOMOUNT-Phase
        durchf\"uhren},label=admin01,language=sqlplus]
[oracle@FEA11-119SRV ~]$ sqlplus / as sysdba

&SQL&*Plus: Release 11.2.0.1.0 Production &on& Tue Aug 27 10:20:29 2013

Copyright (c) 1982, 2009, Oracle.  All rights reserved.

Connected to an idle instance.

SQL> startup nomount
ORACLE instance started.

Total System Global Area  643084288 bytes
Fixed Size                  2215984 bytes
Variable Size             222302160 bytes
Database Buffers          411041792 bytes
Redo Buffers                7524352 bytes
        \end{lstlisting}

        \bild{Die NOMOUNT-Phase}{startup_nomount}{1.75}

        Nach Erreichen der NOMOUNT-Phase sind drei Dinge geschehen:
        \begin{enumerate}
          \item Die Parameter-/Serverparameterdatei wurde gelesen
          \item Die SGA wurde mit den gelesenen Parametern erstellt
          \item Die Oracle-Hintergrundprozesse wurden gestartet
        \end{enumerate}
      \subsection{Die MOUNT-Phase}
        Der Begriff \enquote{mounten} bedeutet, dass eine Instanz mit ihrer Datenbank verbunden wird. Dieser Vorgang wird dadurch realisiert, dass die Instanz die Kontrolldatei der Datenbank liest, um ihr die Speicherorte der Daten- und der Redo Log Dateien zu entnehmen. Der Pfad zu den Kontrolldateien ist in der Parameterdatei, im Parameter \parameter{control\_files} festgelegt.
        \bild{Die MOUNT-Phase}{startup_mount}{1.8}

        In dieser Phase ist die Instanz exklusiv ge\"offnet, d. h. nur Administratoren haben Zugriff, normale Nutzer noch nicht. Die MOUNT-Phase ist f\"ur administrative T\"agtigkeiten, wie z. B. Recovery nach einem Datenverlust oder das Verschieben von Datenbankdateien vorgesehen.

        Wie eine Instanz die MOUNT-Phase erreichen kann, h\"angt davon ab, ob sie geschlossen ist oder ob sie sich in der NOMOUNT-Phase befindet. F\"ur eine geschlossene Datenbank wird mit Hilfe des Kommandos \languageorasql{startup mount} die Instanz bis in die MOUNT-Phase gebracht. Existiert die Instanz aber bereits in der NOMOUNT-Phase, muss sie mittels des Befehls \languageorasql{ALTER DATABASE MOUNT} in die MOUNT-Phase versetzt werden.
        \begin{merke}
          Eine Instanz die bereits mit \languageorasql{startup} gestartet wurde, kann nicht nochmal gestartet werden. Ihr Status muss stattdessen mit \languageorasql{ALTER DATABASE} ge\"andert werden.
        \end{merke}
        \beispiel{admin02} zeigt wie eine Instanz reagiert, wenn sie nach einen Start-Up erneut gestartet werden soll.
\clearpage
        \begin{lstlisting}[caption={ORACLE l\"auft noch. Erst
        stoppen.},label=admin02,language=sqlplus]
[oracle@FEA11-119SRV ~]$ sqlplus / as sysdba

&SQL&*Plus: Release 11.2.0.1.0 Production &on& Tue Aug 27 10:23:58 2013

Copyright (c) 1982, 2009, Oracle.  All rights reserved.


Connected to:
Oracle Database 11g Enterprise Edition Release 11.2.0.1.0 - 64bit Production
With the Partitioning, OLAP, Data Mining and Real Application Testing options

SQL> startup mount
&\textbf{\textcolor{red}{ORA-01081: cannot start already-running ORACLE - shut it down first}}&
        \end{lstlisting}
        In \beispiel{admin01} ist zu sehen, dass das \languagesqlplus{startup nomount}-Kommando, die Instanz startet und in die NOMOUNT-Phase versetzt. Das zweite Kommando, \languagesqlplus{startup mount} quittiert Oracle mit der Fehlermeldung: \enquote{\oscommand{ORA-01081: cannot start already-running ORACLE - shut it down first}}. Dies geschieht, da die Instanz bereits gestartet worden war.
        \begin{merke}
          Eine Instanz muss erst heruntergefahren werden, bevor sie erneut gestartet werden kann.
        \end{merke}
        Im folgenden Beispiel wird die Instanz korrekt von der NOMOUNT-Phase in die MOUNT-Phase gehoben.
        \begin{lstlisting}[caption={Das Kommando ALTER DATABASE MOUNT},label=admin03,,language=oracle_sql]
[oracle@FEA11-119SRV ~]$ sqlplus / as sysdba

&SQL&*Plus: Release 11.2.0.1.0 Production &on& Tue Aug 27 10:26:35 2013

Copyright (c) 1982, 2009, Oracle.  All rights reserved.


Connected to:
Oracle Database 11g Enterprise Edition Release 11.2.0.1.0 - 64bit Production
With the Partitioning, OLAP, Data Mining and Real Application Testing options

SQL> ALTER DATABASE MOUNT;

Database altered.
        \end{lstlisting}
    \subsection{Die OPEN-Phase}
      Eine Oracle-Datenbank zu \"offnen bedeutet, die Daten- und Redo Log Dateien zu \"offnen und sie den \enquote{normalen Nutzern} zug\"anglich zu machen. Das \"Offnen einer geschlossenen Datenbank geschieht mit dem Kommando \languagesqlplus{startup} oder wahlweise auch mit \languagesqlplus{startup open}.

      \bild{Die OPEN-Phase}{startup_open}{1.75}

      \begin{lstlisting}[caption={Starten der Instanz und \"offnen der
      Datenbank},label=admin04,language=sqlplus]
[oracle@FEA11-119SRV ~]$ sqlplus / as sysdba

&SQL&*Plus: Release 11.2.0.1.0 Production &on& Tue Aug 27 10:36:12 2013

Copyright (c) 1982, 2009, Oracle.  All rights reserved.

Connected to an idle instance.

SQL> startup open
ORACLE instance started.

Total System Global Area  643084288 bytes
Fixed Size                  2215984 bytes
Variable Size             222302160 bytes
Database Buffers          411041792 bytes
Redo Buffers                7524352 bytes
Database mounted.
Database opened.
        \end{lstlisting}
        F\"ur den Wechsel zwischen MOUNT- und OPEN-Phase gilt das Gleiche, wie f\"ur den Wechsel zwischen NOMOUNT- und MOUNT-Phase, er muss mittels \languageorasql{ALTER DATABASE} erfolgen, da die Instanz bereits gestartet ist.
        \begin{lstlisting}[caption={Wechsel zwischen MOUNT- und OPEN-Phase},label=admin05,language=oracle_sql]
ORACLE instance started.

Total System Global Area  643084288 bytes
Fixed Size                  2215984 bytes
Variable Size             222302160 bytes
Database Buffers          411041792 bytes
Redo Buffers                7524352 bytes
Database mounted.
SQL> ALTER DATABASE OPEN;
Database altered.
        \end{lstlisting}
      \subsection{Das Hochfahren einer Instanz erzwingen}
        In einigen F\"allen kommt es vor, dass eine Instanz nicht auf normalem Wege gestartet werden kann. Sollte dies geschehen, muss \enquote{der Instanzstart erzwungen werden}. Sinnvoll ist ein solches Vorgehen aber nur in den folgenden Situationen:
        \begin{itemize}
          \item Wenn die Instanz nicht gestartet werden kann
          \item Wenn ein Herunterfahren der Instanz nicht m\"oglich ist
        \end{itemize}
        Mit dem Kommando \languagesqlplus{startup force} wird eine Instanz gezwungen, neu zu starten. Durch einen solchen \enquote{gewaltsamen} Neustart entsteht jedoch Datenverlust, weil die Instanz ohne vorherigen Checkpoint von ihrer Datenbank getrennt wird. Alle nicht abgeschlossenen Transaktionen der Nutzer und s\"amtliche Sessions werden abgebrochen.

        Zur Behebung des Datenverlusts, wird beim Neustart ein
        \enquote{Instance-} bzw. \enquote{Crash recovery} durchgef\"uhrt. Dies
        wird durch einen Eintrag im Alert-Log File belegt.
        \begin{lstlisting}[caption={Ein erzwungener Neustart der Instanz und
        seine Folgen - Der Eintrag im Alert Log
        File},label=admin07,language=terminal]
Mon Aug 26 08:59:27 2013 &ALTER DATABASE OPEN& 
&\textbf{\textcolor{red}{Beginning crash recovery of 1 threads}}&
 parallel recovery started with 2 processes
Started redo scan
Completed redo scan
 read 119 KB redo, 82 data blocks need recovery
Started redo application at
 Thread 1: logseq 4, block 277
Recovery of Online Redo Log: Thread 1 Group 1 Seq 4 Reading mem 0
  Mem# 0: /u02/oradata/orcl/redo01.log
Completed redo application of 0.11MB
&\textbf{\textcolor{red}{Completed crash recovery}}& at
 Thread 1: logseq 4, block 515, scn 1037873
 82 data blocks read, 82 data blocks written, 119 redo k-bytes read
        \end{lstlisting}
    \section{Der Shutdown-Vorgang}
      Als \enquote{Shutdown} wird der Vorgang des Herunterfahrens der Instanz bezeichnet. Welche Einzelschritte bei einem Shutdown geschehen, h\"angt davon ab, welche der vier Arten des Shutdowns gew\"ahlt wurde. Folgende Shutdown-Arten gibt es:
      \begin{itemize}
        \item Shutdown Normal
        \item Shutdown Transactional
        \item Shutdown Immediate
        \item Shutdown Abort
      \end{itemize}
      Ein Shutdown kann mit Hilfe eines Datenbank-Tools eingeleitet werden, z. B. SQL*Plus.
      \begin{merke}
        Nur Nutzer mit dem Privileg \privileg{SYSDBA} oder \privileg{SYSOPER} k\"onnen einen Shutdown durchf\"uhren.
      \end{merke}
      \subsection{Shutdown NORMAL}
        Der Shutdown NORMAL ist die gr\"undlichste Art des Shutdown. Er f\"uhrt folgende Einzelschritte durch:
        \begin{enumerate}
          \item Es werden keine neuen Connections zur Datenbank zugelassen
          \item Es wird gewartet, bis alle Transaktionen der Nutzer abgeschlossen sind
          \item Es wird gewartet, bis alle Nutzer ihre Session beendet haben
          \item Ein Checkpoint wird ausgel\"ost
        \end{enumerate}
        Durch das Ausl\"osen des Checkpoints werden alle ge\"anderten Daten aus dem Database Buffer Cache in die Datendateien zur\"uckgeschrieben. Daraus folgt, dass ein Shutdown NORMAL die Datenbank in einem konsistenten Zustand bel\"asst.

        Durchgef\"uhrt wird ein Shutdown NORMAL mit dem Kommando
       \languagesqlplus{shutdown normal} oder einfach nur \languagesqlplus{shutdown}.
        \begin{lstlisting}[caption={Durchf\"uhren eines Shutdown
        NORMAL},label=admin08,language=sqlplus]
SQL> shutdown normal
Database closed.
Database dismounted.
ORACLE instance shut down.
        \end{lstlisting}
         \begin{merke}
           Problematisch an einem Shutdown NORMAL ist, dass in einer Umgebung mit sehr vielen Nutzern, vermutlich nie alle Nutzer ihre Session beenden, was bedeutet, dass der Shutdown-Vorgang auch nie vollendet werden kann.
         \end{merke}
      \subsection{Shutdown TRANSACTIONAL}
        Der Shutdown TRANSACTIONAL ist eine Abstufung des Shutdown NORMAL. Er f\"uhrt folgende Einzelschritte durch:
        \begin{enumerate}
          \item Es werden keine neuen Connections zur Datenbank zugelassen
          \item Es wird gewartet, bis alle Transaktionen der Nutzer abgeschlossen sind
          \item Noch aktive Nutzersessions werden automatisch beendet
          \item Ein Checkpoint wird ausgel\"ost
        \end{enumerate}
        Hier wird also lediglich auf das Ende aller noch offnen Transaktionen gewartet, nicht aber darauf, dass sich alle Nutzer vom System abmelden. Dies kann einen Shutdown-Vorgang deutlich beschleunigen, bzw. es wird dadurch eine realistische Chance f\"ur die Vollendung des Shutdowns einger\"aumt.
        \begin{merke}
          Bei einem Shutdown TRANSACTIONAL ist es sinnvoll die Nutzer vorher zubenachrichtigen, da diese ab einem Zeitpunkt X, ihre Arbeit nicht mehr fortsetzen k\"onnen.
        \end{merke}
        \begin{lstlisting}[caption={Durchf\"uhren eines Shutdown
        TRANSACTIONAL},label=admin09,language=sqlplus]
[oracle@FEA11-119SRV ~]$ sqlplus / as sysdba

&SQL&*Plus: Release 11.2.0.1.0 Production &on& Tue Aug 27 10:50:05 2013

Copyright (c) 1982, 2009, Oracle.  All rights reserved.


Connected to:
Oracle Database 11g Enterprise Edition Release 11.2.0.1.0 - 64bit Production
With the Partitioning, OLAP, Data Mining and Real Application Testing options

SQL> shutdown transactional
Database closed.
Database dismounted.
ORACLE instance shut down.
        \end{lstlisting}
        Der Unterschied zwischen einem Shutdown NORMAL und einem Shutdown TRANSACTIONAL ist im Alert Log File zu sehen.
        \begin{lstlisting}[caption={Der Shutdown TRANSACTIONAL im Alert
        Log},label=admin10,language=terminal]
&\textbf{\textcolor{red}{Shutting down instance (transactional)}}&
Shutting down instance: further logons disabled
Stopping background process CJQ0
Stopping background process QMNC
Stopping background process MMNL
Stopping background process MMON
&\textbf{\textcolor{red}{All transactions complete. Performing immediate shutdown}}&
License high water mark = 3
All dispatchers and shared servers &shutdown&
&ALTER DATABASE CLOSE NORMAL&
        \end{lstlisting}
        Der Eintrag \enquote{All transactions complete. Performing immediate shutdown} weisst darauf hin, dass nach dem Ende der letzten Transaktion sofort mit dem Shutdown begonnen wird. Bei einem Shutdown NORMAL fehlt diese Zeile, wie in \beispiel{admin11} zu sehen ist.
        \begin{lstlisting}[caption={Der Shutdown NORMAL im Alert Log},label=admin11,language=terminal]
&\textbf{\textcolor{red}{Shutting down instance (normal)}}&
Stopping background process SMCO
Shutting down instance: further logons disabled
Mon Aug 26 10:41:26 2013
Stopping background process CJQ0
Stopping background process QMNC
Stopping background process MMNL
Stopping background process MMON
License high water mark = 4
All dispatchers and shared servers &shutdown&
&ALTER DATABASE CLOSE NORMAL&
       \end{lstlisting}
       \begin{merke}
        Ein Shutdown TRANSACTIONAL \"uberf\"uhrt die Datenbank in einen konsistenten Zustand.
       \end{merke}
      \subsection{Shutdown IMMEDIATE}
        Der Shutdown IMMEDIATE ist genau das, was sein Name besagt. Die Datenbank wird sofort, ohne auf offene Transaktionen oder Nutzer zu warten heruntergefahren. Da diese Art des Shutdowns sehr radikal ist, sollten auf jeden Fall alle Nutzer informiert werden.

        Nur in den folgenden Situationen sollte ein Shutdown IMMEDIATE durchgef\"uhrt werden:
        \begin{itemize}
          \item Vor einem automatisierten Backup
          \item Um im Falle eines Stromausfalles die DB so schnell wie m\"oglich herunterzufahren
          \item Im Falle einer Datenbankfehlfunktion, wenn es nicht m\"oglich ist, alle Nutzer vorher zu benachrichtigen
        \end{itemize}
        Bei diesem Shutdown werden folgende Einzelschritte durchgef\"uhrt:
        \begin{itemize}
          \item Es werden keine neuen Connections zur Datenbank zugelassen
          \item Alle aktiven Transaktionen werden zur\"uckgerollt
          \item Noch aktive Nutzersessions werden automatisch beendet
          \item Es wird ein Checkpoint gesetzt
        \end{itemize}
        \begin{merke}
          Obwohl alle Transaktionen abgebrochen und alle Sessions geschlossen werden, wird die Datenbank in einem konsistenten Zustand hinterlassen.
        \end{merke}
        \begin{lstlisting}[caption={Durchf\"uhren eines Shutdown IMMEDIATE},label=admin12,language=sqlplus]
[oracle@FEA11-119SRV ~]$ sqlplus / as sysdba

&SQL&*Plus: Release 11.2.0.1.0 Production &on& Tue Aug 27 10:52:09 2013

Copyright (c) 1982, 2009, Oracle.  All rights reserved.

Connected to:
Oracle Database 11g Enterprise Edition Release 11.2.0.1.0 - 64bit Production
With the Partitioning, OLAP, Data Mining and Real Application Testing options

SQL> shutdown immediate
Database closed.
Database dismounted.
ORACLE instance shut down.
        \end{lstlisting}
      \subsection{Shutdown ABORT}
        Der Shutdown ABORT ist die einzige Shutdown-Variante, bei der die
        Datenbank in einem inkonsistenten Zustand hinterlassen wird. Hier wird
        die Instanz einfach von der Datenbank getrennt, ohne dass vorher ein
        Checkpoint ausgel\"ost wird. Aus diesem Grund sollte ein Shutdown ABORT
        nur dann eingesetzt werden, wenn es notwendig ist.

        Folgende Einzelschritte werden bei einem Shutdown ABORT ausgef\"uhrt:
        \begin{itemize}
          \item Es werden keine neuen Connections zur Datenbank zugelassen
          \item Alle aktiven Transaktionen werden nicht zur\"uckgerollt, sondern sofort abgebrochen (Inkonsistenz!)
          \item Noch aktive Nutzersessions werden abgebrochen
        \end{itemize}
        \begin{lstlisting}[caption={Durchf\"uhren eines Shutdown ABORT},label=admin13,language=sqlplus]
[oracle@FEA11-119SRV ~]$ sqlplus / as sysdba

&SQL&*Plus: Release 11.2.0.1.0 Production &on& Tue Aug 27 10:52:09 2013

Copyright (c) 1982, 2009, Oracle.  All rights reserved.

Connected to:
Oracle Database 11g Enterprise Edition Release 11.2.0.1.0 - 64bit Production
With the Partitioning, OLAP, Data Mining and Real Application Testing options

SQL> shutdown abort
ORACLE instance shut down.
        \end{lstlisting}
        \begin{merke}
          Ein Neustart der Datenbank nach einem \languagesqlplus{shutdown abort} erfordert ein Instance recovery.
        \end{merke}
        \begin{literaturinternet}
          \item \cite{i1006091}
        \end{literaturinternet}
    \section{Verwalten der Parameterdatei}
      Die Parameterdatei bzw. Serverparameterdatei ist die erste Datei, die ben\"otigt wird, um eine Instanz hochfahren zu k\"onnen. Sie enth\"alt eine Liste mit Konfigurationsparametern, den sogenannten \enquote{Init\-iali\-sierungs\-pa\-ra\-me\-tern}.

      Oracle sucht seine Parameterdatei/Serverparameterdatei beim Hochfahren der Instanz in einem betriebssystemabh\"angigen Standardverzeichnis. Dieses ist:
      \begin{itemize}
        \item \textbf{Windows}: \oscommand{\%ORACLE\_HOME\%\textbackslash database}
        \item \textbf{UNIX}: \oscommand{\$ORACLE\_HOME/dbs}
      \end{itemize}
      Beim Durchsuchen des entsprechenden Standardpfades wird nach drei verschiedenen Dateinamen gesucht:
      \begin{itemize}
        \item \oscommand{spfile\$ORACLE\_SID.ora} (z. B. \oscommand{spfileorcl.ora})
        \item \oscommand{spfile.ora}
        \item \oscommand{init\$ORACLE\_SID.ora} (z. B. \oscommand{initorcl.ora})
      \end{itemize}
      Bei den ersten beiden Dateinamen handelt es sich um Serverparameterdateien, beim Dritten um eine Parameterdatei.
      \begin{merke}
        Kann Oracle keine dieser Dateien finden, wird der Startvorgang
        abgebrochen.
      \end{merke}

      \begin{literaturinternet}
        \item \cite{i1124822}
      \end{literaturinternet}
      \subsection{Initialisierungsparameter administrieren}
        Initialisierungsparameter haben unterschiedliche Aufgaben:
        \begin{itemize}
          \item Benennen von Objekten wie z. B. Kontrolldateien oder Betriebssystemverzeichnissen
          \item Beeinflussen von Kapazit\"aten, wie z. B. der Gr\"o\ss e der SGA
        \end{itemize}
        Es kann aus unterschiedlichen Gr\"unden notwendig sein, die Werte von Initialisierungsparametern zu ver\"andern. Welche Auswirkungen diese \"Anderungen haben, h\"angt von den Charakteristiken der Datenbank und anderen Faktoren ab.
        \subsubsection{Statische und dynamische Parameter}
          Initialisierungsparameter werden in zwei Gruppen eingeteilt:
          \begin{itemize}
            \item \textbf{Statische Parameter}: Die \"Anderung an einem solchen Parameter wird erst nach einem Instanzneustart wirksam.
            \item \textbf{Dynamische Parameter}: \"Anderungen an dynamischen Parametern werden sofort wirksam.
          \end{itemize}
          Die View \identifier{v\$system\_parameter} gibt Aufschluss dar\"uber, ob ein Parameter statisch oder dynamisch ist.
          \begin{lstlisting}[caption={Unterscheiden zwischen dynamischen und statischen Parametern},label=admin14,language=oracle_sql,alsolanguage=sqlplus]
SQL> col name format a30

SQL> SELECT name, issys_modifiable
  2  FROM   v$system_parameter;
          \end{lstlisting}
          In der Spalte \identifier{issys\_modifiable} k\"onnen drei verschiedene Werte vorkommen:
          \begin{itemize}
            \item \textbf{IMMEDIATE}: Es handelt sich um einen \textbf{dynamischen} Parameter.
            \item \textbf{FALSE}: Der Parameter ist \textbf{statisch}.
            \item \textbf{DEFERRED}: \"Anderungen an einem so markierten Parameter haben nur auf neue Sessions, die nach der \"Anderung erstellt wurden eine Auswirkung.
          \end{itemize}
        \subsubsection{Initialisierungsparameter \"andern}
          Initialisierungsparameter k\"onnen per \languageorasql{ALTER SYSTEM}-SQL-Kommando oder im Enterprise Manager ge\"andert werden. \beispiel{admin15} zeigt wie der Parameter \identifier{license\_max\_sessions} von 0 auf 50 ge\"andert wird.
          \begin{merke}
            Der Parameter \identifier{license\_max\_sessions} legt fest, wie viele gleichzeitige Verbindungen zur Datenbank m\"oglich sind. Ist der Schwellenwert von \identifier{license\_max\_sessions} erreicht, k\"onnen sich nur noch Nutzer an der Datenbank anmelden, die das Privileg \privileg{restricted\_session} haben.
          \end{merke}
          \begin{lstlisting}[caption={\parameter{license\_max\_sessions} wird ge\"andert},label=admin15,language=oracle_sql,alsolanguage=sqlplus]
SQL> show parameter license_max_sessions

NAME                                 &TYPE&        VALUE
------------------------------------ ----------- ------------------------------
license_max_sessions                 integer     0

SQL> ALTER SYSTEM
  2  SET license_max_sessions = 50 SCOPE=both;

System altered.
          \end{lstlisting}
          Das \languageorasql{ALTER SYSTEM}-Kommando hat in \beispiel{admin15} zwei Klauseln:
          \begin{itemize}
            \item \languageorasql{SET <parameter> = wert}: Gibt den zu ver\"andernden Parameter und den neuen Wert an. Welche Werte zul\"assig sind, h\"angt vom jeweiligen Parameter ab.
            \item \languageorasql{SCOPE = <scope>}: Mit der \languageorasql{SCOPE}-Klausel wird geregelt, wo die \"Anderung vollzogen wird.
          \end{itemize}
          Die \languageorasql{SCOPE}-Klausel kennt drei Werte f\"ur \languageorasql{<scope>}:
          \begin{itemize}
            \item \languageorasql{SCOPE=both}: Die \"Anderung erfolgt in der SGA und im SPFILE.
            \item \languageorasql{SCOPE=memory}: Die \"Anderung erfolgt nur in der SGA.
            \item \languageorasql{SCOPE=spfile}: Die \"Anderung  erfolgt nur im SPFile. Damit der neue Wert wirksam wird, muss die Instanz neu gestartet werden.
          \end{itemize}
          \begin{merke}
            \languageorasql{SCOPE=both} ist die Standardeinstellung und kann deshalb entfallen.
          \end{merke}

          \begin{literaturinternet}
            \item \cite{i2053602}
          \end{literaturinternet}
      \subsection{Sessionparameter}
        Im Gegensatz zu Initialisierungsparametern, die systemweite G\"ultigkeit haben, sind Sessionparameter nur innerhalb der Session eines Nutzers g\"ultig. Beispielsweise kann der Sessionparameter \parameter{nls\_language} von jedem Nutzer, der das \privileg{ALTER SESSION} besitzt, ge\"andert werden. D. h. w\"ahrend Nutzer A mit deutschen Spracheinstellungen arbeitet, kann Nutzer B zur gleichen Zeit in englischer Sprache arbeiten.
        \subsubsection{Sessionparameter \"andern}
          \"Anderungen an Sessionparametern werden mit dem SQL-Kommando \languageorasql{ALTER SESSION} durchgef\"uhrt, dessen Syntax dem \languageorasql{ALTER SYSTEM}-Kommando sehr \"ahnlich ist.

          Ge\"andert werden k\"onnen einige dynamische Initialisierungsparameter, sowie alle Sessionparameter. Welche Initialisierungsparameter betroffen sind, kann wiederum mit Hilfe der View \identifier{v\$system\_parameter} und der Spalte \identifier{isses\_modifiable} ermittelt werden. Eine Liste der Sessionparameter kann aus der Oracle-Onlinedokumentation entnommen werden.
          \begin{lstlisting}[caption={Sessionmodifiable
          Initialisierungsparameter},label=admin16,language=oracle_sql,alsolanguage=sqlplus]
 SQL> col name format a30

SQL> SELECT name, isses_modifiable
  2  FROM   v$system_parameter;
          \end{lstlisting}
          \begin{literaturinternet}
            \item \cite{autoId0}
            \item \cite{sthref3228}
          \end{literaturinternet}
          \begin{lstlisting}[caption={Beispiel f\"ur ALTER SESSION},label=admin17,language=oracle_sql,alsolanguage=sqlplus]
SQL>show parameter nls_language

NAME                                 &TYPE&        VALUE
------------------------------------ ----------- ------------------------------
nls_language                         string      GERMAN

SQL>ALTER SESSION
  2 SET nls_language = 'AMERICAN';

Session altered.
          \end{lstlisting}
\clearpage
          Das Ergebnis des \languageorasql{ALTER SESSION}-Statements aus \beispiel{admin17} kann nicht mit Hilfe des SQL*Plus-Kommandos \languagesqlplus{show parameter} gepr\"uft werden, da dieses nur Systemparameter anzeigt, aber keine Sessionparameter. Die Werte aller NLS-Sessionparameter k\"onnen mit Hilfe der View \identifier{v\$nls\_parameters} angezeigt werden.
          \begin{lstlisting}[caption={Sessionparameter mit Hilfe von \identifier{v\$nls\_parameters} ermitteln},label=admin17a,language=oracle_sql,alsolanguage=sqlplus]
SQL> col parameter format a30
SQL> col value format a30

SQL> SELECT *
  2  FROM   v$nls_parameters
  3  WHERE  parameter LIKE 'NLS_LANGUAGE';

PARAMETER                         VALUE
--------------------------------- ---------------------------------
NLS_LANGUAGE                      AMERICAN
          \end{lstlisting}
      \subsection{SPFiles/PFiles generieren}
        Oracle bietet verschiedene M\"oglichkeiten, um ein PFile bzw. ein SPFile zu erzeugen:
        \begin{itemize}
          \item SPFile aus einem PFile
          \item SPFile aus den aktuellen Initialisierungsparametern der Instanz
          \item PFile aus einem SPFile
          \item PFile aus den aktuellen Initialisierungsparametern der Instanz
        \end{itemize}
        \subsubsection{Ein SPFile generieren}
          Da ein SPFile eine Bin\"ardatei ist, kann es nicht von Hand erstellt werden. Oracle stellt das Kommando \languageorasql{CREATE SPFILE} zur Verf\"ugung, um ein SPFile aus einem PFile  zu generieren.

          \beispiel{admin18} zeigt die Syntax des \languageorasql{CREATE SPFILE}-Kommandos. Angaben in eckigen Klammern sind optional.
          \begin{lstlisting}[caption={\languageorasql{CREATE SPFILE}},label=admin18,language=oracle_sql]
CREATE SPFILE = [pfad/dateiname.ora]
FROM   PFILE  = [pfad/dateiname.ora];
          \end{lstlisting}
          Die \languageorasql{SPFILE}-Klausel kann wahlweise den Dateinamen der Zieldatei annehmen. Wird kein Dateinamen angegeben, wird die Datei \oscommand{ORACLE\_HOME/dbs/spfile<SID>.ora} angelegt. Dies funktioniert jedoch nur, wenn die Instanz heruntergefahren ist, da Oracle sonst den Speicherort der Datei \oscommand{ORACLE\_HOME/dbs/spfile<SID>.ora} sch\"utzt, so dass diese nicht \"uberschrieben werden kann.
          \begin{merke}
            Ist die Instanz gestartet, muss f\"ur die \languageorasql{SPFILE}-Klausel ein Dateiname angegeben werden, da Oracle sonst die Fehlermeldung \enquote{\oscommand{ORA-32002: cannot create SPFILE already being used by the instance}} anzeigt.
          \end{merke}

          Dient ein PFile als Quelle kann  hier optional der Name der Quelldatei angegeben werden. Ohne Dateiname wird die Datei \oscommand{ORACLE\_HOME/dbs/init<SID>.ora} als Quelle genutzt. Ist diese Datei nicht vorhanden, antwortet Oracle mit der Fehlermeldung:

          \oscommand{ORA-01078: failure in processing system parameters}\\
          \oscommand{LRM-00109: could not open parameter file}\\
          \oscommand{'/u01/app/oracle/product/11.2.0/ORCL/dbs/initorcl.ora'}

          Durch die Angabe von MEMORY, statt PFILE, kann ein SPFile mit den Werten der aktuellen Initialisierungsparameter erzeugt werden.\beispiel{admin19} zeigt verschiedene Varianten des \languageorasql{CREATE SPFILE}-Kommandos.
          \begin{lstlisting}[caption={Beispiele f\"ur \languageorasql{CREATE SPFILE}},label=admin19,language=oracle_sql]
SQL> CREATE SPFILE
  2  FROM   PFILE;

File created.

SQL> CREATE SPFILE = '/home/oracle/spfileorcl.ora'
  2  FROM   PFILE;

File created.

SQL> CREATE SPFILE = '/home/oracle/spfileorcl.ora'
  2  FROM  PFILE   = '?/dbs/initorcl.ora'

File created.

SQL> CREATE SPFILE = '/home/oracle/spfileorcl.ora'
  2  FROM  MEMORY;

File created.
          \end{lstlisting}
          \begin{merke}
            Das \oscommand{?} in einer Pfadangabe dient als Synonym f\"ur die Umgebungsvariable \oscommand{ORACLE\_HOME}.
          \end{merke}

          \begin{merke}
            Wird kein Quell-SPFile angegeben, wird der Wert des Initialisierungsparameters \parameter{spfile} benutzt, um das aktuelle SPFile zu ermitteln. Wird kein Dateiname f\"ur das PFile angegeben, wird im Verzeichnis \oscommand{ORACLE\_HOME/dbs} die Datei \oscommand{init<SID>.ora} angelegt.
          \end{merke}
        \subsubsection{Ein PFile generieren}
          Analog zum Kommando \languageorasql{CREATE SPFILE} existiert das \languageorasql{CREATE PFILE}-Statement.
          \begin{lstlisting}[caption={Beispiele f\"ur \languageorasql{CREATE PFILE}},label=admin20,language=oracle_sql]
SQL> CREATE PFILE
  2  FROM   SPFILE;

SQL> CREATE PFILE = '/home/oracle/initorcl.ora'
  2  FROM   SPFILE;

SQL> CREATE PFILE = '/home/oracle/initorcl.ora'
  2  FROM  SPFILE   = '?/dbs/initorcl.ora'

SQL> CREATE PFILE = '/home/oracle/initorcl.ora'
  2  FROM  MEMORY;
          \end{lstlisting}
      \subsection{Hochfahren einer Instanz mit alternativer Parameterdatei}
        Soll beim Hochfahren der Instanz eine alternative Parameterdatei genutzt werden.          Das Schl\"usselwort \languagesqlplus{pfile} dient dazu, den Namen der alternativen Parameterdatei anzugeben. Es kann nur eine Parameterdatei verarbeiten, keine Serverparameterdatei!
        \begin{lstlisting}[caption={Start mit alternativer Parameterdatei},label=admin21,language=sqlplus]
SQL> startup pfile='/home/oracle/initorcl.ora';
ORACLE instance started.

Total System Global Area  643084288 bytes
Fixed Size                  2215984 bytes
Variable Size             222302160 bytes
Database Buffers          411041792 bytes
Redo Buffers                7524352 bytes
Database mounted.
Database opened.
        \end{lstlisting}
        \begin{literaturinternet}
          \item \cite{i1006091}
        \end{literaturinternet}

    \section{Memory Management}
      Wie in \abschnitt{memorymanagement} bereits beschrieben, kennt Oracle drei verschiedene Arten des Memory Managements: Manual Shared Memory Management, Automatic Shared Memory Management und seit Oracle 11g auch Automatic Memory Management. Die Konfiguration dieser Memory Management Modi funktioniert mit Hilfe von Initialisierungsparametern.
      \subsection{Manual Shared Memory Management}
        Beim Manual Shared Memory Management m\"ussen alle Komponenten der SGA bzw. der PGAs einzeln definiert werden. F\"ur die SGA-Komponenten existieren die folgenden Initialisierungsparameter:
        \begin{itemize}
          \item \parameter{db\_cache\_size}: Legt die Gr\"o\ss{}e des Database Buffer Caches fest.
          \item \parameter{shared\_pool\_size}: Dimensioniert den Shared Pool.
          \item \parameter{large\_pool\_size}: Gibt die Gr\"o\ss{}e des Large Pool an.
          \item \parameter{java\_pool\_size}: Definiert die Gr\"o\ss{}e des Java Pools.
          \item \parameter{streams\_pool\_size}: Legt die Gr\"o\ss{}e des Streams Pools fest.
          \item \parameter{log\_buffer}: Gibt die Gr\"o\ss{}e des Redo Log Buffers an.
        \end{itemize}
        \begin{merke}
          Auf einige Komponenten der SGA, wie z. B. den Large Pool oder den Streams Pool wird in dieser Unterlage nicht n\"aher eingegangen! Zudem existieren noch weitere Parameter, die \"uber den Horizont dieses Skriptes hinausgehen.
        \end{merke}
        Die aktuellen Werte dieser Parameter k\"onnen mit der View \identifier{v\$sgainfo} abgefragt werden.
        \begin{lstlisting}[caption={Gr\"o\ss{}e der SGA-Komponenten ermitteln},label=admin22,language=oracle_sql]
SQL> SELECT name, bytes
  2  FROM   v$sgainfo;

NAME                                  BYTES
-------------------------------- ----------
Fixed SGA Size                      2217264
Redo Buffers                        6926336
Buffer Cache Size                 260046848
Shared Pool Size                  121634816
Large Pool Size                     4194304
Java Pool Size                      4194304
Streams Pool Size                         0
Shared IO Pool Size                       0
Granule Size                        4194304
Maximum SGA Size                  764121088
Startup overhead in Shared Pool    71303168
Free SGA Memory Available         364904448
        \end{lstlisting}
      \subsection{Automatic Shared Memory Management (ASMM)}
        Mit dem Automatic Shared Memory Management kamen in Oracle 10g zwei neue Initialisierungsparameter: \parameter{sga\_target} und \parameter{pga\_aggregate\_target}.

        Um ASMM zu aktivieren, m\"ussen folgende Voraussetzungen gegeben sein:
        \begin{itemize}
          \item Der Parameter \parameter{sga\_target} muss einen Wert gr\"o\ss{}er 0 haben.
          \item Es muss ein SPFile benutzt werden.
          \item Der Parameter \parameter{statistics\_level} muss einen der beiden Werte \enquote{typical} oder \enquote{all} haben.
          \item Der Parameter \parameter{shared\_pool\_size} muss einen Wert gr\"o\ss{}er 0 haben.
        \end{itemize}
        Mit dem \parameter{sga\_target} wird die Speichergesamtmenge f\"ur alle Komponenten der SGA gesetzt. Er ist dynamisch und kann somit jederzeit ge\"andert werden. Seine maximale Gr\"o\ss{}e wird durch den statischen Parameter \parameter{sga\_max\_size} begrenzt (Hardlimit).

        Damit die Einstellungen f\"ur die SGA-Komponenten auch nach einem Neustart der Instanz erhalten bleiben, werden in der Parameterdatei zus\"atzliche Initialisierungsparameter gef\"uhrt: \parameter{\_\_db\_cache\_size}, \parameter{\_\_shared\_pool\_size}, \parameter{\_\_large\_pool\_size}, \parameter{\_\_java\_pool\_size} und \parameter{\_\_streams\_pool\_size}. Diese Parameter, die an dem doppelten Unterstrich vor ihrem Namen zu erkennen sind, dienen als Zwischenspeicher bis zum n\"achsten Instanzstart.

        Zus\"atzlich zu \parameter{sga\_target} kann der DBA Parameter, wie \parameter{db\_cache\_size}, \parameter{shared\_pool\_size} oder \parameter{large\_pool\_size} dazu benutzen, um Mindestgr\"o\ss{}en f\"ur diese Speicherbereiche zu setzen. Hierzu ein Beispiel:
        \begin{lstlisting}[caption={Ein Rechenbeispiel},label=admin23,language=oracle_sql]
SQL> ALTER SYSTEM
  2  SET sga_target = 600M;

System altered.

SQL> ALTER SYSTEM
  2  SET shared_pool_size = 50M;

System altered.

SQL> ALTER SYSTEM
  2  SET db_cache_size = 200M;
        \end{lstlisting}
        Wird eine Instanz, wie in \beispiel{admin23} konfiguriert, entfallen mindestens 50 Megabyte auf den Shared Pool, mindestens 200 Megabyte auf den Database Buffer Cache und 350 Megabyte k\"onnen auf alle anderen Komponenten der SGA verteilt werden.
\clearpage
        Zur Verwaltung der PGAs gibt es \parameter{pga\_aggregate\_target}. Dieser Parameter legt die gesamte Speichermenge fest, die f\"ur alle PGAs zur Verf\"ugung steht. Statt einer fixen PGA-Gr\"o\ss{}e, die der DBA selbst festlegen muss, kann die Datenbank nun selbst entscheiden, wie gro\ss{} jede einzelne PGA werden kann. Serverprozesse die eine gr\"o\ss{}ere PGA ben\"otigen k\"onnen somit mehr Speicherplatz erhalten, als andere.

        Problematisch an dieser Form der Speicherverwaltung ist, dass wenn ein zu kleiner Wert f\"ur \parameter{sga\_target} und ein zu gro\ss{}er f\"ur \parameter{pga\_aggregate\_target} gesetzt wird, die Datenbank keinen automatischen Ausgleich schaffen kann, falls die SGA mehr Memory ben\"otigt. Hier muss dann der DBA eingreifen und selbstst\"andig die Werte anpassen.
      \subsection{Automatic Memory Management}
        Das mit Oracle 11g neu eingef\"uhrte Automatic Memory Management vereinfacht die Situation des DBAs erneut. Es kommen zwei neue Parameter hinzu:
        \begin{itemize}
          \item \parameter{memory\_target}: Dieser dynamische Parameter definiert die Gr\"o\ss{}e aller Speicherkomponenten der Instanz (SGA + PGA).
          \item \parameter{memory\_max\_target}: Dies ist ein statischer Parameter, der als hartes Limit f\"ur \parameter{memory\_target} fungiert.
        \end{itemize}
        Der Vorteil dieser neuen Methode ist, dass die Datenbank automatisch einen Ausgleich zwischen SGA und PGA schaffen kann, falls eine der beiden Seiten zu wenig Speicher hat. F\"ur das Automatic Memory Management gelten folgende Regeln:
        \begin{center}
          \tablecaption{Regeln f\"ur das Automatic Memory Management}
          \label{rulesforautomaticmemorymanagement}
          \begin{small}
            \tablefirsthead{
              \multicolumn{1}{c}{\parameter{memory\_target}} &
              \multicolumn{1}{c}{\parameter{sga\_target}} &
              \multicolumn{1}{c}{\parameter{pga\_aggregate\_target}} &
              \multicolumn{1}{c}{\parameter{sga\_max\_size}} &
              \multicolumn{1}{c}{\textbf{Auswirkung}} \\
              \hline
            }
            \tablehead{
              \multicolumn{1}{c}{\parameter{memory\_target}} &
              \multicolumn{1}{c}{\parameter{sga\_target}} &
              \multicolumn{1}{c}{\parameter{pga\_aggregate\_target}} &
              \multicolumn{1}{c}{\parameter{sga\_max\_size}} \\
              \hline
            }
            \tabletail{
              \hline
            }
            \tablelasttail {
              \hline
            }
            \begin{supertabular}{|c|c|c|c|p{4.41cm}|}
              $x \neq 0$ & $x \neq 0$ & $x \neq 0$ & - & Die 2 Parameter \parameter{sga\_target} sowie \parameter{pga\_aggregate\_target} sind Min\-dest\-werte. \\
              \hline
              $x \neq 0$ & $x \neq 0$ & n. a. & - & \parameter{memory\_target} - \parameter{sga\_target} = \parameter{pga\_aggregate\_target}\\
              \hline
              $x \neq 0$ & n. a. & n. a. & - & \parameter{memory\_target} - Wert von  \parameter{pga\_aggregate\_target} ergibt \parameter{sga\_target} \\
              \hline
              $x \neq 0$ & 0 & 0 & 0 & \parameter{pga\_aggregate\_target}: 40 \% und \parameter{sga\_target}: 60 \% von \parameter{memory\_target}\\
            \end{supertabular}
          \end{small}
        \end{center}
\clearpage
        \begin{lstlisting}[caption={Auszug aus einer Parameterdatei},label=admin24,language=terminal]
orcl.__db_cache_size=260046848
orcl.__java_pool_size=4194304
orcl.__large_pool_size=4194304
orcl.__pga_aggregate_target=268435456
orcl.__sga_target=402653184
orcl.__shared_pool_size=121634816
orcl.__streams_pool_size=0
*.compatible='11.2.0.0.0'
*.db_block_size=8192
*.db_domain='local'
*.db_name='orcl'
*.memory_max_target=730M
*.memory_target=640M
        \end{lstlisting}
    \section{Verwaltung von Kontrolldateien}
      Eine Kontrolldatei ist eine Bin\"ardatei, welche die Struktur einer Oracle-Datenbank aufzeichnet. Jede Oracle-Datenbank ben\"otigt eine Kontrolldatei. Sie beinhaltet folgende Informationen:
      \begin{itemize}
        \item Globaler Datenbankname
        \item Dateinamen und Speicherorte der Daten- und Redo Log Dateien
        \item Zeitstempel der Datenbankerstellung
        \item Die aktuelle Log Sequence Number
        \item Checkpoint-Informationen
      \end{itemize}
      Um eine Datenbank \"offnen zu k\"onnen, muss der Schreibzugriff auf die Kontrolldateien m\"oglich sein. Generiert werden diese Dateien w\"ahrend der Datenbankerstellung. Standardm\"a\ss ig wird immer nur eine erstellt, der Administrator sollte jedoch daf\"ur sorgen, dass mehrere Sicherheitskopien der Kontrolldatei auf mehreren Speichermedien (Spiegelung) zur Verf\"ugung stehen.
      \subsection{Namensgebung f\"ur Kontrolldateien}
        Welche Kontrolldateien die Datenbank benutzt, wird durch den Initialisierungsparameter \parameter{control\_files} in der Serverparameterdatei festgelegt. Er kann eine Liste von Dateinamen enthalten, wie das folgende Beispiel zeigt.
\clearpage
				\begin{lstlisting}[caption={Der Parameter \parameter{control\_files}},label=admin25,language=terminal]
...
CONTROL_FILES='/u02/oradata/orcl/control01.ctl',
              '/u05/fast_recovery_area/orcl/control02.ctl'
...
        \end{lstlisting}
        \begin{merke}
          Es werden alle angegebenen Kontrolldateien ge\"offnet und parallel in diese geschrieben.
        \end{merke}
      \subsection{Multiplexing der Kontrolldateien}
        Jede Oracle-Datenbank sollte mindestens zwei Kopien einer Kontrolldatei
        haben, die auf verschiedenen Datentr\"agern gespeichert sind.
        Wurde eine Kontrolldateikopie w\"ahrend des laufenden Betriebs
        besch\"adigt, st\"urzt die Instanz meist sofort ab.

        Nach der Behebung des Medienfehlers kann die kaputte Kontrolldatei durch eine funktionsf\"ahige Kopie erneuert werden.

        Die Datenbank verwendet ihre Kontrolldateien wie folgt:
        \begin{itemize}
          \item Es wird gleichzeitig in alle Kontrolldateien geschrieben.
          \item Nur die erste aufgelistete Kontrolldatei wird gelesen
          \item Wird eine Kopie besch\"adigt, st\"urzt die Instanz meistens ab.
        \end{itemize}
        Eine Variante Kontrolldateien zu spiegeln ist, sie auf allen Datentr\"agern, die eine Redo Log Datei enthalten zu verteilen, da auch die Redo Logs gespiegelt werden sollten.
      \subsection{Hinzuf\"ugen und l\"oschen von Kontrolldateikopien}
        \subsubsection{Hinzuf\"ugen einer weiteren Kontrolldateikopie}
          Die zum Hinzuf\"ugen einer Kontrolldateikopie notwendigen Schritte sind:
        \begin{enumerate}
          \item Instanz herunterfahren und in die NOMOUNT-Phase bringen
            \begin{lstlisting}[caption={Hinzuf\"ugen von Kontrolldateikopien 1},label=admin26,language=sqlplus]
SQL> shutdown immediate
Database closed.
Database dismounted.
ORACLE instance shut down.

SQL> startup nomount
ORACLE instance started.
            \end{lstlisting}
          \item Kopieren einer funktionsf\"ahigen Kontrolldateikopie an den neuen Speicherort
            \begin{lstlisting}[caption={L\"oschen von Kontrolldateikopien 3},label=admin26a,language=sqlplus]
SQL> host cp /u02/oradata/orcl/control01.ctl /u03/oradata/orcl/control03.ctl
            \end{lstlisting}
          \item Den Parameter \parameter{control\_files} mit \languageorasql{ALTER SYSTEM} bearbeiten.
            \begin{lstlisting}[caption={Hinzuf\"ugen von Kontrolldateikopien 2},label=admin27,language=oracle_sql,alsolanguage=sqlplus]
SQL> show parameter control_files

NAME                                 &TYPE&     VALUE
------------------------------------ -------- ------------------------------
control_files                        string   /u02/oradata/orcl/control01.ct
                                              l, /u05/fast_recovery_area/orc
                                              l/control02.ctl
SQL> ALTER SYSTEM
  2  SET control_files='/u02/oradata/orcl/control01.ctl',
                       '/u05/fast_recovery_area/orcl/control02.ctl',
                       '/u03/oradata/orcl/control03.ctl'
  3  SCOPE=spfile;
            \end{lstlisting}
          \item Neustart der Instanz
        \end{enumerate}

        \subsubsection{L\"oschen von Kontrolldateikopien}
        Zum L\"oschen einer Kontrolldateikopie sind drei Schritte notwendig:
        \begin{enumerate}
          \item Instanz herunterfahren und in die NOMOUNT-Phase bringen
            \begin{lstlisting}[caption={L\"oschen von Kontrolldateikopien 1},label=admin28,language=sqlplus]
SQL> shutdown immediate
Database closed.
Database dismounted.
ORACLE instance shut down.

SQL> startup nomount
ORACLE instance started.
            \end{lstlisting}
          \item Den Parameter \parameter{control\_files} mit \languageorasql{ALTER SYSTEM} bearbeiten.
            \begin{lstlisting}[caption={L\"oschen von Kontrolldateikopien 2a},label=admin29,language=sqlplus]
SQL> show parameter control_files

NAME                                 &TYPE&     VALUE
------------------------------------ -------- ------------------------------
control_files                        string   /u02/oradata/orcl/control01.ct
                                              l, /u05/fast_recovery_area/orc
                                              l/control02.ctl
						\end{lstlisting}
\clearpage
						\begin{lstlisting}[caption={L\"oschen von Kontrolldateikopien
						2b},language=oracle_sql]
SQL> ALTER SYSTEM
  2  SET control_files='/u02/oradata/orcl/control01.ctl'
  3  SCOPE=spfile;
            \end{lstlisting}
          \item L\"oschen der Kontrolldateikopie mit Hilfe des Betriebssystems
            \begin{lstlisting}[caption={L\"oschen von Kontrolldateikopien 3},label=admin29a,language=sqlplus]
SQL> host rm -f /u05/fast_recovery_area/orcl/control02.ctl
            \end{lstlisting}
          \item Neustart der Instanz
        \end{enumerate}

        \begin{literaturinternet}
          \item \cite{i1006088}
        \end{literaturinternet}

    \section{Verwalten der Redo Logs}
      F\"ur den Betrieb einer Oracle-Datenbank wird ein Set aus \enquote{Redo Log Gruppen}, umgangssprachlich als \enquote{Redo Logs} bezeichnet, ben\"otigt. Eine Redo Log Gruppe besteht aus einer oder mehreren Redo Log Dateien, die auch als Redo Log Member bezeichnet werden. Die prim\"are Funktion der Redo Logs ist das Aufzeichnen aller \"Anderungen, die an den Daten vorgenommen wurden (Nutz- und Metadaten).
      \subsection{Redo Log Gruppen und Member erstellen}
        Obwohl die Konfiguration der Redo Log Gruppen bereits vor der Installation einer Datenbank erdacht werden sollte, kann es notwendig werden, weitere Redo Log Gruppen oder neue Member zu erstellen. F\"ur die Erstellung von Redo Log Gruppen und Membern muss der Nutzer das Privileg \privileg{ALTER DATABASE} besitzen.

        Das Erzeugen einer neuen Redo Log Gruppe geschieht mit \languageorasql{ALTER DATABASE ADD LOGFILE}.
          \begin{lstlisting}[caption={Erzeugen einer Redo Log Gruppe},label=admin30,language=oracle_sql]
SQL> ALTER DATABASE
  2  ADD LOGFILE '/u02/oradata/orcl/redo04a.log'
  3  SIZE 50M;
          \end{lstlisting}
          \begin{merke}
            Es sollte immer eine vollst\"andige Pfadangabe f\"ur die Member verwendet werden, da diese sonst im Verzeichnis \oscommand{ORACLE\_HOME/dbs} erstellt werden.
          \end{merke}
          Das obige Statement kann durch die Angabe der Redo Log Gruppen Nummer erweitert werden:
          \begin{lstlisting}[caption={Erzeugen einer Redo Log Gruppe mit Angabe der Gruppennummer},label=admin31,language=oracle_sql]
SQL> ALTER DATABASE
  2  ADD LOGFILE GROUP 5
  3  ('/u02/oradata/orcl/redo05a.log',
  4   '/u03/oradata/orcl/redo05b.log')
  5  SIZE 50M;
          \end{lstlisting}
          Die Angabe von \languageorasql{GROUP 5} sorgt daf\"ur, das die Datenbank versucht, die neue Redo Log Gruppe mit der Nummer 5 zu erstellen. Dies kann aber nur funktionieren, wenn diese Nummer noch nicht belegt ist. Die beiden neu erstellten Gruppen k\"onnen durch die View \identifier{v\$log} sichtbar gemacht werden.
          \begin{lstlisting}[caption={Die View \identifier{v\$log}},label=admin32,language=oracle_sql]
SQL> SELECT   group&\#&, members,
  2           bytes / POWER(1024, 2) AS Megabytes
  3  FROM     v$log
  4  ORDER BY group&\#&;

    &GROUP\#&     MEMBERS  MEGABYTES
---------- ---------- ----------
         1          1         50
         2          1         50
         3          1         50
         4          1         50
         5          2         50
          \end{lstlisting}
          Die Spalte \identifier{members} zeigt, dass die Gruppen 1 bis 4 mit jeweils nur einem Member angelegt wurden, was eine sehr gef\"ahrliche Konfiguration darstellt. Wird dieser eine Member besch\"adigt, ist folglich die gesamte Gruppe besch\"adigt. Dies f\"uhrt zu Problemen bei einem Datenbank-Recovery-Vorgang, wenn der Memberausfall nicht rechtzeitig bemerkt wird.
        \subsubsection{Redo Log Member einer Gruppe hinzuf\"ugen}
          Um einer Redo Log Gruppe einen neuen Member hinzuzuf\"ugen, wird das SQL-Kommando \languageorasql{ALTER DATABASE ADD LOGFILE MEMBER}, zusammen mit dem Dateinamen des Members, sowie der Nummer der betroffenen Redo Log Gruppe verwendet.
          \begin{lstlisting}[caption={Hinzuf\"ugen eines Members zu einer Redo Log Gruppe},label=admin33,language=oracle_sql]
SQL> ALTER DATABASE
  2  ADD LOGFILE MEMBER '/u03/oradata/orcl/redo01b.log'
  3  TO GROUP 1;
          \end{lstlisting}
          Passend zur View \identifier{v\$log}, die Informationen zu allen Log Gruppen anzeigt, gibt es auch die View \identifier{v\$logfile}. Dies befasst sich mit den Redo Log Members.
\clearpage
          \begin{lstlisting}[caption={Informationen \"uber Redo Log Member sammeln},label=admin34,language=oracle_sql,alsolanguage=sqlplus]
SQL> col member format a50
SQL> SELECT group&\#&, member
  2  FROM   v$logfile;

    GROUP&\#& MEMBER
---------- --------------------------------------------------
         3 /u02/oradata/orcl/redo03.log
         2 /u02/oradata/orcl/redo02.log
         1 /u02/oradata/orcl/redo01.log
         1 /u03/oradata/orcl/redo01b.log
         4 /u02/oradata/orcl/redo04a.log
         5 /u02/oradata/orcl/redo05a.log
         5 /u03/oradata/orcl/redo05b.log
          \end{lstlisting}
      \subsection{Redo Logs verschieben/umbenennen}
        Es gibt Situationen, die es notwendig machen, Redo Log Member an einen anderen Speicherort zu verschieben. Dies ist beispielsweise dann der Fall, wenn ein Datentr\"ager aus dem System entfernt werden soll oder wenn ein Datentr\"ager eine zu hohe Auslastung hat, weil sich zu viele Redo Log Dateien, Datendateien und Kontrolldateien darauf befinden.

        Das Verschieben/Umbenennen besteht im Wesentlichen aus zwei Schritten:
        \begin{enumerate}
          \item Verschieben/Umbenennen der Redo Log Datei mit Betriebssystemmitteln
          \item \"Andern des Dateipfades in der Datenbank
        \end{enumerate}
        F\"ur diesen Vorgang muss der Nutzer das Privileg \privileg{ALTER DATABASE} auf Seiten der Datenbank und entsprechende Rechte im Betriebssytem haben.
        \begin{enumerate}
          \item Abfragen von \identifier{v\$logfile}
            \begin{lstlisting}[caption={Speicherorte der Member ermitteln},label=admin35,language=oracle_sql,alsolanguage=sqlplus]
SQL> col member format a50
SQL> SELECT group&\#&, member
  2  FROM   v$logfile;

    GROUP&\#& MEMBER
---------- --------------------------------------------------
         3 /u02/oradata/orcl/redo03.log
         2 /u02/oradata/orcl/redo02.log
         1 /u02/oradata/orcl/redo01.log
         1 /u03/oradata/orcl/redo01b.log
         4 /u02/oradata/orcl/redo04a.log
         5 /u02/oradata/orcl/redo05a.log
         5 /u03/oradata/orcl/redo05b.log
             \end{lstlisting}
          \item Instanz herunterfahren
            \begin{lstlisting}[caption={Instanz herunterfahren},label=admin36,language=sqlplus]
SQL> shutdown immediate
            \end{lstlisting}
          \item Verschieben der Datei \oscommand{/u02/oradata/orcl/redo01.log} mit Betriebssystemmitteln nach \oscommand{/u02/oradata/orcl/redo01a.log}
            \begin{lstlisting}[caption={Verschieben der Redo Log Datei mit BS Mitteln},label=admin36a,language=sqlplus]
SQL> host mv /u02/oradata/orcl/redo01.log /u02/oradata/orcl/redo01a.log
            \end{lstlisting}
          \item Datenbank in die MOUNT-Phase bringen
            \begin{lstlisting}[caption={Datenbank MOUNTen},label=admin37,language=sqlplus]
SQL> startup mount
            \end{lstlisting}
          \item \"Andern des Dateipfades in der Datenbank
                  \begin{lstlisting}[caption={Redo Log Datei umbenennen},label=admin38,language=oracle_sql]
SQL> ALTER DATABASE
  2  RENAME FILE '/u02/oradata/orcl/redo01.log'
              TO '/u02/oradata/orcl/redo01a.log';
            \end{lstlisting}
          \item Datenbank \"offnen
            \begin{lstlisting}[caption={Datenbank \"offnen},label=admin39,language=oracle_sql]
SQL> ALTER DATABASE OPEN;
            \end{lstlisting}
        \end{enumerate}
      \subsection{L\"oschen von Redo Log Gruppen und Membern}
        \subsubsection{Eine Redo Log Gruppe l\"oschen}
          Sollte eine Redo Log Gruppe nicht mehr von N\"oten sein, kann diese gel\"oscht werden. Hierf\"ur ben\"otigt der Nutzer wiederrum das \privileg{ALTER DATABASE}-Privileg. Au\ss erdem sollten vor dem L\"oschen die folgenden Punkte bedacht werden:
          \begin{itemize}
            \item Jede Instanz ben\"otigt mindestens zwei Redo Log Gruppen mit einer beliebigen Anzahl von Membern.
            \item Nur eine Redo Log Gruppe, die den Status Inactive oder Unused hat, kann gel\"oscht werden.
            \item Falls die Archivierung f\"ur Redo Logs aktiviert ist, sollte vorher gepr\"uft werden, ob die zu l\"oschende Gruppe bereits archiviert wurde.
          \end{itemize}
\clearpage
            \begin{enumerate}
              \item Pr\"ufen, ob die Redo Log Gruppe inaktiv ist und evtl. bereits archiviert wurde
                \begin{lstlisting}[caption={Status der Redo Logs pr\"ufen},label=admin40,language=oracle_sql]
SQL> SELECT group&\#&, archived, status
  2  FROM   v$log;
                \end{lstlisting}
              \item Durchf\"uhren eines Log Switches und eines Checkpoints, damit die Gruppe in den Status \enquote{inactive} wechselt.
                \begin{lstlisting}[caption={Log Switch + Checkpoint durchf\"uhren},label=admin41,language=oracle_sql]
SQL> ALTER SYSTEM SWITCH LOGFILE;

SQL> ALTER SYSTEM CHECKPOINT;
                \end{lstlisting}
              \item Redo Log Gruppe l\"oschen
              \begin{lstlisting}[caption={Log Gruppe l\"oschen},label=admin42,language=oracle_sql]
SQL> ALTER DATABASE DROP LOGFILE GROUP 5;
                \end{lstlisting}
              \item L\"oschen aller Memberdateien der Redo Log Gruppe mit Betriebssystemmitteln.
              \begin{lstlisting}[caption={Log Gruppe l\"oschen},label=admin42a,language=sqlplus]
SQL> host rm /u02/oradata/orcl/redo05a.log /u03/oradata/orcl/redo05b.log
                \end{lstlisting}
            \end{enumerate}
        \subsubsection{Redo Log Member l\"oschen}
          Um einen Member aus einer Redo Log Gruppe l\"oschen zu k\"onnen, ben\"otigt der Nutzer das Privileg \privileg{ALTER DATABASE}. Au\ss erdem sollten vor dem L\"oschen die folgenden Punkte bedacht werden:
          \begin{itemize}
            \item Beim L\"oschen eines Members aus einer Redo Log Gruppe wird die Redo Log Konfiguration kurzzeitig asymetrisch. Dieser Zustand sollte so schnell wie m\"oglich bereinigt werden.
            \item Nur in einer Redo Log Gruppe, die den Status Inactive oder Unused hat, k\"onnen Member gel\"oscht werden.
            \item Falls die Archivierung f\"ur Redo Logs aktiviert ist, sollte vorher gepr\"uft werden, ob die zu l\"oschende Gruppe bereits archiviert wurde.
            \item Der letzte Member einer Redo Log Gruppe kann nicht gel\"oscht werden. Es muss dann die ganze Gruppe gel\"oscht werden.
          \end{itemize}
\clearpage
          \begin{enumerate}
            \item Pr\"ufen, ob die Redo Log Gruppe inaktiv ist und evtl. bereits archiviert wurde
              \begin{lstlisting}[caption={Status der Redo Logs pr\"ufen},label=admin43,language=oracle_sql]
SQL> SELECT group&\#&, archived, status
  2  FROM   v$log;
              \end{lstlisting}
            \item Durchf\"uhren eines Log Switches und eines Checkpoints, damit die Gruppe in den Status \enquote{inactive} wechselt.
              \begin{lstlisting}[caption={Log Switch + Checkpoint durchf\"uhren},label=admin44,language=oracle_sql]
SQL> ALTER SYSTEM SWITCH LOGFILE;

SQL> ALTER SYSTEM CHECKPOINT;
              \end{lstlisting}
            \item Redo Log Member l\"oschen
            \begin{lstlisting}[caption={Log Member l\"oschen},label=admin45,language=oracle_sql]
SQL> ALTER DATABASE
  2  DROP LOGFILE MEMBER '/u03/oradata/orcl/redo01b.log';
            \end{lstlisting}
            \begin{lstlisting}[caption={Log Member l\"oschen},label=admin45a,language=sqlplus]
SQL> host rm -f /u03/oradata/orcl/redo01b.log
            \end{lstlisting}
          \end{enumerate}
          Zuletzt muss die betreffende Datei noch betriebssystemseitig gel\"oscht werden.
      \subsection{Defekte Member bearbeiten}
        \subsubsection{Status von Redo Log Membern}
          Wird ein Redo Member besch\"adigt, erh\"alt er einen Fehlerstatus. Dieser kann in der View \identifier{v\$logfile} aus der Spalte \identifier{status} ersehen werden. Es gibt folgende Stati:
          \begin{itemize}
          \item \textbf{NULL} (Kein Wert): Ist die Statusspalte leer, ist der Redo Log Member voll funktionsf\"ahig.
          \item \textbf{INVALID}: Aus einem nicht n\"aher definierten Grund, kann auf die Datei nicht zugegriffen werden.
          \item \textbf{STALE}: Der Inhalt der Redo Log Member Datei ist nicht vollst\"andig. Dies kann entstehen, wenn w\"ahrend der Nutzung einer Redo Log Gruppe die Instanz abst\"urzt.
          \item \textbf{DELETED}: Dieser Status zeigt an, dass die Datei gel\"oscht wurde.
        \end{itemize}
\clearpage
        \subsubsection{Leeren einer Redo Log Gruppe}
          Wenn im laufenden Betrieb der Datenbank eine Redo Log Gruppe zerst\"ort wird und damit die Archivierung unm\"oglich geworden ist, muss die besch\"adigte Gruppe geleert werden. Der Vorteil dieser Methode ist, dass die Datenbank hierzu nicht heruntergefahren werden muss. Weiterhin gibt es zwei F\"alle, in denen das Leeren einer Redo Log Gruppe die einzige M\"oglichkeit darstellt, das Problem zu l\"osen:
          \begin{itemize}
            \item Wenn es nur zwei Redo Log Gruppen gibt
            \item Wenn in der Redo Log Gruppe mit dem Status Current eine Redo Log Datei defekt ist
          \end{itemize}
          Um eine Redo Log Gruppe zu leeren, die gerade in der Archivierung befindlich ist, wird folgendes Kommando verwendet:
          \begin{lstlisting}[caption={Redo Log Gruppe leeren},label=admin46,language=oracle_sql]
SQL> ALTER DATABASE
  2  CLEAR LOGFILE GROUP 3;
        \end{lstlisting}
        Zum Leeren einer noch nicht archivierten Redo Log Gruppe, muss das SQL-Schl\"usselwort \languageorasql{UNARCHIVED} hinzugef\"ugt werden.
        \begin{lstlisting}[caption={Eine nicht archivierte Redo Log Gruppe leeren},label=admin47,language=oracle_sql]
SQL> ALTER DATABASE
  2  CLEAR UNARCHIVED LOGFILE GROUP 3;
        \end{lstlisting}
        Dieses Statement verhindert, dass die betreffende Redo Log Gruppe jemals archiviert wird.
        \begin{merke}
          Zu beachten ist: Wird eine Redo Log Gruppe geleert, die zum Recovery der Datenbank ben\"otigt wird, sind alle Backups, die dieses Redo Log ben\"otigen nutzlos und es sollte sofort ein neues Backup der Datenbank angefertigt werden. In der Alert Log Datei werden die ung\"ultigen Datenbankbackups aufgelistet (nur mit RMAN erstellte Backups).
        \end{merke}
      \subsection{Informationen \"uber Redo Log Gruppen/Member sammeln}
        \begin{literaturinternet}
          \item \cite{i1007497}
        \end{literaturinternet}
    \section{Verwalten der archivierten Redo Logs}
    \label{administeringarchivelogs}
      \subsection{Was sind archivierte Redo Logs}
        Es ist m\"oglich, die gef\"ullten Redo Log Gruppen einer Oracle-Datenbank an einem oder mehreren anderen Speicherorten zu sichern. Diese gesicherten Redo Logs werden als \enquote{Archived Redo Logs} oder einfach als \enquote{Archive Logs} bezeichnet. Der Prozess des Sicherns der Redo Logs hei\ss{}t \enquote{Archiving}\footnote{archiving = engl. Archivierung, Sicherung}.
        \begin{merke}
          Wird eine Redo Log Gruppe durch multiplexing auf mehrere Speicherorte verteilt, archiviert der Archiver-Prozess immer nur eine der identischen Kopien der Redo Log Dateien.
        \end{merke}
        Damit die Redo Logs einer Oracle-Datenbank archiviert werden, muss die Datenbank in den \enquote{Archivelog-Modus} versetzt werden. Das Gegenst\"uck zum Archivelog-Modus ist der \enquote{Noarchivelog-Modus}. Im Noarchivelog-Modus findet keine Archivierung statt.
        Welcher der beiden Modi f\"ur die Datenbank verwendet werden sollte, h\"angt davon ab, ob Datenverlust akzeptabel ist oder nicht.

        \bild{Archivierung der Redo Logs}{archiving}{3}
      \subsection{F\"ur die Archivierung notwendige Initialisierungsparameter}
        \subsubsection{Anzahl der Archiver-Prozesse festlegen}
          Der Initialisierungsparameter \parameter{log\_archive\_max\_processes} legt die Anzahl der zu startenden Archiver-Prozesse fest (Standardwert ist 4). Eine Ver\"anderung dieses Standardwertes ist normalerweise nicht notwendig, da die Datenbank selbstst\"andig zus\"atzliche Archiver-Prozesse startet, wenn dies erforderlich erscheint.
\clearpage
					Ein m\"oglicher Grund f\"ur eine \"Anderung dieses Parameters ist z. B., dass es  performanter ist, gleich die richtige Anzahl Archiver-Prozesse zu starten. Es k\"onnen bis zu 30 Archiver-Prozesse gleichzeitig gestartet werden.

        \subsubsection{Speicherorte der Archiver-Prozesse festlegen}
         Zust\"andig f\"ur die Festlegung der Speicherorte der Archive Logs sind die Initialisierungsparameter \parameter{log\_archive\_dest\_1} bis \parameter{log\_archive\_dest\_31}. Um beispielsweise die beiden Speicherorte \oscommand{/u02/backup/archive\_logs} und \oscommand{/u03/backup/archive\_logs} festzulegen, m\"ussen die beiden Initialisierungsparameter \parameter{log\_archive\_dest\_1} und \parameter{log\_archive\_dest\_2} wie folgt angepasst werden:
          \begin{lstlisting}[caption={log\_archive\_dest-Parameter setzen},label=admin48,language=oracle_sql]
SQL> ALTER SYSTEM
  2  SET log_archive_dest_1='LOCATION=/u02/backup/archive_logs';
SQL> ALTER SYSTEM
  2  SET log_archive_dest_2='LOCATION=/u03/backup/archive_logs';
          \end{lstlisting}
        \subsubsection{Dateinamen der Archive Logs konfigurieren}
          Um das Format des Dateinamens f\"ur die Archive Logs zu setzen wird der Parameter \parameter{log\_archive\_format} verwendet. Der komplette Name eines Archive Logs setzt sich dann aus den Parameter \parameter{log\_archive\_dest\_n} + \parameter{log\_archive\_format} zusammen.
          \begin{lstlisting}[caption={\parameter{log\_archive\_format}-Parameter
          setzen},label=admin49,language=oracle_sql,alsolanguage=sqlplus]
SQL> ALTER SYSTEM
  2  SET log_archive_format='arch_%s_%r_%t.log'
  3  SCOPE=spfile;
          \end{lstlisting}
          \begin{merke}
            Hat der Initialisierungsparameter \parameter{compatible} einen Wert gr\"o\ss er oder gleich 10.0, m\"ussen im Parameter \parameter{log\_archive\_format} die Platzhalter \%r, \%s und \%t zwingend verwendet werden. Falls nicht wird die Fehlermeldung \enquote{\oscommand{ORA-19905: log\_archive\_format must contain \%s, \%t and \%r}} ausgel\"ost.
          \end{merke}
          Die Konfiguration der Archive Log Destination kann mittels der View \identifier{v\$archive\_dest} abgefragt werden.
          \begin{lstlisting}[caption={Die Konfiguration der Archive Log Destination abfragen},label=admin50,language=sqlplus]
SQL> col dest_name format a20
SQL> col destination format a30
SQL> col error format a40
SQL> set linesize 300
					\end{lstlisting}
\clearpage
\begin{lstlisting}[caption={Die Konfiguration der Archive Log Destination abfragen - Fortsetzung},label=admin50a,language=oracle_sql]
SQL> SELECT dest_name, status, destination, error
  2  FROM   v$archive_dest
  3  WHERE  dest_id < 3;

DEST_NAME            STATUS    &DESTINATION&               ERROR
------------------- --------- -------------------------- -------------------
LOG_ARCHIVE_DEST_1  VALID     /u02/backup/archive_logs
LOG_ARCHIVE_DEST_2  VALID     /u03/backup/archive_logs
          \end{lstlisting}
        \subsubsection{Status der Speicherorte}
          Jede Archive Log Destination ist immer mit einem Status versehen, der etwas \"uber ihre Funktionsf\"ahigkeit und ihre Nutzung aussagt.
        \begin{itemize}
          \item \textbf{Valid/Invalid}: Es wurde ein g\"ultiger bzw. ein ung\"ultiger Speicherort angegeben. G\"ultig hei\ss t, dass der Speicherort existieren muss.
          \item \textbf{Enabled/Disabled}: Ist der Speicherort aktiviert (wird genutzt) oder deaktiviert (wird nicht genutzt)?
          \item \textbf{Active/Inactive}: Ist der Speicherort zug\"anglich oder aufgrund eines Fehlers unzug\"anglich?
        \end{itemize}
        Tabelle \tabelle{redofehlerstatus} zeigt verschiedene Kombinantionen
        dies Stati.
        \begin{center}
          \tablecaption{M\"ogliche Zust\"ande der LOG\_ARCHIVE\_DEST\_n-Speicherorte}
          \tablefirsthead{%
            \hline
            \multicolumn{1}{|c|}{ } &
            \multicolumn{3}{|c|}{\textbf{Eigenschaften}} &
            \multicolumn{1}{|l|}{ }
            \\
            \cline{2-4}
            \multicolumn{1}{|c|}{\textbf{Status}} &
            \multicolumn{1}{c|}{\textbf{Valid}} &
            \multicolumn{1}{c|}{\textbf{Enabled}} &
            \multicolumn{1}{c|}{\textbf{Active}} &
            \multicolumn{1}{l|}{\textbf{Bedeutung}}
            \\
            \hline
          }
          \tablehead {%
          }
          \tabletail{%
            \hline
          }
          \begin{supertabular}[h]{|l|c|c|c|p{6.2cm}|}
            \label{redofehlerstatus}
            \multirow{2}{3cm}{VALID} & \multirow{2}{1.5cm}{Ja} & \multirow{2}{1.5cm}{Ja} & \multirow{2}{1.5cm}{Ja} & \footnotesize Es wurde ein g\"ultiger Speicherort angegeben, der auch erreichbar ist. \\
            \hline
            INACTIVE & \multirow{1}{1.5cm}{Nein} & -- & -- & \footnotesize Es wurde kein Speicherort angegeben. \\
            \hline
            \multirow{2}{3cm}{ERROR} & \multirow{2}{1.5cm}{Ja} & \multirow{2}{1.5cm}{Ja} & \multirow{2}{1.5cm}{Nein} & \footnotesize Es trat ein Fehler auf, als versucht wurde, am angegebenen Speicherort eine Datei zu erstellen. \\
            \hline
            FULL & \multirow{1}{1.5cm}{Ja} & \multirow{1}{1.5cm}{Ja} & \multirow{1}{1.5cm}{Nein} & \footnotesize Kein freier Speicher am Speicherort verf\"ugbar. \\
            \hline
            \multirow{2}{3cm}{DEFERRED} & \multirow{2}{1.5cm}{Ja} & \multirow{2}{1.5cm}{Nein} & \multirow{2}{1.5cm}{Ja} & \footnotesize Der Speicherort wurde durch den Nutzer zeitweilig deaktiviert. \\
            \hline
            \multirow{2}{3cm}{DISABLED} & \multirow{2}{1.5cm}{Ja} & \multirow{2}{1.5cm}{Nein} & \multirow{2}{1.5cm}{Nein} & \footnotesize Der Speicherort musste aufgrund eines Zugriffsfehlers zeitweilig deaktiviert werden. \\
            \hline
            \multirow{3}{3cm}{BAD PARAM} & \multirow{3}{1.5cm}{--} & \multirow{3}{1.5cm}{--} & \multirow{3}{1.5cm}{--} & \footnotesize Ein nicht n\"aher definierbarer Fehler ist aufgetreten (z. B. wurde ein ung\"ultiger Wert f\"ur einen Parameter angegeben).\\
          \end{supertabular}
        \end{center}
        \subsubsection{Probleme mit fehlerhaften Speicherorten}
          Es kann vorkommen, dass die Speicherung eines Archive Logs an einem der angegebenen Speicherorte fehlschl\"agt. Oracle stellt verschiedene M\"oglichkeiten bereit, wie auf solche Ausf\"alle reagiert werden kann. Eine M\"oglichkeit besteht darin, mit dem Parameter \parameter{log\_archive\_min\_succeed\_dest} anzugeben, auf wie vielen Speicherorten die Archivierung als erfolgreich gelten muss, ehe die zu archivierende Redo Log Gruppe wieder verwendet werden kann. Standardwert f\"ur diesen Parameter ist 1. Der Maximalwert ist 10.

          In Kombination dazu kann mit jedem \parameter{log\_archive\_dest\_n}-Parameter ein Speicherort als \textit{optional} (Standardwert) oder als \textit{mandatory} \footnote{mandatory = engl. obligatorisch, zwingend} deklariert werden. Je nachdem, wie der Parameter \parameter{log\_archive\_min\_succeed\_dest} eingestellt ist, muss die Archivierung mindestens an allen als mandatory deklarierten Speicherorten erfolgreich gewesen sein, bevor der Log Writer-Prozess die betreffende Redo Log Gruppe wieder verwenden kann.

          Die folgende Tabelle soll das Verhalten der Datenbank verdeutlichen:
          \begin{itemize}
            \item Es wurden 4 Speicherorte deklariert, 2 als optional und 2 als mandatory
            \item \parameter{log\_archive\_min\_succeed\_dest} wird auf die Werte 1 bis 5 eingestellt. Die Spalte Wert zeigt an, auf welchen Wert dieser Parameter gesetzt wurde. Die Spalte Auswirkungen beschreibt welche Auswirkungen diese Einstellung hat.
          \end{itemize}
          \begin{center}
            \tablecaption{Auswirkungen des Parameters log\_archive\_min\_succeed\_dest}
            \tablefirsthead{%
              \hline
              \multicolumn{1}{|c}{\textbf{Wert}}&
              \multicolumn{1}{|l|}{\textbf{Auswirkungen}}
              \\
              \hline
            }
            \tablehead{%
              \hline
              \multicolumn{1}{|c}{\textbf{Wert}}&
              \multicolumn{1}{|l|}{\textbf{Auswirkungen}}
              \\
              \hline
            }
            \tabletail{
              \hline
            }
            \begin{supertabular}[h]{|c|p{13cm}|}
              \multirow{2}{1.5cm}{1} & \footnotesize Die Datenbank ignoriert den f\"ur \parameter{log\_archive\_min\_succeed\_dest} eingestellten Wert und nimmt stattdessen die Anzahl der als mandatory deklarierten Speicherorte. \\
              \hline
              \multirow{3}{1.5cm}{2} & \footnotesize Die Archivierung gilt als erfolgreich, wenn an beiden als mandatory deklarierten Speicherorten die Speicherung erfolgreich war. Die als optional deklarierten Speicherorte werden bei der \"Uberpr\"ufung ignoriert. \\
              \hline
              \multirow{2}{1.5cm}{3} & \footnotesize Es m\"ussen beide als mandatory und mindestens einer der als optional deklarierten Speicherorte erfolgreich gewesen sein, bevor die Archivierung als erfolgreich gilt. \\
              \hline
              \multirow{2}{1.5cm}{4} & \footnotesize Die Archivierung ist erst dann erfolgreich, wenn an allen Speicherorten die Archived Logs gespeichert werden konnten. \\
              \hline
              \multirow{2}{1.5cm}{5} & \footnotesize Es tritt ein Fehler auf, da die Anzahl der Speicherorte geringer ist, als der Wert f\"ur \parameter{log\_archive\_min\_succeed\_dest}. \\
            \end{supertabular}
          \end{center}
          F\"ur das Zusammenspiel zwischen den beiden als mandatory deklarierten Speicherorten und dem Parameter \parameter{log\_archive\_min\_succeed\_dest} gelten die folgenden Regeln:
          \begin{itemize}
            \item Wird ein Speicherort nicht explizit als mandatory deklariert, wird er als optional angesehen.
            \item Wird f\"ur den Parameter \parameter{log\_archive\_min\_succeed\_dest} ein Wert angegeben, wird mindestens ein Speicherort als mandatory betrachtet.
            \item Der Wert f\"ur \parameter{log\_archive\_min\_succeed\_dest} kann nicht gr\"o\ss er sein, als die Anzahl der konfigurierten Speicherorte.
          \end{itemize}
          Das folgende Beispiel zeigt, wie ein Speicherort als mandatory deklariert wird:
          \begin{lstlisting}[caption={\parameter{log\_archive\_dest\_n} als mandatory deklarieren},label=admin51,language=oracle_sql]
SQL> ALTER SYSTEM
  2  SET log_archive_dest_1='LOCATION=/u02/backup/archive_logs MANDATORY';
SQL> ALTER SYSTEM
  2  SET log_archive_dest_2='LOCATION=/u03/backup/archive_logs OPTIONAL';
          \end{lstlisting}
      \subsection{Eine Datenbank in den Archivelog-Modus versetzen}
        Eine Datenbank kann bereits bei ihrer Erstellung oder auch nachtr\"aglich in den Archivelog-Modus versetzet werden. Dazu sind folgende Schritte notwendig:
        \begin{enumerate}
          \item Anpassen der Intialisierungsparameter, die f\"ur das Archiving notwendig sind
          \item Instanz konsistent herunterfahren
          \item Die Datenbank in die MOUNT-Phase versetzen.
          \item Den Archivierungsmodus einschalten.
          \item Datenbank \"offnen
          \item (Herunterfahren der Datenbank und durchf\"uhren eines Backups)
          \item (Datenbank hochfahren)
        \end{enumerate}
        \begin{lstlisting}[caption={Archivelog-Modus
        aktivieren},label=admin52,language=oracle_sql,alsolanguage=sqlplus] Connected to:
Oracle Database 11g Enterprise Edition Release 11.2.0.1.0 - 64bit Production
With the Partitioning, OLAP, Data Mining and Real Application Testing options

SQL> shutdown immediate
Database closed.
Database dismounted.
ORACLE instance shut down.
				\end{lstlisting}
\clearpage
\begin{lstlisting}[caption={Archivelog-Modus aktivieren -
Fortsetzung},label=admin52a,language=oracle_sql,alsolanguage=sqlplus]
SQL> startup mount 
ORACLE instance started.

Total System Global Area  764121088 bytes
Fixed Size                  2217264 bytes
Variable Size             503319248 bytes
Database Buffers          251658240 bytes
Redo Buffers                6926336 bytes
Database mounted.
SQL> ALTER DATABASE ARCHIVELOG;

Database altered.

SQL> ALTER DATABASE OPEN;

Database altered.
        \end{lstlisting}
    \section{Das Data Dictionary}
      Der wichtigste Teile einer Oracle-Datenbank ist das \enquote{Data
      Dictionary}. Es besteht aus einer Menge von Tabellen mit wichtigen
      Metainformationen, auf die alle Nutzer Lesezugriff haben.
      Darunter fallen:
      \begin{itemize}
        \item Die Definitionen aller Schemaobjekte der Datenbank (Tabellen, Views, Indizes, usw.)
        \item Wie viel Speicherplatz f\"ur Schemaobjekte reserviert und aktuell
        genutzt wird
        \item Standardwerte f\"ur Tabellenspalten
        \item Integrit\"ats Constraints
        \item Benutzernamen aller Oracle-Nutzer
        \item Privilegien und Rollen mit Zuordnung zu den Nutzern
        \item Auditing Informationen
        \item Weitere generelle Metainformationen
      \end{itemize}
      Die Tabellen des Data Dictionary sind auf die gleiche Art und Weise organisiert, wie andere Tabellen auch. Sie sind jedoch im \identifier{System}-Tablespace gespeichert und geh\"oren dem Nutzer \identifier{SYS}.

      Nicht nur, dass das Data Dictionary ein zentraler Punkt in der Datebank ist, es stellt auch ein wichtiges Hilfsmittel f\"ur alle Nutzer, vom Endnutzer bis zum Administrator dar. Es kann mit Hilfe von SQL-Befehlen abgefragt, nicht aber ge\"andert werden.

      Der Oracle Nutzer \identifier{SYS} ist Eigent\"umer der Basistabellen und Nutzer-Views des Data Dictionary. Kein anderer Oracle Nutzer au\ss er ihm sollte Zugriff auf diese Tabellen haben, da dies die Datenbankintegrit\"at verletzen k\"onnte.
      \subsection{Benutzung und Struktur des Data Dictionary}
        Das Data Dictionary wird auf drei unterschiedliche Arten genutzt:
        \begin{itemize}
          \item Oracle selbst greift lesend auf das Data Dictionary zu, um Informationen \"uber Nutzer, Privilegien und Schemaobjekte zu erhalten.
          \item Immer wenn ein DDL-Statement abgesetzt wird, wird das Data Dictionary durch Oracle modifiziert
          \item Jeder Oracle Nutzer kann das Data Dictionary als Nachschlagewerk nutzen.
        \end{itemize}
        Das Data Dictionary besteht aus den folgenden Bestandteilen:
        \subsubsection{Dynamic Performance Views (X\$-Views)}
          Oracle verwaltet eine Menge von Pseudo-Views, die als Dynamic Performance Views bezeichnet werden. Es handelt sich hierbei nicht um echte Views. Sie zeigen Informationen \"uber die Instanz und die Datenbank an und werden dynamisch im laufenden Betrieb durch den Kern von Oracle selbst aktualisiert.

          Diese View tragen das Pr\"afix \enquote{x\$} in ihrem Namen. Sie werden bei der Erstellung der Datenbank automatisch mit erstellt. Um den Inhalt der \identifier{x\$}-Views f\"ur Administratoren nutzbar zu machen, hat Oracle zus\"atzlich sogenannte \identifier{v\$}-Views geschaffen.
        \subsubsection{Dynamic Performance Views (V\$-Views)}
          \identifier{v\$}-Views benutzen als Informationsgrundlage die eben beschriebenen \identifier{x\$-Views}. W\"ahrend die Namensgebung bei \identifier{x\$}-Views sehr undurchschaubar ist (z. B. \identifier{x\$ksmfs}, \identifier{x\$ksmss} oder \identifier{x\$kcbwait}) ist die der \identifier{v\$}-Views klar verst\"andlich (z. B. \identifier{v\$session}, \identifier{v\$log} oder \identifier{v\$logfile}). Auch die Struktur der \identifier{v\$}-Views ist f\"ur Administratoren besser zu durchschauen, au\ss{}erdem sind \identifier{v\$}-Views dokumentiert, im Gegensatz zu den \identifier{x\$}-Views.
\clearpage
        \subsubsection{Basis Tabellen}
          Dies sind die zugrundeliegenden Tabellen, die die Metainformationen enthalten. Sie werden nur von der Datenbank selbst genutzt, da sie durch Normalisierung und kryptischen Inhalt f\"ur normale Nutzer wenig durchschaubar sind. Einige Beispiele f\"ur Basistabellen sind \identifier{user\$}, \identifier{tab\$}, \identifier{obj\$} oder \identifier{aud\$}.
        \subsubsection{Nutzer-Views}
          Diese Views stellen eine Zusammenfassung des Inhalts der Basistabellen dar. Sie dekodieren den un\"ubersichtlichen Inhalt der Basistabellen und machen ihn f\"ur Nutzer lesbar. Nicht jeder Nutzer hat auf alle Views Zugriff.
      \subsection{View-Klassen im Data Dictionary}
        Es gibt unterschiedliche Arten von Views im Data Dictionary. Einige sind f\"ur alle Nutzer zug\"anglich, andere nur f\"ur Administratoren. Die einzelnen Klassen k\"onnen an dem Pr\"afix in ihrem Namen erkannt werden. Folgende Klassen gibt es:
        \begin{itemize}
          \item \textbf{USER}: User-Views zeigen, welche Objekte im Schema des Nutzers existieren
          \item \textbf{ALL}: Erweiterte User-Views zeigen, auf welche Objekte der Nutzer Zugriff hat
          \item \textbf{DBA}: Administrative Views zeigen den gesamten Datenbankinhalt
        \end{itemize}
        \begin{merke}
          Nicht von allen Views existieren immer alle drei Klassen, z. B. \identifier{DBA\_LOCKS}.
        \end{merke}
        \subsubsection{Views mit dem Pr\"afix USER}
          Diese Views sind die Interesantesten f\"ur normale Nutzer. Sie haben folgende Eigenschaften:
          \begin{itemize}
            \item Sie zeigen das private Umfeld eines Nutzers, mit all seinen Objekten in der Datenbank.
            \item Sie zeigen nur Tabellenspalten die f\"ur den Nutzer relevant sind.
            \item Sie sind eine Untermenge der Views mit dem Pr\"afix ALL.
					\end{itemize}
\clearpage
					\subsubsection{Views mit dem Pr\"afix ALL}
          ALL-Views zeigen das f\"ur den Nutzer sichtbare Umfeld in der Datenbank. Sie liefern Informationen \"uber alle Datenbankobjekte, auf die der Nutzer Zugriff hat, inklusive der Informationen die durch USER-Views angezeigt werden. Im Gegensatz zu den USER-Views haben die ALL-Views eine Spalte \identifier{OWNER}, die den Eigent\"umer eines Objekts anzeigt.
        \subsubsection{Views mit dem Pr\"afix DBA}
          Views mit dem Pr\"afix DBA zeigen eine umfassende Ansicht der Datenbank mit allen Objekten, Privilegien und Nutzern. Auf diese Views hat nur administratives Personal Zugriff.
        \subsubsection{Die Tabelle DUAL}
          Die Tabelle \identifier{dual} ist eine Tabelle des Data Dictionary mit einer einzigen Spalte names \identifier{dummy}, die den Wert \enquote{X} enth\"alt. Sie ist f\"ur die Durchf\"uhrung von Berechnungen gedacht, wie das folgende Beispiel zeigt.
        \begin{lstlisting}[caption={Die Tabelle DUAL},label=admin53,language=oracle_sql]
SQL> SELECT SYSDATE AS Datum
  2  FROM   dual;

  DATUM
---------
29.08.13
        \end{lstlisting}

        \begin{literaturinternet}
          \item \cite{i125539}
          \item \cite{i2112}
        \end{literaturinternet}
\clearpage
    \section{Informationen}
      \subsection{Verzeichnis der relevanten Initialisierungsparameter}
        \begin{literaturinternet}
          \item \cite{REFRN10019}
          \item \cite{REFRN10021}
          \item \cite{REFRN10033}
          \item \cite{REFRN10075}
          \item \cite{REFRN10079}
          \item \cite{REFRN10086}
          \item \cite{REFRN10089}
          \item \cite{REFRN10090}
          \item \cite{REFRN10091}
          \item \cite{REFRN10094}
          \item \cite{REFRN10284}
          \item \cite{REFRN10285}
          \item \cite{REFRN10123}
          \item \cite{REFRN10165}
          \item \cite{REFRN10202}
          \item \cite{REFRN10198}
          \item \cite{REFRN10256}
          \item \cite{REFRN10243}
          \item \cite{REFRN10214}
        \end{literaturinternet}
\clearpage
      \subsection{Verzeichnis der relevanten Data Dictionary Views}
        \begin{literaturinternet}
          \item \cite{REFRN29014}
          \item \cite{REFRN30007}
          \item \cite{sthref3187}
          \item \cite{REFRN30089}
          \item \cite{sthref3267}
          \item \cite{sthref3423}
          \item \cite{REFRN30129}
          \item \cite{REFRN30128}
          \item \cite{REFRN30159}
          \item \cite{REFRN30314}
          \item \cite{REFRN30275}
        \end{literaturinternet}
\clearpage

    \input{uebungen/dbadmin_04_verwalten_einer_oracle_instanz_uebung}
    \input{loesungen/dbadmin_04_verwalten_einer_oracle_instanz_loesung}
    \chapter{Datenbank Storage-Strukturen verwalten}
    \setcounter{page}{1}
    \kapitelnummer{chapter}
    \minitoc
\newpage
      Die Hauptaufgabe einer Datenbank ist es, gro\ss{}e Datenmengen effizient
      zu verwalten und diese dem Nutzer \"uber eine einheitliche Schnittstelle
      zur Verf\"ugung zu stellen. Dieses Ziel wird von Oracle dadurch erreicht,
      dass der Datenzugriff und die Datenverwaltung in einem mehrschichtigen
      System gekapselt sind. Als Vorlage f\"ur dieses System dient die
      ANSI-SPARC-Architekur.
      \bild{Die ANSI-SPARC Architektur}{ansi_sparc_model}{1.7}
      Diese Architektur beschreibt, dass Benutzer mittels eines
      vereinheitlichten Zugriffsmechanismusses (hier die Sprache SQL) auf eine
      konzeptionelle Ebene zugreifen (hier dargestellt durch Tabellen), wobei
      die Daten physikalisch in Dateien, auf einem Dateisystem abgelegt werden.

      F\"ur den Benutzter bedeutet das, dass er Tabellen sieht, obwohl in den Dateien tats\"achlich nur \enquote{Bits und Bytes} existieren, keine Spur von Tabellen. Die Tabellen dienen nur als Darstellungsmittel f\"ur den Menschen, zur leichteren Verarbeitung der Daten.

      Der Vorteil dieses Modelles ist, dass sich beide Ebenen, die konzeptionelle und die interne Ebene, getrennt von einander ver\"andern k\"onnen, ohne den jeweils anderen zu beeinflussen. Diesen Vorteil, der durch die Gliederung in drei Ebenen zustande kommt, hat Oracle sich zu Nutze gemacht und seine Storage-Struktur ebenfalls auf einer mehrschichtigen Architektur aufgebaut.

      Die Oracle-Storage-Struktur gliedert sich in zwei Ebenen, die logische und die physikalische. W\"ahrend die physikalische mit Dateien, Bl\"ocken und Dateisystemen zu tun hat, beschreibt die logische den internen Aufbau der Datendateien.

      Unter der physikalen Storage-Struktur versteht man den Zusammenhang zwischen Dateisystem, Betriebssystembl\"ocken und Dateien. Der physikalische Aufbau einer Datei wird durch das Dateisystem gestaltet. Im Regelfall besteht eine Datei aus mehreren \enquote{Betriebssystembl\"ocken}, der kleinsten Speichereinheit, die ein Dateisystem verwalten kann.
      \bild{Die physi\-ka\-li\-sche Storage-Struktur}{oracle_storage_structure_1}{1.5}
      \begin{merke}
        Die eigentliche Dateiablage geschieht im Dateisystem. Welche Mechanismen, Buffer und Caches hier genutzt werden und welche Effekte dadurch zum Tragen kommen, ist f\"ur die Datenbank selbst unwesentlich. Sie k\"ummert sich nur um die interne Struktur der Datendateien. Der Datenbankadministrator jedoch sollte auch \"uber das benutzte Dateisystem Bescheid wissen, da es durch Fehler in diesem Bereich zu Performance\-einbu\ss{}en und sogar zu Datenverlust kommen kann!
      \end{merke}
      Die logische Storage-Struktur ist, genau wie die konzeptionelle Ebene des ANSI-SPARC-Modells, eine Abstrahierung der physikalischen Struktur. Das  hei\ss{}t, dass die Datenbank keine BS-Bl"ocke allokiert, sondern Datenbl\"ocke, Extents und Segmente. Diese drei Gr\"o\ss{}en sind der logischen Struktur zugeordnet, da sie nicht physikalisch auf dem Datentr\"ager existieren, sondern lediglich als Verwaltungseinheiten innerhalb der Oracle-Datenbank.
    \section{Die Komponenten der Oracle-Storage-Struktur}
      \subsection{Datenbl\"ocke}
        Die kleinste Einheit in der Oracle Speicherplatz verwaltet, ist der Oracle-Block oder auch Datenblock genannt. Die Standardgr\"o\ss{}e dieser Bl\"ocke betr\"agt 8 K und kann zwischen 2 K und 32 K variieren.
        \begin{merke}
          Die Gr\"o\ss{}e eines Oracle-Blocks muss zwingend ein Vielfaches der Gr\"o\ss{}e eines  Betriebssystemblocks entsprechen.
        \end{merke}
        Welche Blockgr\"o\ss{}e f\"ur eine Datenbank benutzt wird, muss bei der Datenbankerstellung mit Hilfe des Parameters \parameter{db\_block\_size} festgelegt werden und kann sp\"ater nicht mehr ver\"andert werden. Meist werden Werte wie 4 K oder 8 K genutzt.
        \begin{lstlisting}[caption={Der Parameter \parameter{db\_block\_size}},label=admin100,language=oracle_sql,alsolanguage=sqlplus]
SQL> col name format a30 SQL> col value format a10
SQL> SELECT name, value
  2  FROM   v$system_parameter
  3  WHERE  name LIKE 'db_block_size';

NAME                           VALUE
------------------------------ ----------
db_block_size                  8192
        \end{lstlisting}
        \bild{Die logische Storage-Struktur: Oracle-Bl\"ocke}{oracle_storage_structure_2}{1.4}
        \subsubsection{Das Format eines Oracle-Blocks}
          Das Format eines Datenblocks ist unabh\"angig von seinem Inhalt. Es gibt vier Bereiche in einem Oracle-Block:
          \begin{itemize}
            \item Block header
            \item Table directory
            \item Row directory
            \item Row data
          \end{itemize}
          Die drei Teile Block header, Table directory und Row directory werden unter dem Begriff \enquote{Data Block Overhead} zusammengefasst und stehen nicht f\"ur die Speicherung von Nutzdaten zur Verf\"ugung.

          Der Block Header enth\"alt allgemeine Informationen \"uber den Block,
          wie z. B. die Blockadresse und die Art des Blocks. Er ist
          durchschnittlich 48 bis 107 Byte gro\ss{}. Das Table Directory ist
          eine Auflistung aller Tabellen, die Zeilen in diesem Block besitzen
          und das Row Directory speichert die Positionen aller Tabellenzeilen im
          Block.
          \bild{Format eines Oracle-Blocks}{datablock_format}{1.1}

          Eine festgelegte Speichermenge wird in jedem Datenblock als Puffer f\"ur \languageorasql{UPDATE}-Operationen freigehalten. Dies erm\"oglicht das Wachstum bestehender Zeilen. In dem als \enquote{Row data} bezeichneten Teil eines Oracle-Blocks werden die Nutzdaten in Form von Zeilen abgelegt. Eine Tabellenzeile hat ein festgelegtes Format.
        \subsubsection{Aufbau einer Tabellenzeile}
          Oracle speichert jede Zeile einer Tabelle, nach M\"oglichkeit in einem St\"uck, das als \enquote{Row Piece} bezeichnet wird. Ein Row Piece setzt sich aus verschiedenen Feldern zusammen, wie sie in \abbildung{row_format} zu sehen sind.
          \bild{Format einer Tabellenzeile}{row_format}{1.3}
          Der Row Header ist der Verwaltungsteil einer Zeile. Hier werden Informationen gespeichert, wie z. B. die Anzahl der Spalten innerhalb der Zeile. Nach dem Header, werden in jeder Zeile deren L\"ange und die Nutzdaten gespeichert.

          Die Anordnung der Spalten einer Tabelle ist f\"ur alle Zeilen der Tabelle gleich. \"Ublicherweise werden die Spalten einer Tabelle in der Reihenfolge, in der sie im \languageorasql{CREATE TABLE}-Statement angegeben wurden, gespeichert. Dies muss jedoch nicht so sein.
        \subsubsection{Speicherung von NULL-Werten}
          Oracle speichert einen NULL-Wert in einer Tabellenspalte, in dem es die L\"ange der Spalte mit dem Wert 0 angibt (ben\"otigt 1 Byte). Enth\"alt eine Spalte am Ende der Tabelle einen NULL-Wert, wird f\"ur diese Spalte kein Wert und keine L\"ange gespeichert, wodurch Speicherplatz gespart wird.
          \begin{merke}
            Aus diesem Grund, sollten Spalten die sehr oft einen NULL-Wert enthalten immer an das Ende einer Tabelle gestellt werden.
          \end{merke}

          \bild{Null-Werte in Zeilen}{null_werte}{0.75}
        \subsubsection{Die RowID}
          Die \enquote{RowID} identifiziert eine Zeile eindeutig. Sie ist die 64-Bit-Kodierung des Zeilenspeicherorts, was bedeutet, dass  sie alle Angaben enth\"alt, die Oracle braucht, um eine Tabellenzeile in der Datenbank zu finden. F\"ur die Kodierung werden die Zeichen A-Z, a-z, 0-9, + und / verwendet.
\clearpage
          Das folgende Beispiel zeigt den Aufbau einer RowID:

          \languageorasql{AAAAaoAATAAABrXAAA}

          \begin{itemize}
            \item Stelle 1 - 6 \textbf{Data object number} AAAAao: Sie identifiziert das Segment (siehe \abschnitt{segments}), in dem die Zeile gespeichert ist.
            \item Stelle 7 - 9 \textbf{Relativ datafile number} AAT: Dieser Teil der RowID kodiert die Datendatei, in der das Segment gespeichert ist.
            \item Stelle 10 - 15 \textbf{Data block number} AAABrX: Die Data block number ist die Nummer des Datenblocks, der die Zeile enth\"alt.
            \item Stelle 16 - 18 \textbf{Row number} AAA: Die Nummer der Zeile.
          \end{itemize}
        \subsubsection{Rowchaining and Migration}
          Grunds\"atzlich speichert Oracle alle Tabellenzeilen immer in einem St\"uck. Es gibt jedoch Situationen, in denen dies nicht geht. Ist eine Zeile z. B. von Anfang an zu gro\ss{} f\"ur einen Oracle-Block oder umfasst sie mehr als 255 Spalten, so muss sie \"uber mehrere Datenbl\"ocke verteilt werden.
          \begin{merke}
            Eine Zeile die aus mehreren verketteten Row Pieces besteht wird als \enquote{Chained Row} bezeichnet.
          \end{merke}
          \bild{Eine verkettete Tabellenzeile}{chained_row}{1.3}
          Ein anderes Problem ist, dass Zeilen wachsen k\"onnen. Wird eine Tabellenzeile, z. B. aufgrund eines \languageorasql{UPDATE}-Statements l\"anger, kann es vorkommen, dass sie \"uber den Oracle-Block hinaus w\"achst. In so einem Fall wird die Zeile in einen neuen Block migrieren und an ihrer Originalposition verbleibt nur ein Pointer, ein Wegweiser zum neuen Speicherort. Man spricht von \enquote{Row migration}.
          \bild{Eine migrierte Tabellenzeile}{migrated_row}{1.3}
      \subsection{Extents}
        Extents sind eine weitere logische Speicherverwaltungseinheit. Sie bestehen aus einer ununterbrochenen Kette von Datenbl\"ocken. Wenn z. B. f\"ur  eine Tabelle Speicher ben\"otigt wird, sind Extents die kleinsten Einheiten die angefordert werden.
        \bild{Die logische Storage-Struktur: Extents}{oracle_storage_structure_3}{1.3}
        \subsubsection{Wann werden Extents angefordert?}
          Wird ein neues Objekt, wie z. B. eine Tabelle erstellt, wird ein erstes Extent, das sogenannte \textit{Initial Extent} angefordert. Auch wenn das Objekt zu diesem Zeitpunkt noch keine Daten enth\"alt, wird der Speicherplatz den das Initial Extent belegt als verbraucht angezeigt. Ein weiteres Extent wird erst dann angefordert, wenn das aktuelle Extent zu 100 \% gef\"ullt ist.
        \subsubsection{Wann werden Extents wieder freigegeben?}
          Im Allgemeinen werden Extents nur dann freigegeben, wenn das Objekt, dem sie zugewiesen wurden wieder gel\"oscht wird. Zus\"atzlich hat der DBA die M\"oglichkeit, die Datenbank oder Teile von ihr zu Reorganisieren und somit nicht mehr ben\"otigte Extents freizugeben.
      \subsection{Segmente}
        \label{segments}
        Segmente sind das dritte Glied in der Kette der logischen Speicherstrukturen. Sie bestehen aus einer Menge von Extents und enthalten alle Daten eines Datenbankobjekts, wie z. B. einer Tabelle oder eines Indexes. Jedes Segment besteht aus mindestens einem Extent, dass bei der Erstellung des Segments angefordert wird (Initial Extent).

        Seit Oracle 11g wird das Verfahren der \enquote{Deferred Segment Creation} angewandt. Dies bedeutet, dass bei der Erstellung eines Datenbankobjektes nur Metadaten in das Data Dictionary eingetragen werden und ein Segment erst beim Einf\"ugen von Daten angelegt wird. So wird die Verschwendung von Speicherplatz vermieden, falls bei der Installation einer Anwendung viele Tabellen angelegt werden, die u. U. nie gebraucht werden.

        Oracle fordert den Speicher f\"ur Segmente, Extent f\"ur Extent, an. Ist
        das Segment komplett gef\"ullt, wird ein weiteres Extent angefordert. Da
        die Extents f\"ur ein Segment nur bei Bedarf angefordert werden, sind
        diese meist nicht in einer fortlaufenden Reihenfolge.
        \bild{Die logische Storage-Struktur, Segmente}{oracle_storage_structure_4}{1.2}
      \subsection{Tablespaces und Datendateien}
        Die gr\"o\ss{}te existierende, logische Datenstruktur in Oracle sind die Tablespaces. Ein Tablespace besteht physikalisch aus einer oder mehreren Datendateien.
        \begin{itemize}
          \item Eine Oracle Datenbank besteht aus einem oder mehreren Tablespaces, die die gesamten Daten enthalten.
          \item Ein Tablespace besteht aus einer oder mehreren Datendateien.
        \end{itemize}
        \bild{Tablespaces und Datendateien}{tablespaces_and_datafiles}{1.3}
        \subsubsection{Verwaltung der Extents im Tablespace}
          Seit Oracle 9i gibt es zwei M\"oglichkeiten, wie die Verwaltung der freien Extents eines Tablespaces geschehen kann: Im Data Dictionary oder im Tablespace selbst.
          \begin{itemize}
            \item \textbf{Locally Managed Tablespaces}: Bei Locally Managed Tablespaces geschieht die Verwaltung der Extents direkt im Tablespace.
            \item \textbf{Dictionary Managed Tablespaces}: Dictionary Managed Tablespaces verwalten ihre Extents im Data Dictionary
          \end{itemize}
          Wurde ein Tablespace als Dictionary Managed Tablespace erstellt, kann sp\"ater die Verwaltung auf Locally Managed umgestellt werden. Umgekehrt jedoch nicht. Ist der \identifier{System}-Tablespace ein Locally Managed Tablespace, k\"onnen keine Dictionary Managed Tablespaces in der Datenbank erstellt werden. Im Normalfall werden alle Tablespaces als Locally Managed Tablespaces angelegt (erst seit Oracle 10g).

          Locally Managed Tablespaces verwalten, im Gegensatz zu Dictionary Managed Tablespaces, ihren Speicherplatz selbstst\"andig. Dies erfolgt mit Hilfe von Bitmaps. Jede Datendatei eines Tablespaces enth\"alt eine eigene Bitmap, um den F\"ullstatus ihrer Extents zu speichern.
          \bild{Verwaltung freier Extents mittels Bitmap}{locally_managed_ts}{1}
          \begin{merke}
            Da die Nutzung von Dictionary Managed Tablespaces nicht mehr zeitgem\"a\ss{} ist und von Oracle auch nicht mehr empfohlen wird, wird diese Art der Speicherverwaltung hier nicht n\"aher erl\"autert.
          \end{merke}
        \subsubsection{Die wichtigsten Tablespaces im \"Uberblick}
          Bei der Erzeugung einer Datenbank werden einige Tablespaces automatisch erstellt. Im Folgenden werden die vier Standardtablespaces einer Oracle-Datenbank erl\"autert.
          \subsubsection{Der System-Tablespace}
            Jede Oracle Datenbank enth\"alt einen Tablespace mit dem Namen \identifier{System}. Dieser wird bei der Datenbankerstellung erzeugt und muss f\"ur den Betrieb der Datenbank immer verf\"ugbar sein. Er enth\"alt das Herzst\"uck einer Oracle Datenbank, das Data Dictionary.
          \subsubsection{Der Sysaux-Tablespace}
            Der \identifier{Sysaux}-Tablespace stellt eine Erweiterung zum \identifier{System}-Tablespace dar. Er enth\"alt wichtige Komponenten f\"ur einige Datenbankanwendungen wie z. B. den Enterprise Manager. Auch dieser Tablespace wird automatisch bei der Datenbankerstellung erzeugt und kann im laufenden Betrieb nicht gel\"oscht werden. Ist er nicht verf\"ugbar, kann es zu unerwarteten Datenbankfehlern kommen.
          \subsubsection{Undo-Tablespaces}
            Undo-Tablespaces sind spezielle Tablespaces, die nur f\"ur die Aufnahme von Undo-In\-for\-ma\-tionen genutzt werden. Es k\"onnen keinerlei Segmente, wie z. B. Tabellen oder Cluster in einem Undo-Tablespace erstellt werden.
          \subsubsection{Tempor\"are Tablespaces}
            Tempor\"are Tablespaces enthalten tempor\"are Segmente und werden f\"ur gro\ss{}e Sortieroperationen ben\"otigt, die in der PGA eines Nutzers nicht durchgef\"uhrt werden k\"onnen. Auch in dieser Art von Tablespace k\"onnen manuell keine Objekte angelegt werden.
    \section{Automatic Segment Space Management (ASSM)}
      Die Verwaltung des Speicherplatzes umfasst nicht nur die Verwaltung der freien Extents, sondern auch die der freien Oracle-Bl\"ocke in den einzelnen Segmenten. Hierf\"ur gibt es zwei Verfahren. Das eine arbeitet mit \enquote{Freelists} und das andere mit Bitmaps. Beim Erstellen eines Tablespaces kann die Art der Speicherverwaltung gew\"ahlt werden.

      \begin{itemize}
        \item \textbf{Automatic Segment Space Management}: Der Speicherplatz in den Segmenten wird durch eine Bitmap verwaltet.
        \item \textbf{Manual Segment Space Management}: Der Speicherplatz in den Segmenten wird durch Freelists verwaltet.
      \end{itemize}
      \begin{merke}
        Die Standardvorgabe ist immer das Automatic Segment Space Management. Oracle empfiehlt bei dieser Einstellung zu bleiben.
      \end{merke}
      \subsection{Die Auxiliary Map}
        Beim Automatic Segment Space Management werden die Bl\"ocke eines Segments unter zu Hilfenahme einer Bitmap, der \enquote{Auxiliary Map} verwaltet. Sie untergliedert sich in bis zu vier Ebenen. \abbildung{auxiliary_map} zeigt den Aufbau einer dreistufigen Auxiliary Map.
        \subsubsection{Level 1 Bitmap-Bl\"ocke}
          Die Verwaltung des freien Speichers in den Oracle-Bl\"ocken, geschieht in den Level 1 Bitmaps (L1B). Jeder L1B verwaltet mehrere Oracle-Bl\"ocke (zwischen 16 und 1024 Bl\"ocke). F\"ur jeden Oracle-Block wird sein F\"ullgrad im L1B festgehalten.

          Die folgende Abbildung zeigt einen gek\"urzten Ausschnitt aus einem L1B.
\clearpage
          \bild{Architektur des Automatic Segment Space Management}{auxiliary_map}{1.45}

          \begin{lstlisting}[caption={Der Inhalt eines Level 1 Bitmap-Blocks},label=admin101,emph={[9]FULL},emphstyle={[9]\color{black}},language=terminal]
Freeness Status:  nf1 0      nf2 1      nf3 0      nf4 3
Extent Map Block Offset: 4294967295
First free datablock : 2
---------------------------------------------------------------------
DBA Ranges :
---------------------------------------------------------------------
 0x01c00029  Length: 8      Offset : 0
 0x01c00031  Length: 8      Offset : 8

 0:Metadata   1:FULL   2:26-50% free   3:FULL
 4:FULL   5:FULL   6:FULL   7:FULL
 8:FULL   9:FULL   10:76-100% free   11:76-100% free
 12:FULL   13:76-100% free   14:FULL   15:FULL

          \end{lstlisting}
          Die Zeile \enquote{Freeness Status} gibt an, wieviele Bl\"ocke mit welchem Belegungsgrad existieren. Dabei gilt:
          \begin{center}
            \tablecaption{Belegungsgrad von Oracle-Bl\"ocken}
            \tablefirsthead{%
              \hline
              \multicolumn{1}{|c}{\textbf{Freeness Status}}&
              \multicolumn{1}{|c|}{\textbf{Belegungsgrad}}
              \\
              \hline
            }
            \tablehead{%
              \hline
              \multicolumn{1}{|c}{\textbf{Freeness Status}}&
              \multicolumn{1}{|c|}{\textbf{Belegungsgrad}}
              \\
              \hline
            }
            \tabletail{
              \hline
            }
            \begin{supertabular}[h]{|c|l|}
              nf1 & Belegungsgrad zwischen 0 \% und 25 \% \\
              \hline
              nf2 & Belegungsgrad zwischen 26 \% und 50 \%  \\
              \hline
              nf3 & Belegungsgrad zwischen 51 \% und 75 \%  \\
              \hline
              nf4 & Belegungsgrad zwischen 76 \% und 100 \% \\
            \end{supertabular}
          \end{center}
          Somit existieren in \abbildung{admin101} 1 Block mit einem F\"ullstand von 26 \% bis 50 \% und 3 Bl\"ocke mit einem F\"ullgrad zwischen 76 \% und 100~\%. Bl\"ocke die noch leer oder zu 100 \% gef\"ullt sind, werden in dieser Zeile nicht ber\"ucksichtigt.

          Die Zeile \enquote{First free block} gibt, bezogen auf das Segment, die Blocknummer des ersten freien Blocks an.

          In den letzten vier Zeilen des Blockabbildes sind die exakten F\"ullgrade der einzelnen Bl\"ocke zu sehen.
        \subsubsection{Level 2 Bitmap-Bl\"ocke}
          Die Level 2 Bitmap-Bl\"ocke stehen eine Ebene \"uber den L1B in der hierarchischen Ordnung der Auxiliary Map. Sie enthalten die Adressen der Datenbl\"ocke, in denen sich die L1B befinden, die sie verwalten.
        \subsubsection{Level 3 Bitmap Bl\"ocke}
          Falls die Auxiliary Map stark anwachsen sollte, kann Oracle eine weitere Ebene in die Hierarchie einf\"ugen, die Level 3 Bitmap-Bl\"ocke. Sie stehen eine Stufe \"uber den Level 2 Bitmap-Bl\"ocken und enthalten demzufolge eine Liste der Blockadressen, in denen sich die Level 2 Bitmaps befinden.
      \subsection{Freie und Belegte Bl\"ocke - PCTFREE}
        Ob ein Oracle-Block als frei oder als belegt gilt, h\"angt stark davon ab wie viel Speicher als \enquote{Puffer f\"ur wachsende Zeilen} definiert wurde. Beim Erstellen einer Tabelle kann diese Puffergr\"o\ss{}e mittels des Storageparameters \languageorasql{pctfree} definiert werden. Wie die Abk\"urzung \enquote{PCT} in seinem Namen sagt, nimmt er einen Prozentwert als Angabe entgegen. \beispiel{admin102} zeigt seine Anwendung.
        \begin{lstlisting}[caption={Der Storageparameter \languageorasql{pctfree}},label=admin102,language=oracle_sql]
SQL> CREATE TABLE bank.bankfiliale
  2  (
  3    bankfiliale_id                NUMBER,
  4    strasse                       VARCHAR2(50 CHAR),
  5    hausnummer                    VARCHAR2(15 CHAR),
  6    plz                           CHAR(5 CHAR),
  7    ort                           VARCHAR2(20 CHAR)
  8  )
  9  PCTFREE 10;
        \end{lstlisting}
        \languageorasql{PCTFREE} wird in \beispiel{admin102} auf 10 \% gesetzt, was bedeutet, dass 10 \% des Speichers in jedem Datenblock als Puffer f\"ur wachsende Zeilen frei bleiben.
        \subsubsection{PCTFREE richtig setzen}
          Welcher Wert f\"ur \languageorasql{PCTFREE} gew\"ahlt werden sollte, h\"angt davon ab, ob h\"aufige \"Anderungen an einer Tabelle vorgenommen werden (viele \languageorasql{UPDATE}-Statements) oder ob die Tabelle lediglich mit neuen Daten bef\"ullt wird (viele \languageorasql{INSERT}-Statements). Was aber passiert, wenn \languageorasql{PCTFREE} falsch gesetzt wurde:
          \begin{center}
            \begin{small}
              \tablecaption{PCTFREE und seine Auswirkungen}
              \tablefirsthead{%
                \cline{2-3}
                \multicolumn{1}{c}{\textbf{}}&
                \multicolumn{1}{|c}{\textbf{PCTFREE zu klein}}&
                \multicolumn{1}{|c|}{\textbf{PCTFREE zu gro\ss{}}}
                \\
                \hline
              }
              \tablehead{%
                \cline{2-3}
                \multicolumn{1}{c}{\textbf{}}&
                \multicolumn{1}{|c}{\textbf{PCTFREE zu klein}}&
                \multicolumn{1}{|c|}{\textbf{PCTFREE zu gro\ss{}}}
                \\
                \hline
              }
              \tabletail{
                \hline
              }
              \begin{supertabular}[h]{|l|p{5cm}|p{5cm}|}
                Viele Updates & Die Tabellenzeilen haben nicht genug Platz zum Wachsen und migrieren (Migrated Rows!) & Es wird Speicherplatz verschwendet, da mehr Puffer zur Verf\"ugung gestellt wird, als n\"otig. \\
                \hline
                Viele Insert & Keine negativen Auswirkungen! & Es wird Speicherplatz verschwendet, da die Zeilen kaum wachsen. \\
              \end{supertabular}
            \end{small}
          \end{center}
        \subsubsection{Bl\"ocke werden wieder frei, wenn \dots}
          Der Parameter PCTFREE ist f\"ur die Tabelle \identifier{bankfiliale} auf den Wert 10 gesetzt. Daraus folgt, dass ein Datenblock als belegt gilt, wenn sein F\"ullgrad die 90 \% Marke \"uberschreitet. Diese Marke liegt in der Sektion von 76 \% bis 100 \%. Der Block wird wieder frei, wenn er die n\"achste untere Sektionsgrenze unterschreitet. Sein F\"ullgrad muss somit unter die Marke von 51 \% fallen, um diese Bedingung zu erf\"ullen, da die n\"achste untere Sektion von 51 \% bis 75 \% reicht.

          Mit dieser Vorgehensweise soll verhindert werden, das der Zustand eines Blockes in kurzen Intervallen von belegt auf frei und wieder zur\"uck wechselt.
      \subsection{Vor- und Nachteile von ASSM}
        Die Verwendung von Locally Managed Tablespaces mit Automatic Segment Space Management hat folgende Vorteile gegen\"uber der Verwendung von Dictionary Managed Tablespaces oder Locally Managed Tablespaces mit Manual Segment Space Management:
        \begin{itemize}
          \item \textbf{Rekursive Speicherverwaltungsoperationen werden vermieden.}

          In Dictionary Managed Tablespaces werden die Datenbl\"ocke in Tabellen im Data Dictionary verwaltet. Dabei kann es vorkommen, dass das Anfordern eines Extents f\"ur eine Tabelle mit Nutzdaten, die Vergr\"o\ss{}erung einer der beiden Tabellen \identifier{uet\$} oder \identifier{fet\$}, die f\"ur die Verwaltung der Extents zust\"andig sind, nach sich zieht.
          \item \textbf{\"Anderungen an der Extent-Bitmap erzeugen keine Undoinformationen.}
\clearpage
          \item \textbf{Der F\"ullgrad der einzelnen Bl\"ocke ist schneller verf\"ugbar.}

          Bei Automatic Segment Space Management kann der F\"ullgrad eines Block direkt aus der L1B entnommen werden, w\"ahrend beim Manual Segment Space Management der F\"ullgrad dem Blockheader des betreffenden Blockes entnommen werden muss.
          \item \textbf{Die freien Bl\"ocke m\"ussen nicht mehr in einer bestimmten Reihenfolge genutzt werden.}
        \end{itemize}
    \section{Tablespaces verwalten}
       \subsection{Das CREATE TABLESPACE-Kommando}
        \subsubsection{Anlegen eines Tablespaces mit Standardwerten}
          Tablespaces werden mit dem Kommando \languageorasql{CREATE TABLESPACE} angelegt. In seiner einfachsten Form ben\"otigt dieses Kommando nur drei Angaben:
          \begin{itemize}
            \item den Namen des Tablespaces in der \languageorasql{TABLESPACE}-Klausel,
            \item einen Pfad und einen Dateinamen f\"ur den Tablespace in der \languageorasql{DATAFILE}-Klausel
            \item und eine Gr\"o\ss{}e f\"ur die Datendatei in der \languageorasql{SIZE}-Klausel.
          \end{itemize}
          \begin{lstlisting}[caption={Das \languageorasql{CREATE TABLESPACE}-Kommando},label=admin103,language=oracle_sql]
SQL> CREATE TABLESPACE bank
  2  DATAFILE '/u02/oradata/orcl/bank.dbf' SIZE 50M;
          \end{lstlisting}
          \begin{merke}
            In der \languageorasql{DATAFILE}-Klausel k\"onnen eine oder mehrere Datendateien angegeben werden, aus denen der Tablespace bestehen wird. Mit dem Schl\"usselwort \languageorasql{SIZE} wird die Gr\"o\ss{}e jeder einzelnen Datendatei festgelegt.
          \end{merke}
          \begin{lstlisting}[caption={Ein Tablespace mit mehreren Datendateien},label=admin104,language=oracle_sql]
SQL> CREATE TABLESPACE bank
  2  DATAFILE '/u02/oradata/orcl/bank01.dbf' SIZE 50M,
  3           '/u03/oradata/orcl/bank02.dbf' SIZE 200M;
          \end{lstlisting}
        \subsubsection{Gestaltung der Extents beeinflussen}
          Seit Oracle 10g ist jeder Tablespace von Haus aus lokal verwaltet. Locally Managed Tablespaces k\"onnen ihre Extents auf zwei unterschiedliche Arten gestalten:
          \begin{itemize}
            \item \textbf{wachsend}: Die Extents im Tablespace wachsen automatisch. Das erste hat eine Gr\"o\ss{}e von 64 K und das zweite Extent ebenso. Ab dem dritten Extent steuert Oracle das Wachstum. Je mehr Extents angefordert werden, desto gr\"o\ss{}er werden diese. Sie k\"onnen Gr\"o\ss{}en von 1 M, 8 M und sogar 64 M annehmen.
            \item \textbf{gleichf\"ormig}: Gleichf\"ormig bedeutet, dass alle Extents mit einer einheitlichen Gr\"o\ss{}e angelegt werden. \languageorasql{UNIFORM SIZE 1M} ist der Standard, falls nichts n\"aher definiert wird.
          \end{itemize}
          F\"ur beide Verfahren muss das \languageorasql{CREATE TABLESPACE}-Kommando um die \languageorasql{EXTENT MANAGEMENT LOCAL}-Klausel erweitert werden. Sollen automatisch wachsende Extents erzeugt werden, wird zus\"atzlich das Schl\"usselwort \languageorasql{AUTOALLOCATE} ben\"otigt.
          \begin{lstlisting}[caption={Ein Tablespace mit automatisch wachsenden Extents},label=admin105,language=oracle_sql]
SQL> CREATE TABLESPACE bank
  2  DATAFILE '/u02/oradata/orcl/bank01.dbf' SIZE 50M,
  3           '/u03/oradata/orcl/bank02.dbf' SIZE 200M
  4  EXTENT MANAGEMENT LOCAL AUTOALLOCATE;
          \end{lstlisting}
          Die Angabe von \languageorasql{EXTENT MANAGEMENT LOCAL AUTOALLOCATE} ist immer dann vorteilhaft, wenn in einem Tablespace Objekte mit sehr unterschiedlichem Volumen gespeichert werden. Durch das Anwachsen der Extents wird der Verwaltungsaufwand f\"ur das Anfordern verringert, da mit zunehmender Extentgr\"o\ss{}e weniger Extents angefordert werden m\"ussen.
          \begin{merke}
            \languageorasql{EXTENT MANAGEMENT LOCAL AUTOALLOCATE} ist der Standard!
          \end{merke}
          Uniform-Sized-Extents sind immer dann die bessere Wahl, wenn die Gr\"o\ss{}e der Daten, die in die Tabelle eingef\"ugt werden, in etwa vorausgesagt werden kann. Dies kann z. B. dann der Fall sein, wenn PDF oder DOC-Dateien in den Tabellen gespeichert werden sollen.
          \begin{lstlisting}[caption={Uniform-Sized-Extents},label=admin106,language=oracle_sql]
SQL> CREATE TABLESPACE bank
  2  DATAFILE '/u02/oradata/orcl/bank01.dbf' SIZE 50M,
  3           '/u03/oradata/orcl/bank02.dbf' SIZE 200M
  4  EXTENT MANAGEMENT LOCAL UNIFORM SIZE 128K;
          \end{lstlisting}
          \beispiel{admin106} zeigt wie der Tablespace \identifier{bank} mit gleichf\"ormigen 128 K Extents angelegt wird.
          \begin{merke}
            Die Standardgr\"o\ss{}e f\"ur Uniform-Sized-Extents ist 1M!
          \end{merke}
        \subsubsection{Bigfile Tablespaces}
          Bigfile Tablespaces haben die Besonderheit, dass sie nur aus einer einzigen Datendatei bestehen. Somit entf\"allt der dreistellige Anteil der Relative File Number aus der RowID, wodurch die Datendatei bis zu $2^{32}$-Datenbl\"ocke (ca. 4 Milliarden) haben kann. Bei einer maximalen Blockgr\"o\ss{}e von 32 K erm\"oglicht dies eine Kapazit\"at von 128 TB pro Datendatei, bei einer Blockgr\"o\ss{}e von 8 K, sind es 32 TB pro Datendatei.

          Da eine Oracle Datenbank bis zu 65.536 Datendateien haben darf, wird eine Datenbankgr\"o\ss{}e von bis zu 8 Exabyte m\"oglich.

          Im Vergleich dazu, kann ein Smallfile Tablespace nur bis zu $2^{22}$-Datenbl\"ocke enthalten, was bei einer Blockgr\"o\ss{}e von 32 K nur 128 GB gesamtgr\"o\ss{}e ausmacht. Die Gesamtgr\"o\ss{}e der Datenbank, bei 65.536 Datendateien, w\"are dann nur 8 Petabyte, statt 8 Exabyte.
          \begin{merke}
            Seit dem Bigfile Tablespaces existieren, werden \enquote{normale Tablespaces} als Smallfile Tablespaces bezeichnet.
          \end{merke}
          \begin{literaturinternet}
            \item \cite{REFRN0042}
            \item \cite{doctbtthtm}
          \end{literaturinternet}

          \beispiel{admin107} zeigt, wie ein Bigfile Tablespace angelegt wird.
          \begin{lstlisting}[caption={Einen Bigfile Tablespace anlegen},label=admin107,language=oracle_sql]
SQL> CREATE BIGFILE TABLESPACE big_mac_ts
  2  DATAFILE '/u02/oradata/orcl/big_mac_ts.dbf' SIZE 100G;
          \end{lstlisting}
          Ob es sich bei einem Tablespace um einen Bigfile Tablespace handelt, kann mittels der Spalte \identifier{bigfile}, der View \identifier{dba\_tablespaces} ermittelt werden.
          \begin{lstlisting}[caption={Die View \identifier{dba\_tablespaces}},label=admin108,language=oracle_sql]
SQL> SELECT tablespace_name, bigfile
  2  FROM   dba_tablespaces;

TABLESPACE_NAME                BIG
------------------------------ ---
&SYSTEM&                          NO
SYSAUX                         NO
UNDOTBS1                       NO
TEMP                           NO
EXAMPLE                        NO
BIG_MAC_TS                     YES
          \end{lstlisting}
          \begin{merke}
            Ein Bigfile Tablespace wird automatisch als Locally Managed Tablespace mit Automatic Segment Space Management angelegt. Durch Dateigr\"o\ss{}enbeschr\"ankungen in Dateisystemen kann die Erstellung von Bigfile Tablespaces auf eine bestimmte Gr\"o\ss{}e limitiert sein.
          \end{merke}
        \subsubsection{Tempor\"are Tablespaces}
          Tempor\"are Tablespaces werden immer dann ben\"otigt, wenn Sortier- oder Hashingoperationen nicht in der PGA eines Serverprozesses durchgef\"uhrt werden k\"onnen, weil mehr Speicherplatz ben\"otigt wird, als zur Verf\"ugung steht. Ihre Funktion ist vergleichbar mit der der Windows Auslagerungsdatei.

          Durch Hinzuf\"ugen des Schl\"usselwortes \languageorasql{TEMPORARY} zum \languageorasql{CREATE TABLESPACE}-Statement wird ein tempor\"arer Tablespace angelegt. Die \languageorasql{DATAFILE}-Klausel wird bei tempor\"aren Tablespaces durch die \languageorasql{TEMPFILE}-Klausel ersetzt.
          \begin{lstlisting}[caption={Einen tempor\"aren Tablespace anlegen},label=admin109,language=oracle_sql]
SQL> CREATE TEMPORARY TABLESPACE bank_temp
  2  TEMPFILE '/u02/oradata/orcl/bank_temp_01.dbf' SIZE 20 M;
          \end{lstlisting}
          \begin{merke}
            Tempor\"are Tablespaces k\"onnen nur mit Uniform-Sized-Extents angelegt werden. Die \languageorasql{EXTENT MANAGEMENT LOCAL UNIFORM SIZE}-Klausel kann wahlweise dazu benutzt werden, um die Gr\"o\ss{}e der Extents zu beeinflussen. Das Schl\"usselwort \languageorasql{AUTOALLOCATE} darf bei tempor\"aren Tablespaces nicht verwendet werden.
          \end{merke}
          Ein tempor\"arer Tablespace kann auch als Big-File Tablespace angelegt werden.
          \begin{lstlisting}[caption={Einen tempor\"aren Big-File Tablespace anlegen},label=admin110,language=oracle_sql]
SQL>  CREATE BIGFILE TEMPORARY TABLESPACE bank_temp
  2   TEMPFILE '/u02/oradata/orcl/bank_temp.dbf' SIZE 20G;
          \end{lstlisting}
      \subsection{Das ALTER TABLESPACE-Kommando}
        Mit Hilfe des \languageorasql{ALTER TABLESPACE}-Kommandos ist es m\"oglich, die Definition eines Tablespaces zu ver\"andern. Im Folgenden werden einige Anwendungsf\"alle f\"ur dieses Kommando gezeigt.
        \subsubsection{Tablespaces On- und Offline setzen}
          Ein DBA kann alle Tablespaces, mit Ausnahme des \identifier{System}-Tablespace, online (verf\"ugbar) und offline (nicht verf\"ugbar) setzen. Dies kann im laufenden Betrieb geschehen und ist f\"ur Wartungst\"atigkeiten, wie z. B. das Recovery eines einzelnen Tablespaces notwendig.

          Wird ein Tablespace offline gesetzt, verbietet Oracle den Zugriff auf alle Objekte in diesem Tablespace. Laufende Transaktionen werden durch das Offlinesetzen nicht automatisch beendet, ein abschlie\ss{}en der Transaktion mit \languageorasql{COMMIT} ist nach wie vor m\"oglich.

          In manchen F\"allen kann es vorkommen, das Oracle einen Tablespace automatisch offline setzt, z. B. wenn der Tablespace durch einen Medienfehler besch\"adigt wurde.
          \begin{lstlisting}[caption={Einen Tablespace offline setzen},label=admin111,language=oracle_sql]
SQL> ALTER TABLESPACE bank OFFLINE;
          \end{lstlisting}
          Ein Tablespace kann auf drei verschiedene Arten offline gesetzt werden:
          \begin{itemize}
            \item \textbf{OFFLINE NORMAL}: Der Tablespace wird so offline gesetzt, dass kein Recovery ben\"otigt wird, wenn der Tablespace wieder online gesetzt werden soll. Dies funktioniert, jedoch nur dann, wenn alle Datendateien des Tablespaces fehlerfrei sind.
            \item \textbf{OFFLINE TEMPORARY}: Mit diesem Modus k\"onnen alle noch verbliebenen, fehlerfreien Datendateien eines Tablespaces konsistent offline gesetzt werden. Fehlerhafte Datendateien werden ignoriert.
            \item \textbf{OFFLINE IMMEDIATE}: Der Tablespace wird sofort, ohne Checkpoint offline gesetzt. Beim Onlinesetzen wird in jedem Falle Recovery ben\"otigt.
          \end{itemize}
          \beispiel{admin112} verdeutlicht den Unterschied zwischen \languageorasql{OFFLINE NORMAL} und \languageorasql{OFFLINE TEMPORARY}.
          \begin{lstlisting}[caption={Der Unterschied zwischen NORMAL und TEMPORARY},label=admin112,language=oracle_sql]
SQL> CREATE TABLESPACE defekt_ts
  2  DATAFILE '/u02/oradata/orcl/defekt_ts_01.dbf' SIZE 5M,
  3           '/u03/oradata/orcl/defekt_ts_02.dbf' SIZE 10M;

-- Die Datei defekt_ts_01.dbf wird zerstoert!

SQL> SELECT file_name, online_status
  2  FROM   dba_data_files
  3  WHERE  file_name LIKE 'DEFEKT_TS_01.DBF';

FILE_NAME                           ONLINE_STATUS
----------------------------------  -------------
/u02/oradata/orcl/defekt_ts_01.dbf   &RECOVER&
          \end{lstlisting}
\clearpage
          \begin{lstlisting}[language=oracle_sql,alsolanguage=sqlplus]
SQL> ALTER TABLESPACE defekt_ts OFFLINE NORMAL;

ERROR at line 1:
ORA-01191: file 6 is already offline - cannot do a &normal& offline
ORA-01110: data file 6: '/u02/defekt_ts_01.dbf'

SQL> ALTER TABLESPACE defekt_ts OFFLINE TEMPORARY;

Tablespace altered.
          \end{lstlisting}
          Soll der Tablespace wieder Online gehen, wird das Schl\"usselwort \languageorasql{ONLINE} verwendet.
          \begin{lstlisting}[caption={Einen Tablespace online setzen},label=admin113,language=oracle_sql]
SQL> ALTER TABLESPACE defekt_ts ONLINE;
          \end{lstlisting}
        \subsubsection{Read Only Tablespaces}
          Wird ein Tablespace mit der \languageorasql{READ ONLY}-Klausel
          schreibgesch\"utzt, hat das zwei Effekte:
          \begin{itemize}
            \item Unbeabsichtigte \"Anderungen an den Daten werden verhindert.
            \item Der Oracle Recovery Manager erkennt Read Only Tablespaces. Nachdem ein solcher Tablespace einmal gesichert wurde, wird er bei allen folgenden Backups ausgelassen.
          \end{itemize}
          \begin{lstlisting}[caption={Einen Tablespace Read Only setzen},label=admin114,language=oracle_sql]
SQL> ALTER TABLESPACE bank READ ONLY;
          \end{lstlisting}
          Um den Schreibzugriff auf den Tablespace wieder zu erm\"oglichen, wird die Klausel \languageorasql{READ ONLY} durch die Klausel \languageorasql{READ WRITE} ersetzt.
          \begin{lstlisting}[caption={Einen Tablespace Read Write setzen},label=admin115,language=oracle_sql]
SQL> ALTER TABLESPACE bank READ WRITE;
          \end{lstlisting}
          Zur Durchf\"uhrung dieser T\"atigkeiten, muss der Nutzer eines der beiden Systemprivilegien \privileg{ALTER TABLESPACE} oder \privileg{MANAGE TABLESPACE} besitzen.
        \subsubsection{Tablespaces umbenennen}
          Mit der \languageorasql{RENAME TO}-Klausel, des \languageorasql{ALTER TABLESPACE}-Kommandos, k\"onnen Tablespaces umbenannt werden. Die Umbenennung ist sowohl f\"ur permanente, als auch f\"ur tempor\"are Tablespaces m\"oglich.
          \begin{lstlisting}[caption={Einen Tablespace umbenennen},label=admin116,language=oracle_sql]
SQL> ALTER TABLESPACE bank RENAME TO bank_ts;
          \end{lstlisting}
          \begin{merke}
            Soll ein \identifier{Undo}-Tablespace umbenannt werden, ist es unbedingt notwendig, dass zum Betrieb der Instanz ein SPFile verwendet wird.
          \end{merke}
        \subsubsection{Big-File Tablespaces vergr\"o\ss{}ern}
          Ein Big-File Tablespace kann mit Hilfe der \languageorasql{RESIZE}-Klausel der \languageorasql{ALTER TABLESPACE}-An\-wei\-sung vergr\"o\ss{}ert werden. Da er nur eine Datendatei hat, wirkt sich die \"Anderung direkt auf diese aus.
          \begin{lstlisting}[caption={Einen Big-File Tablespace vergr\"o\ss{}ern},label=admin117,language=oracle_sql]
SQL> ALTER TABLESPACE big_mac_ts RESIZE 200G;
          \end{lstlisting}
      \subsection{Das DROP TABLESPACE-Kommando}
          Ein Tablespace kann mitsamt seinem Inhalt gel\"oscht werden, wenn er nicht mehr ben\"otigt wird. Um einen Tablespace l\"oschen zu k\"onnen, ben\"otigt der Nutzer das \privileg{DROP TABLESPACE}-System Privileg. Vor dem L\"oschen sollte unbedingt ein Backup der Datenbank gemacht werden, so dass der Tablespace im Zweifelsfalle wieder hergestellt werden kann.

          Ein leerer Tablespace kann einfach mit dem \languageorasql{DROP TABLESPACE}-Kommando gel\"oscht werden. Um Probleme oder Verz\"ogerungen beim L\"oschen zu vermeiden, sollte der Tablespace immer zuerst offline gesetzt werden.
          \begin{lstlisting}[caption={Einen leeren Tablespace l\"oschen},label=admin118,language=oracle_sql]
SQL> ALTER TABLESPACE big_mac_ts OFFLINE IMMEDIATE;
SQL> DROP TABLESPACE big_mac_ts;
          \end{lstlisting}
          Enth\"alt der Tablespace Segmente, muss an das \languageorasql{DROP TABLESPACE}-Kommando die Klausel \languageorasql{INCLUDING CONTENTS} angeh\"angt werden.
          \begin{lstlisting}[caption={Einen Tablespace mit Inhalt l\"oschen},label=admin119,language=oracle_sql]
SQL> CREATE TABLE empty_table
  2  (
  3    empty_id NUMBER
  4  )
  5  TABLESPACE big_mac_ts;

SQL> DROP TABLESPACE big_mac_ts;

ERROR at line 1:
ORA-01549: tablespace not empty, use INCLUDING CONTENTS option

SQL> DROP TABLESPACE big_mac_ts INCLUDING CONTENTS;

Tablespace dropped.
          \end{lstlisting}
          \begin{merke}
            Beide Male werden die Datendateien der betreffenden Tablespaces nicht mit gel\"oscht. Diese k\"onnen entweder manuell mit Betriebssystemmitteln gel\"oscht werden oder es kann die \languageorasql{AND DATAFILES}-Klausel des \languageorasql{DROP TABLESPACE}-Kommandos verwendet werden.
          \end{merke}
          \begin{lstlisting}[caption={Einen Tablespace mit Inhalt und Datendateien l\"oschen},label=admin120,language=oracle_sql]
SQL> DROP TABLESPACE big_mac_ts
  2  INCLUDING CONTENTS AND DATAFILES;
          \end{lstlisting}
          Enthalten die Segmente im Tablespace Fremdschl\"ussel-Constraints, die auf Segmente in einem anderen Tablespace verweisen, muss zus\"atzlich die \languageorasql{CASCADE CONSTRAINTS}-Klausel angeh\"angt werden.
          \begin{lstlisting}[caption={Einen Tablespace mit Inhalt, Datendateien und Constraints l\"oschen},label=admin121,language=oracle_sql]
SQL> DROP TABLESPACE big_mac_ts
  2  INCLUDING CONTENTS AND DATAFILES
  3  CASCADE CONSTRAINTS;
          \end{lstlisting}
    \section{Datendateien verwalten}
      \subsection{Hinzuf\"ugen einer Datendatei zu einem Tablespace}
        Datendateien werden mit Hilfe der \languageorasql{ADD DATAFILE}-Klausel an einen Tablespace angef\"ugt.
        \begin{lstlisting}[caption={Hinzuf\"ugen einer Datendatei},label=admin122,language=oracle_sql]
SQL> ALTER TABLESPACE bank
  2  ADD DATAFILE '/u03/oradata/orcl/bank02.dbf' SIZE 100M;
        \end{lstlisting}
        \begin{merke}
          Bei einem Bigfile Tablespace kann keine weitere Datendatei hinzugef\"ugt werden.
        \end{merke}
      \subsection{Das Wachstum von Datendateien kontrollieren}
        \subsubsection{Automatisches Wachstum erlauben}
          Eine Datendatei kann so erstellt werden, bzw. ihr Verhalten kann so ge\"andert werden, dass sie bei Bedarf automatisch w\"achst. Dies hat folgende Vorteile:
          \begin{itemize}
            \item Der Administrator wird entlastet, da er nicht sofort eingreifen muss, wenn die Datendatei zu klein ist.
            \item Anwendungen bleiben nicht im Betrieb stehen, weil zu wenig Platz in einem Tablespace zur Verf\"ugung steht.
          \end{itemize}
          Ob f\"ur eine Datendatei das automatische Wachstum bereits aktiviert wurde, kann mit Hilfe der View \identifier{dba\_data\_files} ermittelt werden. Das folgende Beispiel zeigt die Erstellung eines Tablespaces mit einer Datendatei, f\"ur die das automatische Wachstum aktiviert wird.
          \begin{lstlisting}[caption={Erstellen eines Tablespaces mit automatisch wachsender Datendatei},label=admin123,language=oracle_sql]
SQL> CREATE TABLESPACE auto_growing_ts
  2  DATAFILE '/u02/oradata/orcl/auto_growing_ts01.dbf' SIZE 100M
  3  AUTOEXTEND ON MAXSIZE 250M;
          \end{lstlisting}
          In \beispiel{admin123} wird das Wachstum der Datendatei auf 250 Megabyte begrenzt. Durch die Angabe von \languageorasql{MAXSIZE UNLIMITED} ist es m\"oglich, ein unbegrenztes Wachstum der Datendatei einzurichten.

          Auch im Nachhinein kann einem Tablespace eine Datendatei hinzugef\"ugt werden, die automatisch w\"achst.
          \begin{lstlisting}[caption={Hinzuf\"ugen einer Datendatei mit automatischem Wachstum},label=admin124,language=oracle_sql]
SQL> ALTER TABLESPACE auto_growing_ts
  2  ADD DATAFILE '/u03/oradata/orcl/auto_growing_ts02.dbf' SIZE 100M
  3  AUTOEXTEND ON MAXSIZE 250M;
          \end{lstlisting}
          Um das automatische Wachstum f\"ur eine Datendatei abzuschalten, muss die betroffene Datendatei mit dem \languageorasql{ALTER DATABASE}-Kommando angefasst werden.
          \begin{lstlisting}[caption={Automatisches Wachstum f\"ur eine Datendatei abschalten},label=admin125,language=oracle_sql]
SQL> ALTER DATABASE
  2  DATAFILE '/u02/oradata/orcl/auto_growing_ts02.dbf'
  3  AUTOEXTEND OFF;
          \end{lstlisting}
        \subsubsection{Eine Datendatei manuell vergr\"o\ss{}ern}
          Eine Datendatei kann mit Hilfe des \languageorasql{ALTER DATABASE}-Kommandos manuell vergr\"o\ss{}ert werden. Dadurch wird erm\"oglicht, einen Tablespace zu vergr\"o\ss{}ern, ohne neue Datendateien zur Datenbank hinzuzuf\"ugen. Dies ist dann vorteilhaft, wenn die Maximalanzahl an Datendateien f\"ur eine Datenbank fast erreicht ist. Das folgende Beispiel zeigt, wie die 100 M gro\ss{}e Datendatei \identifier{bank02.dbf} auf 350 M vergr\"o\ss{}ert wird.
          \begin{lstlisting}[caption={Eine Datendatei manuell vergr\"o\ss{}ern},label=admin126,language=oracle_sql]
SQL> ALTER DATABASE
  2  DATAFILE '/u03/oradata/orcl/bank02.dbf'
  3  RESIZE 350M;
          \end{lstlisting}
          Mit Hilfe des gleichen Statements kann eine Datendatei auch wieder verkleinert werden. Bedingung daf\"ur ist, dass der hintere Bereich der Datendatei leer ist. Die folgende Grafik veranschaulicht dies.
          \bild{Datendateien verkleinern}{shrink_datafiles}{1.75}

          Die \abbildung{shrink_datafiles} zeigt zwei Datendateien. Die Datei \identifier{bank01.dbf} kann um 150 M reduziert werden, aber \identifier{bank02.dbf} lediglich um 70 M, da belegte Datenbl\"ocke in ihr eine weitere Verkleinerung verhindern.
      \subsection{Datendateien umbenennen und verschieben}
        Datendateien k\"onnen umbenannt werden, um ihren Namen oder ihren Speicherort zu \"andern.
        \begin{merke}
          Wird eine Datendatei in Oracle umbenannt, werden nur ihre Eintr\"age in der Kontrolldatei und im Data Dictionary ge\"andert. Auf dem Datentr\"ager geschieht keine Ver\"anderung. Diese muss manuell nachgeholt werden.
        \end{merke}
        Um eine Datendatei umzubenennen gehen Sie wie folgt vor:
        \begin{enumerate}
          \item Den betreffenden Tablespace Offline setzen.
          \begin{lstlisting}[caption={Tablespace Offline setzen},label=admin127,language=oracle_sql]
SQL> ALTER TABLESPACE bank OFFLINE NORMAL;
          \end{lstlisting}
          \item Umbenennen und evtl. auch verschieben der Datendatei auf dem Datentr\"ager.
          \begin{lstlisting}[caption={Tablespace Offline setzen},label=admin127a,language=oracle_sql,alsolanguage=sqlplus]
SQL> host mv /u02/oradata/orcl/bank02.dbf /u03/oradata/orcl/bank02.dbf
          \end{lstlisting}
          \item Die Datendatei in der Datenbank umbenennen.
          \begin{lstlisting}[caption={Datendatei umbenennen},label=admin128,language=oracle_sql]
SQL> ALTER TABLESPACE bank
  2  RENAME DATAFILE '/u02/oradata/orcl/bank02.dbf'
  3  TO              '/u03/oradata/orcl/bank02.dbf';
          \end{lstlisting}
        \end{enumerate}
      \subsection{Datendateien l\"oschen}
        Datendateien k\"onnen unter der Voraussetzung gel\"oscht werden, dass sie noch unbenutzt sind. Hierf\"ur existiert die \languageorasql{DROP DATAFILE}-Klause des \languageorasql{ALTER TABLESPACE}-Kommandos. Zur Angabe der Datendatei, kann sowohl der vollst\"andige Dateiname (Pfad + Dateiname), als auch die Dateinummer verwendet werden.
        \begin{merke}
          Zum L\"oschen einer Datendatei, muss der Tablespace online sein!
        \end{merke}
        \begin{lstlisting}[caption={Eine Datendatei l\"oschen},label=admin129,language=oracle_sql]
SQL> ALTER TABLESPACE bank ONLINE;
SQL> ALTER TABLESPACE bank
  2  DROP DATAFILE '/u02/oradata/orcl/bank02.dbf';
        \end{lstlisting}
        \begin{lstlisting}[caption={Benutzen der Dateinummer zum l\"oschen einer Datendatei},label=admin130,language=oracle_sql,alsolanguage=sqlplus]
SQL> col tablespace_name format a15
SQL> col file_name format a50
SQL> col file_id format 999999
SQL> set linesize 200

SQL> SELECT   tablespace_name, file_name, file_id
  2  FROM     dba_data_files
  3  WHERE    LOWER(tablespace_name) LIKE 'bank'
  3  ORDER BY file_id;

TABLESPACE_NAME FILE_NAME                                          FILE_ID
--------------- -------------------------------------------------- -------
BANK            /u02/oradata/orcl/bank01.dbf                             6
BANK            /u03/oradata/orcl/bank02.dbf                             7

SQL> ALTER TABLESPACE bank
  2  DROP DATAFILE 6;
        \end{lstlisting}
    \section{Tempor\"are Datendateien}
      Tempor\"are Tablespaces verwenden keine gew\"ohnlichen Datendateien, sondern Tempfiles. Diese unterscheiden sich wie folgt von permanenten Datendateien:
      \begin{itemize}
        \item Sie k\"onnen nicht Read Only gesetzt werden
        \item Es wird kein Recovery f\"ur Tempfiles durchgef\"uhrt
        \item Anders als bei permanenten Datendateien hat ein Tempfile bei seiner Erstellung noch nicht seine volle Gr\"o\ss{}e.
      \end{itemize}
      \subsection{Tempfiles on- und offline setzen}
        Um die Tempfiles eines tempor\"aren Tablespaces offline zu setzen, wird das Schl\"usselwort \languageorasql{DATAFILE} durch das Schl\"usselwort \languageorasql{TEMPFILE} ersetzt. Auch f\"ur Tempfiles kann die Dateinummer verwendet werden. Diese findet man in der View \identifier{dba\_temp\_files}.

        \begin{lstlisting}[caption={Ein Tempfile mit Hilfe des Dateinamens offline setzen},label=admin132,language=oracle_sql]
SQL> ALTER DATABASE
  2  TEMPFILE '/u02/oradata/orcl/temp01.dbf' OFFLINE;
        \end{lstlisting}
        \begin{lstlisting}[caption={Offline-/Onlinesetzen eines Tempfiles mittels der Dateinummer},label=admin133,language=oracle_sql,alsolanguage=sqlplus]
SQL> col tablespace_name format a15
SQL> col file_name format a50
SQL> col file_id format 999999
SQL> set linesize 200
SQL> SELECT   tablespace_name, file_name, file_id
  2  FROM     dba_temp_files;

TABLESPACE_NAME FILE_NAME                                          FILE_ID
--------------- -------------------------------------------------- -------
TEMP            /u02/oradata/orcl/temp01.dbf                             1

SQL> ALTER DATABASE
  2  TEMPFILE 1 OFFLINE;

Database altered.

SQL> ALTER DATABASE
  2  TEMPFILE 1 ONLINE;

Database altered.
        \end{lstlisting}
      \subsection{Tempfiles zu einem tempor\"aren Tablespace hinzuf\"ugen}
        Beim tempor\"aren Tablespaces wird die \languageorasql{ADD DATAFILE}-Klausel durch \languageorasql{ADD TEMPFILE} ersetzt.
        \begin{lstlisting}[caption={Hinzuf\"ugen eines Tempfiles},label=admin134,language=oracle_sql]
SQL> ALTER TABLESPACE temp
  2  ADD TEMPFILE '/u02/oradata/orcl/temp02.dbf' SIZE 100M;
        \end{lstlisting}
      \subsection{Tempfiles l\"oschen}
        Ein unbenutztes Tempfile wird mit der \languageorasql{DROP TEMPFILE}-Klausel des \languageorasql{ALTER TABLESPACE}-Ko\-mman\-dos gel\"oscht.
        \begin{lstlisting}[caption={L\"oschen eines Tempfiles},label=admin135,language=oracle_sql]
SQL> ALTER TABLESPACE temp
  2  DROP TEMPFILE '/u02/oradata/orcl/temp02.dbf';
        \end{lstlisting}
        Auch bei Tempfiles kann statt dem Dateinamen, die Dateinummer genutzt werden.
        \begin{lstlisting}[caption={L\"oschen eines Tempfiles mit Hilfe der Dateinummer},label=admin136,language=oracle_sql]
SQL> ALTER TABLESPACE temp
  2  DROP TEMPFILE 2;
        \end{lstlisting}
    \section{Informationen}
      \subsection{Verzeichnis der relevanten Initialisierungsparameter}
        \begin{literaturinternet}
          \item \cite{REFRN10029}
        \end{literaturinternet}
      \subsection{Verzeichnis der relevanten Data Dictionary Views}
        \begin{literaturinternet}
          \item \cite{sthref1938}
          \item \cite{sthref1988}
          \item \cite{REFRN23076}
          \item \cite{sthref2435}
          \item \cite{sthref2550}
          \item \cite{sthref2555}
          \item \cite{sthref2563}
          \item \cite{sthref2579}
          \item \cite{sthref3281}
          \item \cite{sthref3785}
          \item \cite{sthref3790}
        \end{literaturinternet}
\clearpage

    \section{\"Ubungen - Datenbank Storage Strukturen verwalten}
  \begin{enumerate}
        \item F\"uhren Sie ein Recovery bei Verlust einer Redo Log Datei durch!
      \begin{enumerate}
        \item Starten Sie das Skript \oscommand{lab\_delete\_redolog\_member.sql}! Es l\"oscht einen beliebigen Member einer Ihrer Redo Log Gruppen.
          \begin{lstlisting}[language=terminal]
SQL> @/home/oracle/labs/lab_delete_redolog_member.sql
          \end{lstlisting}
        \item Die Datenbank l\"auft weiterhin normal und es liegen keinerlei Probleme vor. \"Offnen Sie die Alert Log Datei, um herauszufinden welcher Redo Log Member gel\"oscht wurde!
        \item F\"uhren Sie geeignete Ma\ss{}nahmen zur Problembehebung durch!
      \end{enumerate}


      \rule{0.94\textwidth}{0.5pt}

        \item Konfigurieren Sie das Auditing f\"ur die Tabelle \identifier{bank.mitarbeiter} so, dass erfolgreiche \languageorasql{INSERT}- und \languageorasql{UPDATE}-Statements auditiert werden! Es soll jeweils nur ein Auditeintrag pro Session f\"ur jedes  \languageorasql{INSERT}/\languageorasql{UPDATE} gemacht werden.


      \rule{0.94\textwidth}{0.5pt}

      \rule{0.94\textwidth}{0.5pt}

        \item Wechseln Sie den Undo-Tablespace zur\"uck zum Original Undo-Tablespace und l\"oschen Sie \identifier{undotbs02} und \identifier{undotbs03}.


      \rule{0.94\textwidth}{0.5pt}

        \item Pr\"ufen Sie in welchem Modus der Result Cache l\"auft (Manual oder Force)


      \rule{0.94\textwidth}{0.5pt}

        \item Pr\"ufen Sie den Audittrail auf neue Informationen! Welche T\"atigkeiten wurden an der auditierten Tabelle ausgef\"uhrt?

        \item Machen Sie alle Auditingeinstellungen r\"uckg\"angig und l\"oschen Sie den Inhalt des Auditingtrails!

        \item Schalten Sie das Auditing f\"ur alle erfolgreichen Anmeldeversuche ein! Lassen Sie diese Auditingeintr\"age im \textit{DB, Extended} Auditingtrail speichern!


      \rule{0.94\textwidth}{0.5pt}

        \item F\"uhren Sie ein Recovery mit einem Backup Controlfile durch.
      \begin{enumerate}
        \item Starten Sie das Skript \oscommand{lab\_delete\_all\_controlfiles.sql}. Es wird alle Kontrolldateien l\"oschen.
          \begin{lstlisting}[language=terminal]
SQL> @/home/oracle/labs/lab_delete_all_controlfiles.sql
          \end{lstlisting}
        \item Ergreifen Sie geeignete Ma\ss{}nahmen, um die Datenbank wieder lauff\"ahig zu machen.
      \end{enumerate}


      \rule{0.94\textwidth}{0.5pt}

        \item Legen Sie den tempor\"aren smallfile Tablespace \identifier{temp\_ts} an. Sein Tempfile  soll 500 M gro\ss{} sein, auf bis zu 4 G anwachsen k\"onnen, \oscommand{temp\_ts01.dbf} hei\ss{}en und auf Laufwerk \oscommand{/u02} liegen. Benutzen Sie 1 M gro\ss{}e Uniform-Sized Extents.

        \item L\"oschen Sie das Backup Set des \identifier{bank}-Tablespaces von der Festplatte, so dass es nur noch auf SBT verf\"ugbar ist!

        \item Der Tablespace \identifier{bank} liegt derzeit in unverschl\"usselter Form vor. Dies muss zuk\"unftig anders sein. Bereiten Sie einen Tablespace \identifier{bank\_encrypted} vor, der mittels \identifier{AES256} verschl\"usselung gesichert ist und der die gleichen Dimensionen hat, wie der Originaltablespace \identifier{bank}! Rufen Sie alle notwendigen Informationen \"uber den Tablespace \identifier{bank} aus dem Data Dictionary ab! Welche Views helfen Ihnen dabei?

\clearpage
       \item F\"uhren Sie mit Hilfe von RMAN ein Block-Media-Recovery durch, um die be\-sch\"a\-dig\-ten Bl\"ocke zu reparieren.


      \rule{0.94\textwidth}{0.5pt}

      \rule{0.94\textwidth}{0.5pt}

          \item \"Andern Sie die Sprache Ihrer Session auf Deutsch.

        \item L\"oschen Sie das Nutzerprofil \identifier{p\_clerk} in einem Arbeitsschritt!


      \rule{0.94\textwidth}{0.5pt}

        \item L\"oschen Sie den Tablespace fullts mit all seinen Datendateien.
        \item Wie gro\ss{} ist die Auslastung Ihrer FRA durch Backup Pieces (prozentual)?
        \item F\"uhren Sie zum Abschluss das Skript \oscommand{labs/lab\_dbadmin05\_cleardb.sql} aus, um die Datenbank aufzur\"aumen. Es werden alle Tablespaces gel\"oscht, die im Rahmen dieser \"Ubung angelegt werden sollten.

  \end{enumerate}
\clearpage

    \input{loesungen/dbadmin_05_storage_strukturen_verwalten_loesung}
  \input{admin/06_lokale_benutzerverwaltung}
    \section{\"Ubungen - Lokale Benutzerverwaltung}
  \begin{enumerate}
        \item F\"uhren Sie ein Recovery bei Verlust einer Redo Log Datei durch!
      \begin{enumerate}
        \item Starten Sie das Skript \oscommand{lab\_delete\_redolog\_member.sql}! Es l\"oscht einen beliebigen Member einer Ihrer Redo Log Gruppen.
          \begin{lstlisting}[language=terminal]
SQL> @/home/oracle/labs/lab_delete_redolog_member.sql
          \end{lstlisting}
        \item Die Datenbank l\"auft weiterhin normal und es liegen keinerlei Probleme vor. \"Offnen Sie die Alert Log Datei, um herauszufinden welcher Redo Log Member gel\"oscht wurde!
        \item F\"uhren Sie geeignete Ma\ss{}nahmen zur Problembehebung durch!
      \end{enumerate}

        \item Konfigurieren Sie das Auditing f\"ur die Tabelle \identifier{bank.mitarbeiter} so, dass erfolgreiche \languageorasql{INSERT}- und \languageorasql{UPDATE}-Statements auditiert werden! Es soll jeweils nur ein Auditeintrag pro Session f\"ur jedes  \languageorasql{INSERT}/\languageorasql{UPDATE} gemacht werden.


      \rule{0.94\textwidth}{0.5pt}

        \item Wechseln Sie den Undo-Tablespace zur\"uck zum Original Undo-Tablespace und l\"oschen Sie \identifier{undotbs02} und \identifier{undotbs03}.

        \item Pr\"ufen Sie in welchem Modus der Result Cache l\"auft (Manual oder Force)


      \rule{0.94\textwidth}{0.5pt}

        \item Pr\"ufen Sie den Audittrail auf neue Informationen! Welche T\"atigkeiten wurden an der auditierten Tabelle ausgef\"uhrt?

        \item Machen Sie alle Auditingeinstellungen r\"uckg\"angig und l\"oschen Sie den Inhalt des Auditingtrails!

        \item Schalten Sie das Auditing f\"ur alle erfolgreichen Anmeldeversuche ein! Lassen Sie diese Auditingeintr\"age im \textit{DB, Extended} Auditingtrail speichern!

        \item F\"uhren Sie ein Recovery mit einem Backup Controlfile durch.
      \begin{enumerate}
        \item Starten Sie das Skript \oscommand{lab\_delete\_all\_controlfiles.sql}. Es wird alle Kontrolldateien l\"oschen.
          \begin{lstlisting}[language=terminal]
SQL> @/home/oracle/labs/lab_delete_all_controlfiles.sql
          \end{lstlisting}
        \item Ergreifen Sie geeignete Ma\ss{}nahmen, um die Datenbank wieder lauff\"ahig zu machen.
      \end{enumerate}

        \item Legen Sie den tempor\"aren smallfile Tablespace \identifier{temp\_ts} an. Sein Tempfile  soll 500 M gro\ss{} sein, auf bis zu 4 G anwachsen k\"onnen, \oscommand{temp\_ts01.dbf} hei\ss{}en und auf Laufwerk \oscommand{/u02} liegen. Benutzen Sie 1 M gro\ss{}e Uniform-Sized Extents.


      \rule{0.94\textwidth}{0.5pt}

    \begin{merke}
      Sollten Sie noch Zeit haben, k\"onnen Sie sich jetzt mit den folgenden Aufgaben befassen!
    \end{merke}
\clearpage
        \item L\"oschen Sie das Backup Set des \identifier{bank}-Tablespaces von der Festplatte, so dass es nur noch auf SBT verf\"ugbar ist!

        \item Der Tablespace \identifier{bank} liegt derzeit in unverschl\"usselter Form vor. Dies muss zuk\"unftig anders sein. Bereiten Sie einen Tablespace \identifier{bank\_encrypted} vor, der mittels \identifier{AES256} verschl\"usselung gesichert ist und der die gleichen Dimensionen hat, wie der Originaltablespace \identifier{bank}! Rufen Sie alle notwendigen Informationen \"uber den Tablespace \identifier{bank} aus dem Data Dictionary ab! Welche Views helfen Ihnen dabei?

       \item F\"uhren Sie mit Hilfe von RMAN ein Block-Media-Recovery durch, um die be\-sch\"a\-dig\-ten Bl\"ocke zu reparieren.

          \item \"Andern Sie die Sprache Ihrer Session auf Deutsch.

        \item L\"oschen Sie das Nutzerprofil \identifier{p\_clerk} in einem Arbeitsschritt!

        \item L\"oschen Sie den Tablespace fullts mit all seinen Datendateien.

      \rule{0.94\textwidth}{0.5pt}

      \rule{0.94\textwidth}{0.5pt}

        \item Wie gro\ss{} ist die Auslastung Ihrer FRA durch Backup Pieces (prozentual)?
        \item F\"uhren Sie zum Abschluss das Skript \oscommand{labs/lab\_dbadmin05\_cleardb.sql} aus, um die Datenbank aufzur\"aumen. Es werden alle Tablespaces gel\"oscht, die im Rahmen dieser \"Ubung angelegt werden sollten.

        \item Erstellen Sie RMAN-Skripte, um die im Folgenden beschriebene Backup Strategie zu implementieren! Benutzen Sie f\"ur die Umsetzung der Skripte, RUN-Bl\"ocke und die manuelle Kanalanforderung!
      \begin{itemize}
        \item Jeden Sonntag muss ein komprimiertes Level 0 Backup der Datenbank erstellt werden!
        \item Von Montag bis Freitag werden im Drei-Stunden-Rhythmus Backups der Archive Logs erzeugt. Aus Sicherheitsgr\"unden m\"ussen diese Backups, an den Speicherorten \oscommand{/u02/backup} und \oscommand{/u03/backup} dupliziert werden! Alle Kopien der gesicherten Archive Logs sollen automatisch aus der FRA entfernt werden!
        \item Jeden Montag, Dienstag, Donnerstag und Freitag werden Level 1 Backups (inkrementell) der Datenbank erstellt und in der FRA gespeichert!
        \item Jeden Mittwoch muss ein kumulatives Level 1 Backup der Datenbank erzeugt werden!
        \item Jeden Samstag sind der komplette Inhalt der FRA und die Backups der Archive Logs auf das SBT-Ger\"at zu verschieben! Es ist sicherzustellen, dass im Anschluss daran, alle obsoleten Backups gel\"oscht werden!
        \item F\"ur diese Backup Strategie ist ein Recovery Window von 7 Tagen notwendig.
        \item Aktivieren Sie das Block Change Tracking, um die Dauer der inkrementellen Backups zu beschleunigen!
      \end{itemize}

  \end{enumerate}
\clearpage

    \input{loesungen/dbadmin_06_lokale_benutzerverwaltung_loesung}
  \input{admin/07_schemaobjekte_verwalten}
    \chapter{Transaktionen}
    \setcounter{page}{1}\kapitelnummer{chapter}
    \minitoc
\newpage
      Eine Transaktionen ist eine logische Arbeitseinheit, die eines oder mehrere SQL-State\-ments enth\"alt. Transaktionen sind in sich geschlossene Einheiten. Die Ergebnisse aller SQL-Statements einer Transaktionen k\"onnen entweder in die Datenbank \"ubernommen (committed) oder r\"uckg\"angig gemacht (rolled back) werden.

      Eine Transaktion beginnt implizit mit der ersten DML- oder DDL-Anweisung  und endet mit einer der Anweisungen \languageorasql{COMMIT} (\"ubernahme der Daten) oder \languageorasql{ROLLBACK} (r\"uckg\"angig machen), bzw. implizit wenn eine DDL-Anweisung abgesetzt wird (auto commit).

      Um das Konzept einer Transaktion bildlich darzustellen, stelle man sich die Datenbank eines Kreditinstitutes vor. Wenn ein Kunde Geld von seinem eigenen Konto auf ein anderes \"uberweist, geschehen drei verschiedene Dinge:
      \begin{enumerate}
        \item Kontostand des Zahlenden herabsetzen
        \item Kontostand des Zahlungsempf\"angers anpassen
        \item Die Transaktion in einem Journal dokumentieren
      \end{enumerate}
      Die Datenbank muss zwei verschiedene F\"alle abdecken k\"onnen:
      \begin{enumerate}
        \item Alle drei SQL-Statements k\"onnen erfolgreich abgesetzt werden
        \item Durch ein Problem kann mindestens eines der drei Statements nicht korrekt abgesetzt werden (falsche Kontonummer, Hardwarefehler, usw.)
      \end{enumerate}
      Im ersten Fall muss die Datenbank die \"Anderungen der Transaktion in der Datenbank speichern, damit die Bankkonten der Kunden korrekt verwaltet werden. Tritt jedoch wie in Fall zwei ein Fehler auf, muss die gesamte Transaktion zur\"uckgerollt werden.
    \section{Eigenschaften einer Transaktion (ACID)}
      Damit ein transaktionsbasiertes System, funktionieren kann, m\"ussen alle
      Transaktionen grundlegende Eigenschaften aufweisen. Diese k\"onnen durch das Akronym
      \enquote{ACID} beschrieben werden. ACID steht f\"ur \enquote{atomicity},
     \enquote{consistency}, \enquote{isolation} und \enquote{durability}. Im
     Deutschen wird statt ACID auch h\"aufig AKID verwendet.
      \begin{itemize}
        \item \textbf{atomicity} (Atomarit\"at): Eine Transaktion gilt als
        atomar, wenn Sie ganz oder gar nicht ausgef\"uhrt wird. F\"ur den
        Benutzer muss es so aussehen, als w\"are eine Transaktion eine einzelne
        elementare Anweisung, die nicht unterbrochen werden kann. Da die
        einzelnen Anweisungen, aus denen sich eine Transaktion zusammensetzt,
        tats\"achlich aber nacheinander ausgef\"uhrt werden m\"ussen, muss im
        Falle dessen, dass die Transaktion nicht vollst\"andig ausgef\"uhrt
        werden kann, jede einzelne Anweisung wieder r\"uckg\"angig gemacht
        werden.
        \item \textbf{consistency} (Konsistenz): Konsistenz bedeutet, dass sich die Datenbank nach der Aus\-f\"uh\-rung einer Transaktion in einem konsistenten Zustand befinden muss, davon ausgehend, dass die Datenbank auch vor der Transaktion schon konsistent war. F\"ur die Konsistenz einer Datenbank sorgen die Integrit\"ats Constraints, die bei der Ausf\"uhrung einer Transaktion nicht verletzt werden d\"urfen.
        \item \textbf{isolation} (Isolation): Das Prinzip der Isolation
        bedeutet, dass parallel ausgef\"uhrte Transaktionen nicht sich nicht       
        gegenseitig beeinflussen d\"urfen. Umgesetzt wird dies durch
        verschiedene Mechanismen, wie z. B.~Sperren, Zeitstempelverfahren oder,
        im Falle von Oracle, das Multiversioning.
        \item \textbf{durability} (Dauerhaftigkeit): Das Ergebnis einer
        abgeschlossenen Transaktion muss dauerhaft in der Datenbank verf\"ugbar
        sein, auch nach System\-ab\-st\"ur\-zen.
      \end{itemize}
    \section{Transaktionen und ihre Ph\"anomene}
      Die Isolation von Transaktionen ist eine der wesentlichen ACID-Eigenschaften, die an eine Transaktion gestellt werden. Fehlt die Isolation vollst\"andig oder ist diese nur mangelhaft umgesetzt, k\"onnen Probleme bei der Bearbeitung und dem Abfragen von Datens\"atzen auftreten. Diese Ph\"anomene sind im ANSI/ISO SQL-Standard (SQL92) definiert.
      \subsection{Dirty Reads}
        Gr\"unde f\"ur Dirty Reads sind:
        \begin{itemize}
          \item Das DBMS implementiert keine oder nur mangelhafte Isolation f\"ur Transaktionen.
          \item Konkurrierende Lese- und Schreibzugriffe
        \end{itemize}
        \begin{center}
          \tablecaption{Dirty Reads}
          \tablefirsthead{
            \multicolumn{2}{c}{Transaktion 1} &
            \multicolumn{2}{c}{Transaktion 2}\\
            \hline
          }
          \tablehead{}
          \tabletail{}
          \tablelasttail{}
          \begin{supertabular}{lr|ll}
            \small{\texttt{SELECT * FROM employees;}} & \scriptsize{t1} & &\\
            & & \scriptsize{t2} & \small{\texttt{UPDATE employees SET salary= 1000;}}\\
            \small{Anzeigen der Datens\"atze} & & & \\
            \small{mit einem Gehalt von 1000} & \scriptsize{t3} & & \\
            & & \scriptsize{t4} & \small{\texttt{ROLLBACK;}}\\
          \end{supertabular}
        \end{center}
        Das vorangegangene Beispiel zeigt zwei konkurrierende Transaktionen. Transaktion 1 liest die Daten der Tabelle \identifier{employees} zum Zeitpunkt t1. W\"ahrend Transaktion 1 noch liest, beginnt Transaktion 2 zum Zeitpunkt t2 diese Daten zu ver\"andern. Zum Zeitpunkt t3 erfolgt f\"ur Transaktion 1 die Ausgabe der Daten.

				Aufgrund nicht vorhandener Isolation werden auch die noch nicht best\"atigten \"Anderungen von Transaktion 2 mit ausgegeben. Zum Zeitpunkt t4 macht die Transaktion 2 ihr \languageorasql{UPDATE}-Statement wieder r\"uckg\"angig.

        Die Schlussfolgerung aus diesem Szenario zeigt, dass Transaktion 1 zum
        Zeitpunkt t3 nicht committete Daten ausgegeben hat.
        \begin{merke}
          Dieses Szenario kann in Oracle nicht durchgef\"uhrt werden!
        \end{merke}
      \subsection{Non-Repeatable Reads}
        Non-Repeatable Reads sind ein Phänomen, dass immer dann auftritt, wenn:
        \begin{itemize}
          \item das DBMS keine oder nur mangelhafte Isolation f\"ur Transaktionen implementiert
          \item zwei gleiche Lesevorg\"ange \textit{eines Datensatzes} in einer Transaktion unterschiedliche Ergebnisse liefern.
        \end{itemize}
        \begin{center}
          \tablecaption{Non-Repeatable Reads}
          \tablefirsthead{
            \multicolumn{2}{c}{Transaktion 1} &
            \multicolumn{2}{c}{Transaktion 2}\\
            \hline
          }
          \tablehead{
          }
          \tabletail{}
          \begin{supertabular}{lr|ll}
            \small{\texttt{SELECT salary FROM employees}} & & & \\
            \small{\texttt{WHERE employee\_id = 100;}} & \scriptsize{t1} & & \\
            \small{Anzeigen des Datensatzes (24000)} & \scriptsize{t2} & & \\
            & & \scriptsize{t3} & \small{\texttt{UPDATE employees SET salary = 25000}} \\
            & & & \small{\texttt{WHERE employee\_id = 100;}} \\
            & & & \texttt{COMMIT;} \\
            \cline{3-4}
            \small{\texttt{SELECT salary FROM employees}} & & & \\
            \small{\texttt{WHERE employee\_id = 100;}} & \scriptsize{t4} & & \\
            \small{Anzeigen des Datensatzes (25000)} & \scriptsize{t5} & & \\
          \end{supertabular}
        \end{center}
        Transaktion 1 f\"uhrt in diesem Beispiel zweimal den gleichen
        Lesevorgang durch. Dieser liefert zum Zeitpunkt t2 ein Gehalt von 24.000
        \$. Bei Zeitpunkt t3 ver\"andert Transaktion 2 die Basistabelle. Der
        erneute Lesevorgang liefert zum Zeitpunkt t5  ein Gehalt von
        25.000~\$. Der gleiche Lesevorgang \textit{ein und des selben
        Datensatzes} konnte also nicht zweimal mit dem gleichen Ergebnis
        durchgef\"uhrt werden, was als Non-Repeatable Read bezeichnet wird.
\clearpage
      \subsection{Phantom Reads}
        Phantom Reads treten immer dann auf, wenn:
        \begin{itemize}
          \item das DBMS nicht die h\"ochst m\"ogliche Isolation f\"ur Transaktionen implementiert und
          \item zwei gleiche Lesevorg\"ange in einer Transaktion eine \textit{unterschiedliche Menge an Ergebniszeilen} liefern. Dies bedeutet, dass zu den bereits gelesenen Zeilen neue hinzugekommen oder bestehende weggefallen sind, sich also die Menge der gelesenen Zeilen ver\"andert hat.
        \end{itemize}
        \begin{center}
          \tablecaption{Phantom Reads}
          \tablefirsthead{
            \multicolumn{2}{c}{Transaktion 1} &
            \multicolumn{2}{c}{Transaktion 2}\\
            \hline
          }
          \tabletail{}
          \begin{supertabular}{lr|ll}
            \small{\texttt{SELECT * FROM employees}} & & & \\
            \small{\texttt{WHERE department\_id = 30;}} & \scriptsize{t1} & & \\
            \small{Anzeigen der Datens\"atze: 6 St\"uck} & \scriptsize{t2} & & \\
            & & & \small{\texttt{INSERT INTO employees}} \\
            & & \scriptsize{t3} & \small{\texttt{VALUES (...);}}\\
            & & & \texttt{COMMIT;} \\
            \cline{3-4}
            \small{\texttt{SELECT * FROM employees}} & & & \\
            \small{\texttt{WHERE department\_id = 30;}} & \scriptsize{t4} & & \\
            \small{Anzeigen der Datens\"atze: 7 St\"uck} & \scriptsize{t5} & & \\
          \end{supertabular}
        \end{center}
        Transaktion 1 f\"uhrt in diesem Beispiel zweimal den gleichen Lesevorgang aus. Lesevorgang 1 liefert zum Zeitpunkt t2 6 Zeilen. Zum Zeitpunkt t3 ver\"andert Transaktion 2 diese Tabelle. Der erneute Lesevorgang von Transaktion 1 liefert jetzt, zum Zeitpunkt t5 ein ver\"andertes Ergebnis. Es sind nun 7 Zeilen. Der gleiche Lesevorgang konnte also \textit{nicht zweimal mit der gleichen Ergebnismenge} durchgef\"uhrt werden.
        \begin{merke}
          Der Unterschied zwischen Non-Repeatable Reads und Phantom Reads ist der, dass bei den Non-Repeatable Reads bestehende Datens\"atze ver\"andert werden. Dadurch kann sich auch die Ergebnismenge \"andern. Bei den Phantom Reads werden neue Datens\"atze hinzugef\"ugt oder bestehende gel\"oscht. Auch hier wird die Ergebnismenge ge\"andert, aber auf eine andere Art.
        \end{merke}
    \section{Transaktionslevel}
      Um die beschriebenen Transaktionsph\"anomene zu umgehen, wurden im ANSI/ISO SQL-Standard (SQL92) vier verschiedene Transaktionslevel festgelegt. Diese Level legen unterschiedliche Einschr\"ankungen fest, um die genannten Ph\"anomene zu unterdr\"ucken.
\clearpage
      \begin{small}
        \tablecaption{Transaktionslevel gem\"a\ss{} SQL92-Standard}
        \tablefirsthead{
          \hline
          \multicolumn{1}{|c}{\textbf{Isolationslevel}} &
          \multicolumn{1}{|c}{\textbf{Dirty Reads}} &
          \multicolumn{1}{|c}{\textbf{Non-Repeatable Reads}} &
          \multicolumn{1}{|c|}{\textbf{Phantom Reads}}\\
          \hline
        }
        \begin{supertabular}{|l|c|c|c|}
          Read Uncommitted & \textcolor{red}{Ja} & \textcolor{red}{Ja} & \textcolor{red}{Ja} \\
          \hline
          Read Committed & \textcolor{green}{Nein} & \textcolor{red}{Ja} & \textcolor{red}{Ja} \\
          \hline
          Repeatable Read & \textcolor{green}{Nein} & \textcolor{green}{Nein} & \textcolor{red}{Ja} \\
          \hline
          Serializable & \textcolor{green}{Nein} & \textcolor{green}{Nein} & \textcolor{green}{Nein}  \\
          \hline
        \end{supertabular}
      \end{small}
      \subsection{Read Uncommitted}
        Dies ist der Transaktionslevel mit den geringsten Einschr\"ankungen (es gibt n\"amlich keine). Es findet keinerlei Isolation statt, so dass eine Transaktion unbest\"atigte Informationen einer anderen Transaktion lesen kann. Dieser Level ist in Oracle nicht implementiert!
      \subsection{Read Committed}
        Diese Stufe bringt die ersten Einschr\"ankungen mit sich. Es k\"onnen nur noch best\"atigte Informationen anderer Transaktionen gelesen werden. Eine Transaktion wird also lediglich vor fehlerhaften Daten einer anderen Transaktion gesch\"utzt, da fehlerhafte Daten meist zur\"uckgerollt werden. Dieser Level ist der Standard in Oracle.
      \subsection{Repeatable Read}
        Durch eine verbesserte Isolation der Transaktionen wird in diesem Level sichergestellt, dass auch das Ph\"anomen der Non-Repeatable Reads verhindert werden kann. Dieser Level ist in Oracle nicht implementiert!
      \subsection{Serializable}
        Serializable ist der strengste Transaktionslevel. Er verhindert jegliche Trans\-akt\-ions\-ph\"a\-no\-me\-ne. Er tut dies allerdings auf Kosten der Performance, da, wie sein Name sagt, eine Serialisierung der einzelnen Transaktionen durchgef\"uhrt wird, d. h. Oracle versucht zwei parallel ausgef\"uhrte Transaktionen so auszuf\"uhren, als w\"urden sie hintereinander ausgef\"uhrt.

        Zus\"atzlich zu den Transaktionsleveln die der SQL92-Standard festgelegt hat, kennt Oracle noch einen weiteren Level, den Read Only Level. Wird eine Transaktion in diesem Level gestartet, kann sie nur Abfragen, aber kein DML Statement durchf\"uhren. Er hat die gleichen Auswirkungen wie der Level Serializable und ist f\"ur sehr lange laufende Abfragen gedacht, die ein hohes Ma\ss{} an Lesekonsistenz ben\"otigen.

        \bild{Se\-ri\-a\-li\-sie\-rung von Trans\-ak\-tio\-nen}{serializable}{1.2}

    \section{Transaktionssteuerung}
      \subsection{Eine Transaktion starten}
        Eine Transaktion kann auf zwei verschiedene Arten gestartet werden: implizit oder explizit. Implizit wird eine Transaktion durch ein beliebiges DML-Kommando gestartet. D. h. eine Transaktion beginnt implizit, sobald der Nutzer ein DML-Statement abgesetzt hat.

        Um das Transaktionslevel einer Transaktion zu setzen, kann das
        \languageorasql{SET TRANSACTION ISOLATION LEVEL}-Kommandos benutzt
        werden. Die drei m\"oglichen Isolationslevel werden wie folgt gesetzt:
        \begin{lstlisting}[caption={Isolationslevel einer Transaktion
        w\"ahlen},label=admin401,language=oracle_sql]
SET TRANSACTION ISOLATION LEVEL READ COMMITTED;

SET TRANSACTION ISOLATION LEVEL SERIALIZABLE;

SET TRANSACTION READ ONLY;
        \end{lstlisting}
        Soll das Transaktionslevel für eine gesamte Session geändert werden,
        geschieht dies mit \languageorasql{ALTER SESSION}.
\clearpage
        \begin{lstlisting}[caption={Isolationslevel einer Session w\"ahlen},label=admin402,language=oracle_sql]
ALTER SESSION SET isolation_level = READ COMMITTED;

ALTER SESSION SET isolation_level = SERIALIZABLE;
        \end{lstlisting}
      \subsection{Eine Transaktion beenden}
        Eine Transaktion kann auf unterschiedliche Art und Weise beendet werden:
        \begin{itemize}
          \item Durch das Kommando \languageorasql{COMMIT}.

          Ein \languageorasql{COMMIT} sorgt daf\"ur, dass eine Transaktion beendet und ihre \"Anderungen in der Datenbank dauerhaft gemacht werden. Ein \languageorasql{COMMIT} kann nur dann erfolgreich sein, wenn keine Verletzung der Datenkonsistenz vorliegt.
          \item Durch das Kommando \languageorasql{ROLLBACK}.

          Mit \languageorasql{ROLLBACK} werden alle \"Anderungen, die eine Transaktion an einer Datenbank vorgenommen hat, r\"uckg\"angig gemacht. Die Datenbank wird in den letzten konsistenten Zustand zur\"uckversetzt.
          \item Durch das Abbrechen einer Session.

          Wird eine Session unerwartet abgebrochen, werden alle offnen Transaktionen der Session beendet. Es erfolgt kein \languageorasql{COMMIT}.
        \end{itemize}
      \subsection{Transaktionsteile zur\"uckrollen}
        Innerhalb einer Transaktion k\"onnen Marken gesetzt werden, die als Savepoints bezeichnet werden. Dadurch wird eine Transaktion in einzelne Teile zerlegt. Der Nutzer hat so die M\"oglichkeit eine Transaktion nur teilweise, bis zu einem bestimmten Savepoint, zur\"uckzurollen. Dies kann bei langen und komplexen Transaktion sehr n\"utzlich sein.

        Wird ein Rollback zu einem Savepoint durchgef\"uhrt, hebt Oracle nur die Sperren auf, die f\"ur die zur\"uckgerollten Statements notwendig waren. Die Transaktion bleibt, trotz der teilweisen Rollbacks, erhalten. Andere Transaktionen, die Zugriff auf die bisher gesperrten Daten ben\"otigen, k\"onnen dann mit ihrer Arbeit fortfahren.

        \begin{lstlisting}[caption={Einen Savepoint setzen},label=admin403,language=oracle_sql]
UPDATE departments
SET    department_ID = 11
WHERE  department_ID = 10;

SAVEPOINT dept;
        \end{lstlisting}
        \begin{lstlisting}[caption={Rollback zu einem Savepoint},label=admin404,language=oracle_sql]
        
UPDATE departments
SET    department_ID = 21
WHERE  department_ID = 20;

SAVEPOINT dept2;

ROLLBACK TO SAVEPOINT dept;
        \end{lstlisting}
        \begin{literaturinternet}
          \item \cite[Konkurrierende Zugriffe und Datenkonsistenz]{dataconcurrencyandconsistency}.
          \item \cite[Transaktionsverwaltung]{transactionmanagement}.
        \end{literaturinternet}
      \subsection{Deadlocks}
        Ein Deadlock tritt immer dann auf, wenn zwei oder mehr Nutzer Sperren auf ein und die selbe Ressource legen m\"ochten. Deadlocks verhindern, dass die betreffenden Transaktionen weiterarbeiten k\"onnen. Diese Situation wird deshalb als Deadlock bezeichnet, weil es egal ist, wie lange jede Transaktion warten w\"urde, da jeder auf den anderen wartet.

        Oracle kann automatisch Deadlock-Situationen erkennen und aufl\"osen. Dies geschieht, in dem die Transaktion, die den Deadlock bemerkt auf Statementebene zur\"uckgerollt wird (Statement-level rollback). So werden die betreffenden Sperren freigegeben und die andere Transaktion kann weiter arbeiten.

        \begin{literaturinternet}
          \item \cite[Wie Oracle Daten sperrt]{howoraclelocksdata}
        \end{literaturinternet}
    \section{Multiversion Concurrency Control}
      Multiversion Concurrency Control ist ein Mechanismus, den Oracle zur Ge\-w\"ahr\-leis\-tung der Lesekonsistenz nutzt. Dabei werden verschiedene Versionen von Datenbankobjekten aufbewahrt, so dass jeder Nutzer die ben\"otigte Sicht seiner Daten bekommt. Die Umsetzung dieses Verfahrens wird durch verschiedene Methoden, wie z. B. Zeitstempelverfahren oder Snapshots erreicht. Oracle benutzt hierf\"ur sogenannte Before Images seiner Datenbl\"ocke.
\clearpage
      \subsection{Lesekonsistenz}
        \subsubsection{Lesekonsistenz auf Statementebene (Statementlevel Read Consistency)}
          \begin{merke}
            Unter dem Begriff \enquote{Statementlevel Read Consistency} versteht man, dass eine Abfrage nur die Daten sieht, die zum Startzeitpunkt der Abfrage g\"ultig (committed) waren. Die L\"ange der Laufzeit der Abfrage darf dabei keine Rolle spielen.
          \end{merke}
          \tablecaption{Statementlevel Read Consistency}
          \tablefirsthead{
            \multicolumn{2}{c}{Transaktion 1} &
            \multicolumn{2}{c}{Transaktion 2}\\
            \hline
          }
          \tabletail{}
          \begin{supertabular}{lr|ll}
            \label{statementlevelreadconsistency}
            \small{\texttt{SELECT * FROM employees}} & & & \\
            \small{\texttt{WHERE department\_id = 30;}} & \scriptsize{t1} & & \\
            & & & \small{\texttt{DELETE employees}} \\
            & & \scriptsize{t2} & \small{\texttt{WHERE employee\_id = 117;}}\\
            & & & \texttt{COMMIT;} \\
            \cline{3-4}
            \small{Anzeigen der Datens\"atze: 6 St\"uck} & \scriptsize{t3} & & \\
            \small{\texttt{SELECT * FROM employees}} & & & \\
            \small{\texttt{WHERE department\_id = 30;}} & \scriptsize{t4} & & \\
            \small{Anzeigen der Datens\"atze: 5 St\"uck} & \scriptsize{t5} & & \\
          \end{supertabular}

          Das \beispiel{statementlevelreadconsistency} zeigt die beiden Transaktionen 1 und 2. Transaktion 1 startet zum Zeitpunkt t1. Unmittelbar nach dem Start von Transaktion 1 startet Transaktion 2 zum Zeitpunkt t2. Der Delete-Vorgang von Transaktion 2 darf die Abfrage von Transaktion 1 nicht beeinflussen und die Ausgabe der Abfrageergebnisse zum Zeitpunkt t3 zeigt auch korrekte 6 Datens\"atze an. Erst die zum Zeitpunkt t4 gestartete Abfrage in Transaktion 1 registriert die durch Transaktion 2 vorgenommenen \"Anderungen. Die Lesekonsistenz auf Statementebene ist gew\"ahrleistet.
        \subsubsection{Lesekonsistenz auf Transaktionsebene (Transactionlevel Read Consistency)}
          \begin{merke}
            Unter dem Begriff \enquote{Transactionlevel Read Consistency} versteht man, dass eine Abfrage nur die Daten sieht, die zum Startzeitpunkt der Transaktion g\"ultig (committed) waren. Die L\"ange der Laufzeit der Transaktion darf dabei keine Rolle spielen. Um Transactionlevel Read Consistency zu erwirken, muss das Isolationslevel Serializable verwendet werden.
          \end{merke}
\clearpage
          Das \beispiel{transactionlevelreadconsistency} zeigt die beiden Transaktionen 1 und 2. Transaktion 1 startet zum Zeitpunkt t1. Unmittelbar nach dem Start von Transaktion 1 startet Transaktion 2 zum Zeitpunkt t2. Der Delete-Vorgang von Transaktion 2 darf die gesamte Transaktion 1 nicht beeinflussen und die beiden Ausgaben der Abfrageergebnisse, zu den Zeitpunkten t4 und t6, zeigen auch korrekte 6 Datens\"atze an. Die Lesekonsistenz auf Transaktionsebene ist gew\"ahrleistet.
          \tablecaption{Transactionlevel Read Consistency}
          \tablefirsthead{
            \multicolumn{2}{c}{Transaktion 1} &
            \multicolumn{2}{c}{Transaktion 2}\\
            \hline
          }
          \tabletail{}
          \begin{supertabular}{lr|ll}
            \label{transactionlevelreadconsistency}
            \small{\texttt{SET TRANSACTION ISOLATION LEVEL}} & & & \\
            \small{\texttt{SERIALIZABLE;}} & \scriptsize{t1} & & \\
            \small{\texttt{SELECT * FROM employees}} & & & \\
            \small{\texttt{WHERE department\_id = 30;}} & \scriptsize{t2} & & \\
            & & & \small{\texttt{DELETE employees}} \\
            & & \scriptsize{t3} & \small{\texttt{WHERE employee\_id = 117;}}\\
            & & & \texttt{COMMIT;} \\
            \cline{3-4}
            \small{Anzeigen der Datens\"atze: 6 St\"uck} & \scriptsize{t4} & & \\
            \small{\texttt{SELECT * FROM employees}} & & & \\
            \small{\texttt{WHERE department\_id = 30;}} & \scriptsize{t5} & & \\
            \small{Anzeigen der Datens\"atze: 6 St\"uck} & \scriptsize{t6} & & \\
          \end{supertabular}

      \subsection{Undo-Segmente}
        Um die Lesekonsistenz f\"ur Abfragen zu gew\"ahrleisten, benutzt Oracle Undo-Segmente. Undo-Segmente sind spezielle Segmente, die anders als z. B.~Tabellensegmente oder Clustersegmente, nicht direkt durch den Nutzer bearbeitet werden k\"onnen. Seit Oracle 9i sind alle Undo-Segmente in einem Undo-Tablespace zusammengefasst, der nur minimale Administration ben\"otigt.

        Undo-Segmente werden von der Datenbank genutzt, um Before Images von Datenbl\"ocken zu speichern.
        \subsubsection{Before Images}
          Unter dem Begriff Before Image versteht Oracle eine Kopie eines Oracle blocks, bevor dieser ge\"andert wird. Ein Before Image kann im weitesten Sinne als eine \enquote{Kopie der Originalwerte} verstanden werden.
\clearpage
          Before Images werden f\"ur verschiedene Zwecke ben\"otigt. Dies sind im wesentlichen:
          \begin{itemize}
            \item Zur\"uckrollen einer Transaktion, wenn ein \languageorasql{ROLLBACK}-Kommando ausgef\"uhrt werden soll
            \item Recovery der Datenbank
            \item Lesekonsistenz
            \item Oracle Flashback Query
            \item Oracle Flashback Table
          \end{itemize}
          Wird ein \languageorasql{ROLLBACK}-Kommando am Ende einer Transaktion abgesetzt, m\"ussen alle durch die Transaktion verursachten \"Anderungen r\"uckg\"angig gemacht werden. Hierzu werden die Originalwerte aus den Before Images, die in den Undo-Segmente liegen, benutzt.

          Zu beachten ist, dass es mehrere Before Images eines Oracle blocks, mit unterschiedlichen Versionsst\"anden geben kann. Da jeder Oracle block im Database Buffer Cache mit einer SCN versehen wird, kann anhand dieser das Alter des Blocks (und somit seine Version) bestimmt werden. Je h\"oher die SCN, desto neuer ist der Block.

          \subsubsection{Multiversion Concurrency Control durch Before Images}
            \abbildung{readconsistency} zeigt die Nutzung der Before Images f\"ur das Erzeugen von Lesekonsistenz.

            \bild{Lesekonsistenz durch Multiversion Concurrency Control}{readconsistency}{0.7}

            Eine Transaktion wird bei SCN 4711 gestartet. Um Lesekonsistenz zu gew\"ahrleisten, muss die Datenbank daf\"ur sorgen, dass diese Transaktion nur solche Datenbl\"ocke liest, deren SCN kleinergleich 4711 lautet. Ein Oracle block tr\"agt bereits die SCN 4866. F\"ur diesen muss Oracle ein Before Image, mit einer SCN kleiner oder gleich 4711, aus den Undo-Segmenten holen.

    \chapter{Undo-Daten verwalten}
    \setcounter{page}{1}\kapitelnummer{chapter}
    \label{undodata}
    \minitoc
\newpage
    \section{Undo-Tablespaces verwalten}
      \subsection{Automatic Undo Management aktivieren}
        Oracle ist in der Lage, Undo-Daten voll automatisiert zu verwalten. Der
        dazu verwendete Mechanismus hei\ss{}t \enquote{Automatic Undo
        Management}. Wird er benutzt, muss nur ein Undo-Tablespace erstellt
        werden und Oracle k\"ummert sich um den Rest.

        Zur Aktivierung dieses Features, muss der Parameter \parameter{undo\_management} auf den Wert \enquote{auto} gesetzt werden. Dadurch wird bei der Erstellung der Datenbank automatisch ein \enquote{Default Undo-Tablespace} angelegt. Ebenfalls werden auch alle Initialisierungsparameter, die mit manuellem Undo-Management zu tun haben ignoriert.
        \begin{merke}
          Manuelles Undo Management ist zwar noch immer m\"oglich, wird aber von Oracle keinesfalls mehr empfohlen!
        \end{merke}
        Ist der Undo-Tablespace nicht verf\"ugbar, kann die Datenbank nicht
        geöffnet werden. Tritt dieser Fall ein, wird ein Eintrag in der Alert
        log Datei gemacht.
        \begin{lstlisting}[caption={Fehlermeldung bei nicht vorhandenem
        Undo-Tablespace},label=admin501,language=terminal]
Mon Sep 23 15:52:19 2007 
Errors in file /u01/app/oracle/diag/rdbms/orcl/orcl/trace/orcl_ora_544.trc:
ORA-30012: Undo Tablespace 'UNDOTBS01' ist nicht vorhanden oder hat den falschen
Typ
Mon Sep 23 15:52:19 2007
Error 30012 happened during db &open&, shutting down database
&user&: terminating instance due to error 30012
Mon Sep 23 15:52:19 2007
Error in file /u01/app/oracle/diag/rdbms/orcl/orcl/trace/orcl/orcl_pmon_3000.trc
ORA-30012: undo tablespace '' does not exist or of wrong type
        \end{lstlisting}
        Der Parameter \parameter{undo\_tablespace} legt den Undo-Tablespace der Instanz fest.
        \begin{lstlisting}[caption={Der Parameter \parameter{undo\_management}},label=admin502,language=oracle_sql]
UNDO_TABLESPACE = undotbs_01
        \end{lstlisting}
      \subsection{Einen Undo-Tablespace erstellen}
        Es gibt zwei M\"oglichkeiten einen Undo-Tablespace zu erstellen:
        \begin{itemize}
          \item Durch das \languageorasql{CREATE DATABASE}-Kommando, beim Anlegen der Datenbank.
          \item Mit dem \languageorasql{CREATE UNDO TABLESPACE}-Statement
        \end{itemize}
        Wird ein Undo-Tablespace mit dem \languageorasql{CREATE UNDO TABLESPACE}-Statement erstellt, sind die beiden einzigen Parameter die dabei ge\"andert werden k\"onnen:
        \begin{itemize}
          \item die Datafile-Klausel
          \item die Extent-Management-Klausel
        \end{itemize}
        Im folgenden Beispiel wird ein automatisch wachsender Undo-Tablespace erstellt.
        \begin{lstlisting}[caption={Undo-Tablespace erstellen},label=admin503,language=oracle_sql]
SQL> CREATE UNDO TABLESPACE undotbs02
  2  DATAFILE '/u02/oradata/orcl/undotbs02_01.dbf' SIZE 10M
  3  AUTOEXTEND ON;
        \end{lstlisting}
        \begin{merke}
          Es k\"onnen mehrere Undo-Tablespaces erstellt werden, jedoch ist immer
          nur einer aktiv.
        \end{merke}
      \subsection{Einen Undo-Tablespace ver\"andern}
        Undo-Tablespaces werden mit dem Kommando \languageorasql{ALTER TABLESPACE} ver\"andert. Da die meisten Parameter f\"ur einen Undo-Tablespace direkt vom System verwaltet werden, muss sich der DBA nur um folgende Dinge k\"ummern:
        \begin{itemize}
          \item Datendateien zum Tablespace hinzuf\"ugen
          \item Datendateien des Tablespaces umbenennen
          \item Datendateien des Tablespaces on-/offline setzen
          \item (De-)Aktivieren der Retention Guarantee
        \end{itemize}
        Andere Parameter k\"onnen nicht durch den Administrator ver\"andert werden.

        Um zu verhindern, dass der Platz in einem Undo-Tablespace nicht mehr ausreicht, kann eine Datendatei hinzugef\"ugt werden.
          \begin{lstlisting}[caption={Datendatei zum Undo-Tablespace hinzuf\"ugen},label=admin504,language=oracle_sql]
SQL> ALTER TABLESPACE undotbs02
  2  ADD DATAFILE '/u01/app/oracle/oradata/orcl/undotbs02_02.dbf' SIZE 100M
  3  AUTOEXTEND ON MAXSIZE 5G;
          \end{lstlisting}
          Eine andere M\"oglichkeit besteht darin, eine bestehende Datendatei mit dem \languageorasql{ALTER DATABASE}-Kommando in ihrer Gr\"o\ss{}e zu ver\"andern.
          \begin{lstlisting}[caption={Datendatei in ihrer Gr\"o\ss{}e ver\"andern},label=admin505,language=oracle_sql]
SQL> ALTER DATABASE
  2  DATAFILE '/u01/app/oracle/oradata/orcl/undotbs02_02.dbf'
  3  RESIZE 250M;
          \end{lstlisting}
      \subsection{Einen Undo-Tablespace l\"oschen}
        Um einen Undo-Tablespace zu l\"oschen, wird das \languageorasql{DROP TABLESPACE}-Kommando verwendet.
        \begin{lstlisting}[caption={Undo-Tablespace l\"oschen},label=admin506,language=oracle_sql]
SQL> DROP TABLESPACE undotbs02;
        \end{lstlisting}
        \begin{merke}
          Ein Undo-Tablespace kann nur gel\"oscht werden, wenn keine offenen Transaktionen ihn mehr verwenden. Zu beachten ist dabei, dass ein Undo-Tablespace, trotz der Tatsache, dass als unexpired markierte Undo-Records existieren, gel\"oscht werden kann. Daher ist beim L\"oschen von Undo-Tablespaces gr\"o\ss{}te Vorsicht geboten.
        \end{merke}
      \subsection{Den Undo-Tablespace wechseln}
        Da der \parameter{undo\_tablespace}-Initialisierungsparameter dynamisch ist, ist es m\"oglich den Undo-Tablespace im laufenden Betrieb zu wechseln.
        \begin{lstlisting}[caption={Undo-Tablespace wechseln},label=admin507,language=oracle_sql]
SQL> ALTER SYSTEM SET undo_tablespace = undotbs02;
        \end{lstlisting}
        \subsubsection{Auswirkungen des Wechsels}
          Hat der Wechsel funktioniert, wird der neue Undo-Tablespace, ohne f\"ur die Nutzer sp\"urbare \"Anderungen, verwendet.

          Ein Fehlschlagen des Wechsels kann aus folgenden Gr\"unden geschehen:
          \begin{itemize}
            \item Der neue Tablespace existiert nicht
            \item Der neue Tablespace ist kein Undo-Tablespace
            \item Der neue Undo-Tablespace wird bereits von einer anderen Instanz benutzt (RAC)
          \end{itemize}
          Wenn nach dem Wechsel neue Transaktionen durch Nutzer gestartet werden, werden die Undo-Daten hierf\"ur im neuen Undo-Tablespace verwaltet. Der alte Undo-Tablespace bekommt den Status \enquote{pending offline}. Transaktionen, die vor dem Wechsel bereits bestanden, werden noch aus dem alten Undo-Tablespace bedient, es k\"onnen jedoch keine neuen mehr hinzukommen.

          W\"ahrend ein Undo-Tablespace im pending offline-Modus ist, kann er nicht gel\"oscht werden. Erst wenn alle diesen Undo-Tablespace betreffenden Transaktionen beendet wurden, wird der Status des Tablespaces von pending offline auf offline gesetzt. Jetzt kann dieser Undo-Tablespace entweder einer anderen Instanz zur Verf\"ugung gestellt oder gel\"oscht werden.
    \section{Undo Retention}
      \label{undoretention}
      Durch das Ausf\"uhren von Transaktionen sammeln sich Undo-Daten an, die f\"ur das Zur\"uckrollen der Transaktionen oder f\"ur Recoveryzwecke gebraucht werden. Auch nach dem Abschluss einer Transaktion ist es notwendig, diese Daten weiterhin vorzuhalten, da sie zur Aufrechterhaltung der Lesekonsistenz anderer Transaktionen wichtig sind.

      Die Undo Retention bestimmt, wie lange Undo-Daten im Undo-Tablespace verweilen. Je nach dem, wie die Datenbank genutzt wird (viele Schreibvorg\"ange, lange Lesevorg\"ange, usw.) ist es wichtig diesen Wert h\"oher oder niedriger einzustellen. Mit der Aktivierung der automatischen Undo-Verwaltung wird auch ein Standardwert von 900 Sekunden f\"ur die Undo Retention gesetzt.      Abh\"angig von der Undo Retention und davon, wie lange ein Datenblock bereits im Undo-Tablespace verweilt, kann er zwei verschiedene Markierungen erhalten:
      \begin{itemize}
        \item \textbf{expired}: Alte Undo-Daten, deren Transaktionen bereits best\"atigt wurden und deren Alter gr\"o\ss{}er ist, als die aktuelle Undo Retention, werden als expired\footnote{expired = engl. abgelaufen} markiert.
        \item \textbf{unexpired}: Undo-Daten, deren Transaktionen bereits best\"atigt wurden, die aber noch j\"unger sind, als die aktuelle Undo Retention, werden als unexpired\footnote{unexpired = engl. noch nicht abgelaufen} markiert.
      \end{itemize}

      Die Undo Retention wird durch Oracle automatisch, basierend auf der Gr\"o\ss{}e des Undo-Tablespace und der aktuellen Aktivit\"aten im System, richtig gesetzt. Eine manuelle Einstellung der Undo Retention kann \"uber den Parameter \parameter{undo\_retention} geschehen. Die Angabe erfolgt in Sekunden. Vorausgesetzt der Undo-Tablespace ist gro\ss{} genug, h\"alt die Datenbank diese Vorgabe ein.

      Wird der Platz f\"ur neue Transaktionen zu knapp, werden als expired markierte Undo-Daten \"uberschrieben. Sollte dann immer noch nicht genug Platz vorhanden sein, beginnt die Datenbank, die als unexpired markierten Daten zu \"uberschreiben. Dies kann dazu f\"uhren, dass die \"uberschriebenen Daten, noch von einer lang laufenden Abfrage ben\"otigt w\"urden, weshalb die Datenbank die Abfrage mit der Fehlermeldung \enquote{Snapshot too old} abbricht.

      Die beiden folgenden Punkte beschreiben noch einmal kurz, wie sich die Undo Retention auf die Lesekonsistenz der Datenbank auswirkt.
\clearpage
			\bild{Die Fehler\-meldung Snapshot too old}{snapshot_too_old}{0.2}
			\begin{itemize}
        \item Hat der Undo-Tablespace eine vordefinierte Gr\"o\ss{}e und kann nicht wachsen (autoextend wurde nicht benutzt), wird der Parameter \parameter{undo\_retention} ignoriert, sobald der Platz im Undo-Tablespace zu knapp wird. In dieser Situation werden als unexpired markierte Undo-Daten \"uberschrieben.
        \item Kann der Undo-Tablespace wachsen, versucht die Datenbank sich an den Vorgabewert f\"ur die Undo Retention zu halten. Wird der Platz im Undo-Tablespace zu knapp wird der Tablespace vergr\"o\ss{}ert, bis eine evtl. definierte Maximalgr\"o\ss{}e erreicht wird. Danach werden unter Umst\"anden wieder als unexpired markierte Undo-Records gel\"oscht.
      \end{itemize}
      \subsection{Automatisches Tuning der Undo Retention}
        Wie bereits erw\"ahnt, versucht Oracle den Wert f\"ur die Undo Retention automatisch zu tunen. Dies geschieht nach den folgenden Vorgaben:
        \begin{itemize}
          \item Hat der Undo-Tablespace eine feste Gr\"o\ss{}e, tuned Oracle die Undo Retention so, das sie f\"ur die Gr\"o\ss{}e des Undo-Tablespaces und die aktuelle Transaktionslast optimal ist.
          \item Wurde der Undo-Tablespace so konfiguriert, das er wachsen kann, versucht Oracle die Undo Retention so zu tunen, das sie gro\ss\ genug f\"ur die bisher gr\"o\ss{}te gelaufene Abfrage ist.
        \end{itemize}
\clearpage
				Die aktuelle \textit{Tuned Undo Retention} kann in der dynamischen Performance View\\ \identifier{v\$undostat} in 10-Minuten-Intervallen \"uber die letzten 4 Tage hinweg abgefragt werden. Der Wert wird dort in Sekunden angegeben. Soll die Tuned Undo Retention \"uber einen l\"angeren Zeitraum, als 4 Tage, hinweg beobachtet werden, kann hierzu die View \identifier{dba\_hist\_undostat} verwendet werden.
        \begin{lstlisting}[caption={Die Tuned Undo Retention abfragen},label=admin508,language=oracle_sql]
SQL> SELECT   TO_CHAR(begin_time, 'DD.MM.YY HH24:MI') AS "Begin Time",
  2           TO_CHAR(end_time,   'DD.MM.YY HH24:MI') AS "End Time",
  3           tuned_undoretention
  4  FROM     v$undostat
  5  ORDER BY end_time;
        \end{lstlisting}
      \bild{Undo Retention}{undo_retention}{0.58}

      \subsection{Retention Guarantee}
        Um den Erfolg von besonders lang laufenden Abfragen garantieren zu k\"onnen, bietet Oracle die \enquote{Retention Guarantee}. Sie wird beim Anlegen des Undo-Tablespaces durch die Klausel \languageorasql{RETENTION GUARANTEE} aktiviert. Ist dies der Fall, werden die als unexpired markierten Undo-Records solange gespeichert, wie sie ben\"otigt werden, um die Lesekonsistenz f\"ur alle laufenden Abfragen aufrecht zu erhalten.
\clearpage
        Die Aufrechterhaltung der Lesekonsistenz geschieht jedoch nicht ohne Kosten. Durch die Retention Guarantee ist es m\"oglich, dass f\"ur DML-Statements bzw. deren Transaktionen nicht mehr gen\"ugend Speicher im Undo-Tablespace vorhanden ist. Eine solche Transaktion muss dann abgebrochen und zur\"uckgerollt werden. Aus diesem Grund wird von Oracle empfohlen, diesen Mechanismus mit Bedacht zu benutzen.

        Um die Retention Guarantee zu deaktivieren gibt es das \languageorasql{ALTER TABLESPACE}-Statement zusammen mit der \languageorasql{RETENTION NOGUARANTEE}-Klausel.
        \begin{lstlisting}[caption={(De-)aktivieren der Retention Guarantee},label=admin509,language=oracle_sql]
SQL> ALTER TABLESPACE undotbs02 RETENTION GUARANTEE;

SQL> ALTER TABLESPACE undotbs02 RETENTION NOGUARANTEE;
        \end{lstlisting}
        Ob f\"ur den Undo-Tablespace die Retention Guarantee aktiviert wurde, kann der View \identifier{dba\_tablespaces} entnommen werden.
        \begin{lstlisting}[caption={Den Status der Retention Guarantee abfragen},label=admin510,language=oracle_sql]
SQL> SELECT tablespace_name, retention
  2  FROM   dba_tablespaces
  3  WHERE  contents = 'UNDO';

TABLESPACE_NAME                &RETENTION&
------------------------------ -----------
UNDOTBS1                       &NOGUARANTEE&
        \end{lstlisting}
    \section{Informationen}
      \subsection{Verzeichnis der relevanten Initialisierungsparameter}
        \begin{literaturinternet}
          \item \cite{REFRN10224}
          \item \cite{REFRN10225}
          \item \cite{REFRN10227}
        \end{literaturinternet}
\clearpage
      \subsection{Verzeichnis der relevanten Data Dictionary Views}
        \begin{literaturinternet}
          \item \cite{sthref3583}
          \item \cite{sthref3800}
          \item \cite{sthref2575}
          \item \cite{sthref3804}
          \item \cite{REFRN23460}
        \end{literaturinternet}
\clearpage

    \section{\"Ubungen - Undo-Daten verwalten}
  \begin{enumerate}
        \item F\"uhren Sie ein Recovery bei Verlust einer Redo Log Datei durch!
      \begin{enumerate}
        \item Starten Sie das Skript \oscommand{lab\_delete\_redolog\_member.sql}! Es l\"oscht einen beliebigen Member einer Ihrer Redo Log Gruppen.
          \begin{lstlisting}[language=terminal]
SQL> @/home/oracle/labs/lab_delete_redolog_member.sql
          \end{lstlisting}
        \item Die Datenbank l\"auft weiterhin normal und es liegen keinerlei Probleme vor. \"Offnen Sie die Alert Log Datei, um herauszufinden welcher Redo Log Member gel\"oscht wurde!
        \item F\"uhren Sie geeignete Ma\ss{}nahmen zur Problembehebung durch!
      \end{enumerate}


        \item Konfigurieren Sie das Auditing f\"ur die Tabelle \identifier{bank.mitarbeiter} so, dass erfolgreiche \languageorasql{INSERT}- und \languageorasql{UPDATE}-Statements auditiert werden! Es soll jeweils nur ein Auditeintrag pro Session f\"ur jedes  \languageorasql{INSERT}/\languageorasql{UPDATE} gemacht werden.

    \begin{enumerate}
              \item Vergr\"o\ss ern Sie \identifier{undotbs1} auf 200 M.


              \item \"Offnen Sie sich ein zweites Terminalfenster und starten Sie in diesem Fenster SQL*Plus als sysdba.


      \input{uebungen/dbadmin_09/item_b3}

              \item Warten Sie, bis f\"ur alle f\"unf Transaktionen die Meldung \textit{Connected.} angezeigt wird. Dies kann ein paar Sekunden (ca. 20) dauern.


      \input{uebungen/dbadmin_09/item_b5}

              \item Pr\"ufen Sie den Status des Original-Undo-Tablespace
        \identifier{undotbs1}. Wann findet der Wechsel von \identifier{undotbs1}
        zu \identifier{undotbs03} tats\"achlich statt? F\"uhren Sie hierzu die
        folgende Abfrage mehrfach in ca. 30 Sekunden Abst\"anden aus:

        \begin{lstlisting}[language=oracle_sql]
SQL> SELECT usn, status
  2  FROM   v$rollstat;
        \end{lstlisting}

    \end{enumerate}
        \item Wechseln Sie den Undo-Tablespace zur\"uck zum Original Undo-Tablespace und l\"oschen Sie \identifier{undotbs02} und \identifier{undotbs03}.

  \end{enumerate}
\clearpage
    \input{loesungen/dbadmin_09_undo-daten_verwalten_loesung}
    \chapter{Konfigurieren der Oracle-Netzwerkumgebung}
    \setcounter{page}{1}\kapitelnummer{chapter}
    \minitoc
\newpage
    \section{Der Listener}
      Der Listener ist ein eigenst\"andiger, auf dem Datenbankserver laufender
      Prozess. Er empf\"angt eingehende Verbindungsanforderungen der Clients und
      leitet den Verbindungsaufbau zur Datenbank ein. Er speichert seine
      Konfiguration in der Datei \oscommand{listener.ora}.

      Da es f\"ur alle verbindungsspezifischen Parameter Standardwerte gibt, ist es m\"oglich einen Listener ohne eine Konfiguration zu starten. Dieser Default-Listener tr\"agt den Namen \enquote{LISTENER} und hat keine Informationen \"uber vorhandene Instanzen zur Verf\"ugung. Die Datenbankinstanzen m\"ussen sich dynamisch beim Listener registrieren (PMON, Service registration). Daraus ergeben sich folgende Vorteile:
      \begin{itemize}
        \item \textbf{Vereinfachte Konfiguration}: Es sind keine zus\"atzlichen Konfigurationseinstellungen f\"ur dieses Feature notwendig.
        \item \textbf{Connect-time failover}: Da bei dynamischer Registrierung einer Instanz dem Listener auch der Status der Instanz (started, shutdown) mitgegeben wird, kann er im Falle dessen, dass eine Instanz ausf\"allt, den Nutzerprozess an eine andere Instanz der gleichen Datenbank (RAC) verweisen. Dies geschieht transparent f\"ur den Nutzer. Bei der statischen Registrierung einer Instanz ist kein Connect-time failover m\"og\-lich.
        \item \textbf{Runtime connection load balancing}: Service registration erm\"oglicht dem Listener die Verbindungsanforderungen der Clients immer an den am schw\"achsten ausgelasteten Service handler (Dispatcher, Dedicated Server Prozess) weiterzuleiten.
      \end{itemize}
      \subsection{Kommunikation zwischen Client und Server}
        F\"ur die Kommunikation zwischen Client und Datenbankserver verwendet Oracle eine Architektur, die auf dem ISO/OSI Modell basiert.
        In \abbildung{network_architecture} wird sowohl der client- als auch der serverseitige Aufbau der Kommunikationsarchitektur gezeigt.

        Die Gliederung der F\"ahigkeiten in einzelne Schichten bzw. Ebenen hat den Vorteil, dass eine \"Anderung an einer Ebene sich nicht auf die anderen auswirkt.
        \subsubsection{Clientanwendung}
          Anwendungen die mit einer Oracle-Datenbank kommunizieren wollen, m\"ussen das Oracle Call Interface (OCI) oder das Oracle C++ Call Interface (OCCI) benutzen. Diese Interfaces stellen der Clientanwendung alle Methoden zur Verf\"ugung, um z. B. eine Session aufzubauen, SQL-Statements an einen Serverprozess zu senden und vieles mehr.
        \subsubsection{Presentation - TTC}
          Wenn der Datenbankserver und die Clientanwendung auf unterschiedlichen Betriebssystemen laufen, kann es zu Schwierigkeiten kommen, da der Client und der Server unterschiedliche Zeichens\"atze zur Darstellung der Informationen verwenden. Der Two-Task-Common-Presentation Layer, kurz TTC, beseitigt diese Probleme durch Konvertierung der Zeichens\"atze.
          \begin{merke}
            Der TTC konvertiert nur Zeichens\"atze. Er ist kein \enquote{Sprach\"ubersetzer}.
          \end{merke}

        \bild{Die Oracle Netzwerkarchitektur}{network_architecture}{1.5}

        \subsubsection{Oracle Net Foundation Layer}
          Der \textit{Oracle Net Foundation Layer (ONFL)} ist daf\"ur Verantwortlich, die Kommunikation zwischen der Clientanwendung und dem Datenbankserver zu etablieren und aufrecht zu erhalten, sowie den Nachrichtenaustausch zwischen beiden Seiten zu erm\"oglichen. Diese Anforderungen kann der ONFL erf\"ullen, da er eine Technik verwendet, die sich \enquote{Transparent Network Substrate (TNS)} nennt. TNS stellt die M\"oglichkeit f\"ur eine Peer-To-Peer Kommunikation zwischen Client und Server bereit.

          Eine weitere F\"ahigkeit des ONFL ist es, Benennungsmethoden f\"ur die Kommunikation zwischen Client und Server bereitzustellen.
        \subsubsection{Oracle Protocol Support}
          Auf der Position zwischen dem Oracle Network Foundation Layer und den Netzwerkprotokollen ist es seine Aufgabe als Gateway zwischen TNS, TCP/IP und den anderen Protokollen zu dienen. Er sorgt daf\"ur, das die TNS-F\"ahigkeiten des ONFL mit Hilfe der Netzwerkprotokolle \"uber das Netzwerk \"ubertragen werden k\"onnen.
      \subsection{Service Registration}
        Ob eine Datenbank im Netzwerk verf\"ugbar ist, bestimmt der Listener durch einen Mechanismus, der \enquote{Service registration} genannt wird. Es gibt zwei unterschiedliche Arten von Service Registration:
        \begin{itemize}
          \item Static Service Registration
          \item Dynamic Service Registration
        \end{itemize}
        \subsubsection{Dynamic Service Registration}
          Die Dynamic Service Registration wird durch den Oracle Hintergrundprozess PMON durchgef\"uhrt. Er \"ubergibt folgende Informationen an den Listener:
          \begin{itemize}
            \item Servicenamen der verbundenen Datenbank
            \item Name der Instanz(en) die zu diesem Servicenamen geh\"oren
            \item Service handler die f\"ur diese Instanz verf\"ugbar sind (siehe \ref{connectionmodels})
          \end{itemize}
          Mit Hilfe dieser Informationen ist der Listener in der Lage, die Clientanwendung mit dem Datenbankserver zu verbinden.
          \begin{merke}
            Der Listener muss vor dem Hochfahren der Instanz gestartet werden, da der PMON sonst die dynamic service registration nicht ordnungsgem\"a\ss\ durchf\"uhren kann.
          \end{merke}
          Der PMON versucht in Intervallen von ca. 60 Sekunden eine Verbindung zu einem Listener herzustellen, um die Registrierung nachzuholen. Dieser Mechanismus kann auch manuell, mit dem Kommando \languageorasql{ALTER SYSTEM REGISTER} durchgef\"uhrt werden.

          Erh\"alt ein Listener eine Verbindungsanforderung bevor die angeforderte Instanz registriert wurde, wird er den Client abweisen.
\clearpage
          \bild{Ablauf des Service Registration Mechanismus}{serviceregistration}{1.5}

          Um zu erfahren, ob eine Datenbank dynamisch bei ihrem Listener registriert wurde, kann das Listener control utility (siehe \ref{lsnrctl}) genutzt werden. Die Anzeige von \enquote{status READY} sagt aus, dass die Datenbank dynamisch registriert wurde.
          \begin{lstlisting}[caption={Wurde die Datenbank registriert?},label=admin600,language=terminal]
Service "orcl" has 1 instance(s).
  Instance "orcl", status READY, has 1 handler(s) for this service...
          \end{lstlisting}
        \subsubsection{Static service registration}
          Bei der Static Service Registration m\"ussen die Informationen, die der Listener ben\"otigt, in die Konfigurationsdatei des Listeners eingetragen werden. Die Ausgabe des Listener control utilities sieht dann wie folgt aus:
          \begin{lstlisting}[caption={Statische Registrierung},label=admin601,language=terminal]
Service "orcl" has 1 instance(s).
  Instance "orcl", status UNKNOWN, has 1 handler(s) for this service...
          \end{lstlisting}
          Die Anzeige von \enquote{status UNKNOWN} zeigt an, dass die Datenbank statisch registriert wurde, da der Listener nichts \"uber den Zustand der Verbindung zur Datenbank wei\ss{}.
      \subsection{Die Datei listener.ora}
        Die Datei \oscommand{listener.ora} dient als Konfigurationsdatei f\"ur den Listener. Sie befindet sich im Verzeichnis \oscommand{\$ORACLE\_HOME/network/admin}. Sie enth\"alt folgende Angaben:
        \begin{itemize}
          \item Namen der Listener
          \item Hostname, Port und Protokoll f\"ur Verbindungen
          \item Namen der Datenbankservices
          \item Listenerkontroll-Parameter
        \end{itemize}
        \begin{merke}
          Mit Hilfe der Umgebungsvariable \oscommand{TNS\_ADMIN} kann ein alternativer Speicherort f\"ur die Datei \oscommand{listener.ora} angegeben werden.
        \end{merke}
        \beispiel{admin602} zeigt den Inhalt einer Listenerkonfigurationsdatei.

        In Zeile 1 steht der Name des Listeners: \enquote{LISTENER}. Die Zeilen zwei und drei definieren die \enquote{Protokolladresse} des Listeners. Sie enth\"alt Angaben \"uber das zu verwendende Netzwerkprotokoll, den Namen des Hosts, auf dem sich der Listener befindet und den Port, auf dem er lauscht.
        \begin{lstlisting}[caption={Die Datei \oscommand{listener.ora}},label=admin602,language=configfile]
LISTENER =
  (DESCRIPTION =
    (ADDRESS = (PROTOCOL = TCP)(HOST = FEA11-119SRV.oracle.com)(PORT = 1521))
  )

SID_LIST_LISTENER =
  (SID_LIST =
    (SID_DESC =
      (GLOBAL_DBNAME = orcl.local)
      (ORACLE_HOME = /u01/app/oracle/product/11.2.0/orcl)
      (SID_NAME = orcl)
    )
  )

ADR_BASE_LISTENER = /u01/app/oracle
        \end{lstlisting}
        In Zeile 6 beginnt mit dem Parameter \oscommand{SID\_LIST\_\textless listenername\textgreater} die statische Registrierung eines Datenbankservices. Die Angaben \oscommand{GLOBAL\_DBNAME}, \oscommand{ORACLE\_HOME} und \oscommand{SID\_NAME} sind notwendig, um Clients mit einer Datenbank verbinden zu k\"onnen.

        In der letzten Zeile wir durch den Parameter \oscommand{ADR\_BASE\_\textless listenername\textgreater} dem Listener mitgeteilt, wo sich das ADR (Automatic Diagnostic Repository) befindet.
      \subsection{Der Oracle Net Manager}
        Der Oracle Net Manager ist eine in Java geschriebene Software, welche seit den Tagen von Oracle 8i existiert. Er erm\"oglicht die einfache und schnelle Konfiguration eines Listeners. Gestartet wird er durch das Kommando \oscommand{netmgr} auf der Shell.
        \begin{merke}
          Genau wie f\"ur das SQL*Plus-Tool muss auch f\"ur den Net Manager zuerst das \oscommand{oraenv}-Skript ausgef\"uhrt werden.
        \end{merke}
        Mit Hilfe des Net Managers k\"onnen verschiedene Einstellungen an der Oracle Netzwerkkonfiguration vorgenommen werden.
\clearpage
        \begin{itemize}
          \item Listener konfigurieren
          \item Benennungsmethoden konfigurieren
          \item Net Service Names erstellen
        \end{itemize}
        \subsubsection{Einen Listener erstellen}
          \begin{enumerate}
            \item In der Baumansicht auf das Pluszeichen neben \enquote{Lokal} klicken, um den Rest der Baumansicht auf zuklappen.
            \bild{Start\-bild\-schirm des Net Managers}{net_mgr_1}{2.4}
            \item Auf den Knotenpunkt \enquote{Listener} klicken.
            \bild{Das Register Listener erweitern}{net_mgr_2}{2.4}
            \item Links in der Symbolleiste auf das gr\"une Pluszeichen klicken.
            \bild{Neuen Listener hinzuf\"ugen}{net_mgr_3}{0.8}
\clearpage
            \item Geben Sie eine Bezeichnung f\"ur den Listener ein und klicken Sie auf \enquote{OK}.
            \bild{Den neuen Listener benennen}{net_mgr_4}{1.1}
            \item Der neue Listener ist fertig.
            \bild{Der neue Listener ist da}{net_mgr_5}{2.5}
          \end{enumerate}
          \begin{merke}
            Der erste Listener den Sie erstellen sollte immer \enquote{LISTENER} hei\ss{}en, da dies der Standardname f\"ur einen Listener ist.
          \end{merke}
        \subsubsection{Eine weitere Protokolladresse konfigurieren}
          \label{protocoladdresses}
          Die M\"oglichkeit eine weitere Protokolladresse zu konfigurieren erlaubt es dem Administrator, den einen Listener auf einem alternativen Port lauschen zu lassen oder aber einen weiteren Listening-Port hinzuzuf\"ugen. Laut Aussage von Oracle gen\"ugt jedoch in den meisten Umgebungen ein einziger Listener mit einem Port. Interessant ist diese Option daher nur dann, falls der Listener auf zwei unterschiedliche Netzwerkprotokolle konfiguriert werden soll.

          Das Konfigurieren einer weiteren Protokolladresse geschieht im Net Manager im Bereich \enquote{Listening-Adressen}.
          \begin{enumerate}
            \item Klicken Sie auf die Bezeichnung des Listener, den Sie konfigurieren m\"ochten.
            \item W\"ahlen Sie aus dem Dropdownfeld rechts oben, die Option \enquote{Listening-Adressen} aus.
            \bild{Die Option Listening-Adressen ausw\"ahlen}{net_mgr_6}{1}
            \item Klicken Sie rechts unten auf die Schaltfl\"ache \enquote{Adresse hinzuf\"ugen}.
            \bild{Eine neue Listening-Adresse hinzuf\"ugen}{net_mgr_7}{1}
            \item Geben Sie den Hostnamen und die Listener Portnummer im jeweiligen Textfeld ein.
            \bild{Fehlende Angaben erg\"anzen}{net_mgr_8}{1}
          \end{enumerate}
        \subsubsection{Statische Registrierung konfigurieren}
          \begin{enumerate}
            \item In der Baumansicht auf das Pluszeichen neben \enquote{Lokal} klicken, um den Rest der Baumansicht auf zuklappen.
            \item Auf den Knotenpunkt \enquote{Listener} klicken.
            \item Klicken Sie auf die Bezeichnung des Listener, den Sie konfigurieren m\"ochten.
\clearpage
            \item W\"ahlen Sie aus dem Dropdownfeld rechts oben, die Option \enquote{Datenbank-Services} aus.
            \bild{Die Option \enquote{Datenbank-Services} ausw\"ahlen}{net_mgr_9}{1}
            \item Klicken Sie rechts unten auf die Schaltfl\"ache \enquote{Datenbank hinzuf\"ugen}.
            \bild{Die Schaltfl\"ache \enquote{Datenbank hinzuf\"ugen}}{net_mgr_10}{1}
            \item F\"ullen Sie die drei Textfelder aus, um eine Instanz statisch zu registrieren. Dies geht auch dann, wenn die Instanz noch gar nicht existiert.
            \bild{Angaben f\"ur die statische Registrierung}{net_mgr_11}{1}
          \end{enumerate}
        \subsubsection{Logging konfigurieren}
          \begin{enumerate}
            \item In der Baumansicht auf das Pluszeichen neben \textit{Lokal} klicken, um den Rest  aufzuklappen.
            \item Auf den Knotenpunkt \textit{Listener} klicken.
            \item Klicken Sie auf die Bezeichnung des Listener, den Sie konfigurieren m\"ochten.
\clearpage
            \item W\"ahlen Sie aus dem Dropdownfeld rechts oben, die Option \textit{Allgemeine Parameter} aus.
            \bild{Die Option \enquote{Allgemeine Parameter} ausw\"ahlen}{net_mgr_12}{1.1}
            \item W\"ahlen Sie die Registerkarte \textit{Logging und Tracing} aus.
            \bild{Die Registerkarte \enquote{Logging und Tracing}}{net_mgr_13}{1.1}
            \item Klicken Sie auf \textit{Logging aktiviert} um das Logging zu aktivieren.
            \item Passen Sie evtl. Pfad und Dateiname der Log-Datei an (zuerst muss der Haken bei \textit{ADR aktivieren} entfert werden).
          \end{enumerate}
      \subsection{Das Tool lsnrctl}
        \label{lsnrctl}
        Das Listener Control Utility, kurz \oscommand{lsnrctl} steht dem Administrator als Konfigurationswerkzeug f\"ur den Listener zur Verf\"ugung. Mit seiner Hilfe kann der Listener zur Laufzeit beeinflusst und seine aktuelle Konfiguration abgefragt werden. Es ist im Wesentlichen f\"ur das Starten und Stoppen des Listeners, sowie das Abfragen von Verbindungsdaten gedacht. Es befindet sich im Verzeichnis: \oscommand{\$ORACLE\_HOME/bin}
\clearpage
        Aufgerufen wird es mit dem Kommando \oscommand{lsnrctl}. Die vier
        wichtigsten Befehle dieses Tools sind:
        \begin{itemize}
          \item \textbf{start} [listener]: Startet den angegebenen Listener.
          Wird kein Listener angegeben, wird der Standardlistener
          \identifier{LISTENER} gestartet.
          \item \textbf{stop} [listener]: Stoppt den angegebenen Listener. Wird
          kein Listener angegeben, wird der Standardlistener
          \identifier{LISTENER} gestoppt.
          \item \textbf{status} [listener]: Zeigt eine Statusmeldung \"uber den
          angegebenen Listener. Wird kein Listener angegeben, wird
          eine Statusmeldung zum Standardlistener \identifier{LISTENER} gezeigt.
          \item \textbf{service} [listener]: Zeigt an, welche Services ein
          Listener unterst\"utzt und um welche Services es sich handelt. Dieses
          Kommando unterscheidet sich vom Kommando status darin, dass es mehr
          Informationen zu den einzelnen Services ausgibt.
        \end{itemize}

        \begin{literaturinternet}
          \item \cite{i486171}
        \end{literaturinternet}
    \section{Benennungsmethoden}
      Um eine Verbindung zu einer Datenbank aufbauen zu k\"onnen, m\"ussen Nutzer einen sogenannten \enquote{connect string} angeben. Dieser setzt sich zusammen aus:
      \begin{itemize}
        \item einem Nutzernamen,
        \item einem Passwort und
        \item einem \enquote{Connect Identifier}
      \end{itemize}
      Der Connect Identifier ist eine Zeichenkette, die alle Angaben zu der
      Oracle-Instanz enth\"alt, mit der die Verbindung hergestellt werden soll.
      Er wird mittels einer Benennungsmethode in einen Connect Descriptor
      aufgel\"ost. Der Listener erh\"alt dann die Verbindungsanforderung des
      Clients und \"ubernimmt den Verbindungsaufbau. Oracle kennt vier
      verschiedene Benennungsmethoden:
      \begin{itemize}
        \item Hostnaming

          Hostnaming ist die einfachste Benennungsmethode. Wenn auf einem Server
          nur eine einzige Datenbank l\"auft, gen\"ugt die IP-Adresse/der
          DNS-Name des Servers, um die Datenbank zu kontaktieren. Es wird
          daf\"ur keine weitere Konfiguration auf dem Clientrechner ben\"otigt.
        \item Easy Naming

          Easy Naming erm\"oglicht es, einen TCP-Connect String, der aus einer
          IP-Adresse, einem Port und einem Servicename besteht, zum
          Verbindungsaufbau zu verwenden. Diese Methode arbeitet v\"ollig
          konfigurationslos.
        \item Local Naming

         Hier wird der Net Service Name in einer lokalen Datei mit dem Namen:
         \oscommand{tnsnames.ora} gespeichert.
        \item Directory Naming

          Beim Directory Naming werden die Connect Identifier in einem zentralen LDAP-Service, dem \enquote{Oracle Internet Directory (OID)} gespeichert.
        \item External Naming

          Bei dieser Methode, werden Net Service Names in einem externen, von Oracle unter\-st\"ut\-zten, Verzeichnis (NIS, CDS) gespeichert.
      \end{itemize}
        \begin{literaturinternet}
          \item \cite{CIHGGHEE}
          \item \cite{i1047762}
        \end{literaturinternet}
      \subsection{Die Datei sqlnet.ora}
        Die Datei \oscommand{sqlnet.ora} ist f\"ur die Konfiguration der Oracle Netzwerkeinstellungen da. Hier kann unter anderem auch festgelegt werden, welche Benennungsmethoden aktiviert sind. \beispiel{admin603} zeigt beispielsweise den Parameter \parameter{NAMES.DIRECTORY\_PATH}.
        \begin{lstlisting}[caption={Der Parameter NAMES.Directory\_path in der Datei sqlnet.ora},label=admin603,emph={[9]NAMES},emphstyle={[9]\color{black}},language=configfile]
NAMES.DIRECTORY_PATH=(HOSTNAME, TNSNAMES, EZCONNECT)
        \end{lstlisting}
          Zwischen den beiden Klammern werden die Namen der Benennungsmethoden in der Reihenfolge angegeben, in der sie genutzt werden sollen. Damit einen Benennungsmethode funktioniert, muss sie an dieser Stelle aufgelistet sein. Folgende Werte sind f\"ur \parameter{NAMES.DIRECTORY\_PATH} zul\"assig:
          \begin{itemize}
            \item HOSTNAME  : Hostnaming
            \item TNSNAMES  : Local Naming
            \item EZCONNECT : Easy Connect Naming
            \item LDAP      : Directory Naming
            \item NIS       : Network Information Server (auch als Yellow Page bekannt)
          \end{itemize}
\clearpage
      \subsection{Die Benennungsmethode Easy Naming}
        In einer TCP/IP-Umgebung ist das Easy Naming die einfachste Methode eine Verbindung zu einer Datenbank aufzubauen. Es besteht dabei keine Notwendigkeit, Net Service Names in der Datei \oscommand{tnsnames.ora} zu speichern. Es wird einfach die TCP/IP-Methode des Hostnamings erweitert. Die Syntax f\"ur die Nutzung des Easy Namings ist folgende:
        \begin{lstlisting}[caption={Easy Naming},label=admin606,language=terminal]
connect username/password@[//]host[ :port][/services_name]
        \end{lstlisting}
        Einige Beispiele f\"ur das Easy Naming:
        \begin{lstlisting}[caption={Beispiele f\"ur Easy Naming},label=admin607,language=terminal]
connect hr/hr@FEA11-119SRV.oracle.com:1521/orcl.it-training-alt.fus
connect sys/oracle@//FEA11-119SRV.oracle.com/orcl.it-training-alt.fus as sysdba
        \end{lstlisting}
        \begin{merke}
          Diese Verbindungsmethode kann nur dann verwendet werden, wenn keine Advanced Features (Verschl\"usselung u. \"a.) f\"ur die Verbindung zur Datenbank gefordert sind.
        \end{merke}
        \begin{literaturinternet}
          \item \cite{i507143}
        \end{literaturinternet}
      \subsection{Die Benennungsmethode Hostnaming}
        Die einfachste Benennungsmethode ist das Hostnaming. Hier ist keine Konfigurationsarbeit auf dem Clientrechner notwendig. Alle notwendigen Einstellungen werden nur einmal auf dem Server vorgenommen.

        \subsubsection{Funktionsweise}
          Damit das Hostnaming funktioniert, m\"ussen folgende Bedingungen erf\"ullt sein:
          \begin{itemize}
            \item Das Hostnaming muss in der Datei \textit{sqlnet.ora} aktiviert sein.
            \item Auf dem Datenbankserver muss ein Listener gestartet sein, der TCP/IP benutzt.
            \item Der Listeningport muss 1521 (Standardport) sein.
            \item Der Listener muss einen Service kennen, dessen Name gleichlautend mit dem Hostnamen des DB-Servers ist. Das bedeutet, wenn der Hostname des Datenbankservers FEA11-119SRV.oracle.com lautet, dann muss folglich auch der Service Name FEA11-119SRV.oracle.com sein.
          \end{itemize}
          Die ersten beiden Bedingungen sind standardm\"a\ss{}ig immer erf\"ullt, die Dritte muss \"uberpr\"uft werden. Die vierte Bedingung kann auf zwei unterschiedliche Arten erf\"ullt werden.
          \begin{itemize}
            \item Eine Instanz, deren Initialisierungsparameter \parameter{db\_name} gleichlautend mit dem Hostnamen ist, hat sich dynamisch beim Listener registriert.
            \item Es wurde eine statische Registrierung beim Listener vorgenommen und der Listenerparameter \textit{GLOBAL\_DBNAME} wurde passend gesetzt:
          \end{itemize}
          \begin{lstlisting}[caption={Eine statische Registrierung und der Parameter GLOBAL\_DBNAME},label=admin604,language=configfile]
LISTENER =
  (DESCRIPTION =
    (ADDRESS = (PROTOCOL = TCP)(HOST = FEA11-119SRV.oracle.com)(PORT = 1521))
  )

SID_LIST_LISTENER =
  (SID_LIST =
    (SID_DESC =
      (GLOBAL_DBNAME = FEA11-119SRV.oracle.com)
      (ORACLE_HOME = /u01/app/oracle/product/11.2.0)
      (SID_NAME = orcl)))
          \end{lstlisting}
        \subsubsection{Fehler im System - Bug.6374523}
          In der Oracle Version 11.2.0.1 liegt ein Bug vor, der die Benutzung des Hostnamings erschwert. Der Fehler besteht darin, dass der Client, bei seiner Verbindungsanfrage, den Service Name der gew\"unschten Datenbank nicht mitsendet. Dies kann durch die Anwendung des Tools \oscommand{tnsping} sichtbar gemacht werden.
          \begin{lstlisting}[caption={Der Parameter DEFAULT\_SERVICE\_listener\_name},label=admin604a,language=terminal]
TNS Ping Utility for Linux: Version 11.2.0.1.0

Copyright (c) 1997, 2011, Oracle.  All rights reserved.

Used parameter files:
/u01/app/oracle/product/11.2.0/orcl/network/admin/sqlnet.ora

Used HOSTNAME adapter to resolve the alias
Attempting to contact (DESCRIPTION=(CONNECT_DATA=(&\textcolor{red}{SERVICE\_NAME=}&))
(ADDRESS=(PROTOCOL=TCP)(HOST=192.168.20.100)(PORT=1521)))
OK (20 msec)
          \end{lstlisting}
          Die L\"osung hierf\"ur ist im Oracle MySupport Forum beschrieben. Es muss der Listener-Parameter \languageconfigfile{DEFAULT\_SERVICE\_listener\_name} in die Datei \oscommand{listener.ora} eingef\"ugt werden. Hei\ss{}t der Listener \enquote{LISTENER}, so muss folgende Eintragung gemacht werden.
\clearpage
          \begin{lstlisting}[caption={Der Parameter DEFAULT\_SERVICE\_listener\_name},label=admin604b,language=configfile]
LISTENER =
  (DESCRIPTION =
    (ADDRESS = (PROTOCOL = TCP)(HOST = FEA11-119SRV.oracle.com)(PORT = 1521))
  )

SID_LIST_LISTENER =
  (SID_LIST =
    (SID_DESC =
      (GLOBAL_DBNAME = FEA11-119SRV.oracle.com)
      (ORACLE_HOME = /u01/app/oracle/product/11.2.0)
      (SID_NAME = orcl)))

&\textcolor{red}{DEFAULT\_SERVICE\_LISTENER=FEA11-119SRV.oracle.com}&
          \end{lstlisting}
          Durch die Angabe dieses Parameters, zusammen mit dem Hostnamen, wird jeder Client, dessen Anfrage keinen Service Name enth\"alt, an den statisch registrierten Service \enquote{FEA11-119SRV.oracle.com} verwiesen.
        \subsubsection{Benutzung des Hostnamings}
          Um diese Benennungsmethode zu nutzen, muss der Nutzer an seinem Clientrechner einen Nutzernamen, ein Passwort und den Hostnamen des Datenbankservers angeben. Dies k\"onnte z. B. so aussehen:
          \begin{lstlisting}[caption={Ein Beispiel f\"ur Hostnaming},label=admin605,language=sqlplus]
connect hr/hr@FEA11-119SRV.oracle.com
          \end{lstlisting}
          Der Hostname wird durch ein @ Zeichen vom Passwort getrennt. Die Namensaufl\"osung erfolgt, wie gewohnt durch einen DNS-Server oder evtl. auch durch eine Hosts-Datei.
      \subsection{Die Benennungsmethode Local Naming}
        Beim Local Naming werden Net Service Names in einer lokalen Datei mit dem Namen \oscommand{tnsnames.ora} gespeichert. Jeder Net Service Name geh\"ort zu einem Connect Descriptor. Wie Net Service Names und Connect Descriptors in der Datei \oscommand{tnsnames.ora} gespeichert werden, ist im folgenden Beispiel zu sehen.
\clearpage
        \subsubsection{Connect Descriptoren}
          Ein Connect Descriptor wird zusammengesetzt aus:
          \begin{itemize}
            \item einer oder mehreren Protokolladressen eines Listeners
            \item zus\"atzlichen Verbindungsdaten f\"ur die Zieldatenbank
          \end{itemize}
          \begin{lstlisting}[caption={Ein Connect Descriptor},label=admin608,language=configfile]
  (DESCRIPTION=
    (ADDRESS_LIST=
      (ADDRESS= (PROTOCOL=tcp)(HOST=FEA11-119SRV.oracle.com)(PORT=1521))
    )
  (CONNECT_DATA=
    (SERVICE_NAME=orcl)))
          \end{lstlisting}
          Der ADDRESS-Abschnitt enth\"alt die Protokolladresse des Listeners. Im zweiten Abschnitt, der die Bezeichnung \enquote{CONNECT\_DATA} tr\"agt, sind die zus\"atzlichen Verbindungsdaten f\"ur die Zieldatenbank enthalten. In diesem Fall wird die Zieldatenbank durch ihren Service Name \enquote{orcl} identifiziert.

          Um die Nutzung solcher Connect Descriptoren zu vereinfachen, k\"onnen Sie mit einem \enquote{Net Service Name} versehen werden. Im folgenden Beispiel, wird der Net Service Name \enquote{sales} f\"ur einen Connect Descriptor vergeben. Dieser Net Service Name kann dann anstatt des kompletten Connect Descriptors verwendet werden.
          \begin{lstlisting}[caption={Aufbau eines Net Service Names},label=admin609,language=configfile]
sales=
  (DESCRIPTION=
    (ADDRESS_LIST=
     (ADDRESS= (PROTOCOL=tcp)(HOST=FEA11-119SRV.oracle.com)(PORT=1521))
    )
  (CONNECT_DATA=
    (SERVICE_NAME=orcl)))
          \end{lstlisting}
          Um sich bei einer Datenbank anzumelden, kann, wie in den folgenden
          Beispielen, entweder der komplette Connect Descriptor oder aber der
          Net Service Name genutzt werden.
          \begin{lstlisting}[caption={Verwendung eines Connect Descriptors als Connect Identifier},label=admin610,language=sqlplus,alsolanguage=terminal]
connect scott/tiger@(DESCRIPTION=(ADDRESS=(PROTOCOL=tcp)
(HOST=FEA11-119SRV.oracle.com)(PORT=1521)) (CONNECT_DATA=(SERVICE_NAME=orcl))) \
          \end{lstlisting}
          \begin{lstlisting}[caption={Verwendung eines Net Service Name als Connect Identifier},label=admin611,language=sqlplus]
connect scott/tiger@orcl
          \end{lstlisting}
        \subsubsection{Konfiguration des Local Naming im Net Manager}
          \begin{enumerate}
            \item In der Baumansicht auf das Pluszeichen neben \textit{Lokal} klicken, um den Rest der Baumansicht auf zuklappen.
            \item Auf den Knotenpunkt \enquote{Dienstebenennung} klicken.
            \bild{Der Knotenpunkt \enquote{Dienstebenennung}}{net_mgr_14}{2.2}
            \item Links in der Symbolleiste auf das gr\"une Pluszeichen klicken.
            \item Geben Sie den Net Service Namen ein und klicken Sie auf \enquote{Weiter}.
            \bild{Eingabe des Net Service Name}{net_mgr_15}{1.4}
\clearpage
            \item W\"ahlen Sie die gew\"unschten Netzwerkprotokolle aus, mit deren Hilfe eine Verbindung zur Datenbank erstellt werden soll und klicken Sie dann auf \enquote{Weiter}
            \bild{Auswahl des Netzwerkprotokolls}{net_mgr_16}{1.4}
            \item Geben Sie die IP-Adresse oder den Hostnamen des Datenbankservers ein. Der Listenerport ist standardm\"a\ss{}ig immer 1521. Falls Sie auf dem Server einen anderen Listenerport konfiguriert haben, tragen Sie diesen jetzt ein. Klicken Sie anschlie\ss end auf \enquote{Weiter}.
            \bild{IP-Adresse und Port des Listeners festlegen}{net_mgr_17}{1.4}
\clearpage
            \item Geben Sie zur Identifikation der Datenbankinstanz den Service Name der Instanz an und klicken Sie auf \enquote{Weiter}.
            \bild{Der Service Name der Datenbank}{net_mgr_18}{1.4}
            \item Um die Verbindung zu testen, klicken Sie auf \enquote{Testen...}. Klicken Sie anschlie\ss end auf \enquote{Beenden}.
            \bild{Der abschlie\ss{}ende Verbindungstest}{net_mgr_19}{1.4}
\clearpage
            \bild{Den Assistenten beenden}{net_mgr_20}{1.3}

            \bild{Der fertig konfigurierte Net Service Name}{net_mgr_21}{1.3}
          \end{enumerate}
      \subsection{Die Benennungsmethode Directory Naming}
        Beim Directory Naming wird die gleiche Technik wie beim Local Naming verwendet: TNS. Der Unterschied zum Local Naming besteht jedoch darin, das die TNS-Eintr\"age nicht mehr in einer Textdatei gespeichert werden, die lokal auf jedem Clientrechner verf\"ugbar sein muss, sondern in einem LDAP-Verzeichnisdienst, dem Oracle Internet Directory.

        Um das Directory Naming nutzen zu k\"onnen, m\"ussen die folgenden Bedingungen erf\"ullt werden:
\clearpage
        \begin{itemize}
          \item Das Directory Naming muss in der Datei \oscommand{sqlnet.ora} aktiviert sein.
          \item Die Datei \oscommand{ldap.ora} muss auf den Clientrechnern konfiguriert werden.
        \end{itemize}
        Eine neue Konfigurationsdatei die hier ins Spiel kommt, ist die Datei \oscommand{ldap.ora}. Sie steuert den Zugriff auf einen LDAP-Dienst und sieht folgenderma\ss{}en aus:
        \begin{lstlisting}[caption={Die Datei ldap.ora},label=admin612,language=configfile]
DEFAULT_ADMIN_CONTEXT = "ou=OracleContext"
DIRECTORY_SERVERS = (FEA11-119OID20.oracle.com:389:636)
DIRECTORY_SERVER_TYPE = OID
        \end{lstlisting}
        Die drei gezeigten Parameter haben folgende Bedeutung:
        \begin{itemize}
          \item \parameter{DEFAULT\_ADMIN\_CONTEXT}: Gibt an, wo in der Verzeichnisstruktur des LDAP die TNS-Daten gespeichert werden.
          \item \parameter{DIRECTORY\_SERVERS}: Rechnername/IP-Adresse und Port/SSL-Port des Directoryservers.
          \item \parameter{DIRECTORY\_SERVER\_TYPE}: Der Typ des Directoryserver (OID = Oracle Internetdirectory oder AD = Active Directory)
        \end{itemize}

        Die Aufl\"osung eines in einem Directory gespeicherten Net Service Name l\"auft wie folgt ab:

        \bild{Directory Naming}{network_directory_naming}{1.5}

        \begin{itemize}
          \item Auf dem Clientrechner wird mittels der Datei \oscommand{sqlnet.ora} ermittelt, dass eine Namensaufl\"osung durch einen LDAP-Dienst gemacht werden soll. Die Datei \oscommand{ldap.ora} gibt vor, welcher LDAP-Dienst zust\"andig ist.
          \item Der LDAP-Dienst beantwortet die Anfrage nach dem Net Service Name \enquote{orcl} mit den Verbindungsdaten f\"ur den Listener des Datenbankservers (Protokoll, IP-Adresse, Port, Service Name).
          \item Der Client kann im dritten Schritt eine Verbindung zum Datenbankserver aufbauen. Wie der Verbindungsaufbau genau abl\"auft, ist davon abh\"angig, ob eine Dedicated Server Umgebung oder eine Shared Server Umgebung aufgebaut wurde.
        \end{itemize}
    \section{Netzwerk Verbindungsmodelle}
    \label{connectionmodels}
      Unter dem Begriff \enquote{Verbindungsmodell} versteht man die Art und Weise, wie ein Clientprozess mit einer Oracle-Instanz verbunden wird. Seit Oracle 11g gibt es drei verschiedene Verbindungsmodelle:

      \begin{itemize}
        \item Dedicated Server Architektur
        \item Shared Server Architektur
        \item Database Resident Connection Pooling (Oracle 11g New Feature)
      \end{itemize}

      \begin{merke}
        Bei Dedicated Serverprozessen bzw. Dispatchern handelt es sich um sogenannte Service Handler. Ein Service Handler ist ein serverseitiger Prozess, der Anfragen eines Clients entgegen nimmt und bearbeitet (Serverprozess) oder nur zur Verarbeitung weiterleitet (Dispatcher).
      \end{merke}
      \subsection{Die Dedicated Server Architektur}
        In einer Dedicated Server Architektur startet der Listener f\"ur jeden Client einen eigenen Serverprozess. Nach dem die Session des Clients beendet wurde, wird der Serverprozess ebenfalls beendet. Diese Konfiguration ben\"otigt sehr viele Ressourcen, da jeder Client seinen eigenen Serverprozess bekommt. Grunds\"atzlich ist dies aber die zu bevorzugende Variante, da sie die beste Performance erzielt.
        \begin{enumerate}
          \item Der Client schickt eine Verbindungsanforderung an den Datenbankserver. Der Listener nimmt diese entgegen.
          \item Der Listener erstellt einen Serverprozess.
          \item Der Serverprozess f\"uhrt die Authentifizierung des Clients durch.
          \item Es erfolgt ein \textit{Connection-Redirect}. D. h. der Listener vermittelt die Verbindung zwischen dem Clientprozess und dem Serverprozess.
        \end{enumerate}
        Im Anschluss an diese vier Schritte, ist der Client mit dem Datenbankserver verbunden und kann seine Arbeit aufnehmen.
        \bild{Aufbau der Verbindung in einer Dedicated Server Architektur}{network_dedicated_server}{0.9}
      \subsection{Die Shared Server Architektur}
        In einer Shared Server Architektur werden Clients nicht mit einem Serverprozess, sondern mit einem Dispatcher verbunden. Ein Dispatcher funktioniert als Verwalter und Verteiler von Arbeitsauftr\"agen. Er nimmt Anforderungen von Clients entgegen und leitet sie an einen Serverprozess weiter, der dann die eigentliche Arbeit verrichtet.

        \bild{Aufbau der Verbindung in einer Shared Server Architektur}{network_shared_server_architecture}{0.9}

        Jeder Dispatcher kann mehrere Clientverbindungen annehmen. Jede Clientanforderung wird von den Dispatchern in eine Warteschlange (Request Queue) aufgenommen. Ein Shared Serverprozess, der eine Anforderung abgearbeitet hat, nimmt sich aus dieser Warteschlange die n\"achste Anforderung, um sie abzuarbeiten und legt das Ergebnis der Anforderung in die Response Queue des Dispatchers, von dem die Anforderung kam. Die Dispatcher, lesen ihre Response Queue aus, um die Ergebnisse an ihre Clients auszuliefern. Auf diese Weise kann eine kleine Menge Shared Server Prozesse eine gro\ss e Anzahl Clients bedienen.

        Der in \abbildung{network_shared_server_architecture} gezeigte Verbindungsaufbau zwischen einem Client und der Datenbank l\"auft wie folgt ab:
\clearpage
          \begin{enumerate}
            \item Ein Client baut eine Verbindung zum Listener auf und fordert eine Verbindung zur Datenbank an.
            \item Der Listener leitet den Client an einen der bestehenden Dispatcher-Prozesse weiter.
            \item Der Dispatcher platziert die Anforderung in der Request Queue.
            \item Ein Serverprozess nimmt die Anforderung zur Verarbeitung aus der Request Queue.
            \item Der Serverprozess legt die Anforderung in die Response Queue des Dispatchers, von dem die Anforderung kam.
            \item Der Dispatcher leitet das Ergebnis aus seiner Response Queue an den Client weiter, der die Anforderung gestellt hat.
          \end{enumerate}
      \subsection{Database Resident Connection Pooling (DRCP)}
        DRCP ist eine neue Technologie, die den Versuch darstellt, die Vorteile des Dedicated Server- und des Shared Server Modells zu vereinen. Beim DRCP erstellt die Datenbank einen Pool von Dedicated Serverprozessen, die als \enquote{Pooled Server} bezeichnet werden. Dieser Pool kann von einer sehr gro\ss{}en Anzahl Clients genutzt werden, sofern die Anwendungen die gleichen Credentials (Nutzername, Passwort) f\"ur den Aufbau der Session nutzen. Im Vergleich zum Shared Server Modell werden hier nicht nur die Serverprozesse geteilt, sondern auch die Sessions. Besonders vorteilhaft ist dies bei Webanwendungen, wo in kurzer Zeit sehr viele und zeitlich sehr begrenzte Connections aufgebaut werden.

        \bild{Database Resident Connection Pooling}{database_resident_connection_pooling}{0.8}

        \subsubsection{Der Connection Broker}
          Der Connection Broker ist ein in Oracle 11g neu hinzugekommener Hintergrundprozess, der die Verwaltung der Pooled Server und der Sessions \"ubernimmt. Muss ein Client an der Datenbank arbeiten, hat der Connection Broker die Aufgabe, ihm einen Serverprozess zuzuweisen. Der Serverprozess verh\"alt sich wie ein Dedicated Serverprozess f\"ur den Client. Hat der Client seine Arbeit beendet, \"ubernimmt der Connection Broker wieder die Verwaltung von Serverprozess und Session.
        \subsubsection{Die Ressourcenersparnis}
          Die durch die Nutzung von DRCP entstehende Ressourcenersparnis, kann an einem einfachen Rechenbeispiel dargestellt werden:

          In einem Firmenintranet soll eine Datenbankinfrastruktur f\"ur bis zu 3.000 Clients geschaffen werden. Bei der Nutzung des Dedicated Server Modells werden f\"ur jeden Serverprozess ca. 4 MB RAM ben\"otigt, plus 400 KB f\"ur die Session. Dies ergibt in Summe: $3000 * (4MB + 400KB) \approx 12,86 GB$

          Wird die gleiche Infrastruktur als Shared Serverumgebung realisiert, werden weniger Serverprozesse ben\"otigt, da sich mehrere Clients einen Serverprozess teilen. Werden f\"ur 3.000 Clients insgesamt 75 Serverprozesse zur Verf\"ugung gestellt, ergeben sich folgende Zahlen: $(3000 * 400 KB) + (75 * 4 MB) \approx 1,44 GB$ Arbeitsspeicher.

          Das Database Resident Connection Pooling reduziert den Ressourcenverbrauch noch einmal, da mehrere Clients sich einen Serverprozess und eine Session teilen. Hinzu kommt jedoch ein Verwaltungsoverhead, von ca. 35 KB pro Client. Das Ergebnis sieht so aus: $(75 * (400 KB + 4 MB) + (3000 * 35 KB)) \approx 0,42 GB \mathop{\widehat{=}} 431 MB$ Arbeitsspeicher.

          \begin{merke}
            DRCP kann genauso wie die Shared Server Architektur nicht f\"ur Administratoren genutzt werden!
          \end{merke}

      \subsection{Dedicated Server, Shared Server und DRCP im Vergleich}
        \begin{center}
          \begin{small}
            \tablecaption{Einsatz von Dedicated und Shared Server Modell}
            \tablefirsthead{%
              \hline
              \multicolumn{1}{|c}{\textbf{Dedicated Server}}&
              \multicolumn{1}{|c|}{\textbf{Shared Server}} &
              \multicolumn{1}{c|}{\textbf{DRCP}} \\
              \hline
            }
            \tabletail{
              \hline
            }
            \begin{supertabular}[h]{|p{4.9cm}|p{4.9cm}|p{4.9cm}|}
              Alle Clients haben einen eigenen Serverprozess und eine eigene Session.& Clients geben ihre Anforderungen \"uber Dispatcher an einen Pool von Serverprozessen weiter. Sie haben nur noch eine eigene Session, keinen eigenen Serverprozess mehr. & Clients werden vom Connection Broker Prozess an einen Serverprozess mit einer bestehenden Session vermittelt. \\
              \hline
              Soll die Client-Connection beendet werden, m\"ussen der Serverprozess und die Session zerst\"ort werden. & Es muss nur die Session des Clients beendet werden, der Serverprozess bleibt erhalten. & Beim Verbindungsabbau \"ubernimmt der Connection Broker wieder die Verwaltung des Serverprozesses und der Session. \\
              \hline
            \end{supertabular}
          \end{small}
        \end{center}
    \section{Informationen}
      \subsection{Verzeichnis der Konfigurationsdateien}
        \begin{literaturinternet}
          \item  \cite{NETRF011}
          \item  \cite{NETRF008}
          \item  \cite{NETRF007}
          \item  \cite{NETRF006}
        \end{literaturinternet}
\clearpage

    \section{\"Ubungen - Konfigurieren der Oracle-Netzwerkumgebung}
  \begin{enumerate}
        \item F\"uhren Sie ein Recovery bei Verlust einer Redo Log Datei durch!
      \begin{enumerate}
        \item Starten Sie das Skript \oscommand{lab\_delete\_redolog\_member.sql}! Es l\"oscht einen beliebigen Member einer Ihrer Redo Log Gruppen.
          \begin{lstlisting}[language=terminal]
SQL> @/home/oracle/labs/lab_delete_redolog_member.sql
          \end{lstlisting}
        \item Die Datenbank l\"auft weiterhin normal und es liegen keinerlei Probleme vor. \"Offnen Sie die Alert Log Datei, um herauszufinden welcher Redo Log Member gel\"oscht wurde!
        \item F\"uhren Sie geeignete Ma\ss{}nahmen zur Problembehebung durch!
      \end{enumerate}

        \item Konfigurieren Sie das Auditing f\"ur die Tabelle \identifier{bank.mitarbeiter} so, dass erfolgreiche \languageorasql{INSERT}- und \languageorasql{UPDATE}-Statements auditiert werden! Es soll jeweils nur ein Auditeintrag pro Session f\"ur jedes  \languageorasql{INSERT}/\languageorasql{UPDATE} gemacht werden.

        \item Wechseln Sie den Undo-Tablespace zur\"uck zum Original Undo-Tablespace und l\"oschen Sie \identifier{undotbs02} und \identifier{undotbs03}.

        \item Pr\"ufen Sie in welchem Modus der Result Cache l\"auft (Manual oder Force)

        \item Pr\"ufen Sie den Audittrail auf neue Informationen! Welche T\"atigkeiten wurden an der auditierten Tabelle ausgef\"uhrt?

        \item Machen Sie alle Auditingeinstellungen r\"uckg\"angig und l\"oschen Sie den Inhalt des Auditingtrails!

        \item Schalten Sie das Auditing f\"ur alle erfolgreichen Anmeldeversuche ein! Lassen Sie diese Auditingeintr\"age im \textit{DB, Extended} Auditingtrail speichern!

        \item F\"uhren Sie ein Recovery mit einem Backup Controlfile durch.
      \begin{enumerate}
        \item Starten Sie das Skript \oscommand{lab\_delete\_all\_controlfiles.sql}. Es wird alle Kontrolldateien l\"oschen.
          \begin{lstlisting}[language=terminal]
SQL> @/home/oracle/labs/lab_delete_all_controlfiles.sql
          \end{lstlisting}
        \item Ergreifen Sie geeignete Ma\ss{}nahmen, um die Datenbank wieder lauff\"ahig zu machen.
      \end{enumerate}


      \rule{0.94\textwidth}{0.5pt}

      \rule{0.94\textwidth}{0.5pt}

      \rule{0.94\textwidth}{0.5pt}
\clearpage
        \item Legen Sie den tempor\"aren smallfile Tablespace \identifier{temp\_ts} an. Sein Tempfile  soll 500 M gro\ss{} sein, auf bis zu 4 G anwachsen k\"onnen, \oscommand{temp\_ts01.dbf} hei\ss{}en und auf Laufwerk \oscommand{/u02} liegen. Benutzen Sie 1 M gro\ss{}e Uniform-Sized Extents.

\end{enumerate}
\clearpage

    \input{loesungen/dbadmin_10_konfigurieren_der_oracle_netzwerkumgebung_loesung}
  \input{admin/11_verwalten_des_result_caches}
    \section{\"Ubungen - Verwalten des Result Caches}
  \begin{enumerate}
        \item F\"uhren Sie ein Recovery bei Verlust einer Redo Log Datei durch!
      \begin{enumerate}
        \item Starten Sie das Skript \oscommand{lab\_delete\_redolog\_member.sql}! Es l\"oscht einen beliebigen Member einer Ihrer Redo Log Gruppen.
          \begin{lstlisting}[language=terminal]
SQL> @/home/oracle/labs/lab_delete_redolog_member.sql
          \end{lstlisting}
        \item Die Datenbank l\"auft weiterhin normal und es liegen keinerlei Probleme vor. \"Offnen Sie die Alert Log Datei, um herauszufinden welcher Redo Log Member gel\"oscht wurde!
        \item F\"uhren Sie geeignete Ma\ss{}nahmen zur Problembehebung durch!
      \end{enumerate}


      \rule{0.94\textwidth}{0.5pt}

        \item Konfigurieren Sie das Auditing f\"ur die Tabelle \identifier{bank.mitarbeiter} so, dass erfolgreiche \languageorasql{INSERT}- und \languageorasql{UPDATE}-Statements auditiert werden! Es soll jeweils nur ein Auditeintrag pro Session f\"ur jedes  \languageorasql{INSERT}/\languageorasql{UPDATE} gemacht werden.


      \rule{0.94\textwidth}{0.5pt}

        \item Wechseln Sie den Undo-Tablespace zur\"uck zum Original Undo-Tablespace und l\"oschen Sie \identifier{undotbs02} und \identifier{undotbs03}.

        \item Pr\"ufen Sie in welchem Modus der Result Cache l\"auft (Manual oder Force)


      \rule{0.94\textwidth}{0.5pt}

        \item Pr\"ufen Sie den Audittrail auf neue Informationen! Welche T\"atigkeiten wurden an der auditierten Tabelle ausgef\"uhrt?

        \item Machen Sie alle Auditingeinstellungen r\"uckg\"angig und l\"oschen Sie den Inhalt des Auditingtrails!

        \item Schalten Sie das Auditing f\"ur alle erfolgreichen Anmeldeversuche ein! Lassen Sie diese Auditingeintr\"age im \textit{DB, Extended} Auditingtrail speichern!

        \item F\"uhren Sie ein Recovery mit einem Backup Controlfile durch.
      \begin{enumerate}
        \item Starten Sie das Skript \oscommand{lab\_delete\_all\_controlfiles.sql}. Es wird alle Kontrolldateien l\"oschen.
          \begin{lstlisting}[language=terminal]
SQL> @/home/oracle/labs/lab_delete_all_controlfiles.sql
          \end{lstlisting}
        \item Ergreifen Sie geeignete Ma\ss{}nahmen, um die Datenbank wieder lauff\"ahig zu machen.
      \end{enumerate}


      \rule{0.94\textwidth}{0.5pt}

      \rule{0.94\textwidth}{0.5pt}

        \item Legen Sie den tempor\"aren smallfile Tablespace \identifier{temp\_ts} an. Sein Tempfile  soll 500 M gro\ss{} sein, auf bis zu 4 G anwachsen k\"onnen, \oscommand{temp\_ts01.dbf} hei\ss{}en und auf Laufwerk \oscommand{/u02} liegen. Benutzen Sie 1 M gro\ss{}e Uniform-Sized Extents.


      \rule{0.94\textwidth}{0.5pt}

      \rule{0.94\textwidth}{0.5pt}

        \item L\"oschen Sie das Backup Set des \identifier{bank}-Tablespaces von der Festplatte, so dass es nur noch auf SBT verf\"ugbar ist!

\clearpage
        \item Der Tablespace \identifier{bank} liegt derzeit in unverschl\"usselter Form vor. Dies muss zuk\"unftig anders sein. Bereiten Sie einen Tablespace \identifier{bank\_encrypted} vor, der mittels \identifier{AES256} verschl\"usselung gesichert ist und der die gleichen Dimensionen hat, wie der Originaltablespace \identifier{bank}! Rufen Sie alle notwendigen Informationen \"uber den Tablespace \identifier{bank} aus dem Data Dictionary ab! Welche Views helfen Ihnen dabei?

  \end{enumerate}
\clearpage

    \input{loesungen/dbadmin_11_verwalten_des_result_caches_loesung}
    \chapter{Implementing Security}
    \setcounter{page}{1}\kapitelnummer{chapter}
    \minitoc
\newpage
    \section{Database Auditing}
      Auditing ist das \"Uberwachen und Aufzeichnen von ausgew\"ahlten Aktionen,
      die innerhalb der Datenbank stattfinden. Es kann auf Einzelnen oder auf
      einer Kombination von Faktoren (z. B. Nutzername, Anwendung, Anmeldezeit,
      usw.) basieren. Auditing \"ublicherweise f\"ur die folgenden Zwecke
      genutzt:
      \begin{itemize}
        \item Abschreckung von Nutzern, vor unerlaubten Zugriffen auf Objekte,,
        au\ss erhalb ihres Verantwortungsbereiches.
        \item Daten \"uber bestimmte Aktivit\"aten in der Datenbank sammeln
        \item Verd\"achtige Nutzeraktivit\"aten aufdecken
        \item Aufdecken von Problemen mit Autorisations- und Zugriffskontrollmechanismen
      \end{itemize}
      In Oracle gibt es f\"unf Arten des Auditings:
      \begin{itemize}
        \item \textbf{Statementauditing}: Die Nutzung bestimmter SQL-Statements wird \"uberwacht.
        \item \textbf{Privilegeauditing}: Die Nutzung von Privilegien wird \"uberwacht.
        \item \textbf{Objectauditing}: Der Zugriff auf bestimmte Objekte wird \"uberwacht.
        \item \textbf{Networkauditing}: \"Uberwachung von Fehlern im Netzwerk
        \item \textbf{Standardauditing}: Standardm\"a\ss{}ige \"Uberwachung, die immer aktiv ist.
      \end{itemize}
      \begin{merke}
        Jeder Datenbankadministrator kann das Auditing jederzeit aktivieren.
      \end{merke}
      \subsection{Grunds\"atze und Vorgehensweisen beim Auditing}
        Oracle 11g erm\"oglicht es Auditinginformationen in den Auditingtrail
        der Datenbank oder den des Betriebssystems (Windowsereignisanzeige oder
        SysLog Daemon) zu schreiben. Je nachdem welche Nutzergruppe \"uberwacht
        werden muss, normale Nutzer oder Administratoren, empfiehlt es sich
        einen ad\"aquaten Auditingtrail auszuw\"ahlen, so dass dieser nicht
        durch den betroffenen Nutzerkreis ver\"andert/gef\"alscht werden kann.
        Datenbankadministratoren z. B. haben in einer Oracledatenbank
        unbeschränkten Zugriff, weshalb ein Sammeln der Auditingdaten innerhalb
        der Datenbank nicht sinnvoll ist.
        \subsubsection{Auditinginformationen \"uberschaubar halten}
          Auch wenn Auditing keine gro\ss en Kosten erzeugt, sollte die Menge der Auditinginformationen auf das Notwendigste beschr\"ankt werden. Auf diese Weise soll der Auditingtrail \"uberschaubar gehalten und eine einfache Auswertung der Informationen erm\"oglicht werden.

          Das Wichtigste f\"ur die Entwicklung einer Auditingstrategie ist: Der genaue Zweck des Auditings muss bestimmt werden. Erst wenn festgelegt wurde, warum die Datenbank \"uberwacht werden muss, kann eine zielf\"uhrende \"Uberwachungsstrategie entwickelt werden.

          Wenn das Ziel des Auditings das Sammeln von historischen Informationen \"uber die Datenbank ist, sollten folgende Regeln eingehalten werden:
          \begin{itemize}
            \item Auditinginformationen archivieren und den Auditingtrail leeren

              Wurden alle ben\"otigten Informationen gesammelt, sollten diese archiviert und der Auditingtrail geleert werden, damit dieser sp\"ater f\"ur neue \"Uberwachungen zur Ver\-f\"u\-gung steht.
            \item Datenschutzrichtlinien beachten
            
              Das Bundesdatenschutzgesetz (BDSG) regelt sehr genau, auf welche
              Weise Daten erhoben werden dürfen wie lange diese gespeichert
              werden dürfen und wer in welchem Falle Zugriff auf die gesammelten
              Daten hat.
          \end{itemize}
        \subsubsection{\"Uberwachen von verd\"achtigen Aktivit\"aten}
          Beim \"Uberwachen verd\"achtiger Aktivit\"aten sollte wie folgt vorgegangen werden:
          \begin{itemize}
            \item Erst umfassend und dann immer spezieller auditieren

              Wenn verd\"achtige Aktivit\"aten in einem System, wie z. B. einer Datenbank, festgestellt werden, liegen meist nicht viele Informationen \"uber diese vor. Daher muss am Anfang ein umfassenderes Auditing stattfinden, bis alle notwendigen Informationen gesammelt wurden, um die \"Uberwachung auf bestimmte Objekte, Nutzer oder Aktivit\"aten einschr\"anken zu k\"onnen.

              Die Auditinginformationen sollten immer wieder dahingehend ausgewertet werden, ob nicht eine weitere Spezialisierung des Auditings stattfinden kann.
            \item Absichern des Auditingtrails vor unauthorisierten Zugriffen

              Der Auditingtrail sollte vor unerw\"unschten Ver\"anderungen von au\ss erhalb ge\-sch\"u\-tzt werden, z. B. durch eine \"Uberwachung des Auditingtrails.
          \end{itemize}
        \subsubsection{\"Uberwachen administrativer Nutzer}
          Sessions des Nutzers \identifier{sys} oder anderer Nutzer, die sich
          mit den Zus\"atzen \languagesqlplus{as sysdba} oder
          \languagesqlplus{as sysoper} anmelden, werden standardm\"a\ss{}ig
          nicht auditiert. Ist es notwendig, eine \"Uberwachung f\"ur diese
          Nutzer zu aktivieren, geschieht dies mit Hilfe des
          Initialisierungsparameters \parameter{audit\_sys\_operations}
          (statischer Parameter). Der Standardwert f\"ur diesen Parameter ist
          \enquote{false}. Wird dieser Wert auf \enquote{true} ge\"andert,
          aktiviert sich dadurch das Auditing f\"ur administratives Personal.
          \begin{merke}
            Alle Auditinginformationen \"uber den Nutzer \identifier{sys} werden im Auditingtrail des Betriebssystems und nicht in der Datenbank aufbewahrt.
          \end{merke}
      \subsection{Der Audittrail}
        \subsubsection{Informationen im Auditingtrail}
          Ein Audittrail enth\"alt unterschiedliche Informationen, abh\"angig davon, welche Ereignisse \"uberwacht werden und wie das Auditing konfiguriert wurde. Der folgende Literaturhinweis zeigt eine Auflistung der Informationen, die immer im Auditingtrail, sowohl in der Datenbank, also auch im Betriebssystem, gespeichert werden.

          \begin{literaturinternet}
            \item \cite{BCGIDBFI}
          \end{literaturinternet}

          Ist es dem Audittrail nicht mehr m\"oglich neue Eintr\"age aufzunehmen, kann eine \"uberwachte Aktion nicht durchgef\"uhrt werden und bricht mit einer Fehlermeldung ab.

          Der Auditingtrail enth\"alt keine Informationen \"uber Werte, die in einem SQL-Statement verwendet wurden. Wird beispielsweise ein \languageorasql{UPDATE}-Statement \"uberwacht, werden die alten und neuen Werte der Tabelle nicht mit aufgezeichnet. Dies kann jedoch durch das sogenannte \enquote{Fine-Grained-Auditing} erreicht werden.
        \subsubsection{Die Wahl des Auditttrails}
          Welcher Auditingtrail verwendet werden sollte, ist von verschiedenen Faktoren abh\"angig:
          \begin{itemize}
            \item Welche Aktivit\"aten sollen \"uberwacht werden?
            \item Kann Oracle den Auditingtrail des verwendeten Betriebssystems benutzen?
            \item Welcher Auditingtrail kann besser gegen unerw\"unschte Zugriffe gesichert werden?
          \end{itemize}
          Beide Audittrails haben Vor- und Nachteile, die bei der Wahl ber\"ucksichtigt werden sollten. Die folgende Liste ist eine Gegen\"uberstellung der Vorteile beider Trails.

          Datenbankauditingtrail:
          \begin{itemize}
            \item Die Ergebnisse des Auditings k\"onnen, mit Hilfe von SQL oder PL/SQL und den vordefinierten Data Dictionary Views, schnell und einfach, in der Datenbank, analysiert werden.
            \item Es kann sehr einfach ein Report \"uber die Auditinginformationen aus der Datenbank erzeugt werden.
          \end{itemize}
          Betriebssystemauditingtrail:
          \begin{itemize}
            \item Auditinginformationen die im Auditingtrail des Betriebssystems geschrieben wurden, sind unter Umst\"anden sicherer als im Datenbankauditingtrail, da f\"ur den Zugriff auf den Trail des Betriebssystems Berechtigungen ben\"otigt werden, die meist nur ein Systemadministrator hat.
            \item Die Informationen haben eine gr\"o\ss{}ere Verf\"ugbarkeit, da sie auch dann noch erreichbar sind, wenn die Datenbank heruntergefahren wurde.
            \item Die Auditinginformationen k\"onnen als XML-Dateien an einen sicheren Ort im Netzwerk geschrieben werden. Mit Hilfe der View \identifier{v\$xml\_audit\_trail} k\"on\-nen diese Dateien ganz einfach durch SQL-Kommandos abgefragt werden.
            \item Die Nutzung des Betriebssystemauditingtrails konsolidiert alle Auditinginformationen. Es k\"onnen auf diese Art und Weise alle Auditinginformationen, aller Anwendungen an einer Quelle zusammengef\"uhrt werden, was eine zentrale Auswertung  erm\"oglicht.
          \end{itemize}
        \subsubsection{Konfiguration des Auditingtrails}
          Der statische Parameter \parameter{audit\_trail} legt fest, welcher Audittrail genutzt wird. Er kann folgende Werte annehmen:
          \begin{itemize}
            \item \textbf{NONE}: Das Auditing ist deaktiviert, da kein Auditingtrail verwendet wird.
            \item \textbf{DB}: Standardverhalten. Es wird der Auditingtrail der Datenbank benutzt, mit Ausnahme der Informationen, die immer in den Betriebssystemauditingtrail geschrieben werden.
            \item \textbf{DB, EXTENDED}: Es wird der Auditingtrail der Datenbank benutzt und es werden zus\"atzliche Informationen gespeichert.
            \item \textbf{OS}: Alle Auditinginformationen werden in den Betriebssystemauditingtrail geschrieben.
            \item \textbf{XML}: Wie OS, nur das alle Informationen in XML-Dateien geschrieben werden.
            \item \textbf{XML, EXTENDED} Wie XML, aber es werden zus\"atzliche Informationen gespeichert.
          \end{itemize}
          \begin{merke}
            Der Hauptunterschied zwischen den beiden Einstellungen \enquote{DB} und \enquote{XML} zu ihren \enquote{EXTENDED} Varianten ist, dass bei \enquote{EXTENDED} der genaue Wortlaut des \"uberwachten SQL-Statements mitgespeichert wird.
          \end{merke}
          \begin{lstlisting}[caption={Der Parameter \parameter{audit\_trail}},label=admin800,language=oracle_sql]
SQL> ALTER SYSTEM
  2  SET audit_trail=XML, EXTENDED SCOPE=spfile;
          \end{lstlisting}
        \subsubsection{Einen Speicherort f\"ur den Betriebssystemauditingtrail festlegen}
          Wenn der Parameter \parameter{audit\_trail} einen der beiden Werte \enquote{OS} oder \enquote{XML} hat, legt der Parameter \parameter{audit\_file\_dest} ein Verzeichnis auf dem Datentr\"ager fest, in dem die Datenbank ihre Auditinginformationen ablegt. Der \parameter{audit\_file\_dest}-Parameter ist dynamisch und kann mit Hilfe des Kommando \languageorasql{ALTER SYSTEM} wie folgt ge\"andert werden:
          \begin{lstlisting}[caption={Der Parameter \parameter{audit\_file\_dest}},label=admin801,language=oracle_sql]
SQL> ALTER SYSTEM
  2  SET audit_file_dest='/u01/app/oracle' DEFERRED;
          \end{lstlisting}
          \begin{literaturinternet}
            \item \cite{i2282157}
          \end{literaturinternet}

          Der Standardwert f\"ur den Parameter \parameter{audit\_file\_dest} ist:
          \begin{itemize}
            \item Windows: \oscommand{\%ORACLE\_BASE\%\textbackslash admin\textbackslash \%ORACLE\_SID\%\textbackslash adump}
            \item Linux/Unix: \oscommand{\$ORACLE\_BASE/admin/\$ORACLE\_SID/adump}
          \end{itemize}

      \begin{literaturinternet}
        \item \cite{085271}
      \end{literaturinternet}
      \subsection{Standardauditing}
        Unabh\"angig davon, ob das Datenbankauditing aktiviert wurde oder nicht, f\"uhrt Oracle standardm\"a\ss{}ig ein Auditing f\"ur bestimmte Aktivit\"aten durch und schreibt die Informationen in den Auditingtrail des Betriebssystems. Dies wird als \enquote{Standardauditing} bezeichnet. Es handelt sich dabei um die folgenden Aktivit\"aten:
        \begin{itemize}
          \item Verbindungen zur Instanz mit administrativen Privilegien

            Es wird ein Eintrag im Betriebssystemauditingtrail erzeugt, der den Nutzernamen des Betriebssystemnutzers enth\"alt, der sich mit einem der Zus\"atze \languagesqlplus{as sysdba} oder \languagesqlplus{as sysoper} angemeldet hat.
          \item Hochfahren der Instanz

            Es wird ein Eintrag im Betriebssystemauditingtrail erzeugt, der Informationen \"uber den Nutzer enth\"alt, der das Hochfahren der Instanz veranlasst hat.
          \item Herunterfahren der Instanz

            Es wird ein Eintrag im Betriebssystemauditingtrail erzeugt, der Informationen \"uber den Nutzer enth\"alt, der das Herunterfahren der Instanz veranlasst hat.
        \end{itemize}
        \begin{merke}
          Als Auditingtrail wird das System bezeichnet, welches die Auditinginformationen aufnimmt. Dies kann z. B. eine Datenbanktabelle, eine XML-Datei, die Windowsereignisanzeige oder der Linux SysLog Daemon sein.
        \end{merke}
      \subsection{Statementauditing aktivieren}
        \label{statementauditing}
        Um Statementauditing zu konfigurieren, muss das \languageorasql{AUDIT}-Kommando zusammen mit einem Bezeichner aufgerufen werden, der vorgibt, welche Art von Statements zu auditieren sind.
        \begin{merke}
          Zum \"Uberwachen von Systemprivilegien wird das Privileg \privileg{audit system} ben\"otigt. F\"ur Objektauditing wird das Privileg \privileg{audit any} ben\"otigt.
        \end{merke}
        Ein einfaches Beispiel f\"ur Statementauditing ist:
        \begin{lstlisting}[caption={Statementauditing aktivieren},label=admin802,language=oracle_sql]
SQL> AUDIT TABLE;
        \end{lstlisting}
        Dieses Kommando sorgt daf\"ur, dass alle \languageorasql{CREATE TABLE}-, \languageorasql{DROP TABLE}- und \languageorasql{TRUNCATE TABLE}-State\-ments auditiert werden.

        \begin{literaturinternet}
          \item \cite{i2059073}
        \end{literaturinternet}

        Mit Hilfe der \languageorasql{AUDITING BY}-Klausel kann das Auditing nach Nutzern eingeschr\"ankt werden. Um beispielsweise alle \languageorasql{ALTER TABLE}-Statements des Nutzers \identifier{bank} zu auditieren wird das \languageorasql{AUDIT}-Statement wie folgt abgewandelt:
        \begin{lstlisting}[caption={Statementauditing auf Nutzer einschr\"anken},label=admin803,language=oracle_sql]
SQL> AUDIT ALTER TABLE 
  2  BY bank;
        \end{lstlisting}
        Um nur erfolgreiche \languageorasql{ALTER TABLE}-Statements des Nutzers \identifier{bank} zu \"uberwachen, wird zus\"atzlich die Option \languageorasql{WHENEVER SUCCESSFUL} verwendet.
        \begin{lstlisting}[caption={Nur erfolgreiche Statements auditieren},label=admin804,language=oracle_sql]
SQL> AUDIT ALTER TABLE 
  2  BY bank
  3  WHENEVER SUCCESSFUL;
        \end{lstlisting}
        \begin{merke}
          Das Gegenst\"uck zur Option \languageorasql{WHENEVER SUCCESSFUL} ist die Option \languageorasql{WHENEVER NOT SUCCESSFUL} (Standardwert).
        \end{merke}
        Es ist m\"oglich Auditingrichtlinien f\"ur mehrere Nutzer und mehrere
        Ereignisse zu kombinieren. In einem solchen Fall, wird dann eine Liste
        mit Benutzernamen in der \languageorasql{BY}-Klausel angegeben.
        \begin{lstlisting}[caption={Statementauditingoptionen
        kombinieren},label=admin805,language=oracle_sql]  
SQL> AUDIT TABLE, ALTER TABLE 
  2  BY bank, hr 
  3  WHENEVER SUCCESSFUL;
        \end{lstlisting}
        \begin{merke}
          Eine neue Session bezieht ihre Auditingeinstellungen aus dem Data Dictionary. Diese Einstellungen bleiben w\"ahrend der gesamten Lebensdauer der Session erhalten. \"Anderungen an den Auditingeinstellungen werden f\"ur eine Session erst nach einem Neustart wirksam.
        \end{merke}
        Um eine \"Ubersicht dar\"uber zu erhalten, welche
        Statementauditingoptionen derzeit aktiviert sind, kann die View
        \identifier{dba\_stmt\_audit\_opts} abgefragt werden.
\clearpage
        \begin{lstlisting}[caption={Aktvierte Statementauditingoptionen}, label=admin806,language=oracle_sql]
SQL> SELECT user_name, audit_option, success, failure
  2  FROM   dba_stmt_audit_opts;

USER_NAME       AUDIT_OPTION                   SUCCESS    &FAILURE&
--------------- ------------------------------ ---------- ----------
                 &TABLE&                           &BY ACCESS&   &BY ACCESS&
BANK             &ALTER TABLE&                      &BY ACCESS&   &BY ACCESS&
HR               &ALTER TABLE&                      &BY ACCESS&   &NOT SET&
BANK             &TABLE&                           &BY ACCESS&   &NOT SET&
HR               &TABLE&                           &BY ACCESS&   &NOT SET&
        \end{lstlisting}
         \beispiel{admin806} zeigt, dass die Auditingstatements aus den
         Beispielen \beispiel{admin802}, \beispiel{admin803},
         \beispiel{admin804} und \beispiel{admin805} kumuliert werden. F\"ur die
         Eintr\"age drei, vier und f\"unf wurde jeweils die Klausel
         \languageorasql{WHENEVER SUCCESSFUL} angegeben, was am Wert
         \enquote{NOT SET} in der Spalte \identifier{failure} erkennbar ist.
      \subsection{Privilegeauditing aktivieren}
           Das Auditieren von Systemprivilegien funktioniert auf die gleiche Art
           und Weise wie das aktivierern von Statementauditing. Einziger
           Unterschied ist, dass dem \languageorasql{AUDIT}-Statement ein oder
           mehrere Systemprivilegien zur \"Uberwachung \"ubergeben werden
           m\"ussen.
          \begin{lstlisting}[caption={Ein einfaches Privilegienauditing konfigurieren},label=admin807,language=oracle_sql]
SQL> AUDIT &connect&;
          \end{lstlisting}
          Wie beim Statementauditing kann auch hier nach Nutzerkonten eingeschr\"ankt werden.
          \begin{lstlisting}[caption={Privilegienauditing nach Nutzern einschr\"anken},label=admin808,language=oracle_sql]
SQL> AUDIT &connect& 
  2  BY bank;
          \end{lstlisting}
          Die Angabe der Klauseln \languageorasql{WHENEVER SUCCESSFUL} und \languageorasql{WHENEVER NOT SUCCESSFUL} ist ebenfalls m\"oglich.
          \begin{lstlisting}[caption={Nur erfolglose Anmeldungen \"uberwachen},label=admin809,language=oracle_sql]
SQL> AUDIT &connect&
  2  BY bank
  3  WHENEVER NOT SUCCESSFUL;
          \end{lstlisting}
          Die View \identifier{dba\_priv\_audit\_opts} dient dazu, die gesetzten
          Privilegeauditingoptionen auszuwerten.
\clearpage
        \begin{lstlisting}[caption={Aktvierte Statementauditingoptionen}, label=admin810,language=oracle_sql]
SQL> SELECT user_name, privilege, success, failure
  2  FROM   dba_priv_audit_opts;

USER_NAME       PRIVILEGE                                SUCCESS    &FAILURE&
--------------- ---------------------------------------- ---------- ----------
BANK             &CREATE SESSION&                             &NOT SET&     &BY ACCESS&
                 &CREATE SESSION&                             &BY ACCESS&   &BY ACCESS&
        \end{lstlisting}
      \subsection{Objectauditing aktivieren}
        Das Objectauditing ist dem Privilegeauditing sehr \"ahnlich. Statt Systemprivilegien werden Objektprivilegien verwendet und es m\"ussen eines oder mehrere Objekte angegeben werden, auf die sich das Auditing bezieht.
        \begin{lstlisting}[caption={Ein einfaches Objectauditing konfigurieren},label=admin811,language=oracle_sql]
SQL> AUDIT SELECT, INSERT, UPDATE, DELETE ON bank.mitarbeiter
  2  BY ACCESS
  3  WHENEVER NOT SUCCESSFUL;
        \end{lstlisting}
        Ein weiterer Unterschied zum Statement- oder Privilegeauditing ist, dass
        beim Objectauditing keine \languageorasql{AUDITING BY}-Klausel verwendet
        werden kann, was bedeutet, dass nicht nach Nutzern eingeschr\"ankt
        werden kann. Neu beim Objectauditing ist, dass mit Hilfe der Angaben
        \languageorasql{BY SESSION} und \languageorasql{BY ACCESS} gesteuert
        werden kann, ob:
        \begin{itemize}
          \item \textbf{BY ACCESS}: Es wird bei jedem Auftreten eines Ereignisses ein Eintrag im Audittrail erzeugt.
          \item \textbf{BY SESSION}: Nur beim ersten Auftreten eines Ereignisses ein Eintrag im Audittrail erzeugt wird.
        \end{itemize}
        \begin{lstlisting}[caption={Nur eine Warnung pro Session f\"ur ein Ereignis},label=admin812,language=oracle_sql]
SQL> AUDIT select ON bank.Buchung
  2  BY SESSION
  3  WHENEVER SUCCESSFUL;
        \end{lstlisting}
        Auch f\"ur diese Art des Auditings existiert eine View, die eine Auswertung der Auditingoptionen erlaubt, \identifier{dba\_obj\_audit\_opts}. Sie gestaltet sich jedoch g\"anzlich anders, als \identifier{dba\_stmt\_audit\_opts} oder \identifier{dba\_priv\_audit\_opts}.

        In der View \identifier{dba\_obj\_audit\_opts} existiert f\"ur jedes Objektprivileg eine eigene Spalte. Die Spaltenbezeichnungen sind Abk\"urzungen der betroffenen Privilegien. Beispielsweise beinhaltet die Spalte \identifier{alt} die Objectauditingoptions f\"ur das Objektprivileg \privileg{alter} oder die Spalte \identifier{sel} die Optionen f\"ur das \privileg{select}-Privileg.
        \begin{lstlisting}[caption={Die View \identifier{dba\_obj\_audit\_opts}},label=admin813,language=oracle_sql]
 Name                                      Null?    Type
 ----------------------------------------- -------- --------------------
 &OWNER&                                               &VARCHAR2&(30)
 OBJECT_NAME                                        &VARCHAR2&(30)
 OBJECT_TYPE                                        &VARCHAR2&(23)
 ALT                                                &VARCHAR2&(3)
 AUD                                                &VARCHAR2&(3)
 COM                                                &VARCHAR2&(3)
 DEL                                                &VARCHAR2&(3)
 GRA                                                &VARCHAR2&(3)
 IND                                                &VARCHAR2&(3)
 INS                                                &VARCHAR2&(3)
 LOC                                                &VARCHAR2&(3)
 REN                                                &VARCHAR2&(3)
 SEL                                                &VARCHAR2&(3)
 UPD                                                &VARCHAR2&(3)
 REF                                                &CHAR&(3)
 EXE                                                &VARCHAR2&(3)
 CRE                                                &VARCHAR2&(3)
 REA                                                &VARCHAR2&(3)
 WRI                                                &VARCHAR2&(3)
 FBK                                                &VARCHAR2&(3)

        \end{lstlisting}
        Um die Auditingoptionen f\"ur die Objektprivilegien \privileg{select}, \privileg{insert}, \privileg{update} und \privileg{delete} abzufragen zu k\"onnen, m\"ussen die Spalten \identifier{sel}, \identifier{ins}, \identifier{upd} und \identifier{del} in die Abfrage einbezogen werden.
        \begin{lstlisting}[caption={Objectauditingoptions abfragen},label=admin814,language=oracle_sql,alsolanguage=sqlplus]
SQL> SELECT sel, ins, upd, del
  2  FROM   dba_obj_audit_opts;

SEL INS UPD DEL
--- --- --- ---
S/S A/- -/S A/A
        \end{lstlisting}
        Diese Angaben sind wie folgt zu interpretieren:

        Der linke Buchstabe, steht f\"ur die \"Uberwachung erfolgreich verlaufener Ereignisse (Klausel \languageorasql{WHENEVER SUCCESSFUL}). Hier gibt es drei M\"oglichkeiten:
        \begin{itemize}
          \item \textbf{S}: Es ist eine \"Uberwachung erfolgreicher Ereignisse im Modus \languageorasql{BY SESSION} konfiguriert.
          \item \textbf{A}: Es ist eine \"Uberwachung erfolgreicher Ereignisse im Modus \languageorasql{BY ACCESS} konfiguriert.
          \item \textbf{-}: Es wurde keine \"Uberwachung erfolgreicher Ereignisse konfiguriert.
        \end{itemize}
        Der rechte Buchstabe, steht f\"ur eine \"Uberwachung erfolglos verlaufener Ereignisse(Klausel \languageorasql{WHENEVER NOT SUCCESSFUL}). Hier gibt es widerum drei M\"oglichkeiten:
        \begin{itemize}
          \item \textbf{S}: Es ist eine \"Uberwachung erfolgloser Ereignisse im Modus \languageorasql{BY SESSION} konfiguriert.
          \item \textbf{A}: Es ist eine \"Uberwachung erfolgloser Ereignisse im Modus \languageorasql{BY ACCESS} konfiguriert.
          \item \textbf{-}: Es wurde keine \"Uberwachung erfolgloser Ereignisse konfiguriert.
        \end{itemize}
        Das Ergebnis aus \beispiel{admin814} zeigt vier verschiedene Varianten:
        \begin{itemize}
          \item \texttt{sel (S/S)}: F\"ur die Nutzung des \privileg{select}-Privilegs sollen sowohl erfolgreiche, als auch erfolglose Ereignisse, im Modus \languageorasql{BY SESSION} \"uberwacht werden.
          \item \texttt{ins (A/-)}: F\"ur die Nutzung des \privileg{insert}-Privilegs werden nur erfolgreiche Ereignisse, im Modus \languageorasql{BY ACCESS} \"uberwacht. Es erfolgt keine \"Uberwachung erfolgloser Ereignisse.
          \item \texttt{upd (-/S)}: F\"ur das \privileg{update}-Privileg werden nur erfolglose Ereignisse, im Modus \languageorasql{BY SESSION}, \"uberwacht. Es erfolgt keine \"Uberwachung erfolgreicher Ereignisse.
          \item \texttt{del (A/A)}: F\"ur die Nutzung des \privileg{delete}-Privilegs sollen sowohl erfolgreiche, als auch erfolglose Ereignisse, im Modus \languageorasql{BY ACCESS} \"uberwacht werden.
        \end{itemize}
      \subsection{Audittrails auswerten}
        Genauso wichtig oder sogar noch wichtiger als die Auditingoptionen sind die Auditingergebnisse, die in den Audittrails stehen. Wie f\"ur viele andere Dinge, stellt auch hier Oracle entsprechende Views zur Verf\"ugung. Die beiden wichtigsten sind \identifier{dba\_audit\_trail} und \identifier{dba\_common\_audit\_trail}.

        \identifier{dba\_audit\_trail} zeigt den Inhalt des Datenbank\-audittrails f\"ur alle Auditingarten an. Um nach den unterschiedlichen Auditingarten zu untergliedern gibt es noch die folgenden Views:
        \begin{itemize}
          \item \identifier{dba\_audit\_statement}: Enth\"alt alle Statementauditing-Eintr\"age im Datenbank\-audit\-trail.
          \item \identifier{dba\_audit\_object}: Enth\"alt alle Objectauditing-Eintr\"age im Datenbank\-audit\-trail.
          \item \identifier{dba\_audit\_session}: Enth\"alt alle connect und disconnect Eintr\"age im Datenbank\-audit\-trail.
        \end{itemize}
\clearpage
        \begin{merke}
          Statt \identifier{dba\_audit\_trail} direkt abzufragen, sollten die spezialisierten Views genutzt werden, da diese die Informationen des Datenbankauditingtrails \"ubersichtlicher darstellen.
        \end{merke}
        \begin{lstlisting}[caption={Den Datenbankauditingtrail nach connects auswerten},label=admin815,language=oracle_sql,alsolanguage=sqlplus]
SQL> col userhost format a30
SQL> SELECT   username, userhost,
              TO_CHAR(timestamp, 'DD.MM.YYYY HH24:MI') AS Time
  2  FROM     dba_audit_session
  3  ORDER BY timestamp;

USERNAME        USERHOST                       &TIME&
--------------- ------------------------------ ----------------
ALICE           FEA11-119WS03.oracle.com       11.09.2013 12:09
CHLOE           FEA11-119WS03.oracle.com       11.09.2013 12:12
BANK            FEA11-119WS03.oracle.com       18.09.2013 10:59
&SYSTEM&           FEA11-119WS03.oracle.com       18.09.2013 11:35
&SYSTEM&           FEA11-119WS03.oracle.com       18.09.2013 11:44
&SYSTEM&           FEA11-119WS03.oracle.com       18.09.2013 11:55
&SYSTEM&           FEA11-119WS03.oracle.com       18.09.2013 12:00
BANK            FEA11-119CL.oracle.com         14.10.2013 10:47
BANK            FEA11-119CL.oracle.com         14.10.2013 10:49
BANK            FEA11-119CL.oracle.com         14.10.2013 10:57
        \end{lstlisting}
        \beispiel{admin815} zeigt eine Auswertung des Audittrails bez\"uglich connect-Ereignissen.

        Der Betriebssystemaudittrail, der XML-Audittrail und das Fine Grained
        Auditing werden nur in der View \identifier{dba\_common\_audit\_trail}
        angezeigt, was bedeutet, dass die drei oben genannten Views
        \identifier{dba\_audit\_statement}, \identifier{dba\_audit\_object} und
        auch  \identifier{dba\_audit\_session} nicht funktionieren. Um die
        gleiche Auswertung zu erreichen, wie in \beispiel{admin814}, m\"usste
        dann die Spalte \identifier{action} der View
        \identifier{dba\_common\_audit\_trail} genutzt werden, um nach connect
        und disconnect Ereignissen zu filtern.
        \begin{lstlisting}[caption={Einen externen Audittrail auswerten},label=admin816,language=oracle_sql,alsolanguage=sqlplus]
SQL> col db_user format a10
SQL> col userhost format a29
SQL> col action format 999999
SQL> col extended_timestamp format a31
SQL> SELECT   db_user, userhost, action, extended_timestamp
  2  FROM     dba_common_audit_trail
  3  WHERE    action IN (100,101)
  4  ORDER BY extended_timestamp;
        \end{lstlisting}
\clearpage
        \begin{lstlisting}[caption={Einen externen Audittrail auswerten - Fortsetzung},language=oracle_sql,alsolanguage=sqlplus]
DB_USER    USERHOST                      ACTION EXTENDED_TIMESTAMP
---------- ----------------------------- ------ -------------------------------
BANK       FEA11-119CL.oracle.com           101 14.10.13 10:57:52,477400 +02:00
BANK       FEA11-119CL.oracle.com           100 14.10.13 11:03:32,648556 +02:00
BANK       FEA11-119SRV.oracle.com          100 16.10.13 10:54:51,029779 +02:00
BANK       FEA11-119SRV.oracle.com          101 16.10.13 10:54:53,737176 +02:00
BANK       FEA11-119SRV.oracle.com          100 16.10.13 10:54:54,059818 +02:00
BANK       FEA11-119SRV.oracle.com          100 16.10.13 10:54:57,221702 +02:00
BANK       FEA11-119SRV.oracle.com          100 16.10.13 10:54:59,997771 +02:00
BANK       FEA11-119SRV.oracle.com          101 16.10.13 10:55:02,779921 +02:00
        \end{lstlisting}
        Die Werte 100 und 101 der Spalte \identifier{action} stehen f\"ur
        connect (100) und disconnect (101). Sollte es so sein, dass der
        OS-Audittrail oder der XML-Audittrail genutzt werden, empfiehlt es sich
        f\"ur den Administrator, die drei Views
        \identifier{dba\_audit\_statement}, \identifier{dba\_audit\_object} und
        \identifier{dba\_audit\_session} nach zu bauen. Als Beispiel wird hier
        die View \identifier{common\_audit\_session} gezeigt, die alle connect
        und disconnect Ereignisse, auf Basis der View
        \identifier{dba\_common\_audit\_trail}   anzeigt.
        \begin{lstlisting}[caption={Eine eigene View f\"ur dba\_common\_audit\_trail},label=admin817,language=oracle_sql]
SQL> CREATE OR REPLACE VIEW common_audit_session
(DB_USER, USERHOST, ACTION, EXTENDED_TIMESTAMP, EXTENDED_ACTION)
  2  AS
  3    SELECT   db_user, userhost, action, extended_timestamp,
  4             DECODE(action, 100, 'CONNECT',
  5                            101, 'DISCONNECT',
  6                              0, 'CONNECT AS SYSDBA')
  7    FROM     dba_common_audit_trail
  8    WHERE    action IN (0, 100, 101)
  9    ORDER BY extended_timestamp;

View created.
        \end{lstlisting}
      \subsection{Auditing deaktivieren}
        Das Kommando \languageorasql{NOAUDIT} deaktiviert eine konfigurierte Auditingeinstellung wieder. Es k\"on\-nen damit alle Auditingarten deaktiviert werden.

        \begin{merke}
          Das Kommando \languageorasql{NOAUDIT} kennt die beiden Optionen \languageorasql{BY ACCESS} und \languageorasql{BY SESSION} nicht. Deshalb k\"onnen diese nicht verwendet werden.
        \end{merke}
        Im Folgenden werden einige Beispiele f\"ur das \languageorasql{NOAUDIT}-Kommando gezeigt.
\clearpage
        \begin{lstlisting}[caption={Statement- und Privilegeauditing deaktivieren},label=admin818,language=oracle_sql]
SQL> NOAUDIT TABLE;
SQL> NOAUDIT ALTER TABLE BY bank;
SQL> NOAUDIT ALTER TABLE BY bank WHENEVER SUCCESSFUL;
SQL> NOAUDIT connect;
        \end{lstlisting}
        \begin{lstlisting}[caption={Alle AUDIT ALL-Statements
        deaktivieren},label=admin819,language=oracle_sql]
SQL> NOAUDIT ALL;
SQL> NOAUDIT ALL BY bank;
SQL> NOAUDIT ALL BY bank WHENEVER SUCCESSFUL;
        \end{lstlisting}
        \begin{merke}
          Um alle Privilegeauditings zu deaktivieren gibt es das Schl\"usselwort \languageorasql{ALL PRIVILEGES}.
        \end{merke}
        \begin{lstlisting}[caption={Alle Privilegeauditings deaktivieren},label=admin820,language=oracle_sql]
SQL> NOAUDIT ALL PRIVILEGES;
SQL> NOAUDIT ALL PRIVILEGES BY bank;
SQL> NOAUDIT ALL PRIVILEGES BY bank
  2  WHENEVER SUCCESSFUL;
        \end{lstlisting}
        Auch beim Objectauditing gibt es die M\"oglichkeit, alle Auditings direkt zu deaktivieren. Dies geschieht mit dem Statement \languageorasql{NOAUDIT ALL ON DEFAULT}.
        \begin{lstlisting}[caption={Objectauditing deaktivieren},label=admin821,language=oracle_sql]
SQL> NOAUDIT DELETE ON bank.mitarbeiter;
SQL> NOAUDIT SELECT, UPDATE, INSERT ON bank.mitarbeiter;
SQL> NOAUDIT SELECT ON bank.mitarbeiter
  2  WHENEVER SUCCESSFUL;

SQL> NOAUDIT ALL ON bank.mitarbeiter;
SQL> NOAUDIT ALL ON DEFAULT;
        \end{lstlisting}
    \section{Die Nutzerauthentifizierung sicherer gestalten}
      \subsection{Sichere Authentifizierung}
        \subsubsection{Brute Force Protection}
          Mit Oracle 11g kommt ein altes/neues Feature, welches als Schutz vor
          Brute Force Attacken fungiert. Oracle verz\"ogert bei fehlerhaften
          Anmeldeversuchen den n\"achsten Anmeldevorgang. Es gilt:
\clearpage
          \begin{itemize}
            \item W\"ahrend der ersten drei Anmeldeversuche existiert noch keine Verz\"ogerung
            \item Ab dem vierten Versuch wird eine stetig ansteigende Verz\"ogerung (max. 10 Sekunden) eingebaut.
          \end{itemize}
          Ein einfacher Test mit einem Java-Programm zeigt die Auswirkungen dieses neuen Features.
          \begin{lstlisting}[caption={Logon Delay Test},language=terminal]
[oracle@FEA11-119CL ~]$ /usr/java/jdk1.7.0_45/bin/java AuthDelayTest
Connecting with URL=jdbc:oracle:oci8:@ORCL as bank/wrong_password
Versuch: 0
Delay: 0 Sek.
Versuch: 1
Delay: 0 Sek.
Versuch: 2
Delay: 0 Sek.
Versuch: 3
Delay: 1 Sek.
Versuch: 4
Delay: 2 Sek.
Versuch: 5
Delay: 3 Sek.
Versuch: 6
Delay: 4 Sek.
Versuch: 7
Delay: 5 Sek.
Versuch: 8
Delay: 6 Sek.
Versuch: 9
Delay: 7 Sek.
Versuch: 10
Delay: 0 Sek.
Versuch: 11
Delay: 0 Sek.
Versuch: 12
Delay: 0 Sek.
Versuch: 13
Delay: 0 Sek.
          \end{lstlisting}
          Das das Delay ab Versuch Nummer 10 wieder auf 0 zur\"uckgeht, h\"angt
         damit zusammen, dass der Account dann gesperrt ist.
\clearpage
        \subsubsection{max\_failed\_login\_attempts}
          Mit dem Profilparameter \languageorasql{MAX\_FAILED\_LOGIN\_ATTEMPTS}
          ist es m\"oglich, f\"ur valide Nutzerkonten eine maximale Anzahl
          fehlerhafter Anmeldeversuche festzulegen. Ein Konto wird gesperrt,
          sobald das Passwort zu oft falsch eingegeben wurde. Dieser Mechanismus
          sch\"utzt vor Brute Force Attacken, nicht aber vor dem Aussp\"ahen von
          Nutzernamen.

          Mit dem neuen \parameter{sec\_max\_failed\_login\_attempts}-Parameter
          kann die maximale Anzahl fehlerhafter Anmeldeversuche f\"ur einen
          Client konfiguriert werden. Sendet der Client zu h\"aufig eine falsche
          Kombination aus Nutzername und Passwort, wird seine Netzwerkverbindung
          zur Datenbank getrennt. Dieser Initialisierungsparameter bezieht sich
          auch auf Anmeldeversuche mit nicht existenten Nutzerkonten, im
          Unterschied zum Profilparameter
          \languageorasql{MAX\_FAILED\_LOGIN\_ATTEMPTS}, der nur f\"ur existente
          Konten gilt.
          
          Wenn beispielsweise ein Hacker einen Serverprozess startet um mittels
          Brute Force unterschiedliche Kombinationen aus Nutzername und Passwort
          zu testen, kann mittels dieses neuen Initialisierungsparameters
          seine Verbindung zum Serverprozess nach n Versuchen getrennt werden. Ohne diesen
          Parameter k\"onnte der Hacker tausende von Versuchen
          starten, ohne dabei unterbrochen zu werden.
          \begin{lstlisting}[caption={Der Parameter \parameter{sec\_max\_failed\_login\_attempts}},label=admin825,language=oracle_sql]
SQL> ALTER SYSTEM
  2  SET sec_max_failed_login_attempts = 3
  3  SCOPE=spfile;
        \end{lstlisting}
          Wird der Parameter wie in \beispiel{admin825} konfiguriert, bedeutet dies f\"ur alle Clients, dass nach drei falschen Anmeldeversuchen die Connection zum Datenbankserver getrennt wird.

          Dieser Parameter ist statisch!
      \subsection{Sichere Passw\"orter}
        Bis zur Einf\"uhrung von Oracle 11g waren alle Nutzerpassw\"orter,
        inklusive der Administratorenpassw\"orter case insensitiv,
        \enquote{HalloWelt} war gleich \enquote{hallowelt}. Durch den neuen
        Initialisierungsparameter \parameter{sec\_case\_sensitive\_logon} ist es
        nun m\"oglich, Passw\"orter case sensitiv in der Datenbank zu speichern.
        Der Kern dieses neuen Verfahrens ist der SHA1 Hashing Algorithmus.
        Dieser ist im Gegensatz zum 3DES-Algorithmus in der Lage die
        Passw\"orter case sensitiv zum hashen. Zus\"atzlich dazu bringt der
        SHA1-Algorithmus ein Salt mit sich, der die Passwort-Hashes noch
        sicherer macht.
\clearpage
        \begin{lstlisting}[caption={\parameter{sec\_case\_sensitive\_logon}},label=admin826,language=sqlplus]
SQL> show parameter sec_case_sensitive_logon

NAME                              &TYPE&         VALUE
--------------------------------- ------------ ----------------
sec_case_sensitive_logon          boolean      &TRUE&
        \end{lstlisting}
        Eine Stolperfalle, die es zu beachten gilt ist, dass die case-sensitivit\"at von Administratorenpassw\"ortern nicht nur von \parameter{sec\_case\_sensitive\_logon} abh\"angig ist, sondern zus\"atzlich auch davon, wie die Passwortdatei erstellt wurde. Wird die Passwortdatei mit dem Parameter \texttt{ignorecase=y} erstellt, bleibt \parameter{sec\_case\_sensitive\_logon} f\"ur die Administratoren wirkungslos. Nur wenn \texttt{ignorecase=n} ist, haben auch Admins case-sensitive Passw\"orter.
        \begin{merke}
          In Oracle 12c ist gilt dieser Parameter als Deprecated (veraltet), da Passw\"orter in der Version 12c nur noch case sensitiv gespeichert werden.
        \end{merke}
    \section{Verschl\"usselte Tablespaces}
      Seit Oracle 11g ist es m\"oglich, statt der gesamten Datenbank, nur
      einzelne Tablespaces zu verschl\"usseln. Dies hat den Vorteil, dass der
      Administrator gezielt, sensitive Daten verschl\"usseln kann. Der von
      Oracle genutzte Verschl\"usselungsmechanismus wird als Transparent Data
      Encryption (TDE) bezeichnet. Das Schl\"usselwort \enquote{Transparent}
      bedeutet, dass alle Anwendungen, ohne Ver\"anderung, die verschl\"usselten
      Daten nutzen k\"onnen, als seien sie unverschl\"usselt. Die Daten werden
      sogar automatisch verschl\"usselt im Undo Tablespace und in den Redo Logs
      abgelegt, so dass auch dort Datendiebstahl nicht ohne weiteres erfolgen
      kann.

      Oracle benutzt zur Verschl\"usselung Industriestandards, wie AES256,
      AES192, AES128 und 3DES168. Welcher Standard genutzt werden soll, muss bei
      der Erstellung des Tablespaces angegeben werden. Unterschiedliche
      Tablespaces k\"onnen unterschiedliche Verschl\"usselungsverfahren nutzen.
      \begin{merke}
        Das Standardverfahren innerhalb der Oracle-Datenbank ist AES128.
      \end{merke}
      Grundlage f\"ur die Kryptierung ist ein Wallet.
      \subsection{Vorbereiten der Tablespaceverschl\"usselung}
        \subsubsection{Wallets}
          Ein Wallet\footnote{engl. Brieftasche} ist ein verschl\"uselter
          Speicher f\"ur Zertifikate und Schl\"ussel. Es kann f\"ur
          unterschiedliche Zwecke eingesetzt werden, wie z. B. Authentifizierung
          oder Verschl\"usselung. Wird das Wallet zur Verschl\"usselung benutzt,
          spricht man von einem \enquote{Encryptionwallet}.
        \subsubsection{Erstellen eines Walletspeicherortes}
          Um ein Wallet f\"ur die Verschl\"usselung eines Tablespaces nutzen zu
          k\"onnen, m\"uss vorher ein \enquote{Walletspeicher} definiert werden.
          Hierbei handelt es sich um ein Verzeichnis, welches der Datenbank
          bekannt gemacht werden muss. Dazu muss der Walletspeicher in der Datei
          \oscommand{sqlnet.ora} eingetragen werden.
          \begin{lstlisting}[caption={Ein Encryptionwallet
          registrieren},label=admin837,language=configfile]
ENCRYPTION_WALLET_LOCATION =
  (SOURCE =
    (METHOD = FILE)
    (METHOD_DATA =
      (DIRECTORY = /u01/app/oracle/wallets/))
  )
          \end{lstlisting}
          Der \oscommand{sqlnet.ora}-Parameter \languageconfigfile{ENCRYPTION_WALLET_LOCATION} gibt an, wo das Wallet gespeichert werden soll. \languageconfigfile{METHOD = FILE} weisst darauf hin, dass das Wallet als Datei auf dem Dateisystem abgelegt wird. \languageconfigfile{DIRECTORY} gibt den eigentlichen Speicherort an.
        \subsubsection{Erstellen des Encryptionwallets}
          Das Encryptionwallet wird direkt mit SQL erzeugt.
          \begin{lstlisting}[caption={Ein Encryptionwallet erzeugen},label=admin838,language=oracle_sql]
SQL> ALTER SYSTEM SET ENCRYPTION KEY
  2  IDENTIFIED BY "P@ssw0rd";
          \end{lstlisting}
      \subsection{Einen verschl\"usselten Tablespace erstellen}
        \subsubsection{\"Offnen des Wallets}
          Um ein Encryptionwallet nutzen zu k\"onnen, muss es zuerst ge\"offnet werden.
          \begin{lstlisting}[caption={Ein Encryptionwallet \"offnen},label=admin839,language=oracle_sql]
SQL> ALTER SYSTEM
  2  SET ENCRYPTION WALLET OPEN
  3  IDENTIFIED BY "P@ssw0rd";
          \end{lstlisting}
          Nach dem \"Offnen des Wallets kann der Tablespace kreiert werden.
        \subsubsection{Erstellen des Tablespaces}
          Die Erstellung des verschl\"usselten Tablespaces erfolgt mit Hilfe des SQL-Kommandos \languageorasql{CREATE TABLESPACE}.
          \begin{lstlisting}[caption={Einen kryptierten Tablespace mit AES128-Verschl\"usselung erstellen},label=admin840,language=oracle_sql]
SQL> CREATE TABLESPACE crypto_ts
  2  DATAFILE '/u02/oradata/orcl/crypto_ts01.dbf' SIZE 50M
  3  ENCRYPTION
  4  DEFAULT STORAGE(ENCRYPT);

Tablespace created.
          \end{lstlisting}
          Die Zeilen 3 und 4 \languageorasql{ENCRYPTION} und \languageorasql{DEFAULT STORAGE(ENCRYPT)} sorgen f\"ur die Verschl\"usselung. Die \languageorasql{ENCRYPTION}-Klausel kann variiert werden, um einen anderen Verschl\"usselungsalgorithmus zu verwenden.
          \begin{lstlisting}[caption={Einen kryptierten Tablespace mit alternativer Verschl\"usselungsmethode erzeugen},label=admin841,language=oracle_sql]
SQL> CREATE TABLESPACE crypto_ts_aes256
  2  DATAFILE '/u02/oradata/orcl/crypto_ts_aes25601.dbf' SIZE 50M
  3  ENCRYPTION USING 'AES256'
  4  DEFAULT STORAGE(ENCRYPT);

Tablespace created.
          \end{lstlisting}
          So wohl das Erstellen, als auch der Zugriff auf kryptierte Tablespaces funktioniert nur, wenn das Encryptionwallet ge\"offnet wurde.
          \begin{lstlisting}[caption={Einen kryptierten Tablespace mit AES128-Verschl\"usselung erstellen},label=admin842,language=oracle_sql]
SQL> CREATE TABLESPACE crypto_ts_false
  2  DATAFILE '/u02/oradata/orcl/crypto_ts_false01.dbf' SIZE 50M
  3  ENCRYPTION
  4  DEFAULT STORAGE(ENCRYPT);

&\textcolor{red}{ORA-28365: wallet is not open}&
          \end{lstlisting}
          Oracle quittiert den Zugriff ohne ge\"offnetes Wallet mit dem Fehler ORA-28365.
        \subsubsection{Schlie\ss{}en des Wallets}
          Sollte das Wallet f\"ur keinen weiteren Vorgang gebraucht werden, ist es ratsam, das Wallet wieder zu schlie\ss{}en.
          \begin{lstlisting}[caption={Das Encryptionwallet schlie\ss{}en},label=admin843,language=oracle_sql]
SQL> ALTER SYSTEM
  2  SET ENCRYPTION WALLET CLOSE
  3  IDENTIFIED BY "P@ssw0rd";
          \end{lstlisting}
      \subsection{Umgang mit verschl\"usselten Tablespaces}
        \subsubsection{Informationen sammeln}
          Informationen \"uber die Verschl\"usselung und den benutzten
          Algorithmus liefert die View \identifier{v\$encrypted\_tablespaces}.
          \begin{lstlisting}[caption={},label=admin844,language=oracle_sql]
SQL> SELECT t.ts#, t.name, e.encryptionalg, e.encryptedts
  2  FROM   v$tablespace t INNER JOIN v$encrypted_tablespaces e
  3           ON (t.ts# = e.ts#);

       TS# NAME                           ENCRYPT ENC
---------- ------------------------------ ------- ---
        16 CRYPTO_TS                      AES128  YES
        17 CRYPTO_TS_AES256               AES256  YES
          \end{lstlisting}
        \subsubsection{Einschr\"ankungen}
          Im Umgang mit kryptierten Tablespaces existieren drei wesentliche Einschr\"ankungen:
          \begin{itemize}
            \item Ein unverschl\"usselter Tablespace kann nicht im Nachhinein verschl\"usselt werden. Dies kann nur \"uber Umwege, mittels der Oracle Data Pump erfolgen.
            \item Verschl\"usselte Tablespaces k\"onnen nicht ohne Weiteres in eine andere Datenbank umgezogen werden.
            \item Beim Wiederherstellen eines kryptierten Tablespaces muss vor der Recovery-Phase das Wallet ge\"offnet werden, da sonst der Oracle Recovery Manager die Datenbl\"ocke nicht verarbeiten kann.
          \end{itemize}

          \begin{literaturinternet}
            \item \cite{ADMIN12327}
          \end{literaturinternet}
    \section{Informationen}
      \subsection{Verzeichnis der relevanten Initialisierungsparameter}
        \begin{literaturinternet}
          \item \cite{REFRN10019}
          \item \cite{REFRN10274}
          \item \cite{REFRN10299}
        \end{literaturinternet}
      \subsection{Verzeichnis der relevanten Data Dictionary Views}
        \begin{literaturinternet}
          \item \cite{sthref1850}
          \item \cite{sthref1854}
          \item \cite{sthref1856}
          \item \cite{sthref1858}
          \item \cite{sthref2293}
          \item \cite{sthref2337}
          \item \cite{sthref2472}
          \item \cite{sthref2579}
          \item \cite{REFRN30496}
          \item \cite{REFRN30405}
        \end{literaturinternet}
      \subsection{Verzeichnis der Konfigurationsdateien und ihrer Parameter}
        \begin{literaturinternet}
          \item  \cite{NETRF006}
        \end{literaturinternet}
\clearpage

    \section{\"Ubungen - Implementing Security}
  \begin{enumerate}
        \item F\"uhren Sie ein Recovery bei Verlust einer Redo Log Datei durch!
      \begin{enumerate}
        \item Starten Sie das Skript \oscommand{lab\_delete\_redolog\_member.sql}! Es l\"oscht einen beliebigen Member einer Ihrer Redo Log Gruppen.
          \begin{lstlisting}[language=terminal]
SQL> @/home/oracle/labs/lab_delete_redolog_member.sql
          \end{lstlisting}
        \item Die Datenbank l\"auft weiterhin normal und es liegen keinerlei Probleme vor. \"Offnen Sie die Alert Log Datei, um herauszufinden welcher Redo Log Member gel\"oscht wurde!
        \item F\"uhren Sie geeignete Ma\ss{}nahmen zur Problembehebung durch!
      \end{enumerate}

        \item Konfigurieren Sie das Auditing f\"ur die Tabelle \identifier{bank.mitarbeiter} so, dass erfolgreiche \languageorasql{INSERT}- und \languageorasql{UPDATE}-Statements auditiert werden! Es soll jeweils nur ein Auditeintrag pro Session f\"ur jedes  \languageorasql{INSERT}/\languageorasql{UPDATE} gemacht werden.

        \item Wechseln Sie den Undo-Tablespace zur\"uck zum Original Undo-Tablespace und l\"oschen Sie \identifier{undotbs02} und \identifier{undotbs03}.


      \rule{0.94\textwidth}{0.5pt}

        \item Pr\"ufen Sie in welchem Modus der Result Cache l\"auft (Manual oder Force)

        \item Pr\"ufen Sie den Audittrail auf neue Informationen! Welche T\"atigkeiten wurden an der auditierten Tabelle ausgef\"uhrt?


      \rule{0.94\textwidth}{0.5pt}

      \rule{0.94\textwidth}{0.5pt}

        \item Machen Sie alle Auditingeinstellungen r\"uckg\"angig und l\"oschen Sie den Inhalt des Auditingtrails!

        \item Schalten Sie das Auditing f\"ur alle erfolgreichen Anmeldeversuche ein! Lassen Sie diese Auditingeintr\"age im \textit{DB, Extended} Auditingtrail speichern!

        \item F\"uhren Sie ein Recovery mit einem Backup Controlfile durch.
      \begin{enumerate}
        \item Starten Sie das Skript \oscommand{lab\_delete\_all\_controlfiles.sql}. Es wird alle Kontrolldateien l\"oschen.
          \begin{lstlisting}[language=terminal]
SQL> @/home/oracle/labs/lab_delete_all_controlfiles.sql
          \end{lstlisting}
        \item Ergreifen Sie geeignete Ma\ss{}nahmen, um die Datenbank wieder lauff\"ahig zu machen.
      \end{enumerate}

        \item Legen Sie den tempor\"aren smallfile Tablespace \identifier{temp\_ts} an. Sein Tempfile  soll 500 M gro\ss{} sein, auf bis zu 4 G anwachsen k\"onnen, \oscommand{temp\_ts01.dbf} hei\ss{}en und auf Laufwerk \oscommand{/u02} liegen. Benutzen Sie 1 M gro\ss{}e Uniform-Sized Extents.


      \rule{0.94\textwidth}{0.5pt}

        \item L\"oschen Sie das Backup Set des \identifier{bank}-Tablespaces von der Festplatte, so dass es nur noch auf SBT verf\"ugbar ist!

\clearpage
        \item Der Tablespace \identifier{bank} liegt derzeit in unverschl\"usselter Form vor. Dies muss zuk\"unftig anders sein. Bereiten Sie einen Tablespace \identifier{bank\_encrypted} vor, der mittels \identifier{AES256} verschl\"usselung gesichert ist und der die gleichen Dimensionen hat, wie der Originaltablespace \identifier{bank}! Rufen Sie alle notwendigen Informationen \"uber den Tablespace \identifier{bank} aus dem Data Dictionary ab! Welche Views helfen Ihnen dabei?


      \rule{0.94\textwidth}{0.5pt}


      \rule{0.94\textwidth}{0.5pt}


      \rule{0.94\textwidth}{0.5pt}

  \end{enumerate}
\clearpage

    \input{loesungen/dbadmin_12_implementing_security_loesung}
  \input{admin/13_verschieben_von_daten}
    \chapter{Grundlagen des Backup und Recovery}
    \setcounter{page}{1}
    \kapitelnummer{chapter}
    \minitoc
\newpage
    \section{Was ist Backup and Recovery?}
      Im Allgemeinen stehen die Begriffe Backup und Recovery f\"ur verschiedenste Strategien und Arbeitsabl\"aufe, mit deren Hilfe versucht wird, eine Datenbank gegen Datenverlust zu sch\"utzen.
      \subsection{Physische und logische Backups}
        Ein Backup ist eine Kopie der Daten der Datenbank, das dazu genutzt werden kann, die Datenbank wiederherzustellen. Backups werden in physische und logische Backups unterteilt.
        \begin{itemize}
          \item \textbf{Physische Backups}: Dies sind Kopien von Datenbankdateien, d. h. sie enthalten sowohl die Datenbankstruktur, als auch die Daten selbst. Sie k\"onnen in den verschiedensten Formen vorliegen, z. B. komprimiert oder auch als inkrementelles Backup.
          \item \textbf{Logische Backups}: Solche Backups enthalten Nutz- und Metadaten von Datenbank\-objekten, die mit Hilfe von Oracle Utilities aus der Datenbank exportiert wurden. Hierbei handelt es sich nur um Datenexporte, nicht aber um vollst\"andige Backups.
        \end{itemize}
        Physische Backups sind die Grundlage f\"ur jede Backup and Recovery Strategie. Logische Backups stellen in einigen Situationen eine hilfreiche Erg\"anzung zu den physischen Backups dar, sind jedoch kein ernstzunehmender Schutz f\"ur eine Datenbank gegen Datenverlust.

        Dieser Teil der Unterrichtsunterlage k\"ummert sich im Wesentlichen nur um physische Backups. Daher wird mit dem Begriff Backup, wenn nicht anders angegeben, auch immer auf ein physisches Backup verwiesen.
      \subsection{Ursachen die ein Recovery notwendig machen}
        Obwohl es viele Ursachen gibt, die die normale Funktion einer Oracle Datenbank beeintr\"achtigen k\"onnen, gibt es im Wesentlichen nur zwei, die das Eingreifen des Administrators erfordern:
        \begin{itemize}
          \item Medienfehler
          \item Bedienerfehler
        \end{itemize}
        \subsubsection{Bedienerfehler}
          Bedienerfehler liegen immer dann vor, wenn es entweder durch eine fehlerhafte Anwendungslogik oder durch eine direkte Fehleingabe eines Nutzers zu Datenverlust kommt. Dies \"au\ss{}ert sich meist in f\"alschlicherweise ge\"anderten oder gel\"oschten Daten. Obwohl durch Nutzerschulungen, sowie dem vorsichtigen und umsichtigen Umgang mit Privilegien, die meisten Bedienerfehler vermieden werden k\"onnen, bestimmt die Backup und Recovery Strategie, wie einfach und unkompliziert der Administrator die verlorenen Daten wiederherstellen kann.
        \subsubsection{Medienfehler}
          Unter dem Begriff Medienfehler versteht man den Verlust von Daten, aufgrund der Fehlfunktion eines Datentr\"agers. Keine Datenbankdatei ist gegen diese Art von Fehler gesch\"utzt. Die passende Recovery Strategie f\"ur solche F\"alle h\"angt davon ab, welche Datenbankdateien betroffen sind.
    \section{Unterschiedliche Arten physischer Backups mit RMAN}
      Physische Backups k\"onnen nach den unterschiedlichsten Gesichtspunkten unterschieden werden, z. B. in welchem Zustand (ge\"offnet oder geschlossen) war die Datenbank zum Zeitpunkt des Backups, welche Teile der Datenbank wurden gesichert und in welcher Form wird das Backup gespeichert.

      \subsection{Image Copies, Backup Sets und Backup Pieces}
        \subsubsection{Image Copy}
          RMAN (Recovery Manager) kann ein Backup entweder als Image Copy oder als Backup Set speichern. Eine Image Copy ist eine 1:1 Kopie einer Datei, die auch mit Hilfe von Betriebssystemmitteln erstellt werden k\"onnte. Der Vorteil der Nutzung von RMAN bei der Erstellung von Image Copies ist, dass er bei diesem Vorgang den Inhalt der Image Copy auf korrupte Bl\"ocke pr\"ufen kann. Image Copies werden im RMAN Repository registriert.

        \subsubsection{Backup Set}
          Eine andere Form der Speicherung von Backups sind Backup Sets. Ein Backup Set besteht aus mehreren Backup Pieces. Ein Backup Piece ist eine bin\"are Archivdatei, welche mit einer ZIP-Datei verglichen werden kann. Ein mit RMAN durchgef\"uhrter Backup-Job kann eines oder mehrere Backup Sets erzeugen. Backup Sets k\"onnen auch nur durch RMAN wieder zur\"uckgeschrieben werden.

        \begin{merke}
          RMAN kann nur Backup Sets auf Bandlaufwerke \"ubertragen.
        \end{merke}

        \bild{Backup Set und Image Copy}{backupset_and_imagecopy}{0.5}

      \subsection{Konsistente und inkonsistente Backups}
        Physische Backups werden in konsistente Backups (Cold Backup) und inkonsistente Back\-ups (Hot Backup) unterschieden. Ein Backup wird als konsistent bezeichnet, wenn die Datenbank zum Zeitpunkt des Backups in einem konsistenten Zustand war. Das ist dann der Fall, wenn alle \"Anderungen der Redo Logs in die Datendateien \"ubertragen wurden. Dies geschieht nur, wenn die Datenbank ordnungsgem\"a\ss{} heruntergefahren wurde. Eine Datenbank, die aus einem konsistenten Backup wiederhergestellt wurde, kann sofort ohne weiteres Recovery ge\"offnet werden.

        Es k\"onnen aber auch Backups einer ge\"offneten Datenbank durchgef\"uhrt werden. Dabei handelt es sich dann um inkonsistente Backups, sogenannte Hot Backups. Wurde eine Datenbank aus einem inkonsistenten Backup wiederhergestellt, muss in jedem Falle ein Media Recovery durchgef\"uhrt werden, um alle \"Anderungen der Redo Logs in die Datendateien zu \"ubernehmen. Da hier auch archivierte Redo Logs ben\"otigt werden, muss sich die Datenbank im Archivelog-Modus befinden.
      \subsection{Vollst\"andige und inkrementelle Backups}
        Als vollst\"andig werden Backups dann bezeichnet, wenn sie komplette
        Datendateien enthalten. Solche Backups k\"onnen mit dem RMAN oder mit
        Betriebssystemmitteln erzeugt werden. Inkrementelle Backups basieren auf
        der Idee, nur die Datenbl\"ocke einer Datendatei zu sichern, die sich
        seit dem letzten vollst\"andigen Backup ge\"andert haben. Der Vorteil
        dieser Methode ist, dass durch das Zur\"uckschreiben kompletter
        Datenbl\"ocke der Zeitaufwand f\"ur das Zur\"uckspielen der Redo Logs
        erheblich reduziert und somit die gesamte Recovery Phase verringert
        wird. Inkrementelle Backups k\"onnen nur mit RMAN durchgef\"uhrt werden.
\clearpage
    \section{Oracle Backup and Recovery L\"osungen}
      Oracle kennt zwei verschiedene M\"oglichkeiten physische Backups zu erstellen.
      \begin{itemize}
        \item \textbf{Recovery Manager} (RMAN): Der RMAN ist ein Programm, das sowohl \"uber die Kommandozeile als auch \"uber den Enterprise Manager bedient werden kann. Er erstellt eigene Sessions auf dem Datenbankserver, um seine Backup- oder Recovery-Operationen durchzuf\"uhren.

        \item \textbf{User-Managed Backup and Recovery}: Bei dieser Methode wird mit Hilfe von SQL*Plus und Betriebssystemkommandos ein Backup der Datenbank erstellt bzw. zur\"uckgespielt. Hierbei ist der Administrator selbst f\"ur die Verwaltung der Backups zust\"andig.
      \end{itemize}
      Beide Methoden werden von Oracle unterst\"utzt und sind vollst\"andig dokumentiert. Die bevorzugte Variante stellt jedoch der RMAN dar. Er liefert eine einheitliche, betriebssystemunabh\"angige Bedienoberfl\"ache und bietet zu dem M\"oglichkeiten die das User-Managed Backup and Recovery nicht kennt.
      \subsection{Backup and Recovery Features des RMAN}
        Wie bereits erw\"ahnt, besitzt der RMAN einige wesentliche Vorteile gegen\"uber dem User-Managed Backup and Recovery. Die Wissenswertesten davon sind:
        \begin{itemize}
          \item \textbf{Inkrementelle Backups}: Inkrementelle Backups sind kompakter als vollst\"andige Backups und k\"onnen schneller durchgef\"uhrt werden. Sie verk\"urzen auch die Recovery Phase, da durch inkrementelle Backups weniger Redo Log-Informationen zur\"uckgespielt werden m\"ussen.
          \item \textbf{Block media recovery}: Eine Datendatei welche nur einige wenige zerst\"orte Bl\"ocke enth\"alt, kann online repariert werden.
          \item \textbf{Unused block compression}: RMAN kann in bestimmten F\"allen unbenutzte Bl\"ocke bei einem Backup auslassen.
          \item \textbf{Binary compression}: Durch einen integrierten Kompressionsalgorithmus werden Backups komprimiert.
          \item \textbf{Encrypted backups}: Backups k\"onnen verschl\"usselt werden.
        \end{itemize}
\clearpage
        Der RMAN reduziert den administrativen Aufwand bei der Verwaltung von
        Backups, da er ein eigenes Verzeichnis, das \textit{RMAN Repository}
        f\"uhrt, in dem  Informationen \"uber Backups gespeichert werden. RMAN
        kann mit Hilfe dieses Verzeichnisses genau bestimmen, welches das f\"ur
        diese Situation optimalste Backup ist, das zum Restore benutzt werden
        soll. Des Weiteren hat der Administrator die M\"oglichkeit, sich
        Berichte aus dem RMAN Repository ausgeben zu lassen.

        Prim\"ar speichert der RMAN seine ben\"otigten Informationen in der Kontrolldatei der Datenbank. Es kann jedoch auch ein \textit{Recovery Catalog} erstellt werden, um das RMAN Repository zu erweitern. Dies ist ein Datenbankschema, in das der RMAN Informationen \"uber eine oder mehrere Datenbanken speichern kann. F\"ur den RMAN Catalog sollte eine eigenst\"andige Datenbank verwendet werden.
      \subsection{Welche Dateien kann RMAN sichern?}
        RMAN kann alle f\"ur ein Recovery der Datenbank ben\"otigten Dateien sichern. Im Einzelnen sind dies folgende:
        \begin{itemize}
          \item Datendateien und Image Copies von Datendateien
          \item Kontrolldateien und Image Copies von Kontrolldateien
          \item Archivierte Redo Logs
          \item Das aktuelle SPFile
          \item Backup pieces anderer RMAN Backups
        \end{itemize}
        Andere Dateien die f\"ur den Betrieb der Datenbank ben\"otigt werden, wie z. B. Netzwerkkonfigurationsdateien (tnsnames.ora, listener.ora sqlnet.ora) oder die Passwortdatei, k\"onnen nicht durch den RMAN gesichert werden. F\"ur diese Dateien muss ein anderer Backupmechanismus eingesetzt werden.

        Der RMAN kann Backups sowohl auf einem Storage-Laufwerk (Festplatte, RAID usw.), als auch auf Tapes erstellen. Bandlaufwerke werden of auch als SBT\footnote{SBT = System Backup to Tape}-Ger\"ate bezeichnet. RMAN kommuniziert mit SBT-Ger\"aten \"uber einen sogenannten \textit{Media Management Layer}.
    \section{Unterschiedliche Formen des Recovery}
      \subsection{Data File Media Recovery}
        Das Data File Media Recovery oder kurz Media Recovery ist die am h\"aufigsten ben\"otigte Form des Recovery. Es wird durchgef\"uhrt, um verlorene Datendateien, SPFiles oder Kontrolldateien wiederherzustellen. In den folgenden Situation ist ein Media Recovery notwendig:
        \begin{itemize}
          \item Es musste das Backup einer Datendatei wieder hergestellt werden.
          \item Es wurde eine Backup-Kontrolldatei wieder hergestellt.
          \item Eine Datendatei wurde ohne die Option \languageorasql{OFFLINE NORMAL} offline geschaltet
        \end{itemize}
        Damit das Recovery einer Datendatei durchgef\"uhrt werden kann, muss mindestens eine der folgenden Bedingungen erf\"ullt sein:
        \begin{itemize}
          \item Die betreffende Datenbank ist heruntergefahren
          \item Die betreffende Datendatei ist offline, falls die Datenbank nicht heruntergefahren werden kann.
        \end{itemize}
        Eine Datendatei die ein Recovery ben\"otigt, kann erst in den Online-Status gebracht werden, wenn das Recovery abgeschlossen wurde. Eine Datenbank kann nicht ge\"offnet werden, wenn eine ihrer Datendateien ein Recovery ben\"otigt.
        \subsubsection{Die Phasen eines Recovery-Prozesses}
          Das Wiederherstellen von Teilen einer Datenbank oder einer kompletten Datenbank wird in zwei Phasen unterteilt:
          \begin{itemize}
            \item \textbf{Restore}: Unter dem Begriff Restore versteht man das Zur\"uckspielen eines Backups auf den Datentr\"ager.
            \item \textbf{Recovery}: Als Recovery bezeichnet man den Prozess des Aktualisierens der zu\-r\"uck\-ge\-spielten Datenbank\-dateien, mit den Informationen aus den (archivierten) Redo Log Dateien.
          \end{itemize}
          \abbildung{backup_and_recovery_basics} zeigt das Grundprinzip von Backup, Restore und Recovery.

          \bild{Grund\-prin\-zi\-pien des Backup and Recovery}{backup_and_recovery_basics}{1}

          In diesem Beispiel wird bei SCN\footnote{SCN = System Change Number} 75 ein vollst\"andiges Backup der Datenbank gemacht. Die von der Datenbank erzeugten Redo Logs enthalten alle \"Anderungen von SCN 75 bis SCN 666. Gef\"ullte Redo Logs werden archiviert. Bei SCN 666 gehen Datendateien der Datenbank durch einen Medienfehler verloren. Durch ein Restore wird die Datenbank dann auf den Stand von SCN 75 zur\"uckgebracht. Beim anschlie\ss{}enden Recovery, mit Hilfe der Redo Logs und der archivierten Redo Logs, wird die Datenbank wieder auf den Stand von SCN 666 \"uberf\"uhrt.
        \subsubsection{Vollst\"andiges und unvollst\"andiges Recovery}
          Beim Data File Media Recovery unterscheidet man zwei verschiedene Arten:
          \begin{itemize}
            \item Vollst\"andiges Recovery
            \item Unvollst\"andiges Recovery
          \end{itemize}
          \begin{merke}
            Unter einem vollst\"andigen Recovery versteht man das Recovern der gesamten Datenbank oder auch nur von Teilen der Datenbank, bis auf den neuesten Stand. D. h. die Datenbank wird ohne den Verlust von abgeschlossenen Transaktionen wiederhergestellt.
          \end{merke}

          In einigen F\"allen kann es jedoch notwendig sein, die Datenbank bis
          zu einem ganz bestimmten Zeitpunkt zur\"uckzusetzen. Wenn
          beispielsweise durch einen Bedienerfehler eine Tabelle gel\"oscht
          wurde, die noch ben\"otigte Informationen enth\"alt, muss die
          Datenbank bis zu dem Zeitpunkt direkt vor dem Bedienerfehler
          zur\"uckgesetzt werden, so dass die betreffende Tabelle wieder
          existiert.
\clearpage
          Diese Form des Recovery wird als unvollst\"andiges Recovery
          oder als Point-In-Time-Recovery bezeichnet. Ziel dieser Recoveryform
          ist es, die Datenbank auf eine ganz bestimmte SCN oder einen
          bestimmten Zeitpunkt zur\"uckzusetzen, um die Folgen von
          Bedienerfehlern zu beheben.

          Point-In-Time-Recovery stellt auch die einzige Reaktionsm\"oglichkeit dar, wenn der Administrator feststellt, das eines oder mehrere archivierte Redo Logs verloren gegangen sind, die f\"ur ein vollst\"andiges Recovery ben\"otigt w\"urden.

          \begin{merke}
            Bemerkt der Administrator im laufenden Betrieb der Datenbank, dass Redo Log Dateien verloren gegangen sind, sollte augenblicklich ein vollst\"andiges Backup der Datenbank durchgef\"uhrt werden.
          \end{merke}
      \subsection{Instance Recovery / Crash Recovery}
        Wird eine Oracle Datenbank neu gestartet, pr\"uft der SMON, ob ein automatisches Recovery der Datendateien notwendig ist. Ziel dieser Form des Recovery ist es, die Datenbank vor dem \"Offnen auf einen konsistenten Stand zu bringen, ohne abgeschlossene Transaktionen zu verlieren.

        Instance Recovery und Media Recovery haben \"ahnliche Ziele. Es existieren jedoch auch einige Unterschiede zwischen den beiden Formen.
        \begin{itemize}
          \item Media Recovery muss durch einen Nutzer eingeleitet werden, es wird niemals automatisch durchgef\"uhrt.
          \item Media Recovery behandelt nur die durch ein Restore wiederhergestellten Datendateien, jedoch keine Datendateien die w\"ahrend des Crashes online waren.
          \item Media Recovery ben\"otigt die Online Redo Logs und die archivierten Redo Logs um die Datenbank vollst\"andig wiederherstellen zu k\"onnen.
        \end{itemize}
        Im Gegensatz zum Media Recovery ben\"otigt das Instance Recovery nur die Online Redo Logs und es wirkt sich nur auf die Datendateien aus, die w\"ahrend des Neustarts der Datenbank den Status online hatten. Es werden keine archivierten Redo Logs ben\"otigt und es wird auch kein Restore durchgef\"uhrt.

        Die Datenbank rollt beim Instance Recovery alle offenen Transaktionen zur\"uck (Rollback-Phase) und wendet anschlie\ss{}end den Inhalt der Online Redo Logs auf die Datendateien an (Rollforward-Phase). So wird die Datenbank auf den neuesten Stand vor dem Crash gebracht.

    \chapter{Konfigurieren des Recovery Manager}
    \setcounter{page}{1}\kapitelnummer{chapter}
    \minitoc
\newpage
    \section{Die Architektur des Recovery Managers}
        \begin{center}
          \tablecaption{Die Komponenten der RMAN-Umgebung}
          \tablefirsthead{%
            \hline
            \multicolumn{1}{|c|}{\textbf{Komponente}} &
            \multicolumn{1}{|c|}{\textbf{Beschreibung}} &
            \multicolumn{1}{|c|}{\textbf{Ben\"otigt?}}
            \\
            \hline
          }
          \tabletail{%
            \hline
          }
          \begin{small}
            \begin{supertabular}[h]{|p{3cm}|p{10cm}|c|}
              Zieldatenbank & Hierbei handelt es sich um die Datenbank
              (Kontrolldateien, Datendateien und optional auch Archive Log
              Dateien), die durch RMAN gesichert werden soll. RMAN verwendet die
              Kontrolldateien der Zieldatenbank, um Informationen \"uber diese
              zu gewinnen und um eigene Informationen zu dieser Datenbank zu
              speichern. Die Backup- bzw. Recoveryjobs werden durch
              Serverprozesse der Zieldatenbank durchgef\"uhrt. & Ja \\
              \hline
              RMAN Client & Dies ist die Clientanwendung, die die Backup and Recovery Operationen ausf\"uhrt. Da der RMAN Oracle Net-f\"ahig ist, kann er sich von jedem beliebigen Rechner aus zur Zieldatenbank verbinden. & Ja \\
              \hline
              \raggedright Recovery Katalog Datenbank & Als Recovery Catalog wird eine Datenbank bezeichnet, welche die von RMAN ben\"otigten Metainformationen speichert. & Nein \\
              \hline
              \raggedright Recovery Katalog Schema & Dieses Schema wird in der Recovery Katalog Datenbank angelegt und enth\"alt die Metainformationen des RMAN. Es wird regelm\"a\ss{}ig durch RMAN mit der Kontrolldatei der Zieldatenbank synchronisiert. & Nein \\
              \hline
              Media Management Anwendung & Dies sind herstellerspezifische Anwendungen, die es dem RMAN erlauben, Backup Sets auf Streamertapes zu kopieren. & Nein \\
              \hline
              Media Management Katalog & Der Media Management Katalog wird in Zusammenhang mit einer Media Management Anwendung verwendet und speichert Angaben, die RMAN ben\"otigt, um auf ein Tapelaufwerk zugreifen zu k\"onnen. & Nein \\
            \end{supertabular}
          \end{small}
        \end{center}
      \subsection{Der Kommandozeilen Client}
        \subsubsection{Globalization Support Variablen f\"ur RMAN setzen}
          Vor dem Aufruf des Recovery Managers ist es notwendig, dass die beiden Umgebungsvariablen \identifier{nls\_date\_format} und \identifier{nls\_lang} gesetzt werden. Sie beeinflussen die Formatierung von Datums- und Zeitangaben im RMAN. Das folgende Beispiel zeigt m\"ogliche Einstellungen f\"ur beide:
          \begin{lstlisting}[caption={Beispiel f\"ur \identifier{nls\_date\_format} und \identifier{nls\_lang}},label=admin1000,language=terminal]
-- Ohne Zeichensatz
[oracle@FEA11-119SRV ~]$ export NLS_LANG=german_germany

-- Mit Zeichensatz
[oracle@FEA11-119SRV ~]$ export NLS_LANG=german_germany.UTF8

[oracle@FEA11-119SRV ~]$ export NLS_DATE_FORMAT='DD.MM.YYYY HH24:MI:SS'
          \end{lstlisting}
\clearpage
          \begin{merke}
            Bei der Angabe eines Zeichensatzes in der Variablen \identifier{nls\_lang} ist immer der Zeichensatz zu verwenden, den das Betriebssystem aktuell in Nutzung hat. Unter Linux wird standardm\"assig der WE8ISO8859P1 Zeichensatz genutzt, unter Windows der WE8MSWIN1252.
          \end{merke}
          Der folgende Literaturhinweis liefert weiterf\"uhrende Informationen zur Konfiguration der National Language Support Umgebung!
          \begin{literaturinternet}
            \item \cite{NLSPG189}
          \end{literaturinternet}
        \subsubsection{Starten von RMAN}
          RMAN wird auf der Kommandozeile durch die Eingabe von \oscommand{rman} gestartet.
          \begin{lstlisting}[caption={Starten des RMAN},label=admin1001,language=terminal]
[oracle@FEA11-119SRV ~]$ rman target /
[oracle@FEA11-119SRV ~]$ rman target sys/oracle@ORCL NOCATALOG
[oracle@FEA11-119SRV ~]$ rman target / catalog rman/cat@CATDB
[oracle@FEA11-119SRV ~]$ rman target / catalog rman/cat@CATDB \
> auxiliary sys/oracle@AUXDB as sysdba
          \end{lstlisting}
          \begin{merke}
            Das Schl\"usselwort \oscommand{target} gibt an, dass sich RMAN mit der Zieldatenbank verbinden soll. Um sinnvoll arbeiten zu k\"onnen, muss sich RMAN mit einer Zieldatenbank verbunden haben.
          \end{merke}

          In den beiden ersten Zeilen wird darauf verzichtet, sich mit einem Recovery Katalog zu verbinden. Das Schl\"usselwort \oscommand{NOCATALOG} dr\"uckt dies explizit aus, es kann jedoch auch weggelassen werden. Zeile Nummer drei verbindet RMAN mit einer Zieldatenbank und zus\"atzlich mit einem Recovery Katalog.

          F\"ur spezielle Arbeiten, wie z. B. das Duplizieren einer Datenbank oder das Aufbauen einer Standby Datenbank, kann es erforderlich sein, sich zus\"atzlich mit einer Hilfsdatenbank\footnote{Auxiliary Database = engl. Hilfsdatenbank} zu verbinden. Die meisten T\"atigkeiten erfordern jedoch nur, dass RMAN sich mit einer Zieldatenbank verbindet. Die Verbindung zu einem Recovery Katalog ist grunds\"atzlich optional.

          Um den RMAN zu verlassen wird das Kommando \languagerman{exit} verwendet.
\clearpage
        \subsubsection{Verbindungen und Authentifizierung}
          \begin{merke}
            Um sich im RMAN mit einer Zieldatenbank oder einer Hilfsdatenbank verbinden zu k\"onnen, be\-n\"o\-tigt der Nutzer das \privileg{SYSDBA}-Privileg. Die Authentifizierung kann dabei \"uber eine Passwortdatei oder das Betriebssystem erfolgen.
          \end{merke}
          Da das Verbinden mit einer Ziel- oder Hilfsdatenbank immer das \privileg{SYSDBA}-Privileg voraussetzt, ist es im RMAN, anders als bei SQL*Plus, nicht notwendig die Klausel \languagesqlplus{as sysdba} mit anzugeben.

          Verwendet die Zieldatenbank eine Passwortdatei, kann zur Authentifizierung im RMAN ein Passwort verwendet werden. Soll die Be\-triebs\-sys\-tem\-authen\-ti\-fi\-zie\-rung in\-ner\-halb von RMAN genutzt werden, kann die Umgebungsvariable \identifier{oracle\_sid} gesetzt werden. Sie muss den Instanznamen der Zieldatenbank enthalten.
        \subsubsection{Eingeben von Kommandos}
          Wenn der RMAN gestartet wurde, wird folgendes Commandprompt angezeigt:
          \begin{lstlisting}[caption={Das RMAN-Commandprompt},label=admin1002,language=rman]
RMAN>
          \end{lstlisting}
          Ab diesem Zeitpunkt k\"onnen beliebige RMAN-Kommandos eingegeben werden:
          \begin{lstlisting}[caption={Beispiel f\"ur einige RMAN-Kommandos},label=admin1003,language=rman]
RMAN> CONNECT target /
RMAN> CONNECT catalog rman/cat@CATDB

RMAN> BACKUP database;
          \end{lstlisting}
          Fast alle RMAN-Kommandos akzeptieren einen oder mehrere Parameter und m\"ussen mit einem Semikolon ';' abgeschlossen werden. Wenige Ausnahmen, wie z. B. \languagesqlplus{startup}, \languagesqlplus{shutdown} oder \languagesqlplus{connect} k\"onnen ohne Semikolon benutzt werden.

          Ein RMAN-Kommando kann auf mehrere Zeilen verteilt werden.
          \begin{lstlisting}[caption={Ein mehrzeiliges RMAN-Kommando},label=admin1004,language=rman]
RMAN> BACKUP database
2>    INCLUDE current controlfile;
          \end{lstlisting}
        \subsubsection{RMAN zum Starten und Herunterfahren der Datenbank benutzen}
          Wenn ein Vorgang es erfordert, dass die Zieldatenbank gestartet, heruntergefahren, in den MOUNT- oder NOMOUNT-Status versetzt wird, kann der RMAN die entsprechende \"Anderung an der Datenbank herbeif\"uhren. Im folgenden Beispiel werden ein Shutdown, ein Startup und einige andere SQL-Statements durchgef\"uhrt:
          \begin{lstlisting}[caption={Startup und Shutdown im RMAN},label=admin1005,language=rman,alsolanguage=sqlplus]
[oracle@FEA11-119SRV ~]$ rman target /

RMAN> shutdown immediate
RMAN> startup nomount
RMAN> SQL 'ALTER DATABASE NOMOUNT';
RMAN> SQL 'ALTER DATABASE MOUNT';
RMAN> SQL 'ALTER DATABASE OPEN';
        \end{lstlisting}
      \subsection{Das RMAN Repository}
        Als RMAN Repository wird eine Sammlung von Metadaten bezeichnet, die der RMAN f\"ur Backup, Recovery und Wartungsarbeiten ben\"otigt. RMAN speichert dieses Repository immer in der Kontrolldatei der Zieldatenbank. Von der Korrektheit der Kontrolldatei h\"angt es ab, welchen Stand der Datenbank der RMAN kennt und welche Backups er wiederherstellen kann. Dies ist einer der Gr\"unde, warum die Kontrolldatei eine besondere Rolle bei der Erstellung einer Backupstrategie spielt. RMAN kann, allein mit Hilfe der Kontrolldatei alle notwendigen Backup und Recovery Operationen ausf\"uhren.

        Optional kann ein Recovery Katalog erstellt werden. Dies ist ein Schema in einer Oracle-Datenbank, welches die gleichen Informationen speichert wie die Kontrolldatei. Im Unterschied zu einer Kontrolldatei, deren Eintr\"age im RMAN Repository nur eine begrenzte Lebensdauer haben, ehe sie \"uberschrieben werden, kann der Recovery Katalog diese Informationen unbegrenzt speichern. Die erh\"ohte Komplexit\"at die durch die Verwaltung eines Recovery Katalogs entsteht, kann schnell durch die Bequemlichkeit ersetzt werden, die er bietet.

        Einige Features des RMAN funktionieren nur mit Hilfe eines Recovery Katalogs, z. B. \enquote{RMAN Stored Scripts} und alle Kommandos, die mit solchen Skripten in Verbindung stehen. Andere RMAN Kommandos sind speziell f\"ur die Verwaltung eines Recovery Katalogs und werden nicht ben\"otigt, wenn keiner verwendet wird.

        Der Recovery Katalog wird durch den RMAN selbst verwaltet. Die Zieldatenbank greift niemals in ihn ein. RMAN gibt automatisch alle notwendigen Informationen, aus der Kontrolldatei der Zieldatenbank an den Recovery Katalog weiter, falls eine \"Anderung eintritt.
        \subsubsection{Das RMAN Repository in der Kontrolldatei}
          Das RMAN Repository kennt zwei Eintragsarten: Circular Reuse Records und Noncircular Reuse Records.

          Circular Reuse Records enthalten Daten, die einer hohen \"Anderungsh\"aufigkeit unterliegen und bei Bedarf \"uberschrieben werden k\"onnen. Sie sind als \enquote{logischer Ring organisiert}, was bedeutet, dass sobald alle Speicherpl\"atze belegt sind, entweder neuer Platz durch eine Erweiterung der Kontrolldatei geschaffen wird oder das bestehende Speicherpl\"atze \"uberschrieben werden. \parameter{control\_file\_record\_keep\_time} ist der Initialisierungsparameter, der vorgibt, nach wie vielen Tagen ein Eintrag \"uberschrieben werden kann.

          Non Circular Records enthalten Informationen, die f\"ur den Betrieb der Datenbank wichtig sind. Deshalb k\"onnen diese nicht durch den RMAN \"uberschrieben werden. Dazu z\"ahlt beispielsweise, welche Daten- und Redo Log Dateien eine Datenbank enth\"alt.
        \subsubsection{Das RMAN Repository im Recovery Katalog}
         Wird ein RMAN Katalog benutzt, sollte er sich nicht in der Zieldatenbank befinden. Wird er  dennoch dort gespeichert, ist er verloren, wenn die Zieldatenbank verloren geht, was das Recovery der Zieldatenbank deutlich erschwert.

          Um den Recovery Katalog mit einer Zieldatenbank zu verkn\"upfen, muss diese bei ihm registriert werden. Es k\"onnen beliebig viele Datenbanken registriert werden. RMAN unterscheidet die einzelnen Datenbanken anhand einer DBID.

          \begin{merke}
            Jede Oracle-Datenbank besitzt eine eigene, eindeutige DBID. Diese kann der View \identifier{v\$database} entnommen werden.
          \end{merke}

          Im Einzelnen beinhaltet der Recovery Katalog Informationen \"uber folgende Dinge:
          \begin{itemize}
            \item Backup Sets und Pieces von Datendateien, Redo Logs und Archive Logs
            \item Kopien von Datendateien
            \item Kopien von Archive Logs
            \item Tablespaces und Datendateien der Zieldatenbank(en)
            \item RMAN stored scripts
            \item Dauerhafte Konfigurationseinstellungen des RMAN
          \end{itemize}
        \subsubsection{Resynchronisieren des Recovery Katalogs}
          Damit der Recovery Katalog aktuell bleibt, muss er immer wieder mit dem RMAN Repository resynchronisiert werden. Dies geschieht mit Hilfe eines \enquote{Snapshot Controlfile}, dass immer dann durch den RMAN automatisch erzeugt wird, wenn eine Resynchronisation notwendig ist. Es stellt eine konsistente Momentaufnahme der Kontrolldatei dar. Da solche Snapshots nur kurzzeitig verwendet werden, werden sie nicht im Recovery Katalog registriert. RMAN speichert die Checkpoint SCN aus dem Snapshot Controlfile im Recovery Katalog, um die Aktualit\"at des Katalogs sicherzustellen.

          Die Datenbank garantiert, das immer nur ein RMAN-Prozess auf ein Snapshot Controlfile zugreift. Dies erm\"oglicht, dass sich unterschiedliche RMAN-Prozesse nicht gegenseitig st\"oren.
    \section{Konfigurieren des RMAN}
      Die Konfiguration des RMAN ist einfach und unkompliziert, da f\"ur fast alle Parameter Standardwerte existieren. Dadurch ist es m\"oglich, ohne manuelle Konfiguration direkt mit dem RMAN zu arbeiten.

      Um den RMAN jedoch effizienter nutzen zu k\"onnen, ist es notwendig die wichtigsten Optionen des RMAN zu kennen. RMAN-Parameter k\"onnen mit dem Kommando \languagerman{CONFIGURE} gesetzt und gespeichert werden, so dass sie nicht bei jedem Backup erneut ge\"andert werden m\"ussen. Das Kommando \languagerman{SHOW} erm\"oglicht die Ausgabe aller Optionen.
      \subsection{Die aktuelle RMAN-Konfiguration anzeigen}
        Mit Hilfe des Kommandos \languagerman{SHOW} k\"onnen die aktuellen Werte der RMAN-Parameter angezeigt werden.
        \begin{lstlisting}[caption={Das SHOW-Kommando},label=admin1006,language=rman]
RMAN> SHOW retention policy;
RMAN> SHOW device type;
RMAN> SHOW default device type;
RMAN> SHOW channel;

RMAN> SHOW all;
        \end{lstlisting}
        Das Kommando \languagerman{SHOW all} zeigt alle Optionen als Serie von \languagerman{CONFIGURE}-Kommandos. Sie kann in eine Textdatei gespeichert werden, die dann dazu dienen kann, die Konfiguration f\"ur eine andere Datenbank einzurichten. Das Spooling in eine Textdatei wird mit Hilfe des \languagerman{SPOOL LOG TO} \oscommand{Dateiname}-Kommandos aktiviert und mit \languagerman{SPOOL LOG OFF} wieder deaktiviert.
        \begin{lstlisting}[caption={Ausgabe des Kommandos SHOW-ALL},label=admin1007,language=rman]
RMAN> SPOOL LOG TO '/home/oracle/rman.cfg';
RMAN> SHOW ALL;
RMAN> SPOOL LOG OFF;
        \end{lstlisting}
      \subsection{Anpassen der RMAN-Konfiguration}
        \subsubsection{Das Backup-Standardger\"at festlegen}
          Standardm\"a\ss{}ig legt der RMAN alle Backups in einem Verzeichnis auf der Festplatte ab. Er kann aber auch so konfiguriert werden, dass er seine Backups auf ein SBT-Ger\"at speichert.

          Nach der Konfiguration des SBT-Ger\"ates, kann es im RMAN zum Backup-Standardger\"at gemacht werden:
          \begin{lstlisting}[caption={Konfigurieren des Backup-Standardger\"ats},label=admin1008,language=rman]
RMAN> CONFIGURE DEFAULT DEVICE TYPE TO sbt;
          \end{lstlisting}
          Anschlie\ss{}end werden alle Backups, bei denen kein Backupger\"at angegeben wurde, auf das Backup-Standardger\"at gespeichert. Um eine Festplatte als Backup-Standardger\"at einzurichten, muss dem \languagerman{DEFAULT DEVICE TYPE}-Parameter der Wert \enquote{disk} gegeben werden:
          \begin{lstlisting}[caption={Konfigurieren des Backup-Standardger\"ats auf Festplatte},label=admin1009,language=rman]
RMAN> CONFIGURE DEFAULT DEVICE TYPE TO disk;
          \end{lstlisting}
          Auch ohne das Einrichten eines Backup-Standardger\"ats, k\"onnen Backups auf SBT-Ger\"at oder Festplatte gespeichert werden. Das folgende Beispiel zeigt zwei unterschiedliche
          \languagerman{BACKUP}-Kommandos:
          \begin{lstlisting}[caption={Backup-Beispiele},label=admin1010,language=rman]
RMAN> BACKUP DEVICE TYPE SBT database;
RMAN> BACKUP DEVICE TYPE DISK database;
RMAN> BACKUP database;
          \end{lstlisting}
          Das erste Kommando \languagerman{BACKUP DEVICE TYPE SBT DATABASE} speichert das Backup der Zieldatenbank auf ein SBT-Ger\"at. Das zweite sichert das Backup auf Festplatte, da als Device Type der Wert \enquote{disk} vorgegeben wurde. Das dritte \languagerman{BACKUP database}-Kommando benutzt das Backup-Standardger\"at.
          \begin{merke}
            Parameter, die einem Kommando, wie z. B. \languagerman{BACKUP}, \languagerman{RESTORE} oder \languagerman{RECOVER} mitgegeben werden, haben immer Vorrang vor einer Einstellung, die mit \languagerman{CONFIGURE} vorkonfiguriert wurde.
          \end{merke}
          \begin{lstlisting}[caption={\"Uberschreiben einer Konfigurationseinstellungen},label=admin1011,language=rman]
RMAN> CONFIGURE DEFAULT DEVICE TYPE TO sbt;
RMAN> BACKUP database
2>    DEVICE TYPE disk;
          \end{lstlisting}
        \subsubsection{Konfigurieren des Standard-Backuptyps}
          Der Standard-Backuptyp legt fest, welche Art von Backup gemacht wird, wenn nichts spezielles definiert wurde. Zur Auswahl stehen:
          \begin{itemize}
            \item Unkomprimierte Backup Sets
            \item Komprimierte Backup Sets
            \item Image Copies
          \end{itemize}
          \begin{merke}
            Wurde hier nichts abweichendes konfiguriert, erstellt RMAN unkomprimierte Backup Sets. Es existiert keine M\"oglichkeit Image Copies f\"ur ein SBT-Ger\"at zu konfigurieren, da RMAN nur Backup Sets auf SBT-Ger\"ate schreiben kann.
          \end{merke}
          \begin{lstlisting}[caption={Standard-Backuptyp festlegen},label=admin1012,language=rman]
RMAN> CONFIGURE DEVICE TYPE disk
2>    BACKUP TYPE TO copy;
RMAN> CONFIGURE DEVICE TYPE disk
2>    BACKUP TYPE TO backupset;

RMAN> CONFIGURE DEVICE TYPE disk
2>    BACKUP TYPE TO compressed backupset;
RMAN> CONFIGURE DEVICE TYPE sbt
2>    BACKUP TYPE TO compressed backupset;
          \end{lstlisting}
          Komprimierte Backup Sets k\"onnen sowohl auf Festplatte, als auch auf einem SBT-Ger\"at gespeichert werden.
      \subsection{RMAN auf seine Standardkonfiguration zur\"ucksetzen}
        Jeder RMAN-Parameter kann auf seinen Standardwert zur\"uckgesetzt werden. Dies geschieht mit dem RMAN-Kommando \languagerman{CONFIGURE ... CLEAR}.
        \begin{lstlisting}[caption={CONFIGURE ... CLEAR},label=admin1013,language=rman]
RMAN> CONFIGURE BACKUP OPTIMIZATION CLEAR;
RMAN> CONFIGURE RETENTION POLICY CLEAR;
RMAN> CONFIGURE CONTROLFILE AUTOBACKUP FORMAT FOR DEVICE TYPE DISK CLEAR;
        \end{lstlisting}
    \section{Kan\"ale f\"ur RMAN konfigurieren}
      Ein Kanal im Sinne von RMAN ist ein Datenstrom zwischen einem Speicherger\"at und einem Serverprozess. Die meisten RMAN-Kommandos werden mit Hilfe eines Serverprozesses ausgef\"uhrt. Wie in \abbildung{rman_channel} zu sehen ist, erstellt jeder Kanal eine Verbindung zu einer Zieldatenbank, in dem er dort einen Serverprozess startet. Der Serverprozess f\"uhrt dann die Backup-, Restore- oder Recovery-T\"atigkeiten durch.
      \bild{RMAN und Channels}{rman_channel}{1}
      Kan\"ale k\"onnen in RMAN auf zwei unterschiedliche Arten erstellt werden: automatisch oder manuell.
      \subsection{Die manuelle Kanalanforderung}
        In RMAN ist es m\"oglich Kan\"ale manuell, f\"ur ganz bestimmte Zwecke anzufordern. Dies geschieht durch das \languagerman{ALLOCATE}-Kommando, wie in \beispiel{admin1014} gezeigt wird.
        \begin{lstlisting}[caption={Manuelle Kanalanforderung},label=admin1014,language=rman]
RMAN> RUN
2>    {
3>      ALLOCATE CHANNEL c1 DEVICE TYPE disk;
4>      BACKUP database PLUS ARCHIVELOG;
5>    }
        \end{lstlisting}
        Die Zeile \languagerman{ALLOCATE CHANNEL c1 DEVICE TYPE disk} fordert einen Kanal namens \identifier{c1} an. Dieser liest seine Daten von einem Disk-Ger\"at. Zu beachten ist an dieser Stelle der RUN-Block.

        \begin{merke}
          RUN-Bl\"ocke sind vergleichbar mit Prozeduren/Methoden aus der Programmierung. Sie stellen eine in sich geschlossene Einheit dar, die eine Abfolge von Befehlen enth\"alt, welche sequenziell abgearbeitet wird. F\"ur die manuelle Kanalanforderung ist es zwingend notwendig, einen RUN-Block zu benutzen.
        \end{merke}

        RMAN benutzt den angeforderten Kanal f\"ur alle Operationen, die innerhalb des RUN-Blocks definiert sind. Nach dem der RUN-Block abgearbeitet worden ist, wird der Kanal automatisch wieder geschlossen.
      \subsection{Die automatische Kanalanforderung}
        Seit Oracle 10g ist es nicht mehr zwingend erforderlich, Kan\"ale manuell anzufordern. Der RMAN h\"alt standardm\"a\ss{}ig immer einen generischen Kanal f\"ur alle Operationen bereit.
        \begin{merke}
          Ein Kanal ist generisch, wenn er nur auf den Standardeinstellungen von RMAN beruht. Das hei\ss{}t, er wird z. B. durch das Standardbackupger\"at beeinflusst.
        \end{merke}
        \subsubsection{Kan\"ale vordefinieren}
          Mit Hilfe des Kommandos \languagerman{CONFIGURE CHANNEL} k\"onnen die Konfigurationseinstellungen f\"ur Kan\"ale ge\"andert werden. Das folgende Beispiel zeigt, wie f\"ur einen Kanal der Speicherort (das Format) konfiguriert wird.
          \begin{lstlisting}[caption={Vordefinieren eines Channels mit Speicherort},label=admin1015,language=rman]
RMAN> CONFIGURE CHANNEL DEVICE TYPE disk
2>    FORMAT '/u04/backup/ora_df%t_%s%p.bkp';
          \end{lstlisting}
          Dieses Kommando legt als Speicherort f\"ur den Channel das Verzeichnis\ \oscommand{/u04/backup/} fest. Das Format f\"ur den Dateinamen der Backups gliedert sich wie folgt:
          \begin{itemize}
            \item Der Dateiname beginnt mit den Buchstaben \oscommand{ora\_df}.
            \item Der Platzhalter \oscommand{\%t} wird durch einen Zeitstempel ersetzt.
            \item Der Platzhalter \oscommand{\%s} wird mit der Nummer des Backup Sets ersetzt.
            \item Der Platzhalter \oscommand{\%p} wird mit der Nummer des Backup Piece ersetzt.
          \end{itemize}

          \begin{merke}
            Zu beachten bei der Konfiguration eines expliziten Speicherorts f\"ur Backup Sets ist: Wird ein expliziter Speicherort f\"ur Backup Sets konfiguriert, werden die Backup Sets au\ss{}erhalb der Fast Recovery Area (siehe \ref{configureflashrecoveryarea}) gespeichert. Damit geht die gesamte Funktionalit\"at der Fast Recovery Area verloren.
          \end{merke}
          Ein anderes Beispiel f\"ur die Konfiguration eines Kanals zeigt \beispiel{admin1016}.
          \begin{lstlisting}[caption={Ein vordefinierter Channel mit Backup Piece Size},label=admin1016,language=rman]
RMAN> CONFIGURE CHANNEL DEVICE TYPE disk
2>    MAXPIECESIZE = 500M;
          \end{lstlisting}
          In diesem Beispiel wird die Maximalgr\"o\ss{}e f\"ur die einzelnen Backup Pieces auf 500 Megabyte festgelegt.  Jedes \"uber diesen Kanal angefertigte Backup Set, wird automatisch in 500 Megabyte St\"ucke, sogenannte Backup Pieces, zerlegt.

          \bild{Ein Backup Set, das aus drei Backup Pieces besteht}{backuppieces}{0.55}

        \subsubsection{Optimierung automatisch angeforderter Kan\"ale}
          RMAN optimiert die Nutzung von automatisch angeforderten Kan\"alen dahin gehend, dass ein solcher Kanal nur so lange offen bleibt, wie er ben\"otigt wird.

          Im folgenden Beispiel sind drei Backup-Operationen zu sehen. Die Erste \"offnet den vordefinierten Kanal. Die zweite und die dritte Operation ben\"otigen einen exakt genauso konfigurierten Kanal, wie die erste Operation. Deshalb wird der Kanal nicht jedes mal neu ge\"offnet, sondern bleibt offen.

          \begin{merke}
            Erst nach der Benutzung eines der beiden Kommandos \languagerman{ALLOCATE} oder \languagerman{CONFIGURE} wird der vordefinierte Kanal wieder geschlossen.
          \end{merke}
          \begin{lstlisting}[caption={Optimierung der automatischen Kanalanforderung},label=admin1017,language=rman]
RMAN> BACKUP datafile 1; --Kanal wird geoeffnet
RMAN> BACKUP current controlfile; -- Kanal wird erneut genutzt
RMAN> BACKUP archivelog all; -- Kanal wird weiter genutzt und bleibt offen
RMAN> CONFIGURE DEFAULT DEVICE TYPE TO disk; -- Der Kanal wird geschlossen
          \end{lstlisting}
      \subsection{Parallelisierung bei der Kanalanforderung}
        Mit Hilfe der Parallelisierung k\"onnen, in einem Arbeitsschritt, mehrere gleichartige Kan\"ale synchron angefordert werden.
        \subsubsection{Parallelisierung manuell angeforderter Kan\"ale}
          Bei der manuellen Kanalanforderung ist die Parallelisierung sehr einfach einzurichten. Im folgenden Beispiel werden zwei Kan\"ale angefordert, \oscommand{c1} und \oscommand{c2}. Da  mehrere Dateien gesichert werden m\"ussen, kann RMAN die Parallelisierung nutzen und die Arbeit wird auf beide Kan\"ale verteilt.
\clearpage
          \begin{lstlisting}[caption={Parallelisierung und die manuelle Kanalanforderung},label=admin1018,language=rman]
RMAN> RUN {
2>      ALLOCATE CHANNEL c1 DEVICE TYPE disk;
3>      ALLOCATE CHANNEL c2 DEVICE TYPE disk;
4>      BACKUP database PLUS ARCHIVELOG;
5>    }
          \end{lstlisting}
          \begin{merke}
            RMAN nutzt immer so viele Kan\"ale, wie er f\"ur seine aktuelle Operation ben\"otigt. Sind zwei Kan\"ale angefordert und RMAN kann nur einen f\"ur seine Arbeit nutzen (z. B. weil nur eine Datei gesichert werden soll), nutzt er auch nur den ersten angeforderten Kanal.
          \end{merke}
        \subsubsection{Parallelisierung automatisch angeforderter Kan\"ale}
          In seiner Standardeinstellung hat RMAN nur einen einzigen Kanal zur Verf\"ugung. Um mehrere Kan\"ale zu nutzen, muss die Parallelisierung aktiviert werden. \beispiel{admin1019} zeigt das Kommando zur parallelen Anforderung von 3 Festplatten-Kan\"alen.
          \begin{lstlisting}[caption={Parallelisierung von Kan\"alen},label=admin1019,language=rman]
RMAN> CONFIGURE DEVICE TYPE disk PARALLELISM 3;
          \end{lstlisting}
          Hier gilt das gleiche Prinzip, wie bei der manuellen Kanalanforderung: Es werden immer nur so viele Kan\"ale genutzt, wie zur Verf\"ugung stehen und ben\"otigt werden. Jeder einzelne dieser Kan\"ale kann mit Hilfe des \languagerman{CONFIGURE CHANNEL}-Kommandos konfiguriert werden.
          \begin{lstlisting}[caption={Unterschiedliche Kan\"ale vordefinieren},label=admin1020,language=rman]
RMAN> CONFIGURE CHANNEL 0 DEVICE TYPE disk
2>    MAXPIECESIZE = 500M;
RMAN> CONFIGURE CHANNEL 1 DEVICE TYPE sbt
2>    PARMS 'SBT_LIBRARY=oracle.disksbt,ENV=(BACKUP_DIR=/u04)';
RMAN> CONFIGURE CHANNEL 2 DEVICE TYPE disk
2   > FORMAT '/u02/backup/backupset_%u.bkp';
          \end{lstlisting}
          \begin{merke}
            Mit \languagerman{CONFIGURE CHANNEL n} werden Kan\"ale durch Nummern von 0 bis n angesprochen. Somit ist es m\"oglich, getrennte Konfigurationen f\"ur Kan\"ale anzulegen. Die Kommandos \languagerman{CONFIGURE CHANNEL} und \languagerman{CONFIGURE CHANNEL 0} sind identisch.
          \end{merke}
          Wenn alle parallel angeforderten Kan\"ale die gleichen Einstellungen aufweisen sollen, kann das Kommando \languagerman{CONFIGURE CHANNEL} genutzt werden, um Kanal Nummer 0 zu konfigurieren. Wurde nur dieser eine Kanal eingerichtet, besitzen alle ge\"offneten Kan\"ale die gleichen Vorgaben.
          \begin{lstlisting}[caption={Parallele Anforderung von Kan\"alen mit gleicher Konfiguration},label=admin1021,language=rman]
RMAN> CONFIGURE DEVICE TYPE disk
2>    PARALLELISM 3;
RMAN> CONFIGURE CHANNEL DEVICE TYPE disk
2>    MAXPIECESIZE = 500M
3>    FORMAT '/u02/backup/%d_%D-%M-%Y_%u.bkp';
          \end{lstlisting}
          Im vorangegangenen Beispiel werden drei Kan\"ale parallel genutzt. Die Konfiguration f\"ur alle drei Kan\"ale wird von der Konfiguration des Kanals Nummer 0 abgeleitet.
        \subsubsection{Namenskonventionen der automatischen Kanalanforderung}
          RMAN benutzt die Konvention \enquote{ORA\_ger\"at\_n} f\"ur die Benennung automatisch angeforderter Kan\"ale. Der Platzhalter \enquote{ger\"at} steht dabei f\"ur das Ger\"at, auf das der Kanal zeigt, beispielsweise \enquote{disk} oder \enquote{sbt\_tape}. n steht f\"ur die laufende Nummer des Kanals. RMAN benennt den ersten Festplatten-Kanal \enquote{ORA\_DISK\_1}, den zweiten \enquote{ORA\_DISK\_2} usw. F\"ur Streamerger\"ate lautet der Name des ersten automatisch angeforderten Kanals \enquote{ORA\_SBT\_TAPE\_1}.

          Das folgende Beispiel zeigt den Zusammenhang zwischen automatischer Kanalanforderung, Parallelisierung und deren Namenskonventionen.

          Der Nutzer setzt folgende Kommandos ab:
          \begin{lstlisting}[caption={Namenskonventionen f\"ur RMAN-Kan\"ale},label=admin1022,language=rman]
RMAN> CONFIGURE DEVICE TYPE disk PARALLELISM 3;
RMAN> BACKUP database PLUS ARCHIVELOG;
          \end{lstlisting}
          Effektiv tut RMAN folgendes:
          \begin{lstlisting}[caption={Namenskonventionen f\"ur vordefinierte RMAN-Kan\"ale 2},label=admin1023,language=rman]
RMAN> RUN {
2>      ALLOCATE CHANNEL ORA_DISK_1 DEVICE TYPE disk;
3>      ALLOCATE CHANNEL ORA_DISK_2 DEVICE TYPE disk;
4>      ALLOCATE CHANNEL ORA_DISK_3 DEVICE TYPE disk;
5>
6>      BACKUP database PLUS ARCHIVELOG;
7>    }
          \end{lstlisting}
\clearpage
          \begin{merke}
            Beachtet werden muss, das Kan\"ale die mit dem Pr\"afix \textit{ORA\_} beginnen, nur f\"ur die interne Nutzung durch RMAN angefordert werden k\"onnen. Wird ein Kanal mit einem solchen Namen manuell angefordert, quittiert RMAN dies mit der folgenden Fehlermeldung:
\begin{verbatim}
RMAN-00571: ===========================================================
RMAN-00569: =============== ERROR MESSAGE STACK FOLLOWS ===============
RMAN-00571: ===========================================================
RMAN-03002: Fehler bei allocate Befehl auf 07/22/2013 14:06:01
RMAN-06472: Kanal-ID ORA_DISK_1 wird automatisch zugewiesen
\end{verbatim}
          \end{merke}
        \subsubsection{Automatisches Channel-Failover}
          Sind f\"ur eine Operation mehrere Kan\"ale angeforderte, kann RMAN im Falle dessen, dass ein Kanal ausf\"allt, auf einen anderen Kanal wechseln. Somit kann die Operation unter Umst\"anden trotz des Ausfalls eines Kanals fortgesetzt werden. Dieser Mechanismus wird als \enquote{Channel-Failover} bezeichnet.
    \section{Autobackup f\"ur Kontrolldateien und SPFiles konfigurieren}
      \label{controlfileautobackup}
      In vielen Recoverysituationen ist es sehr wichtig, ein Backup der aktuellen Kontrolldatei zu besitzen. Um die Wahrscheinlichkeit zu erh\"ohen, dass ein solches Backup existiert, bietet Oracle die M\"oglichkeit, das \enquote{Controlfile Autobackup} einzurichten. Es wird ein automatisches Backup der Kontrolldatei angefertigt, sobald \"Anderungen am RMAN Repositoy gemacht wurden.

      Mit Hilfe eines Controlfile Autobackups kann RMAN die Datenbank auch dann wiederherstellen, wenn die aktuelle Kontrolldatei, der Recovery Katalog und das SPFile nicht mehr verf\"ugbar sind.  Um das automatische Controlfile Autobackup zu konfigurieren, werden die beiden folgenden Kommandos verwendet:
      \begin{lstlisting}[caption={Controlfile Autobackup konfigurieren},label=admin1024,language=rman]
RMAN> CONFIGURE CONTROLFILE AUTOBACKUP ON;
RMAN> CONFIGURE CONTROLFILE AUTOBACKUP OFF;
      \end{lstlisting}
      \begin{merke}
        Oracle empfiehlt, dass das Controlfile Autobackup immer aktiviert sein sollte.
      \end{merke}

      Der Ablauf eines Controlfile Autobackup gliedert sich wie folgt:
      \bild{Ablauf eines Controlfile Autobackups}{controlfile_autobackup}{0.35}

      Nachdem das Controlfile Autobackup abgeschlossen wurde, wird der komplette Pfad und der Dateiname des Backup Sets in die Alert.log Datei eingetragen.

      Wie in \abbildung{controlfile_autobackup} zu sehen, wird nur in zwei Situationen ein Controlfile Autobackup durchgef\"uhrt:
      \begin{enumerate}
        \item Wenn ein Backup erfolgreich verlaufen ist.
        \item Wenn eine strukturelle \"Anderung an der Datenbank vorgenommen wurde (hinzuf\"ugen oder entfernen von Tablespaces, \"andern von Redo Log Gruppen, u. \"a.).
      \end{enumerate}
      Im ersten Fall wird das Autobackup durch einen RMAN Channel durchgef\"uhrt. Im zweiten Fall f\"uhrt der Serverprozess, der die Struktur\"anderung an der Datenbank vorgenommen hat das Autobackup durch.
      \begin{merke}
        Sobald Datendatei Nummer 1, der System-Tablespace in einem Backup enthalten ist, wird immer die Kontrolldatei mitgesichert, auch wenn das Controlfile Autoback deaktiviert ist.
      \end{merke}

      \begin{literaturinternet}
        \item \cite[Control File and Server Parameter File Autobackups]{cfandspfileautobackups}
      \end{literaturinternet}
    \section{Die Backup Retention Policy}
      \label{backupretentionpolicy}
      Die Backup Retention Policy\footnote{Retention Policy (w\"ortliche \"Ubersetzung): Erhaltungsregel} legt fest, welche Backups erhalten bleiben m\"ussen, um ein Recovery der Datenbank durchf\"uhren zu k\"onnen. Diese Regel kann auf einem Zeitfenster (Recovery Window) basieren oder auf Redundanz.
      \subsection{Zeitfensterbasierte Retention Policy einrichten}
        Als Recovery Window wird ein Zeitraum bezeichnet, der vom aktuellen Datum ausgehend x Tage in die Vergangenheit reicht. Innerhalb dieses Zeitfensters kann die Datenbank mit einem Point-In-Time-Recovery in jeden beliebigen Zustand zur\"uckversetzt werden. Der fr\"uhest m\"ogliche Zeitpunkt, an den eine Datenbank zur\"uckversetzt werden kann, wird als \textit{Point Of Recoverability} bezeichnet.
        \begin{lstlisting}[caption={RECOVERY WINDOW setzen},label=admin1025,language=rman]
RMAN> CONFIGURE RETENTION POLICY TO RECOVERY WINDOW OF 7 DAYS;
        \end{lstlisting}
        Diese Regel stellt sicher, dass f\"ur jede Datendatei ein Backup existiert, das \"alter ist, als der Point of Recoverability, in diesem Falle also \"alter als sieben Tage.

        Alle Backups, die \"alter sind, als das aktuellste Backup, das die genannte Bedingung erf\"ullt gelten als obsolet (nicht mehr ben\"otigt).

        Das folgende Beispiel soll die Nutzung eines Recovery Window verdeutlichen. Folgendes gilt:
        \begin{itemize}
          \item Es wurden Backups am 01.06.20XX und am 13.06.20XX gemacht.
          \item Die Datenbank befindet sich im Archivelog Modus und alle notwendigen Archive sind vorhanden.
        \end{itemize}
        In \abbildung{recovery_window_example_1} ist das aktuelle Datum der 30.06.20XX. Der Point of Recoverability liegt am 23.06.20XX, sieben Tage vom aktuellen Datum zur\"uck. Um ein Recovery gem\"a\ss{}\ der g\"ultigen Retention Policy machen zu k\"onnen, wird das Backup vom 13.06.20X, sowie die Archive Logs von Sequenz 550 bis 800, ben\"otigt. Das Backup vom 01.06.20XX und alle Archive Logs vor Sequenz 550 sind obsolet und k\"onnen gel\"oscht werden.
        \bild{Nutzung eines Recovery Window}{recovery_window_example_1}{1.5}
\clearpage
        \abbildung{recovery_window_example_2} zeigt ein ver\"andertes Szenario:

        \bild{Nutzung eines Recovery Window}{recovery_window_example_2}{1.5}

        Das aktuelle Datum ist der 11.07.20XX und der Point of Recoverability ist der 04.07.20XX. Um die Retention Policy zu erf\"ullen, werden nun das Backup vom 30.06.20XX und alle Archive von Logs von Sequenz 800 an ben\"otigt. Das Backup vom 06.07.20X ist zwar aktueller als das vom 30.06.20XX, es kann aber die Retention Policy nicht erf\"ullen, da es j\"unger ist als der Point of Recoverability.
      \subsection{Eine redundanzbasierte Retention Policy einrichten}
        Eine redundanzbasierte Retention Policy legt fest, wie viele Backups von einer Datendatei vorhanden sein m\"ussen, um die Policy zu erf\"ullen.
        \begin{lstlisting}[caption={REDUNDANCY setzen},label=admin1026,language=rman]
RMAN> CONFIGURE RETENTION POLICY TO REDUNDANCY 2;
        \end{lstlisting}
        Dieses Kommando legt fest, dass von jeder Datendatei mindestens zwei Backups vorhanden sein m\"ussen. Hierbei wird kein Unterschied zwischen Image Copy und Backup Set gemacht.

        Welche der beiden Verwaltungsmethoden vorzuziehen ist, h\"angt davon ab, ob ein Point-In-Time-Recovery der Datenbank in Frage kommt oder nicht. Falls nicht, erweist sich die redundanzbasierte Methode als platzsparender. Im anderen Falle ist die zeitfensterbasierte Variante besser, da mittels des Point Of Recoverability genau bestimmt werden kann, wie weit ein Point-In-Time-Recovery in die Vergangenheit zur\"uckreichen kann.
        \begin{merke}
          Die Standardeinstellung f\"ur die Retention Policy ist REDUNDANCY = 1.
        \end{merke}
\clearpage
			\subsection{Die Retention Policy anzeigen lassen}
        Mit Hilfe des \languagerman{SHOW}-Kommandos kann die aktuelle Retention Policy angezeigt werden.
        \begin{lstlisting}[caption={Anzeigen der Retention Policy},label=admin1027,language=rman]
RMAN> SHOW RETENTION POLICY;
        \end{lstlisting}
      \subsection{Die Retention Policy abschalten}
        Mit dem folgenden Kommando kann die Retention Policy abgeschaltet werden:
        \begin{lstlisting}[caption={Abschalten der Retention Policy},label=admin1028,language=rman]
RMAN> CONFIGURE RETENTION POLICY TO NONE;
        \end{lstlisting}
      \subsection{Backups aus der Retention Policy ausschlie\ss{}en}
        Es kann vorkommen, dass man bestimmte Backups f\"ur besondere Zwecke l\"anger aufbewahren muss, als die Retention Policy dies zul\"asst. Solche Langzeit Backups k\"onnen im RMAN Repository verzeichnet sein, m\"ussen aber von der Retention Policy ausgenommen werden.

        Ein Backup kann von der Retention Policy ausgeschlossen werden, indem:
        \begin{itemize}
          \item Die \languagerman{KEEP}-Option des \languagerman{BACKUP}-Kommandos verwendet wird
            \begin{lstlisting}[caption={Ein neues Backup aus der Retention Policy ausschlie\ss{}en},label=admin1029,language=rman]
RMAN> BACKUP database KEEP UNTIL TIME "31.12.2014" NOLOGS
2>    FORMAT "/u03/backup/ORCL/backup_%s_%p.bkp";
            \end{lstlisting}
          \item Das \languagerman{CHANGE}-Kommando mit der \languagerman{KEEP}-Option verwendet wird.
            \begin{lstlisting}[caption={Ein bestehendes Backup aus der Retention Policy ausschlie\ss{}en},label=admin1030,language=rman]
RMAN> CHANGE BACKUPSET 2 KEEP UNTIL TIME "31.12.2014"
2>    NOLOGS;
            \end{lstlisting}
        \end{itemize}
        Backups die von der Retention Policy ausgeschlossen sind, sind nach wie vor vollst\"andig g\"ultig und nutzbar. Um ein Backup wieder in die Retention Policy einzuschlie\ss{}en, wird die \languagerman{NOKEEP}-Option des \languagerman{CHANGE}-Kommandos verwendet.
        \subsubsection{LOGS und NOLOGS}
          Die Optionen \languagerman{LOGS} und \languagerman{NOLOGS} legen fest, ob die zu einem Backup ben\"otigten Archive Logs (alle Archive Logs die neuer sind als das Backup) mit von der Retention Policy ausgeschlossen werden (\languagerman{LOGS}) oder ob die Archive Logs nicht ber\"ucksichtigt werden sollen (\languagerman{NOLOGS}).

          Es gibt die M\"oglichkeit, ein Backup f\"ur immer aus der Retention Policy auszuschlie\ss{}en. Dies funktioniert mit der Option \languagerman{KEEP FOREVER}. Hierbei sollte jedoch immer die Option \languagerman{NOLOGS} verwendet werden, da sonst alle Archive Logs, die j\"unger als das Backup sind f\"ur immer aufbewahrt werden m\"ussten.
        \subsubsection{Einschr\"ankungen der KEEP-Option}
          F\"ur die Nutzung der \languagerman{KEEP}-Option gibt es die folgenden Einschr\"ankungen.
          \begin{itemize}
            \item Werden Backup Sets mit einem Keep-Attribut versehen, k\"onnen diese nicht in der Fast Recovery Area (sieh \ref{configureflashrecoveryarea}) gespeichert werden. Aus diesem Grund muss hier zwingend ein alternativer Speicherort mit dem \languagerman{FORMAT}-Schl\"usselwort gesetzt werden. Anderenfalls bricht RMAN den Backupjob mit der Fehlermeldung ab:
            \begin{lstlisting}[caption={Keine Dateien mit KEEP-Attribut in der FRA aufbewahren!},label=admin1031,language=terminal]
% ORA-19811: Dateien in DB_RECOVERY_FILE_DEST mit Keep-Attribut
           sind nicht moeglich.
            \end{lstlisting}
            \item Backups, die Archive Logs beinhalten, k\"onnen nicht aus der Retention Policy ausgeschlossen werden. Oracle bricht einen solchen Versuch mit der folgenden Fehlermeldung ab:
            \begin{lstlisting}[caption={Backups mit Archive Logs d\"urfen kein KEEP-Attribut haben!},label=admin1032,language=terminal]
RMAN-06530: &CHANGE& ... &KEEP& not supported for backup &set& which contains
            archive logs.
            \end{lstlisting}
          \end{itemize}
    \section{Die Fast Recovery Area}
      \label{configureflashrecoveryarea}
      \begin{merke}
        Mit Oracle 10g wurde die \enquote{Flash Recovery Area} eingef\"uhrt. Da es immer wieder zu Verwechslungen mit der Technologie \enquote{Flashback Database} kam, wurde die Flash Recovery Area, in Oracle 11g in \enquote{Fast Recovery Area} umbenannt.
      \end{merke}
      Wie bereits erw\"ahnt, ist die Fast Recovery Area ein Speicherbereich auf einem Daten\-tr\"ager, der verschiedene, f\"ur das Recovery der Datenbank ben\"otigte Dateien zwischenspeichert (Cache-Funktion).

      Die Fast Recovery Area erleichtert die Administration der Datenbank dahingehend, dass Backupdateien automatisch benannt werden, dass sie automatisch solange vorgehalten werden, wie sie f\"ur ein Restore and Recovery ben\"otigt werden und dass sie automatisch gel\"oscht werden, wenn sie nicht mehr ben\"otigt werden.
      \subsection{Planen der Fast Recovery Area}
        Die in der Fast Recovery Area gespeicherten Dateien werden in zwei Kategorien eingeteilt.
        \subsubsection{Permanente Dateien}
          Die einzigen permanenten Dateien sind gespiegelte Kopien der Kontrolldatei und der Redo Log Dateien. Diese Dateien k\"onnen nicht gel\"oscht werden, ohne einen Absturz der Instanz zu bewirken.
        \subsubsection{Fl\"uchtige Dateien}
          Alle anderen Dateien gelten als fl\"uchtig und werden von Oracle automatisch gel\"oscht, wenn Sie aufgrund der Backup Retention Policy als veraltet gelten oder wenn sie durch ein Backup erfasst wurden.

          Dies schlie\ss{}t folgende Dateien ein:
          \begin{itemize}
            \item Archivierte Redo Logs
            \item Image Copies von Datendateien
            \item Image Copies von Kontrolldateien
            \item Kontrolldatei-Autobackups
            \item Backup Pieces
          \end{itemize}
        \subsubsection{Den geeigneten Speicherplatz f\"ur die Recovery Area finden}
          Bevor eine Fast Recovery Area erstellt werden kann, muss der geeignete
          Speicherort f\"ur sie gefunden werden (Verzeichnis oder ASM Disk
          Group).
\clearpage
          \begin{merke}
            Eine Fast Recovery Area kann nicht auf einem Raw-Datentr\"ager gespeichert werden.
          \end{merke}
          Des Weiteren muss f\"ur die Fast Recovery Area eine Disk Quota festgelegt werden, die bestimmt, wie gro\ss{} die Area werden darf.

          Die Fast Recovery Area sollte auf einem anderen Datentr\"ager liegen, als die Datenbank, da sonst die Gefahr besteht, das Datenbank und Fast Recovery Area, verloren gehen.
        \subsubsection{Die Gr\"o\ss{}e der Fast Recovery Area planen}
          Die Fast Recovery Area sollte gro\ss{} genug sein, um folgende Dateien aufzunehmen:
          \begin{itemize}
            \item Kopien aller Datendateien
            \item Inkrementelle Backups
            \item Online Redo Logs
            \item Archivierte Redo Logs, die in noch keinem Backup gesichert wurden
            \item Kontrolldateien
            \item Kontrolldatei-Autobackups
          \end{itemize}
          Sollte so viel Speicherplatz nicht zur Verf\"ugung stehen, ist es das Beste, die Fast Recovery Area gro\ss{} genug zu machen, um Backups der wichtigsten Tablespaces zu speichern und alle archivierten Redo Logs, die in noch keinem Backup gesichert wurden. Im Minimum muss die Fast Recovery Area jedoch so gro\ss{} sein, dass sie die noch nicht durch ein Backup gesicherten archivierten Redo Logs aufnehmen kann.

          \begin{merke}
            Wie gro\ss{}\ man seine Fast Recovery Area planen sollte h\"angt von verschiedenen Faktoren ab. Diese sind unter anderem:
            \begin{itemize}
              \item Finden viele oder nur sehr wenige \"Anderungen an der Datenbank statt?
              \item Werden Backups nur auf Festplatte oder auch auf SBT-Ger\"ate gespeichert?
              \item Wie viele Backups m\"ussen st\"andig verf\"ugbar sein?
              \item Soll der Flashback Database Mechanismus genutzt werden oder nicht?
            \end{itemize}
          \end{merke}
      \subsection{Konfigurieren der Fast Recovery Area}
        Um die Fast Recovery Area zu aktivieren, m\"ussen die beiden Initialisierungsparameter \parameter{DB\_RECOVERY\_FILE\_DEST\_SIZE} und \parameter{DB\_RECOVERY\_FILE\_DEST} gesetzt werden.
        \begin{merke}
          Der Parameter \parameter{DB\_RECOVERY\_FILE\_DEST\_SIZE} muss zuerst gesetzt werden. Er legt den maximal zur Verf\"ugung stehen Speicherplatz f\"ur die Fast Recovery Area fest. Der Wert f\"ur diesen Parameter sollte immer so geplant werden, das ein Verwaltungsoverhead von ca. 10\% auf dem betreffenden Datentr\"ager \"ubrig bleibt.
        \end{merke}
        \begin{lstlisting}[caption={DB\_RECOVERY\_FILE\_DEST\_SIZE setzen},label=admin1033,language=oracle_sql]
SQL> ALTER SYSTEM SET db_recovery_file_dest_size=4G;
        \end{lstlisting}
        \begin{merke}
          \parameter{DB\_RECOVERY\_FILE\_DEST} legt ein Verzeichnis oder eine ASM Disk Group als Speicherort f\"ur die Fast Recovery Area fest.
        \end{merke}
        \begin{lstlisting}[caption={DB\_RECOVERY\_FILE\_DEST\_SIZE setzen},label=admin1034,language=oracle_sql]
SQL> ALTER SYSTEM SET db_recovery_file_dest='/u05/fast_recovery_area';
        \end{lstlisting}
        \subsubsection{Die Views v\$recovery\_file\_dest und v\$recovery\_area\_usage}
          Diese beiden Views enthalten n\"utzliche Informationen dar\"uber, ob gen\"ugend Speicherplatz f\"ur die Fast Recovery Area zur Verf\"ugung gestellt wird.
          \begin{itemize}
            \item \identifier{v\$recovery\_file\_dest}: Enth\"alt Informationen
            \"uber den Speicherort, die Disk Quota, den genutzten Speicher und
            den noch zur Verf\"ugung stehenden Speicher in der Fast Recovery
            Area.
            \begin{lstlisting}[caption={\identifier{v\$recovery\_file\_dest}},label=admin1035,language=oracle_sql]
SQL> col name format a23
SQL> SELECT name, space_limit / POWER(1024, 2) AS Space_Limit,
  2         ROUND(space_used / POWER(1024, 2),2) AS space_used, 
  3         number_of_files
  4  FROM   v$recovery_file_dest;

NAME                    SPACE_LIMIT SPACE_USED NUMBER_OF_FILES
----------------------- ----------- ---------- ---------------
/u05/fast_recovery_area        4096     640,09               4

            \end{lstlisting}

           \item \identifier{v\$recovery\_area\_usage}: Zeigt Informationen dar\"uber an, wie viel Speicherplatz jede Dateiart in der Fast Recovery Area belegt und wie viel Speicherplatz durch das L\"oschen als veraltet markierter Dateien gewonnen werden kann.
            \begin{lstlisting}[caption={\identifier{v\$recovery\_area\_usage}},label=admin1036,language=oracle_sql]
SQL> SELECT file_type,
  2         TO_CHAR(percent_space_used, '90.00') AS pct_space_used,
  3         TO_CHAR(percent_space_reclaimable, '90.00') AS pct_space_recl,
  4         number_of_files
  5  FROM   v$recovery_area_usage;
  
FILE_TYPE            PCT_SP PCT_SP NUMBER_OF_FILES
-------------------- ------ ------ ---------------
CONTROL &FILE&           0.00   0.00               0
REDO &LOG&               0.00   0.00               0
ARCHIVED &LOG&           0.00   0.00               0
&BACKUP& PIECE           15.63   0.23               4
IMAGE COPY             0.00   0.00               0
&FLASHBACK& &LOG&            0.00   0.00               0
&FOREIGN& ARCHIVED &LOG&     0.00   0.00               0
            \end{lstlisting}
          \end{itemize}

          \begin{literaturinternet}
            \item \cite{sthref3529}
            \item \cite{sthref3528}
          \end{literaturinternet}

        \subsubsection{Die Fast Recovery Area abschalten}
          Um die Fast Recovery Area abzuschalten, muss lediglich der Initialisierungsparameter \parameter{db\_recovery\_file\_dest} auf einen Null-String ('') zur\"uckgesetzt werden.
          \begin{lstlisting}[caption={Die Fast Recovery Area abschalten},label=admin1037,language=oracle_sql]
SQL> ALTER SYSTEM
  2  SET db_recovery_file_dest='';
          \end{lstlisting}
          Sowohl die in der Fast Recovery Area gespeicherten Dateien als auch die Eintr\"age im RMAN Repository bleiben dabei erhalten.
        \subsubsection{Verhalten der Fast Recovery Area bei einem Instanz-Crash}
          Grunds\"atzlich verwaltet sich die Fast Recovery Area selbst. Im Falle eines Instanz-Crashes kann es jedoch vorkommen, das aktuell ge\"offnete Dateien unvollst\"andig in der Fast Recovery Area zur\"uckbleiben. Oracle zeigt dies mit der folgenden Fehlermeldung im Alert Log:
          \begin{lstlisting}[caption={Besch\"adigte Dateien in der Fast Recovery Area},label=admin1038,language=rman]
ORA-19816: WARNING: Files may exist in location that are not known to database.
          \end{lstlisting}
          \textit{location} wird ersetzt durch den aktuellen Speicherort der Fast Recovery Area.

          Liegt ein solcher Fall vor, gibt es zwei M\"oglichkeiten, um die Situation zu bereinigen:
          \begin{itemize}
            \item Die betreffende Datei kann, falls der Dateiheader unbesch\"adigt ist, einfach erneut durch RMAN erfasst und im Repository gespeichert werden.
              \begin{lstlisting}[caption={Besch\"adigte Dateien in der Fast Recovery Area neu katalogisieren},label=admin1039,language=rman]
RMAN> CATALOG RECOVERY AREA;
              \end{lstlisting}
            \item Die Datei muss manuell gel\"oscht werden, da der Dateiheader zerst\"ort/ung\"ultig ist.
          \end{itemize}
      \subsection{Speicherplatzverwaltung in der Fast Recovery Area}
        Oracle l\"oscht keine Dateien in der Fast Recovery Area, bevor nicht weiterer Speicherplatz ben\"otigt wird. Der daraus resultierende Effekt ist, dass Dateien die bereits auf ein SBT-Ger\"at gesichert wurden, meist noch einige Zeit danach in der Fast Recovery Area vorhanden sind. Somit fungiert die Area als Cache.
        \subsubsection{Welche Dateien werden wann gel\"oscht}
          Es sind vier einfache Regeln, von denen drei hier genannt werden, wann welche Dateien aus der Fast Recovery Area entfernt werden:
          \begin{itemize}
            \item Permanente Dateien werden nie gel\"oscht.
            \item Dateien, die durch die Retention Policy als veraltet markiert wurden werden ge\-l\"oscht, wenn neuer Speicherplatz ben\"otigt wird.
            \item Fl\"uchtige Dateien, die von einem Backup erfasst wurden, werden gel\"oscht, wenn weiterer Speicherplatz ben\"otigt wird.
          \end{itemize}
          \begin{merke}
            Welche Dateien in der Fast Recovery Area zuerst gel\"oscht werden, kann nicht genau bestimmt werden.
          \end{merke}
        \subsubsection{Wenn nicht genug Speicher zur Verf\"ugung steht...}
          Ist die Retention Policy des RMAN so eingestellt, das mehr Speicherplatz f\"ur Backups ben\"otigt wird, als in der Fast Recovery Area vorhanden ist oder wurde die Retention Policy abgeschaltet, kann es passieren, dass die Fast Recovery Area sich zu 100\% f\"ullt.

          Sind nur noch weniger als 15\% freier Speicher in der Fast Recovery Area vorhanden, wird eine Warnung durch die Datenbank ausgegeben. Sind es nur noch 3\%, wird eine kritische Warnung angezeigt. Die Datenbank wird solange neuen Speicher anfordern, bis keiner mehr vorhanden ist.

          Eine komplett gef\"ullte Fast Recovery Area l\"ost folgende Fehlermeldung aus:
          \begin{lstlisting}[caption={100\% gef\"ullte Fast Recovery Area},label=admin1040,language=terminal]
ORA-19809: limit exceeded for recovery files
ORA-19804: cannot reclaim nnnnn bytes disk space from mmmmm limit
          \end{lstlisting}
          \textit{nnnnn} stellt dabei die ben\"otigte Speichermenge und \textit{mmmmm} die Disk Quota der Fast Recovery Area dar.

          Oft reagiert die Datenbank auf so einen Fehler, in dem sie stehen bleibt. Wenn zum Beispiel ein als zwingend markierter Speicherort der Archived Logs in der Fast Recovery Area liegt, kann die Datenbank nicht mit ihrem Betrieb fortfahren, bis das Problem behoben ist.
        \subsubsection{Bereinigen einer vollen Fast Recovery Area}
          Es gibt verschieden M\"oglichkeiten, eine zu 100 \% gef\"ullte FRA zu bereinigen:
          \begin{itemize}
            \item Mehr Speicherplatz verf\"ugbar machen, in dem der Wert des Initialisierungsparameters \parameter{db\_recovery\_file\_dest\_size} erh\"oht wird.
            \item Backups von den Dateien in der Fast Recovery Area auf Band anfertigen(siehe \ref{FRABackups}) und anschlie\ss{}end, mit Hilfe des RMAN Kommandos \languagerman{DELETE} die Fast Recovery Area entleeren.
          \end{itemize}
          \begin{merke}
            Manuelle Eingriffe in die Fast Recovery Area mit Betriebssystemmitteln sollten vermieden werden, da die Datenbank derartige \"Anderungen nicht registriert und der Speicherplatz nach wie vor als belegt gilt. Solche Fehler k\"onnen anschlie\ss{}end nur mittels des RMAN Kommandos \languagerman{CROSSCHECK} behoben werden.
          \end{merke}
        \subsubsection{Die Fast Recovery Area an einen neuen Platz verschieben}
          Muss die Fast Recovery Area an einen neuen Platz verschoben werden, gen\"ugt es, das SQL*Plus-Tool zu \"offnen und den Parameter \parameter{DB\_RECOVERY\_FILE\_DEST} anzupassen:
          \begin{lstlisting}[caption={Den Speicherort der Fast Recovery Area \"andern},label=changeFRAlocation,language=oracle_sql]
SQL> ALTER SYSTEM
  2  SET db_recovery_file_dest='/u04/fast_recovery_area'
  3  SCOPE=both;
          \end{lstlisting}
          \begin{merke}
            Anschlie\ss{}end an die \"Anderung dieses Parameters, werden alle Dateien am neuen Speicherort erstellt. Alle bereits bestehenden Dateien verbleiben am Alten und verbrauchen weiterhin Speicherplatz in der Fast Recovery Area, bis sie durch die automatische Verwaltung der Fast Recovery Area gel\"oscht werden.
          \end{merke}
          Die n\"achsten Schritte sind nur dann notwendig, wenn auch die bestehenden Dateien an den neuen Speicherort \"uberf\"uhrt werden m\"ussen. Mit Hilfe des \languagerman{BACKUP}-Kommandos k\"onnen einzelne Dateiarten in die neue Fast Recovery Area gesichert werden.
          \begin{lstlisting}[caption={Verschieben von Dateien in eine neue Fast Recovery Area},label=admin1041,language=rman]
RMAN> BACKUP AS COPY ARCHIVELOG ALL DELETE INPUT;

RMAN> BACKUP DEVICE TYPE DISK BACKUPSET ALL DELETE INPUT;

RMAN> BACKUP AS COPY DATAFILECOPY ALL DELETE INPUT;
          \end{lstlisting}
          Hiermit werden alle Archive Logs, Backup Sets und Image Copies in die neue Fast Recovery Area gesichert und am alten Speicherort gel\"oscht. Bei allen anderen Dateitypen sorgt die automatische Verwaltung f\"ur das L\"oschen, wenn die Dateien nicht mehr ben\"otigt werden.
        \subsubsection{Zusammenspiel zwischen Retention Policy und der Fast Recovery Area}
          Wird ein Backup als obsolet gekennzeichnet, ist dies immer auf eine im RMAN konfigurierte Retention Policy zur\"uckzuf\"uhren. Wird die Fast Recovery Area konfiguriert, benutzt die Datenbank einen internen Algorithmus, um Dateien bestimmen zu k\"onnen, die von der Festplatte gel\"oscht werden k\"onnen. Es gibt zwei F\"alle, wann Dateien durch die Verwaltung der Fast Recovery Area gel\"oscht werden:
          \begin{itemize}
            \item Dateien tragen den Status obsolete, da sie gegen die konfigurierte Retention Policy versto\ss{}en.
            \item Dateien sind noch nicht obsolete, wurden aber anderweitig, z. B. auf Tape, gesichert.
          \end{itemize}
\clearpage

    \input{uebungen/dbadmin_15_konfigurieren_des_recovery_manager_uebung}
    \input{loesungen/dbadmin_15_konfigurieren_des_recovery_manager_loesung}
    \chapter{Verwalten des Recovery Katalogs}
    \setcounter{page}{1}\kapitelnummer{chapter}
    \minitoc
\newpage
    \section{Einrichten eines Recovery Katalogs}
    \label{createrecoverycatalog}
      Das Einrichten eines Recovery Katalogs umfasst vier Schritte:
      \begin{itemize}
        \item Erstellen/Konfigurieren der Datenbank, die den Recovery Katalog aufnimmt
        \item Schaffen einer Netzwerkverbindung (TNS) zwischen der Zieldatenbank und der Recovery Katalog Datenbank.
        \item Den Eigent\"umer des Recovery Katalogs erstellen
        \item Den Recovery Katalog erstellen
      \end{itemize}
      Bevor der Katalog genutzt werden kann, muss noch die Zieldatenbank registriert werden.
      \subsection{Erstellen der Katalogdatenbank}
        F\"ur die Planung des Recovery Katalogs ist es wichtig zu wissen, wie viel Speicherplatz zur Verf\"ugung gestellt werden muss. Dies ist davon abh\"angig, wie viele Datenbanken durch den Katalog verwaltet werden oder wie gro\ss{} die Anzahl der Archive Logs und der Backupdateien einer jeden Datenbank ist. Als drittes werden auch noch RMAN stored Scripts im Recovery Katalog gespeichert, f\"ur die ebenfalls geringe Platzreserven ben\"otigt werden.

        \begin{merke}
          Laut Oracle sind ca. $15 MB * Anzahl\ reg\ DB = Wachstum/Jahr$ ein guter Ansatz.
        \end{merke}

        Der typische Speicherplatzverbrauch f\"ur einen Katalog teilt sich wie folgt auf:
        \begin{center}
          \tablecaption{Speicherplatzbedarf einer Katalogdatenbank}
          \tablefirsthead{%
          \hline
          \multicolumn{1}{|c}{\textbf{Art des Speicherplatzes}} &
          \multicolumn{1}{|c|}{\textbf{Verbrauch/Jahr}} \\
          \hline
          }
          \tablehead{%
          \hline
          \multicolumn{1}{|c}{\textbf{Art des Speicherplatzes}} &
          \multicolumn{1}{|c|}{\textbf{Verbrauch/Jahr}} \\
          \hline
          }
          \tabletail{%
            \hline
          }
          \begin{supertabular}[h]{|l|c|c|c|p{7cm}|}
          System-Tablespace & 90 MB \\
          \hline
          Temp-Tablespace & 5 MB \\
          \hline
          Undo-Tablespace & 5 MB \\
          \hline
          Recovery Katalog Tablespace & 15 MB pro registrierter Datenbank \\
          \hline
          Online Redo Logs & 10 MB pro Member (3 Gruppen mit je 2 Membern) \\
          \end{supertabular}
        \end{center}
\clearpage
      \subsection{Die TNS-Netzwerkverbindung erstellen}
        Um eine Netzwerkverbindung mittles TNS zwischen der Zieldatenbank und dem Recovery Katalog zu erm\"oglichen, sollte das Local Naming oder das Directory Naming genutzt werden. Im Folgenden wird eine Beispielkonfiguration f\"ur das Local Naming gezeigt.

        Es werden folgende Annahmen getroffen:
        \begin{itemize}
          \item Der Rechner auf dem sich der Recovery Katalog befindet benutzt TCP/IP.
          \item Der Name des Rechners, der den Recovery Katalog enth\"alt ist: FEA11-119CAT.
          \item Der Listener dieses Rechners l\"auft auf Port 1521.
          \item Der Servicename der Katalogdatenbank ist: CATDB.
        \end{itemize}
        Die Datei \oscommand{tnsnames.ora} der Zieldatenbank sollte folgenden Abschnitt beinhalten:
        \begin{lstlisting}[caption={Der Net Service Name der CATDB},label=admin1200,language=configfile]
CATDB =
  (DESCRIPTION=
    (ADDRESS_LIST=
     (ADDRESS= (PROTOCOL=tcp)(HOST=FEA11-119CAT)(PORT=1521))
    )
  (CONNECT_DATA=
    (SERVICE_NAME=CATDB)))
        \end{lstlisting}
      \subsection{Den Katalogeigent\"umer erstellen}
        Ist die Katalogdatenbank erst einmal erstellt und die TNS-Netzwerkverbindung geschaffen, kann der dritte Schritt, das Erstellen des Katalogeigent\"umers vorgenommen werden.
        \begin{enumerate}
          \item Als Nutzer sys mit der Datenbank verbinden.
            \begin{lstlisting}[caption={Mit der Katalogdatenbank verbinden},label=admin1201,language=terminal]
[oracle@FEA11-119SRV ~]$ sqlplus sys/oracle@CATDB as sysdba
            \end{lstlisting}
          \item Erstellen des Katalogtablespaces
            \begin{lstlisting}[caption={Den Katalogtablespace erstellen},label=admin1202,language=oracle_sql]
SQL> CREATE TABLESPACE catts
  2  DATAFILE '/u02/oradata/catdb/catts01.dbf' SIZE 15 M
  3  AUTOEXTEND ON MAXSIZE 30 M;
            \end{lstlisting}
\clearpage
          \item Erstellen des Katalogeigent\"umers.
            \begin{lstlisting}[caption={Den Katalogeigent\"umer erstellen},label=admin1203,language=oracle_sql]
SQL> CREATE USER catowner
  2  IDENTIFIED BY catpass
  3  DEFAULT TABLESPACE catts
  4  QUOTA unlimited ON catts;
            \end{lstlisting}
          \item Dem Katalogeigent\"umer die Rolle \privileg{recovery\_catalog\_owner} geben.
            \begin{lstlisting}[caption={Die Rolle \privileg{recovery\_catalog\_owner} \"ubergeben},label=admin1204,language=oracle_sql]
SQL> GRANT create session, recovery_catalog_owner TO catowner;
            \end{lstlisting}
        \end{enumerate}
        \begin{merke}
          Damit ein Nutzer Eigent\"umer des Recovery Katalogschemas sein kann, muss er zwingend die Rolle \privileg{recovery\_catalog\_owner} besitzen.
        \end{merke}

      \subsection{Den Recovery Katalog erstellen}
        Das Erstellen des Kataloges geschieht im RMAN.
        \begin{enumerate}
          \item RMAN starten und mit der Katalogdatenbank verbinden
            \begin{lstlisting}[caption={Mit der Katalogdatenbank verbinden},label=admin1205,language=rman,language=terminal]
[oracle@FEA11-119SRV ~]$ rman catalog catowner/catpass@CATDB
            \end{lstlisting}
          \item Den Katalog erstellen
            \begin{lstlisting}[caption={Katalog erstellen},label=admin1206,language=rman]
RMAN> CREATE CATALOG;
            \end{lstlisting}
        \end{enumerate}
        Das Kommando \languagerman{CREATE CATALOG} erstellt den Recovery Katalog im Default Tablespace des Katalogeigent\"umers. Wahlweise kann dieses Kommando auch um die \languageorasql{TABLESPACE}-Klausel erweitert werden.
      \subsection{Registrieren einer Datenbank}
        Der letzte Schritt, bevor eine Datenbank mit dem Recovery Katalog
        genutzt werden kann, ist die Datenbank zu registrieren.
\clearpage
        \begin{enumerate}
          \item Bei  Zieldatenbank und  Recovery Katalog anmelden
            \begin{lstlisting}[caption={Mit Ziel- und Katalogdatenbank verbinden},label=admin1207,language=terminal]
[oracle@FEA11-119SRV ~]$ rman target / catalog catowner/catpass@CATDB
            \end{lstlisting}
          \item Die Zieldatenbank in den Mount-Status versetzen
          \item Registrieren der Zieldatenbank
            \begin{lstlisting}[caption={Zieldatenbank registrieren},label=admin1208,language=rman]
RMAN> REGISTER DATABASE;
            \end{lstlisting}
        \end{enumerate}
        W\"ahrend dieses Vorgangs erstellt RMAN Zeilen in den Katalogtabellen, die alle notwendigen Informationen aus der Kontrolldatei der Zieldatenbank enthalten. So wird der Recovery Katalog mit der Zieldatenbank synchronisiert. Um zu \"uberpr\"ufen, ob die Synchronisation erfolgreich war, kann das Kommando \languagerman{REPORT SCHEMA} genutzt werden.
            \begin{lstlisting}[caption={Schemainformationen anzeigen},label=admin1209,language=rman]
RMAN> REPORT SCHEMA;

File Size(MB)   Tablespace       RB segs Datafile Name
---- ---------- ---------------- ------- -------------------
1        307200 SYSTEM             NO    /u02/oradata/ORCL/system01.dbf
2        303500 SYSAUX             NO    /u02/oradata/ORCL/sysaux01.dbf
3         20480 UNDOTBS            YES   /u02/oradata/ORCL/undotbs01.dbf
4         10240 USERS              NO    /u02/oradata/ORCL/users01.dbf
5         10240 EXAMPLE            NO    /u02/oradata/ORCL/example01.dbf
            \end{lstlisting}
   \section{Datenbanken verwalten}
      \subsection{Eine weitere Datenbank registrieren}
        Es k\"onnen mehrere Zieldatenbank in einem Recovery Katalog verwaltet werden, wenn sie unterschiedliche Datenbank-IDs aufweisen. Beim Vorgang des Duplizierens einer neuen Datenbank, mit dem RMAN-Kommando \languagerman{DUPLICATE} oder bei der Erstellung mit Hilfe des SQL-Statements \languageorasql{CREATE DATABASE}, wird automatisch eine neue DBID generiert. Lediglich bei Datenbankkopien, die auf anderem Wege erstellt wurden, kann es zu Problemen kommen. Hierzu muss zuerst mit dem RMAN-Kommando \languagerman{DBNEWID} eine neue DatenbankID erstellt werden.
        \begin{merke}
          Eine Datenbank kann auch in mehreren Recovery Katalogen registriert werden.
        \end{merke}
\clearpage
      \subsection{Aufheben einer Datenbankregistrierung}
        Zum Aufheben einer Datenbankregistrierung im Recovery Katalog gibt es das Kommando \languagerman{UNREGISTER DATABASE}. Dabei werden alle Eintr\"age zu einer Datenbank aus dem Katalog gel\"oscht.
        \begin{merke}
          Informationen die zwar im Recovery Katalog gespeichert waren, aber nicht in der Kontroldatei (\parameter{control\_file\_record\_keep\_time}) gespeichert sind, gehen bei diesem Vorgang verloren.
        \end{merke}
        Die Registrierung einer Datenbank wird wie folgt aufgehoben:

        \begin{enumerate}
          \item RMAN starten und eine Verbindung zur betreffenden Zieldatenbank und dem Recovery Katalog herstellen.
            \begin{lstlisting}[caption={Starten des RMAN},label=admin1210,language=rman]
[oracle@FEA11-119SRV ~]$ rman target / catalog catowner/catpass@CATDB
            \end{lstlisting}
            Es ist nicht zwingend notwendig, sich bei der Zieldatenbank anzumelden. Sind jedoch mehrere Datenbanken im Recovery Katalog registriert, muss die Zieldatenbank durch ihre DBID identifiziert werden.
          \item Auflisten aller Backup Sets und Image Copies der Zieldatenbank. Dies sollte aus Sicherheitsgr\"unden erfolgen, damit Backups wieder katalogisiert werden k\"onnen.
            \begin{lstlisting}[caption={Backup Sets und Image Copies auflisten},label=admin1211,language=rman]
RMAN> LIST BACKUP SUMMARY;
RMAN> LIST COPY;
            \end{lstlisting}
          \item Soll die Datenbank dauerhaft aus dem Recovery Katalog gel\"oscht werden, m\"ussen auch alle Backups der Zieldatenbank gel\"oscht werden.
            \begin{lstlisting}[caption={L\"oschen aller Backups},label=admin1212,language=rman]
RMAN> DELETE BACKUP DEVICE TYPE sbt;
RMAN> DELETE BACKUP DEVICE TYPE DISK;
RMAN> DELETE COPY;
            \end{lstlisting}
            \begin{merke}
              Dieser Schritt darf nur dann ausgef\"uhrt werden, wenn auch die Zieldatenbank selbst gel\"oscht werden soll!!!
            \end{merke}
\clearpage
          \item Aufheben der Datenbankregistrierung
            \begin{lstlisting}[caption={Aufheben der Datenbankregistrierung},label=admin1213,language=rman]
RMAN> UNREGISTER DATABASE;
            \end{lstlisting}
        \end{enumerate}
        Wird eine Datenbankregistrierung aufgehoben, ohne dass eine Verbindung zur Zieldatenbank m\"oglich ist, muss Schritt 4 wie folgt ver\"andert werden.
        \begin{lstlisting}[caption={Aufheben der Datenbankregistrierung ohne Verbindung zur Zieldatenbank},label=admin1214,language=rman]
RMAN> RUN {
2>      SET DBID=93457265485;
3>      UNREGISTER DATABASE;
4>    }
        \end{lstlisting}
    \section{Synchronisation des Recovery Katalogs}
      Wenn RMAN eine Synchronisation des Recovery Katalogs durchf\"uhrt, vergleicht er den Inhalt der Kontrolldatei der Zieldatenbank, mit dem Inhalt des Recovery Katalogs und gleicht beide einander an. RMAN f\"uhrt dabei folgende Schritte durch:
      \begin{enumerate}
        \item Erstellen eines Snapshot Controlfiles
        \item Snapshot Controlfile und Recovery Katalog vergleichen
        \item Angleichen von Snapshot Controlfile und Recovery Katalog
      \end{enumerate}
      RMAN f\"uhrt automatisch die Synchronisation des Recovery Katalogs durch. Dies geschieht in verschiedenen Situationen, wie z. B. bei Ausf\"uhrung des \languagerman{BACKUP}-Kommandos. Die Synchronisation kann jedoch auch im Bedarfsfall manuell durchgef\"uhrt werden. Dies geschieht mit dem \languagerman{RESYNC CATALOG}-Kommando.
      \subsection{Vollst\"andige und teilweise Resynchronisation}
        Die Synchronisation des Recovery Katalogs kann vollst\"andig oder nur teilweise erfolgen. In einer teilweisen Resynchronisation liest RMAN die Kontrolldatei der Zieldatenbank und wertet diese nach neuen Backups, neuen Archive Logs und der Redo Log Historie aus. Es werden keine Schemadaten der Zieldatenbank synchronisiert. Diese werden nur bei einer vollst\"andigen Synchronisation abgeglichen.
      \subsection{Wann sollte manuell synchronisiert werden?}
        Da RMAN die Synchronisation des Recovery Katalogs automatisch durchf\"uhrt, ist ein manuelles Synchronisieren meist nicht notwendig. Es gibt jedoch einige Ausnahmesituationen, die im Folgenden beschrieben werden.
        \begin{merke}
          Grundsatz: Manuelle Synchronisation ist immer dann notwendig, wenn \"uber einen l\"angeren Zeitraum keine Verbindung zwischen dem Katalog und der Zieldatenbank bestanden hat.
        \end{merke}
        \subsubsection{Resynchronisation wenn der Recovery Katalog nicht verf\"ugbar war}
          Es kann vorkommen, das Backup oder Recovery Arbeiten notwendig sind, ohne dass der Recovery Katalog verf\"ugbar ist. In so einem Fall kann keine automatische Synchronisation erfolgen.
        \subsubsection{Resynchronisation bei unregelm\"a\ss{}igen Backups}
          Folgendes Beispiel:
          \begin{itemize}
            \item Die Zieldatenbank l\"auft im Archive Logs Modus.
            \item Die Datenbank wird in unregelm\"a\ss{}igen Abst\"anden gesichert.
            \item Zwischen den einzelnen Backups wird eine hohe Anzahl Redo Log Switches erzeugt.
          \end{itemize}
          Bezugnehmend auf das obige Beispiel, kann es sinnvoll sein, den Recovery Katalog manuell zu synchronisieren, da er nicht automatisch bei jedem Redo Log Switch aktualisiert wird. Informationen \"uber Log Switches werden nur im Controlfile der Zieldatenbank gesichert. Wie aktuell der Recovery Katalog gehalten wird, h\"angt dann davon ab, wie oft er manuell synchronisiert wird.
        \subsubsection{Resynchronisation nach einer \"Anderung der Datenbankstruktur}
          Wie schon im vorhergehenden Fall, wird der Recovery Katalog auch bei
          einer Struk\-tur\-\"an\-der\-ung der Datenbank nicht automatisch
          resynchronisiert. Nach einer Struktur\"anderung an der Datenbank
          sollte wie folgt vorgegangen werden:
\clearpage
          \begin{enumerate}
            \item RMAN starten und eine Verbindung zur betreffenden Zieldatenbank und dem Recovery Katalog herstellen.
              \begin{lstlisting}[caption={Starten des RMAN},label=admin1215,language=rman]
[oracle@FEA11-119SRV ~]$ rman target / catalog catowner/catpass@CATDB
              \end{lstlisting}
            \item Resynchronisieren des Recovery Katalogs
              \begin{lstlisting}[caption={Resynchronisieren des Recovery Katalogs},label=admin1216,language=rman]
RMAN> RESYNC CATALOG;
              \end{lstlisting}
          \end{enumerate}
      \subsection{CONTROL\_FILE\_RECORD\_KEEP\_TIME}
        Der Initialisierungsparameter
        \parameter{control\_file\_record\_keep\_time} legt fest, wie lange die
        Eintr\"age des RMAN Repositories in der Kontrolldatei der Zieldatenbank
        erhalten bleiben. \"Uberschreitet ein Eintrag diesen Wert, so wird er
        aus der Kontrolldatei gel\"oscht. Deshalb sollte darauf geachtet werden,
        dass die Synchronisationsvorg\"ange innerhalb der angegebenen Zeitspanne
        bleiben. Das hei\ss{}t, dass der Wert f\"ur den
        Initialisierungsparameter \parameter{control\_file\_record\_keep\_time}
        so gew\"ahlt werden sollte, dass das RMAN Repository ohne Verlust in den
        Recovery Katalog synchronisiert werden kann.

    \chapter{Backups mit dem RMAN}
    \setcounter{page}{1}\kapitelnummer{chapter}
    \minitoc
\newpage
    \section{Backups anfertigen}
      Backups werden im RMAN mit dem Kommando \languagerman{BACKUP}
      durchgef\"uhrt. Dabei sind  unterschiedliche Arten von Backups, wie z. B.
      Hot- oder Coldbackups, Full oder Incremental Backups m\"oglich.
      \subsection{Hot- oder Coldbackup}
        Ob ein Backup ein Hot- oder ein Coldbackup wird, h\"angt vom Status der Datenbank ab. Ist die Datenbank ge\"offnet, wird das Backups als Hotbackup (inkonsistentes Backup) bezeichnet. Ist die Datenbank im Mount-Status, ist es ein Coldbackup (konsistentes Backup). Die beiden folgenden Beispiele zeigen, wie Hot- bzw. Coldbackups durchgef\"uhrt werden k\"onnen.
        \begin{lstlisting}[caption={Ein Coldbackup durchf\"uhren},label=admin1300,language=rman,alsolanguage=sqlplus]
RMAN> shutdown immediate

RMAN> startup mount

RMAN> BACKUP database;
        \end{lstlisting}
        \begin{lstlisting}[caption={Ein Hotbackup durchf\"uhren},label=admin1301,language=rman,emph={[9]ALTER,OPEN,DATABASE},emphstyle={[9]\color{magenta}\bfseries}]
RMAN> SQL 'ALTER DATABASE OPEN';

RMAN> BACKUP database;
          \end{lstlisting}
      \subsection{Das BACKUP-Kommando}
        Das \languagerman{BACKUP}-Kommando verwendet die vorkonfigurierten Einstellungen und Standardeinstellungen des RMAN. Soll jedoch von diesen Einstellungen abgewichen werden, kann dies durch die Angabe von Parametern geschehen.
        \subsubsection{Das Backup-Ausgabeger\"at festlegen}
          Das \languagerman{BACKUP}-Kommando kennt die Klausel \languagerman{DEVICE TYPE}. Sie gibt an, ob das Backup auf einen Datentr\"ager oder einem SBT-Ger\"at gespeichert werden soll. In den beiden folgenden Beispielen wird die \languagerman{SHOW}-Anweisung dazu benutzt, um die aktuelle Channel-Einstellung vor dem Backup anzuzeigen.
\clearpage
          \begin{lstlisting}[caption={Ein Backup to Disk},label=admin1302,language=rman]
RMAN> SHOW CHANNEL;

RMAN configuration parameters for database with db_unique_name ORCL are:
RMAN configuration has no stored or default parameters

RMAN> BACKUP DEVICE TYPE disk database;
          \end{lstlisting}
          \begin{lstlisting}[caption={Ein Backup to Tape},label=admin1303,language=rman]
RMAN> SHOW CHANNEL;

using target database control file instead of recovery catalog
RMAN configuration parameters for database with db_unique_name ORCL are:
CONFIGURE CHANNEL DEVICE TYPE 'SBT_TAPE'
PARMS  'SBT_LIBRARY=oracle.disksbt,ENV=(BACKUP_DIR=/u04)';

RMAN> BACKUP DEVICE TYPE sbt database;
          \end{lstlisting}
        \subsubsection{Image Copy oder Backup Set?}
          Um ein Backup als Image Copy anzufertigen gibt es die \languagerman{AS COPY}-Klausel. Image Copies k\"onnen nur auf Disk-, aber nicht auf SBT-Ger\"ate gespeichert werden.
          \begin{lstlisting}[caption={Eine Image Copy der Datenbank erstellen},label=admin1304,language=rman]
RMAN> BACKUP AS COPY database;
          \end{lstlisting}
          Das Gegenst\"uck zur \languagerman{AS COPY}-Klausel ist die \languagerman{AS BACKUPSET}-Klausel.
          \begin{lstlisting}[caption={Ein Backup Set der Datenbank erstellen},label=admin1305,language=rman]
RMAN> BACKUP AS BACKUPSET database;
        \end{lstlisting}
        \subsubsection{Kompression verwenden}
          RMAN ist in der Lage Backup Sets zu komprimieren, um Speicherplatz zu sparen. Das Erstellen komprimierter Backup Sets ben\"otigt jedoch mehr Zeit als die Erstellung normaler Backup Sets.
          \begin{merke}
            Kompression kann f\"ur Image Copies nicht angewendet werden.
          \end{merke}
          \begin{lstlisting}[caption={Ein komprimiertes Backup Set der Datenbank erstellen},label=admin1306,language=rman]
RMAN> BACKUP AS COMPRESSED BACKUPSET database;
          \end{lstlisting}
        \subsubsection{Den Speicherort des Backups angeben}
          Mit Hilfe der \languagerman{FORMAT}-Klausel des \languagerman{BACKUP}-Kommandos kann angegeben werden, wo das Backup gespeichert werden soll. Standardm\"a\ss{}ig werden Backups in der Fast Recovery Area gespeichert.
          \begin{lstlisting}[caption={Den Speicherort des Backups \"andern},label=admin1307,language=rman]
RMAN> BACKUP database FORMAT = '/u04/backup/%d_%D-%M-%Y_%t_%s_%p.bkp';
          \end{lstlisting}
          Der folgenden Literaturhinweis erl\"autert die Bedeutung der einzelnen Platzhalter.
          \begin{literaturinternet}
            \item \cite[formatSpec]{formatSpec}.
          \end{literaturinternet}
      \subsubsection{Backups mit Tags versehen}
          RMAN h\"angt an jedes Backup einen kurzen Kommentar an, der als \enquote{Tag}\footnote{tag = engl. Schild} bezeichnet wird und zur besseren Identifizierung dient. Um manuell ein Tag an ein Backup anzuh\"angen, gibt es die \languagerman{TAG}-Klausel im \languagerman{BACKUP}-Kommando.
          \begin{lstlisting}[caption={Ein Backup mit manuellem Tag versehen},label=admin1308,language=rman]
RMAN> BACKUP database TAG = 'Full-Backup 28.10.13';
          \end{lstlisting}
      \subsection{Full-Backups}
        Unter einem Full-Backup (Vollsicherung) wird die vollst\"andige Sicherung
        \begin{itemize}
          \item einer Datendatei,
          \item eines Tablespaces,
          \item oder einer Datenbank verstanden.
        \end{itemize}
        Das Gegenst\"uck zur Vollsicherung ist die inkrementelle Sicherung.
        \subsubsection{Backup einer Datenbank}
          \beispiel{admin1309} zeigt die einfachste Variante eines kompletten Backups einer Datenbank.
          \begin{lstlisting}[caption={Backup einer ganzen Datenbank},label=admin1309,language=rman,emph={[9]ALTER,SYSTEM,ARCHIVE,LOG,CURRENT},emphstyle={[9]\color{magenta}\bfseries}]
RMAN> BACKUP database;
RMAN> SQL 'ALTER SYSTEM ARCHIVE LOG CURRENT';
          \end{lstlisting}
          Das Kommando \languagerman{BACKUP database} fertigt das Backup der Datenbank an. Im zweiten Schritt werden mit dem Kommando \languageorasql{ALTER SYSTEM ARCHIVE LOG CURRENT} die aktiven Redo Logs gewechselt (Log Switch) und archiviert. Somit wird sichergestellt, dass alle f\"ur das Recovery mit diesem Backup ben\"otigten Redo Logs archiviert wurden. Wird dieses Backup als Hotbackup durchgef\"uhrt, sollten anschliessend noch die Archive Logs gesichert werden.
        \subsubsection{Backups einzelner Tablespaces}
          Um ein Backup eines oderer mehrerer Tablespaces durchzuf\"uhren, gibt es das Kommando \languagerman{BACKUP tablespace}. Im folgenden Beispiel werden die beiden Tablespaces \identifier{users} und \identifier{bank} auf ein SBT-Ger\"at gesichert.
          \begin{lstlisting}[caption={Backup eines Tablespace},label=admin1310,language=rman,emph={[9]ALTER,SYSTEM,ARCHIVE,LOG,CURRENT},emphstyle={[9]\color{magenta}\bfseries}]
RMAN> BACKUP DEVICE TYPE sbt tablespace users, bank;
RMAN> SQL 'ALTER SYSTEM ARCHIVE LOG CURRENT';
          \end{lstlisting}
        \subsubsection{Backups einzelner Datendateien}
          Um einzelne Datendateien sichern zu k\"onnen, gibt es das Kommando \languagerman{BACKUP datafile}. Wenn beispielsweise die Datendateien mit den Nummern 1, 2 und 4 gesichert werden sollen, k\"onnte dies wie folgt geschehen:
          \begin{lstlisting}[caption={Backup einzelner Datendateien},label=admin1311,language=rman]
RMAN> BACKUP datafile 1, 2, 4;
          \end{lstlisting}
          \begin{merke}
            Ist das Controlfile Autobackup aktiviert, wird anschliessend an dieses Backup, automatisch ein Backup Set mit der aktuellen Kontrolldatei und der Serverparameterdatei angelegt. Ist das Controlfile Autobackup deaktiviert, werden die aktuelle Kontrolldatei und die Serverparameterdatei automatisch in das Backupset der Datendateien integriert, da Tablespace Nummer 1 gesichert wurde.
          \end{merke}
          Um die Nummern der Datendateien herauszufinden kann die View \identifier{dba\_data\_files} wie folgt benutzt werden:
          \begin{lstlisting}[caption={Herausfinden der Datendateinummern},label=admin1312,language=oracle_sql]
SQL> SELECT   file_id, file_name, tablespace_name
  2  FROM     dba_data_files
  3  ORDER BY tablespace_name, file_id;
          \end{lstlisting}
      \subsection{Inkrementelle Backups}
        F\"uhrt der RMAN ein inkrementelles Backup durch, sichert er nur die Datenbl\"ocke, die sich seit einem bestimmten Backup, welches als Referenz angegeben wurde, ge\"andert haben. Inkrementelle Backups k\"onnen sowohl von ganzen Datenbanken als auch von Tablespaces und Datendateien durchgef\"uhrt werden.

        Es gibt verschiedene Begr\"undungen, warum inkrementelle Backups in eine Backup-Stra\-te\-gie mit einbezogen werden sollten:
        \begin{itemize}
          \item Reduzieren des Zeitaufwands f\"ur t\"agliche Backups
          \item Netzwerkbandbreite durch kleinere Datenmengen einsparen
        \end{itemize}
        \subsubsection{Wie funktionieren inkrementelle Backups?}
          Jeder Datenblock in einer Oracle Datenbank f\"uhrt in seinem Header eine System Change Number (SCN). Dies ist immer die SCN, die w\"ahrend der letzten Ver\"anderung am Datenblock aktuell war. W\"ahrend eines inkrementellen Backups liest RMAN die SCN jedes Datenblocks und vergleicht diese mit der SCN des Referenzbackups. Ist die SCN im Datenblock gr\"o\ss{}er, als die SCN im Referenzbackup, wird der Datenblock gesichert.

          Inkrementelle Backups k\"onnen als Level 0 oder als Level 1 Backups angelegt werden. Ein Level 0 Backup sichert alle Datenbl\"ocke genau wie ein normales Full-Backup. Der einzige Unterschied ist, dass ein normales Full-Backup nicht in eine inkrementelle Backup-Strategie integriert werden kann.

          F\"ur Level 1 Backups gibt es zwei verschiedene Arten:
          \begin{itemize}
            \item \textbf{Inkrementelles Backup}: Es werden alle Datenbl\"ocke gesichert, die sich seit dem letzten Level 1 oder Level 0 Backup ge\"andert haben.
            \item \textbf{Kumulatives Backup}: Es werden alle Datenbl\"ocke gesichert, die sich seit dem letzten Level 0 Backup ge\"andert haben.
          \end{itemize}
           Inkrementelle Backups sparen Speicherplatz und Zeit w\"ahrend des Backupvorgangs, da unter Umst\"anden weniger Datenbl\"ocke gesichert werden m\"ussen, als bei einem kumulativen Backup.

          Der Vorteil der kumulativen Backups liegt darin, das die Recovery-Zeit geringer ist, als bei den Inkrementellen. Bei inkrementellen Backups m\"ussen alle Backups, seit dem letzten Level 0 Backup in der richtigen Reihenfolge wiederhergestellt werden. Bei kumulativen Backups muss nur das aktuellste Backup seit dem letzten Level 0 Backup aufgespielt werden.
        \subsubsection{Level 0 Backups durchf\"uhren}
          Die Grundlage einer jeden inkrementellen Backupstrategie sind Level 0 Backups. Egal welche Art von Level 1 Backups gemacht wird, die Art und Weise wie Level 0 Backups erstellt werden, bleibt davon unber\"uhrt. \beispiel{admin1313} zeigt verschiedene Varianten eines Level 0 Backups.
          \begin{lstlisting}[caption={Inkrementelles Level 0 Backup},label=admin1313,language=rman]
-- Level 0 Backup der Datenbank
RMAN> BACKUP INCREMENTAL LEVEL 0 database;
-- Level 0 Backup eines Tablespaces
RMAN> BACKUP INCREMENTAL LEVEL 0 tablespace bank;
-- Level 0 Backup einer Datendatei
RMAN> BACKUP INCREMENTAL LEVEL 0 datafile 1;
          \end{lstlisting}
        \subsubsection{Level 1 Backups durchf\"uhren}
          Zur Durchf\"uhrung eines Level 1 Backups wird lediglich die Klausel \languagerman{LEVEL 0} durch \languagerman{LEVEL 1} ersetzt. Hierbei handelt es sich dann um ein inkrementelles Backup.
          \begin{lstlisting}[caption={Inkrementelles Level 1 Backup der Datenbank},label=admin1314,language=rman]
RMAN> BACKUP INCREMENTAL LEVEL 1 database;
          \end{lstlisting}
          \abbildung{differentiel_backup} zeigt ein Beispiel, in dem an jedem Sonntag ein inkrementelles Level 0 Backup durchgef\"uhrt wird. Dieses dient dann als Referenzbackup f\"ur das Level 1 Backup, welches am darauf folgenden Montag erstellt wird. Das Level 1 Backup vom Montag gilt dann wiederum als Referenzbackup f\"ur das Level 1 Backup am Dienstag, bis am n\"achsten Sonntag wieder ein Level 0 Backup erstellt wird und die Kette von vorne beginnt.

          \bild{Eine inkrementelle Backup\-strategie}{differentiel_backup}{1}

        \subsubsection{Kumulative Backups durchf\"uhren}
          Um kumulative Backups durchzuf\"uhren, muss zus\"atzlich zur \languagerman{INCREMENTAL LEVEL N}-Klausel noch das Schl\"usselwort \languagerman{CUMULATIVE} angegeben werden.
          \bild{Eine kumulative Backup\-strategie}{cumulative_backup}{1}
          \begin{lstlisting}[caption={Kumulatives Level 1 Backup der Datenbank},label=admin1315,language=rman]
RMAN> BACKUP INCREMENTAL LEVEL 1 CUMULATIVE database;
          \end{lstlisting}
      \subsection{Backups der Archive Logs durchf\"uhren}
        \label{backupderarchivelogsdurchfuehren}
        \subsubsection{Separate Sicherung der Archive Logs}
          Archive Logs sind der Schl\"ussel zu einer erfolgreichen Backup und Recoverystrategie, weshalb sie regelm\"assig gesichert werden sollten. \"Ahnlich wie Kontrolldateien k\"onnen Archive Logs als eigenes Backup Set gesichert oder in einem anderen Backup Set mitgesichert werden. Um ein manuelles Backup der Archive Logs durchzuf\"uhren gibt es die \languagerman{ARCHIVELOG-ALL}-Klausel des \languagerman{BACKUP}-Kommandos.
          \begin{lstlisting}[caption={Manuelles Backup aller Archive Logs},label=admin1316,language=rman]
RMAN> BACKUP archivelog ALL;
          \end{lstlisting}
          Anders als bei den Datendateien, muss bei der Sicherung der Archive Logs immer angegeben werden, welche Archive Logs zu sichern sind. In \beispiel{admin1316} wird das Schl\"usselwort \languagerman{ALL} benutzt, um alle Archive Logs zu sichern. Es gibt jedoch noch andere M\"oglichkeiten, um die Menge der zu sichernden Archive Logs einzuschr\"anken.
          \begin{lstlisting}[caption={Alle Archive Logs ab Sequenz Nummer 4711 sichern},label=admin1317,language=rman]
RMAN> BACKUP archivelog
2>    FROM SEQUENCE 4711;
          \end{lstlisting}
          \begin{lstlisting}[caption={Alle Archive Logs bis Sequenz Nummer 4711 sichern},label=admin1318,language=rman]
RMAN> BACKUP archivelog
2>    UNTIL SEQUENCE 4711;
          \end{lstlisting}
          \begin{lstlisting}[caption={Alle Archive Logs zwischen Sequenz Nummer 666 und 4711 sichern},label=admin1319,language=rman]
RMAN> BACKUP archivelog
2>    SEQUENCE BETWEEN 666 AND 4711;
          \end{lstlisting}
          Wie in \ref{administeringarchivelogs} bereits erw\"ahnt, sollte aus Sicherheitsgr\"unden die Archivierung der Log Dateien immer an mehreren Speicherorten erfolgen. Wenn nun mehrere identische Kopien einer Archive Log Datei existieren, wird der RMAN immer nur die erste verf\"ugbare Kopie sichern. Archive Logs gelten als identische wenn folgende Attribute gleich sind:
          \begin{itemize}
            \item die Datenbank ID (DBID),
            \item die Log Thread Number (Laufende Nummer des LGWr-Prozesses, nur in RAC-Umgebungen relevant)
            \item die Log Sequence Number
            \item die Reset Logs SCN
          \end{itemize}
        \subsubsection{Archive Logs mitsichern}
          Eine andere Variante ist die Archive Logs beim Backup der Datenbank oder eines Teils der Datenbank mitzusichern. Dies geschieht durch die Klausel \languagerman{PLUS archivelog}, die dem \languagerman{BACKUP}-Kommando angeh\"angt werden kann.
          \begin{lstlisting}[caption={Manuelles Backup des Archive Logs},label=admin1320,language=rman]
RMAN> BACKUP database PLUS archivelog;
        \end{lstlisting}
        Nach dem Absetzen dieses Kommandos f\"uhrt der RMAN folgende Schritte durch:
        \begin{enumerate}
          \item Es wird ein Log Switch herbeigef\"uhrt.
          \item RMAN f\"uhrt das Kommando \languagerman{BACKUP archivelog ALL} aus.
          \item Das Backup der Datenbank wird angelegt.
          \item Es wird erneut ein Log Switch herbeigef\"uhrt
          \item Das Kommando \languagerman{BACKUP archivelog ALL} wird ein zweites mal ausgef\"uhrt.
        \end{enumerate}
        Damit wird sichergestellt, das alle Archive Logs, die vor und w\"ahrend des Backups entstanden sind mitgesichert werden.
        \subsubsection{Gesicherte Archive Logs automatisch l\"oschen}
          RMAN bietet die M\"oglichkeit, Archive Logs automatisch, nach ihrer Sicherung zu l\"oschen. Dies erspart ein manuelles Eingreifen des Administrators. Um die Archive Logs automatisch zu l\"oschen, muss eine der beiden Klauseln \languagerman{DELETE INPUT} oder \languagerman{DELETE ALL INPUT} an das \languagerman{BACKUP ARCHIVELOG}-Kommando ange\"angt werden. Der Unterschied zwischen diesen Klauseln liegt darin, welche Archive Logs gel\"oscht werden:
          \begin{itemize}
            \item \languagerman{DELETE INPUT}: Es werden nur die Kopien einer Archive Log Datei gel\"oscht, die vom RMAN zur Sicherung benutzt wurden. Alle anderen Kopien bleiben erhalten.
            \item \languagerman{DELETE ALL INPUT}: Es werden alle Kopien einer Archive Log Datei gel\"oscht, auch die, welche nicht zur Sicherung benutzt wurden.
          \end{itemize}
          \bild{Auto\-matisches l\"oschen der Archive Logs}{delete_input_and_delete_all_input}{1.5}
          \beispiel{admin1321} zeigt die Anwendung der \languagerman{DELETE ALL INPUT}-Klausel.
          \begin{lstlisting}[caption={Manuelles Backup und gleichzeitiges L\"oschen der Archive Logs},label=admin1321,language=rman]
RMAN> BACKUP archivelog ALL
2>    DELETE ALL INPUT;
          \end{lstlisting}
    \subsection{Backups der Fast Recovery Area anfertigen}
      \label{FRABackups}
      In verschiedenen F\"allen kann es sinnvoll sein, den Inhalt der Fast
      Recovery Area auf Band zu sichern. Eine einfache Vorgehensweise beim
      Sichern der Fast Recovery Area k\"onnte sein:
\clearpage
      \begin{enumerate}
        \item Festlegen einer passenden Backup Retention Policy
        \item Backup der Datenbank durchf\"uhren
        \item Sichern der Fast Recovery Area auf Band
        \item Obsolete Backups der Fast Recovery Area von den B\"andern l\"oschen
        \item Obsolete Datenbankbackups aus der Fast Recovery Area l\"oschen
      \end{enumerate}

      \begin{lstlisting}[caption={Sichern der Fast Recovery Area},label=admin1322,language=rman]
RMAN> CONFIGURE RETENTION POLICY TO REDUNDANCY 2;

RMAN> RUN {
2>      ALLOCATE CHANNEL c_disk
3>      DEVICE TYPE disk;
4>      ALLOCATE CHANNEL c_sbt
5>      DEVICE TYPE sbt
6>      PARMS  'SBT_LIBRARY=oracle.disksbt,ENV=(BACKUP_DIR=/u04)';
7>
8>      BACKUP CHANNEL c_disk database;
9>      BACKUP CHANNEL c_sbt  recovery area;
10>    }
RMAN> DELETE obsolete;
      \end{lstlisting}
      \begin{merke}
        Der Inhalt der Fast Recovery Area kann mittels \languagerman{BACKUP RECOVERY AREA} auf SBT gesichert werden. Eine Sicherung auf Festplatte ist nicht m\"oglich.
      \end{merke}
      \subsection{Manuelle Kontrolldatei-Backups}
        \subsubsection{Backup mit dem RMAN}
          RMAN benutzt ein \enquote{Snapshot Controfile}, um konsistente Backups der Kontrolldatei anzufertigen. Ist das Controlfile Autobackup-Feature des RMAN aktiviert, macht RMAN automatisch Backups der Kontrolldatei und der Serverparameterdatei nach jedem Backup und jeder Struktur\"anderung der Datenbank. Ist dieses Feature nicht aktiviert, sollten regelm\"assig manuelle Backups der Kontrolldatei erfolgen. Dies kann auf unterschiedliche Arten geschehen:
        \begin{itemize}
          \item Durch das RMAN-Kommando \languagerman{BACKUP current controlfile}
          \item In dem eine Kopie der Kontrolldatei in das aktuelle Backup aufgenommen wird
          \item Durch das Sichern von Datendatei Nummer 1, da dabei die Kontrolldatei immer automatisch mitgesichert wird. Dies geschieht zum Beispiel im Rahmen eines Daten\-bank-Backups.
        \end{itemize}
        Damit die Kontrolldatei bei einem beliebigen Backup mitgesichert werden
        kann, gibt es die Klausel \languagerman{INCLUDE current controlfile}.
        \begin{lstlisting}[caption={Kontrolldatei in ein Backup mit einschlie\ss{}en},label=admin1323,language=rman]
RMAN> BACKUP tablespace bank
2>    INCLUDE current controlfile;
        \end{lstlisting}
        \subsubsection{Backup to Trace}
          Oracle stellt eine zweite Variante zur Sicherung des Controlfiles zur Verf\"ugung: Das Backup to Trace. Hierbei wird keine Bin\"ar-Kopie, sondern ein SQL-Skript erstellt, mit dessen Hilfe die Kontrolldatei neu kreiert werden kann. Erzeugt wird dieses Skript mit Hilfe des SQL-Kommandos \languageorasql{ALTER DATABASE BACKUP}.
        \begin{lstlisting}[caption={Kontrolldatei in ein Backup mit einschlie\ss{}en},label=admin1324,language=oracle_sql]
SQL> ALTER DATABASE
  2  BACKUP CONTROLFILE TO TRACE
  3  AS '/home/oracle/control.bkp';
        \end{lstlisting}
        Und so sieht es aus:
        \begin{lstlisting}[caption={Das \languageorasql{CREATE CONTROLFILE}-Skript},label=admin1325,language=oracle_sql]
startup nomount
CREATE CONTROLFILE REUSE DATABASE "ORCL" NORESETLOGS  ARCHIVELOG
    MAXLOGFILES 16
    MAXLOGMEMBERS 3
    MAXDATAFILES 100
    MAXINSTANCES 8
    MAXLOGHISTORY 292
LOGFILE
  GROUP 1 '/u01/app/oracle/oradata/orcl/redo01.log'  SIZE 50M BLOCKSIZE 512,
  GROUP 2 '/u01/app/oracle/oradata/orcl/redo02.log'  SIZE 50M BLOCKSIZE 512,
  GROUP 3 '/u01/app/oracle/oradata/orcl/redo03a.log'  SIZE 50M BLOCKSIZE 512,
  GROUP 4 '/u01/app/oracle/oradata/orcl/redo04a.log'  SIZE 50M BLOCKSIZE 512,
  GROUP 5 (
    '/u01/app/oracle/oradata/orcl/redo05a.log',
    '/u02/oradata/orcl/redo05b.log'
  ) SIZE 50M BLOCKSIZE 512

DATAFILE
  '/u01/app/oracle/oradata/orcl/system01.dbf',
  '/u01/app/oracle/oradata/orcl/sysaux01.dbf',
  '/u01/app/oracle/oradata/orcl/undotbs01.dbf',
  '/u01/app/oracle/oradata/orcl/users01.dbf',
  '/u01/app/oracle/oradata/orcl/example01.dbf',
  '/u01/app/oracle/oradata/orcl/bank01.dbf'
CHARACTER SET WE8MSWIN1252;
          \end{lstlisting}
          \beispiel{admin1325} zeigt nur einen Ausschnitt aus dem \languageorasql{CREATE CONTROLFILE}-Skript.

          Interessant ist ein solches Skript immer dann, wenn Probleme mit der Datenbank auftreten, die sich auf dem \enquote{offiziellen} Weg nicht l\"osen lassen. Dies ist unter anderem dann der Fall, wenn alle Dateien der Current Redo Log Group besch\"adigt sind.
      \subsection{Backup des SPFile}
        Das SPFile wird in verschiedenen Situationen automatisch mitgesichert (siehe \ref{controlfileautobackup}). Um ein manuelles Backup des SPFile durchzuf\"uhren gibt es das folgende Kommando:
        \begin{lstlisting}[caption={Manuelles Backup des SPFile},label=admin1326,language=rman]
RMAN> BACKUP spfile;
          \end{lstlisting}
      \subsection{Backupduplexing konfigurieren}
        Es ist m\"oglich RMAN so zu konfigurieren, dass von jedem Backuppiece mehrere identische Kopien angefertigt werden. Dieses Features ist als \enquote{duplexing} bekannt und bezieht sich nur auf Backup Sets, nicht aber auf Image Copies.
        \begin{lstlisting}[caption={Konfigurieren des Backupduplexing},label=admin1327,language=rman]
RMAN> CONFIGURE DATAFILE BACKUP COPIES
2>    FOR DEVICE TYPE disk TO 2;
RMAN> CONFIGURE ARCHIVELOG BACKUP COPIES
2>    FOR DEVICE TYPE sbt TO 2;

RMAN> BACKUP database PLUS ARCHIVELOG
2>    FORMAT '/u02/backups/%U', '/u03/backups/%U';
        \end{lstlisting}
        \beispiel{admin1327} zeigt, wie sowohl f\"ur Datendateien, als auch f\"ur Archive Log Dateien, das Backup Duplexing auf den Wert zwei konfiguriert wird. RMAN erstellt zwei identische Kopien der Datendateien und verteilt diese auf die beiden Speicherorte \oscommand{/u02/backups} und \oscommand{/u03/backups}. Ein solcher Formatstring kann mit Hilfe eines der Kommandos \languagerman{BACKUP}, \languagerman{CONFIGURE CHANNEL} oder \languagerman{ALLOCATE CHANNEL} angegeben werden.
        \begin{merke}
          Zu beachten ist dabei, dass das Backup Duplexing nicht zusammen mit der Fast Recovery Area funktioniert. Dies wird durch die folgende Fehlermeldung angezeigt:
          \begin{verbatim}
ORA-19806: cannot make duplex backups in recovery area.
          \end{verbatim}
          Aus diesem Grund muss immer ein Format-String verwendet werden, der die Duplexkopien ausserhalb der Fast Recovery Area speichert. Es k\"onnen maximal 4 identische Kopien eines Backups erzeugt werden.
        \end{merke}
    \section{Backups verwalten}
      \subsection{Backups katalogisieren}
        In bestimmten Situationen kann es notwendig sein, dass Image Copies, Backupsets oder Archive Logs im RMAN Repository oder im RMAN Katalog neu erfasst werden m\"ussen. Ein solcher Fall tritt ein, wenn:
        \begin{itemize}
          \item Kopien von Datendateien ohne Zuhilfenahme des RMAN angefertigt wurden. Diese k\"onnen dann als Datafile Copies im Repository/Katalog registriert werden.
          \item Backup Pieces ohne RMAN auf dem Datentr\"ager verschoben wurden.
          \item Backup Pieces aus dem Repository gel\"oscht wurden, die Dateien selbst aber noch existieren und wiederverwendet werden sollen.
          \item die Kontrolldatei verloren geht und kein Recovery Katalog genutzt wird.
          \end{itemize}
        \begin{lstlisting}[caption={Backup Pieces katalogisieren},label=admin1328,language=rman]
RMAN> CATALOG BACKUPPIECE  '/u03/backup/backup_820.bkp',
2>                         '/u04/backup/backup_821.bkp';
        \end{lstlisting}
        In \beispiel{admin1328} wird RMAN versuchen beide Backup Pieces zu katalogisieren. Er wird seine Arbeit auch dann fortsetzen, wenn eines der beiden Pieces defekt ist. Es wird dann nur das Funktionsf\"ahige ins Repository aufgenommen. Soll ein aus mehreren Pieces bestehendes Backup Set katalogisiert werden, gelingt dies nur, wenn alle Pieces fehlerfrei sind.
        \begin{lstlisting}[caption={Datendatei-Kopien katalogisieren},label=admin1329,language=rman]
RMAN> CATALOG DATAFILECOPY '/u03/backup/bank01.dbf';
        \end{lstlisting}
        \begin{lstlisting}[caption={Archive Logs katalogisieren},label=admin1330,language=rman]
RMAN> CATALOG ARCHIVELOG   '/u03/backup/archive1_731.dbf',
2>                         '/u03/backup/archive1_732.dbf';
        \end{lstlisting}
        Es ist m\"oglich, den Inhalt eines gesamten Verzeichnisses, in einem Arbeitsschritt zu katalogisieren.
        \begin{lstlisting}[caption={Verzeichnisinhalt katalogisieren},label=admin1331,language=rman]
RMAN> CATALOG START WITH '/disk1/backups/';
        \end{lstlisting}
        Bei dieser Art des Katalogisierens, muss beachtet werden, dass es sich bei der Angabe von \oscommand{/disk1/backups/} nicht um einen Verzeichnisnamen handelt, sondern nur um ein Prefix. Wird \oscommand{/disk1/backups} angegeben, werden alle Verzeichnisse, die mit dieser Zeichenkette beginnen, z. B. \oscommand{/disk1/backupsets} oder \oscommand{/disk1/backups-jahr-2013} ebenfalls katalogisiert. Dies kann dazu f\"uhren, dass unbeabsichtigt  Dateien in den Recovery Katalog aufgenommen werden.

        \begin{merke}
          Um solche Probleme zu vermeiden, sollte das Prefix immer mit einem / abgeschlossen werden, also z. B. \oscommand{/disk1/backups/}.
        \end{merke}
      \subsection{Backup-Crosschecks durchf\"uhren}
        Ein Crosscheck vergleicht den Recovery Katalog mit dem Inhalt des Dateisystems, um Unterschiede festzustellen. Wenn z. B. ein Nutzer ein Backup Set mit Betriebssystemmitteln von der Festplatte l\"oscht, existiert trotzdem noch der Eintrag f\"ur dieses Backup Set im Recovery Katalog. Mit Hilfe des \languagerman{CROSSCHECK}-Kommandos k\"onnen solche Inkonsistenzen bereinigt werden.
        Nach einem Crosscheck bekommt jeder Eintrag im RMAN einen Status:
        \begin{itemize}
          \item \textbf{EXPIRED}: Das Backup ist nicht auf dem Dateisystem verf\"ugbar.
          \item \textbf{AVAILABLE}: Das Backup ist verf\"ugbar und darf genutzt werden.
          \item \textbf{UNAVAILABLE}: Das Backup ist verf\"ugbar, darf aber nicht durch den RMAN genutzt werden.
        \end{itemize}
        \begin{merke}
          Der Status expired sollte nicht mit dem Status obsolete verwechselt werden.
        \end{merke}
        Die Ergebnisse eines Crosschecks k\"onnen direkt im RMAN, mit Hilfe des \languagerman{LIST}-Kommandos oder in SQL*PLUS, mit Hilfe der View \identifier{v\$backup\_files} betrachtet werden. W\"ahrend eines Crosschecks werden keine Dateien von der Festplatte und auch keine RMAN-Eintr\"age gel\"oscht. Lediglich der Status eines Eintrags wird ver\"andert. Um Backups zu l\"oschen, muss das \languagerman{DELETE}-Kommando des RMAN verwendet werden (siehe \ref{deletingbackups}).
        Einfache Beispiele f\"ur die Nutzung des \languagerman{CROSSCHECK}-Kommandos k\"onnten so aussehen:
        \begin{lstlisting}[caption={\languagerman{CROSSCHECK} aller Backups},label=admin1332,language=rman]
RMAN> CROSSCHECK backup;
        \end{lstlisting}
        Bei der Angabe von \languagerman{BACKUP} werden alle Arten von Backups
        \"uberpr\"uft. Es gibt verschieden M\"oglichkeiten, um die Menge der zu
        \"uberpr\"ufenden Backups einzuschr\"anken.
\clearpage
        \begin{lstlisting}[caption={\languagerman{CROSSCHECK} auf Backup Sets beschr\"anken},label=admin1333,language=rman]
RMAN> CROSSCHECK backupset;

RMAN> CROSSCHECK backupset 666, 815, 4711;

RMAN> CROSSCHECK backupset TAG = 'nightly_backup';
        \end{lstlisting}
        Das erste Kommando pr\"uft alle vorhandenen Backup Sets. Das zweite nur die Backup Sets mit den IDs 666, 815 und 4711. Das dritte \languagerman{CROSSCHECK}-Kommando checkt nur das Backup Piece mit dem Tag \enquote{nightly\_backup}.
        \begin{lstlisting}[caption={Nur Datafile Copies checken},label=admin1334,language=rman]
RMAN> CROSSCHECK datafilecopy ALL;

RMAN> CROSSCHECK datafilecopy 113, 114, 115;
        \end{lstlisting}
        Wird der Crosscheck auf Datafilecopies, Archive Logs oder Controlfilecopies angewendet, muss in jedem Fall die Menge der zu pr\"ufenden Dateien angegeben werden.
        \begin{lstlisting}[caption={Alle Arten von Copies (Datafilecopy, Archive Logs, Controlfilecopy) checken},label=admin1335,language=rman]
RMAN> CROSSCHECK copy;
        \end{lstlisting}
      \subsection{Manuelle Status\"anderungen an Backups}
        Mit Hilfe des \languagerman{CHANGE}-Kommandos kann der Status eines Backups manuell auf \enquote{available} oder \enquote{unavailable} gesetzt werden. Da der RMAN sich selbst die f\"ur eine Wiederherstellung notwendigen Backups sucht, ist es immer dann sinnvoll, ein Backup Set als \enquote{unavailable} zu markieren, wenn der RMAN es nicht f\"ur das Restore benutzen darf.
        \begin{merke}
          Zu beachten ist, dass Dateien die sich in der Fast Recovery Area befinden, nicht als \textit{unavailable} markiert werden k\"onnen.
        \end{merke}
        \begin{lstlisting}[caption=Manuelle Status\"anderung eines Backup Sets auf unavailable,label=admin1336,language=rman]
RMAN> CHANGE BACKUPSET 12 UNAVAILABLE;
        \end{lstlisting}
        \begin{lstlisting}[caption=Das Backup Set wieder verf\"ugbar machen,label=admin1337,language=rman]
RMAN> CHANGE BACKUPSET 12 AVAILABLE;
        \end{lstlisting}
        \begin{merke}
          Manuelle Status\"anderungen sind auch f\"ur Datafilecopies, Controlfilecopies und alle anderen Arten von Backups m\"oglich.
        \end{merke}
      \subsection{Backups l\"oschen}
        \label{deletingbackups}
        \subsubsection{Bestimmte Backups l\"oschen}
          In regelm\"a\ss{}igen Abst\"anden ist es notwendig, Backups zu l\"oschen und Platz f\"ur Neue zu schaffen. Diese Arbeit sollte immer mit dem RMAN-Kommando \languagerman{DELETE} verrichtet werden, da nicht nur die Dateien gel\"oscht, sondern auch das RMAN Repository gepflegt werden muss. Es bedient sich grunds\"atzlich der gleichen Syntax, wie der \languagerman{CROSSCHECK}-Befehl.
          \begin{lstlisting}[caption=Alle vorhandenen Backups l\"oschen,label=admin1338,language=rman]
RMAN> DELETE backup;
          \end{lstlisting}
          \begin{lstlisting}[caption=Nur Backup Sets l\"oschen,label=admin1339,language=rman]
RMAN> DELETE backupset;

RMAN> DELETE backupset 666, 815, 4711;
          \end{lstlisting}
          \languagerman{DELETE backupset} l\"oscht alle Backup Sets, w\"ahrend durch die Angabe von Backup Set IDs nur bestimmte Backup Sets gel\"oscht werden.
          \begin{lstlisting}[caption=Archive Logs l\"oschen,label=admin1340,language=rman]
RMAN> DELETE ARCHIVELOG ALL;

RMAN> DELETE ARCHIVELOG
2>    UNTIL SEQUENCE = 200;
          \end{lstlisting}
          \begin{merke}
            Gerade beim L\"oschen von Archive Logs ist gr\"o\ss{}te Vorsicht geboten!
          \end{merke}
          RMAN fragt vor dem L\"oschen jeder Datei nach. Diese Nachfrage kann durch die Angabe von \languagerman{DELETE NOPROMPT} unterbunden werden.
        \subsubsection{Obsolete Backups l\"oschen}
          Die Backup Retention Policy legt fest, welche Backups f\"ur ein Recovery ben\"otigt werden und welche nicht (siehe \ref{backupretentionpolicy}). Verst\"o\ss{}t ein Backup gegen die Retention Policy, wird es als \enquote{obsolete} markiert. Obsolete Backups k\"onnen mittels \languagerman{DELETE}-Befehl gel\"oscht werden.
          \begin{lstlisting}[caption=Obsolete Backups l\"oschen,label=admin1341,language=rman]
RMAN> DELETE obsolete;
          \end{lstlisting}
        \subsubsection{Expired-Backups l\"oschen}
          Wird ein Crosscheck ausgef\"uhrt, um den Inhalt des Recovery Katalogs
          mit dem Dateisystem zu vergleichen, werden Backups die zwar noch im
          Katalog eingetragen sind, aber auf dem Dateisystem nicht mehr
          existieren auf den Status \textit{expired} gesetzt. Mit dem Kommando
          \languagerman{DELETE EXPIRED} k\"onnen solche Eintr\"age anschliessend
          entfernt werden. Sollte die zu dem Eintrag geh\"orende Datei noch
          existieren, wird sie ebenfalls gel\"oscht.

          Folgender Vorgang l\"oscht alle als \textit{expired} markierten
          Backups:
          \begin{enumerate}
            \item Durchf\"uhren eines Crosschecks.
              \begin{lstlisting}[caption={CROSSCHECK durchf\"uhren},label=admin1342,language=rman]
CROSSCHECK backup;
              \end{lstlisting}
            \item L\"oschen der als \textit{expired} markierten Backups.
              \begin{lstlisting}[caption={L\"oschen der Backups},label=admin1343,language=rman]
DELETE EXPIRED backup;
              \end{lstlisting}
          \end{enumerate}
      \section{Block Change Tracking}
        Das \enquote{Block Change Tracking}-Feature des RMAN erh\"oht die Performance inkrementeller Backups dadurch, dass Informationen \"uber ge\"anderte Bl\"ocke in einem \enquote{Trackingfile} gespeichert werden. Wenn RMAN das Trackingfile benutzt, entf\"allt das Durchsuchen der Datenbank nach ge\"anderten Bl\"ocken.

        Das erste Level 0 Backup, nach dem Einschalten des Block Change Tracking, zieht noch keinen Nutzen daraus, da das Trackingfile noch nicht den aktuellen Stand in der Datenbank wiedergeben kann. Aber bereits das n\"achste Level 1 Backup, kann die Informationen im Trackingfile nutzen.

        \begin{merke}
          Durch die Nutzung von Block Change Tracking \"andert sich nichts an den Backup- oder Recovery-Kommandos.
        \end{merke}

        Standardm\"a\ss{}ig ist das Block Change Tracking deaktiviert, da es geringe Performanceeinbussen, im normalen Betrieb der Datenbank mit sich bringt. Der Performancegewinn bei der Durchf\"uhrung inkrementeller Backups, mit gro\ss{}en Datenmengen, ist jedoch sehr hoch.

        Das Trackingfile kann an einem beliebigen Ort auf dem Datentr\"ager gespeichert werden. Oracle empfiehlt es in der Fast Recovery Area (Standardspeicherort) abzulegen.

        \begin{merke}
          Im Trackingfile werden f\"ur bis zu acht inkrementelle Backups (ein Level 0 plus sieben Level 1) die Trackinginformationen gespeichert. Sobald ein weiteres Level 1 Backup angelegt wird, werden die Informationen zum ersten Backup (Level 0) \"uberschrieben.
        \end{merke}
        Um dieses Feature zu aktivieren, muss ein Trackingfile angegeben werden. Dies kann nur geschehen, wenn die Datenbank ge\"offnet oder in der Mount-Phase ist.
        \begin{lstlisting}[caption={Block Change Tracking
        aktivieren},label=admin1344,language=oracle_sql]
SQL> ALTER DATABASE
  2  ENABLE BLOCK CHANGE TRACKING
  3  USING FILE '/u02/rman_change_track.f' REUSE;
        \end{lstlisting}
        Die \languageorasql{USING FILE}-Klausel gibt den Pfad und den Dateinamen des Tracking\-files an. Das Schl\"us\-sel\-wort \languageorasql{REUSE} sorgt daf\"ur, dass eine evtl. bereits bestehende Datei \"uberschrieben wird.

        Verwenden Sie die Klausel \languageorasql{DISABLE BLOCK CHANGE TRACKING}, um das Block Change Tracking zu deaktivieren.
        \begin{lstlisting}[caption={Block Change Tracking deaktivieren},label=admin1345,language=oracle_sql]
SQL> ALTER DATABASE
  2  DISABLE BLOCK CHANGE TRACKING;
        \end{lstlisting}
        Mit Hilfe der View \identifier{v\$block\_change\_tracking} kann man ersehen, ob das Block Change Tracking aktiviert ist. Sie enth\"alt eine Spalte \identifier{status}, die den Zustand des Block Change Trackings angibt.
        \begin{lstlisting}[caption={Den Status des Block Change Trackings \"uberpr\"ufen},label=admin1346,language=oracle_sql,alsolanguage=sqlplus]
SQL> col status format a8 
SQL> col filename format a30
SQL> col bytes format 999,999.00
SQL> SELECT *
  2  FROM v$block_change_tracking;

STATUS   FILENAME                             BYTES
-------- ------------------------------ -----------
DISABLED
        \end{lstlisting}
        \begin{literaturinternet}
          \item \cite{BRADV89537}
        \end{literaturinternet}
    \section{Das Kommando LIST}
      Das Kommando \languagerman{LIST} benutzt das RMAN Repository, um
      verschiedene Informationen \"uber Backups und Archive Logs anzuzeigen. Es
      ist nur dann funktionsf\"ahig, wenn eine Verbindung zur Zieldatenbank
      besteht.
      \subsection{Auflistungen gruppieren}
        \languagerman{LIST} kann Backups auf zwei Arten anzeigen:
        \begin{itemize}
          \item Gruppiert nach Backup: Zu jedem Backup werden die zugeh\"origen Dateien angezeigt.
          \item Gruppiert nach Datei: Zu jeder Datei wird das betreffende Backup angezeigt.
        \end{itemize}
        \subsubsection{Nach Backup gruppieren}
          \begin{lstlisting}[caption={\languagerman{LIST} - Alle Backups, gruppiert nach Backup anzeigen},label=admin1347, language=rman]
RMAN> LIST backup;

using target database control file instead of recovery catalog

List of Backup Sets
===================


BS Key  Type LV Size       Device Type Elapsed Time Completion Time
------- ---- -- ---------- ----------- ------------ ---------------
1       Full    591.36M    &DISK&        00:01:00     24-OCT-13
        BP Key: 1   Status: &AVAILABLE&  Compressed: NO  Tag: TAG20131024T141243
        Piece Name: /u05/fast_recovery_area/ORCL/backupset/201...
  List of Datafiles in backup &set& 1
  File LV Type Ckp &SCN&    Ckp Time  Name
  ---- -- ---- ---------- --------- ----
  1       Full 2054849    24-OCT-13 /u01/app/oracle/oradata/orcl/system01.dbf

BS Key  Type LV Size       Device Type Elapsed Time Completion Time
------- ---- -- ---------- ----------- ------------ ---------------
2       Full    9.36M      &DISK&        00:00:04     24-OCT-13
        BP Key: 2   Status: &AVAILABLE&  Compressed: NO  Tag: TAG20131024T141243
        Piece Name: /u05/fast_recovery_area/ORCL/backupset/201...
  &SPFILE& Included: Modification time: 23-OCT-13
  &SPFILE& db_unique_name: ORCL
  Control File Included: Ckp &SCN&: 2054873      Ckp time: 24-OCT-13
          \end{lstlisting}
          \begin{lstlisting}[caption={\languagerman{LIST} - Bestimmte Backup Sets, gruppiert nach Backup anzeigen},label=admin1348,language=rman]
RMAN> LIST backupset 666, 815, 4711;
          \end{lstlisting}
        \subsubsection{Nach Datei gruppieren}
          Um in umgekehrter Folge zu Gruppieren, muss dem
          \languagerman{LIST}-Kommando der Zusatz \languagerman{BY FILE}
          mitgegeben werden.

          \begin{lstlisting}[caption={\languagerman{LIST} - Backups, gruppiert nach Backup Datei anzeigen},label=admin1349,language=rman]
RMAN> LIST backup BY FILE;


List of Datafile Backups
========================

File Key     TY LV S Ckp &SCN&    Ckp Time  #Pieces #Copies Compressed Tag
---- ------- -  -- - ---------- --------- ------- ------- ---------- ---
1    1       B  F  A 2054849    24-OCT-13 1       1       NO         TAG2013...
2    7       B  F  A 2076573    27-OCT-13 1       1       NO         TAG2013...
6    5       B  F  A 2062383    25-OCT-13 1       1       NO         TAG2013...
     3       B  F  A 2054959    24-OCT-13 1       1       NO         TAG2013...
7    3       B  F  A 2054959    24-OCT-13 1       1       NO         TAG2013...

List of Control File Backups
============================

CF Ckp &SCN& Ckp Time  BS Key  S #Pieces #Copies Compressed Tag
---------- --------- ------- - ------- ------- ---------- ---
2076573    27-OCT-13 8       A 1       1       NO         TAG20131028T093159
2062392    25-OCT-13 6       A 1       1       NO         TAG20131025T112454
2054967    24-OCT-13 4       A 1       1       NO         TAG20131024T141608
2054873    24-OCT-13 2       A 1       1       NO         TAG20131024T141243
List of &SPFILE& Backups
======================

Modification Time BS Key  S #Pieces #Copies Compressed Tag
----------------- ------- - ------- ------- ---------- ---
27-OCT-13         8       A 1       1       NO         TAG20131028T093159
23-OCT-13         6       A 1       1       NO         TAG20131025T112454
23-OCT-13         4       A 1       1       NO         TAG20131024T141608
23-OCT-13         2       A 1       1       NO         TAG20131024T141243
          \end{lstlisting}
        \subsubsection{Anzeigen einer Zusammenfassung}
          Mit \languagerman{LIST ... SUMMARY} kann f\"ur jeden beliebigen Backuptyp eine Zusammenfassung angezeigt werden. Diese enth\"alt nur noch die wesentlichsten Informationen, in geraffter Form.
\clearpage
          \begin{lstlisting}[caption={Anzeigen einer Zusammenfassung},label=admin1350,language=rman]
RMAN> LIST backup SUMMARY;

List of Backups
===============
Key     TY LV S Device Type Completion Time #Pieces #Copies Compressed Tag
------- -- -- - ----------- --------------- ------- ------- ---------- ---
1       B  F  A &DISK&        24-OCT-13       1       1       NO        TAG2013...
2       B  F  A &DISK&        24-OCT-13       1       1       NO        TAG2013...
3       B  F  A &DISK&        24-OCT-13       1       1       NO        TAG2013...
4       B  F  A &DISK&        24-OCT-13       1       1       NO        TAG2013...
5       B  F  A SBT_TAPE    25-OCT-13       1       1       NO        TAG2013...
6       B  F  A SBT_TAPE    25-OCT-13       1       1       NO        TAG2013...
7       B  F  A &DISK&        28-OCT-13       1       1       NO        TAG2013...
8       B  F  A &DISK&        28-OCT-13       1       1       NO        TAG2013...
          \end{lstlisting}
      \subsection{Auflisten der Expired Backups}
        Es ist m\"oglich sich solche Backups anzeigen zu lassen, die im Repository als \enquote{expired} markiert sind.
          \begin{lstlisting}[caption={Expired Backups auflisten},label=admin1351,language=rman]
RMAN> LIST expired backup;

List of change, crosscheck, and delete Backup Sets
===================
BS Key  Size       Device Type Elapsed Time Completion Time
------- ---------- ----------- ------------ ---------------
7       136M       disk        00:00:20     04-NOV-12
        BP Key: 7   Status: available  Compressed: NO  Tag: TAG2012...
        Piece Name: /u05/fast_recovery_area/backup/2003_11_04/...
  List of Archived Logs in backup set 7
  Thrd Seq     Low scn    Low Time  Next scn   Next Time
  ---- ------- ---------- --------- ---------- ---------
  1    1       173832     21-OCT-12 174750     21-OCT-03
  1    2       174750     21-OCT-12 174755     21-OCT-03
  1    3       174755     21-OCT-12 174758     21-OCT-03

  1    37      533321     01-NOV-12 575472     03-NOV-03
  1    38      575472     03-NOV-12 617944     04-NOV-03
  1    39      617944     04-NOV-12 631495     04-NOV-03
BS Key  Type LV Size       Device Type Elapsed Time Completion Time
------- ---- -- ---------- ----------- ------------ ---------------
8       Full    2M         disk        00:00:01     04-NOV-12
        BP Key: 8   Status: available  Compressed: NO  Tag: TAG2012...
        Piece Name: /u05/fast_recovery_area/c-774627068-20121104-01
  Controlfile Included: Ckp scn: 631510       Ckp time: 04-NOV-12
  spfile Included: Modification time: 21-OCT-12
          \end{lstlisting}
      \subsection{Ausgew\"ahlte Backups anzeigen}
        Anbei noch einige Beispiele, die zeigen, wie verschiedenste Backups zur Auflistung ausgew\"ahlt werden k\"onnen.
        \begin{lstlisting}[caption={Auflistungen eingrenzen},label=admin1352,language=rman]
# Nur Full-Backups der Datenbank anzeigen
RMAN> LIST backup OF database;

# Image Copies einer bestimmten Datendatei anzeigen
RMAN> LIST copy OF datafile '/u02/oradata/ORCL/system01.dbf';

# Ein bestimmtes Backup Set anzeigen
RMAN> LIST backupset 213;

# Ein Backup Set nach seinem Tag aussuchen
RMAN> LIST backupset TAG 'weekly_full_db_backup';

# Alle Image Copies einer Datendatei, die auf einem sbt-Ger&\"a&t liegen anzeigen
RMAN> LIST copy OF datafile '/u02/oradata/ORCL/system01.dbf'
2>    DEVICE TYPE sbt;

# Alle Backups anzeigen, die in einem bestimmten Verzeichnis liegen
RMAN> LIST backup LIKE '/tmp/%';

# Image Copies anzeigen, die in einem bestimmten Zeitraum fertiggestellt wurden
RMAN> LIST copy OF datafile 2 COMPLETED BETWEEN '10-DEC-2002' AND '17-DEC-2002';

# Alle Archive Logs, die mindestens zweimal auf Band gesichert wurden anzeigen
RMAN> LIST archivelog ALL BACKED UP 2 TIMES TO DEVICE TYPE sbt;

# Auflisten eines bestimmten Backupsets nach Tag
RMAN> LIST backupset TAG 'weekly_full_db_backup';
          \end{lstlisting}
    \section{Das Kommando REPORT}
      Das \languagerman{REPORT}-Kommando analysiert das RMAN Repository der Zieldatenbank, um Antworten auf verschiedene Fragen zu liefern.
      \subsection{Welche Dateien wurden noch nicht gesichert?}
        Erf\"ullt eine Datendatei die aktuelle Backup Retention Policy nicht, weil noch nicht gen\"ugende Backups vorliegen, bzw. das \"alteste Backup noch nicht alt genug ist, um das Recovery Window zu gew\"ahrleisten, kann dies mit dem Kommando \languagerman{REPORT NEED BACKUP} ermittelt werden.
        \begin{lstlisting}[caption={Wer verst\"o\ss{}t gegen die Retention Policy?},label=admin1353,language=rman]
RMAN> REPORT NEED BACKUP;

RMAN retention policy will be applied to the command
RMAN retention policy is set to redundancy 1
Report of files with less than 1 redundant backups
File #bkps Name
---- ----- -----------------------------------------------------
3    0     /u01/app/oracle/oradata/orcl/undotbs01.dbf
4    0     /u01/app/oracle/oradata/orcl/users01.dbf
5    0     /u01/app/oracle/oradata/orcl/example01.dbf
        \end{lstlisting}
        Die Ausgabe aus \beispiel{admin1353} zeigt an, dass bei einer redundanzbasierenden Retention Policy mit dem Wert 1, noch insgesamt 3 Datendateien gesichert werden m\"ussen, ehe die konfigurierte Policy eingehalten wird.

        Es ist auch m\"oglich, mit der \languagerman{NEED BACKUP}-Klausel eine andere Retention Policy zu unterstellen. Soll zum Beispiel gepr\"uft werden, welche Datendateien noch gesichert werden m\"ussten, wenn eine redundanzbasierende Retention Policy mit dem Wert 2 G\"ultigkeit h\"atte, wird die \languagerman{NEED BACKUP}-Klausel wie folgt erg\"anzt:
        \begin{lstlisting}[caption={Datenbankdateien die gesichert werden m\"ussen},label=admin1354,language=rman]
RMAN> REPORT NEED BACKUP REDUNDANCY 2;

Report of files with less than 2 redundant backups
File #bkps Name
---- ----- -----------------------------------------------------
1    1     /u01/app/oracle/oradata/orcl/system01.dbf
2    1     /u01/app/oracle/oradata/orcl/sysaux01.dbf
3    0     /u01/app/oracle/oradata/orcl/undotbs01.dbf
4    0     /u01/app/oracle/oradata/orcl/users01.dbf
5    0     /u01/app/oracle/oradata/orcl/example01.dbf
7    1     /u01/app/oracle/oradata/orcl/bank01.dbf
        \end{lstlisting}
        Der gleiche Vorgang ist auch f\"ur eine Zeitfensterbasierende Retention Policy m\"oglich.
        \begin{lstlisting}[caption={Datenbankdateien die gesichert werden m\"ussen},label=admin1355,language=rman]
RMAN> REPORT NEED BACKUP RECOVERY WINDOW OF 3 DAYS;

Report of files that must be backed up to satisfy 3 days recovery window
File Days  Name
---- ----- -----------------------------------------------------
1    4     /u01/app/oracle/oradata/orcl/system01.dbf
3    1535  /u01/app/oracle/oradata/orcl/undotbs01.dbf
4    1535  /u01/app/oracle/oradata/orcl/users01.dbf
5    67    /u01/app/oracle/oradata/orcl/example01.dbf
7    4     /u01/app/oracle/oradata/orcl/bank01.dbf
        \end{lstlisting}
        Die Spalte \identifier{Tage} gibt an, wie viele Archive Log Informationen, in Tagen gerechnet, eine Datendatei ben\"otigt, um vollst\"andig wiederhergestellt werden zu k\"onnen.
      \subsection{Nicht wiederherstellbare Datendateien aufsp\"uren}
        Nicht rekonstruierbare Operationen (unrecoverable operations) sind
        Schreibvorg\"ange in Datendateien, die unter Umgehung der SGA
        stattfinden. Dabei werden Informationen direkt in Datendateien
        geschrieben, ohne vorher durch den Database Buffer Cache geschleust zu
        werden. Ein solcher Schreibvorgang wird als \enquote{Direct Load}
        bezeichnet und hat den Vorteil, dass er sehr viel schneller ist, als ein
        normaler Schreibvorgang. Der Nachteil an einer solchen Vorgehensweise
        ist jedoch, dass keine Redo-Informationen erzeugt werden.

        Daraus resultiert, dass eine so behandelte Datendatei im Falle eines
        Crashes auch nicht mit Hilfe der Archive Logs wiederhergestellt werden
        kann. Die Empfehlung seitens Oracle ist deshalb, eine solche Datei
        umgehend in einem Backup zu sichern.
        \begin{lstlisting}[caption={Datenbankdateien die nicht rekonstruierbare Operationen enthalten aufsp\"uren},label=admin1356,language=rman]
RMAN> REPORT UNRECOVERABLE;

Report of files that need backup due to unrecoverable operations
File Type of Backup Required Name
---- ----------------------- -----------------------------------
5    full                    /u01/app/oracle/oradata/orcl/example01.dbf
        \end{lstlisting}
      \subsection{Nicht mehr ben\"otigte Backups}
        Wenn eine ausreichende Anzahl Backups vorhanden ist, um die geltende
        Retention Policy zu erf\"ullen, werden alle Backups, die nicht mehr
        ben\"otigt werden, als \enquote{obsolete} markiert. Eine Auflistung der
        obsoleten Backups kann mit \languagerman{REPORT OBSOLETE} angezeigt werden.
        \begin{lstlisting}[caption={Backups die nicht mehr ben\"otigt werden anzeigen},label=admin1357,language=rman]
RMAN> REPORT obsolete;

RMAN retention policy will be applied to the command
RMAN retention policy is set to redundancy 1
Report of obsolete backups and copies
Type                 Key    Completion Time    Filename/Handle
-------------------- ------ ------------------ --------------------
Backup Set           2      24-OCT-13
  Backup Piece       2      24-OCT-13          /u05/fast_recovery_area/ORCL/b...
Backup Set           4      24-OCT-13
  Backup Piece       4      24-OCT-13          /u05/fast_recovery_area/ORCL/...
        \end{lstlisting}
        Auch hierbei kann wieder eine ver\"anderte Retention Policy unterstellt werden, so wie es auch bei der \languagerman{NEED BACKUP}-Klausel m\"oglich ist.
        \begin{lstlisting}[caption={Backups die nicht mehr ben\"otigt werden anzeigen},label=admin1358,language=rman]
RMAN> REPORT obsolete RECOVERY WINDOW OF 3 DAYS;

Report of obsolete backups and copies
Type                 Key    Completion Time    Filename/Handle
-------------------- ------ ------------------ --------------------
Backup Set           2      24-OCT-13
  Backup Piece       2      24-OCT-13          /u05/fast_recovery_area/ORCL/b...
Backup Set           4      24-OCT-13
  Backup Piece       4      24-OCT-13          /u05/fast_recovery_area/ORCL/a...
        \end{lstlisting}
    \section{Erarbeiten von Backup und Recovery Strategien}
      \subsection{Redundancy Sets}
        Das Set der Dateien, das dazu ben\"otigt wird, um eine Datenbank im Fehlerfall zu Recovern wird als \enquote{Redundancy Set} bezeichnet. Ein solches Set sollte folgende Dateien enthalten:
        \begin{itemize}
          \item Das letzte Backup der Kontrolldatei und aller Datendateien
          \item Alle Archive Logs, die nach dem letzten Vollbackup entstanden sind
          \item Multiplexing Duplikate von Kontrolldateien und Redo Log Dateien
          \item Kopien von Konfigurationsdateien (SPFile, tnsnames.ora, listener.ora)
        \end{itemize}
        Aus Sicherheitsgr\"unden ist es wichtig, das Redundancy Set nicht auf dem gleichen Datentr\"ager aufzubewahrt, wie die Zieldatenbank selbst. Dies wird am Einfachsten dadurch erreicht, das eine Fast Recovery Area auf einem von der Datenbank getrennten Datentr\"ager eingerichtet wird. Unabh\"angig davon sollten folgende Empfehlungen beachtet werden:
        \begin{itemize}
          \item Kontroll- und Redo Log Dateien sollten auf Datenbankebene verteilt werden (Multiplexing).
          \item Wird die Datenbank im Archive Log Modus betrieben, sollten die Archive an unterschiedliche Stellen verteilt werden.
          \item F\"ur die Speicherung der Datendateien sollte Mirroring auf Betriebssystem- oder Hardwareebene genutzt werden.
          \item Es sollte immer mindestens eine Kopie des Redundancy Sets verf\"ugbar sein. Wird das Redundancy Set auf einem Band gespeichert, sollten aufgrund der hohen Ausfallwahrscheinlichkeit eines Bands immer mindestens zwei Kopien des Redundancy Sets vorhanden sein.
        \end{itemize}
        Die Nutzung einer Fast Recovery Area ist nicht  verpflichtend, wird
        jedoch seitens Oracle empfohlen.
      \subsection{Planen einer Backupstrategie}
        Eine Backupstrategie legt fest, welche Teile der Datenbank gesichert werden m\"ussen, welche Tools daf\"ur herangezogen werden und wie die Datenbank m\"oglichst robust konfiguriert werden kann, um Backup und Recovery Operationen zu erleichtern. Die Entwicklung einer Backupstrategie ist auch nicht zu letzt eine Frage des Budgets, da hochwertige Backupsysteme meist sehr kostspielig sind.
        \subsubsection{Archive Log oder Noarchive Log Modus}
          Die Datenbank kann in zwei verschiedenen Modi betrieben werden: Archivelog und Noarchivelog. Der Archivelog Modus erweitert die M\"oglichkeiten des DBA bei einem Recovery wesentlich, hat jedoch auch Nachteile, wie z. B. einen h\"oheren Speicherplatzverbrauch. Im Folgenden werden die Vor- und Nachteile dieser beiden Modi gegen\"ubergestellt. So kann eine Entscheidung getroffen werden, welcher Modus der Richtige f\"ur den Betrieb der Datenbank ist.

          Die Datenbank im Noarchivelog Modus zu betreiben, hat folgende Auswirkungen:
          \begin{itemize}
            \item Es k\"onnen keine Online-Backups gemacht werden. Die Datenbank muss vor einem Backup heruntergefahren werden.
            \item Erweiterte Recovery Techniken, wie z. B. Point-In-Time Recovery oder Flashback Database k\"onnen nicht genutzt werden.
          \end{itemize}
          Im Gegensatz dazu, hat der Betrieb der Datenbank im Archivelog Modus diese Auswirkungen:
          \begin{itemize}
            \item Die gesamte Breite der Recovery Techniken steht dem Administrator zur Verf\"ugung.
            \item Zus\"atzlicher Speicherplatz f\"ur die Speicherung der Archive Logs wird ben\"otigt.
            \item Die Archive Logs m\"ussen verwaltet werden (Bereitstellung von Speicherorten, Sicherung auf Band, usw.)
            \item Durch die Archivierung entsteht ein Performance Overhead.
          \end{itemize}
        \subsubsection{Festlegen der Backupfrequenz}
          Abh\"angig von der H\"aufigkeit, mit der eine Datenbank ge\"andert wird, sollte auch die Backupfrequenz bestimmt werden. Sie richtet sich nach den folgenden Gesichtspunkten:
          \begin{itemize}
            \item Wie h\"aufig werden Schemaobjekte gel\"oscht und neu angelegt?
            \item Wie h\"aufig werden Tabellenzeilen bearbeitet (eingef\"ugt, gel\"oscht, ge\"andert)?
            \item Wie h\"aufig wird die Struktur der Datenbank ge\"andert?
          \end{itemize}
          Von den Antworten auf diese Fragen, den vorhandenen Mitteln und der maximalen Down-Time, welche die Datenbank aufweisen darf, h\"angt ab, wie h\"aufig ein Backup der Datenbank erfolgen muss. Bei strukturellen \"Anderungen der Datenbank sollte grunds\"atzlich immer vorher und nachher ein Backup erfolgen.

          Wenn die \"Anderungsh\"aufigkeit f\"ur bestimmte Objekte sehr hoch ist und im Rest der Datenbank eher niedrig, kann auch ein partielles Backup der Datenbank in Frage kommen, so dass h\"aufig ge\"anderte Objekte auch h\"aufiger gesichert werden, als andere.
        \subsubsection{Wartungsfenster}
          Die Planung einer Backupstrategie h\"angt nicht zuletzt von evtl. vorhanden Wartungsfenstern ab.
          \begin{merke}
            Als Wartungsfenster wird ein Zeitraum bezeichnet, innerhalb dessen die Datenbank zu Wartungszwecken heruntergefahren werden kann.
          \end{merke}
          Da jede Datenbank einen bestimmten Verf\"ugbarkeitsgrad haben muss, also zu bestimmten Zeiten arbeitsbereit sein soll, werden die M\"oglichkeiten f\"ur Wartungsfenster automatisch eingeschr\"ankt.
        \subsubsection{Maximale Down-Time}
          \begin{merke}
            Als Down-Time wird der Zeitraum bezeichnet, in dem die Datenbank aufgrund eines Ausfalls nicht mehr verf\"ugbar ist. Hier gilt es m\"oglichst realistische Sch\"atzungen, evtl. auch basierend auf Erfahrungswerten, zu machen, wie lange ein schnellst m\"ogliches Recovery der Datenbank dauert und wie lange die Datenbank somit nicht verf\"ugbar sein wird. Die max. Down-Time sollte so gering wie m\"oglich gehalten werden.
          \end{merke}

          Da eine Backupstrategie die Basis f\"ur jede Recoverystrategie darstellt, muss die max. Down-Time nicht erst beim Erstellen der Recoverystrategie(n) ber\"ucksichtig werden, sondern bereits hier.
          \abbildung{backupstrategie} zeigt eine m\"ogliche Vorgehensweise bei der Planung einer Backupstrategie.
          \bild{Planung einer Backupstrategie}{backupstrategie}{0.45}
\clearpage
      \subsection{RMAN-Skripte benutzen}
        RMAN stellt die M\"oglichkeit bereit, Skripte f\"ur st\"andig wiederkehrende Aufgaben zu benutzen. Dabei handelt es sich um einfache Textdateien, die auf jeder Betriebssystemplattform erstellt und genutzt werden k\"onnen. Aus diesem Grund stellen RMAN-Skripte das ideale Hilfsmittel zur Implementierung einer Backupstrategie dar.
        \subsubsection{RMAN mit Skript-Datei starten}
          \begin{lstlisting}[caption={Aufrund eines RMAN-Skripts},label=admin1359,language=rman]
[oracle@FEA11-119SRV ~]$ rman target / cmdfile backup_sunday_full.cmd
          \end{lstlisting}
          \begin{merke}
            Die Dateiendung \oscommand{.cmd} wurde willk\"urlich gew\"ahlt.
          \end{merke}
          Nach der Ausf\"uhrung aller Kommandos im Skript, wird RMAN automatisch beendet. Um die Ausgaben des RMAN-Skripts sehen zu k\"onnen, kann zus\"atzlich eine Logdatei Angegeben werden.
          \begin{lstlisting}[caption={RMAN-Skript mit Log-Datei benutzen},label=admin1360,language=terminal]
[oracle@FEA11-119SRV ~]$ rman target / cmdfile backup_sunday_full.cmd \
> log backup_sunday_full.log
          \end{lstlisting}
        \subsubsection{Skript-Dateien in RMAN starten}
          RMAN-Skripte k\"onnen auch innerhalb von RMAN aufgerufen werden:
          \begin{lstlisting}[caption={Ein Skript-Datei in RMAN aufrufen},label=admin1361,language=rman]
RMAN> @backup_sunday_full.cmd
        \end{lstlisting}
          Wurden die Kommandos im RMAN-Skript ausgef\"uhrt, zeigt RMAN die Meldung \enquote{**end-of-file**} und wird nicht beendet.
        \subsubsection{Syntax Check}
          RMAN bietet die M\"oglichkeit, einen Syntax-Check an Skript-Dateien durchzuf\"uhren. Dies geschieht mit Hilfe des Parameters \oscommand{CHECKSYNTAX}. Wird RMAN mit diesem Parameter aufgerufen, werden alle Kommandos nur getestet und nicht ausgef\"uhrt. Ist ein Kommando fehlerhaft, wird die RMAN-Fehlermeldung \enquote{RMAN-00558} ausgegeben.
          \begin{lstlisting}[caption={Ein RMAN-Skript auf korrekte Syntax \"uberpr\"ufen},label=admin1362,language=rman]
[oracle@FEA11-119SRV ~]$ rman checksyntax cmdfile backup_sunday_full.cmd

Recovery Manager: Release 11.2.0.1.0 - Production on Sun Oct 27 08:51:42 2013

Copyright (c) 1982, 2009, Oracle and/or its affiliates.  All rights reserved.

RMAN> BACKUP database;
The cmdfile has no syntax errors
Recovery Manager complete.
          \end{lstlisting}
      \subsection{Planen einer Recoverystrategie}
        Die Fehler, die in einer Datenbank entstehen k\"onnen, decken die
        gesamte Skala von Nutzer\-fehlern, \"uber Blockfehler in Datendateien
        bis hin zu Medienfehlern ab. Wie schnell ein Fehler behoben
        werden kann, h\"angt im Wesentlichen von der Recoverystrategie ab.

        Vor der Erstellung einer Recoverystrategie sollten folgende Fragen beantwortet werden:
        \begin{itemize}
          \item Wie soll auf den Ausfall eines gesamten Speichermediums reagiert werden?
          \item Wie kann ein logischer Fehler, der durch eine Anwendung erzeugt wurde, aufgesp\"urt und behoben werden?
            \begin{itemize}
              \item Welche Auswirkungen entstehen dabei auf zwischenzeitlich durchgef\"uhrte Updates?
              \item Wie kann ein erneutes Auftreten des gleichen Fehlers verhindert werden?
            \end{itemize}
          \item Wie soll reagiert werden, wenn das Alert.log File einen oder mehrere defekte Datenbl\"ocke in den Datendateien anzeigt und wie k\"onnen diese Bl\"ocke repariert werden?
          \item Welches Disaster Recovery ist im Falle der Zerst\"orung des kompletten Datenbankservers notwendig und wie lange w\"urde ein solches Recovery dauern?
          \item Kann eine andere Person, die Datenbank Recovern, falls der DBA abwesend ist?
        \end{itemize}
        Wurden Antworten auf diese Fragen gefunden, kann gekl\"art werden, welche Techniken beim Recovery zum Einsatz kommen sollen. Hier stehen zur Verf\"ugung:
        \begin{itemize}
          \item User Managed Backup and Recovery
          \item Der Recovery Manager
          \item Oracle Flashback Database
          \item Block Media Recovery
        \end{itemize}
        Dies sind jedoch nur einige wenige \"Uberlegungen, die bei der Planung einer Recoverystrategie gemacht werden sollten. Zus\"atzlich kommt es immer auf Hardware, Personal und Budget an.
    \section{Ein Backupszenario}
      \label{backupszenarios}
      \subsection{Backupstrategie bei hohem \"Anderungsvolumen}
        In diesem Szenario wird eine Datenbank beschrieben, f\"ur die eine w\"ochentliche Sicherung eingerichtet werden soll. Das \"Anderungsvolumen dieser Datenbank, innerhalb einer Woche, ist hoch. Da in einem solchen Szenario selbst inkrementelle Backups sehr gro\ss\ werden k\"onnen, wird hier ein w\"ochentliches Full-Backup mit zus\"atzlicher Sicherung der Archive Logs und einer Zeitfensterbasierten Retention Policy empfohlen.

        Die Strategie setzt sich aus den folgenden einzelnen Elementen zusammen:
        \begin{itemize}
          \item Die Redo Logs werden mit Hilfe der Fast Recovery Area gesichert.
          \item W\"ochentlich wird ein Full-Backup der Datenbank in die Fast Recovery Area gesichert.
          \item T\"aglich werden alle Backupdateien (einschlie\ss{}lich der Archive Logs), die sich in der Fast Recovery Area befinden auf ein SBT-Ger\"at gesichert. Obsolete Backups werden vom SBT gel\"oscht.
        \end{itemize}
        Wird f\"ur die Backupstrategie eine Recovery Window basierte Retention Policy verwendet, ist sichergestellt, dass alle Backups, solange sie ben\"otigt werden, vorr\"atig sind.
        \subsubsection{Vorbereitungen}
          Vorbereitend f\"ur diese Backupstrategie muss eine Fast Recovery Area mit der richtigen Gr\"o\ss{}e erstellt werden. Diese kann nach folgender Formel berechnet werden:

          \begin{small}
            Speicherplatzbedarf = (Gr\"o\ss{}e eines Full-Backups) + (Gr\"o\ss{}e aller Archive Logs f\"ur Y+1 Tage)
          \end{small}

          Die Variable $Y$ steht in dieser Formel f\"ur die Anzahl der Tage, die zwischen den Sicherungen der Fast Recovery Area vergehen. In diesem Szenario erfolgt eine t\"agliche Sicherung, d. h. es gilt $Y=1$. Wichtig ist ebenfalls, dass die Fast Recovery Area als Speicherort f\"ur die Redo Logs angegeben wird.

          Zur Umsetzung dieser Strategie werden zwei verschiedene RMAN-Skripte benutzt. Eines f\"ur das w\"ochentliche Full-Backup und eines f\"ur die t\"aglichen Backups der Fast Recovery Area. Das w\"ochentliche Skript wird jeden Sonntag ausgef\"uhrt. Das t\"agliche Skript wird sechs Tage die Woche ausgef\"uhrt.
\clearpage
        \subsubsection{Das w\"ochentliche Skript}
          \begin{lstlisting}[caption={Das w\"ochentliche Skript},label=admin1363,language=rman]

RMAN> RUN {
2>      ALLOCATE CHANNEL c1 DEVICE TYPE disk;
3>      ALLOCATE CHANNEL c2 DEVICE TYPE sbt
4>      PARMS 'SBT_LIBRARY=oracle.disksbt,ENV=(BACKUP_DIR=/u04)';
5>
6>      BACKUP AS COMPRESSED BACKUPSET database
7>      CHANNEL c1;
8>
9>      BACKUP recovery area
10>      CHANNEL c2;
11>
12>     DELETE obsolete DEVICE TYPE sbt;
13>   }
          \end{lstlisting}
        \subsubsection{Das t\"agliche Skript}
          \begin{lstlisting}[caption={Das t\"agliche Skript},label=admin1364,language=rman]
RMAN> RUN {
3>      ALLOCATE CHANNEL c2 DEVICE TYPE sbt
4>      PARMS 'SBT_LIBRARY=oracle.disksbt,ENV=(BACKUP_DIR=/u04)';
5>
5>      BACKUP recovery area
6>      CHANNEL c2;
7>
8>      DELETE obsolete DEVICE TYPE sbt;
9>    }
          \end{lstlisting}
          Da t\"aglich die gesamte Fast Recovery Area gesichert wird, kann der RMAN, Dateien aus der Recovery Area l\"oschen, sobald dies notwendig ist. Zwischen den einzelnen Backups ist es schwierig zu sagen, welche Dateien noch in der Fast Recovery Area sind.
    \section{Informationen}
      \subsection{Verzeichnis der relevanten Data Dictionary Views}
        \begin{literaturinternet}
          \item \cite{REFRN30022}
          \item \cite{REFRN30030}
        \end{literaturinternet}
\clearpage

    \section{\"Ubungen - Backups mit dem RMAN}
  \begin{merke}
    Zur Vorbereitung auf diese \"Ubung l\"oschen Sie bitte alle vorhandenen Backups!
  \end{merke}
  \begin{enumerate}
        \item F\"uhren Sie ein Recovery bei Verlust einer Redo Log Datei durch!
      \begin{enumerate}
        \item Starten Sie das Skript \oscommand{lab\_delete\_redolog\_member.sql}! Es l\"oscht einen beliebigen Member einer Ihrer Redo Log Gruppen.
          \begin{lstlisting}[language=terminal]
SQL> @/home/oracle/labs/lab_delete_redolog_member.sql
          \end{lstlisting}
        \item Die Datenbank l\"auft weiterhin normal und es liegen keinerlei Probleme vor. \"Offnen Sie die Alert Log Datei, um herauszufinden welcher Redo Log Member gel\"oscht wurde!
        \item F\"uhren Sie geeignete Ma\ss{}nahmen zur Problembehebung durch!
      \end{enumerate}

        \item Konfigurieren Sie das Auditing f\"ur die Tabelle \identifier{bank.mitarbeiter} so, dass erfolgreiche \languageorasql{INSERT}- und \languageorasql{UPDATE}-Statements auditiert werden! Es soll jeweils nur ein Auditeintrag pro Session f\"ur jedes  \languageorasql{INSERT}/\languageorasql{UPDATE} gemacht werden.

        \item Wechseln Sie den Undo-Tablespace zur\"uck zum Original Undo-Tablespace und l\"oschen Sie \identifier{undotbs02} und \identifier{undotbs03}.

        \item Pr\"ufen Sie in welchem Modus der Result Cache l\"auft (Manual oder Force)

        \item Pr\"ufen Sie den Audittrail auf neue Informationen! Welche T\"atigkeiten wurden an der auditierten Tabelle ausgef\"uhrt?


      \rule{0.94\textwidth}{0.5pt}

        \item Machen Sie alle Auditingeinstellungen r\"uckg\"angig und l\"oschen Sie den Inhalt des Auditingtrails!

        \item Schalten Sie das Auditing f\"ur alle erfolgreichen Anmeldeversuche ein! Lassen Sie diese Auditingeintr\"age im \textit{DB, Extended} Auditingtrail speichern!


      \rule{0.94\textwidth}{0.5pt}

      \rule{0.94\textwidth}{0.5pt}

      \rule{0.94\textwidth}{0.5pt}

      \rule{0.94\textwidth}{0.5pt}
\clearpage
        \item F\"uhren Sie ein Recovery mit einem Backup Controlfile durch.
      \begin{enumerate}
        \item Starten Sie das Skript \oscommand{lab\_delete\_all\_controlfiles.sql}. Es wird alle Kontrolldateien l\"oschen.
          \begin{lstlisting}[language=terminal]
SQL> @/home/oracle/labs/lab_delete_all_controlfiles.sql
          \end{lstlisting}
        \item Ergreifen Sie geeignete Ma\ss{}nahmen, um die Datenbank wieder lauff\"ahig zu machen.
      \end{enumerate}

        \item Legen Sie den tempor\"aren smallfile Tablespace \identifier{temp\_ts} an. Sein Tempfile  soll 500 M gro\ss{} sein, auf bis zu 4 G anwachsen k\"onnen, \oscommand{temp\_ts01.dbf} hei\ss{}en und auf Laufwerk \oscommand{/u02} liegen. Benutzen Sie 1 M gro\ss{}e Uniform-Sized Extents.

        \item L\"oschen Sie das Backup Set des \identifier{bank}-Tablespaces von der Festplatte, so dass es nur noch auf SBT verf\"ugbar ist!

        \item Der Tablespace \identifier{bank} liegt derzeit in unverschl\"usselter Form vor. Dies muss zuk\"unftig anders sein. Bereiten Sie einen Tablespace \identifier{bank\_encrypted} vor, der mittels \identifier{AES256} verschl\"usselung gesichert ist und der die gleichen Dimensionen hat, wie der Originaltablespace \identifier{bank}! Rufen Sie alle notwendigen Informationen \"uber den Tablespace \identifier{bank} aus dem Data Dictionary ab! Welche Views helfen Ihnen dabei?

       \item F\"uhren Sie mit Hilfe von RMAN ein Block-Media-Recovery durch, um die be\-sch\"a\-dig\-ten Bl\"ocke zu reparieren.

          \item \"Andern Sie die Sprache Ihrer Session auf Deutsch.

        \item L\"oschen Sie das Nutzerprofil \identifier{p\_clerk} in einem Arbeitsschritt!


      \rule{0.94\textwidth}{0.5pt}

        \item L\"oschen Sie den Tablespace fullts mit all seinen Datendateien.

      \rule{0.94\textwidth}{0.5pt}

        \item Wie gro\ss{} ist die Auslastung Ihrer FRA durch Backup Pieces (prozentual)?

      \rule{0.94\textwidth}{0.5pt}

        \item F\"uhren Sie zum Abschluss das Skript \oscommand{labs/lab\_dbadmin05\_cleardb.sql} aus, um die Datenbank aufzur\"aumen. Es werden alle Tablespaces gel\"oscht, die im Rahmen dieser \"Ubung angelegt werden sollten.

        \item Erstellen Sie RMAN-Skripte, um die im Folgenden beschriebene Backup Strategie zu implementieren! Benutzen Sie f\"ur die Umsetzung der Skripte, RUN-Bl\"ocke und die manuelle Kanalanforderung!
      \begin{itemize}
        \item Jeden Sonntag muss ein komprimiertes Level 0 Backup der Datenbank erstellt werden!
        \item Von Montag bis Freitag werden im Drei-Stunden-Rhythmus Backups der Archive Logs erzeugt. Aus Sicherheitsgr\"unden m\"ussen diese Backups, an den Speicherorten \oscommand{/u02/backup} und \oscommand{/u03/backup} dupliziert werden! Alle Kopien der gesicherten Archive Logs sollen automatisch aus der FRA entfernt werden!
        \item Jeden Montag, Dienstag, Donnerstag und Freitag werden Level 1 Backups (inkrementell) der Datenbank erstellt und in der FRA gespeichert!
        \item Jeden Mittwoch muss ein kumulatives Level 1 Backup der Datenbank erzeugt werden!
        \item Jeden Samstag sind der komplette Inhalt der FRA und die Backups der Archive Logs auf das SBT-Ger\"at zu verschieben! Es ist sicherzustellen, dass im Anschluss daran, alle obsoleten Backups gel\"oscht werden!
        \item F\"ur diese Backup Strategie ist ein Recovery Window von 7 Tagen notwendig.
        \item Aktivieren Sie das Block Change Tracking, um die Dauer der inkrementellen Backups zu beschleunigen!
      \end{itemize}

        \item F\"uhren Sie das Skript labs/lab\_delete\_backups.sql aus. Dieses Skript wird eine will\-k\"ur\-liche Anzahl Ihrer Backups l\"oschen.

          \item Pr\"ufen Sie, ob sich Ihre Datenbank im Archive-Log-Modus befindet. Falls nicht wechseln Sie in den Archive-Log-Modus.

        \item L\"oschen Sie die Eintr\"age f\"ur alle nicht mehr verf\"ugbaren Backups.

    \input{uebungen/dbadmin_17/item_v}
  \end{enumerate}
\clearpage

    \input{loesungen/dbadmin_17_backups_mit_dem_rman_loesung}
    \chapter{Recovery mit dem RMAN}
  \label{recoverywithrman}
    \setcounter{page}{1}\kapitelnummer{chapter}
    \minitoc
\newpage
    Obwohl der RMAN das Recovery einer Oracle-Datenbank wesentlich vereinfacht, sind im Vorfeld trotzdem noch einige Planungsarbeiten notwendig. Welche dies sind, ist im Wesentlichen von der Art des Datenverlustes abh\"angig.
    \section{Ein Recovery planen und vorbereiten}
      Die Durchf\"uhrung eines Restore und Recovery Prozesses besteht aus 5 Schritten.
      \begin{enumerate}
        \item Ermitteln welche Datenbankdateien wiederhergestellt werden m\"ussen und welche Backups daf\"ur herangezogen werden k\"onnen. Dies kann auch Archive Logs, dass SPFile und die Kontrolldatei einschlie\ss{}en.
        \item Die Datenbank in den ben\"otigten Zustand versetzen. F\"ur ein vollst\"andiges Restore and Recovery der ganzen Datenbank ist dies meist der MOUNT-Status. M\"ussen nur ein einzelner Tabelspace oder einzelne Datendateien wiederhergestellt werden, gen\"ugt es, den betreffenden Tablespace in den Offline-Status zu versetzen.
        \item Durchf\"uhren der Restore-Phase. Hierbei kann es notwendig sein, die Datenbankdateien an einem neuen Speicherort wiederherzustellen, weil der Alte nicht mehr verf\"ugbar ist. Eventuelle Anpassungen am SPFile d\"urfen dabei nicht vergessen werden.
        \item Durchf\"uhren der Recovery-Phase.
        \item Durchf\"uhren von T\"atigkeiten, die zum weiteren Betrieb der Datenbank notwendig sind, z. B. \"offnen der Datenbank.
      \end{enumerate}
      Nicht jeder Recovery-Prozess ben\"otigt immer alle 5 Schritte. Muss beispielsweise nur das SPFile aus einem Backup wiederhergestellt werden, ist ein Recovery der Datenbank nicht notwendig.
    \section{Pflege- und Erhaltungsma\ss{}nahen}
      \subsection{Datenbankdateien auf Fehler pr\"ufen}
        Mit dem \languagerman{VALIDATE}-Kommando k\"onnen Datenbankdateien vor einem Backup auf Funktionsf\"ahigkeit gepr\"uft werden. Dabei wird getestet, ob die Datenbankdateien existieren, ob sie sich am richtigen Speicherort befinden und ob sie physische bzw. logische Besch\"adigungen aufweisen. RMAN verf\"ahrt dabei genauso wie bei einem Backup, nur wird beim Validieren kein Backup Set erzeugt.
\clearpage
        \begin{lstlisting}[caption={Eine Datendatei validieren},label=admin1400,language=rman]
RMAN> VALIDATE datafile 1;

Starting validate at 29-OCT-13
allocated channel: ORA_DISK_1
channel ORA_DISK_1: SID=25 device type=&DISK&
allocated channel: ORA_DISK_2
channel ORA_DISK_2: SID=145 device type=&DISK&
channel ORA_DISK_1: starting validation of datafile
channel ORA_DISK_1: specifying datafile(s) for validation
input datafile file number=00001 name=/u01/app/oracle/oradata/orcl/system01.dbf
channel ORA_DISK_2: starting validation of datafile
channel ORA_DISK_2: specifying datafile(s) for validation
including current &SPFILE& in backup set
channel ORA_DISK_2: validation complete, elapsed time: 00:00:01
List of Control File and &SPFILE&
===============================
File Type    Status Blocks Failing Blocks Examined
------------ ------ -------------- ---------------
&SPFILE&        OK     0              2
channel ORA_DISK_2: starting validation of datafile
channel ORA_DISK_2: specifying datafile(s) for validation
including current control file for validation
channel ORA_DISK_2: validation complete, elapsed time: 00:00:01
List of Control File and &SPFILE&
===============================
File Type    Status Blocks Failing Blocks Examined
------------ ------ -------------- ---------------
Control File OK     0              594
channel ORA_DISK_1: validation complete, elapsed time: 00:00:48
List of Datafiles
=================
File Status Marked Corrupt Empty Blocks Blocks Examined High &SCN&
---- ------ -------------- ------------ --------------- ----------
1    OK     0              13310        92201           2086674
  File Name: /u01/app/oracle/oradata/orcl/system01.dbf
  Block Type Blocks Failing Blocks Processed
  ---------- -------------- ----------------
  Data       0              59907
  Index      0              12711
  Other      0              6232

Finished validate at 29-OCT-13
        \end{lstlisting}
        Die Syntax des \languagerman{VALIDATE}-Befehls ist der des \languagerman{BACKUP}-Kommandos sehr \"ahnlich. Es k\"onnen Archive Logs, Kontrolldateien, SPFiles, Datendateien, Tablespaces oder auch die gesamte Datenbank gepr\"uft werden.
\clearpage
        \begin{lstlisting}[caption={Eine ganze Datenbank validieren},label=admin1401,language=rman]
RMAN> VALIDATE database;
        \end{lstlisting}
        \begin{lstlisting}[caption={Validieren der Archive Logs},label=admin1402,language=rman]
RMAN> VALIDATE archivelog all;
Starting validate at 29-OCT-13
released channel: ORA_SBT_TAPE_1
using channel ORA_DISK_1
using channel ORA_DISK_2
channel ORA_DISK_1: starting validation of archived log
channel ORA_DISK_1: specifying archived log(s) for validation
input archived log thread=1 sequence=56 RECID=89 STAMP=830101540
channel ORA_DISK_1: validation complete, elapsed time: 00:00:01
List of Archived Logs
=====================
Thrd Seq     Status Blocks Failing Blocks Examined Name
---- ------- ------ -------------- --------------- ---------------
1    56      OK     0              130             /u02/backup/archive_logs/1...
Finished validate at 29-OCT-13
        \end{lstlisting}
        Wird w\"ahrend der Validierung ein Fehler in einer Datenbankdatei entdeckt, wird die View \identifier{v\$database\_block\_corruption} mit Informationen gef\"ullt. Diese Fehler k\"onnen dann evtl. durch ein Block-Media Recovery behoben werden.
      \subsection{Funktionsf\"ahigkeit eines Backup Sets ermitteln}
        Das Kommando \languagerman{VALIDATE BACKUPSET} erm\"oglicht die Validierung eines Backup Sets. Dabei wird das Backup Set auf logische und physikalische Fehler \"uberpr\"uft.
        \begin{lstlisting}[caption={Ein Backup Set validieren},label=admin1403,language=rman]
RMAN> VALIDATE backupset 4711;

Starting validate at 29-OCT-13
using channel ORA_DISK_1
using channel ORA_DISK_2
using channel ORA_SBT_TAPE_1
channel ORA_SBT_TAPE_1: starting validation of datafile backup set
channel ORA_SBT_TAPE_1: reading from backup piece 06on9l7u_1_1
channel ORA_SBT_TAPE_1: piece handle=06on9l7u_1_1 tag=TAG20131025T112446
channel ORA_SBT_TAPE_1: restored backup piece 1
channel ORA_SBT_TAPE_1: validation complete, elapsed time: 00:00:04
Finished validate at 29-OCT-13
        \end{lstlisting}
      \subsection{Wiederherstellbarkeit einer Datenbankdatei pr\"ufen}
        Neben den M\"oglichkeiten Datenbankdateien und Backups auf Fehlerfreiheit zu pr\"ufen, kann noch ein weiterer Pr\"ufschritt unternommen werden. Mit \languagerman{RESTORE VALIDATE} kann festgestellt werden, ob eine Datenbankdatei, mit Hilfe der bestehenden Backups wiederhergestellt werden kann.
        \begin{lstlisting}[caption={Kann der Tablespace \identifier{bank} wiederhergestellt werden?},label=admin1404,language=rman]
RMAN> RESTORE VALIDATE tablespace bank;

Starting restore at 29-OCT-13
using channel ORA_DISK_1
using channel ORA_DISK_2
allocated channel: ORA_SBT_TAPE_1
channel ORA_SBT_TAPE_1: SID=21 device type=SBT_TAPE
channel ORA_SBT_TAPE_1: WARNING: Oracle Test Disk API

channel ORA_DISK_1: starting validation of datafile backup set
channel ORA_DISK_2: starting validation of datafile backup set
channel ORA_DISK_1: reading from backup piece /u05/fast_recovery_area/ORCL/...
channel ORA_DISK_2: reading from backup piece /u05/fast_recovery_area/ORCL/...
channel ORA_DISK_2: piece handle=/u05/fast_recovery_area/ORCL/...
channel ORA_DISK_2: restored backup piece 1
channel ORA_DISK_2: validation complete, elapsed time: 00:00:03
channel ORA_DISK_1: piece handle=/u05/fast_recovery_area/ORCL/backupset/...
channel ORA_DISK_1: restored backup piece 1
channel ORA_DISK_1: validation complete, elapsed time: 00:00:15
Finished restore at 29-OCT-13
        \end{lstlisting}
        In \beispiel{admin1404} wird der Restore-Prozess f\"ur den Tablespace bank validiert. Dabei wird gepr\"uft, ob alle notwendigen Backup Sets, Image Copies und Archive Logs vorhanden und funktionsf\"ahig sind.
      \subsection{Der Paramter DB\_Ultra\_Safe}
        Der Initialisierungsparameter \parameter{db\_ultra\_safe} wurde mit Oracle 11g neu eingef\"uhrt. Er soll helfen, Besch\"adigungen an Datenbl\"ocken zu vermeiden. Er selbst stellt keine neuen Mechanismen zur Verf\"ugung, sondern beeinflusst drei andere Initialisierungsparameter, die teilweise bereits seit Oracle 8i vorhanden sind. Dies sind:
        \begin{itemize}
          \item \parameter{db\_block\_checking}
          \item \parameter{db\_lost\_write\_protect}
          \item \parameter{db\_block\_checksum}
        \end{itemize}
        \subsubsection{DB\_Block\_Checking}
          Der Parameter \parameter{db\_block\_checking} legt fest, ob Oraclebl\"ocke einer semantischen Pr\"ufung unterzogen werden. Unter einer semantischen Pr\"ufung versteht man den Abgleich der Blockstruktur mit seinem Inhalt. Wenn beispielsweise eine Spalte den Wert \enquote{Florian Weidinger} enth\"alt, aber den Datentyp \identifier{DATE} aufweist, muss dies ein Fehler sein. Ein anderes Beispiel sind Spalten des Typs \identifier{VARCHAR2}. Diese k\"onnen in Oracle maximal 4.000 Byte lang sein. Taucht pl\"otzlich eine Spalte auf, die gr\"o\ss{}er als 4.000 Byte ist, so muss auch dies ein Fehler sein. Diese Art von Fehlern wird zumeist durch fehlerhafte Speichermedien hervorgerufen.

          \parameter{db\_block\_checking} kennt folgende Werte:
          \begin{itemize}
            \item \textbf{OFF}: Das Blockchecking wird lediglich im \identifier{system}-Tablespace durchgef\"uhrt.
            \item \textbf{FALSE}: Dieser Wert hat die gleiche Bedeutung wie \enquote{OFF} und wird nur aus Kompatibilit\"atsgr\"unden bereitgestellt.
            \item \textbf{LOW}: Es findet eine semantische \"Uberpr\"ufung der Blockheader, in allen Tablespaces statt.
            \item \textbf{MEDIUM}: Es findet eine vollst\"andige semantische Pr\"ufung der Oraclebl\"ocke, in allen Tablespaces statt. Ausgenommen sind Indexstrukturen.
            \item \textbf{FULL}: Es findet eine vollst\"andige semantische Pr\"ufung der Oraclebl\"ocke, in allen Tablespaces statt. Auch Indexstrukturen werden gepr\"uft.
            \item \textbf{TRUE}: Dieser Wert hat die gleiche Bedeutung wie \enquote{FULL} und wird nur aus Kompatibilit\"atsgr\"unden bereitgestellt.
          \end{itemize}
        \subsubsection{DB\_Lost\_Write\_Protect}
          Dieser Parameter ist nur in einer Oracle Data Guard Umgebung sinnvoll und hilft dort, Datenverlust durch Fehlfunktionen von Datentr\"agern zu vermeiden. Er kennt die Werte \enquote{NONE}, \enquote{TYPICAL} und \enquote{FULL}.
        \subsubsection{DB\_Block\_Checksum}
          Mit Hilfe von \parameter{db\_block\_checksum} kann die Datenbank dazu veranlasst werden, jedem Block, der in eine Datendatei geschrieben wird, eine Checksumme zu geben. Mit Hilfe dieser Checksumme kann beim Lesen des Blockes sichergestellt werden, dass der Inhalt gelesen wird, der vorher geschrieben wurde (dass der Block unver\"andert geblieben ist).
\clearpage
          Anders als mit \parameter{db\_block\_checking} kann mit diesem Parameter nicht verhindert werden, dass ein fehlerhafter Block gelesen wird. Wird ein Block beim Schreiben auf den Datentr\"ager besch\"adigt, wird von diesem besch\"adigten Block eine Checksumme gebildet. Beim erneuten Lesen des Blockes wird nur die Checksumme gepr\"uft und es wird festgestellt, dass der Block unver\"andert (fehlerhaft) geblieben ist.

          \parameter{db\_block\_checksum} kennt folgende Werte:
          \begin{itemize}
            \item \textbf{OFF}: Nur im \identifier{system}-Tablespace werden Blockchecksummen gebildet.
            \item \textbf{FALSE}: Dieser Wert hat die gleiche Bedeutung wie \enquote{OFF} und wird nur aus Kompatibilit\"atsgr\"unden bereitgestellt.
            \item \textbf{TYPICAL}: In allen Tablespaces werden Blockchecksummen gebildet und vor jedem Lesevorgang verglichen.
            \item \textbf{FULL}: Wie \enquote{TYPICAL}, aber zus\"atzlich wird vor jedem \languageorasql{UPDATE} oder \languageorasql{DELETE}, dass auf einen Block angewandt wird, die Checksumme \"uberpr\"uft und direkt nach der \"Anderung neu berechnet, nicht erst, wenn der Block auf den Datentr\"ager geschrieben wird.
            \item \textbf{TRUE}: Dieser Wert hat die gleiche Bedeutung wie \enquote{TYPICAL} und wird nur aus Kompatibilit\"atsgr\"unden bereitgestellt.
          \end{itemize}
          \bild{Der Unter\-schied zwischen TYPICAL und FULL}{db_block_checksum_typical_and_full}{1.3}
        \subsubsection{Werte f\"ur DB\_Ultra\_Safe}
          Der Parameter \parameter{db\_ultra\_safe} kann folgende Werte annehmen:
          \begin{itemize}
            \item \textbf{OFF}: Es gelten die Einstellungen der Initialisierungsparameter \parameter{db\_block\_checking}, \parameter{db\_lost\_write\_protect} und \parameter{db\_block\_checksum}.
            \item \textbf{DATA\_ONLY}: Die Einstellung der Parameter werden ver\"andert:
              \begin{itemize}
                \item \parameter{db\_block\_checking}: MEDIUM
                \item \parameter{db\_lost\_write\_protect}: TYPICAL
                \item \parameter{db\_block\_checksum}: FULL
              \end{itemize}
            \item \textbf{DATA\_AND\_INDEX}: Die Einstellung der Parameter werden ver\"andert:
              \begin{itemize}
                \item \parameter{db\_block\_checking}: FULL
                \item \parameter{db\_lost\_write\_protect}: TYPICAL
                \item \parameter{db\_block\_checksum}: FULL
              \end{itemize}
          \end{itemize}
          Es wird von Seiten Oracle empfohlen, \parameter{db\_ultra\_safe} auf den Wert \enquote{DATA\_ONLY} oder h\"oher einzustellen. Da dies kein dynamischer Parameter ist, muss ein Neustart der Datenbank erfolgen.
    \section{Recovery unkritischer Verluste}
      Bei unkritischen Verlusten handelt es sich um Sch\"aden, welche die Datenbank nicht unmittelbar zum Stillstand bringen. Die Datenbank kann trotz Besch\"adigung weiterarbeiten, m\"oglicherweise auch nur f\"ur begrenzte Zeit. Beispielsweise wird der Verlust eines SPFiles die Datenbank nicht beeintr\"achtigen, solange kein Neustart erfolgen soll. Auch eine Passwortdatei ist unkritisch f\"ur den Betrieb der Datenbank. Lediglich die Anmeldung der DBAs wird dadurch beeintr\"achtigt.
      \subsection{Verlust eines SPFile beheben (mit Recovery Katalog)}
        \subsubsection{Wiederherstellung aus einem Controlfile Autobackup}
          \begin{enumerate}
            \item Mit dem RMAN an der Zieldatenbank und am Recovery Katalog anmelden.
              \begin{lstlisting}[caption={An der Zieldatenbank und am Recovery Katalog anmelden},label=admin1405,language=rman]
[oracle@FEA11-119SRV ~]$ rman target / catalog catowner/catpass@CATDB
              \end{lstlisting}
            \item Wurde die Zieldatenbank heruntergefahren, muss erst eine Instanz erzeugt werden. Der RMAN benutzt Standardeinstellungen, um eine Minimalinstanz zuerstellen. Diese kann f\"ur das weitere Recovery genutzt werden.
              \begin{lstlisting}[caption={Zieldatenbank im RMAN in den NOMOUNT-Status bringen},label=admin1406,language=rman,alsolanguage=sqlplus]
RMAN> startup nomount
              \end{lstlisting}
              \begin{merke}
                Dieser Schritt ist in SQL*Plus nicht m\"oglich!
              \end{merke}
            \item Wiederherstellen des SPFiles aus dem Controlfile Autobackup. Dieser Schritt unterscheidet sich, je nachdem, ob die Instanz in der NOMOUNT-Phase oder ge\"offnet ist.
              \begin{itemize}
                \item \textbf{NOMOUNT-Phase}
                  \begin{lstlisting}[caption={Wiederherstellen des SPFiles aus dem Controlfile Autobackup},label=admin1407,language=rman]
RMAN> RESTORE spfile FROM AUTOBACKUP;
                  \end{lstlisting}
                \item \textbf{MOUNT-Phase oder ge\"offnete Instanz}

                Bei ge\"offneter oder gemounteter Instanz muss im Anschluss an diesen Arbeits\-schritt, dass wiederhergestellte SPFile in das \oscommand{\$ORACLE\_HOME/dbs}-Verzeichnis verschoben werden. Dies ist notwendig, da Oracle das (vermeindlich noch vorhandene) SPFile mit einer Schreibsperre, im Dateisystem belegt.
								\begin{lstlisting}[caption={Wiederherstellen des SPFiles aus dem Controlfile Autobackup},label=admin1408,language=rman]
RMAN> RESTORE spfile
2>    TO '/u01/app/oracle/product/11.2.0/ORCL/spfileorcl.ora'
3>    FROM AUTOBACKUP;
                  \end{lstlisting}
              \end{itemize}
            \item Durchstarten der Instanz
              \begin{lstlisting}[caption={Neustart der Instanz},label=admin1409,language=rman,alsolanguage=sqlplus]
RMAN> startup force
              \end{lstlisting}
              Nach dem Neustart werden die Initialisierungsparameter gem\"a\ss{} dem wiederhergestellten SPFile gesetzt und m\"ussen auf ihre Aktualit\"at \"uberpr\"uft werden.
          \end{enumerate}
        \subsubsection{Wiederherstellung aus einem Backup Set}
          \begin{enumerate}
            \item Mit dem RMAN an der Zieldatenbank und am Recovery Katalog anmelden.
              \begin{lstlisting}[caption={An der Zieldatenbank und am Recovery Katalog anmelden},label=admin1410,language=rman]
[oracle@FEA11-119SRV ~]$ rman target / catalog catowner/catpass@CATDB
              \end{lstlisting}
            \item Wurde die Zieldatenbank heruntergefahren, muss erst eine Instanz erzeugt werden. Der RMAN benutzt Standardeinstellungen, um eine Minimalinstanz zuerstellen. Diese kann f\"ur das weitere Recovery genutzt werden.
              \begin{lstlisting}[caption={Zieldatenbank im RMAN in den NOMOUNT-Status bringen},label=admin1411,language=rman,alsolanguage=sqlplus]
RMAN> startup nomount
              \end{lstlisting}
              \begin{merke}
                Dieser Schritt ist in SQL*Plus nicht m\"oglich!
              \end{merke}
            \item Wiederherstellen des SPFiles. Dieser Schritt unterscheidet sich, je nachdem, ob die Instanz in der NOMOUNT-Phase oder ge\"offnet ist.
              \begin{itemize}
                \item \textbf{NOMOUNT-Phase}
                  \begin{lstlisting}[caption={Wiederherstellen des SPFiles in der NOMOUNT-Phase},label=admin1412,language=rman]
RMAN> RESTORE spfile;
                  \end{lstlisting}
                \item \textbf{MOUNT-Phase oder ge\"offnete Instanz}

                Bei ge\"offneter oder gemounteter Instanz muss im Anschluss an diesen Arbeitsschritt das wiederhergestellte SPFile in das \oscommand{\$ORACLE\_HOME/dbs}-Verzeichnis verschoben werden. Dies ist notwendig, da Oracle das (vermeindlich noch vorhandene) SPFile mit einer Schreibsperre, im Dateisystem belegt.
								\begin{lstlisting}[caption={Wiederherstellen des SPFiles in der MOUNT-Phase},label=admin1413,language=rman]
RMAN> RESTORE spfile
2>    TO '/u01/app/oracle/product/11.2.0/spfileorcl.ora';
                  \end{lstlisting}
              \end{itemize}
            \item Durchstarten der Instanz
              \begin{lstlisting}[caption={Neustart der Instanz},label=admin1414,language=rman,alsolanguage=sqlplus]
RMAN> startup force
              \end{lstlisting}
              Nach dem Neustart werden die Initialisierungsparameter gem\"a\ss{} dem wiederhergestellten SPFile gesetzt und m\"ussen auf ihre Aktualit\"at \"uberpr\"uft werden.
          \end{enumerate}
\clearpage
      \subsection{Verlust eines SPFiles beheben (ohne Recovery Katalog)}
        \subsubsection{Die Datenbank-ID (DBID)}
          Jede Oracle-Datenbank wird anhand einer zehnstelligen ID, der Datenbank-ID (DBID) identifiziert. Diese ist in der Kontrolldatei gespeichert und wird vom RMAN ben\"otigt, damit dieser erkennen kann, welche Backup Sets zu welcher Datenbank geh\"oren. Ist die Instanz nicht gemountet oder ge\"offnet, kann der RMAN die DBID nicht selbstst\"andig ermitteln.

          Es gibt drei M\"oglichkeiten, an die DBID zu gelangen:
            \begin{enumerate}
              \item Abfragen der View \identifier{v\$database}
                \begin{lstlisting}[caption={Abfragen von \identifier{v\$database}},label=admin1415,language=oracle_sql]
SQL> SELECT dbid
  2  FROM   v$database;

      DBID
----------
1351916467
                \end{lstlisting}
              \item Starten des RMAN: Beim Starten des RMAN wird die DBID der Zieldatenbank rechts unten in der Startmeldung angezeigt, sofern die Instanz hochgefahren ist.
                \begin{lstlisting}[caption={Startmeldung des RMAN},label=admin1416,language=terminal]
Recovery Manager: Release 11.2.0.1.0 - Production &on& Tue Oct 29 17:34:34 2013

Copyright (c) 1982, 2009, Oracle and/or its affiliates.  All rights reserved.

connected to target database: ORCL &\textcolor{red}{(DBID=1351916467)}&
                \end{lstlisting}
              \item Durchsuchen eines unkomprimierten Backups, welches die Datendatei Nummer 1 beinhaltet.
                \begin{lstlisting}[caption={Die DBID in einem Backup Set suchen},label=admin1417,language=terminal]
[oracle@FEA11-119SRV ~]$ strings /u02/backup/ORCL_29-10-2013_4aklgp2d.bkp \
> | grep &MAXVALUE&,

&\textcolor{red}{1351916467}, MAXVALUE,&
                \end{lstlisting}
                Das Linux-Kommando \oscommand{strings} durchsucht eine Bin\"ardatei nach \enquote{lesbaren} Zeichenketten. In einem unkomprimierten und unverschl\"usselten Backup Set kann so die DBID sichtbar gemacht werden.
              \end{enumerate}
        \subsubsection{Wiederherstellung aus einem Controlfile Autobackup}
          \begin{enumerate}
            \item Mit dem RMAN an der Zieldatenbank anmelden.
              \begin{lstlisting}[caption={An der Zieldatenbank anmelden},label=admin1418,language=rman]
[oracle@FEA11-119SRV ~]$ rman target sys/oracle
              \end{lstlisting}
            \item Setzen der DBID
              \begin{lstlisting}[caption={Setzen der DBID},label=admin1419,language=rman]
RMAN> SET DBID 1351916467;
              \end{lstlisting}
            \item Wurde die Zieldatenbank heruntergefahren, muss erst eine Instanz erzeugt werden. Der RMAN benutzt Standardeinstellungen, um eine Minimalinstanz zuerstellen. Diese kann f\"ur das weitere Recovery genutzt werden.
              \begin{lstlisting}[caption={Zieldatenbank im RMAN in den NOMOUNT-Status bringen},label=admin1420,language=rman,alsolanguage=sqlplus]
RMAN> startup nomount
              \end{lstlisting}
              \begin{merke}
                Dieser Schritt ist in SQL*Plus nicht m\"oglich!
              \end{merke}
            \item Wiederherstellen des SPFiles aus dem Controlfile Autobackup. Dieser Schritt unterscheidet sich, je nachdem, ob die Instanz in der NOMOUNT-Phase oder ge\"offnet ist.
              \begin{itemize}
                \item \textbf{NOMOUNT-Phase}
                  \begin{lstlisting}[caption={Wiederherstellen des SPFiles aus dem Controlfile Autobackup},label=admin1421,language=rman]
RMAN> RESTORE spfile FROM AUTOBACKUP
2>    RECOVERY AREA '/u05/fast_recovery_area'
3>    DB_NAME       'orcl';
                  \end{lstlisting}
                  \begin{merke}
                    Wenn die Instanz nicht gemountet oder ge\"offnet ist, ist es dem RMAN unm\"oglich, den Speicherort der Controlfile Autobackups zu ermitteln. Mit Hilfe der beiden Parameter \languagerman{RECOVERY AREA} und \languagerman{DB_NAME} wird der Speicherort, die Fast Recovery Area, dem RMAN mitgeteilt.
                  \end{merke}
                \item \textbf{MOUNT-Phase oder ge\"offnete Instanz}
                  \begin{lstlisting}[caption={Wiederherstellen des SPFiles aus dem Controlfile Autobackup},label=admin1422,language=rman]
RMAN> RESTORE spfile
2>    TO '/u01/app/oracle/product/11.2.0/ORCL/spfileorcl.ora'
3>    FROM AUTOBACKUP;
                  \end{lstlisting}
              \end{itemize}
              Bei ge\"offneter oder gemounteter Instanz muss im Anschluss an diesen Arbeitsschritt das wiederhergestellte SPFile in das \oscommand{\$ORACLE\_HOME/dbs}-Verzeichnis verschoben werden. Dies ist notwendig, da Oracle das (vermeindlich noch vorhandene) SPFile mit einer Schreibsperre, im Dateisystem belegt.
            \item Durchstarten der Instanz
              \begin{lstlisting}[caption={Neustart der Instanz},label=admin1423,language=rman,alsolanguage=sqlplus]
RMAN> startup force
              \end{lstlisting}
              Nach dem Neustart werden die Initialisierungsparameter gem\"a\ss{} dem wiederhergestellten SPFile gesetzt und m\"ussen auf ihre Aktualit\"at \"uberpr\"uft werden.
            \end{enumerate}
        \subsubsection{Wiederherstellung aus einem Backup Set}
          \begin{enumerate}
            \item Mit dem RMAN an der Zieldatenbank anmelden und diee DBID setzen.
              \begin{lstlisting}[caption={An der Zieldatenbank anmelden und die DBID setzen},label=admin1424,language=rman]
[oracle@FEA11-119SRV ~]$ rman target /

RMAN> SET DBID 1351916467;
                  \end{lstlisting}
            \item Wurde die Zieldatenbank heruntergefahren, muss erst eine Instanz erzeugt werden. Der RMAN benutzt Standardeinstellungen, um eine Minimalinstanz zuerstellen. Diese kann f\"ur das weitere Recovery genutzt werden.
              \begin{lstlisting}[caption={Zieldatenbank im RMAN in den NOMOUNT-Status bringen},label=admin1425,language=rman]
RMAN> startup nomount
              \end{lstlisting}
              \begin{merke}
                Dieser Schritt ist in SQL*Plus nicht m\"oglich!
              \end{merke}
            \item Wiederherstellen des SPFiles. Dieser Schritt unterscheidet sich, je nachdem, ob die Instanz in der NOMOUNT-Phase oder ge\"offnet ist.
              \begin{itemize}
                \item \textbf{NOMOUNT-Phase}
                  \begin{lstlisting}[caption={Wiederherstellen des SPFiles in der NOMOUNT-Phase},label=admin1426,language=rman]
RMAN> RESTORE spfile
2>    FROM '/u02/backup/3ukkpd6p.bkp';
                  \end{lstlisting}
\clearpage
                \item \textbf{MOUNT-Phase oder ge\"offnete Instanz}

								Bei ge\"offneter oder gemounteter Instanz muss im Anschluss an diesen Arbeitsschritt das wiederhergestellte SPFile in das \oscommand{\$ORACLE\_HOME/dbs}-Verzeichnis verschoben werden. Dies ist notwendig, da Oracle das (vermeindlich noch vorhandene) SPFile mit einer Schreibsperre, im Dateisystem belegt.
								\begin{lstlisting}[caption={Wiederherstellen des SPFiles in der MOUNT-Phase},label=admin1427,language=rman]
RMAN> RESTORE spfile
2>    TO '/u01/app/oracle/product/11.2.0/spfileorcl.ora'
3>    FROM '/u02/backup/3ukkpd6p.bkp';
                  \end{lstlisting}
              \end{itemize}
            \item Durchstarten der Instanz
              \begin{lstlisting}[caption={Neustart der Instanz},label=admin1428,language=rman,alsolanguage=sqlplus]
RMAN> startup force
              \end{lstlisting}
              Nach dem Neustart werden die Initialisierungsparameter gem\"a\ss{} dem wiederhergestellten SPFile gesetzt und m\"ussen auf ihre Aktualit\"at \"uberpr\"uft werden.
            \end{enumerate}
        \subsubsection{Wiederherstellen eines PFiles}
          Da ein PFile nicht automatisch durch den Datenbankserver verwaltet wird, gibt es auch keinen datenbankeigenen Backupmechanismus. F\"ur die Sicherung und Wiederherstellung eines PFiles muss manuell gesorgt werden. Es empfiehlt sich daher, bei Benutzung eines PFiles, dieses durch Betriebssystemmittel vor Verlust zu sch\"utzen.
          \begin{merke}
            Schon seit Oracle 9i wird davon abgeraten, ein PFile zu benutzen.
          \end{merke}
      \subsection{Verlust einer Passwortdatei beheben}
        Die Passwortdatei geh\"ort zu den Dateien, die nicht durch RMAN gesichert werden k\"onnen. Daher sollte auch diese auf anderem Wege gesichert werden. Geht die Passwortdatei verloren und es existiert kein Backup, muss sie wie neu erstellt werden.
      \subsection{Verlust von Konfigurationsdateien beheben}
        F\"ur Konfigurationsdateien wie, z. B. die \oscommand{tnsnames.ora}, \oscommand{listener.ora} oder die Datei \oscommand{sqlnet.ora} gilt Gleiches, wie f\"ur die Passwortdatei. Bei einem Verlust m\"ussen sie neu angelegt werden und sollten daher auf jeden Fall gesichert werden.
      \subsection{Verlust eines Tempfiles beheben}
        Geht im laufenden Betrieb der Datenbank ein Tempfile verloren, kann bei der Ausf\"uhrung eines SQL-Statements folgende Fehlermeldung auftreten:
        \begin{lstlisting}[caption={Verlust eines Tempfiles},label=admin1429,language=oracle_sql]
ORA-01565: error in identifying file
'/u02/oradata/ORCL/temp01.dbf'
ORA-27037: unable to obtain file status
Linux Error: 2: No such file or directory
        \end{lstlisting}
        Es gibt zwei M\"oglichkeiten auf diese Situation zu reagieren:
        \begin{itemize}
          \item Neustarten der Instanz. Hierbei wird das Tempfile automatisch neu erstellt. In der Alert Log Datei wird eine Meldung ausgegeben, \"ahnlich dieser:
          \begin{lstlisting}[caption={Automatisches Neuerstellen eines Tempfiles},label=admin1430,language=oracle_sql]
Recreating tempfile
'/u02/oradata/ORCL/temp01.dbf'
          \end{lstlisting}
          \item Erstellen eines neuen Tempfiles und l\"oschen der verlorenen Datei aus dem Data Dictionary.
            \begin{enumerate}
              \item Ermitteln zu welchem Temp-Tablespace die besch\"adigte Datei geh\"ort.
                \begin{lstlisting}[caption={Ermitteln des richtigen Temp-Tablespace},label=admin1431,language=oracle_sql]
SQL> SELECT tablespace_name
  2  FROM   dba_temp_files
  3  WHERE  file_name LIKE '/u02/oradata/ORCL/temp01.dbf'

TABLESPACE_NAME
------------------------------
TEMP
                \end{lstlisting}
              \item Neues Tempfile erstellen
              \begin{lstlisting}[caption={Neues Tempfile erstellen},label=admin1432,language=oracle_sql]
SQL> ALTER TABLESPACE temp
  2  ADD TEMPFILE '/u02/oradata/ORCL/temp02.dbf'
  3  SIZE 20 M AUTOEXTEND ON MAXSIZE 500M;
              \end{lstlisting}
              \item Das besch\"adigte Tempfile l\"oschen.
              \begin{lstlisting}[caption={Besch\"adigtes Tempfile l\"oschen},label=admin1433,language=oracle_sql]
SQL> ALTER TABLESPACE temp
  2  DROP TEMPFILE '/u02/oradata/ORCL/temp01.dbf';
              \end{lstlisting}
            \end{enumerate}
        \end{itemize}
    \section{Datafile Media Recovery}
      Ein Media Recovery k\"ummert sich immer um die Behebung von Datenbankfehlern, die durch den Verlust einer der folgenden Dateiarten zustande gekommen sind:
      \begin{itemize}
        \item Datendateien
        \item Kontrolldatei
        \item Archive Log Dateien
      \end{itemize}
      Archive Log Dateien bilden dabei eine Ausnahme, da sie meist nur tempor\"ar f\"ur ein Recovery ben\"otigt werden, m\"ussen sie nicht zwingend sofort wiederhergestellt werden, sondern es gen\"ugt eine funktionsf\"ahige Kopie dieser Dateien zu besitzen.
      \subsection{Identifizieren besch\"adigter Dateien}
        \label{identifyfiles}
        Um zu analysieren welche Dateien ein Recovery brauchen, ist Folgendes notwendig:
        \begin{enumerate}
          \item Starten von SQL*Plus
          \item Status der Instanz abfragen
          \begin{lstlisting}[caption={Status der Instanz abfragen},label=admin1434,language=oracle_sql]
SQL> SELECT status FROM v$instance;

STATUS
------------
&OPEN&
          \end{lstlisting}
          Auch wenn der Status \enquote{OPEN} angezeigt wird, ist es m\"oglich,
          dass Teile der Datenbank ein Recovery ben\"otigen.
          \item Damit die folgenden Abfragen sinnvolle Ergebnisse liefern, muss zuerst ein Checkpoint stattfinden, da erst dann der Status der Datendateien aktualisiert wird.
          \begin{lstlisting}[caption={Einen Checkpoint absetzen},label=admin1435,language=oracle_sql]
SQL> ALTER SYSTEM CHECKPOINT;
          \end{lstlisting}
          \item Den Dateistatus abfragen. Hierf\"ur gibt es mehrere M\"og\-lich\-kei\-ten:
          \begin{enumerate}
            \item Abfragen der View \identifier{v\$datafile\_header}

						Jede zur\"uckgegebene Zeile steht stellvertretend f\"ur eine Datendatei. Die Spalte \identifier{recover} zeigt an, ob eine Datendatei ein Recovery ben\"otigt oder nicht. In der Spalte \identifier{error} wird angezeigt, ob ein Problem mit der Datendatei vorliegt (z. B. ob der Datendateiheader nicht gelesen werden konnte).
						\begin{lstlisting}[caption={Status der Datendateien abfragen mit	V\$DATAFILE\_HEADER},label=admin1436,language=oracle_sql,alsolanguage=sqlplus,emph={[9]FILE,OFFLINE,TABLESPACE, NOT},emphstyle={[9]\color{black}}]
SQL> col file# format 999
SQL> col status format A7
SQL> col error format A10
SQL> col tablespace_name format A10
SQL> col name format A30
SQL> SELECT file#, status, error, recover, tablespace_name, name
  2  FROM   v$datafile_header
  3  WHERE  recover = 'YES'
  4    OR   (recover IS NULL
  5    AND  error IS NOT NULL);

&\textcolor{black}{FILE\#}&  STATUS  ERROR           REC &\textcolor{black}{TABLESPACE}&   NAME
----- ------- --------------- --- ---------- ----------------
6     &\textcolor{black}{OFFLINE}&   &\textcolor{black}{FILE}& &\textcolor{black}{NOT}& FOUND
              \end{lstlisting}

              Zeigt die Spalte \identifier{error} einen Wert an, sollte zuerst nach Hardware- oder Betriebssystemproblemen gesucht werden. Liegen keine derartigen Probleme vor, muss die betreffende Datendatei wiederhergestellt werden.

              Zeigt die Spalte \identifier{recover} den Wert YES und die Spalte \identifier{error} ist leer, so ist meist nur ein Recovery der Datendatei notwendig. Da diese View aber nur die Header der Datendateien liest, kann sie nicht alle Probleme anzeigen, die ein Recovery der Datendatei notwendig machen!
              \begin{merke}
                Die Anzeige defekter Datendateien funktioniert nur dann
                zuverl\"assig, wenn der Parameter
                \parameter{filesystemio\_options} den Wert \textit{directIO}
                hat. Diese Einstellung sorgt daf\"ur, dass Informationen direkt
                auf den Datentr\"ager geschrieben und Caching-Mechanismen des
                Dateisystems umgangen werden.
              \end{merke}
            \item Abfragen der View \identifier{v\$recover\_file}
              \begin{lstlisting}[texcl=true, caption={Status der Datendateien abfragen mit V\$RECOVER\_FILE},label=admin1437,language=oracle_sql,emph={[10]FILE,OFFLINE,CHANGE, TIME, NOT},emphstyle={[10]\color{black}}]
SQL> SELECT file#, error, online_status, change#, time 
  2  FROM   v$recover_file;

&\textcolor{black}{FILE\#}& ERROR         ONLINE_ &\textcolor{black}{CHANGE\#}& &\textcolor{black}{TIME}& 
----- ------------------------------- ----------------- ------------------- -------
6    &\textcolor{black}{FILE}& &\textcolor{black}{NOT}& FOUND &\textcolor{black}{OFFLINE}&         0
              \end{lstlisting}
              Zus\"atzlich zu \identifier{v\$datafile\_header} zeigt sie an, dass eine Datendatei defekte Bl\"ocke enth\"alt. Im Gegensatz zu \identifier{v\$datafile\_header} kann sie nur dann verwendet werden, wenn die Kontrolldatei nicht wiederhergestellt werden musste.
\clearpage
            \item Abfragen der Views \identifier{v\$datafile} und \identifier{v\$tablespace}
              \begin{lstlisting}[caption={Status der Datendateien abfragen},label=admin1438,language=oracle_sql,alsolanguage=sqlplus,emph={[9]FILE,OFFLINE,CHANGE, TIME, NOT},emphstyle={[9]\color{black}}]
col df#       format 999
col df_name   format A35
col tbsp_name format A7
col status    format A7
col error     format A10
col change#   format 99999999
SELECT r.file# AS df#, d.name AS df_name, t.name AS tbsp_name,
      d.status, r.error
FROM   v$recover_file r, v$datafile d, v$tablespace t
WHERE  t.ts# = d.ts#
  AND  d.file# = r.file#;

 DF# DF_NAME                             TBSP_NA STATUS  ERROR       
---- ----------------------------------- ------- ------- ----------
   6 /u02/oradata/ORCL/example01.dbf     EXAMPLE RECOVER &\textcolor{black}{FILE}& &\textcolor{black}{NOT}&
                                                         FOUND
              \end{lstlisting}
          \end{enumerate}
        \end{enumerate}
    \subsection{Ermitteln der ben\"otigten Backups}
      RMAN bietet mit dem Kommando \languagerman{RESTORE PREVIEW} die M\"oglichkeit festzustellen, welche Backups f\"ur einen Restore-Prozess ben\"otigt werden. Dies ist gerade dann wichtig, wenn Backup Sets auf SBT-Tapes liegen und erst wieder in die FRA zur\"uckgeholt werden m\"ussen. Im folgenden Beispiel wird angenommen, dass der Tablespace \identifier{bank} vollst\"andig wiederhergestellt werden muss.
      \begin{lstlisting}[caption={Vorschau auf ein Restore},label=admin1439,language=rman]
RMAN> RESTORE PREVIEW tablespace bank;

Starting restore at 30-OCT-13
using channel ORA_DISK_1
using channel ORA_DISK_2
using channel ORA_SBT_TAPE_1


List of Backup Sets
===================


BS Key  Type LV Size
------- ---- -- ----------
16      Full    26.34M
					\end{lstlisting}
\clearpage
					\begin{lstlisting}[language=rman]
  List of Datafiles in backup set 16
  File LV Type Ckp &SCN&    Ckp Time  Name
  ---- -- ---- ---------- --------- ----
  6       Full 2084386    29-OCT-13 /u01/app/oracle/oradata/orcl/bank01.dbf

...

BS Key  Size       Device Type Elapsed Time Completion Time
------- ---------- ----------- ------------ ---------------
30      240.25M    SBT_TAPE    00:00:03     30-OCT-13
        BP Key: 48   Status: &AVAILABLE&  Compressed: NO  Tag: TAG20131030T111928
        Handle: 18onmqq1_1_1   Media: /u04,18onmqq1_1_1

  List of Archived Logs in backup set 30
  Thrd Seq     Low &SCN&    Low Time  Next &SCN&   Next Time
  ---- ------- ---------- --------- ---------- ---------
  1    56      2085711    29-OCT-13 2086381    29-OCT-13
  1    57      2086381    29-OCT-13 2111298    30-OCT-13
  1    58      2111298    30-OCT-13 2111915    30-OCT-13
  1    59      2111915    30-OCT-13 2112126    30-OCT-13
validation succeeded for backup piece
Media recovery start &SCN& is 2084386
Recovery must be done beyond &SCN& 2084393 to clear datafile fuzziness
validation succeeded for backup piece
Finished restore at 30-OCT-13
            \end{lstlisting}
      Die Ausgabe des \languagerman{RESTORE PREVIEW}-Befehls zeigt eine komplette Auflistung aller Backup Sets und Copies, die der RMAN f\"ur den angegebenen Restore-Prozess heranziehen w\"urde. Besonderes Augenmerk muss auf die Backup Sets gelegt werden, die auf SBT-Tapes gespeichert sind, wie hier z. B. Backup Set Nummer 30. Dessen Inhalt sollte vor dem Restore des Tablespaces in die FRA zur\"uckgeholt werden.
    \subsection{Wiedherstellen der gesamten Datenbank}
      \begin{merke}
        Sollte die Datenbank eines oder mehrere verschl\"usselte Elemente (Tabellenspalten, Tablespaces, o. \"a.) enthalten, muss vor dem Recovery das Encryption Wallet ge\"offnet werden. \ref{recoverencryptedts} zeigt wie ein verschl\"usselter Tablespace recovered wird.
      \end{merke}
\clearpage
      \subsubsection{Alle ben\"otigten Dateien befinden sich in der FRA}
        Das Wiederherstellen der gesamten Datenbank ist der mit Sicherheit zeitaufwendigste Recovery-Fall der entstehen kann. Ob und wann die gesamte Datenbank wiederherzustellen ist, kann von verschiedenen Faktoren abgeleitet werden. Die folgende Liste zeigt nur einige dieser Gr\"unde:
        \begin{itemize}
          \item Es wurden mehr als 70 \% aller Datendateien zerst\"ort.
          \item Die Datenbank muss auf einen anderen Server umgezogen werden.
        \end{itemize}
        Im Folgenden wird ein einfaches vollst\"andiges Restore and Recovery einer ganzen Datenbank, ohne Recovery Katalog durchgef\"uhrt. Die Kontrolldatei, dass SPFile und mindestens eine gespiegelte Kopie der Redo Logs der Datenbank sind noch erhalten. Des Weiteren befinden sich alle ben\"otigten Archive Logs in der FRA.
        \begin{enumerate}
          \item Mit dem RMAN an der Zieldatenbank anmelden
            \begin{lstlisting}[caption={An der Zieldatenbank anmelden},label=admin1440,language=rman]
[oracle@FEA11-119SRV ~]$ rman target /
            \end{lstlisting}
          \item Die Datenbank in den Mount-Status bringen
            \begin{lstlisting}[caption={Datenbank mounten},label=admin1441,language=rman,alsolanguage=sqlplus]
RMAN> startup force mount
          \end{lstlisting}
          \item Das Restore durchf\"uhren
            \begin{lstlisting}[caption={Restore der Datendateien},label=admin1442,language=rman]
RMAN> RESTORE database;
            \end{lstlisting}
          \item Recovern der Datenbank
            \begin{lstlisting}[caption={Recovery der Datenbank},label=admin1443,language=rman]
RMAN> RECOVER database;
            \end{lstlisting}
          \item \"Offnen der Datenbank
            \begin{lstlisting}[caption={\"Offnen der Datenbank nach dem Recovery},label=admin1444,language=rman,emph={[10]ALTER,DATABASE,OPEN},emphstyle={[10]\color{magenta}\bfseries}]
RMAN> SQL 'ALTER DATABASE OPEN';
            \end{lstlisting}
        \end{enumerate}
        \begin{merke}
          Da in diesem Fall die Kontrolldatei der Zieldatenbank erhalten blieb,
          gibt es keinen Unterschied zwischen einem Recovery mit Hilfe eines
          Recovery Katalogs und einem Recovery ohne Nutzung des Recovery
          Katalogs.
        \end{merke}
        \subsubsection{Archive Logs  m\"ussen von SBT zur\"uckgeholt werden}
          Das vorangegangene Beispiel wird nun dahingehend abgewandelt, dass nicht alle ben\"otigten Archive Logs in der FRA vorliegen.
          \begin{enumerate}
            \item Mit dem RMAN an der Zieldatenbank anmelden
              \begin{lstlisting}[caption={An der Zieldatenbank anmelden},label=admin1445,language=rman]
[oracle@FEA11-119SRV ~]$ rman target /
              \end{lstlisting}
            \item Die Datenbank in den Mount-Status bringen
              \begin{lstlisting}[caption={Datenbank mounten},label=admin1446,language=rman,alsolanguage=sqlplus]
RMAN> startup force mount
            \end{lstlisting}
            \item Ermitteln welche Backup Sets auf SBT-Tape liegen.
              \begin{lstlisting}[caption={Ermitteln der ben\"otigten Dateien},label=admin1447,language=rman]
RMAN> RESTORE PREVIEW database;
              \end{lstlisting}
            \item Zur\"uckholen der ben\"otigten Archive Logs.
              \begin{lstlisting}[caption={Restore der Archive Logs von SBT in die FRA},label=admin1448,language=rman]
RMAN> RESTORE archive log UNTIL SEQUENCE 60;
              \end{lstlisting}
            \item Das Restore der Datenbank durchf\"uhren
              \begin{lstlisting}[caption={Restore der Datendateien},label=admin1449,language=rman]
RMAN> RESTORE database;
              \end{lstlisting}
            \item Recovern der Datenbank
              \begin{lstlisting}[caption={Recovery der Datenbank},label=admin1450,language=rman]
RMAN> RECOVER database;
              \end{lstlisting}
            \item \"Offnen der Datenbank
              \begin{lstlisting}[caption={\"Offnen der Datenbank nach dem Recovery},label=admin1451,language=rman,emph={[10]ALTER,DATABASE,OPEN},emphstyle={[10]\color{magenta}\bfseries}]
RMAN> SQL 'ALTER DATABASE OPEN';
              \end{lstlisting}
            \item L\"oschen der nicht mehr ben\"otigten Archive Logs.
              \begin{lstlisting}[caption={L\"oschen der Archive Logs, die bereits auf SBT-Tape gesichert wurden},label=admin1452,language=rman]
RMAN> DELETE archivelog all BACKED UP 1 TIMES
2>    TO DEVICE TYPE sbt;
              \end{lstlisting}
              \begin{lstlisting}[caption={L\"oschen der Archive Logs, die bereits auf Disk gesichert wurden},label=admin1453,language=rman]
RMAN> DELETE archivelog all BACKED UP 1 TIMES
2>    TO DEVICE TYPE disk;
              \end{lstlisting}
          \end{enumerate}
          Durch die Klausel \languagerman{BACKUPED UP 1 TIMES} wird beim
          L\"oschen sichergestellt, dass nur solche Archive Logs aus der FRA entfernt werden, die bereits mindestens 1x gesichert worden sind.
      \subsection{Wiederherstellung eines Tablespaces}
        \subsubsection{Wenn der Tablespace unverschl\"usselt ist}
          Die Wiederherstellung eines Tablespaces kann sowohl im MOUNT- als auch im OPEN-Status der Datenbank durchgef\"uhrt werden. Im folgenden Beispiel wird angenommen, dass der Tablespace \identifier{bank} besch\"adigt wurde, w\"ahrend die Datenbank ge\"offnet ist. Wie schon beim Recovery der gesamten Datenbank, so muss auch hier erst gepr\"uft werden, ob alle ben\"otigten Dateien in der FRA vorliegen.
          \begin{enumerate}
            \item Mit dem RMAN an der Zieldatenbank anmelden
              \begin{lstlisting}[caption={An der Zieldatenbank anmelden},label=admin1454,language=terminal]
[oracle@FEA11-119SRV ~]$ rman target /
              \end{lstlisting}
            \item \"Uberpr\"ufen ob alle ben\"otigten Dateien in der FRA vorliegen
              \begin{lstlisting}[caption={Voraussetzungen \"uberpr\"ufen},label=admin1455,language=rman]
RMAN> RESTORE PREVIEW tablespace bank;
              \end{lstlisting}
            \item Nicht vorliegende Archive Logs von SBT-Tapes holen
              \begin{lstlisting}[caption={Voraussetzungen \"uberpr\"ufen},label=admin1456,language=rman]
RMAN> RESTORE archive log UNTIL SEQUENCE 60;
              \end{lstlisting}
            \item Offline setzen des betreffenden Tablespaces
              \begin{lstlisting}[caption={Betreffenden Tablespace Offline
              setzen},label=admin1457,language=rman,emph={[10]ALTER,TABLESPACE,OFFLINE,IMMEDIATE},emphstyle={[10]\color{magenta}\bfseries}]
RMAN> SQL 'ALTER TABLESPACE bank OFFLINE IMMEDIATE';
              \end{lstlisting}
            \item Das Restore des Tablespaces durchf\"uhren
              \begin{lstlisting}[caption={Restore des betreffenden Tablespaces},label=admin1458,language=rman]
RMAN> RESTORE tablespace bank;
              \end{lstlisting}
            \item Recovern des Tablespaces
              \begin{lstlisting}[caption={Recovery des Tablespaces},label=admin1459,language=rman]
RMAN> RECOVER tablespace bank DELETE archivelog;
              \end{lstlisting}
              Die Klausel \languagerman{DELETE archivelog} sorgt daf\"ur, dass alle nicht mehr ben\"otigten Archive Logs sofort gel\"oscht werden. Dabei werden keine Logs gel\"oscht, die noch nicht gesichert wurden.
            \item Online setzen des betreffenden Tablespaces nach dem Recovery
              \begin{lstlisting}[caption={Betreffenden Tablespace Online setzen},label=admin1460,language=rman,emph={[10]ALTER,TABLESPACE,ONLINE},emphstyle={[10]\color{magenta}\bfseries}]
RMAN> SQL 'ALTER TABLESPACE bank ONLINE';
              \end{lstlisting}
          \end{enumerate}
        \subsubsection{Wenn der Tablespace verschl\"usselt ist}
          \label{recoverencryptedts}
          Bei einem verschl\"usselten Tablespace sind grunds\"atzlich die gleichen Schritte f\"ur ein Restore and Recovery notwendig, wie bei einem Unverschl\"usselten. Der einzige Unterschied ist, dass ein kryptierter Tablespace nur dann recovered werden kann, wenn das Wallet vorher ge\"offnet wurde.
          \begin{enumerate}
            \item Mit dem RMAN an der Zieldatenbank anmelden
              \begin{lstlisting}[caption={An der Zieldatenbank anmelden},label=admin1461,language=terminal]
[oracle@FEA11-119SRV ~]$ rman target /
              \end{lstlisting}
            \item Das Wallet \"offnen
              \begin{lstlisting}[caption={\"Offnen des Wallets},label=admin1462,language=rman,emph={[9]ALTER,SYSTEM,SET,ENCRYPTION,WALLET,OPEN,IDENTIFIED,BY},emphstyle={[9]\color{magenta}\bfseries}]
RMAN> SQL'ALTER SYSTEM SET ENCRYPTION WALLET OPEN IDENTIFIED BY "P@ssw0rd"';
              \end{lstlisting}
            \item \"Uberpr\"ufen ob alle ben\"otigten Dateien in der FRA vorliegen
              \begin{lstlisting}[caption={Voraussetzungen \"uberpr\"ufen},label=admin1463,language=rman]
RMAN> RESTORE PREVIEW tablespace bank;
              \end{lstlisting}
            \item Nicht vorliegende Archive Logs von SBT-Tapes holen
              \begin{lstlisting}[caption={Voraussetzungen \"uberpr\"ufen},label=admin1464,language=rman]
RMAN> RESTORE archive log UNTIL SEQUENCE 60;
              \end{lstlisting}
            \item Offline setzen des betreffenden Tablespaces
              \begin{lstlisting}[caption={Betreffenden Tablespace Offline setzen},label=admin1465,language=rman,emph={[9]ALTER,TABLESPACE,OFFLINE,IMMEDIATE},emphstyle={[9]\color{magenta}\bfseries}]
RMAN> SQL 'ALTER TABLESPACE bank OFFLINE IMMEDIATE';
              \end{lstlisting}
            \item Das Restore des Tablespaces durchf\"uhren
              \begin{lstlisting}[caption={Restore des betreffenden Tablespaces},label=admin1466,language=rman]
RMAN> RESTORE tablespace bank;
              \end{lstlisting}
            \item Recovern des Tablespaces
              \begin{lstlisting}[caption={Recovery des Tablespaces},label=admin1467,language=rman]
RMAN> RECOVER tablespace bank DELETE archivelog;
              \end{lstlisting}
              Die Klausel \languagerman{DELETE archivelog} sorgt daf\"ur, dass alle nicht mehr ben\"otigten Archive Logs sofort gel\"oscht werden. Dabei werden keine Logs gel\"oscht, die noch nicht gesichert wurden.
            \item Online setzen des betreffenden Tablespaces nach dem Recovery
              \begin{lstlisting}[caption={Betreffenden Tablespace Online setzen},label=admin1468,language=rman,emph={[9]ALTER,TABLESPACE,ONLINE},emphstyle={[9]\color{magenta}\bfseries}]
RMAN> SQL 'ALTER TABLESPACE bank ONLINE';
              \end{lstlisting}
            \item Das Wallet schlie\ss{}en
              \begin{lstlisting}[caption={Schlie\ss{}en des Wallets},label=admin1469,language=rman,emph={[9]ALTER,SYSTEM,SET,ENCRYPTION,WALLET,CLOSE,IDENTIFIED,BY},emphstyle={[9]\color{magenta}\bfseries}]
RMAN> SQL'ALTER SYSTEM SET ENCRYPTION WALLET CLOSE IDENTIFIED BY "P@ssw0rd"';
              \end{lstlisting}
          \end{enumerate}
      \subsection{Wiederherstellen einzelner Datendateien}
        Sind nur einzelne Datendateien besch\"adigt, ist es meist wirtschaftlicher, explizit die besch\"adigten Dateien wiederherzustellen und nicht die gesamten Tablespaces.
        \subsubsection{Datendateien Offline setzen}
          Datendateien k\"onnen genauso wie Tablespaces Offline gesetzt werden. Wenn dies geschieht, ist der Tablespace der sie enth\"alt nicht verf\"ugbar, bis die Datendatei wieder Online gesetzt wurde.

          Um eine Datendatei Offline zu setzen, wird das \languageorasql{ALTER DATABASE}-Kommando und das \privileg{alter database}-System Privileg ben\"otigt.
          \begin{lstlisting}[caption={Eine Datendatei offline setzen},label=admin1470,language=oracle_sql]
SQL> ALTER DATABASE
  2  DATAFILE '/u02/oradata/orcl/bankl01.dbf' OFFLINE;
          \end{lstlisting}
          Eine Datendatei kann nicht nur mit ihrem Dateinamen, sondern auch mit ihrer internen Dateinummer angesprochen werden. Herausfinden kann man die Dateinummer mit Hilfe der View \identifier{dba\_data\_files}.
          \begin{lstlisting}[caption={Herausfinden der file\_id einer Datendatei},label=admin1471,language=oracle_sql]
SQL> SELECT file_id
  2  FROM   dba_data_files
  3  WHERE  file_name LIKE '%bank01%';

   FILE_ID
----------
         6
          \end{lstlisting}
          Im \languageorasql{ALTER DATABASE}-Kommando wird der Dateiname durch die Dateinummer ersetzt.
          \begin{lstlisting}[caption={Eine Datendatei mit Hilfe der file\_id Offline setzen},label=admin1472,language=oracle_sql]
SQL> ALTER DATABASE
  2  DATAFILE 6 OFFLINE;
          \end{lstlisting}
          Soll die Datendatei wieder Online gebracht werden, wird das \languageorasql{ALTER DATABASE}-Statement zusammen mit dem Schl\"usselwort \languageorasql{ONLINE} verwendet.
          \begin{lstlisting}[caption={Eine Datendatei online setzen},label=admin1473,language=oracle_sql]
SQL> ALTER DATABASE
  2  DATAFILE 6 ONLINE;
          \end{lstlisting}
        \subsubsection{Restore and Recovery von Datendateien}
          Im folgenden Beispiel wird angenommen, dass die Datenbank ge\"offnet ist.
          \begin{enumerate}
            \item Mit dem RMAN an der Zieldatenbank anmelden
              \begin{lstlisting}[caption={An der Zieldatenbank anmelden},label=admin1474,language=rman]
[oracle@FEA11-119SRV ~]$ rman target sys/oracle
              \end{lstlisting}
            \item Offline setzen der betreffenden Datendatei
              \begin{lstlisting}[caption={Betreffende Datendatei Offline setzen},label=admin1475,language=rman,emph={[9]ALTER,DATABASE,DATAFILE,OFFLINE,IMMEDIATE},emphstyle={[9]\color{magenta}\bfseries}]
RMAN> SQL 'ALTER DATABASE DATAFILE 6 OFFLINE';
              \end{lstlisting}
            \item Das Restore der Datendatei durchf\"uhren
              \begin{lstlisting}[caption={Restore der betreffenden Datendatei},label=admin1476,language=rman]
RMAN> RESTORE datafile 6;
              \end{lstlisting}
            \item Recovern der Datendatei
              \begin{lstlisting}[caption={Recovery der Datendatei},label=admin1477,language=rman]
RMAN> RECOVER datafile 6 DELETE archivelog;
              \end{lstlisting}
            \item Online setzen der Datendatei nach dem Recovery
              \begin{lstlisting}[caption={Betreffende Datendatei Online setzen},label=admin1478,language=rman,emph={[9]ALTER,DATABASE,DATAFILE,ONLINE},emphstyle={[9]\color{magenta}\bfseries}]
RMAN> SQL 'ALTER DATABASE DATAFILE 6 ONLINE';
              \end{lstlisting}
          \end{enumerate}
        \subsubsection{Restore and Recovery einer Datendatei an einem neuen Speicherort}
          Um eine Datendatei an einem neuen Speicherort wiederherzustellen, muss dieser in der Kontrolldatei angegeben werden. RMAN bietet f\"ur das Angeben des neuen Speicherorts das \languagerman{SET NEWNAME}-Kommando und \languagerman{SWITCH DATAFILE} zum Umschalten auf den neuen Speicherort.

          Die Kombination der beiden RMAN-Kommandos \languagerman{SET NEWNAME} und \languagerman{SWITCH DATAFILE} ist vergleichbar mit dem SQL-Kommando \languageorasql{ALTER DATABASE RENAME FILE}.

          Im folgenden Beispiel wird davon ausgegangen, dass der Tablespace \identifier{bank} besch\"adigt wurde. Eine der beiden Datendateien dieses Tablespaces, muss an einem anderen Speicherort wiederhergestellt werden, da der Originalspeicherort derzeit nicht verf\"ugbar ist. Mit Hilfe der unter \ref{identifyfiles} beschriebenen Methode wurde herausgefunden, dass die Datendatei \oscommand{bank02.dbf} die Nummer 7 hat.
          \begin{enumerate}
            \item Mit dem RMAN an der Zieldatenbank anmelden
              \begin{lstlisting}[caption={An der Zieldatenbank anmelden},label=admin1479,language=rman]
[oracle@FEA11-119SRV ~]$ rman target sys/oracle
              \end{lstlisting}
            \item Offline setzen der betreffenden Datendatei
              \begin{lstlisting}[caption={Betreffende Datendatei Offline setzen},label=admin1480,language=rman,emph={[9]ALTER,TABLESPACE,OFFLINE,TEMPORARY},emphstyle={[9]\color{magenta}\bfseries}]
RMAN> SQL 'ALTER TABLESPACE bank OFFLINE TEMPORARY';
              \end{lstlisting}
            \item \"Offnen eines RUN-Blocks
            \begin{lstlisting}[caption={Einen RUN-Block \"offnen},label=admin1481,language=rman]
RMAN> RUN {
            \end{lstlisting}
            \item Neuen Speicherort der Datendatei festlegen
            \begin{lstlisting}[caption={Neuen Speicherort festlegen},label=admin1482,language=rman]
2>      SET NEWNAME FOR datafile 7 TO '/u02/oradata/ORCL/bank02.dbf';
            \end{lstlisting}
            \item Durchf\"uhren des Restores
            \begin{lstlisting}[caption={Restore durchf\"uhren},label=admin1483,language=rman]
3>      RESTORE datafile 7;
            \end{lstlisting}
            \item Umschalten auf den neuen Speicherort der Datendatei
            \begin{lstlisting}[caption={Auf den neuen Speicherort umschalten},label=admin1484,language=rman]
4>      SWITCH DATAFILE ALL;
            \end{lstlisting}
            \item Recovery der Datendatei durchf\"uhren und den RUN-Block schlie\ss{}en.
            \begin{lstlisting}[caption={Recovery durchf\"uhren},label=admin1485,language=rman]
5>      RECOVER datafile 7;
6>   }
            \end{lstlisting}
            \item Datendatei in den ONLINE-Status bringen
            \begin{lstlisting}[caption={Datendatei in den ONLINE-Status bringen},label=admin1486,language=rman,emph={[9]ALTER,TABLESPACE,ONLINE},emphstyle={[9]\color{magenta}\bfseries}]
RMAN> SQL 'ALTER TABLESPACE bank ONLINE';
            \end{lstlisting}
          \end{enumerate}
      \subsection{Wiederherstellen von Archive Logs}
        Ben\"otigt der RMAN Archive Logs f\"ur ein Recovery, stellt er diese selbstst\"andig wieder her. In manchen F\"allen ist es jedoch notwendig, Archive Logs manuell wiederherzustellen. Im Folgenden werden zwei Szenarios beschrieben, bei denen die Archive Logs manuell wiederherzustellen sind.
        \subsubsection{Die Archive Logs an einem neuen Speicherort wiederherstellen}
          Standardm\"assig geben \parameter{log\_archive\_format} und \parameter{log\_archive\_dest\_n} vor, unter welchem Namen und an welchem Speicherort die Archive Logs wiederhergestellt werden. Wenn es sinnvoll ist, kann mit dem \languagerman{SET ARCHIVELOG DESTINATION}-Kommando des RMAN ein anderer Ort definiert werden.
          \begin{enumerate}
            \item \"Offnen eines RUN-Blocks
              \begin{lstlisting}[caption={Einen RUN-Block \"offnen},label=admin1488,language=rman]
RMAN> RUN {
              \end{lstlisting}
            \item Neuen Speicherort der Archive Logs festlegen
              \begin{lstlisting}[caption={Neuen Speicherort festlegen},label=admin1489,language=rman]
2>      SET ARCHIVELOG DESTINATION TO '/tmp';
              \end{lstlisting}
            \item Wiederherstellen der Archive Logs und schlie\ss{}en des RUN-Blockes.
              \begin{lstlisting}[caption={Wiederherstellen der Archive Logs},label=admin1490,language=rman]
3>      RESTORE archivelog ALL;
4>    }
              \end{lstlisting}
          \end{enumerate}
        \subsubsection{Archive Logs auf mehrere Speicherorte verteilen}
          Es ist m\"oglich die Menge der Archive Logs auf mehrere Orte aufzuteilen, falls dies aus Speicherplatzgr\"unden notwendig ist. Das folgende Beispiel stellt 150 Archive Logs wieder her und verteilt diese auf zwei Locations.
          \begin{lstlisting}[caption={Wiederherstellen der Archive Logs an verschiedenen Speicherorten},label=admin1491,language=rman]
RMAN> RUN {
2>      SET ARCHIVELOG DESTINATION TO '/u02/backup';
3>      RESTORE ARCHIVELOG FROM SEQUENCE 1 UNTIL SEQUENCE 75;
4>
5>      SET ARCHIVELOG DESTINATION TO '/u03/backup';
6>      RESTORE ARCHIVELOG FROM SEQUENCE 76 UNTIL SEQUENCE 150;
7>    }
          \end{lstlisting}
          Wird innerhalb dieses RUN-Blocks ein Recovery durchgef\"uhrt, findet RMAN automatisch seine Archive Logs an den unterschiedlichen Speicherorten.
          \begin{merke}
            Durch den Befehl \languagerman{SET} get\"atigte Einstellungen gelten immer nur innerhalb des RUN-Blockes, in dem Sie vorgenommen wurden.
          \end{merke}
    \section{Block Media Recovery}
      Es kann vorkommen, dass durch einen Fehler, einzelne Bl\"ocke in einer Datendatei besch\"a\-digt werden. Um solche Sch\"aden zu beheben, ist es zum einen m\"oglich, die gesamte Datendatei wiederherzustellen oder aber nur die defekten Bl\"ocke zu recovern. Werden nur einzelne Bl\"ocke einer Datendatei repariert, spricht man von \enquote{Block Media Recovery}.

      Block Media Recovery hat dem Datafile Media Recovery (Recovery einer
      ganzen Datendatei) gegen\"uber folgende Vorteile:
      \begin{itemize}
        \item Die Zeit f\"ur das Instance Recovery verringert sich, da nur die wiederhergestellten Bl\"ocke dem Instance Recovery unterzogen werden m\"ussen.
        \item Die betroffene Datendatei kann w\"ahrend eines Block Media Recovery Online bleiben.
      \end{itemize}
      Das Block Media Recovery unterliegt aber auch einer ganzen Reihe von
      Einschränkungen, die beachtet werden müssen:
      \begin{itemize}
        \item Block Media Recovery kann nur aus dem RMAN heraus gesteuert werden.
        \item Ein einmal begonnenes Block Media Recovery muss vollst\"andig durchgef\"uhrt werden.
				\item F\"ur Block Media Recovery k\"onnen nur Full-Backups herangezogen werden, keine inkrementellen Backups.
        \item Es k\"onnen nur solche Bl\"ocke mit Block Media Recovery repariert werden, die als corrupt\footnote{corrupt = engl. fehlerhaft} markiert wurden. In \identifier{v\$database\_block\_corruption} kann ersehen werden, welche Bl\"ocke dies sind. Diese View wird aber nicht automatisch bef\"ullt. Die betroffene Datendatei muss erst mit \languagerman{VALIDATE} validiert werden.
      \end{itemize}
      \begin{merke}
        Block Media Recovery ist nur in der Enterprise Version von Oracle 11g m\"oglich!
      \end{merke}
      \subsection{Wann sollte Block Media Recovery angewendet werden?}
        Block Media Recovery stellt eine Erweiterung zum Datafile Media Recovery dar. In F\"allen, in denen die Anzahl der defekten Bl\"ocke sehr hoch ist, kann Block Media Recovery das Datafile Media Recovery nicht ersetzen.

        Das ein defekter Block vorliegt, kann an der Fehlermeldung ORA-01578 festgestellt werden. Hier ein Beispiel f\"ur eine solche Fehlermeldung:
        \begin{lstlisting}[caption={Der Fehler ORA-01578},label=admin1492,language=terminal]
ORA-01578: ORACLE data block corrupted (file # 7, block # 3)
ORA-01110: data file 7: '/u02/oradata/orcl/bank02.dbf'
ORA-01578: ORACLE data block corrupted (file # 7, block # 235)
ORA-01110: data file 2: '/u02/oradata/orcl/bank02.dbf'
        \end{lstlisting}
      \subsection{Block Media Recovery und die Redo Logs}
        F\"ur Datafile Media Recovery ist immer eine ununterbrochene Kette von Archived Logs notwendig, da das Recovery sonst nicht erfolgreich sein kann. Block Media Recovery kann unter Umst\"anden auch dann noch erfolgreich sein, wenn einzelne Archived Logs verloren gegangen sind. Die einzige Bedingung f\"ur Block Media Recovery ist, dass alle Archived Logs, die Informationen \"uber die wiederherzustellenden Bl\"ocke enthalten, vorhanden sein m\"ussen.

        Wenn der RMAN das Fehlen eines Archived Logs feststellt, bricht er seine Arbeit nicht sofort ab. Es kann vorkommen, dass ein Datenblock in einem sp\"ateren Archive Log als \enquote{newed block} gef\"uhrt wird. Dies geschieht beispielsweise dann, wenn alle Zeilen einer Tabelle gel\"oscht wurden oder die gesamte Tabelle aus dem Block gel\"oscht wurde.
        In so einem Fall formatiert Oracle den Block neu, wodurch alle alten Redo Informationen irrelevant werden.
      \subsection{Backups defekter Datendateien anfertigen}
        \label{corruptfile}
        Bevor ein Block Media Recovery durchgef\"uhrt wird, sollte zuerst ein Backup der betroffenen Datendatei(en) gemacht werden. Standardm\"a\ss{}ig weigert sich der RMAN eine Datendatei mit besch\"adigten Bl\"ocken zu sichern. Dies resultiert aus der Standardeinstellung des RMAN-Parameters \languagerman{MAXCORRUPT} mit dem Wert 0. Dieser Parameter kann so konfiguriert werden, dass eine bestimmte Anzahl defekter Bl\"ocke f\"ur eine Datendatei zul\"assig ist.
        \begin{lstlisting}[caption={Hotbackup einer defekten Datendatei anfertigen},label=admin1493,language=rman]
RMAN> RUN {
2>      SET MAXCORRUPT FOR datafile 6 TO 999;
3>      BACKUP AS BACKUPSET DATAFILE 6;
4>    }
        \end{lstlisting}
      \subsection{Block Media Recovery durchf\"uhren}
        Das RMAN-Kommando \languagerman{BLOCKRECOVER} kann einzelne Datenbl\"ocke in der Datenbank wiederherstellen. Das folgende Szenario zeigt eine einfache Nutzung dieses Kommandos.
        \begin{enumerate}
          \item Zwei defekte Bl\"ocke werden angezeigt.
          \begin{lstlisting}[caption={Der Fehler ORA-01578},label=admin1494,language=terminal]
ORA-01578: ORACLE data block corrupted (file # 7, block # 3)
ORA-01110: data file 7: '/u02/oradata/orcl/bank02.dbf'
ORA-01578: ORACLE data block corrupted (file # 7, block # 235)
ORA-01110: data file 2: '/u02/oradata/orcl/bank02.dbf'
          \end{lstlisting}
          \item Die Bl\"ocke k\"onnen recovered werden, w\"ahrend die Datenbank ge\"offnet ist.
          \begin{lstlisting}[caption={Das Kommando BLOCKRECOVER},label=admin1495,language=rman]
RMAN> BLOCKRECOVER datafile 7 block 3
2>    datafile 2 block 235;
          \end{lstlisting}
        \end{enumerate}
        Ist die Liste der defekten Bl\"ocke sehr gro\ss{}, kann das folgende Kommando genutzt werden, um alle korrupten Bl\"ocke in einem Arbeitsgang zu recovern:
          \begin{lstlisting}[caption={Das Kommando BLOCKRECOVER},label=admin1496,language=rman]
RMAN> BLOCKRECOVER CORRUPTION LIST;
          \end{lstlisting}
        \begin{literaturinternet}
          \item \cite{i1016424}
        \end{literaturinternet}
    \section{Unvollst\"andiges Recovery (Point-In-Time Recovery)}
      Point-In-Time Recovery versetzt die Datenbank in einen Stand zur\"uck, der zeitlich vor dem Aktuellen liegt. Diese Art des Recovery wird auch \textit{unvollst\"andiges Recovery} genannt, da nicht alle vorhandenen Archive Logs und inkrementellen Backups zur Wiederherstellung der Datenbank benutzt werden. Ein Point-In-Time-Recovery kann gezielt erfolgen oder unfreiwillig notwendig werden, da bestimmte Teile der Datenbank verloren gegangen sind.
      \subsection{Die Datenbank im Verlauf der Zeit}
        Ist Datenverlust f\"ur eine Datenbank inakzeptabel, m\"ussen zwei Dinge sichergestellt werden:
        \begin{enumerate}
          \item Die Datenbank muss sich im ARCHIVELOG Modus befinden.
          \item Es m\"ussen regelm\"assig Backups der Datenbank angefertigt werden.
        \end{enumerate}
        Wenn diese beiden Bedingungen erf\"ullt sind, k\"onnte ein Ausschnitt aus dem Lebenszyklus einer Datenbank wie folgt aussehen:
        \bild{Lebenszyklus einer Datenbank erster Teil}{lebenszyklus1}{0.4}
        \abbildung{lebenszyklus1} zeigt eine Datenbank f\"ur die die drei Backups, Backup 1, Backup 2 und Backup 3 existieren. Desweitern sind insgesamt 16 Archive Logs mit der Logsequenz Nummern 1 bis 16 vorhanden. In diesem Szenario kann die Datenbank auf jeden beliebigen Zeitpunkt zwischen der Erstellung von Backup 1 und JETZT zur\"uckgesetzt werden.

        In der folgenden \abbildung{lebenszyklus2} wird der Fall dargestellt, dass zum Zeitpunkt X f\"alschlicher Weise eine Tabelle gel\"oscht wurde, die unbedingt wiederhergestellt werden muss.
        \bild{Lebenszyklus einer Datenbank zweiter Teil}{lebenszyklus2}{0.4}
        Um den Verlust der Tabelle wieder auszugleichen, muss die Datenbank auf einen Zeitpunkt zur\"uckgesetzt werden, der vor dem Zeitpunkt X liegt.
        \bild{Lebenszyklus einer Datenbank dritter Teil}{lebenszyklus3}{0.4}
        Um dieses Ziel zu erreichen, m\"ussen folgende Schritte durchgef\"uhrt
        werden:
        \begin{enumerate}
          \item RESTORE: Wiederherstellen von Backup Nummer 3
          \item RECOVER: Benutzen der Archive Logs Nummer 12, 13, 14 und von Teilen des Archive Log Nummer 15 um sich an den Zeitpunkt (X - n) heranzutasten.
          \item \"Offnen der Datenbank
        \end{enumerate}
        Die Folgen dieses Szenarios w\"aren:
        \begin{itemize}
          \item Die Datenbank wurde zur\"uckgesetzt auf die Logsequenz Nummer 14.
          \item Der Inhalt der aktuellen Redo Log Datei, mit der Logsequenz Nummer 17 ist unbrauchbar.
        \end{itemize}
        Daraus ergibt sich Folgendes:
        \begin{enumerate}
          \item Die Redo Log Dateien m\"ussen geleert/formatiert werden, da ihr Inhalt unbrauchbar ist. Die Redo Logs werden im weiteren Verlauf mit neuem Inhalt gef\"ullt.
          \item Bei einem Logswitch w\"urde erneut eine Archive Log Datei mit der Nummer 15 erzeugt werden, die es aber bereits gibt.
        \end{enumerate}
        Um die Problematik der Erzeugung von Archive Log Dateien mit gleicher Log Sequenze Nummer (in diesem Beispiel LSN 15) zu vermeiden, hat Oracle den Begriff der \enquote{Inkarnation} eingef\"uhrt.
        \begin{merke}
          Unter einer Inkarnation versteht Oracle den Lebenszyklus einer Datenbank. Der Anfang eines Lebenszykluses wird durch den Zeitpunkt markiert, an dem die Logsequenz Nummer 1 vergeben wurde. Das Ende wird durch ein Point-In-Time-Recovery markiert.
        \end{merke}
        Mit diesen neuen Erkenntnissen ergibt sich nach dem Recovery der Datenbank, aus dem obigen Beispiel, ein anderes Bild:

        \bild{Lebens\-zyklus einer Datenbank vierter Teil}{lebenszyklus4}{0.5}

        Durch das \"Offnen der Datenbank und das Zur\"ucksetzen der Logsequenz Nummer auf 1 wird ein neuer Lebenszyklus der Datenbank begonnen, also eine neue Inkarnation.
        \begin{merke}
          Eine neue Datenbankinkarnation wird immer dann erstellt, wenn beim \"Offnen der Datenbank das Kommando \languageorasql{ALTER DATABASE OPEN RESETLOGS} verwendet wird. Die Z\"ahlung der Inkarnationen beginnt bei 1 und wird fortlaufend durchgef\"uhrt.
        \end{merke}
        Obwohl in dieser Situation jetzt zwei Archive Log Dateien mit der Logsequenz Nummer 1 existieren, kann die Datenbank diese Beiden, anhand der Inkarnationsnummer 1 bzw. 2 auseinanderhalten.

        \begin{merke}
          Um sehen zu k\"onnen, welche Datenbankinkarnationen existieren, kann im RMAN das Kommando \languagerman{LIST INCARNATION} oder in SQL*Plus die View  \identifier{v\$database\_incarnation} benutzt werden.
        \end{merke}

        \begin{lstlisting}[caption={Die View v\$database\_incarnation},label=admin1497,language=oracle_sql,emph={[9]CURRENT},emphstyle={[9]\color{black}}]
SQL> SELECT incarnation#, resetlogs_change#, prior_incarnation#
  2  FROM   v$database_incarnation

INCARNATION# RESETLOGS_CHANGE# PRIOR_INCARNATION#
------------ ----------------- ------------------
           1                 1                  0
           2            446075                  1
           3            498982                  2
           4            500243                  2

          \end{lstlisting}
\clearpage
          In \beispiel{admin1497} zeigt die Spalte \textit{RESETLOGS\_CHANGE\#}
          die SCN, des Startzeitpunktes der jeweiligen Inkarnation an. Wie zu
          sehen ist, beginnt Inkarnation Nummer 1 mit SCN 1 und jede weitere
          Inkarnation mit einer h\"oheren SCN. Somit l\"ast sich folgendes Bild
          zeichnen:

          \bild{Da\-ten\-bank\-in\-kar\-na\-tion\-en}{incarnations_scn}{0.375}

        \subsubsection{Inkarnationen und die Logsequenz Nummer (LSN)}
          Eine neue Inkarnation wird mit dem Kommando \lstinline[language=oracle_sql]{ALTER DATABASE OPEN RESETLOGS} erzeugt. Das Schl\"usselwort \lstinline[language=oracle_sql]{RESETLOGS} deutet darauf hin, dass die Logsequenz Nummer bei jeder neuen Inkarnation auf den Wert 1 zur\"uckgesetzt wird. Tats\"achlich geschehen folgende Schritte:
          \begin{enumerate}
            \item Die aktuellen Redo Logs werden archiviert,
            \item die Log Sequenze Nummer wird auf den Wert 1 zur\"uckgesetzt und
            \item die Redo Logs erhalten einen neuen Zeitstempel, sowie eine neue SCN.
          \end{enumerate}
      \subsection{Voraussetzungen f\"ur Database Point-In-Time Recovery}
        Um ein Database-Point-In-Time-Recovery durchf\"uhren zu k\"onnen, m\"ussen die folgenden Voraussetzungen gegeben sein:
        \begin{itemize}
          \item Die betreffende Datenbank muss sich im Archivelog-Modus befinden.
          \item Es m\"ussen Backups aller Datendateien bestehen, die vor dem gew\"unschten Reco\-very\-zeit\-punkt entstanden sind.
          \item Alle Archive Logs, die zum Roll-Forward des Backups bis zum gew\"unschten Reco\-very\-zeit\-punkt ben\"otigt werden, m\"ussen vorhanden sein.
        \end{itemize}
      \subsection{Database Point-In-Time Recovery vorbereiten}
        Die folgenden Schritte sollten f\"ur ein Datenbank Point-In-Time Recovery vorbereitet werden.
        \begin{itemize}
          \item Festlegen des Zielzeitpunktes, der SCN, des Restore-Points oder der Log Sequence Number, bei der das Recovery stoppen soll. Dies kann mit Hilfe der Oracle Flashback Features geschehen. Auch die Alert.log-Datei kann hierbei dienlich sein.
          \item Soll als Abbruchkriterium f\"ur das Recovery ein Zeitpunkt verwendet werden, sollten die beiden Umgebungsvariablen \oscommand{NLS\_LANG} und \oscommand{NLS\_DATE\_FORMAT} gesetzt sein.
        \end{itemize}
      \subsection{Point-In-Time Recovery durchf\"uhren}\label{dbpitrinc}
        \begin{enumerate}
          \item Starten des RMAN und mit der Zieldatenbank verbinden.
            \begin{lstlisting}[caption={Starten und Anmelden},label=admin1498,language=rman]
[oracle@FEA11-119SRV ~]$ rman target sys/oracle
            \end{lstlisting}
          \item \"Uberf\"uhren der Datenbank in die MOUNT-Phase.
            \begin{lstlisting}[caption={Shutdown und Mounten},label=admin1499,language=rman,alsolanguage=sqlplus]
RMAN> shutdown immediate
RMAN> startup mount
            \end{lstlisting}
          \item Die folgenden Schritte sollten in einem RUN-Block ausgef\"uhrt werden:
            \begin{enumerate}
              \item Zielzeitpunkt mit dem \languagerman{SET UNTIL}-Kommando festlegen.
              \item Kan\"ale zum Zugriff auf die Datenbank konfigurieren, falls keine automatisch Vorkonfigurierten vorhanden sind.
              \item Restore and Recovery
            \end{enumerate}
            \begin{lstlisting}[caption={Restore and Recovery},label=admin1500,language=rman]
RMAN> RUN {
2>      SET UNTIL TIME '31.10.2013 10:30:00';
#Alternativen
# SET UNTIL SCN 487159;
# SET UNTIL SEQUENCE 62;
# SET UNTIL RESTORE POINT before_update;
3>
4>      RESTORE database;
5>      RECOVER database;
6>    }
            \end{lstlisting}
          \item Datenbank mit \languageorasql{OPEN RESETLOGS} \"offnen
            \begin{lstlisting}[caption={Datenbank mit open resetlogs \"offnen},label=admin1501,language=rman,emph={[9]ALTER,DATABASE,OPEN,RESETLOGS},emphstyle={[9]\color{magenta}\bfseries}]
RMAN> SQL 'ALTER DATABASE OPEN RESETLOGS';
            \end{lstlisting}
        \end{enumerate}
      \subsection{Recovery nach Verlust der Kontrolldatei (mit Recovery Katalog)}
        \label{recoverywithabackupcontrolfile}
        Wenn es vorkommt, dass alle Kopien der Kontrolldatei einer Datenbank zerst\"ort werden oder verloren gehen, muss eine Kontrolldatei aus einem Backup wiederhergestellt werden. Diese wird dann als \enquote{Backupcontrolfile} bezeichnet. Da die Kontrolldatei aus einem Backup, nicht mehr den aktuellsten Stand der Datenbank wiederspiegelt, muss die gesamte Datenbank auf den Stand des Backupcontrolfile recovered werden.
        \subsubsection{Wiederherstellen eines Controlfiles aus einem Controlfile Autobackup}
          \begin{enumerate}
            \item Mit dem RMAN an der Zieldatenbank und am Recovery Katalog anmelden.
              \begin{lstlisting}[caption={An der Zieldatenbank und am Recovery Katalog anmelden},label=admin1502,language=rman]
[oracle@FEA11-119SRV ~]$ rman target sys/oracle catalog catowner/catpass@CATDB
              \end{lstlisting}
            \item Wurde die Zieldatenbank heruntergefahren, muss erst eine Instanz erzeugt werden. Der RMAN benutzt Standardeinstellungen, um eine Minimalinstanz zuerstellen. Diese kann f\"ur das weitere Recovery genutzt werden.
              \begin{lstlisting}[caption={Zieldatenbank im RMAN in den NOMOUNT-Status bringen},label=admin1503,language=rman,alsolanguage=sqlplus]
RMAN> startup nomount
              \end{lstlisting}
              \begin{merke}
                Dieser Schritt ist in SQL*Plus nicht m\"oglich!
              \end{merke}
            \item Wiederherstellen des Controlfiles aus dem Controlfile Autobackup
              \begin{lstlisting}[caption={Wiederherstellen des Controlfiles},label=admin1504,language=rman]
RMAN> RESTORE controlfile
2>    FROM AUTOBACKUP;
              \end{lstlisting}
            \item Die Zieldatenbank in den MOUNT-Status versetzen
              \begin{lstlisting}[caption={Zieldatenbank mounten},label=admin1505,language=rman,emph={[9]ALTER,DATABASE,MOUNT},emphstyle={[9]\color{magenta}\bfseries}]
RMAN> SQL 'ALTER DATABASE MOUNT';
              \end{lstlisting}
            \item Wiederherstellen aller Datendateien
              \begin{lstlisting}[caption={Datendateien wiederherstellen},label=admin1506,language=rman]
RMAN> RESTORE database;
              \end{lstlisting}
            \item Recovern der Datenbank
              \begin{lstlisting}[caption={Recovern der Datenbank},label=admin1507,language=rman]
RMAN> RECOVER database;
              \end{lstlisting}
            \item \"Offnen der Datenbank mit der Option \languageorasql{OPEN RESETLOGS}
              \begin{lstlisting}[caption={Datenbank mit open resetlogs \"offnen},label=admin1508,language=rman,emph={[9]ALTER,DATABASE,OPEN,RESETLOGS},emphstyle={[9]\color{magenta}\bfseries}]
RMAN> SQL 'ALTER DATABASE OPEN RESETLOGS';
              \end{lstlisting}
            \end{enumerate}
        \subsubsection{Wiederherstellen eines Controlfiles aus einem Backup Set}
          \begin{enumerate}
            \item Mit dem RMAN an der Zieldatenbank und am Recovery Katalog anmelden.
              \begin{lstlisting}[caption={An der Zieldatenbank und am Recovery Katalog anmelden},label=admin1509,language=rman]
[oracle@FEA11-119SRV ~]$ rman target sys/oracle catalog catowner/catpass@CATDB
              \end{lstlisting}
            \item Wurde die Zieldatenbank heruntergefahren, muss erst eine Instanz erzeugt werden. Der RMAN benutzt Standardeinstellungen, um eine Minimalinstanz zuerstellen. Diese kann f\"ur das weitere Recovery genutzt werden.
              \begin{lstlisting}[caption={Zieldatenbank im RMAN in den NOMOUNT-Status bringen},label=admin1510,language=rman,alsolanguage=sqlplus]
RMAN> startup nomount
              \end{lstlisting}
              \begin{merke}
                Dieser Schritt ist in SQL*Plus nicht m\"oglich!
              \end{merke}
            \item Wiederherstellen des Controlfiles.
              \begin{lstlisting}[caption={Wiederherstellen des Controlfiles},label=admin1511,language=rman]
RMAN> RESTORE controlfile;
              \end{lstlisting}
            \item Die Zieldatenbank in den MOUNT-Status versetzen
              \begin{lstlisting}[caption={Zieldatenbank mounten},label=admin1512,language=rman,emph={[9]ALTER,DATABASE,MOUNT},emphstyle={[9]\color{magenta}\bfseries}]
RMAN> SQL 'ALTER DATABASE MOUNT';
              \end{lstlisting}
            \item Wiederherstellen aller Datendateien
              \begin{lstlisting}[caption={Datendateien wiederherstellen},label=admin1513,language=rman]
RMAN> RESTORE database;
              \end{lstlisting}
            \item Recovern der Datenbank
              \begin{lstlisting}[caption={Recovern der Datenbank},label=admin1514,language=rman]
RMAN> RECOVER database;
              \end{lstlisting}
            \item \"Offnen der Datenbank mit der Option \languageorasql{OPEN RESETLOGS}
              \begin{lstlisting}[caption={Datenbank mit open resetlogs \"offnen},label=admin1515,language=rman,emph={[9]ALTER,DATABASE,OPEN,RESETLOGS},emphstyle={[9]\color{magenta}\bfseries}]
RMAN> SQL 'ALTER DATABASE OPEN RESETLOGS';
              \end{lstlisting}
            \end{enumerate}
\clearpage
      \subsection{Recovery nach Verlust der Kontrolldatei (ohne Recovery Katalog)}
        \subsubsection{Wiederherstellen eines Controlfiles aus einem Controlfile Autobackup}
          \begin{enumerate}
            \item Mit dem RMAN an der Zieldatenbank anmelden.
              \begin{lstlisting}[caption={An der Zieldatenbank anmelden},label=admin1516,language=rman]
[oracle@FEA11-119SRV ~]$ rman target sys/oracle
              \end{lstlisting}
            \item Setzen der DBID
              \begin{lstlisting}[caption={Setzen der DBID},label=admin1517,language=rman]
RMAN> SET DBID 1351916467;
              \end{lstlisting}
            \item Wurde die Zieldatenbank heruntergefahren, muss erst eine Instanz erzeugt werden. Der RMAN benutzt Standardeinstellungen, um eine Minimalinstanz zuerstellen. Diese kann f\"ur das weitere Recovery genutzt werden.
              \begin{lstlisting}[caption={Zieldatenbank im RMAN in den NOMOUNT-Status bringen},label=admin1518,language=rman,alsolanguage=sqlplus]
RMAN> startup nomount
              \end{lstlisting}
              \begin{merke}
                Dieser Schritt ist in SQL*Plus nicht m\"oglich!
              \end{merke}
            \item Wiederherstellen des Controlfiles aus dem Controlfile Autobackup
              \begin{lstlisting}[caption={Wiederherstellen des Controlfiles},label=admin1519,language=rman]
RMAN> RESTORE controlfile FROM AUTOBACKUP
2>    RECOVERY AREA '/u05/fast_recovery_area'
3>    DB_NAME       'orcl';
              \end{lstlisting}
              \begin{merke}
                Wenn die Instanz nicht gemountet oder ge\"offnet ist, ist es dem
                RMAN unm\"oglich, den Speicherort der Controlfile Autobackups zu
                ermitteln. Mit Hilfe der beiden Parameter \languagerman{RECOVERY
                AREA} und \languagerman{DB_NAME} wird der Speicherort, die Fast
                Recovery Area, dem RMAN mitgeteilt.
              \end{merke}
            \item Die Zieldatenbank in den MOUNT-Status versetzen
              \begin{lstlisting}[caption={Zieldatenbank
              mounten},label=admin1520,language=rman,emph={[9]ALTER,DATABASE,MOUNT},emphstyle={[9]\color{magenta}\bfseries}]
RMAN> SQL 'ALTER DATABASE MOUNT'
              \end{lstlisting}
\clearpage
            \item Wiederherstellen aller Datendateien
              \begin{lstlisting}[caption={Datendateien wiederherstellen},label=admin1521,language=rman]
RMAN> RESTORE database;
              \end{lstlisting}
            \item Recovern der Datenbank
              \begin{lstlisting}[caption={Recovern der Datenbank},label=admin1522,language=rman]
RMAN> RECOVER database;
              \end{lstlisting}
            \item \"Offnen der Datenbank mit der Option \languageorasql{OPEN RESETLOGS}
              \begin{lstlisting}[caption={Datenbank mit open resetlogs \"offnen},label=admin1523,language=rman,emph={[9]ALTER,DATABASE,OPEN,RESETLOGS},emphstyle={[9]\color{magenta}\bfseries}]
RMAN> SQL 'ALTER DATABASE OPEN RESETLOGS';
              \end{lstlisting}
            \end{enumerate}
        \subsubsection{Wiederherstellen eines Controlfiles aus einem Backup Set}
          \begin{enumerate}
            \item Mit dem RMAN an der Zieldatenbank anmelden und die DBID setzen.
              \begin{lstlisting}[caption={An der Zieldatenbank anmelden und die DBID setzen},label=admin1524,language=rman]
[oracle@FEA11-119SRV ~]$ rman target sys/oracle

RMAN> SET DBID 1351916467;
                  \end{lstlisting}
            \item Wurde die Zieldatenbank heruntergefahren, muss erst eine Instanz erzeugt werden. Der RMAN benutzt Standardeinstellungen, um eine Minimalinstanz zuerstellen. Diese kann f\"ur das weitere Recovery genutzt werden.
              \begin{lstlisting}[caption={Zieldatenbank im RMAN in den NOMOUNT-Status bringen},label=admin1525,language=rman,alsolanguage=sqlplus]
RMAN> startup nomount
              \end{lstlisting}
              \begin{merke}
                Dieser Schritt ist in SQL*Plus nicht m\"oglich!
              \end{merke}
            \item Wiederherstellen des Controlfiles.
              \begin{lstlisting}[caption={Wiederherstellen des Controlfiles},label=admin1526,language=rman]
RMAN> RESTORE controlfile
2>    FROM '/u02/3ukkpd6p.bkp';
              \end{lstlisting}
            \item Die Zieldatenbank in den MOUNT-Status versetzen
              \begin{lstlisting}[caption={Zieldatenbank mounten},label=admin1527,language=rman,emph={[9]ALTER,DATABASE,MOUNT},emphstyle={[9]\color{magenta}\bfseries}]
RMAN> SQL 'ALTER DATABASE MOUNT'
              \end{lstlisting}
            \item Wiederherstellen aller Datendateien
              \begin{lstlisting}[caption={Datendateien wiederherstellen},label=admin1528,language=rman]
RMAN> RESTORE database;
              \end{lstlisting}
            \item Recovern der Datenbank
              \begin{lstlisting}[caption={Recovern der Datenbank},label=admin1529,language=rman]
RMAN> RECOVER database;
              \end{lstlisting}
            \item \"Offnen der Datenbank mit der Option \languageorasql{OPEN RESETLOGS}
              \begin{lstlisting}[caption={Datenbank mit open resetlogs \"offnen},label=admin1530,language=rman,emph={[9]ALTER,DATABASE,OPEN,RESETLOGS},emphstyle={[9]\color{magenta}\bfseries}]
RMAN> SQL 'ALTER DATABASE OPEN RESETLOGS';
              \end{lstlisting}
            \end{enumerate}
      \subsection{Wiederherstellen einer Datenbank im NOARCHIVELOG Modus}
        Das Wiederherstellen einer Datenbank im NOARCHIVELOG Modus ist \"ahnlich dem Wiederherstellen einer Datenbank im ARCHIVELOG Modus. Die Hauptunterschiede dabei sind:
        \begin{itemize}
          \item Es k\"onnen nur Coldbackups zur Wiederherstellung verwendet werden.
          \item Block Media Recovery ist nicht m\"oglich, da keine Archive Logs existieren.
        \end{itemize}
        Mit Hilfe von inkrementellen Backups ist auch im NOARCHIVELOG Modus eine eingeschr\"ankte Form des Recovery m\"oglich.

        Das folgende Szenario verdeutlicht das Recovery einer Datenbank im NOARCHIVELOG Modus.
        \begin{itemize}
          \item Die Datenbank l\"auft im NOARCHIVELOG Modus.
          \item Es wird ein Recovery Katalog benutzt.
          \item Die Datenbank wird konsistent heruntergefahren und es wird ein Level 0 Backup am Sonntagabend gemacht.
          \item Am darauf folgenden Mittwoch wird die Datenbank ebenfalls konsistent heruntergefahren und es wird ein Level 1 Backup angefertigt.
          \item Donnerstagabend crashed die Datenbank. Es gehen 50 \% aller Datendateien und s\"amtliche Redo Logs verloren.
        \end{itemize}
        In solch einem Fall muss ein Media Recovery unter Nutzung des Level 0 und des Level 1 Backups durchgef\"uhrt werden. Weiterhin muss ber\"ucksichtig werden, dass die Redo Logs komplett verloren gegangen sind.
        \begin{enumerate}
          \item Mit dem RMAN an der Zieldatenbank und am Recovery Katalog anmelden.
            \begin{lstlisting}[caption={An der Zieldatenbank und am Recovery Katalog anmelden},label=admin1531,language=rman]
[oracle@FEA11-119SRV ~]$ rman target / catalog catowner/catpass@CATDB
            \end{lstlisting}
          \item Herunterfahren der Datenbank und starten im NOMOUNT-Status.
            \begin{lstlisting}[caption={Shutdown und Mounten},label=admin1532,language=rman,alsolanguage=sqlplus]
RMAN> startup force nomount;
            \end{lstlisting}
            Ein \languagesqlplus{startup force nomount} kann ohne Weiteres durchgef\"uhrt werden, da die Datenbank bereits irreparabel besch\"adigt ist. Ein konsistentes Herunterfahren ist nicht mehr n\"otig/m\"oglich.
          \item Wiederherstellen der Kontrolldatei auf einem Autobackup oder einem Backup Set.
            \begin{lstlisting}[caption={Kontrolldatei wiederherstellen},label=admin1533,language=rman]
RMAN> RESTORE controlfile;
            \end{lstlisting}
          \item Die Datenbank in den MOUNT-Status \"uberf\"uhren
            \begin{lstlisting}[caption={MOUNT-Status erreichen},label=admin1534,language=rman,emph={[9]ALTER, DATABASE,MOUNT},emphstyle={[9]\color{magenta}\bfseries}]
RMAN> SQL 'ALTER DATABASE MOUNT';
            \end{lstlisting}
          \item Wiederherstellen der Datenbankdateien
            \begin{lstlisting}[caption={Datenbankdateien wiederherstellen},label=admin1535,language=rman]
RMAN> RESTORE database;
            \end{lstlisting}
          \item Recovern der Datenbank
            \begin{lstlisting}[caption={Datenbankdateien wiederherstellen},label=admin1536,language=rman]
RMAN> RECOVER database NOREDO;
            \end{lstlisting}
          \item \"Offnen der Datenbank
            \begin{lstlisting}[caption={Datenbank mit open resetlogs \"offnen},label=admin1537,language=rman,emph={[9]ALTER,DATABASE,OPEN,RESETLOGS},emphstyle={[9]\color{magenta}\bfseries}]
RMAN> SQL 'ALTER DATABASE OPEN RESETLOGS';
            \end{lstlisting}
        \end{enumerate}
        Bei diesem Vorgang werden alle \"Anderungen, die bis einschlie\ss{}lich des Level 1 Backups vorgenommen wurden recovered. Die Angabe von \languagerman{NOREDO} sorgt daf\"ur, dass RMAN nicht versucht, die Redo Logs beim Recovery zu nutzen, da diese nicht mehr existieren.

        Selbst wenn die Redo Logs noch existieren w\"urden, m\"usste trotzdem die \languagerman{NOREDO}-Klausel angegeben werden, da zwischen den letzten \"Anderung im Level 1 Backup und den Redo Logs eine gro\ss{}e L\"ucke klafft.
    \section{Der Data Recovery Advisor}
      Der Data Recovery Advisor ist eine mit Oracle 11g neu eingef\"uhrte PL/SQL-Anwendung, die den Adminstrator bei einem Recovery-Szenario unterst\"utzt. Er kann automatisch nach Fehlern und L\"osungen suchen und, per Anweisung des Admins, eine Reparatur ausf\"uhren.

      Ohne dieses Tool muss der Admin selbstst\"andig eine Fehlerdiagnose durchf\"uhren und anschlie\ss{}end geeignete Ma\ss{}nahmen ergreifen, um die Sch\"aden zu beheben. Eine solches Prozedere ist sehr umst\"andlich und fehleranf\"allig, weshalb es gute Kenntnisse der Materie und ein hohes Ma\ss{} an Erfahrung erfordert.
      \subsection{Fehler}
        Fehler, im Sinne des Data Recovery Advisors sind persistente St\"orungen der Datenbank die durch einen \enquote{Data Integrity Check} gefunden wurden. Sobald ein Fehler festgestellt wurde, kann der Data Recovery Advisor Informationen dar\"uber liefern und ihn gegeben\-en\-falls beheben. Gespeichert werden Fehler nicht innerhalb der Datenbank, sondern im Automatic Diagnostic Repository, kurz ADR, einer Verzeichnisstruktur im Dateisystem. Daher wird die Funktionsweise des Data Recovery Advisor nicht beeintr\"achtigt, falls sich die Datenbank in der NOMOUNT-Phase befindet.
        \subsubsection{Fehlerarten}
          Der Data Recovery Advisor ist in der Lage, verschiedenste Fehlerarten zu erkennen. Beispielsweise kann er feststellen, dass Datenbankdateien (Datendateien, Kontroll- und Redo Log Dateien) nicht ge\"offnet werden k\"onnen, weil sie nicht existieren oder Zugriffsrechte fehlen. Des Weiteren kann er Besch\"adigungen von Dateien erkennen, wenn z. B. eine Datendatei nicht mehr synchron mit der Datenbank ist (Inkonsistenz) oder wenn einzelne Oraclebl\"ocke physikalisch besch\"adigt sind. Teilweise kann er sogar Fehler erkennen, die nicht in der Datenbank, sondern im Betriebssystem auftreten (Disk I/O, Treiberfehler, usw.).
        \subsubsection{Status und Priorit\"at}
          Jeder Fehler hat einen Status und eine Priorit\"at. Wenn ein Fehler entsteht, erh\"alt er automatisch den Status \enquote{Open}. Sobald der Fehler behoben wurde, wird der Status auf \enquote{Closed} ge\"andert. Diese Status\"anderung geschieht entweder durch das RMAN-Kommando \languagerman{LIST FAILURE} oder durch \languagerman{CHANGE FAILURE}.

          Wie schwerwiegend ein Fehler ist, wird in drei Priorit\"atsklassen ausgedr\"uckt: \enquote{LOW}, \enquote{HIGH} und \enquote{CRITICAL}.
          \begin{itemize}
            \item \textbf{CRITICAL}: Dies sind schwerwiegende Fehler, die die Verf\"ugbarkeit der gesamten Datenbank beeinflussen k\"onnen und die deshalb sofortiger Aufmerksamkeit bed\"urfen (Ausfall von Kontrolldateien oder des \identifier{system}-Tablespaces).
            \item \textbf{HIGH}: Dies Fehlerklasse umfasst alle Ereignisse, welche die Verf\"ugbarkeit von Teilen der Datenbank beeinflussen k\"onnen (Korrupte Bl\"ocke, Ausfall von Nicht-System-Tablespaces, fehlen von Archive Logs).
            \item \textbf{LOW}: Die Fehlerklasse \enquote{LOW} wird nicht vom System, sondern nur vom Administrator vergeben, wenn er Fehler der Klasse \enquote{HIGH} als \enquote{nicht besonders wichtig} einstuft.
          \end{itemize}
      \subsection{Die Fehlersuche - LIST FAILURE}
        Das Kommando \languagerman{LIST FAILURE} dient dazu, die Ergebnisse von automatischen oder manuellen Fehlerdiagnosen anzuzeigen. \languagerman{LIST FAILURE} selbst diagnostiziert keine Fehler.
        \begin{lstlisting}[caption={Das Kommando \languagerman{LIST FAILURE}},label=admin1538,language=rman]
RMAN> LIST FAILURE;

List of Database Failures
=========================

Failure ID Priority Status    Time Detected Summary
---------- -------- --------- ------------- -------
42         HIGH     &OPEN&      01-NOV-13     One or more non-system
datafiles are missing
1141       HIGH     &OPEN&      01-NOV-13     Datafile 7:
'/u01/app/oracle/oradata/orcl/bank02.dbf' contains one or more corrupt blocks
        \end{lstlisting}
        \beispiel{admin1538} zeigt zwei Fehler der Kategorie \enquote{HIGH}, versehen mit den IDs 42 und 1141. In der Spalte \enquote{Summary} wird eine kurze Erkl\"arung zu diesen Fehlern angezeigt.
      \subsection{Der Ratschlag - ADVISE FAILURE}
        Mit dem \languagerman{ADVISE FAILURE}-Befehl k\"onnen durch den RMAN generierte Reparaturskripte angezeigt und bereits behobene Fehler geschlossen werden (Status \enquote{CLOSED}).
        \begin{merke}
          Es muss immer zuerst das Kommando \languagerman{LIST FAILURE} ausgef\"uhrt werden, bevor der Befehl \languagerman{ADVISE FAILURE} ausgef\"uhrt werden kann.
        \end{merke}

        \begin{lstlisting}[caption={Das Kommando \languagerman{ADVISE FAILURE}},label=admin1539,language=rman]
RMAN> ADVISE FAILURE;

List of Database Failures
=========================

Failure ID Priority Status    Time Detected Summary
---------- -------- --------- ------------- -------
42         HIGH     &OPEN&      01-NOV-13     One or more non-system
datafiles are missing
1141       HIGH     &OPEN&      01-NOV-13     Datafile 7:
'/u01/app/oracle/oradata/orcl/bank02.dbf' contains one or more corrupt blocks

analyzing automatic repair options; this may take some time
using channel ORA_DISK_1
using channel ORA_DISK_2
allocated channel: ORA_SBT_TAPE_1
channel ORA_SBT_TAPE_1: SID=134 device type=SBT_TAPE
channel ORA_SBT_TAPE_1: WARNING: Oracle Test Disk API
analyzing automatic repair options complete

Mandatory Manual Actions
========================
no manual actions available

Optional Manual Actions
=======================
1. If file /u01/app/oracle/oradata/orcl/bank01.dbf was unintentionally
renamed or moved, restore it

Automated Repair Options
========================
Option Repair Description
------ ------------------
1      Restore and recover datafile 6; Recover multiple corrupt blocks
       in datafile 7
  Strategy: The repair includes complete media recovery with no data loss
  Repair script: /u01/app/oracle/diag/rdbms/orcl/orcl/hm/reco_184824511.hm
        \end{lstlisting}
        Zuerst wird eine Liste der erkannten Fehler angezeigt, bevor dann eine Analyse durchgef\"uhrt wird, um automatische Reparaturma\ss{}nahmen zu finden. Diese werden dann in einem Reparaturskript zusammengefasst, dessen Name ganz unten im Bericht angezeigt wird.
        \begin{lstlisting}[caption={Das Reparaturskript},label=admin1540,language=rman]
Strategy: The repair includes complete media recovery with no data loss
Repair script: /u01/app/oracle/diag/rdbms/orcl/orcl/hm/reco_184824511.hm
        \end{lstlisting}
        \begin{merke}
          RMAN versucht immer seine Reparaturskripte zu konsolidieren, was bedeutet, dass m\"oglichst viele Fehler mit m\"oglichst wenigen Reparaturschritten behoben werden sollen. In manchen F\"allen ist dies jedoch nicht m\"oglich, weshalb RMAN dann eine Meldung anzeigt, dass aktuell einige Fehler nicht behoben werden k\"onnen.
        \end{merke}
        \languagerman{ADVISE FAILURE} zeigt aber nicht nur automatische Reparaturma\ss{}nahmen an, sondern wo immer es sich anbietet auch Manuelle. Diese werden in \enquote{optionale} und \enquote{zwingend notwendige} Ma\ss{}nahmen unterteilt.

        \beispiel{admin1539} zeigt f\"ur Datendatei Nummer 6 einen optionalen Schritt an. Sollte die Datei nur aus Versehen umbenannt oder verschoben worden sein, kann dieser Schritt manuell r\"uckg\"angig gemacht werden. Eine solche Vorgehensweise ist unter Umst\"anden viel zeit- und ressourcensparender als ein Restore \& Recovery der Datendatei.

        Mandatory manual options werden immer dann angezeigt, wenn keine
        automatischen Reparaturschritte erzeugt werden k\"onnen. Dies k\"onnte
        z. B. dann der Fall sein, wenn ein Archive Log fehlt, dass f\"ur das
        Recovery einer Datendatei ben\"otigt wird.
        \begin{lstlisting}[caption={Mandatory manual options},label=admin1541,language=rman]
RMAN> ADVISE FAILURE;

List of Database Failures
=========================

Failure ID Priority Status    Time Detected Summary
---------- -------- --------- ------------- -------
42         HIGH     &OPEN&      01-NOV-13     One or more non-system
datafiles are missing
1141       HIGH     &OPEN&      01-NOV-13     Datafile 7:
'/u01/app/oracle/oradata/orcl/bank02.dbf' contains one or more corrupt blocks

analyzing automatic repair options; this may take some time
using channel ORA_DISK_1
using channel ORA_DISK_2
using channel ORA_SBT_TAPE_1
        \end{lstlisting}
\clearpage
        \begin{lstlisting}[language=rman]
analyzing automatic repair options complete

Mandatory Manual Actions
========================
1. If file /u01/app/oracle/oradata/orcl/bank01.dbf was unintentionally
   renamed or moved, restore it
2. If you have an export of tablespace BANK, then drop and re-create
   the tablespace and import the data.
3. No backup of block 232 in file 7 was found. Drop and re-create the
   associated object (if possible), or use the DBMS_REPAIR package to
   repair the block corruption
4. No backup of block 233 in file 7 was found. Drop and re-create the
   associated object (if possible), or use the DBMS_REPAIR package to
  repair the block corruption
5. No backup of block 234 in file 7 was found. Drop and re-create the
   associated object (if possible), or use the DBMS_REPAIR package to
   repair the block corruption
6. No backup of block 235 in file 7 was found. Drop and re-create the
   associated object (if possible), or use the DBMS_REPAIR package to
   repair the block corruption
7. Contact Oracle Support Services if the preceding recommendations
   cannot be used, or if they do not fix the failures selected for repair

Optional Manual Actions
=======================
no manual actions available

Automated Repair Options
========================
no automatic repair options available
        \end{lstlisting}
      \subsection{Die Reparatur - REPAIR FAILURE}
        Der \enquote{Dritte im Bunde} ist \languagerman{REPAIR FAILURE}. Dieses
        Kommando benutzt das von \languagerman{ADVISE FAILURE} erstellte
        Reparaturskript, um die festgestellten Fehler zu beheben. Vor der
        eigentlichen Reparatur kann mit Hilfe von \languagerman{REPAIR FAILURE
        PREVIEW} zuerst das komplette Repairscript angzeigt werden.
\clearpage
        \begin{lstlisting}[caption={Eine Vorschau auf das Repairscript},label=admin1542,language=rman]
RMAN> REPAIR FAILURE PREVIEW;

Strategy: The repair includes complete media recovery with no data loss
Repair script: /u01/app/oracle/diag/rdbms/orcl/orcl/hm/reco_1758777565.hm

contents of repair script:
   # restore and recover datafile
   sql 'alter database datafile 6 offline';
   restore datafile 6;
   recover datafile 6;
   sql 'alter database datafile 6 online';
   # block media recovery for multiple blocks
   recover datafile 7 block 232 to 235;
        \end{lstlisting}
        Bevor der Befehl \languagerman{REPAIR FAILURE} ausgef\"uhrt wird, sollte immer zuerst eine Endkontrolle des Reparaturskripts erfolgen.
        \begin{lstlisting}[caption={Die Fehler reparieren},label=admin1543,language=rman]
RMAN> REPAIR FAILURE;

Strategy: The repair includes complete media recovery with no data loss
Repair script: /u01/app/oracle/diag/rdbms/orcl/orcl/hm/reco_1758777565.hm

contents of repair script:
   # restore and recover datafile
   sql 'alter database datafile 6 offline';
   restore datafile 6;
   recover datafile 6;
   sql 'alter database datafile 6 online';
   # block media recovery for multiple blocks
   recover datafile 7 block 232 to 235;

Do you really want to execute the above repair (enter YES or NO)? YES
executing repair script
        \end{lstlisting}
\clearpage
    \section{Informationen}
      \subsection{Verzeichnis der relevanten Initialisierungsparameter}
        \begin{literaturinternet}
          \item \cite{REFRN10029}
          \item \cite{REFRN10030}
          \item \cite{REFRN10268}
          \item \cite{REFRN10295}
          \item \cite{REFRN10061}
          \item \cite{REFRN10089}
          \item \cite{REFRN10086}
        \end{literaturinternet}
      \subsection{Verzeichnis der relevanten Data Dictionary Views}
        \begin{literaturinternet}
          \item \cite{REFRN30047}
          \item \cite{REFRN30048}
          \item \cite{REFRN30049}
          \item \cite{sthref3281}
          \item \cite{REFRN30052}
          \item \cite{REFRN30196}
          \item \cite{sthref3785}
        \end{literaturinternet}
\clearpage

    \input{uebungen/dbadmin_18_recovery_mit_dem_rman_uebung}
    \input{loesungen/dbadmin_18_recovery_mit_dem_rman_loesung}
% % %   \input{admin/xx_advanced_diagnostic_repository}
  \chapter{Oracle Flashback}
    \setcounter{page}{1}\kapitelnummer{chapter}
    \minitoc
\newpage
    \section{Die Oracle Flashback Technologie}
      Die Oracle Flashback Technologie stellt M\"oglichkeiten bereit,
      \enquote{in die Vergangenheit} von Daten zu sehen, ohne ein Recovery
      durchf\"uhren zu m\"ussen. Die meisten der Oracle Flash\-back Features
      arbeiten auf der Ebene logischer Backups:
      \begin{itemize}
        \item \textbf{Oracle Flashback Query}: Diese Technologie erm\"oglicht es, Daten einer Tabelle zu einem definierten Zeitpunkt zu sehen. Damit ist es m\"oglich, fehlerhafte \languageorasql{UPDATE}- oder \languageorasql{DELETE}-Operationen r\"uckg\"angig zu machen.
        \item \textbf{Oracle Flashback Version Query}: Mit diesem Feature k\"onnen alle Versionen einer Tabellenzeile innerhalb eines spezifizierten Zeitintervalls betrachtet werden. Es werden nicht nur die Nutzdaten, sondern auch Metadaten (Startzeit, Endzeit und TransaktionsID der Transaktion) angezeigt. Damit ist die M\"oglichkeit gegeben, sowohl Datenverlust zu beheben, als auch Auditing zu betreiben.
        \item \textbf{Oracle Flashback Transaction Query}: Hiermit kann man sich alle \"Anderungen betrachten, die eine einzelne Transaktion oder alle Transaktionen innerhalb eines bestimmten Zeitraums durchgef\"uhrt haben.
        \item \textbf{Oracle Flashback Transaction Backout}: Mit dieser neuen Technologie ist es m\"oglich, bereits committete Transaktionen r\"uckg\"angig zu machen.
        \item \textbf{Oracle Flashback Table}: Dieser Mechanimus erm\"oglicht es, eine Tabelle online in einen Zustand zur\"uck zuversetzen, der durch einen Zeitpunkt definiert wird.
        \item \textbf{Oracle Flashback Drop}: Der Flashback Drop macht es m\"oglich, die Auswirkungen eines \languageorasql{DROP TABLE}-Statements r\"uckg\"angig zu machen.
        \item \textbf{Oracle Flashback Database}: Diese Option ist die \enquote{Rewind-Taste} an der Datenbank. Die gesamte DB kann innerhalb k\"urzester Zeit auf einen genau definierten Stand zur\"uckversetzt werden.
      \end{itemize}
      Die Features Flashback Table, Flashback Query, Flashback Transaction Query und Flash\-back Version Query basieren alle auf Undo-Daten. Der Mechanismus Flashback Drop verwendet einen Speicherbereich der \enquote{Recycle Bin} gennant wird. In diesem Speicherbereich werden gel\"oschte Tabellen gespeichert, bis der Speicher \"uberl\"auft und Platz f\"ur neue Objekte geschaffen werden muss. Alle diese Features sind unabh\"angig von RMAN.

      Auf einer anderen Ebene existiert das Feature \enquote{Oracle Flashback Database}. Es ist eine Alternative zum Point-In-Time-Recovery. Wenn in der Datenbank eine gr\"o\ss ere Anzahl fehlerhafter \"Anderungen gespeichert ist, kann Flashback Database diese Datendateien in einen fr\"uheren Zustand zur\"uckversetzen.

			Die Auswirkungen des Oracle Flashback Database Mechanismus sind denen eines Point-In-Time-Recovery sehr \"ahnlich, jedoch funktioniert der Flashback Mechanismus deutlich schneller als ein Recovery, da das Restore der Datendateien wegf\"allt und nur wenige Redo Logs f\"ur das Flashback ben\"otigt werden.

      Flashback Database benutzt \enquote{Flashback logs} und zus\"atzlich die archivierten Redo Logs, um fr\"uhere Versionen eines Datenblocks wieder herzustellen. Zur Speicherung von Flashback Logs muss eine \enquote{Fast Recovery Area} erstellt werden. Flashback logging ist standardm\"assig deaktiviert.

      Flashback Database ist in den RMAN integriert. Er kann sich automatisch aus den vorhandenen Backups die ben\"otigten archivierten Redo Logs und Flashback Logs holen, um das Flashback durchzuf\"uhren. Auch eine manuelle Nutzung von Flashback Database mit SQL*Plus ist m\"oglich.
    \section{Oracle Flashback Query}
      Flashback Query macht es m\"oglich, einen in der Vergangenheit liegenden Stand einer Tabelle abzufragen. Wenn beispielsweise um 12:30 festgestellt wird, dass aus der Tabelle \identifier{mitarbeiter} der Angestellte \enquote{Wolf} versehentlich gel\"oscht wurde, er aber um 09:30 Uhr noch existierte, kann der Stand der Tabelle \identifier{mitarbeiter} von 09:30 abgefragt werden.

      Das folgende Beispiel zeigt eine Flashback Query auf der Tabelle \identifier{mitarbeiter} mit Hilfe der \languageorasql{AS OF}-Klausel:
      \begin{lstlisting}[caption={Flashback Query mit AS OF TIMESTAMP},label=admin1700,language=oracle_sql]
SQL> SELECT Mitarbeiter_ID, Vorname, Nachname
  2  FROM   mitarbeiter
  3  WHERE  Nachname LIKE 'Wolf'
  4    AND  Mitarbeiter_ID = 98;

no rows selected.

SQL> SELECT Mitarbeiter_ID, Vorname, Nachname
  2  FROM   mitarbeiter AS OF TIMESTAMP
  3         TO_TIMESTAMP('03.11.2013 09:30:00', 'DD.MM.YYYY HH24:MI:SS')
  4  WHERE  Nachname LIKE 'Wolf'
  5    AND  Mitarbeiter_ID = 98;

MITARBEITER_ID VORNAME                        NACHNAME
-------------- ------------------------------ ---------------------------
            98 Louis                          Wolf
      \end{lstlisting}
      In diesem Beispiel wird die \languageorasql{AS OF}-Klausel zusammen mit einem Zeitpunkt benutzt. Es sind aber noch andere Angaben, wie z. B. Restore Points und SCNs m\"oglich.
      \begin{merke}
        Die \languageorasql{AS OF}-Klausel kann an jedes beliebige \languageorasql{SELECT}-Statement angeh\"angt werden.
      \end{merke}
      \begin{literaturinternet}
        \item \cite{ADFNS01003}
      \end{literaturinternet}
    \section{Oracle Flashback Version Query}
      Mit der Oracle Flashback Version Query k\"onnen verschiedene Versionen einer oder mehrerer Tabellenzeilen angezeigt werden. Diese Technologie eignet sich hervorragend, um \"Anderungen an einer Tabellenzeile mitzuverfolgen. In \beispiel{admin1701} werden die Kontobewegungen am Konto 447 zur\"uckverfolgt.
      \begin{lstlisting}[caption={Verschiedene Versionen einer Zeile mit Flashback Version Query},label=admin1701,language=oracle_sql]
SQL> SELECT Konto_ID, Guthaben
  2  FROM   bank.girokonto VERSIONS BETWEEN TIMESTAMP
  3         TO_TIMESTAMP('03.11.2013 10:55:00', 'DD.MM.YYYY HH:MI:SS')
  4    AND  TO_TIMESTAMP('03.11.2013 11:10:40', 'DD.MM.YYYY HH:MI:SS')
  5  WHERE  Konto_ID = 447;

  KONTO_ID   GUTHABEN
---------- ----------
       447     1231,4
       447     1182,5
      \end{lstlisting}
      Die Klausel \languageorasql{VERSIONS BETWEEN} macht aus einer normalen Abfrage eine Flashback Version Query. Das Schl\"usselwort \languageorasql{TIMESTAMP} kann, genau wie bei der \languageorasql{AS OF}-Klausel, durch das Schl\"usselwort \languageorasql{SCN} ersetzt werden.

      Bei einer Flashback Version Query k\"onnen mit Hilfe von Pseudospalten noch weitere Informationen, \"uber eine Zeile abgefragt werden. \tabelle{flashtable1} zeigt die Pseudospalten, die bei einer Flashback Version Query zur Verf\"ugung stehen.
      \begin{center}
        \tablecaption{Pseudospalten bei der Flashback Version Query}
        \tablefirsthead{%
          \hline
          \multicolumn{1}{|c|}{\textbf{Pseudospalte}} &
          \multicolumn{1}{|c|}{\textbf{Beschreibung}}
          \\
          \hline
        }
        \tablehead{%
        }
        \tabletail{%
          \hline
        }
        \begin{supertabular}[h]{|p{5cm}|p{9,5cm}|}
          \label{flashtable1}
          \raggedright VERSIONS\_STARTSCN VERSIONS\_STARTTIME & \footnotesize Diese beiden Spalten zeigen die SCN oder den Zeitpunkt, an dem der erste Datenblock der betreffenden Zeile durch eine DML-Operation ver\"andert wurde. Diese Information kann f\"ur das Wiederherstellen einer Zeile mit Hilfe einer Flashback Query benutzt werden. \\
          \hline
          \raggedright VERSIONS\_ENDSCN VERSIONS\_ENDTIME & \footnotesize Zeigt die SCN oder den Zeitpunkt, an dem die noch aktuelle Zeile durch die neue Version vollst\"andig ersetzt worden ist. \\
          \hline
          VERSIONS\_XID & \footnotesize Idenifiziert die Transaktion, die die Zeile ver\"andert hat \\
          \hline
          VERSIONS\_OPERATION & \footnotesize Zeigt die Art der DML-Operation an, die auf der Zeile ausgef\"uhrt wurde (I f\"ur  INSERT, U f\"ur UPDATE und D f\"ur DELETE)\\
        \end{supertabular}
      \end{center}
      \begin{lstlisting}[caption={Informationsgewinnung mit Flashback Version Query},label=admin1702,language=oracle_sql,alsolanguage=sqlplus]
SQL> col VERSIONS_STARTTIME format a18 
SQL> col VERSIONS_ENDTIME format a18

SQL> SELECT Konto_ID, Guthaben, VERSIONS_STARTTIME, VERSIONS_ENDTIME,
  2         VERSIONS_XID, VERSIONS_OPERATION
  2  FROM   bank.girokonto VERSIONS BETWEEN TIMESTAMP
  3         TO_TIMESTAMP('03.11.2013 10:55:00', 'DD.MM.YYYY HH:MI:SS')
  4    AND  TO_TIMESTAMP('03.11.2013 11:10:40', 'DD.MM.YYYY HH:MI:SS')
  5  WHERE  Konto_ID = 447;

  KONTO_ID   GUTHABEN VERSIONS_STARTTIME VERSIONS_ENDTIME   VERSIONS_XID     V
---------- ---------- ------------------ ------------------ ---------------- -
       447     1231,4 03.11.13 11:09:52                     0900190059040000 U
       447     1182,5                    03.11.13 11:09:52
      \end{lstlisting}
      \begin{literaturinternet}
        \item \cite{ADFNS01004}
      \end{literaturinternet}
    \section{Oracle Flashback Transaction}
      W\"ahrend mit Flashback Query die Vergangenheit einer Tabellenzeile und mit Flashback Version Query die Bearbeitungshistorie einer Zeile angezeigt werden kann, erm\"oglicht das Flashback Transaction Feature:
        \begin{itemize}
          \item zu sehen, was eine spezifische Transaktion an einer Tabellenzeile gemacht hat.
          \item Bestehende Transaktionen wieder r\"uckg\"angig zu machen.
        \end{itemize}
      \subsection{Flashback Transaction Query}
        \subsubsection{Voraussetzungen}
          Flashback Transaction Query basiert, im Gegensatz zu Flashback Query und Flashback Version Query nicht nur auf den Undo Daten, sondern auch auf Redo Records (Archive Logs). Damit diese Technologie einwandfrei funktionieren kann, muss das Supplemental Logging aktiviert werden. Ohne Supplemental Logging kann Flashback Transaction Query nicht funktionieren.
\clearpage
        \subsubsection{Supplemental Logging}
          Supplemental Logging sorgt daf\"ur, dass bei DML-Operationen zus\"atzliche Informationen in den Redo Logs festgehalten werden. Normalerweise wird bei einem \languageorasql{UPDATE} auf eine Zeile nur deren RowID als identifizierendes Merkmal festgehalten. Bei aktiviertem Supplemental Logging werden alle Spalten aufgezeichnet, die zur Identifikation einer Tabellenzeile benutzt wurden. \beispiel{admin1703} zeigt ein \languageorasql{UPDATE}-Statement auf die Tabelle \identifier{girokonto}.
          \begin{lstlisting}[caption={Supplemental Logging - Demo Schritt 1},label=admin1703,language=oracle_sql]
SQL> UPDATE girokonto
  2  SET    Guthaben = 1256.37
  3  WHERE  Konto_ID = 447;

SQL> COMMIT;
          \end{lstlisting}
          Ohne das Supplemental Logging wird zu dieser Transaktion nur das aufgezeichnet, was in \beispiel{admin1704} gezeigt wird.
          \begin{lstlisting}[caption={Supplemental Logging - Demo Schritt 2 },label=admin1704,language=oracle_sql]
update "BANK"."GIROKONTO"
  set
    "GUTHABEN" = 1256,37
  where
    "GUTHABEN" = 1231,40 and
    ROWID = 'AAASWaAAHAAAADUACG'
          \end{lstlisting}
          Mit eingeschaltetem Supplemental Logging wird zus\"atzlich zu diesen Informationen noch die Prim\"arschl\"usselspalte festgehalten, die zur Identifizierung der Tabellenzeile ben\"otigt wird.
          \begin{lstlisting}[caption={Supplemental Logging - Demo Schritt 3},label=admin1705,language=oracle_sql]
update "BANK"."GIROKONTO"
  set
    "GUTHABEN" = 1256,37
  where
    "KONTO_ID" = 447 and
    "GUTHABEN" = 1231,40 and
    ROWID = 'AAASWaAAHAAAADUACG'
          \end{lstlisting}
        \subsubsection{Supplemental Logging aktivieren}
          \begin{lstlisting}[caption={Supplemental Logging},label=admin1706,language=oracle_sql]
SQL> ALTER DATABASE ADD SUPPLEMENTAL LOG DATA;
SQL> ALTER DATABASE ADD SUPPLEMENTAL LOG DATA (PRIMARY KEY) COLUMNS;
SQL> ALTER DATABASE ADD SUPPLEMENTAL LOG DATA (FOREIGN KEY) COLUMNS;
          \end{lstlisting}
          Das erste Kommando aktiviert das \enquote{Minimal supplemental logging}. Dies ist die Grundlage f\"ur die beiden anderen Varianten. Mit dem zweiten Befehl wird das Supplemental Logging f\"ur Primary Key values aktiviert. Dies ist zwingend erforderlich, um Flashback Transaction nutzen zu k\"onnen.

          Zeile drei enth\"alt einen Befehl, der das Supplemental Logging f\"ur Foreign Keys values aktiviert. Dieser ist nicht in allen F\"allen notwendig und sollte auch nur mit Bedacht genutzt werden, da das Supplemental Logging f\"ur Foreign Key die Redo Logs immens aufbl\"ahen kann.
        \subsubsection{Funktionsweise}
          Oracle Flashback Transaction Query erlaubt es zu sehen, was eine spezifische Transaktion an einer Tabellenzeile gemacht hat. Realisiert wird dieses Feature mit Hilfe der Data Dictionary View \identifier{flashback\_transaction\_query}.

          Um zu erfahren, welche Transaktionen Ver\"anderungen an der Tabelle \identifier{girokonto} vorgenommen haben, muss nur die Tabelle \identifier{flashback\_transaction\_query} abgefragt werden.

          Das Interessante an diesem Feature ist, dass zu jedem SQL-Statement ein Undo-SQL-Statement generiert wird. Mit dessen Hilfe kann die bereits bestehende \"Anderung r\"uckg\"angig gemacht werden.
          \begin{lstlisting}[caption={Die Tabelle \identifier{flashback\_transaction\_query}},label=admin1707,language=oracle_sql,alsolanguage=sqlplus,emph={[9]RAW,NUMBER,DATE,VARCHAR2},emphstyle={[9]\color{black}}]
SQL> desc flashback_transaction_query
 Name                                      Null?    Type
 ----------------------------------------- -------- ----------------------
 XID                                                RAW(8)
 START_SCN                                          NUMBER
 START_TIMESTAMP                                    DATE
 COMMIT_SCN                                         NUMBER
 COMMIT_TIMESTAMP                                   DATE
 LOGON_USER                                         VARCHAR2(30)
 UNDO_CHANGE#                                       NUMBER
 OPERATION                                          VARCHAR2(32)
 TABLE_NAME                                         VARCHAR2(256)
 TABLE_OWNER                                        VARCHAR2(32)
 ROW_ID                                             VARCHAR2(19)
 UNDO_SQL                                           VARCHAR2(4000)
          \end{lstlisting}
\clearpage
          \begin{lstlisting}[caption={Informationsgewinnung mit Flashback Transaction Query},label=admin1708,language=oracle_sql,alsolanguage=sqlplus]
col logon_user format a10
col undo_sql format a50

SQL> SELECT xid, start_timestamp, commit_scn, logon_user
  2         undo_sql
  3  FROM   flashback_transaction_query
  4  WHERE  LOWER(table_name) LIKE 'girokonto';

XID              START_TIMESTAMP     COMMIT_SCN LOGON_USER
---------------- ------------------- ---------- ----------
UNDO_SQL
--------------------------------------------------
0A00130036030000 03.11.2013 14:13:09    2278483 BANK
update "BANK"."GIROKONTO" set "GUTHABEN" = '1231,40'
1' where ROWID = 'AAASWaAAHAAAADUACG';

          \end{lstlisting}

          \begin{literaturinternet}
            \item \cite{ADFNS01005}
          \end{literaturinternet}
      \subsection{Flashback Transaction Backout}
        Flashback Transaction Backout ist eine Weiterentwicklung der Flashback Transaction Query. Mit Flashback Transaction Backout k\"onnen Transaktionen r\"uckg\"angig gemacht werden, \"ahnlich wie mit Flashback Transaction Query. Der Unterschied zwischen beiden ist, dass Flashback Transaction Backout in Form einer PL/SQL-Prozedur existiert, die Abh\"angigkeiten zwischen Transaktionen erkennt.
        \subsubsection{Abh\"angigkeiten zwischen Transaktionen}
          Die gr\"o\ss{}te Gefahr beim Zur\"uckrollen bereits best\"atigter Transaktionen ist, dass zwei Transaktionen von einander abh\"angig sein k\"onnen. Beim R\"uckg\"angigmachen einer Transaktion kann eine Zweite in Mitleidenschaft gezogen werden. Es muss in jedem Falle gepr\"uft werden, ob das Ergebnis noch korrekte Daten wiedergibt.

          Eine Transaktion B kann von einer Transaktion A auf drei
          unterschiedlichen Wegen abh\"angig sein:
\clearpage
          \begin{itemize}
            \item \textbf{Write-After-Write}: Bei einer solchen Abh\"angigkeit f\"ugt Transaktion A eine Zeile in eine Tabelle ein, die sp\"ater von Transaktion B ge\"andert wird.
              \begin{lstlisting}[caption={Eine Write-After-Write Abh\"angigkeit},label=admin1709,language=oracle_sql]
-- Transaktion A
SQL> UPDATE TABLE girokonto
  2  SET    Guthaben = 1256.80
  3  WHERE  Konto_ID = 447;
SQL> COMMIT;

-- Transaktion B
SQL> UPDATE TABLE girokonto
  2  SET    Guthaben = 1481.16
  3  WHERE  Konto_ID = 447;
SQL> COMMIT;
              \end{lstlisting}
            \item \textbf{Primary Key Abh\"angigkeit}: Transaktion A l\"oscht eine Zeile aus einer Tabelle. Der Wert des Prim\"arschl\"ussels dieser Zeile war \enquote{n}. Anschlie\ss{}end f\"ugt Transaktion B eine Zeile in die Tabelle ein. Der Wert des Prim\"arschl\"ussels der neuen Zeile ist ebenfalls \enquote{n}. Beide Zeilen haben somit den gleichen Wert als Prim\"arschl\"ussel.
              \begin{lstlisting}[caption={Eine Primary Key Abh\"angigkeit},label=admin1710,language=oracle_sql]
-- Transaktion A
SQL> DELETE girokonto
  2  WHERE  Konto_ID = &\textcolor{red}{447}&;
SQL> COMMIT;
-- Transaktion B
SQL> INSERT INTO girokonto
  2  VALUES (&\textcolor{red}{447}&, 10.5, 1256.37, 10, 15000);
SQL> COMMIT;
              \end{lstlisting}
            \item \textbf{Foreign Key Abh\"angigkeit}: Transaktion A erstellt eine Zeile in Tabelle A. Transaktion B erstellt eine von Tabelle A abh\"angige Zeile in Tabelle B.

            \beispiel{admin1711} zeigt, dass die in Tabelle \identifier{girokonto} erstellte Zeile,  durch ein Foreign Key Constraint zwischen den beiden Tabellen \identifier{konto} und \identifier{girokonto}, von der Tabelle \identifier{konto} abh\"angig ist.
            \begin{lstlisting}[caption={Eine Foreign Key Abh\"angigkeit},label=admin1711,language=oracle_sql]
-- Transaktion A
SQL> INSERT INTO Konto
  2  VALUES (&\textcolor{red}{666}&, 'DE2355880228095276211', NULL);

SQL> COMMIT;
-- Transaktion B
SQL> INSERT INTO Girokonto
  2  VALUES (&\textcolor{red}{666}&, 10.5, 2400, 10, 15000);

SQL> COMMIT;
              \end{lstlisting}
          \end{itemize}
          Flashback Transaction Backout kann alle drei Abh\"angigkeiten erkennen und den Nutzer mit einer Fehlermeldung warnen, bevor eine Transaktion r\"uckg\"angig gemacht wird.

          Die einzige Angabe, die ben\"otigt wird, um eine Transaktion zur\"uckzurollen, ist die XID der betroffenen Transaktion. Diese kann mit Flashback Version Query oder Flashback Transaction Query ermittelt werden.
          \begin{lstlisting}[caption={Die XID einer Transaktion ermitteln},label=admin1712,language=oracle_sql]
SQL> SELECT xid, start_timestamp, logon_user
  2  FROM   flashback_transaction_query
  3  WHERE  LOWER(table_name) LIKE 'girokonto';

XID              START_TIMESTAMP     LOGON_USER
---------------- ------------------- ----------
02000D0056040000 03.11.2013 14:20:10 BANK
04000D0048030000 03.11.2013 14:00:59 BANK
05001400F6040000 03.11.2013 13:41:55 BANK
05000000F6040000 03.11.2013 13:20:49 BANK
0600020049040000 03.11.2013 14:05:25 BANK
0800020049040000 03.11.2013 14:00:41 BANK
0800210048040000 03.11.2013 13:21:22 BANK
090020005B040000 03.11.2013 11:09:52 SYS
0900190059040000 03.11.2013 10:56:50 SYS
0A00130036030000 03.11.2013 14:13:09 BANK

10 rows selected.
          \end{lstlisting}
          Flashback Transaction Backout kann eine oder mehrere Transaktionen gleichzeitig zur\"uckrollen. \beispiel{admin1713} zeigt die Anwendung der Prozedur \identifier{transaction\_backout}.
          \begin{lstlisting}[caption={Eine best\"atigte Transaktion zur\"uckrollen},label=admin1713,language=plsql]
SQL> BEGIN
  2  DBMS_FLASHBACK.TRANSACTION_BACKOUT (
  3    numberofxids => 1,
  4    xids         => xid_array('02000D0056040000'),
  5    options      => DBMS_FLASHBACK.NOCASCADE
  6  );
  7  END;
  8  /
          \end{lstlisting}
          Die Prozedur \identifier{transaction\_backout} wird in \beispiel{admin1713} mit drei Parametern aufgerufen:
          \begin{itemize}
            \item \textbf{numberofxids}: Die Anzahl der Transaktionen die zur\"uckgerollt werden sollen.
            \item \textbf{xids}: Ein Array mit den XIDs aller zur\"uckzurollenden Transaktionen
            \item \textbf{options}: Optionen die das Verhalten von \identifier{transaction\_backout} steuern.
          \end{itemize}
\clearpage
          Es gibt folgende Optionen:
          \begin{itemize}
            \item \textbf{NOCASCADE}: Der Vorgang wird beim Auftreten der ersten Abh\"angigkeit sofort abgebrochen.
            \item \textbf{NOCASCADE\_FORCE}: Trotz Abh\"angigkeiten wird die Transaktion r\"uckg\"angig gemacht. Sollte dabei kein Constraint verletzt werden, kann der Vorgang erfolgreich abgeschlossen werden.
            \item \textbf{NONCONFLICT\_ONLY}: Es werden nur die Transaktionen zur\"uckgerollt, die keine Abh\"angigkeiten aufweisen. Die anderen bleiben bestehen. Dies l\"ost kein Problem mit der Datenbankkonsistenz aus, jedoch muss sehr genau gepr\"uft werden, ob das Ergebnis den eigenen Vorstellungen entspricht.
            \item \textbf{CASCADE}: Mit dieser Option werden alle betroffenen Transaktion der Reihe nach zu\"urckgerollt.
          \end{itemize}
        \subsubsection{Einschr\"ankungen}
          Flashback Transaction Backout hat folgende Einschr\"ankungen:
          \begin{itemize}
            \item DDL Operationen k\"onnen nicht r\"uckg\"angig gemacht werden.
            \item Transaktionen mit LOB Typen (CLOB, BLOB, BFILE, NCLOB) k\"onnen nicht r\"uckg\"angig gemacht werden.
            \item Die Undo Segmente im Undo Tablespace m\"ussen gro\ss{} genug sein.
          \end{itemize}

          \begin{literaturinternet}
            \item \cite{dflashbhtm}
            \item \cite{ADFNS01009}
          \end{literaturinternet}
    \section{Oracle Flashback Table}
      Oracle Flashback Table erm\"oglicht es dem DBA eine Tabelle, ohne kompliziertes Recovery, in einen durch einen Zeitpunkt oder eine SCN bestimmten Zustand zur\"uckzuversetzen. Flashback Table ersetzt dadurch meist ein wesentlich komplizierteres Point-In-Time Recovery.

      Folgende Voraussetzungen m\"ussen f\"ur den Einsatz von Flashback Table geschaffen werden:
\clearpage
      \begin{itemize}
        \item Einschalten von \textit{Row Movement} f\"ur die betreffenden Tabellen.
        \begin{lstlisting}[caption={Einschalten von Row Movement},label=admin1714,language=oracle_sql]
SQL> ALTER TABLE mitarbeiter ENABLE ROW MOVEMENT
        \end{lstlisting}
        \item Der Nutzer muss das Systemprivileg \privileg{flashback any table} haben.
        \item Der Nutzer muss die Objektprivilegien \privileg{select}, \privileg{insert}, \privileg{delete} und \privileg{alter} auf die betreffenden Tabellen haben.
        \item Es m\"ussen gen\"ugend Undo-Informationen verf\"ugbar sein.
      \end{itemize}
      Die beiden folgenden Beispiele zeigen, wie die Tabelle \identifier{mitarbeiter}, mit Hilfe einer SCN bzw. eines Zeitstempels in einen vergangenen Zustand zur\"uckversetzt wird.
      \begin{lstlisting}[caption={Flashback Table mit SCN},label=admin1715,language=oracle_sql]
SQL> FLASHBACK TABLE mitarbeiter TO SCN 1000;
      \end{lstlisting}
      \begin{lstlisting}[caption={Flashback Table mit Zeitstempel},label=admin1716,language=oracle_sql]
SQL> FLASHBACK TABLE mitarbeiter TO TIMESTAMP
  2            TO_TIMESTAMP('03.11.2013 16:43:25', 'DD.MM.YYYY HH:MI:SS');
      \end{lstlisting}
    \section{Oracle Flashback Drop}
      Mit Hilfe von Flashback Drop kann das L\"oschen einer Tabelle r\"uckg\"angig gemacht werden. Flashback Drop ist schneller als jede Form von Point-In-Time Recovery, denn seit Oracle 10g wird eine Tabelle beim L\"oschen nicht mehr sofort entfernt, sondern:
      \begin{enumerate}
        \item Die Tabelle wird umbenannt und erh\"alt einen Recycle Bin Namen, wie zum Beispiel  \languageorasql{BIN$6kkdfoAiGljgQGRkFBQQiQ==$0}.
        \item Sie wird in einen Speicherbereich verschoben, der sich \enquote{Recycle Bin} nennt.
      \end{enumerate}
      Der Flashback Drop macht diese beiden Schritte wieder r\"uckg\"angig.
      \subsection{Der Recycle Bin}
        Der Recycle Bin ist ein logischer Kontainer f\"ur Tabellen und deren abh\"angige Objekte. Wird eine Tabelle gel\"oscht, speichert die Datenbank diese dann mit all ihren abh\"angigen Objekten im Recycle Bin, so dass die Tabelle sp\"ater wiederhergestellt werden kann.
        \subsubsection{Funktionsweise des Recycle Bin}
          Tabellen erhalten einen neuen Namen und werden solange im Recycle Bin gespeichert, bis dieser bereinigt wird. Die Namensgebung im Recycle Bin folgt diesem Schema:

          \verb+BIN$globalUID$version+

          \enquote{globalUID} steht dabei f\"ur eine Zeichenkette aus 24 Zeichen, die in der ganzen Datenbank eindeutig ist. \enquote{version} ist die Versionsnummer (laufende Nummer) der Tabelle. Die Namen im Recycle Bin sind immer 30 Zeichen lang.

          Die Bereinigung des Recycle Bin geschieht automatisch, kann aber auch manuell durchgef\"uhrt werden. Der Recycle Bin wird in den folgenden Situationen automatisch bereinigt:
          \begin{itemize}
            \item wenn der Tablespace, in dem sich das betreffende Objekt befindet keinen freien Speicher mehr hat und wachsen m\"usste und
            \item wenn die Quota, die der Besitzer des Objekts auf den Tablespace hat, in dem sich das Objekt befindet, ausgereizt ist
          \end{itemize}
          Die Objekte werden dann nach dem First-In-First-Out Prinzip endg\"ultig gel\"oscht.

          Soll der Recycle Bin manuell entleert werden, muss das SQL-Kommando \languageorasql{PURGE} verwendet werden. Es wird auf Objekte angewandt, die sich bereits im Recycle Bin befinden. F\"ur seine Ausf\"uhrung kann der Objektname oder auch der Recycle Bin Name genutzt werden.
          \begin{lstlisting}[caption={Das Kommando \languageorasql{PURGE}},label=admin1717,language=oracle_sql]
-- Tabellen entfernen
SQL> PURGE TABLE mitarbeiter;

-- Oder
SQL> PURGE TABLE "BIN$6kkdfoAiGljgQGRkFBQQiQ==$0";

-- Indizes entfernen
SQL> PURGE INDEX "BIN$4fkefoBjGljgQGRkFBQQiQ==$0";

-- Tablespaces entfernen
SQL> PURGE TABLESPACE example;
          \end{lstlisting}
          Will ein Nutzer alle Objekte, die er gel\"oscht hat, aus dem Recycle Bin entfernen, kann er dies mit dem folgenden Kommando tun:
          \begin{lstlisting}[caption={Das Kommando \languageorasql{PURGE RECYCLEBIN}},label=admin1718,language=oracle_sql]
SQL> PURGE RECYCLEBIN;
          \end{lstlisting}
          F\"ur den DBA gibt es zus\"atzlich das Kommando \languageorasql{PURGE DBA\_RECYCLEBIN}, mit dem alle Objekte aus dem Recycle Bin, unabh\"angig von deren Besitzer, gel\"oscht werden.
          \begin{lstlisting}[caption={Das Kommando \languageorasql{PURGE DBA_RECYCLEBIN}},label=admin1719,language=oracle_sql]
SQL> PURGE DBA_RECYCLEBIN;
          \end{lstlisting}
          Um eine Tabelle so zu l\"oschen, das sie erst gar nicht in den Recycle Bin verschoben wird, kann das \languageorasql{DROP TABLE}-Statement um das Schl\"usselwort \languageorasql{PURGE} erweitert werden.
          \begin{lstlisting}[caption={\languageorasql{DROP TABLE PURGE}},label=admin1720,language=oracle_sql]
SQL> DROP TABLE employees PURGE;
          \end{lstlisting}
        \subsubsection{Privilegien f\"ur die Nutzung von Flashback Drop}
          F\"ur Flashback Drop und Purge Operationen gelten die folgenden Regeln:
          \begin{itemize}
            \item Ein Nutzer, der das \privileg{drop}-Privileg auf eine Tabelle hat, kann diese L\"oschen und somit im Recycle Bin platzieren.
            \item Will ein Nutzer eine Flashback Drop-Operation ausf\"uhren, ben\"otigt er das \privileg{drop}-Privileg f\"ur das betreffende Objekt.
            \item Soll das Objekt endg\"ultig mit \languageorasql{PURGE} entfernt werden, ben\"otigt der Nutzer dazu ebenfalls nur das \privileg{drop}-Privileg f\"ur das betreffende Objekt.
          \end{itemize}
        \subsubsection{Den Inhalt des Recycle Bin anzeigen}
          Um sich den Inhalt des Recycle Bin anzeigen zu lassen, k\"onnen entweder die beiden Views \identifier{user\_recyclebin}, \identifier{dba\_recyclebin} oder das SQL*Plus-Kommando \languagesqlplus{show recyclebin} verwendet werden. Die Ausgabe des \languagesqlplus{show recyclebin}-Kommandos sieht so aus:
          \begin{lstlisting}[caption={\languagesqlplus{show recyclebin}},label=admin1721,language=oracle_sql,emph={[9]RECYCLEBIN,TYPE,TIME,DROP,TABLE},emphstyle={[9]\color{black}}]
ORIGINAL NAME RECYCLEBIN NAME                  TYPE   DROP TIME
------------- -------------------------------- ----- -------------------
MITARBEITER   BIN$6kkdfoAiGljgQGRkFBQQiQ==$0   TABLE 2013-03-11 17:16:59
          \end{lstlisting}
          Zus\"atzlich zu \identifier{user\_recyclebin} und \identifier{dba\_recyclebin}, k\"onnen die Data Dictionary Views \identifier{user\_tables}, \identifier{all\_tables}, \identifier{dba\_tables}, \identifier{user\_indexes}, \identifier{all\_indexes} und \identifier{dba\_indexes} dar\"uber Auskunft geben, welche Objekte gel\"oscht wurden. In allen diesen Views existiert eine Spalte \identifier{dropped}. Wurde ein Objekt gel\"oscht, taucht es nach wie vor in diesen Views auf, nur die Spalte \identifier{dropped} wird auf YES gesetzt.
\clearpage
        \subsubsection{Den Recycle Bin ein- und ausschalten}
          Standardm\"a\ss ig ist der Recycle Bin eingeschaltet. \"Uber den Initialisierungsparameter \parameter{recyclebin} kann er ein- und ausgeschaltet werden. Der Parameter nimmt die beiden Werte \enquote{ON} und \enquote{OFF} entgegen. \parameter{recyclebin} kann sowohl Sessionweit mit \languageorasql{ALTER SESSION} als auch Systemweit mit \languageorasql{ALTER SYSTEM} ge\"andert werden.
          \begin{lstlisting}[caption={\languageorasql{Den Recycle Bin ausschalten}},label=admin1722,language=oracle_sql]
SQL> ALTER SYSTEM
  2  SET recyclebin=off;
          \end{lstlisting}
          Die Auswirkungen der \"Anderung werden erst nach einer Neuanmeldung f\"ur die Nutzer sp\"urbar.
        \subsubsection{Regeln und Einschr\"ankungen f\"ur die Nutzung des Recycle Bin}
          \begin{itemize}
            \item Die Flashback Drop-Funktionalit\"at ist nur f\"ur Objekte in einem Nicht-System und lokal verwalteten Tablespace verf\"ugbar.
            \item Es kann nicht bestimmt werden, wie lange sich ein Objekt im Recycle Bin befinden soll.
            \item Abfragen auf Objekte im Recycle Bin sind erlaubt, DML und DDL Operationen aber nicht.
            \item Flashback Queries auf Objekte im Recycle Bin k\"onnen nur mit deren Recycle-Bin-Namen ausgef\"uhrt werden.
            \item Wird eine Tabelle in den Recycle Bin verschoben, werden automatisch alle von ihr abh\"angigen Objekte mit verschoben.
            \item Fremdschl\"ussel gehen beim Verschieben eines Objekts in den Recycle Bin verloren und werden beim Wiederherstellen des Objekts nicht wiederhergestellt.
          \end{itemize}
      \subsection{Einen Flashback Drop ausf\"uhren}
        Ein Flashback Drop kann sowohl mit dem Originalnamen, als auch mit dem Recycle Bin Namen des Objekts ausgef\"uhrt werden. Das folgende Beispiel zeigt diese beiden M\"oglichkeiten f\"ur die Tabelle \identifier{mitarbeiter}.
\clearpage
        \begin{lstlisting}[caption={Ausf\"uhren eines Flashback Drop},label=admin1723,language=oracle_sql]
SQL> DROP TABLE mitarbeiter;

SQL> FLASHBACK TABLE mitarbeiter TO BEFORE DROP;

SQL> FLASHBACK TABLE "BIN$gk3lsj/3akk5hg3j2lkl5j3d==$0" TO BEFORE DROP;
        \end{lstlisting}
        Ein im Recycle Bin befindliches Objekt kann bei seiner Wiederherstellung direkt umbenannt werden. Hierzu wird das \languageorasql{FLASHBACK TABLE ... TO BEFORE DROP}-Statement um die Klausel \languageorasql{RENAME TO} wie folgt erweitert:
        \begin{lstlisting}[caption={Ausf\"uhren eines Flashback Drop mit gleichzeitiger Umbenennung},label=admin1724,language=oracle_sql]
SQL> FLASHBACK TABLE mitarbeiter TO BEFORE DROP
  2  RENAME TO bank.angestellte;

SQL> FLASHBACK TABLE "BIN$gk3lsj/3akk5hg3j2lkl5j3d==$0" TO BEFORE DROP
  2  RENAME TO bank.angestellte;
        \end{lstlisting}
    \section{Oracle Total Recall}
      Immer mehr Unternehmen haben das Problem, dass eine immer gr\"o\ss{}er werdende Datenflut f\"ur sehr lange Zeit vorgehalten werden muss. Teils geschieht dies aufgrund von gesetzlichen Vorgaben (z. B. 10 Jahre im Bereich Gewerbesteuer), teils ist dies notwendig, um Projekte bis in ihre Anf\"ange zur\"uckverfolgen zu k\"onnen.

      Je gr\"o\ss{}er die Datenmengen, desto h\"oher sind auch die Anforderungen an die Datenbank, Abfragen performant bearbeiten zu k\"onnen. Um diesem Problem wirkungsvoll entgegentreten zu k\"onnen, hat Oracle, in der Version 11g R2 seiner Datenbank, die Option \enquote{Total Recall} eingef\"uhrt. Diese Option enth\"alt das Feature \enquote{Flashback Data Archive}, mit dessen Hilfe historische Daten au\ss{}erhalb der Datenbank gelagert werden k\"onnen, aber mittels Flashback Query im direkten Zugriff der Anwendung bleiben.
      \subsection{Flashback Data Archive (FBDA)- Architektur}
        FBDA ist eine optionale und f\"ur Anwendungen v\"ollig transparente Technologie. Es speichert Daten in \enquote{Historientabellen}, die in eigenen Tablespaces angelegt werden. Um Speicheplatz zu sparen und die Performance von Abfragen zu erh\"ohen, sind Historientabellen partitioniert und komprimiert.
\clearpage
        Bef\"ullt werden die Historientabellen durch einen neuen Hintergrundprozess, den \enquote{FBDA} - Flashback Data Archiver. Dieser Prozess erwacht in bestimmten Zeitabst\"anden und durchforstet den Undo Tablespace nach Beforeimages, die archiviert werden m\"ussen. Seine \enquote{Schlafenszeit} justiert der FBDA automatisch nach dem Transaktionsaufkommen (Standardzeitinterval = 5 Minuten).
        \bild{Flashback Data Archive - Architektur}{fbda_architecture}{1}
        \begin{merke}
          Nur die beiden DML-Befehl \languageorasql{UPDATE} und \languageorasql{DELETE} beeinflu\ss{}en das Flashback Data Archive, da \languageorasql{INSERT}-Kommandos keine Beforeimages erzeugen!
        \end{merke}
        \bild{DML und Flashback Data Archive}{fbda_dml}{1}
      \subsection{Flashback Data Archive administrieren}
        \subsubsection{Voraussetzungen}
          Um Flashback Data Archive einrichten zu k\"onnen, m\"ussen die folgenden Voraussetzungen gegeben sein:
          \begin{itemize}
            \item Der Tablespace f\"ur das Flashback Data Archive muss Automatic Segment Space Management nutzen
            \item Das Automatic Undo Management muss aktiviert sein.
          \end{itemize}
        \subsubsection{Privilegien}
          Es gibt drei Privilegien, welche im Zusammenhang mit der Arbeit mit Flashback Data Archive stehen:
          \begin{itemize}
            \item \privileg{flashback archive administer}: Erlaubt das Erstellen und Verwalten von FBDAs.
            \item \privileg{flashback archive}: Erm\"oglicht das Einschalten FBDA f\"ur eine Tabelle.
            \item \privileg{sysdba}: Beinhaltet \privileg{flashback archive administer} und \privileg{flashback archive}
          \end{itemize}
        \subsubsection{Archive erstellen}
          Das Erstellen eines Archives geschieht mit dem Kommando \languageorasql{CREATE FLASHBACK ARCHIVE}. F\"ur die Archive sollte ein eigenst\"andiger Tablespace erstellt werden.
          \begin{lstlisting}[caption={Ein Flashback Data Archive anlegen},label=admin1725,language=oracle_sql]
SQL> CREATE FLASHBACK ARCHIVE fbda_bank_1
  2  TABLESPACE bank
  3  QUOTA      10G
  4  RETENTION  5 YEAR;
          \end{lstlisting}
          Die Klausel \languageorasql{TABLESPACE bank} legt fest, dass das Archiv \identifier{fbda\_bank\_1} f\"ur den Tablespace \identifier{bank} angelegt wird. Mit \languageorasql{QUOTA 10G} wird die Speicherplatzquota f\"ur das Archiv angegeben. Diese wirkt sich genauso aus, wie die \languageorasql{MAXSIZE}-Klausel bei Tablespaces. Mit \languageorasql{RETENTION 5 YEAR} wird die Verweildauer der Daten im FBDA bestimmt. Daten die \"alter sind, als die angegebene Retention werden automatisch aus dem Archiv entfernt.
          \begin{merke}
            Wird die \languageorasql{QUOTA}-Klausel nicht angegeben gilt automatisch \languageorasql{QUOTA unlimited}!
          \end{merke}
          Beim Anlegen eines FBDA ist es m\"oglich, mit Hilfe des Schl\"usselwortes \languageorasql{DEFAULT} ein Standard Flashback Data Archive zu erzeugen. Wann dies von Nutzen ist, wird in \ref{enablefbda} erl\"autert.
          \begin{lstlisting}[caption={Ein Default Flashback Data Archive anlegen},label=admin1726,language=oracle_sql]
SQL> CREATE FLASHBACK ARCHIVE DEFAULT fbda_bank_def
  2  TABLESPACE archive
  3  QUOTA      10G
  4  RETENTION  5 YEAR;
          \end{lstlisting}
          Das ein Flashback Data Archive angelegt wurde, kann mittels der Data Dictionary View \identifier{dba\_flashback\_archive} gepr\"uft werden.
          \begin{lstlisting}[caption={\identifier{dba\_flashback\_archive} abfragen},label=admin1727,language=oracle_sql,alsolanguage=sqlplus]
SQL> col flashback_archive_name format a30 
SQL> SELECT flashback_archive_name, retention_in_days, status
  2  FROM   dba_flashback_archive

FLASHBACK_ARCHIVE_NAME         RETENTION_IN_DAYS STATUS
------------------------------ ----------------- -------
FBDA_BANK_1                                 1825
FBDA_BANK_DEF                               1825 DEFAULT
          \end{lstlisting}
        \subsubsection{Flashback Data Archive aktivieren}
        \label{enablefbda}
          Das Aktivieren von FBDA erfolgt auf Tabellenebene.
          \begin{lstlisting}[caption={Flashback Data Archive aktivieren},label=admin1728,language=oracle_sql]
SQL> ALTER TABLE bank.buchung
  2  FLASHBACK ARCHIVE fbda_bank_1;
          \end{lstlisting}
          Die Tabelle \identifier{buchung} ist nun mit dem Flashback Data Archive \identifier{fbda\_bank\_1} verkn\"upft. F\"ur sie wird eine Historientabelle im Archiv angelegt.
          \begin{lstlisting}[caption={Wo ist die Historientabelle?},label=admin1729,language=oracle_sql,alsolanguage=sqlplus]
SQL> col table_name format a20
SQL> col flashback_archive_name format a20
SQL> col archive_table_name format a20

SQL> SELECT table_name, flashback_archive_name, archive_table_name, status
  2  FROM   dba_flashback_archive_tables;

TABLE_NAME           FLASHBACK_ARCHIVE_NA ARCHIVE_TABLE_NAME   STATUS
-------------------- -------------------- -------------------- --------
BUCHUNG              FBDA_BANK_1          SYS_FBA_HIST_75155   ENABLED
          \end{lstlisting}
          Die View \identifier{dba\_flashback\_archive\_tables} enth\"alt f\"ur jede Tabelle, f\"ur die FBDA aktiviert wurde, einen Eintrag. Die Historientabelle f\"ur \identifier{bank.buchung} wurde automatisch, unter dem Namen \identifier{SYS\_FBA\_HIST\_75155} angelegt.
          \begin{merke}
            Die Historientabelle wird erst dann erzeugt, wenn der FBDA-Hintergrundprozess anf\"angt, seine Arbeit zu machen!
          \end{merke}
          Die interne Struktur der Historientabelle sieht so aus:
          \begin{lstlisting}[caption={Die Struktur der Historientabelle},label=admin1730,emph={[9]VARCHAR2,NUMBER,RAW,DATE},emphstyle={[9]\color{black}},language=oracle_sql,alsolanguage=sqlplus]
SQL> desc archive.SYS_FBA_HIST_75155
 Name                                      Null?    Type
 ----------------------------------------- -------- -----------------------
 RID                                                VARCHAR2(4000)
 STARTSCN                                           NUMBER
 ENDSCN                                             NUMBER
 XID                                                RAW(8)
 OPERATION                                          VARCHAR2(1)
 BUCHUNGS_ID                                        NUMBER
 BETRAG                                             NUMBER(12,2)
 BUCHUNGSDATUM                                      DATE
 KONTO_ID                                           NUMBER
 TRANSAKTIONS_ID                                    NUMBER
          \end{lstlisting}
          Zus\"atzlich zu den Spalten der Tabelle \identifier{buchung} werden noch weitere Spalten f\"ur Metadaten mitgef\"uhrt.

          In \beispiel{admin1726} wird Flashback Data Archive, unter Angabe eines Archivs aktiviert. Die Archivangabe kann aber auch entfallen.
          \begin{lstlisting}[caption={Flashback Data Archive mit einem Default Archive aktivieren},label=admin1731,language=oracle_sql]
SQL> ALTER TABLE bank.buchung
  2  FLASHBACK ARCHIVE;
          \end{lstlisting}
          Das Statement in \beispiel{admin1731} aktiviert Flashback Data Archive f\"ur die Tabelle \identifier{buchung}, wenn vorher ein Default Flashback Data Archive angelegt wurde. Existiert kein Default Archive, schl\"agt dieses Statement fehl.
          \begin{lstlisting}[caption={Welche Archive wurden benutzt?},label=admin1732,language=oracle_sql,alsolanguage=sqlplus]
SQL> col table_name format a20
SQL> col flashback_archive_name format a20
SQL> col archive_table_name format a20

SQL> SELECT table_name, flashback_archive_name, archive_table_name, status
  2  FROM   dba_flashback_archive_tables;

TABLE_NAME           FLASHBACK_ARCHIVE_NA ARCHIVE_TABLE_NAME   STATUS
-------------------- -------------------- -------------------- --------
BUCHUNG              FBDA_BANK_DEF        SYS_FBA_HIST_75155   ENABLED
          \end{lstlisting}
\clearpage
        \subsubsection{Abfragen des Flashback Archives}
          Die Benutzung der Flashback Archive geschieht mit Hilfe der \languageorasql{AS OF}- und \languageorasql{VERSIONS BETWEEN}-Klauseln.
          \begin{lstlisting}[caption={Flashback Data Archive mit einem Default Archive aktivieren},label=admin1733,language=oracle_sql]
SQL> SELECT *
  2  FROM   buchung AS OF TIMESTAMP
  3         TO_TIMESTAMP('03.11.2010 09:30:00', 'DD.MM.YYYY HH24:MI:SS')
          \end{lstlisting}
          Sofern das Flashback Archive bereits lange genug existiert, werden die Daten so angezeigt, wie sie zum 03.11.2010, um 09:30 vorlagen.
        \subsubsection{Flashback Archive bearbeiten}
          Flashback Archive k\"onnen im laufenden Betrieb ver\"andert werden. Folgende Ver\"anderungen sind m\"oglich:
          \begin{itemize}
            \item Ver\"andern der Gr\"o\ss{}e
            \item Ein Archive zum Standardarchiv machen
            \item Ver\"andern des Parameters \languageorasql{RETENTION}
            \item Historische Daten l\"oschen
          \end{itemize}
          Die Gr\"o\ss{}e eines Archives kann auf zwei Arten ge\"andert werden. Die Quota an einem Tablespace kann erh\"oht oder ein weiterer Tablespace hinzugef\"ugt werden.
          \begin{lstlisting}[caption={Die Quota eines Archives auf 20GB erh\"ohen},label=admin1734,language=oracle_sql]
SQL> ALTER FLASHBACK ARCHIVE fbda_bank_def
  2  MODIFY TABLESPACE archive QUOTA 20G;
          \end{lstlisting}
          \begin{lstlisting}[caption={Dem Archiv einen weiteren Tablespace hinzuf\"ugen},label=admin1735,language=oracle_sql]
SQL> ALTER FLASHBACK ARCHIVE fbda_bank_def
  2  ADD TABLESPACE archive2 QUOTA 10G;
          \end{lstlisting}
          Verkleinert wird ein Archiv, in dem ihm ein Tablespace entzogen oder die Quotas verringert werden.
          \begin{lstlisting}[caption={Einen Tablespace aus dem Archiv entfernen},label=admin1736,language=oracle_sql]
SQL> ALTER FLASHBACK ARCHIVE fbda_bank_def
  2  REMOVE TABLESPACE archive2;
          \end{lstlisting}
          \begin{lstlisting}[caption={Die Quota eines Tablespaces verringern},label=admin1737,language=oracle_sql]
SQL> ALTER FLASHBACK ARCHIVE fbda_bank_def
  2  ADD TABLESPACE archive2 QUOTA 5G;
          \end{lstlisting}
          \begin{merke}
            Das Entfernen eines Tablespaces aus einem Archiv funktioniert nur, wenn der Speicherplatz im betroffenen Tablespace noch nicht belegt ist.
          \end{merke}
          Mit Hilfe der \languageorasql{SET DEFAULT}-Klausel kann ein Archiv zum Standardarchiv gemacht werden.
          \begin{lstlisting}[caption={Ein Archiv zum Standardarchiv machen},label=admin1738,language=oracle_sql]
SQL> ALTER FLASHBACK ARCHIVE fbda_bank_1
  2  SET DEFAULT;
          \end{lstlisting}
          Um die Retention eines Archives zu ver\"andern, wird die \languageorasql{MODIFY RETENTION}-Klausel verwendet.
          \begin{lstlisting}[caption={Die Vorhaltedauer eines Archives ver\"andern},label=admin1739,language=oracle_sql]
SQL> ALTER FLASHBACK ARCHIVE fbda_bank_1
  2  MODIFY RETENTION 3 YEAR;
          \end{lstlisting}
        \subsubsection{Archive manuell bereinigen}
          Oracle bereinigt den Inhalt der Flashback Archives automatisch. Dennoch hat der Administrator die M\"oglichkeit, den Inhalt manuell zu l\"oschen.
          \begin{lstlisting}[caption={Den gesamten Inhalt eines Archives l\"oschen},label=admin1740,language=oracle_sql]
SQL> ALTER FLASHBACK ARCHIVE fbda_bank_1
  2  PURGE ALL;
          \end{lstlisting}
          \begin{lstlisting}[caption={Alle Eintr\"age l\"oschen, die \"alter als ein Jahr und sechs Monate sind},label=admin1741,language=oracle_sql]
SQL> ALTER FLASHBACK ARCHIVE fbda_bank_1
  2  PURGE BEFORE TIMESTAMP (SYSTIMESTAMP - INTERVAL '1-6' YEAR TO MONTH);
          \end{lstlisting}
        \subsubsection{Flashback Data Archive deaktivieren}
          Flashback Data Archive kann jeder Zeit f\"ur eine Tabelle deaktiviert werden.
          \begin{merke}
            Beim Deaktivieren von Flashback Archive wird die Historientabelle gel\"oscht!
          \end{merke}
          \begin{lstlisting}[caption={Flashback Data Archive deaktivieren},label=admin1742,language=oracle_sql]
SQL> ALTER TABLE bank.buchung
  2  NO FLASHBACK ARCHIVE;
          \end{lstlisting}
        \subsubsection{Flashback Archive l\"oschen}
          Beim L\"oschen eines Flashback Archives wird die Historientabelle gel\"oscht, der Tablespace, in dem sie lag, bleibt erhalten.
          \begin{lstlisting}[caption={Flashback Data Archive deaktivieren},label=admin1743,language=oracle_sql]
SQL> DROP FLASHBACK ARCHIVE fbda_bank_1;
          \end{lstlisting}
    \section{Informationen}
      \subsection{Verzeichnis der relevanten Initialisierungsparameter}
        \begin{literaturinternet}
          \item \cite{REFRN10264}
        \end{literaturinternet}
      \subsection{Verzeichnis der relevanten Data Dictionary Views}
        \begin{literaturinternet}
          \item \cite{ADFNS01005}
          \item \cite{REFRN23342}
          \item \cite{sthref2545}
          \item \cite{sthref2202}
          \item \cite{REFRN23719}
          \item \cite{sthref2010}
        \end{literaturinternet}


	\clearpage
	\appendix
	\part{Appendix}
\printbibliography[heading=tsqlreference,keyword=TSQLReference]
\printbibliography[heading=sqlserveradmin,keyword=SQLServerAdmin]
\end{document}