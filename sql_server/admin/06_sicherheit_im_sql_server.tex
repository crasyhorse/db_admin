  \chapter{Sicherheit im SQL Server}
    \setcounter{page}{1}\kapitelnummer{chapter}
    \minitoc
\newpage
  
    \begin{merke}
      Die in diesem Kapitel genannten Informationen beziehen sich nurauf das
      Absichern einer SQL Server-Instanz. Für andere Softwareprodukte können
      unter Umständen andere bzw. zusätzliche Einstellungen und Maßnahmen
      notwendig sein.
    \end{merke}
    Das Absichern von Anwendungen und Diensten ist eine vielschichtige Aufgabe
    und bedarf genauer Planung, sowie guter Kenntnis der Materie. Grundsätzlich
    sind es vier Ebenen, die es abzusichern gilt:
    \begin{itemize}
        \item Der Serverraum
        \item Das Netzwerk
        \item Das Betriebssystem
        \item Die Anwendung / der Dienst
    \end{itemize}
    Abhängig davon, wie sensibel die zu schützenden Daten sind, muss jede
    dieser Ebene mit entsprechendem Aufwand gesichert werden.
    \section{Sicherheit auf Betriebssystemebene}
      Ein wesentlicher Bestandteil der Sicherheit auf Betriebssystemebene ist
      ein gut durchdachtes Rollen und Rechtekonzept. Jeder Benutzer sollte nur
      auf die Bestandteile des Betriebssystems Zugriff haben, auf die er auch
      Zugriff haben muss. Das beginnt damit, das jeder Benutzer ein eigenes
      Nutzerkonto hat, das mit den notwendigen Berechtigungen ausgestattet ist.
      
      Zu beachten ist hierbei, dass nicht nur \enquote{Benutzer aus Fleisch und
      Blut} Nutzerkonten verwenden. Jeder Windowsdienst wird im Kontext eines
      Benutzerkontos ausgeführt, was bedeutet, dass der Dienst die
      Berechtigungen seines Nutzerkontos besitzt und einsetzen kann.
      \begin{merke}
        Benutzerkonten, die von Windowsdiensten verwendet werden, werden im
        Folgenden als \enquote{Dienstkonten} bezeichnet.
      \end{merke}
      Um unliebsame Zwischenfälle zu vermeiden sollte der Administrator das
      \enquote{Priciple of least privilege} (POLP) beachten. Dieses Prinzip
      besagt, dass jeder Benutzer nur die Rechte eingeräumt bekommt, die er für
      seine Arbeit zwingend braucht.
      \subsection{NTFS-Zugriffsrechte}
        Der Microsoft SQL Server speichert verschiedene Dateiarten in
        unterschiedlichen Verzeichnissen. Bei fast allen dieser Ordner muss nur
        der SQL Server-Dienst selbst Zugriff haben, der Datenbankadministrator
        benötigt dagegen nur in Ausnahmefällen Zugriff.
        \subsubsection{Die Standardverzeichnisse des SQL Server}
          In Kapitel \ref{installation_of_sql_server_2014} wurde bereits das
          Standardinstallationsverzeichnis des SQL Server angesprochen. Es
          befindet sich in \oscommand{C:\textbackslash Program Files\textbackslash Microsoft SQL
          Server}. In diesem Ordner liegen die folgenden Bestandteile:
          \begin{itemize}
            \item die instanzabhängigen Komponenten der Database Engine, der
            Reporting Services und der Analysis Services
            \item die gemeinsam genutzten Komponenten,
            \item das Datenverzeichnis \oscommand{data},
            \item das Protokollverzeichnis \oscommand{data}
            \item und das Sicherungsverzeichnis \oscommand{backup}
          \end{itemize}
          Für all diese Verzeichnisse gilt: Normale Benutzer benötigen keinen
          Zugriff und auch Administratoren müssen dort nur in äußerst seltenen
          Fällen zugreifen. Nur die SQL Server-Dienste müssen Vollzugriff auf
          die dort befindlichen Dateien haben.
          \begin{merke}
            Anmerkung des Autors: Im Unterverzeichnis \oscommand{binn} liegen
            die ausführbaren Dateien für das Management Studio und alle anderen
            Komponenten, die Sie während dieses Lehrgangs benötigen. Aus diesem
            Grund werden Kerbdie Zugriffsrechte für dieses Verzeichnis nicht
            verändert. Im Normalfall hat der Administrator einen eigenen
            \enquote{Adminrechner}, auf dem die Client-Tools, samt Management
            Studio, installiert sind, weshalb auch er dort keinen Zugriff haben
            muss.
          \end{merke}
        \subsubsection{Erstellen einer sinnvollen Verzeichnisstruktur}
          Sofern der SQL Server mit Standardeinstellungen betrieben wird, werden
          neue Datenbanken im Datenverzeichnis \oscommand{data}, unterhalb des
          Installationsordners angelegt. Da dieser Ordner üblicherweise auf dem
          Systemdatenträger \oscommand{C:\textbackslash} liegt, sollten dort
          keine Datenbanken liegen, weil sonst die Gefahr besteht, das dieser
          Datenträger volläuft und das gesamte System blockiert wird. Des
          Weiteren sollten, wie in \ref{movedatabases} bereits erwähnt,
          bestimmte Datenbanken, wie z. B. die \identifier{tempdb}, auch aus
          Performancegründen auf andere Datenträger verteilt werden.
          
          Die Aufgabe des Administrators ist es nun, sich eine sinnvolle
          Verzeichnisstruktur für die Verteilung zu überlegen. Besagte Struktur
          sollte dabei die folgenden Punkte berücksichtigen:
          \begin{itemize}
            \item Anwendungsdaten (Programme und Programmbibliotheken) und
            Datenbanken sollten getrennt von einander liegen.
            \item Es sollte möglich sein, ein \enquote{Inplace Upgrade} der
            Datenbanksoftware durchzuführen. Dazu müssen kurzzeitig zwei
            Versionen des SQL Servers auf einem Server betrieben werden können.
            \item Datenbanken und Protokolldateien sollten auf getrennten
            Datenträgern liegen.
            \item Möglichst einfach durch NTFS-Rechte zu sichernde
            Verzeichnisstruktur.
          \end{itemize}
          Verschiedene Softwareanbieter haben bereits solche Lösungen erarbeitet
          und stellen diese der Allgemeinheit zur Verfügung. Ein Lösungsansatz,
          der all diese Punkte berücksichtigt, ist die \enquote{Oracle Flexible
          Architecture} - kurz OFA. Auch wenn die OFA für die Aufteilung von
          Oracle-Datenbankdateien gedacht ist, ist sie dennoch so allgemein
          gehalten, das sie auch auf andere DBMS, wie z. B. den SQL Server,
          anwendbar ist.
          
          Die OFA-Struktur sieht folgendes vor:
          \begin{itemize}
            \item Ein Verzeichnisbaum beginnt mit einem kurzen, eindeutigen
            Namen, der von keinen äußeren Faktoren, wie z. B. einer
            Hardwarebezeichnung, abhängig ist. Außerdem sollte der
            Verzeichnisname eine laufende Nummer enthalten.
            \item Das erste Unterverzeichnis drückt etwas über seinen Inhalt
            aus, z. B. \oscommand{app} für Applications oder \oscommand{data}
            für Datenbankdateien.
            \item In der zweiten Verzeichnisebene kann die Versionsnummer des
            genutzten DBMS dargestellt werden, z. B. \oscommand{120} für den SQL
            Server 2014.
            \item Die nächste Verzeichnisebene gibt den Namen der Datenbank an,
            z. B. \oscommand{Bank\_2014}
          \end{itemize}
          Beispiele für eine OFA-Kompatible Verzeichnisstruktur könnten somit
          sein:
          
          \oscommand{D:\textbackslash u01\textbackslash data\textbackslash
          120\textbackslash Bank\_2014}
          
          \oscommand{D:\textbackslash u01\textbackslash filestream\textbackslash
          120\textbackslash Bank\_2014}
          
          \oscommand{D:\textbackslash u01\textbackslash log\textbackslash
          120\textbackslash Bank\_2014}
          
          \oscommand{D:\textbackslash u01\textbackslash backup\textbackslash
          120\textbackslash Bank\_2014}
          
        \subsubsection{Festlegen der NTFS-Rechte mit Hilfe der OFA}
          Die einfachste Möglichkeit, die passenden NTFS-Berechtigungen
          innerhalb der OFA zu verteilen besteht darin, den Mechanismus der
          \enquote{Vererbung} auszunutzen. Der Administrator muss lediglich die
          NTFS-Rechte an der obersten Verzeichnisebene, im vorangegangenen
          Beispiel war dies \oscommand{u01}, ändern und diese werden weiter
          nach unten vererbt. Orientiert man sie dabei, wie von Microsoft
          empfohlen, am AGDLP-Schema, so bedeutet dies:
          \begin{enumerate}
            \item Erstellen einer globalen Gruppe im AD, z. B. mit dem Namen
            \identifier{g-mssqlserver120}.
            \item Aufnehmen der gMSAs aller SQL Server-Dienste in die globale
            Gruppe.
            \item Erstellen einer domänenlokalen Gruppe im Active Directory,
            beispielsweise mit dem Namen \identifier{l-mssqlserver120}.
            \item Aufnehmen der globalen Gruppe in die domänenlokale Gruppe.
            \item Deaktivieren der Vererbung für den Ordner
            \oscommand{D:\textbackslash u01}
            \item Zuteilen des NTFS-Rechts \enquote{Vollzugriff} an die
            domänenlokale Gruppe auf das Verzeichnis \oscommand{D:\textbackslash
            u01}.
            \item Aufrufen der Erweiteren Sicherheitseinstellungen und
            ausführen der Funktion \enquote{Alle Berechtigungseinträge für
            untergeordnete Objekte durch vererbbare Berechtigungseinträge von
            diesem Objekt ersetzen}. Dadurch werden alle Unterodner mit den
            korrekten Berechtigungen versehen.
            \item Kontrollieren, dass die Vererbung des NTFS-Rechte für alle
            Unterordner von \oscommand{D:\textbackslash u01} aktiviert ist.
            \item Zuteilen des NTFS-Rechts \enquote{Vollzugriff} an die weiteren
            Ordner \oscommand{u02}, \oscommand{u03}, usw.
          \end{enumerate}    
      \subsection{Die Windows-Firewall}
        Mit Hilfe der Windows-Firewall kann ein weiteres Stück Sicherheit erzeugt
        werden. Sie ermöglicht es, die Kommunikation zwischem dem
        Datenbankserver und seinen Clients sowohl mit statischen als auch mit
        dynamischen Ports zu regeln.
        \begin{merke}
          Da es die unterschiedlichsten Firewall-Produkte auf dem Markt gibt,
          wird die Windows-Firewall hier nur als stellvertretendes Beispiel
          herangezogen, um aufzuzeigen, wie die Netzwerkkommunikation zwischen
          Client und Server gesichert werden kann.
        \end{merke}
        Die Verwaltung der Windows-Firewall erfolgt mit dem MMC-Plugin
        \enquote{Windows-Firewall mit erweiterter Sicherheit}. Dieses kann
        durch die \enquote{Ausführen}-Funktion des Windows Startmenüs mit
        \oscommand{WF.msc} gestartet werden.
        \begin{enumerate}
          \item Klicken Sie im Firewallmenü, links außen, auf die Option
          \enquote{Eingehende Regel}.
          \bild{Das Windows-Firewall MMC-Plugin}{windows_firewall_1}{1.8}
          \item Öffnen Sie mit der rechten Maustaste das Kontextmenü und
          wählen Sie die Option \enquote{Neue Regel}.
          \bild{Neue eingehende Regel erstellen}{windows_firewall_2}{1.2}
          \item Wählen Sie in Schritt eins, des \enquote{Assistenten für
          eingehende Regeln} die Option \enquote{Port}.
          \bild{Eine Regel für Ports erstellen}{windows_firewall_3}{1.3}
          \item Klicken Sie auf \enquote{Weiter}
          \item Wählen Sie in Schritt zwei die Optionen \enquote{TCP} und
          \enquote{Bestimmte lokale Ports} und geben Sie die Portnummer 1433
          an.
          \bild{TCP-Port 1433 ansprechen}{windows_firewall_4}{1.3}
          \item Klicken Sie auf \enquote{Weiter}.
          \item Im Schritt \enquote{Aktion} des Assistenten muss die Option
          \enquote{Verbindung zulassen} aktiviert bleiben.
          \bild{Zulassen der Verbindung}{windows_firewall_5}{1.3}
            \item Klicken Sie auf Weiter
\clearpage
            \item Die eingehende Regel soll für alle drei Profile
            gelten.
            \bild{Anwenden der Regel auf alle
          Netzwerkprofile}{windows_firewall_6}{1}
          \item Vergeben Sie den Namen \enquote{Database Engine} für die
          Policy.
          \bild{Fertigstellen des Assistenten}{windows_firewall_7}{1.3}
        \end{enumerate}
        \begin{literaturinternet}
          \item \cite{ms175043}
        \end{literaturinternet}
    \section{Konfigurieren der Authentifizierung}
      Für die Authentifizierung der Clients kann der SQL Server zwei
      verschiedene Protokolle benutzen: NTLM und Kerberos. NTLM ist die
      Standardauthentifizierungsmethode und wird immer dann genutzt, wenn
      Kerberos nicht konfiguriert wurde oder nicht genutzt werden kann.
      \subsection{Unterschiede zwischen NTLM und Kerberos}
        \subsubsection{NT LAN Manager Authentifizierung (NTLM)}
          Das LAN Manager-Protokoll wurde ursprünglich von Microsoft entwickelt
          und kam lange Zeit auch nur in Microsoftprodukten vor. Mit Hilfe von
          Reverse Engineering hat dieses Protokoll aber auch Einzug in andere
          Softwareprodukte erhalten. Ende der 1990er Jahre erfolgte dann die
          Weiterentwicklung des LAN Manager-Protokolls zum NT LAN
          Manager-Protokoll.
          
          NTLM basiert auf der folgenden Vorgehensweise:
          \begin{enumerate}
            \item Der Client sendet seinen Benutzernamen im Klartext an den
            Server.
            \item Der Server sendet im zweiten Schritt eine Zufallszahl an den
            Client. Diese Zahl wird als \enquote{Challenge} bezeichnet.
            \item Der Client sendet eine mit dem Hash-Wert seines
            Benutzerpasswortes verschlüsselte Anwort, die \enquote{Response}
            zurück.
            \item Der Server verschlüsselt nun seinerseits die Challenge mit dem
            Benutzerpasswort und vergleicht das Ergebnis mit der Response. Bei
            Wertgleichheit gilt der Client als authentifiziert.
          \end{enumerate}
          Die Größte Schwachstelle dieses Protokolls ist die unverschlüsselte
          Übertragung des Benutzernamens. Außerdem konnte es bei der ersten
          Version des NTLM-Protokolls vorkommen, dass lange Passwörter schwächer
          waren als Kurz. Um dieses Manko zu überwinden wurde eigens NTLMv2
          entwickelt.
        \subsubsection{Das Kerberos-Protokoll}
          Das Kerberos-Protokoll hat seinen Ursprung im
          \enquote{Needham-Schroeder-Protokoll}, welches 1978 als Protokoll zur
          Authentifizierung in unsicheren Netzwerken entworfen wurde. Kerberos
          gilt als äußerst sicher, weshalb Microsoft die Nutzung von Kerberos
          als Standard vorschlägt. Die Authentifizierung wird durch einen extra
          Kerberos-Dienst vorgenommen, der als \enquote{vertrauenswürdige dritte
          Partei} (Trusted Third Party) oder auch als Key Distribution Center
          (KDC) bezeichnet wird. In der Welt von Microsoft übernimmt die Rolle
          des KDC der Domain Controller einer Active Directory Domäne.
          
          Die Vorteile von Kerberos gegenüber NTLM sind:
          \begin{itemize}
            \item \textbf{Multilaterale Authentifizierung}: Alle Parteien, der
            Kerberos-Dienst, der Zielserver und der Client authentifizieren sich
            gegenseitig. Dadurch werden Man-In-The-Middle-Attacken
            ausgeschlossen.
            \item \textbf{Sichere Informationsübertragung}: Alle
            Authentifizierungsinformationen werden verschlüsselt übertragen.
            \item \textbf{Single-Sign-On}: Kerberos vermeidet mehrfache
            Anmeldevorgänge durch die Verwendung eines
            Single-Sign-On-Mechanismus.
          \end{itemize}
           Der Mechanismus den Kerberos verwendet basiert auf der Verteilung von
           verschlüsselten Tickets. Diese Tickets stellen für die Clients die
           Berechtigung dar auf bestimmte Dienste (die Services) zuzugreifen.
           Für jeden Service den ein Client benutzen möchte muss er sich ein eigenes
           Ticket holen.
           
           Ein Kerberos-Authentifizierungsvorgang läuft wie folgt ab:
          \begin{enumerate}
            \item Der Client meldet sich beim Kerberos-Server und fordert ein
            \enquote{Ticket Granting Ticket} (TGT) an. Bei einem TGT handelt es
            sich um eine kleine verschlüsselte Datei, welche die IP-Adresse des
            Clients, eine Session-ID und eine Gültigkeitsdauer beinhaltet. Zur
            Anforderung des TGT muss der Client unter Umständen ein Passwort
            angeben.
            \item Der Server prüft, ob sich der Client in seiner
            \enquote{Realm DB} befindet und erteilt dem Client das TGT.
            \item Der Client kann nun ein \enquote{Service Ticket} für einen
            speziellen Dienst beim KDC anfordern. Hierfür sendet er sein TGT
            und den SPN des gewünschten Dienstes an den KDC.
            \item Der KDC erteilt dem Client das Service Ticket
            \item Der Client kann mit Hilfe des Service Tickets nun den
            gewünschten Service benutzen.
          \end{enumerate}
          \bild{Das Kerberos\-verfahren}{kerberos}{2}
          \begin{literaturinternet}
            \item \cite{cc280744}
            \item \cite{cc280745}
            \item \cite{wikipTicketGrantingTicket}
          \end{literaturinternet}
      \subsection{Service Principal Names (SPN)}
        In einer Active Directory Umgebung darf kein Zugriff auf einen
        Dienst, ohne eine vorherige Authentifizierung des Clients erfolgen. Für die
        Authentifizierung benutzt Microsoft Windows das Protokoll \enquote{Kerberos}, und ein
        Bestandteil dieses Protokolls sind die sogenannten \enquote{Service
        Principal Names}, abgekürzt \enquote{SPN}. Ein SPN, zu Deutsch
        \enquote{Dienst Prinzipal Name}, ist vereinfacht ausgedrückt ein Name für
        einem Windows Dienst, mit dem dieser sich im Active Directory registriert.
        \subsubsection{Aufbau eines SPN}
          Ein SPN besteht aus maximal vier Teilen, die durch Slashes ( / ) von
          einander getrennt werden.
        
          \texttt{<service class>/<host>:<port number>/<service name>}
        
          \begin{itemize}
            \item \textbf{service class}: Eine Zeichenfolge, welche die Art des
            Dienstes identifiziert, beispielsweise steht \identifier{MSSQLSvc} für
            den SQL Server Dienst.
            \item \textbf{host}: Abhängig vom jeweiligen Dienst ist dies die
            IP-Adresse, der NetBIOS-Name oder der FQDN\footnote{FQDN = Fully
            Qualified Domain Name} des betreffenden Servers auf dem der Dienst
            bereitgestellt wird. Dieser Teil wird manchmal auch als
            \enquote{Instance Name} bezeichnet.
            \item \textbf{port number}: Eine optional anzugebende Portnummer, mit
            deren Hilfe mehrere Instanzen, des gleichen Dienstes, auf einem Host
            unterschieden werden können.
            \item \textbf{service name}: Ebenfalls eine optionale Angabe, die zur
            Unterscheidung unterschiedlicher Instanzen ein und des selben Dienstes
            auf einem Host benutzt werden kann.
          \end{itemize}
          Der Teil Service Class wird oft auch als \enquote{Service Type}
          bezeichnet. Es existieren eine Reihe sogenannter \enquote{Well-known
          services}, wie z. B. \identifier{www} für Web Services, \identifier{http} für
          Webseiten/-server oder auch \identifier{ldap} für Directory Services.
          Grundsätzlich muss der Service Type aber nicht mehr als eine eindeutige
          Zeichenfolge sein.

          Der SPN der SQL Server Standardinstanz besteht aus zwei bis drei
          Bestandteilen:
          
          \texttt{MSSQLSvc/fea11-119srv20.ms-c-ix-04.fus}
          
          Das Beispiel zeigt einen SPN des Typs SQL Server, der auf dem Host
          \identifier{fea11-119srv20.ms-c-ix-04.fus} bereitgestellt wird. Optional
          kann hier auch eine Portnummer mit aufgenommen werden, falls der SQL
          Server nicht auf seinem Standardport 1433 betrieben wird.
          
          \texttt{MSSQLSvc/fea11-119srv20.ms-c-ix-04.fus:1423}

          Da der SQL Server ein replizierbarer Dienst ist, kann es mehrere
          parallele Instanzen auf einem Host geben. In solch einem Fall ist
          der Service Name von Bedeutung, da er den Namen der benannten SQL
          Server Instanz enthält.
          
          \texttt{MSSQLSvc/fea11-119srv20.ms-c-ix-04.fus/CRM}
          
          \texttt{MSSQLSvc/fea11-119srv20.ms-c-ix-04.fus/ERM}
          
          Beidesmals sind die zwei vorderen Teile gleich, nur der Service
          Name ist unterschiedlich.
          \begin{merke}
            Wichtig ist der Service Name besonders dann, wenn ein
            Verbindungsprotokoll zum Einsatz kommt, dass keine Portnummern kennt,
            wie z. B. Named Pipes.
          \end{merke}
          \begin{literaturinternet}
            \item \cite{ms677601}
            \item \cite{qzaidi20101012}
          \end{literaturinternet}
        \subsubsection{Verbreitung von SPNs - Service Publication}
          Wie eingangs bereits erwähnt, muss ein Dienst seinen SPN im Active
          Directory registrieren, damit Clients zu ihm Kontakt aufnehmen können.
          Der Vorgang während dem der Dienst seinen SPN im AD registriert wird
          als \enquote{Service Publication} bezeichnet.

          Unter normalen Umständen wird ein SPN von einem Setup-Programm erzeugt
          und im Active Directory registriert. Bei SQL Server 2014 werden in einem
          Installationsschritt die Dienstkonten für die verschiedenen Konten
          abgefragt und nach Abschluß des Schrittes im AD registriert.
          \begin{merke}
            Damit die Service Publication funktionieren kann, muss der Benutzer,
            der die Installationsroutine ausführt die Berechtigung besitzen
            Objekte im Active Directory anzulegen.
          \end{merke}
          Wird nach erfolgter Service Publication das Konto eines Dienstes
          geändert, muss die SPN beim alten Dienstkonto gelöscht und mit dem neuen
          registriert werden. Dieser Schritt kann entfallen, wenn MSAs genutzt
          werden, da diese ihre SPN selbst verwalten.
        \subsubsection{Wo werden die SPN-Informationen abgelegt?}
          Ein SPN ist immer mit einem Benutzer- oder Computerkonto verknüpft.
          Welche Kontoart benutzt wird hängt davon ab, welcher Kontotyp als
          Dienstkonto festgelegt wurde.
          \begin{itemize}
            \item \textbf{Virtual Account}: Da ein virtuelles Konto das
            Computerkonto des lokalen Computers benutzt, um auf
            Netzwerkresourcen zuzugreifen, wird in diesem Fall der SPN mit dem
            Computerkonto registriert.
            \item \textbf{Benutzerkonto/MSA}: Hier wird der SPN mit dem
            betreffenden Benutzerkonto registriert. Der einzige Unterschied
            zwischen Benutzerkonto und MSA ist der, dass MSAs die notwendigen
            Berechtigungen haben, um ihre SPNs selbst zu verwalten.
          \end{itemize}
          \begin{merke}
            SPNs werden im LDAP-Attribut \enquote{servicePrincipalName}
            aufgelistet. Dessen Inhalt kann im ADSI-Editor sichtbar gemacht
            werden.
          \end{merke}
          \begin{literaturinternet}
            \item \cite{ms191153}
            \item \cite{cc755804}
          \end{literaturinternet}
      \subsection{Umstellung auf Kerberos}
        Um das Authentifizierungsverfahren auf Kerberos umzustellen muss der
        Administrator lediglich einen einzigen Schritt durchführen:
      
        Dem SQL Server-Dienstkonto muss ein gülter Service Principal Name (SPN)
        zugeordnet werden.

        Die Zuordnung des SPN kann manuell oder automatisch erfolgen.
        \subsubsection{Automatische zuordnung eines SPN}
          Die automatische Zuordnung eines SPN für den SQL Server-Dienst geschieht
          beim Starten des Dienstes. Die Voraussetzung dafür ist, dass sein
          Dienstkonto die Berechtigung hat einen SPN zu registrieren.
          Standardmäßig haben nur die Konten \enquote{Lokale System} und
          \enquote{Netzwerk Dienst} die Berechtigung einen SPN zu registrieren.
          Allen anderen Konten muss dieses Recht erst erteilt werden. Gehen Sie
          hierzu wie folgt vor:
          \begin{enumerate}
            \item Starten Sie das MMC-Plugin \enquote{ADSI-Editor}
            \bild{Starten des ADSI-Editors}{adsi_editor_1}{1.3}
            \item Öffnen Sie per Rechtsklick auf \enquote{ADSI-Editor} das
            Kontextmenü und klicken Sie auf \enquote{Verbindung herstellen}.
            \bild{Herstellen einer Verbindung zur Domäne}{adsi_editor_2}{1.3}
            \item Klicken Sie auf OK um die Verbindung zur Domäne herzustellen.
            \bild{Verbindungs\-einstellungen}{adsi_editor_3}{1.3}
            \item Navigieren Sie im Verzeichnisbaum bis zum gewünschten gMSA.
            \bild{Navigation im Verzeichni}{adsi_editor_4}{1.3}
            \item Öffnen Sie das Kontextmenü des gMSA und klicken Sie auf
            \enquote{Eigenschaften}
            \item Wählen Sie die Registerkarte \enquote{Sicherheit} und klicken
            Sie auf das Benutzerkonto \enquote{SELBST}.
            \bild{Der Eigenschaften\-dialog des gMSA}{adsi_editor_5}{1.3}
            \item Klicken Sie auf \enquote{Erweitert}.
            \item Wählen Sie im Dialog \enquote{Erweiterte
            Sicherheitseinstellungen} den Benutzer \enquote{SELBST} aus. Dieser
            Eintrag existiert u. U. mehrfach. Klicken Sie auf die Schaltfläche
            \enquote{Bearbeiten} sobald Sie den richtigen Benutzer gefunden
            haben.
            \bild{Erweiterte
            Sicherheits\-einstellungen}{adsi_editor_6}{1.3}
            \item Scrollen Sie im Dialog \enquote{Berechtigungseintrag} soweit
            nach unten, bis Sie die beiden Berechtigungen
              \begin{itemize}
                \item ServicePrincipalName lesen
                \item ServicePrincipalName schreiben
              \end{itemize}
            gefunden haben und wählen Sie diese aus.
            \bild{Auswählen der Berechtigungen}{adsi_editor_7}{1.3}
            \item Schließen Sie die Eigenschaftsdialoge und den ADSI-Editor.
            \item Starten Sie den SQL Server-Dienst neu.
          \end{enumerate}
          Im ErrorLog des SQL Server kann nach dem Neustart ersehen werden, ob
          die automatische Zuordnung des SPN funktioniert hat. Es erscheint die
          folgende Meldung:
          
          The SQL Server Network Interface library successfully registered the
          Service Principal Name (SPN) MSSQLSvc/FEA11-119SRVXX.ms-c-ix-04.fus
          for the SQL Server service.
          
          Beim Fehlschlagen der Zuordnung wird die folgende Fehlermeldung
          angezeigt:
          
          The SQL Server Network Interface library could nat register the
          Service Principal Name (SPN) MSSQLSvc/FEA11-119SRVXX.ms-c-ix-04.fus
          for the SQL Server service. Windows return code: 0xXXXXX, state: 15.
          Failure to register a SPN might cause integrated authentication to use
          NTLM instead of Kerberos.
          
          Sollte diese Fehlermeldung erscheinen, kann sie durch die manuelle
          Zuordnung eines SPN behoben werden.
          \begin{literaturinternet}
            \item \cite{btmcm210tsnilwutrs}
            \item \cite{mswcttssnilcnrtspns}
            \item \cite{ms191153}
          \end{literaturinternet}
        \subsubsection{Manuelle zuordnung eines SPN}
          \begin{enumerate}
            \item Starten Sie das MMC-Plugin \enquote{ADSI-Editor}
            \item Öffnen Sie per Rechtsklick auf \enquote{ADSI-Editor} das
            Kontextmenü und klicken Sie auf \enquote{Verbindung herstellen}.
            \item Klicken Sie auf OK um die Verbindung zur Domäne herzustellen.
            \item Navigieren Sie im Verzeichnisbaum bis zum gewünschten gMSA.
            \item Öffnen Sie das Kontextmenü des gMSA und klicken Sie auf
            \enquote{Eigenschaften}
            \item Scrollen Sie auf der Registerkarte \enquote{Attribut-Editor}
            soweit nach unten, bis Sie den Eintrag
            \enquote{servicePrincipalName} gefunden haben. Wählen Sie ihn aus
            und klicken Sie auf \enquote{Bearbeiten}.
            \bild{Der Attribut-Editor}{adsi_editor_8}{1.3}
            \item Fügen Sie den SPN MSSQLSvc/FEA11-119SRVXX.ms-c-ix-04.fus
            hinzu.
            \bild{Das Attribut service\-Principal\-Name}{adsi_editor_9}{1.3}
            \item Schließen Sie alle Dialogfenster und den ADSI-Editor.
            \item Starten Sie den SQL Server-Dienst neu.
          \end{enumerate}
        \subsubsection{Die Zuordnung eines SPN überprüfen}
          Ob die Zuordnung des SPN funktioniert hat kann mit Hilfe des Kommandos
          \oscommand{setspn} in einer Eingabeaufforderung überprüft werden.
          
          \oscommand{setspn -L MS-C-IX-04\textbackslash MSSQLSERVER12\$}
          
          Bei erfolgreicher Zuordnung erscheint eine Liste der SPNs, die dem
          gMSA MSSQLSERVER12\$ zugeordnet sind.
          \bild{Ergebnis von setspn.exe}{setspn_1}{1.5}
        \subsubsection{Wird Kerberos nun wirklich verwendet?}
          Die Katalogsicht \identifier{sys.dm\_exec\_connections} kann darüber
          Auskunft geben, ob Kerberos für die Authentifizierung genutzt wird
          oder nicht.
          \begin{lstlisting}[language=ms_sql,caption={Wie wird
        authentifiziert?},label=sql19_02]
SELECT net_transport, client_net_address, encrypt_option, auth_scheme
FROM   sys.dm_exec_connections;
        \end{lstlisting}
        \begin{center}
          \begin{small}
            \changefont{pcr}{m}{n}
            \tablefirsthead{
              \multicolumn{1}{l}{\textbf{NET\_TRANSPORT}} &
              \multicolumn{1}{l}{\textbf{CLIENT\_NET\_ADDRESS}} &
              \multicolumn{1}{l}{\textbf{ENCRYPT\_OPTION}} &
              \multicolumn{1}{l}{\textbf{AUTH\_SCHEME}} \\
              \cmidrule(l){1-1}\cmidrule(l){2-2}\cmidrule(l){3-3}\cmidrule(l){4-4}
              }
            \tablehead{}
            \tabletail{
              \multicolumn{2}{l}{\textbf{1 Zeile ausgew\"ahlt}} \\
            }
            \tablelasttail{
              \multicolumn{2}{l}{\textbf{1 Zeile ausgew\"ahlt}} \\
            }
            \begin{mssql}
              \begin{supertabular}{llll}
                TCP & 192.168.111.43 & TRUE & KERBEROS \\
              \end{supertabular}
            \end{mssql}
          \end{small}
        \end{center}
    \section{Von Privilegien, Rollen, Prinzipalen und Securables}
      \subsection{Privilegien}
        \subsubsection{Zuweisen, Entziehen und Verweigern}
          Um mit Datenbankobjekten arbeiten zu können, müssen Benutzer dazu
          berechtigt werden bestimmte Aktionen auszuführen. Dies kann z. B. das
          Verbinden mit der SQL Server-Instanz, das Lesen von Tabellenzeilen
          oder auch das Ändern von Werten in Tabellen sein. Solche
          Berechtigungen werdem im Datenbankumfeld als \enquote{Privilegien}
          bezeichnet. Privilegien können Nutzern zugewiesen oder entzogen
          werden. Zusätzlich dazu kennt der Microsoft SQL Server auch die
          Möglichkeit einem Benutzer ein Privileg explizit zu verweigern.
          
          Die Auswirkungen beim Zuweisen, Entziehen oder Verweigern von
          Privilegien sind vergleichbar mit den Auswirkungen bei der
          Verwaltung von NTFS-Rechten.
          \begin{itemize}
            \item \textbf{Zuweisung}: Einem Benutzer wird ein Privileg erteilt.
            Er kann dadurch eine bestimmte Aktion ausführen.
            \item \textbf{Entzug}: Einem Benutzer wird ein zugeteiltes Privileg
            weggenommen, er kann eine Aktion nicht mehr ausführen.
            \item \textbf{Verweigerung}: Einem Benutzer wird die Ausübung eines
            Privileges explizit verboten. Es spielt dabei keine Rolle, ob dem
            betroffenen Benutzer ein Privileg erteilt wurde oder nicht. Durch
            die Verweigerung kann er bestimmte Aktionen nicht ausführen.
          \end{itemize}
          \begin{merke}
            Zuteilung, Entzug und Verweigerung von Privilegien wirken sich immer
            augenblicklich auf den oder die betroffenen Benutzer aus.
          \end{merke}
          Bei der Zuteilung von Privilegien an Benutzer sollte immer nach dem
          LUA-Prinzip vorgegangen werden. LUA steht für
          \enquote{\textbf{L}east-Privileged \textbf{U}ser \textbf{A}ccount} und
          bedeutet, dass einem Benutzer immer nur ein minimaler Privilegiensatz
          zugwiesen werden soll. Wichtig ist, dass dieses Prinzip auch schon bei
          der Entwicklung von Datenbankanwendungen genutzt wird, da nur so ein
          konsequent nach LUA erstelltes Rollen- Rechtekonzept entsteht.
        \subsubsection{Durchführung der Berechtigungsprüfung}
          Um zu überprüfen, ob ein Prinzipal den Zugriff auf eine Resource
          erhält oder nicht, muss der SQL Server einen relativ komplexen Vorgang
          ausführen, in dessen Rahmen nach einer Berechtigungszuteilung gesucht
          wird, die einen Prinzipal mit ausreichenden Berechtigungen ausstattet.
          Hierbei müssen aber blockierende \DENY-Anweisungen, überlappende
          Gruppenmitgliedschaften oder Besitzverkettungen berücksichtigt werden.
          
          Die folgende Aufzählung gibt einen Überblick darüber, welche Schritte
          der SQL Server bei der Durchführung der Berechtigungsprüfungen
          vollziehen muss. Die Reihenfolge dieser Schritte kann von Vorgang zu
          Vorgang etwas anders sein.
          \begin{enumerate}
            \item Umgehe die gesamte Berechtigungsprüfung, wenn der betroffene
            Prinzipal ein Administrator oder der Eigentümer einer Datenbank ist
            (Mitglied der festen Serverrolle \privileg{sysadmin}, bzw. der
            Benutzer \privileg{dbo}).
            \item Erteilung des Zugriffs, sofern die Besitzverkettung anwendbar
            ist.
            \item Identiäten, die dem Betroffenen User zuordenbar sind,
            aggregieren, um den Sicherheitskontext zu erstellen.
            \begin{merke}
              Als Sicherheitskontext wird die Gruppe der Prinzipale bezeichnet,
              die sich auf den aktuellen Login bzw. Benutzer beziehen und die
              Berechtigungen für die Zugriffüberprüfung einbringen. 
            \end{merke}
            \item Auflisten der Berechtigungen, die für diesen
            Berechtigungsbereich zugteilt oder verweigert worden sind.
            \begin{merke}
              Als Berechtigungsbereich wird das Securable und alle Securable
              Classes bezeichnet, in denen sich das Securable befindet.
              Beispielsweise befindet sich eine Tabelle (Securable) in einem
              Schema (Securable Class) welches sich in einer Datenbank
              (weitere Securable Class) befindet.)
            \end{merke}
            \item Analysieren, welche Berechtigung für den aktuellen Vorgang
            benötigt wird.
            \item Die Berechtigungsprüfung ist beendet (erfolgreich oder nicht
            erfolgreich).
          \end{enumerate}
          Die Prüfung gilt dann als erfolgreich beendet, wenn die benötigte
          Berechtigung für eine Identität innerhalb des Sicherheitskontextes für
          ein Securable im Berechtigungsbereich gefunden wurde (in Kurzform:
          Dann, wenn der betroffene Benutzer das entsprechende Zugriffsrecht
          hat).
        \subsubsection{Namenskonventionen für Privilegien}
          Microsoft hat die Privilegien, die innerhalb des SQL Server zur
          Verfügung stehen, nach dem folgenden Namensschema benannt:
          \begin{itemize}
            \item \textbf{ALTER}: Räumt einem Prinzipal die Möglichkeit ein alle
            Eigenschaften, mit Ausnahme des Besitzes, eines Securables zu
            ändern. Wird dieses Privileg auf einen Gültigkeitsbereich erteilt,
            so ist damit nicht nur das Ändern, sondern auch das Erstellen und
            das Löschen von Securables enthalten.
            \item \textbf{ALTER ANY}: Durch den Zusatz \textbf{ANY} ist es
            möglich alle Securables eines Types zu Erstellen, zu Ändern oder zu
            Löschen. Diese Berechtigung kann nur für die Gültigkeitsbereiche
            Datenbank oder Server erteilt werden und betrifft, im Gegensatz zu
            \textbf{ALTER}, automatisch immer alle Securables in einem dieser
            Bereiche.
            \item \textbf{VIEW DEFINITION}: Ermöglich dem Empfänger Einsicht in
            die Metadaten eines Securables zu nehmen.
            \item \textbf{CONTROL}: Erteilt Privilegien, die denen des Besitzers
            eines Securables sehr ähnlich sind. Der Empfänger kann alle
            Eigenschaften eines Securables ändern (wie bei \textbf{ALTER}) und
            er kann auch Berechtigungen auf das Securable erteilen.
          \end{itemize}
          Diese Liste zeigt nur einen kurzen Ausschnitt aus dem
          Benennungsschema. Die vollständige Version, sowie eine Liste alle in
          SQL Server enthaltenen Privilegien kann unter dem folgenden
          Literaturhinweis eingesehen werden.
          \begin{literaturinternet}
            \item \cite{ms191291}
          \end{literaturinternet}
      \subsection{Rollen}
        Rollen Sie für eine Datenbank das, was unter Microsoft Windows Gruppen
        sind. Mit Hilfe von Rollen können Benutzer gruppiert und Privilegien
        zugeteilt werden. Sie werden benutzt, um die Erstellung und die
        Verwaltung eines Rechtekonzeptes zu vereinfachen. Grundsätzlich ist es
        einfacher eine Rolle mit Privilegien auszustatten und anschließend
        Benutzer der Rolle zuzuweisen bzw. sie daraus zu entfernen, als jedem
        Benutzer einen eigenen Privilegiensatz zu geben. Übersichtlichkeit
        und Verwaltbarkeit eines Systems werden durch die Nutzung von Rollen
        deutlich gesteigert.
        
        \begin{merke}
          Rollen können geschachtelt werden (Rollen werden Rollen zugewiesen).
          Jedoch sollte der Administrator/Entwicklern von einer allzu tiefen
          Verschachtelung absehen, da sonst die Performance einer Anwendung
          verschlechtert werden kann.
        \end{merke}
        \begin{literaturinternet}
          \item \cite{bb669084}
        \end{literaturinternet}
      \subsection{Prinzipale}
        Ein Prinzipal ist ein Objekt bzw. eine Entität, die in der Lage ist
        Resourcen anzufordern und zu benutzen. In Bezug auf Microsoft Windows
        sind Benutzerkonten und Sicherheitsgruppen Prinzipale, da sowohl
        Benutzern als auch Gruppen Rechte zugewiesen werden können, mit deren
        Hilfe die Prinzipale dann Resourcen, wie z. B. Dateien benutzen
        können.
        
        Der SQL Server versteht unter anderem Rollen, Logins und
        Datenbankbenutzer als Prinzipale.
        \begin{literaturinternet}
          \item \cite{ms181127}
        \end{literaturinternet}
      \subsection{Securables - Sicherungsfähige Elemente}
        Securables (deutsche Übersetzung: Sicherungsfähige Elemente) sind
        Entitäten die durch Privilegien gesichert werden können. Sie stellen
        das Gegenstück zu den Prinzipalen dar. Ein Prinzipal muss bestimmte
        Privilegien besitzen, um auf ein Securable zugreifen zu können. Durch
        die Möglichkeit Securables in einander zu schachteln entstehen
        hierarchische Bereiche in der Datenbank, die wiederum selbst
        Securables darstellen und somit auch durch das Autorisierungssystem
        des SQL Server geschützt werden können.
        \begin{literaturinternet}
          \item \cite{ms190401}
        \end{literaturinternet}
    \section{Sicherheit auf Serverebene}
      \subsection{Securables und Prinzipale}
        \subsubsection{Securables}
          Auf Serverebene kennt der Microsoft SQL Server die folgenden
          Securables:
          \begin{itemize}
            \item Anmeldungen (Logins)
            \item Datenbanken
            \item Serverrollen
            \item Endpunkte
            \item Verfügbarkeitsgruppen
          \end{itemize}
          Einige diese Securables werden im weiteren Verlauf dieser
          Unterrichtsunterlagen noch näher erläutert.
          \begin{literaturinternet}
            \item \cite{ms190401}
          \end{literaturinternet}
        \subsubsection{Prinzipale}
          Auf Serverebene kennt der SQL Server zwei Prinzipale: Anmeldungen
          (Logins) und Serverrollen.
          
          Die Verknüpfung eines Benutzerkontos mit dem SQL Server wird als
          \enquote{Login}, zu Deutsch \enquote{Anmeldung} bezeichnet. Der SQL
          Server kennt drei verschiedene Arten von Anmeldungen:
          \begin{itemize}
            \item Windows-Benutzer-Logins
            \item Windows-Gruppen-Logins
            \item SQL Server-Logins
          \end{itemize}
          Zusätzlich zu den aufgelisteten Anmeldetypen gibt es noch weitere, die
          allerdings lediglich zur Codesignierung genutzt werden können.         
      \subsection{Serverrollen}
        Seit SQL Server 2005 werden Rollen auf Serverebene bereitgestellt, um
        dem Administrator die Verwaltung der Privilegie auf einem Server zu
        vereinfachen.
        \subsubsection{Feste Serverrollen}
          Der SQL Server 2014 bringt 9 feste Serverrollen mit, wobei jede dieser
          Rollen einen unveränderlichen Satz von Berechtigungen besitzt.
          \begin{itemize}
              \item \privileg{sysadmin}: Mitglieder werden zu
              Serveradministratoren, da die Rolle alle Privilegien beinhaltet,
              die auf Serverebene existieren.
              \item \privileg{bulkadmin}: Mitglieder können die SQL-Anweisung
              \languagemssql{BULK INSERT}-Ausführen.
              \item \privileg{dbcreator} Diese Rolle enthält alle Privilegien,
              um ihren Mitgliedern das Erstellen, Ändern, Löschen und
              Wiederherstellen von beliebigen Datenbanken zu ermöglichen.
              \item \privileg{diskadmin}: Mitglieder dieser Rolle können die
              Dateien der Datenbanken auf dem Datenträger verwalten.
              \item \privileg{processadmin} Mitglieder können Prozesse
              beenden, die innerhalb einer SQL Server-Instanz laufen.
              \item \privileg{public}: Diese Rolle verhält sich wie die
              Windows-Benutzergruppe \privileg{Jeder}.
              \item \privileg{securityadmin}: Mitglieder dieser Rolle können
              Anmeldungen verwalten und Berechtigungen auf Serverebene zuteilen,
              entziehen oder verweigern. Sofern ein Mitglied Zugriff auf eine
              Datenbank hat kann es auch dort Privilegien zuteilen, verweigern
              oder entziehen. Die Rolle \privileg{securityadmin} sollte mit
              der Rolle \privileg{sysadmin} gleichgestellt werden.
              \item \privileg{serveradmin}: Mitglieder dieser Rolle können
              Serverkonfigurationsoptionen verändern und den Server
              herunterfahren.
              \item \privileg{setupadmin}: Diese Rolle ermöglicht es ihren
              Mitgliedern Verbindungsserver mit Hilfe von T-SQL hinzuzufügen und
              zu entfernen.
          \end{itemize}
          \begin{merke}
            Bei der Benutzung von Festen Serverrollen ist große Vorsicht
            geboten. Jedes Mitglied einer Festen Serverrolle kann andere
            Anmeldungen als Mitglied zur eigenen Rolle hinzufügen.
          \end{merke}
          \begin{literaturinternet}
            \item \cite{ms188659}
            \item \cite{defsafdr2}
          \end{literaturinternet}
        \subsubsection{Eigene Serverrollen erstellen}
          \begin{enumerate}
            \item Erweitern Sie im Objekt-Explorer die SQL Server-Instanz
            innerhalb derer die neue Serverrolle erstellt werden soll.
            \item Wählen Sie das Register \enquote{Sicherheit} und darin
            \enquote{Serverrollen}.
            \item Öffnen Sie das Kontextmenü des Registers
            \enquote{Serverrrollen} und klicken Sie auf \enquote{Neue
            Serverrolle\ldots}. Es erscheint der Dialog \enquote{Neue
            Serverrolle}
            \bild{\enquote{Neue Serverrolle\ldots}}{create_server_role_1}{1.5}
            \item Geben Sie einen Namen für die Serverrolle ein und wählen Sie
            das Securable, dessen Privilegien Sie verwalten möchten.
            \bild{Der Dialog \enquote{Neue
            Serverrolle}}{create_server_role_2}{1.5}
            \item Wählen Sie im unteren Bereich des Dialogs die benötigten
            Privilegien aus.
            \item Klicken Sie auf \enquote{OK}.
          \end{enumerate}
          \beispiel{sql19_03} zeigt den Vorgang der Rollenerstellung mit Hilfe
          von T-SQL
          \begin{lstlisting}[language=ms_sql,caption={Eine
          Serverrolle erstellen},label=sql19_03]
USE [master]
GO

CREATE SERVER ROLE [UserHelpDesk]
GO

GRANT ALTER ANY LOGIN, ALTER ANY CONNECTION, IMPERSONATE ANY LOGIN
TO [UserHelpDesk]
GO
        \end{lstlisting}
        \begin{literaturinternet}
          \item \cite{ee677627}
          \item \cite{ee677610}
        \end{literaturinternet}          
        \subsubsection{Mitglieder einer Rolle verwalten}
          Anmeldung und Serverrollen können Mitglieder in einer Serverrolle
          werden. Die Verwaltung der Mitglieder einer Rolle kann entweder
          mittels SSMS oder mittels T-SQL erfolgen.
          \begin{enumerate}
            \item Öffnen Sie im Objekt-Explorer das Kontextmenü der Serverrolle
            und wählen Sie den Menüpunkt \enquote{Eigenschaften}.
            \item Wechseln Sie links außen auf die Seite \enquote{Mitglieder}.
            \item Klicken Sie auf \enquote{Hinzufügen} um der Rolle neue
            Mitglieder hinzuzufügen oder wählen Sie ein Mitglied, das Sie
            entfernen möchten und klicken Sie auf \enquote{Entfernen}
            \bild{Der Dialog \enquote{Neue
            Serverrolle}}{create_server_role_3}{1.5}
            \item Wechseln Sie auf die Seite \enquote{Mitgliedschaften}. Hier
            kann die Rolle selbst Mitglied einer anderen Rolle werden.
            \bild{Der Dialog \enquote{Neue
            Serverrolle}}{create_server_role_4}{1.5}
            \item Klicken Sie auf \enquote{OK}.
          \end{enumerate}
          \begin{lstlisting}[language=ms_sql,caption={Hinzufügen
          von Mitgliedern zu einer Serverrolle},label=sql19_04]
ALTER SERVER ROLE [UserHelpDesk]
ADD MEMBER [MS-C-IX-04\G-UserHelpdesk]
GO
          \end{lstlisting}
          \begin{lstlisting}[language=ms_sql,caption={Entfernen
          von Mitgliedern aus einer Serverrolle},label=sql19_05]
ALTER SERVER ROLE [UserHelpDesk]
DROP MEMBER [MS-C-IX-04\G-UserHelpdesk]
GO
          \end{lstlisting}
          \begin{literaturinternet}
            \item \cite{ms189775}
          \end{literaturinternet}
        \subsubsection{Selbsterstellte Serverrollen löschen}
          Um eine selbsterstellte Serverrolle löschen zu können, müssen zuerst
          alle Mitglieder der Rolle gelöscht werden.
          \begin{enumerate}
            \item Öffnen Sie im Objekt-Explorer das Kontextmenü der Serverrolle
            und wählen Sie den Menüpunkt \enquote{Löschen}.
            \item Klicken Sie auf \enquote{OK}.
          \end{enumerate}
          \begin{lstlisting}[language=ms_sql,caption={Entfernen einer
          Serverrolle},label=sql19_05a]
DROP SERVER ROLE [UserHelpDesk]
GO
          \end{lstlisting}
          \begin{literaturinternet}
            \item \cite{ee677643}
          \end{literaturinternet}
      \subsection{Privilegienverwaltung auf Serverebene}
        In SQL Server existieren auf Serverebene ca. 30 bis 40 verschiedene
        Privilegien, welche alle einen rein administrativen Charakter haben. Das
        bedeutet, dass diese Privilegien nur Administratoren und nicht den
        \enquote{gewöhnlichen} Benutzer zugeordnet werden sollten.
        \subsubsection{Securables auf Serverebene}
          Auf Serverebene existieren in SQL Server 2014 insgesamt 5 verschiedene
          Securables:
          \begin{itemize}
            \item Login
            \item Endpoint
            \item Server
            \item Availabilitygroup
            \item Server role
          \end{itemize}
          Für jedes Securable existieren unterschiedliche Privilegien, die zu
          dessen Administration genutzt werden können.
        \subsubsection{Die wichtigsten Server-Privilegien im Überblick}
          \begin{itemize}
            \item \privileg{control server}: Ermöglicht genau wie die Rolle
            \privileg{sysadmin} die vollständige Kontrolle über eine SQL
            Server Instanz. Der einzige Unterschied ist, dass ein Mitglied der
            Serverrolle \privileg{sysadmin} nicht durch ein \identifier{DENY}
            aufgehalten werden kann.
            \item \privileg{alter any database}: Ermöglicht das Anlegen und
            Löschen beliebiger Datenbanken.
            \item  \privileg{alter any login}: Erlaubt das Anlegen,
            Ändern und Löschen von Logins. Logins, welche der Serverrolle
            \privileg{sysadmin} angehören können nur dann verändert werden,
            wenn man zusätzlich das Privileg \privileg{control server}
            besitzt.
            \item \privileg{alter settings}: Gibt dem Benutzer die Möglichkeit,
            die Konfiguration des Servers mit \languagemssql{sp_configure} zu
            verändern.
            \item \privileg{connect sql}: Ermöglicht es eine Verbindung zum
            Server aufzubauen. Dieses Privileg wird einem frisch erstellten
            Login automatisch zugewiesen.
            \item \privileg{view any database}: Erlaubt es die Metadaten aller
            Datenbanken mit Hilfe von \identifier{sys.databases},
            \identifier{sysdatabases} oder \identifier{sp\_helpdb} sehen zu
            können.
            \item \privileg{view server state}: Ermöglicht es Einsicht in die
            Serverkonfiguration nehmen zu können.
          \end{itemize}
          \begin{literaturinternet}
            \item \cite{mssqltips1714}
            \item \cite{practdbacocsp}
          \end{literaturinternet}
        \subsubsection{Die wichtigsten Login-Privilegien im Überblick}
          \begin{itemize}
            \item \privileg{control}: Ermöglicht die vollständige Kontrolle über
            ein Login. Es beinhaltet die anderen drei Privilegien
            \privileg{alter}, \privileg{impersonate} und \privileg{view
            definition}.
            \item \privileg{alter}: Ermöglicht es Veränderungen an einem Login
            vorzunehmen.
            \item \privileg{impersonate}: Der Besitzer dieses Privileges kann
            die Identität des betroffenen Logins annehmen.
            \item \privileg{view definition}: Ermöglicht es die Metadaten des
            betroffenen Logins sehen zu können.
          \end{itemize}
        \subsubsection{Privilegien auf Serverebene erteilen}
          \begin{merke}
            Privilegien auf Serverebene können nur dann erteilt werden, wenn die
            aktuelle Datenbank \identifier{master} ist. Ist dies nicht der Fall,
            antwortet der SQL Server mit der Fehlermeldung 4621.
          \end{merke}
          \begin{lstlisting}[language=ms_sql, caption={Privilegien auf
          ein Server- oder Login-Objekt erteilen}, label=admin19_05a]
USE [master]
GO

GRANT control server TO SQLLogin;

GRANT view any definition TO SQLLogin;

GRANT view any database TO SQLLogin
WITH GRANT OPTION;

GRANT impersonate ON LOGIN::SQLLOGIN TO Adventureworks2012\sqluser;

GRANT view definition ON LOGIN::SQLLOGIN TO Adventureworks2012\sqluser;
          \end{lstlisting}
          Die Angabe \languagemssql{WITH GRANT OPTION} bei der Vergabe eines
          Privileges bewirkt, dass der betroffene Benutzer dieses Privileg
          weitergeben kann.
          \begin{literaturinternet}
            \item \cite{ms178640}
            \item \cite{ms186717}
          \end{literaturinternet}
        \subsubsection{Privilegien auf Serverebene Entziehen/Verwehren}
          Das Entziehen und Verwehren von Privilegien auf Serverebene
          funktioniert mit den beiden Kommandos \languagemssql{REVOKE} und
          \languagemssql{DENY}.
          \begin{lstlisting}[language=ms_sql, caption={Privilegien von
          einem Server- oder Login-Objekt entziehen}, label=admin19_05b]
USE [master]
GO

REVOKE control server TO SQLLogin;

REVOKE GRANT OPTION FOR view any database FROM SQLLogin;

REVOKE view any database FROM SQLLogin CASCADE;

REVOKE impersonate ON LOGIN::SQLLOGIN FROM Adventureworks2012\sqluser;
          \end{lstlisting}
          Das Kommando \languagemssql{REVOKE GRANT OPTION} entzieht nicht das
          Privileg selbst, sondern nur die \languagemssql{GRANT OPTION}. Durch die
          Angaben von \languagemssql{CASCADE} wird ein kaskadierender
          Privilegienentzug des \privileg{view any database} Privileges
          ausgeführt, was bedeutet, dass nicht nur der eigentliche Besitzer dieses
          Privileges, sondern auch alle nachgeordneten Besitzer das
          \languagemssql{view any database} Privileg verlieren.
          
          Die Syntax des \languagemssql{DENY}-Kommandos stimmt mit der des
          \languagemssql{REVOKE}-Kommandos nahezu überein.
          \begin{lstlisting}[language=ms_sql, caption={Privilegien von
          einem Server- oder Login-Objekt entziehen}, label=admin19_05c]
USE [master]
GO

DENY control server TO SQLLogin;

DENY view any database FROM SQLLogin CASCADE;

DENY impersonate ON LOGIN::SQLLOGIN FROM Adventureworks2012\sqluser;
          \end{lstlisting}

          \begin{literaturinternet}
            \item \cite{ms186226}
            \item \cite{ms186710}
          \end{literaturinternet}    
      \subsection{Verwalten von Logins}
        \subsubsection{Erstellen von Windows-Logins}
          \begin{enumerate}
            \item Öffnen Sie im Objekt-Explorer den Reiter \enquote{Sicherheit}!
            \item Öffnen Sie das Kontextmenü des Reiters \enquote{Anmeldungen}
            und klicken Sie auf \enquote{Neue Anmeldung}!
            \bild{Ein neues Login anlegen}{create_login_1}{1.3}
            \item Klicken Sie im Dialog \enquote{Anmeldung - Neu} rechts oben
            auf die Schaltfläche \enquote{Suchen\ldots}!
            \bild{Den gewünschten Windows-Benutzer suchen}{create_login_2}{1.1}
            \item Wählen Sie den gewünschten Objekttyp und den Suchpfad aus!
            \bild{Objekttyp und Suchpfad festlegen}{create_login_3}{1.3}
            \item Geben Sie den vollständigen Namen des Windows-Benutzerkontos
            ein!
            \bild{Eingeben des Benutzernamens}{create_login_4}{1.3}
            \item Klicken Sie auf \enquote{OK}!
            \item Wählen Sie Standarddatenbank für den Login aus. Dies ist die
            Datenbank mit der ein Benutzer nach der Anmeldung automatisch
            verbunden wird!
            \bild{Festlegen der Standard\-datenbank}{create_login_5}{1.3}
            \item Klicken Sie auf \enquote{OK}!
          \end{enumerate}
          Soll das neue Login mit Hilfe von T-SQL angelegt werden, muss hierfür
          das Kommando \languagemssql{CREATE LOGIN} mit dem Zusatz
          \languagemssql{FROM WINDOWS} benutzt.
          \begin{lstlisting}[language=ms_sql, caption={Anlegen eines
          Windows-Logins mit T-SQL}, label=admin19_06]
USE [master]
GO

CREATE LOGIN [MS-C-IX-04\SQL-IX-62-01] 
FROM WINDOWS 
WITH 
  DEFAULT_DATABASE=[master]
GO        
        \end{lstlisting}
        \subsubsection{Erstellen eines SQL Server-Logins}
          \begin{enumerate}
            \item Öffnen Sie im Objekt-Explorer den Reiter \enquote{Sicherheit}! 
            \item Öffnen Sie das Kontextmenü des Reiters \enquote{Anmeldungen}
            und klicken Sie auf \enquote{Neue Anmeldung}!
            \item Wählen Sie im Dialog \enquote{Anmeldung - Neu} die Option
            \enquote{SQL Server-Authentifizierung} aus und geben Sie einen Namen
            für das Login an!
            \bild{Erstellen eines SQL Server Logins}{create_sql_login_1}{1.3}
            \item Geben Sie ein Passwort für das Login an!
            \item Wählen Sie die gewünschten Passwortrichtlinien aus!
            \item Wählen Sie Standarddatenbank für den Login aus. Dies ist die
            Datenbank mit der ein Benutzer nach der Anmeldung automatisch
            verbunden wird!
            \item Klicken Sie auf \enquote{OK}!
          \end{enumerate}
          \begin{lstlisting}[language=ms_sql, caption={Anlegen eines SQL
          Server-Logins mit T-SQL}, label=admin19_07]
USE [master]
GO

CREATE LOGIN [SQL-IX-62-01] 
WITH 
  PASSWORD=N'P@ssw0rd' MUST_CHANGE, 
  DEFAULT_DATABASE=[master], 
  CHECK_EXPIRATION=ON, 
  CHECK_POLICY=ON
GO
        \end{lstlisting}
        \subsubsection{Ändern von Logins}
          Mit Hilfe des \languagemssql{ALTER LOGIN}-Kommandos bzw. mittels
          des SSMS ist es möglich, alle Eigenschaften eines Logins zu verändern.
          Hierbei müssen die folgenden Punkte beachtet werden:
          \begin{itemize}
            \item Die Bezeichner von Domänenanmeldungen/Windows-Anmeldungen
            müssen in eckige Klammern gesetzt werden.
            \item Das Deaktivieren eines Logins wirkt sich nicht auf aktuell
            offene Sessions aus.
          \end{itemize}
          \begin{literaturinternet}
            \item \cite{ms189828}
          \end{literaturinternet}
        \subsubsection{Löschen von Logins}
          Das Löschen eines Logins geschieht mit dem
          SQL-Kommando \languagemssql{DROP LOGIN}. Dabei ist zu beachten, dass
          ein Login, welches gerade benutzt wird, nicht gelöscht werden kann.
          Zuerst müssen alle Sessions, die sich auf das zulöschende Login
          beziehen beendet werden. Auch können Logins, welche Besitzer von
          Objekten sind nicht gelöscht werden. Der Objektbesitz muss erst an
          andere Logins übergeben werden.
          \begin{literaturinternet}
            \item \cite{ms188012}
          \end{literaturinternet}
      \subsection{Den Authentifizierungsmodus wählen}
        Der Microsoft SQL Server unterscheidet zwei verschiedene Arten der
        Authentifizierung:
        \begin{itemize}
            \item \textbf{Windows Authentifizierung}: Bei dieser Art von
            Authentifizierung werden verschiedene Windows Konten bzw. Gruppen
            mit dem SQL Server verknüpft. Durch diese Verknüpfung gelten Sie
            dann als Vertrauenswürdig und es erfolgt keine weitere Abfrage von
            Passwörtern oder anderen Sicherheitsmerkmalen. 
            \item \textbf{Gemischter Modus}: Im gemischten Modus wird zusätzlich
            zur Windows Authentifizierung noch eine SQL Server-Authentifizierung
            unterstütz. Dabei speichert/verwaltet des SQL Server eigene
            Benutzerkonten. Dieser Modus gilt jedoch als unsicher, da bei der
            Benutzung der SQL Server-Authentifzierung der Benutzername und das
            Passwort über das Netzwerk übertragen werden.
        \end{itemize}
        \begin{merke}
          Die Windows Authentifizierung wird auch als \enquote{Integrated
          Security} bezeichnet.
        \end{merke}
        Grundsätzlich empfiehlt Microsoft die Benutzung der Integrierten
        Sicherheit. Die SQL Server-Authentifizierung ist nur in wenigen
        Ausnahmefällen sinnvoll. Beispiele für solche Situationen sind:
        \begin{itemize}
          \item Es ist kein Domänencontroller im Netzwerkvorhanden (arbeiten
          in einer Arbeitsgruppe),
          \item Anmeldungen aus unterschiedlichen, nicht vertrauenlswürdigen
          Domänen sind notwendig,
          \item für Web-Anwendungen
        \end{itemize}
        \subsubsection{Integrated Security}
          Wird dieser Authentifizierungsmodus gewählt, werden
          Logins für Benutzerkonten und -gruppen des lokalen Windows-Servers
          oder einer vertrauenswürdigen Domäne im SQL Server erstellt.
          Hierbei gibt es zwei Punkte die zu beachten sind:
          \begin{itemize}
            \item Bei einem Windows-Gruppen-Login, haben alle Mitglieder dieser
            Gruppe die gleichen Zugriffsrechte auf den SQL Server.
            \item Existiert ein Windows-Benutzer-Login für einen Benutzer und
            ist dieser Benutzer gleichermassen Mitglied in einer Windowsgruppe,
            für die es ein Windows-Gruppen-Login gibt, so haben die
            Einstellungen des Windows-Benutzer-Logins Vorrang.
          \end{itemize}
        \subsubsection{SQL Server Authentifizierung}
          Bei diesem Verfahren muss der Administrator die Benutzerkonten im
          SQL Server erstellen und verwalten. Gespeichert werden der Account
          und ein Hash-Wert des Passwortes in der \identifier{master}-Datenbank.
          Wird der SQL Server auf Windows Server 2003 oder höher betrieben,
          können sogar die Kennwortrichtlinien des Active Directory auf die SQL
          Server Benutzerkonten angewendet werden, um so die Sicherheit dieser Konten
         zu erhöhen.
         
         Bei der Installation des SQL Servers wird standardmäßig des SQL Server
         Benutzerkonto \identifier{sa} angelegt. Dieses Konto hat volle
         administrative Rechte und sollte daher gut gesichert werden. Vergeben
         Sie ein starkes Kennwort und lassen Sie den Account gesperrt.
      \subsection{Distributed Administrator Connection - DAC}
        Die Distributed Administrator Connection wird vom SQL Server zur
        Verfügung gestellt, damit der Administrator in speziellen Fällen, in
        denen keine normale Verbindung zur Instanz mehr möglich ist eine
        Problemdiagnose betreiben kann. Sie unterstützt Sicherheitsfeatures, wie
        z. B. Datenbankverschlüsselung und viele andere.
        \begin{merke}
          Nur Mitglieder der festen Serverrolle \privileg{sysadmin} können
          eine DAC aufbauen.
        \end{merke}
        \subsubsection{Verbindungsaufbau mit der DAC}
          Der Aufbau einer Verbindung mit Hilfe der DAC geschieht nicht über den
          Standardnetzwerkport 1433, sondern über Port 1434. Diese geschieht, um
          im Falle dessen, dass es Probleme auf dem Port 1433 gibt, trotzdem
          noch eine DAC aufgebaut werden kann. Standardmäßig kann eine DAC nur
          lokal erstellt werden, d. h. der Client muss sich direkt auf dem
          Datenbankserver befinden. 
          \begin{merke}
            Wenn Port 1434 nicht zur Verfügung steht, wird beim Starten der
            Instanz ein DAC-Port dynamisch ausgewiesen und im Error Log notiert.
          \end{merke}
          Im Folgenden wird gezeigt, wie mit Hilfe von \oscommand{sqlcmd.exe}
          eine DAC zur Instanz erstellt wird.
          \begin{lstlisting}[language=terminal, caption={SQLCMD.exe mit DAC
          benutzen}, label=admin19_08]
sqlcmd.exe -S 127.0.0.1 -A

sqlcmd.exe -S 127.0.0.1,1434

sqlcmd.exe -S FEA11-119SRV12 -U admin -P xxx -A

sqlcmd.exe -S FEA11-119SRV12 -U admin -P xxx -A -d Master
        \end{lstlisting}
        Der Schalter \oscommand{-S} gibt an, zu welcher Instanz die Verbindung
        aufgebaut werden soll und der spezielle Schalter \oscommand{-A} sagt
        aus, dass versucht werden soll, eine DAC zu benutzen.
        \subsubsection{Remoteverbindungen mit der DAC}
          Remote DAC Verbindungen können zugelassen werden, jedoch wird aus
          Sicherheitsgründen davon abgeraten. Falls remote DAC erforderlich sein
          sollte, wird dieses Feature durch die Einstellung
          \languagemssql{remote admin connections} konfiguriert.          
           \begin{lstlisting}[language=ms_sql, caption={Konfigurieren
          von Remote DAC Verbindungen}, label=admin19_08a]
USE [master]
GO

sp_configure 'remote admin connections', 1;
GO

RECONFIGURE;
GO
        \end{lstlisting}
        Unmittelbar nach dem für \languagemssql{remote admin connections} der
        Wert 1 gesetzt wurde, wir der DAC Listener gestartet. Dies
        geschieht ohne Instanz-Neustart.
        
        Um remote DAC wieder zu deaktiveren, muss für die Einstellung
        \languagemssql{remote admin connections} der Wert 0 gesetzt werden.
        \begin{literaturinternet}
          \item \cite{ms190468}
        \end{literaturinternet}
        \subsubsection{Einschränkungen}
          \begin{itemize}
            \item Es ist nur eine DAC Verbindung pro Instanz erlaubt
            \item Die DAC versucht eine Verbindung zur Standarddatenbank des
            benutzten Logins herzustellen. Gelingt dies nicht, so wird die
            Fehlermeldung 4060 ausgegeben.
            \item SQL-Kommandos, die parallele Ausführung benutzen, wie z. B.
            \languagemssql{BACKUP} oder \languagemssql{RESTORE} sind bei Nutzung
            der DAC verboten.
            \item Für eine DAC stehen nur minimale Resourcen zur Verfügung. Es
            sollten deshalb keine Resourcenintensiven SQL-Kommandos ausführt
            werden.
          \end{itemize}      
    \section{Sicherheit auf Datenbankebene}
      \subsection{Securables und Prinzipale}
        \subsubsection{Securables}
          Auf Datenbankebene kennt der SQL Server 2014 insgesamt 16 veschiedene
          Securables. Im Folgenden werden die Prinzipale Datenbankrolle,
          Schema und Benutzer näher erläutert.
          \begin{literaturinternet}
            \item \cite{ms190401}
          \end{literaturinternet}
        \subsubsection{Prinzipale}
          Innerhalb von Datenbanken existieren drei verschiedene Prinzipale:
          \begin{itemize}
            \item Datenbankbenutzer
            \item Datenbankrollen
            \item Anwendungsrollen
          \end{itemize}
        \subsection{Datenbankbenutzer}
          Damit sich ein Login mit einer Datenbank verbinden kann, muss dem
          Login ein Benutzer in der betreffenden Datenbank zugeordnet werden.
          Der Datenbankbenutzer stellt dann die Identität des Logins innerhalb
          der Datenbank dar. Beide, das Login und auch der Datenbankbenutzer
          können den gleichen Namen tragen. Dies ist aber nicht zwingend
          erforderlich.
          
          Logins können in mehreren Datenbanken unterschiedlichen
          Datenbankbenutzern zugeordnet werden. Innerhalb einer Datenbank kann
          aber nur eine einzige Zuordnung zwischen einem Login und einem
          Benutzer erfolgen.        
          \subsubsection{Der Gast-Benutzer}
            Das Gast-Benutzerkonto bietet die Möglichkeit, die direkte Zuordnung
            von Logins und Datenbankbenutzern auszuhebeln. Jedes Login muss
            normalerweise zwingend einem Datenbankbenutzer zugeordnet worden
            sein, um innerhalb einer Datenbank arbeiten zu können. Ist das
            Gast-Benutzerkonto einer Datenbank aktiviert, kann jedes Login auch
            ohne explizite Benutzerzuordnung eine Verbindung zu der betreffenden
            Datenbank herstellen.
            \begin{merke}
              Das Gast-Benutzerkonto ist ein integriertes Benutzerkonto, welches
              in jeder Datenbank existiert. Standardmäßig ist es deaktiviert und
              seine Verwendung ohne zwingenden Grund sollte vermieden werden.
              Der Gast-Account kann nicht gelöscht werden.
            \end{merke}
          \subsubsection{Der Benutzer \enquote{dbo}}
            Das Benutzerkonto \privileg{dbo} ist ein in jede Datenbank fest
            integriertes Konto, welches implizt alle Berechtigungen zum
            Ausführen beliebiger Operationen auf der jeweiligen Datenbank
            besitzt. Implizit bedeutet, dass für dieses Konto schlicht und
            einfach keinerlei Berechtigungsprüfung durchgeführt wird. Dies hat
            zur Folge, dass auch die explizite Verweigerung eines Privilegs
            keine Auswirkung auf den Nutzer \privileg{dbo} hat.
            
            Von besonderer Bedeutung ist dieses Konto, wenn eine beschädigte
            Datenbank wiederhergestellt werden muss. Da die Zuordnung, zwischen
            Login und Datenbankbenutzer in der Datenbank \identifier{master}
            einer jeden Instanz gespeichert wird, ist trotz Beschädigung der
            Datenbank bekannt, wer dem Benutzer \privileg{dbo} zugeordnet ist.
            Nur so kann eine Wiederherstellung der Datenbank erfolgen.
            \begin{merke}
              Alle Mitglieder der Festen Serverrolle \privileg{sysadmin} sind
              automatisch dem Benutzerkonto \privileg{dbo} zugeordnet.
            \end{merke}
            \begin{literaturinternet}
              \item \cite{bb669065}
            \end{literaturinternet}
          \subsubsection{Verschiedene Benutzertypen}
            Microsoft SQL Server kennt eine Reihe verschiedener
            Datenbankbenutzertypen für unterschiedliche Einsatzzwecke. Es wird
            unterschieden in:
            \begin{itemize}
                \item \textbf{SQL-Benutzer mit Anmeldename}: Diese Art von
                Benutzerkonto kommt häufig dann zum Einsatz, wenn ein Benutzer
                außerhalb der Windows-Domäne Zugriff auf den SQL Server
                benötigt. Ein klassischer Fall wäre z. B. ein
                Außendienstmitarbeiter. Die Verwaltung der Benutzersicherheit
                wird bei dieser Kontoart komplett vom SQL Server übernommen.
                \item \textbf{SQL-Benutzer ohne Anmeldename}: Benutzerkonten
                ohne Anmeldename kommen nur in Partialy Contained Databases zum
                Einsatz. Ein Erläuterung deren Bedeutung/Benutzung erfolg an
                entsprechender Stelle in diesem Manuskript.
                \item \textbf{Benutzer mit Zuordnung zu einem Zertifikat}:
                Benutzer die aus einem Zertifikat erstellt wurden können sich
                nicht am SQL Server anmelden. Sie dienen zur Ausführung von
                Prozessen (z. B. Datenbankspiegelung, Bulk Copy u. ä.) oder zur
                Verbesserung der Sicherheit von Datenbankanwendungen.
                \item \textbf{Benutzer mit Zuordnung zu einem asymmetrischen
                Schlüssel}: Genau wie die Benutzer mit Zuordnung zu einem
                Zertifikat können sich auch diese Benutzer nicht an der
                Datenbank anmelden. Auch ihr Einsatzzweck ist mit dem der
                Benutzer mit Zuordnung zu einem Zertifikat identisch.
                \item \textbf{Windows-Benutzer}: Mit dieser Kontoart ist es
                möglich, eine direkte Zuordnung zwischen einem Windows-Benutzer
                bzw. einer Windows-Benutzergruppe und einem SQL Server
                Benutzerkonto herzustellen. Die Verwaltung der
                Benutzersicherheit wird dabei von Windows übernommen.
            \end{itemize}
            \begin{literaturinternet}
              \item \cite{bb895327}
              \item \cite{somarskoggranperm}
            \end{literaturinternet}
        \subsection{Datenbankbenutzer verwalten}
          \subsubsection{SQL-Benutzer mit Anmeldename erstellen}
            \begin{enumerate}
              \item Wählen Sie im Objekt-Explorer den Ordner
              \enquote{Datenbanken}.
              \item Klicken Sie auf die gewünschte Datenbank und navigieren
              Sie dort zum Ordner \identifier{Sicherheit}.
              \item Öffnen Sie das Kontextmenü der Option \identifier{Benutzer}
              und klicken Sie auf \identifier{Neuer Benutzer}
              \item Wählen Sie im Dropdownmenü \identifier{Benutzertyp} den Punkt
              \identifier{SQL-Benutzer mit Anmeldename}
              \item Geben Sie im Feld \identifier{Benutzername} einen gültigen
              Benutzernamen an.
              \item Geben Sie im Feld \identifier{Anmeldename} den Namen eines
              existierenden Logins an oder klicken Sie auf die Schaltfläche
              rechts neben dem Feld, um ein Login auszuwählen.
              \item Klicken Sie auf OK.
            \end{enumerate}
            Das Anlegen eines SQL-Benutzers mit Anmeldename kann auch mittels
            SQL-Statement geschehen.
            \begin{lstlisting}[language=ms_sql, caption={Anlegen eines
            SQL-Benutzers mit Anmeldename}, label=admin19_08b]
USE [master]
GO

-- Das Login muss im Vorfeld bereits erstellt worden sein.
CREATE LOGIN [SQL-IX-62-01]
WITH PASSWORD=N'password', 
DEFAULT_DATABASE=[model], 
CHECK_EXPIRATION=OFF, 
CHECK_POLICY=OFF
GO

-- Anlegen des Datenbankbenutzers
CREATE USER [SQL-IX-62-01]
FOR LOGIN [SQL-IX-62-01];
GO
            \end{lstlisting}
            \begin{merke}
              Wenn Sie bei der Auswahl des Logins ein Login mit
              Windows-Authentifizierung auswählen wird SQL Server automatisch
              ein Windows-Benutzer erstellen.
            \end{merke}            
            \bild{Anlegen eines SQL-Benutzers mit Anmeldename}{create_sql_user_with_login}{1.3}
          \subsubsection{Anlengen von Windows-Benutzern}
            \begin{enumerate}
              \item Wählen Sie im Objekt-Explorer den Ordner
              \enquote{Datenbanken}.
              \item Klicken Sie auf die gewünschte Datenbank und navigieren
              Sie dort zum Ordner \identifier{Sicherheit}.
              \item Öffnen Sie das Kontextmenü der Option \identifier{Benutzer}
              und klicken Sie auf \identifier{Neuer Benutzer}
              \item Wählen Sie im Dropdownmenü \identifier{Benutzertyp} den Punkt
              \enquote{Windows-Benutzer}
              \item Geben Sie im Feld \identifier{Benutzername} den Namen eines
              existierenden Windows-Benutzeraccounts an. Achten Sier hierbei
              auf die korrekte Schreibweise:
              [Rechnername]\textbackslash [Benutzername] bzw.
              [Domäne]\textbackslash [Benutzername], z. B.
              \enquote{ORA-B-IX-62\textbackslash SQLUser}.
              \item Geben Sie im Feld \identifier{Anmeldename} den Namen eines
              existierenden Logins an oder klicken Sie auf die Schaltfläche
              rechts neben dem Feld, um ein Login auszuwählen.
              \item Klicken Sie auf OK.
            \end{enumerate}
            \begin{merke}
              Achten Sie darauf, dass Sie bei der Auswahl des Logins ein Login
              mit Windows-Authentifizierung auswählen, da sie anderenfalles eine
              Fehlermeldung erhalten werden.
            \end{merke}
            \bild{Anlegen eines Windows-Benutzers}{create_windows_user}{1.3}
            Das Anlegen eines SQL-Benutzers mit Anmeldename kann auch mittels
            SQL-Statement geschehen.
            \begin{lstlisting}[language=ms_sql, caption={Anlegen eines
            Windows-Benutzers}, label=admin19_08c]
USE [master]
GO

-- Das Login muss im Vorfeld bereits erstellt worden sein.
CREATE LOGIN [ORA-B-IX-62\sqluser]
FROM WINDOWS 
WITH DEFAULT_DATABASE=[model]
GO

-- Anlegen des Datenbankbenutzers
CREATE USER [SQL-IX-62-01]
FOR LOGIN [SQL-IX-62-01];
GO
            \end{lstlisting}
            \begin{merke}
              Wird \languagemssql{FOR LOGIN} weggelassen, wird der Benutzer einem
              gleichnamigen Login zugewiesen.
            \end{merke}
            \begin{literaturinternet}
              \item \cite{ms173463}
            \end{literaturinternet}
          \subsubsection{Benutzerzuordnungen}
            Als Benutzerzuordnung wird die Verbindung eines Logins
            zu einem Datenbankbenutzer in einer Datenbank bezeichnet. Diese
            Zuordnung kann im Menü \identifier{Sicherheit} -
            \identifier{Anmeldungen} in den Eigenschaften des jeweiligen Logins
            auf der Seite \identifier{Benutzerzuordnung} eingesehen werden.
            \bild{Ansicht der Benutzerzuordnungen}{user_mapping}{1.3}
            \begin{merke}
              Der Eigenschaftsdialog \identifier{Benutzerzuordnung} bietet einen
              alternativen Weg zum Erstellen von Datenbankbenutzern. Nach der
              Erstellung eines Logins muss lediglich bei der/den gewünschten
              Datenbank(en) ein Häckchen gesetzt werden und SQL Server richtet
              automatisch einen passenden Datenbankbenutzer ein.
            \end{merke}
          \subsubsection{Schemas im Besitz}
            Der Eigenschaftsdialog \identifier{Schemas im Besitz} eines jeden
            Datenbankbenutzers zeigt die Schemata, welcher in der Datenbank
            existieren, zu der der Benutzer gehört. Hier kann der Benutzer den
            Besitz eines Schemas übernehmen.
            \begin{merke}
              Der Nutzen und die Bedeutung von Datenbankschemata wird im
              weiteren Verlauf dieses Kapitels noch erläutert.
            \end{merke}
            \bild{Schemas im Besitz des Benutzers}{schema_ownership}{1.3}
          \subsubsection{Mitgliedschaft}
            Im Eigenschaftsdialog \identifier{Mitgliedschaft} kann der Benutzer
            einer Datenbankrolle als Mitglied hinzugefügt werden.
            \bild{Rollenmitgliedschaften eines Benutzers}{role_membership}{1.3}
          \subsubsection{Sicherungsfähige Elemente}
            Mit Hilfe des Dialoges \identifier{Sicherungsfähige Elemente} können
            einem Datenbankbenutzer direkt Privilegien für Securables zugewiesen
            werden.
            \begin{merke}
              Die Vergabe von Privilegien sollte niemals direkt an Benutzer
              erfolgen. Benutzen Sie hierfür ein Rollen- und Rechtekonzept. Ein
              Beispiel für ein solches Konzept wird noch weiteren Verlauf dieses
              Manuskriptes gegeben.
            \end{merke}
            \bild{Sicherungsfähige Elemente eines
            Benutzers}{users_securables}{1.3}
            \begin{literaturinternet}
              \item \cite{aa337545}
            \end{literaturinternet}
          \subsubsection{Allgemeine Änderungen an einem Benutzerkonto vornehmen}
            Grundsätzlich können alle Attribute eines Benutzers im Nachhinein
            geändert werden. Beispielsweise können der
            Benutzername und das zugeordnete Login verändert werden.
            \begin{lstlisting}[language=ms_sql, caption={Umbenennen eines
            Benutzers}, label=admin19_08d]
USE [model]
GO

ALTER USER [SQLUser]
WITH NAME = [SQLLooser];
GO
            \end{lstlisting}
            \begin{merke}
              Das Umbenennen eines Benutzers funktioniert nicht im Management
              Studio.
              
              Für das Umbenennen eines Windows-Benutzers muss der gültige Name
              eines existenten Windows-Benutzeraccounts angegeben werden.
            \end{merke}            
            Soll dem Benutzer ein anderes Login zugewiesen werden, so muss
            beachtet werden, dass das neue Login den gleichen Typ (SQL Server-
            bzw. Windowsauthentifizierung) haben muss und das es bereits
            existieren muss.
            \begin{lstlisting}[language=ms_sql, caption={Umbenennen eines
            Benutzers}, label=admin19_08e]
USE [model]
GO

ALTER USER [SQLLooser]
WITH LOGIN = [SQLLooserLogin];
GO
            \end{lstlisting}
            \begin{literaturinternet}
              \item \cite{ms176060}
            \end{literaturinternet}
          \subsubsection{Datenbankbenutzer löschen}
            Mit Hilfe des \languagemssql{DROP USER}-Anweisung können Benutzer
            aus einer Datenbank gelöscht werden. Zu Beachten ist dabei, dass
            Besitzer von Securables erst dann gelöscht werden können, wenn der
            Besitz an andere Nutzer übergeben worden ist.
            
            Um einen Datenbankbenutzer im Management Studio zu löschen öffnen
            Sie das Kontextmenü des Benutzers und klicken Sie auf
            \identifier{Löschen}.
            \bild{Sicherungsfähige Elemente eines
            Benutzers}{users_securables1}{1.3}
            \begin{literaturinternet}
              \item \cite{ms189438}
            \end{literaturinternet}            
        \subsection{Das Problem der verwaisten Benutzer}
          Jeder Datenbankbenutzer muss einem Login zugeordnet sein, um für die
          Anmeldung an der SQL Server Instanz genutzt werden zu können. Als
          verwaist wird ein Benutzer dann bezeichnet, wenn es für ihn keine
          Zuordnung zu einem Login gibt. Dies kann folgende Gründe haben:
          \begin{itemize}
            \item Das zugeordnete Login wurde gelöscht
            \item Die Datenbank des Benutzers wurde in einer anderen Instanz
            wiederhergestellt.
            \item Die Datenbank des Benutzers wurde an eine andere Instanz
            angefügt.
          \end{itemize}
          Um zu ermitteln, ob es in einer Datenbank verwaiste Benutzerkonten
          gibt und welche Konten davon betroffen sind, stellt Microsoft die
          Stored Procedure \identifier{sp\_change\_users\_login} zur
          Verfügung.
          \begin{merke}
            Diese Procedure kann nicht für Benutzerkonten genutzt werden, die
            auf Windows-Logins basieren.
            
            Microsoft hat angekündigt, dass diese Procedure in neueren
            Versionen des Microsoft SQL Server nicht mehr zur Verfügung stehen
            wird.
          \end{merke}
          \begin{lstlisting}[language=ms_sql, caption={Ermitteln der
            verwaisten Benutzerkonten}, label=admin19_08f]
USE [model]
GO

sp_change_users_login @Action='Report';
GO
          \end{lstlisting}
          \begin{center}
            \begin{small}
              \changefont{pcr}{m}{n}
              \tablefirsthead {
                \multicolumn{1}{c}{\textbf{UsernName}} &
                \multicolumn{1}{c}{\textbf{UserSID}} \\
                \cmidrule(r){1-1}\cmidrule(r){2-2}
              }
              \tablehead{}
              \tabletail {
              }
              \tablelasttail {
              }
              \begin{mssql}
                \begin{supertabular}{ll}
                  SQLLooser & 0x3CE2559A2F2C674F824B27C672ABB3F1 \\
                \end{supertabular}
              \end{mssql}
            \end{small}
          \end{center}
          Um dem Benutzerkonto \identifier{SQLLooser} ein neues Login
          zuzuweisen sollte das \languagemssql{ALTER USER}-Statement genutzt
          werden, so wie es in \beispiel{admin19_08e} zu sehen ist.
          \begin{literaturinternet}
            \item \cite{ms175475}
            \item \cite{ms174378}
          \end{literaturinternet}            
      \subsection{Privilegien auf Datenbankebene}
        Auf Datenbankebene existieren in SQL Server 2014 insgesamt 16
        Securables, wie z. B. Schemata, Benutzer und Datenbankrollen. Auf dieses
        Securables kann eine Vielzahl von Privilegien angewandt werden. Im
        Folgenden werden die Wichtigsten davon aufgezählt und kurz erläutert:
        \begin{itemize}
          \item \privileg{control}: Dies ist das umfangreichste Privileg
          welches einem Benutzer für ein Securable verliehen werden kann. Es
          schließt alle anderen Privilegen auf Datenbankebene ein.
          \item \privileg{view definition}: Erlaubt dem Benutzer die Metadaten
          der betroffenen Objekte einsehen zu können, ohne jedoch direkten
          Zugriff auf das Securable zu haben.
          \item \privileg{alter}: Dieses Privileg erlaubt es dem Benutzer ein
          Objekt zu verändern. Welcher Art diese Veränderung ist hängt dabei
          vom Objekt selbst ab. Einer Rolle können beispielsweise Mitglieder
          hinzugefügt werden, während bei einem Schema der Besitzer geändert
          werden kann.
          \item \privileg{take ownership}: Erlaubt dem Benutzer den Besitz für
          ein Objekt zu übernehmen.
          \item \privileg{impersonate}: Dieses Privileg kann auf Logins
          (Serverebene) und Benutzer (Datenbankebene) erteilt werden und
          ermöglicht es, die Identität eines anderen Logins/Benutzers
          anzunehmen.
        \end{itemize}
        \subsubsection{Zusammenhänge zwischen Securable- und Datenbankebene}
          Die in SQL Server implementierte Berechtigungshierarchie lässt es zu,
          dass einige Privilegien, die direkt auf eine Datenbank erteilt werden,
          sich auch auf deren Securables auswirken. Beispiele hierfür sind die
          Privilegien \privileg{alter any schema}, \privileg{alter any role} und
          \privileg{alter any user}. Hat ein Benutzer das Privileg
          \privileg{alter any role}, dann besitzt er implizit das
          \privileg{alter}-Privileg für alle Rollen, die in der Datenbank
          existieren bzw. zukünftig existieren werden.
          
          Die folgende Tabelle zeigt ausschnittsweise, welche datenbankseitigen
          Privilegien sich implizit auf Securables auswirken.
          \begin{center}
            \begin{small}
              \changefont{pcr}{m}{n}
              \label{tabdatabasepermimplicit}
              \tablefirsthead{
                \multicolumn{1}{c}{\textbf{Übergeordnetes Privileg}} &
                \multicolumn{1}{c}{\textbf{Basisprivileg}} &
                \multicolumn{1}{c}{\textbf{Securable}} \\
                \hline
                }
              \tablehead{}
              \tabletail{
              }
              \tablelasttail{
              }
              \begin{supertabular}{l|l|l}
                \privileg{control} & \privileg{control} & Zertifikat \\
                \privileg{control} & \privileg{control} & Asymmetrischer Schlüssel \\
                \privileg{control} & \privileg{control} & Schema \\
                \privileg{take ownership} & \privileg{control} & Asymmetrischer Schlüssel \\
                \privileg{alter any certificate} & \privileg{alter} & Zertifikat \\
                \privileg{alter any role} & \privileg{alter} & Datenbankrolle \\
                \privileg{alter any schema} & \privileg{alter} & Schema \\
              \end{supertabular}
            \end{small}
          \end{center}
          Die \tabelle{tabdatabasepermimplicit} ist ein Ausschnitt aus der
          Tabelle \enquote{SQL Server and SQL Database Permissions} aus dem
          Artikel \enquote{Permissions (Database Engine)}
          (\parencite{ms191291}).
        \subsubsection{Zusammenhänge zwischen Datenbank- und Serverebene}
          Ähnliche Zusammenhänge, wie sie gerade beschrieben wurden, existieren
          auch zwischen Server und Datenbankebene.
           \begin{center}
            \begin{small}
              \changefont{pcr}{m}{n}
              \label{tabserverpermimplicit}
              \tablefirsthead{
                \multicolumn{1}{c}{\textbf{Übergeordnetes Privileg}} &
                \multicolumn{1}{c}{\textbf{Basisprivileg}} &
                \multicolumn{1}{c}{\textbf{Securable}} \\
                \hline
                }
              \tablehead{}
              \tabletail{
              }
              \tablelasttail{
              }
              \begin{supertabular}{l|l|l}
                \privileg{control server} & \privileg{alter any asymmetric key} & Datenbank \\
                \privileg{control server} & \privileg{alter any role} & Datenbank \\
                \privileg{control server} & \privileg{alter any schema} & Datenbank \\
                \privileg{alter any database} & \privileg{alter} & Datenbank \\
                \privileg{alter any certificate} & \privileg{alter} & Zertifikat \\
                \privileg{view any definition} & \privileg{view definition} & Datenbank \\
              \end{supertabular}
            \end{small}
          \end{center}
          Die \tabelle{tabserverpermimplicit} ist ein Ausschnitt aus der
          Tabelle \enquote{SQL Server and SQL Database Permissions} aus dem
          Artikel \enquote{Permissions (Database Engine)}
          (\parencite{ms191291}).
          \begin{literaturinternet}
            \item \cite{ms190401}
            \item \cite{ms175536}
            \item \cite{ms191291}
          \end{literaturinternet}
        \subsubsection{Privilegien auf Datenbankebene erteilen/entziehen
        und verweigern}
          Das Erteilen, Entziehen und Verweigern von Privilegien auf
          Datenbankebene entspricht dem Erteilen, Entziehen und Verweigern von
          Privilegie auf Serverebene. Hier einige Beispiele:
          \begin{lstlisting}[language=ms_sql, caption={Privilegien auf
          Datenbankebene erteilen, entziehen und verweigern},label=admin19_08g]
USE [Bank]
GO

GRANT CREATE TABLE TO bob;

GRANT ALTER ANY ROLE TO bob;

GRANT CREATE VIEW TO bob
WITH GRANT OPTION;

REVOKE CREATE TABLE FROM bob;

REVOKE GRANT OPTION FOR CONTROL ON USER::[Bank] FROM [bob] CASCADE
GO

GRANT CONTROL ON USER::[Bank] TO [bob]
GO

DENY IMPERSONATE ON USER::[Bank] TO [bob]
GO
          \end{lstlisting}
          Das Kommando \languagemssql{GRANT CREATE VIEW TO bob WITH GRANT
          OPTION;} erteilt dem Benutzer \identifier{bob} das \privileg{create
          view} und gleichzeitig die Möglichkeit dieses an andere Benutzer
          weiterzugeben. Dies ist der gleiche Mechanismus wie auch auf
          Serverebene.
          
          Um auf Datenbankebene nur die \languagemssql{GRANT OPTION} zu
          entziehen, muss hier zusätzlich das Schlüsselwort
          \languagemssql{CASCADE} angegeben werden. So werden automatisch alle
          Privilegien entzogen, die der Benutzer aufgrund des Besitzes der
          \languagemssql{GRANT OPTION} erteilt hat.
          
          Die Kombination der beiden letzten Kommandos zeigt wie ein Benutzer
          das \privileg{control}-Privileg erhalten und trotzdem nicht die
          Identität des betroffenen Benutzerkontos übernehmen kann. Dies
          funktioniert, in dem ihm das \privileg{control}-Privileg erteilt und
          das \privileg{impersonate}-Privileg verweigert wird.
          \begin{literaturinternet}
            \item \cite{ms188396}
            \item \cite{ms187728}
          \end{literaturinternet}
      \subsection{Rollen}
        \subsubsection{Die Datenbankrolle \enquote{db\_owner}}
          Die Datenbankrolle \privileg{db\_owner} gehört zu den Festen
          Datenbankrollen, die in jeder Datenbank existieren. Sie räumt ihren
          Mitgliedern alle Berechtigungen ein, die zur Administration einer
          Datenbank notwendig sind. Diese Rolle sollte nicht mit dem
          Datenbankbenutzer \privileg{dbo} verwechselt werden. Es existieren
          zwei wesentliche Unterschiede zwischen der Rolle
          \privileg{db\_owner} und dem Benutzer \privileg{dbo}:
          \begin{itemize}
            \item Den Mitgliedern der Rolle \privileg{db\_owner} können
            Priviliegien explizit verweigert werden. Für den Benutzer
            \privileg{dbo} sind Verweigerungen irrelevant, da für ihn keine
            Berechtigungsprüfung erfolgt.
            \item Die Auflistung der Mitglieder der Rolle \privileg{db\_owner}
            wird in jeder Datenbank separat gespeichert. Ist die Datenbank
            beschädigt, kann nicht mehr festgestellt werden, wer ein Mitglied
            dieser Rolle ist. Somit können Mitglieder der Rolle
            \privileg{db\_owner} nur unbeschädigte Datenbanken
            wiederherstellt. Der Benutzer \privileg{dbo} kann auch beschädigte
            Datenbanken wiederherstellen.
          \end{itemize} 
          \begin{literaturinternet}
            \item \cite{gc20110407msdvdo}
          \end{literaturinternet}
        \subsubsection{Die Datenbankrolle \enquote{public}}
          Die Datenbankrolle \privileg{public} ist mit der
          Windows-Benutzergruppe \privileg{Jeder} vergleich bar. Alle aktiven
          Benutzer innerhalb einer Datenbank sind automatisch Mitglied dieser
          Rolle.
          \begin{merke}
            Die Datenbankrolle \privileg{public} kann nicht gelöscht werden.
            Da immer alle aktiven Benutzer automatisch Mitglied dieser Rolle
            werden, sollte diese Rolle nur minimale Privilegien haben.
          \end{merke}
        \subsubsection{Weitere Feste Datenbankrollen}
          Neben den beiden Rollen \privileg{db\_owner} und \privileg{public}
          existieren in SQL Server 2014 noch acht weitere Feste Datenbankrollen.
          Diese sind:
          \begin{itemize}
            \item \privileg{db\_accessadmin}: Ermöglicht es, den Zugriff auf
            eine Datenbank zu regeln. Mitglieder können festlegen, welche Logins
            einem Benutzerkonto in der Datenbank zugeordnet werden.
            \item \privileg{db\_backupoperator}: Mitglieder können eine
            Sicherung der Datenbank anfertigen.
            \item \privileg{db\_datareader}: Erteilt das Recht, alle Tabellen in
            der Datenbank zu lesen.
            \item \privileg{db\_datawriter} Erteilt das Recht, in alle
            Tabellen der Datenbank zu schreiben.
            \item \privileg{db\_ddladmin}: Mitglieder können alle DDL-Befehle
            innerhalb einer Datenbank ausführen.
            \item \privileg{db\_denydatareader}: Verweigert das Recht Tabellen
            in der Datenbank zu lesen. Mitglieder können keinerlei Daten
            abfragen.
            \item \privileg{db\_denydatawriter}: Verweigert das Recht
            DML-Befehle auf die Tabellen einer Datenbank anzuwenden. Mitglieder
            können keine Daten ändern.
            \item \privileg{db\_securityadmin}: Ermöglicht das Ändern von
            Rollenmitgliedschaften und die Verwaltung von Privilegien.
          \end{itemize}
          \begin{literaturinternet}
            \item \cite{ms189121}
            \item \cite{ms189612}
          \end{literaturinternet}
        \subsubsection{Besondere Rollen der MSDB-Datenbank}
          Die Datenbank \identifier{msdb} enhält einige spezielle Feste
          Datenbankrollen die in der MSDB unter \parencite{ms189121} aufgeführt
          sind.
        \subsubsection{Eigene Datenbankrollen erstellen}
          \begin{enumerate}
              \item Erweitern Sie im Objekt-Explorer die Datenbank, innerhalb
              derer die neue Rolle erstellt werden soll.
              \item Wählen Sie das Register \enquote{Sicherheit},
              \enquote{Rollen} und darin \enquote{Datenbankrollen}.
              \bild{Erstellen einer Datenbankrolle}{create_db_role_1}{1.3}
              \item Öffnen Sie das Kontextmenü des Registers
              \enquote{Datenbankrollen} und klicken Sie auf \enquote{Neue
              Datenbankrolle}. Es erscheint der Dialog \enquote{Datenbankrolle
              - Neu}
              \item Tragen Sie im Feld \enquote{Rollenname} den Bezeichner der
              Rolle ein.
              \bild{Den Rollenbezeichner eingeben}{create_db_role_2}{1.3}
              \item Wählen Sie die Seite \enquote{Sicherungsfähige Elemente}
              aus.
              \item Klicken Sie auf die Schaltfläche \enquote{Suchen} und wählen
              Sie die Securables aus, für die Sie Privilegien festlegen
              wollen.
              \item Erteilen Sie in der unteren Fensterhälfte die benötigten
              Privilegien.
              \bild{Auswählen der Securables}{create_db_role_3}{1.3}
              \item Klicken Sie auf \enquote{OK}.
          \end{enumerate}
           \begin{lstlisting}[language=ms_sql, caption={Erstellen einer
           Datenbankrolle}, label=admin19_09]
USE [Bank]
GO

CREATE ROLE userhelpdesk AUTHORIZATION dbo;
GO
        \end{lstlisting}
          \begin{literaturinternet}
            \item \cite{ms187936}
          \end{literaturinternet}
        \subsubsection{Rollenmitglieder verwalten}
          Das Hinzufügen von Mitgliedern zu einer Datenbankrolle kann mittels
          des Managementstudios oder mit Hilfe von T-SQL geschehen. Wenn Sie das
          Management Studio benutzen möchten, gehen Sie wie folgt vor:
          \begin{enumerate}
            \item Erweitern Sie im Objekt-Explorer die Datenbank, innerhalb
            derer die neue Rolle erstellt werden soll.
            \item Wählen Sie das Register \enquote{Sicherheit},
            \enquote{Rollen} und darin \enquote{Datenbankrollen}.
            \item Öffnen Sie das Kontextmenü der Datenbankrolle und klicken Sie
            auf \enquote{Eigenschaften}. Es erscheint der Dialog
            \enquote{Eigenschaften der Datenbankrolle}.
            \item Im unteren Bereich dieses Dialogs können Benutzer oder andere
            Rollen als Mitglieder hinzugefügt oder entfernt werden.
            \bild{Mitglieder zu einer Rolle hinzufügen}{alter_db_role_1}{1.3}
          \end{enumerate}
          Die beiden folgenden T-SQL Statements zeigen, wie einer Rolle
          Mitglieder hinzugefügt bzw. entfernt werden können.
          \begin{lstlisting}[language=ms_sql, caption={Ändern einer
           Datenbankrolle}, label=admin19_10]
USE [Bank]
GO

ALTER ROLE userhelpdesk
ADD MEMBER weidinger;
GO

ALTER ROLE userhelpdesk
DROP MEMBER weidinger;
GO
        \end{lstlisting}
        \subsubsection{Datenbankrollen umbenennen}
          Soll eine Datenbankrolle umbenannt werden, muss das T-SQL Kommando
          \languagemssql{ALTER ROLE} mit der Klausel \languagemssql{WITH NAME}
          benutzt werden.
          \begin{lstlisting}[language=ms_sql, caption={Umbenennen einer
           Datenbankrolle}, label=admin19_11]
USE [Bank]
GO

ALTER ROLE [db_UserHelpDesk] 
WITH NAME = [db_HelpDesk]
GO
          \end{lstlisting}
          Wahlweise kann eine Rolle auch im Objekt-Explorer umbenannt werden.
          \begin{merke}
            Feste Datenbankrollen können nicht umbenannt werden!
          \end{merke}
          \begin{literaturinternet}
            \item \cite{ms189775}
          \end{literaturinternet}
        \subsubsection{Selbsterstellte Datenbankrollen löschen}
          Damit eine Datenbankrolle gelöscht werden kann, müssen zuerst alle
          Mitglieder und alle Securables aus der Rolle entfernt werden.
          \begin{enumerate}
            \item Öffnen Sie im Objekt-Explorer das Kontextmenü der
            Datenbankrolle und wählen Sie den Menüpunkt \enquote{Löschen}.
            \item Klicken Sie auf \enquote{OK}.
          \end{enumerate}
          \begin{lstlisting}[language=ms_sql,caption={Entfernen einer
          Datebankrolle},label=sql19_12]
DROP ROLE [db_UserHelpDesk];
GO
          \end{lstlisting}
          \begin{literaturinternet}
            \item \cite{ms174988}
          \end{literaturinternet}
    \section{Datenbankschemata}
      \subsection{Securables}
        Innerhalb eines Schemas können sich die folgenden Securables befinden:
        \begin{itemize}
          \item Type 
          \item XML-Schemaauflistung
          \item Aggregat (Objekt)
          \item Funktion
          \item Verfahren
          \item Queue 
          \item Synonym
          \item Tabelle
          \item Sicht
        \end{itemize}
        \begin{literaturinternet}
          \item \cite{ms190401}
        \end{literaturinternet}
        \subsubsection{Privilegien auf Schemaebene}
          Auf Schemaebene existieren in SQL Server 2014 insgesamt 12
          Objektprivilegien, wie z. B. \privileg{select}, \privileg{insert},
          \privileg{update} und \privileg{delete}. Diese steuern den direkten
          Zugriff auf die Datensätze innerhalb des Schemas bzw. der Datenbank.
        
          Zugewiesen, entzogen und verweigert werden Objektprivilegien mit den
          SQL-Kommandos \languagemssql{grant}, \languagemssql{revoke} und
          \languagemssql{deny}.
          \begin{lstlisting}[language=ms_sql,caption={Erteilen
          von Objektprivilegien},label=sql19_12a]
USE [Bank]
GO

GRANT select ON viw_eigenkunden_kontaktdaten
TO    bob;
GO

GRANT select ON OBJECT::viw_eigenkunden_kontaktdaten
TO    bob;
GO

GRANT select ON viw_eigenkunden_adressen
TO    bob
WITH GRANT OPTION;
GO        

GRANT insert, update, delete ON eigenkunden
TO    bob;
GO  
          \end{lstlisting}
          Der Entzug und das Verweigern von Objektprivilegien erfolgt parallel
          zum Entzug und zur Verweigerung von Privilegien auf Datenbankebene.
          \begin{literaturinternet}
            \item \cite{ms188371}
            \item \cite{ms187965}
          \end{literaturinternet}
        \subsubsection{Ownership chaining - Besitzverkettungen}
          Wenn in einer SQL Server Datenbank mehrere Objekt ein und des selben
          Besitzers/Schemas sequentiel von einander abhängig sind, spricht man
          von einer Besitzverkettung bzw. engl. Ownership chain.
          \beispiel{sql19_12a} zeigt eine Besitzverkettung, zwei Views die
          sequentiel auf den Tabellen \identifier{Kunde} und
          \identifier{Eigenkunde} aufbauen.
          \begin{lstlisting}[language=ms_sql,caption={Eine
          Besitzverkettung},label=sql19_12b]
USE [Bank]
GO

CREATE VIEW dbo.viw\_eigenkunden\_adressen
AS
  SELECT vorname, nachname, strasse, hausnummer, plz, ort,
         TelefonNr, EmailAdresse
  FROM   eigenkunde ek INNER JOIN Kunde k 
         ON (k.kunden_id = ek.kunden_id);
GO

CREATE VIEW dbo.viw\_eigenkunden\_kontaktdaten
AS
  SELECT TelefonNr, EmailAdresse
  FROM   eigenkunde;
GO
          \end{lstlisting}
          Die beiden Views \identifier{viw\_eigenkunden\_adressen} und
          \identifier{viw\_eigenkunden\_kontaktdaten} werden beide im Schema
          \identifier{dbo} (Besitzer ist der Datenbankbenutzer
          \identifier{dbo}) erstellt, in dem auch die Tabelle
          \identifier{eigenkunde} liegt. Dadurch haben alles drei Objekte den
          gleichen Besitzer und eine Besitzverkettung ist entstanden.
          \bild{Eine Besitzverkettung dreier
          Objekte}{ownership_chain}{1.3}
          Der Benutzer \identifier{bob} erhält nun das
          \privileg{select}-Privileg auf die View
          \identifier{viw\_eigenkunden\_kontaktdaten}.
          \begin{lstlisting}[language=ms_sql,caption={Bob
          erhält Leserechte},label=sql19_12c]
USE [Bank]
GO

GRANT select ON viw\_eigenkunden\_kontaktdaten
TO    bob;
GO
          \end{lstlisting}
          Sobald der Benutzer \identifier{bob} nun ein
          \languagemssql{SELECT}-Kommando auf die View
          \identifier{viw\_eigenkunden\_kontaktdaten} ausführt wird der die
          entsprechenden Datensätze aus der Tabelle \identifier{eigenkunde}
          zurückgeliefert bekommen.
          
          SQL Server prüft im Falle einer Besitzverkettung die Zugriffsrechte
          wie folgt:
          \begin{enumerate}
            \item Wer ist der Besitzer des Objekt, auf das Zugriff genommen
            wird (\identifier{viw\_eigenkunden\_kontaktdaten})? Antwort:
            \identifier{dbo}.
            \item Ist \identifier{dbo} auch der Besitzer der weitern Objekte
            (\identifier{viw\_eigenkunden\_adressen}, \identifier{eigenkunde})?
            Antwort: ja.
            \item Hat der Benutzer \identifier{bob} die notwendigen Rechte um
            auf \identifier{viw\_eigenkunden\_kontaktdaten} in gewünschter Weise
            zuzugreifen? Antwort: ja.
          \end{enumerate}
          Da \identifier{bob} das \privileg{SELECT}-Privileg auf die View
          \identifier{viw\_eigenkunden\_kontaktdaten} hat, welche auf die View
          \identifier{viw\_eigenkunden\_adressen} und die Tabelle
          \identifier{eigenkunde} aufbaut bekommt er Zugriff auf die Datensätze
          der Tabelle, ohne das weitere Zugriffsrechte geprüft werden würden.
          \begin{merke}
            Der Zugriff auf ein Objekt mittels einer Besitzverkettung ist
            performanter, da nur für das Objekt, auf welches unmittelbar
            Zugriff genommen wird die Zugriffsrechte geprüft werden.
          \end{merke}
        \subsubsection{Unterbrochene Besitzverkettungen}
          \bild{Eine unterbrochene
          Besitzverkettung}{interrupted_ownership_chain}{1.3}
          \abbildung{interrupted_ownership_chain} zeigt eine unterbrochene
          Besitzverkettung. Dadurch, dass die View
          \identifier{viw\_eigenkunden\_adressen} einen anderen Besitzer
          (\identifier{alice}) hat, als
          \identifier{viw\_eigenkunden\_kontaktdaten} und \identifier{eingekunde}
          wird der Zugriffsmechanismus nicht mehr funktionieren. SQL Server
          prüft hier wie folgt:
          \begin{enumerate}
            \item Wer ist der Besitzer des Objekt, auf das Zugriff genommen
            wird (\identifier{viw\_eigenkunden\_kontaktdaten})? Antwort:
            \identifier{dbo}.
            \item Ist \identifier{dbo} auch der Besitzer der weitern Objekte
            (\identifier{viw\_eigenkunden\_adressen}, \identifier{eigenkunde})?
            Antwort: \textbf{nein}.
            \item Hat der Benutzer \identifier{bob} die notwendigen Rechte um
            auf \identifier{viw\_eigenkunden\_kontaktdaten} in gewünschter Weise
            zuzugreifen? Antwort: ja.
            \item Hat der Benutzer \identifier{bob} die notwendigen Rechte um
            auf \identifier{viw\_eigenkunden\_adressen} in gewünschter Weise
            zuzugreifen? Antwort: \textbf{nein}.
            \item Auswertung: Der gewünschte Zugriff kann nicht erfolgen.
          \end{enumerate}
          \begin{literaturinternet}
            \item \cite{ms188676}
            \item \cite{bb669084}
          \end{literaturinternet}
      \subsection{Erstellen eines Rollen- und Rechtekonzeptes}
        Eine rollenbasierte Rechtezuteilung ist ein intelligentes und flexibles
        Hilfsmittel für den Administrator, zur effizieten Verwaltung einer
        großen Anzahl von Benutzern, mit dessen Hilfe diesen die benötigten
        Berechtigungen erteilt werden können. Rollen- und Rechtekonzepte eignen
        sich besonders als Autorisierungsverfahren für Systeme mit einer hohen
        Volatilität\footnote{lat. volatilis = fliegend, flüchtig, hier die
        Änderungshäufigkeit}. Nicht oder nur bedingt geeignet sind Rollen- und
        Rechtebasierte Konzepte für Systeme mit wenigen Benutzern, da hier der
        Nutzen schnell durch den zusätzlichen Ausfwand nicht mehr zu
        rechtfertigen ist.
        \subsubsection{Vorteile eines Rollen- und Rechtekonzeptes}
          \begin{itemize}
            \item Viele Einzelberechtigungen werden als \enquote{Tätigkeiten} in
            Rollen zusammengefasst. Beispielsweise können Tätigkeiten wie ein
            Login oder das Verwalten von Benutzern über solche Rollen geregelt
            werden.
            \item Die Komplexität des Autorisierungssystems sinkt, da wenig bis
            garkeine Einzelberechtigungen erteilt und verwaltet werden müssen.
            \item Neue Benutzer können sehr einfach in das System integriert
            werden.
          \end{itemize}
        \subsubsection{Vorteile eines Rollen- und Rechtekonzeptes}
          \begin{itemize}
            \item Hoher Planungsaufwand, um ein konsistentes rollenbasiertes
            Autorisierungssystem zu erstellen.
            \item Erfordert gut ausgebildetes Administrationspersonal
            \item Eventuell vorhandene Überschneidungen müssen geregelt werden
            (eine Person benötig die Rechte aus Abteilung A und nur einen Teil
            der Rechte für Abteilung B)
          \end{itemize}
        \subsubsection{Beispiel für ein Rollen- und Rechtekonzept in SQL Server}
        % Grafik erstellen
        % Auf Seiteneffekte mit Besitzverkettungen eingehen
% TODO Einrichten eines Rollen- und Rechtekonzeptes
    \section{Datenbankverschlüsselung}
%TODO Datenbankverschlüsselung
%https://msdn.microsoft.com/de-de/library/bb510663(v=sql.120).aspx
    \section{Verschlüsselung der Kommunikation - SSL}
      Der Secure Socket Layer (SSL) stellt eine einfache Möglichkeit zur
      Verschlüsselung des Datenverkehrs auf Protokollebene dar. Der SQL Server
      bietet diese Art der Verschlüsselung seit der Version 2005 an. Ob SSL
      genutzt werden kann oder nicht hängt aber im Wesentlichen von der
      genutzten Clientsoftware ab. Beispielsweise können die \enquote{Microsoft
      Data Access Components} (MDAC) SSL erst ab der Version 2.6 SSL benutzen.
      
      Ein zweites, durch den Secure Socket Layer mitgeliefertes Feature ist die
      Identitätskontrolle für den Datenbankserver. Diese funktioniert jedoch
      nur, wenn kein selbstsigniertes Zertifikat für SSL verwendet wird.
      \begin{merke}
        Die zur Verschlüsselung verwendete Schlüsselbreite, 40 oder 128 Bit,
        hängt vom genutzten Betriebssystem ab.
      \end{merke}
      \subsection{Vorbereitende Maßnahmen}
        \begin{merke}
          Für die Nutzung von SSL sollte in jedem Falle ein eigenes Zertifikat
          erstellt werden. Steht des SQL Server kein Zertifikat zur Verfügung,
          nutzt er ein selbstsigniertes Zertifikat. Dies verringert die
          Sicherheit von SSL massiv.
        \end{merke}
        \subsubsection{Erstellen eines Certificate Signing Request}
          Der erste Schritt der bei der Erstellung eines Zertifikats getan
          werden muss, ist die erstellung einer Zertifikatsanforderung, auch
          \enquote{Certificate Signing Request} (CSR) genannt. Dieser wird
          üblicherweise auf dem Rechner erstellt, auf dem auch das Zertifikat
          genutzt werden soll. In einem verschlüsselten Textblock enthält er
          Informationen, die später im Zertifikat vorkommen. Der CSR wird dann
          von einer Zertifizierungstelle bzw. Certificate Authority signiert.
          Aus dem signierten CSR entsteht dann das Zertifikat.
          \begin{literaturinternet}
            \item \cite{sslshopwiaccsr}
          \end{literaturinternet}
          \begin{enumerate}
            \item Starten Sie die Microsoft Management Console (MMC).
            \item Wählen Sie im Menü \enquote{Datei} den Eintrag
            \enquote{snap-In hinzufügen/entfernen}.
            \item Wählen Sie aus der Liste \enquote{Verfügbare Snap-Ins} das
            Snap-In \enquote{Zertifikate} mit einem Doppelklick aus.
            \bild{Das Zertifikats-Snap-In benutzen}{add_snap-in_to_mmc}{1.3}
            \item Wählen Sie im neu erschienenen Dialogfenster die Option
            \enquote{Computerkonto} und klicken Sie auf \enquote{Weiter}.
            \bild{Das Zertifikats-Snap-In
            benutzen}{use_certificate_snap-in_1}{1.3}
            \item Wählen Sie im Dialog \enquote{Computer auswählen} die Option
            \enquote{Lokalen Computer (Computer, auf dem diese Konsole
            ausgeführt wird)} und klicken Sie auf Fertigstellen.
            \bild{Das Zertifikats-Snap-In
            benutzen}{use_certificate_snap-in_2}{1.3}
            \item Klicken Sie links außen den Eintrag \enquote{Eigene
            Zertifikate} an und öffnen Sie das Kontextmenü.
            \item Wählen Sie \enquote{Alle Aufgaben} $\rightarrow$
            \enquote{Erweiterte Vorgänge} $\rightarrow$
            \enquote{Benutzerdefinierte Anforderung erstellen}.
            \bild{Einen benutzer\-definierten Request erstellen}{certmgr_1}{1.3}
            \item Klicken Sie im Vorbereitungs-Dialog auf \enquote{Weiter}.
            \item Wählen Sie im nächsten Schritt den
            Anforderungstyp: \enquote{Den Vorgang ohne Registrierungsrichtlinie
            fortsetzen}
            \bild{Zertifikats\-registrierungs\-richtline wählen}{certmgr_2}{1.2}
            \item Klicken Sie auf \enquote{Weiter}.
\clearpage
            \item Wählen Sie als Vorlagetyp für die benutzerdefinierte
            Anforderung: \enquote{(Keine Vorlage) Legacyschlüssel} und als
            Anforderungsformat \enquote{PKCS\#10}. Klicken Sie auf
            \enquote{Weiter}.
            \bild{Anforderungsvorlage und -format auswählen}{certmgr_3}{1.2}
            \item Klicken Sie im Schritt \enquote{Zertifikatsinformationen}
            auf \enquote{Details} und anschließend auf die
            Schaltfläche \enquote{Eigenschaften}. Es öffnet sich der Dialog
            \enquote{Zertifikateigenschaften}.
            \bild{Anpassen der CSR-Eigenschaften}{certmgr_4}{1.3}
            \item Geben Sie einen Anzeigenamen und eine kurze Beschreibung für
            das Zertifikat ein, so wie in \abbildung{certmgr_5} zu sehen.
            \bild{Allgemeine Angaben zum Zertifikat}{certmgr_5}{1.2}
            \item Wechseln Sie auf die Registerkarte \enquote{Antragsteller}.
            Erstellen Sie die für das Zertifikat notwendigen Einträge:
              
            Antragstellername:
            \begin{itemize}
              \item Allgemeiner Name: FEA11-119SRV12.MS-C-IX-04.FUS
              \item Land/Region: DE
              \item E-Mail: SQL-IX-62-12@MS-C-IX-04.FUS
              \item Ort: Feldafing
              \item Organisation: Bundeswehr
              \item Organisiationseinheit: FueUstgSBw
            \end{itemize}
            \begin{merke}
              Tauschen Sie die Platznummer 12 jeweils durch Ihre eigene
              Platznummer aus!
            \end{merke}
            
            Alternativer Name:
            \begin{itemize}
              \item DNS: FEA11-119SRV12 
              \item DNS: FEA11-119SRV12.MS-C-IX-04.FUS
            \end{itemize}
            \bild{Angaben zum Antragsteller}{certmgr_6}{1.3}
            \item Wechseln Sie auf die Registerkarte \enquote{Erweiterungen}.
            \item Öffnen Sie das Menü \enquote{Erweiterte Schlüsselverwendung
            (Anwendungsrichtlinien)}.
\clearpage
            \item Wählen Sie \enquote{Serverauthentifizierung} bei den
            Verfügbaren Optionen.
            \bild{Zertifikats\-optionen auswählen}{certmgr_7}{1.3}
            \item Wechseln Sie auf die Registerkarte \enquote{Privater
            Schlüssel}.
            \item Öffnen Sie das Menü \enquote{Schlüsseloptionen}.
            \item Wählen Sie 2048 Byte als Schlüsselgröße und aktivieren Sie die
            Option \enquote{Privaten Schlüssel exportierbar machen}.
            \bild{Eigenschaften des privaten Schlüssels ändern}{certmgr_8}{1.3}
            \item Wählen Sie im Menü \enquote{Schlüsseltyp} die Option
            \enquote{Austausch}.
            \item Öffnen Sie das Menü \enquote{Schlüsselberechtigungen}.
            \item Aktivieren Sie die Option \enquote{Benutzerdefinierte
            Berechtigungen verwenden} und klicken Sie auf \enquote{OK}.
            \item Klicken Sie auf \enquote{Weiter}.
            \item Geben Sie als Dateinamen \oscommand{D:\textbackslash
            FEA11-119SRV12.MS-C-IX-04.FUS.csr} ein und achten Sie darauf, dass
            als Dateiformat \enquote{Base 64} gewählt ist.
            \item Klicken Sie auf \enquote{Fertigstellen}
          \end{enumerate}
          Die erstellte Anforderung muss nun durch eine CA signiert werden.
        \subsubsection{Importieren des Serverzertifikats}
          Das Importieren des Serverzertifikats geschieht, genau wie das
          Erstellen des Requests, mit Hilfe des Zertifikatplugins auf dem
          Datenbankserver.
          \begin{enumerate}
            \item Starten Sie das Zertifikatsplugin.
            \item Klicken Sie links außen den Eintrag \enquote{Eigene
            Zertifikate} an und öffnen Sie das Kontextmenü.
            \item Wählen Sie \enquote{Alle Aufgaben} $\rightarrow$
            \enquote{Erweiterte Vorgänge} $\rightarrow$
            \enquote{Importieren}.
            \item Klicken Sie im Willkommensdialog auf \enquote{Weiter}.
            \item Geben Sie als Dateinamen \oscommand{D:\textbackslash
            fea11-119srv12.ms-c-ix-04.fus.pem} ein.
            \bild{Importieren eines Zertifikats}{import_cert_1}{1.3}
            \item Wählen Sie im nächsten Schritt die Option \enquote{Alle
            Zertifikate in folgendem Speicher speichern} und wählen Sie als
            Speicher \enquote{Eigene Zertifikate} aus. Klicken Sie auf
            \enquote{Weiter}.
            \bild{Auswählen des Zertifikats\-speichers}{import_cert_2}{1.3}
            \item Klicken Sie auf \enquote{Fertigstellen}.
          \end{enumerate}
          \bild{Das importierte Server-Zertifikat}{imported_server_cert}{1.5}
          \begin{merke}
            Es ist wichtig, dass das Zertifikat in den Zertifikatsspeicher des
            Computerkontos importiert wird. Nur so können verschiedene
            Benutzerkonten berechtigt werden, auf das Zertifikat zuzugreifen.
          \end{merke}
        \subsubsection{Festlegen der Zugriffsrechte auf den privaten Schlüssel}
          Nach dem Import des Zertifikats in den Speicher des Computerkontos
          können bisher nur die drei folgenden Konten darauf zugreifen:
          \begin{itemize}
            \item Lokaler Dienst (\identifier{nt authority\textbackslash localservice})
            \item Lokales System (\identifier{nt authority\textbackslash localsystem})
            \item Netzwerkdienst (\identifier{nt authority\textbackslash networkservice})
          \end{itemize}
          Damit das gMSA-Dienstkonto des SQL Server
          (\identifier{MS-C-IX-04\textbackslash MSSQLSERVER12\$}) das Zertifikat nutzen kann,
          muss im der Zugriff auf den privaten Schlüssel des Zertifikats
          eingeräumt werden.
          \begin{enumerate}
            \item Öffnen Sie den Zertifikatsmanager für das Computerkonto.
            \item Klicken Sie mit der rechten Maustaste auf das Zertifikat und
            wählen Sie aus dem Kontextmenü \enquote{Alle Aufgaben} $\rightarrow$
            \enquote{Private Schlüssel verwalten}.
            \bild{Verwalten der privaten Schlüssel}{manage_private_keys_1}{1.3}
            \item Fügen Sie das Dienstkonto des SQL Servers hinzu und geben Sie
            im Leserechte auf das Zertifikat.
            \bild{Verwalten der privaten Schlüssel}{manage_private_keys_2}{1.3}
          \end{enumerate}
        \subsubsection{Importieren des CA-Zertifikats}
          Um die Echtheit des Serverzertifikats überprüfen zu können, muss der
          SQL Server auch das CA-Zertifikat importieren.
          \bild{Fehlende Zertifikats\-informationen}{unknown_root_cert}{1.5}
          Die Vorgehensweise ist die gleiche, wie beim Import des
          Serverzertifikats. Es ist lediglich darauf zu achten, dass im zweiten
          Schritt des Assistenten als Zertifikatsspeicher
          \enquote{Vertrauenswürdige Stammzertifizierungsstellen} gewählt
          werden muss. Die Anschließende Sicherheitswarung muss mit \enquote{Ja}
          bestätigt werden.
          \bild{Das importierte CA-Zertifikat}{imported_root_cert}{1.5}
          Mit Hilfe des CA-Zertifikats kann nun die Echtheit des
          Serverzertifikats nachgewiesen werden.
          \bild{Vollständige Certificate chain}{known_root_cert}{1.3}
      \subsection{Konfigurieren der serverseitigen Verschlüsselung}
        Aufseiten des SQL Servers muss in der Netzwerkkonfiguration das
        Serverzertifikat angegeben und die Verschlüsselung aktiviert werden.
        Diese Schritte werden alle im SQL Server Configuration Manager
        ausgeführt.
        \begin{enumerate}
          \item Starten Sie den SQL Server Configuration Manager.
          \item Öffnen Sie die mit Hilfe des Kontextmenüs die Eigenschaften
          der SQL Server Netzwerkprotokolle.
          \bild{Die SQL
          Server-Netzwerk\-protokolle}{sql_server_network_properties}{1.3}
          \item Wechseln Sie im Eigenschaftendialog auf die Registerkarte
          \enquote{Zertifikat} und wählen Sie in der Dropdownbox
          \enquote{Zertifikat} das installierte Serverzertifikat aus.
          \begin{merke}
            Hier können nur solche Zertifikate ausgewählt werden, die vorher mit
            dem Zertifikatsplugin der MMC installiert wurden und die den von
            Microsoft formulierten Anforderungen für ein Serverzertifikat
            entsprechen.
          \end{merke}
          \begin{literaturinternet}
            \item \cite{ms189067}
            \item \cite{ms191192}
          \end{literaturinternet}
\clearpage
          \bild{Die Register\-karte
          \enquote{Zertifikat}}{sql_server_network_properties_1}{1.3}
          \item Wechseln Sie zurück auf die Registerkarte \enquote{Flags}.
          Aktivieren Sie dort die Option \enquote{Verschlüsselung erzwingen}.
          \bild{Die Register\-karte
          \enquote{Flags}}{sql_server_network_properties_2}{1.3}
          \item Klicken Sie auf \enquote{OK} und starten Sie den SQL
          Server-Dienst neu.
        \end{enumerate}
        \begin{merke}
          Eine unverschlüsselte Kommunikation mit dem SQL Server ist nun nicht
          mehr möglich.
        \end{merke}
      \subsubsection{Probleme bei der Auswahl des Zertifikats}
        Es kann vorkommen, dass der SQL Server nicht in der Lage ist, ein
        korrekt erstelltes und in den Speicher des Computerkontos
        importiertes Zertifikat zu erkennen. Dies äußert sich so, dass das
        Zertifikat auf der Registerkarte \enquote{Zertifikat} der
        Netzwerkprotokolleigenschaften des Servers nicht zur Auswahl steht.
        \bild{Kein Zertifikat zur Auswahl}{sql_server_network_properties_3}{1.3}
        In einem solchen Fall, kann zu einer Notlösung gegriffen werden. Der
        Fingerprint des Zertifikats wird direkt in die Registry eingetragen.
        Dieser Vorgang wird von im Problemfall von Microsoft als offizielle
        Lösung empfohlen.
        \begin{itemize}
            \item Lesen Sie die Eigenschaft \enquote{Fingerabdruck} aus Ihrem
            Zertifikat aus.
            \bild{Der Fingerabdruck des
            Zertifikats}{certificate_in_registry_fingerprint}{1.3}
            \item Öffnen Sie den Registrierungseditor mit dem Kommando
            \enquote{regedit.exe}
            \item Navigieren Sie zu dem folgenden Schlüssel:
            
            \begin{small}
            \texttt{\textbackslash\textbackslash
            HKEY\_LOCAL\_MACHINE\textbackslash SOFTWARE\textbackslash
            Microsoft\textbackslash Microsoft SQL Server\textbackslash
            MSSQL12.MSSQLSERVER}
            
            \texttt{\textbackslash MSSQLSerer\textbackslash
            SuperSocketNetLib}
            \end{small}
            \item Wählen Sie dort den Registryeintrag \enquote{Certificate} und
            tragen Sie den Fingerprint Ihres Zertifikats ein.
            \bild{Eintragen des Fingerabdrucks}{certificate_in_registry_1}{1.3}
            \begin{merke}
              Der Fingerabdruck muss ohne Leerzeichen eingetragen werden. Es
              dürfen auch keine unsichtbaren Steuerzeichen in den
              Registryeintrag hineinkopiert werden. Das Eintragen sollte deshalb
              von Hand erfolgen.
            \end{merke}
            Der SQL Server-Dienst muss nun neugestartet werden. Sofern mit dem
            Zertifikat alles in Ordnung ist, wird der Dienst wie erwartet
            starten.
        \end{itemize}
      \subsection{Konfigurieren der clientseitigen Verschlüsselung}
        Auf der Clientseite kann die Verschlüsselung auf zwei unterschiedliche
        Arten konfiguriert werden:
        \begin{itemize}
          \item Die Verschlüsselung kann für jede Applikation einzeln
          aktiviert/deaktiviert werden.
          \item Die Verschlüsselung kann für alle Anwendungen auf einem Client
          erzwungen werden.
        \end{itemize}
        Wichtig ist, dass der Clientrechner entweder das original
        Serverzertifikat oder das Zertifizierungsstammstellenzertifikat zur
        Verfügung haben muss, um eine sichere SSL-Verbindung mit dem
        Datenbankserver herstellen zu können.
        \subsubsection{Aktivieren/Deaktivieren der Verschlüsselung in einer Anwendung}
          Um eine Verschlüsselte Verbindung zu einem SQL Server aufbauen zu
          können, muss einen Client-Anwendung einen Datenbanktreiber benutzen,
          der diese Möglichkeit bietet. Ein Beispiel für eine solche Anwendung
          ist das Microsft SQL Server Management Studio. Es benutzt den
          \enquote{SQL Native Client 11.0} von Microsoft. Um die Verschlüsselung
          zu aktivieren, muss im Anmeldedialog die Option \enquote{Verbindung
          verschlüsseln} auf der Registerkarte
          \enquote{Verbindungseigenschaften}. Dies kann durch einen Klick auf
          die Schaltfläche \enquote{Optionen} erreicht werden.
          \bild{Verschlüsseln der
          Verbin\-dung im
          SSMS}{ssms_connection_settings_encrypt_connection}{1.3}
        \subsubsection{Erzwingen der Verschlüsselung für alle Anwendungen}
          Im SQL Server Configuration Manager können die SQL Native Client
          Komponenten so konfiguriert werden, dass alle Anwendungen, welche
          diesen Datenbanktreiber benutzen, ihre Verbindung verschlüsseln.
          \bild{Erzwingen der
          Verschlüs\-selung auf
          dem Client}{enforce_encryption_on_client}{1.3}
          Dieser Dialog wird über den Eintrag \enquote{Eigenschaften} des
          Kontextmenüs des Eintrags \enquote{SQL Native Client
          11.0-Konfiguration} erreicht. Die beiden dort sichtbaren Optionen
          haben folgende Bedeutung:
          \begin{itemize}
            \item \textbf{Protokollverschlüsselung erzwingen}: Wird diese Option
            auf den Wert \enquote{Ja} eingestellt, fordert der Client eine
            SSL-Verbindung zum Server an. Der Verbindung wird dann in jedem Fall
            verschlüsselt.
            \item \textbf{Vertrauenswürdiges Serverzertifikat}: Solange diese
            Option den Wert \enquote{Nein} hat, überprüft der Client die
            Identität des Servers. So können u. U. Man-In-The-Middle-Attacken
            verhindert werden.
          \end{itemize}
          \begin{merke}
            Für die Überprüfung der Serveridentität muss dem Client entweder das
            original Serverzertifikat oder das
            Zertifizierungsstammstellenzertifikat zur Verfügung stehen. Der
            Import dieses Zertifikats verläuft analog zum Import auf dem
            Datenbankserver.
          \end{merke}
      \subsection{Ist die Verbindung auch wirklich verschlüsselt?}
        Die Antwort auf diese Frage kann nur die Katalogsicht
        \identifier{sys.dm\_exec\_connections} geben. Sie enthält eine Spalte
        \identifier{encrypt\_option}, welche für verschlüsselte Verbindungen den
        Wert TRUE anzeigt.
        \begin{lstlisting}[language=ms_sql,caption={Verschlüsselungseigenschaften
        der Verbindung abfragen},label=sql19_01]
SELECT net_transport, client_net_address, encrypt_option
FROM   sys.dm_exec_connections;
        \end{lstlisting}
        \begin{center}
          \begin{small}
            \changefont{pcr}{m}{n}
            \tablefirsthead{
              \multicolumn{1}{l}{\textbf{NET\_TRANSPORT}} &
              \multicolumn{1}{l}{\textbf{CLIENT\_NET\_ADDRESS}} &
              \multicolumn{1}{l}{\textbf{ENCRYPT\_OPTION}} \\
              \cmidrule(l){1-1}\cmidrule(l){2-2}\cmidrule(l){3-3}
            }
            \tablehead{}
            \tabletail{
              \multicolumn{2}{l}{\textbf{2 Zeilen ausgew\"ahlt}} \\
            }
            \tablelasttail{
              \multicolumn{2}{l}{\textbf{2 Zeilen ausgew\"ahlt}} \\
            }
            \begin{mssql}
              \begin{supertabular}{lll}
                Shared memory & <local machine> & TRUE \\
                TCP & 192.168.111.43 & TRUE \\
              \end{supertabular}
            \end{mssql}
          \end{small}
        \end{center}
        
      \subsection{Nutzen der Option \enquote{Erweiterter Schutz}}
        Der Erweiterte Schutz ist eine Funktion, die seit Windows Server 2008
        R2/Windows 7 durch das Betriebssystem zur Verfügung gestellt wird. Sie
        beinhaltet zwei Komponenten, die Dienstbindung und die Kanalbindung,
        welche in Kombination als Abwehrmaßnahme gegen Relay-Attacken
        funktionieren. Standardmäßig ist der Erweiterte Schutz nicht aktiviert,
        dies muss manuell durch den Administrator geschehen.
        \subsubsection{Wovor schützt der Erweiterte Schutz?}
          Laut Microsoft kann dieser Mechanismus vor zwei Angriffsarten
          schützen:
          \begin{itemize}
            \item \textbf{Lockangriff}: Der Client wird dazu verleitet
            (angelockt) freiwillige eine Verbindung zu einem Angreifer
            herzustellen.
            \item \textbf{Spoofingangriff}: Bei einem solchen Angriff wird dem
            Client eine manipulierte Route zum Ziel zur Verfügung gestellt, so
            dass er unbemerkt beim Angreifer landet.
          \end{itemize}
        \subsubsection{Welche Schutzmaßnahmen bietet der Erweiterte Schutz?}
          Durch das Aktivieren des Erweiterten Schutzes werden zwei Mechanismen
          in Gang gesetzt:
          \begin{itemize}
            \item \textbf{Dienstbindung}: Bei der Dienstbindung signiert der
            Client den Service Principal Name (SPN) des Servers und sendet
            diesen verschlüsselt an den Server. Durch die Signatur des SPN kann
            der Server erkennen von welchem Client der SPN stammt. Durch die
            Verschlüsselung kann kein anderer Client den SPN für seine Zwecke
            mißbrauchen.
            \bild{Schematische Dar\-stellung der
            Dienst\-bindung}{service_binding}{1.3}
            \item \textbf{Kanalbindung}: Bei der Kanalbindung wird ein sicherer
            Kanal zwischen dem Client und dem Server eingerichtet. Diese Methode
            wird mit Hilfe des TLS (ehemals SSL) Protokolls umgesetzt.
          \end{itemize}
          \begin{literaturinternet}
            \item \cite{ff487261}
          \end{literaturinternet}          
        \subsubsection{Den Erweiterten Schutz in Windows aktivieren}
          Die Funktion \enquote{Erweiterer Schutz} kann nur durch zwei
          Registry-Einträge aktiviert werden. Der Administrator muss prüfen, ob
          die Schlüssel
          
          \begin{small}
            \texttt{HKEY\_LOCAL\_MACHINE\textbackslash
            SYSTEM\textbackslash
            CurrentControlSet\textbackslash
            Control\textbackslash Lsa\textbackslash SuppressExtendedProtection}
            
            und
            
            \texttt{HKEY\_LOCAL\_MACHINE\textbackslash
            SYSTEM\textbackslash
            CurrentControlSet\textbackslash Control\textbackslash
            Lsa\textbackslash LmCompatibilityLevel}
          \end{small}
          
          bereits existiert. Falls dies nicht der Fall ist, müssen die Schlüssel
          manuell angelegt werden (Schlüsseltyp = DWORD).
          \begin{enumerate}
            \item Starten Sie den Registrierungseditor mit
            \oscommand{regedit.exe}
            \item Navigieren Sie durch den Registrierungsbaum bis Sie den
            folgenden Schlüssel
            erreichen: \texttt{HKEY\_LOCAL\_MACHINE\textbackslash
            SYSTEM\textbackslash CurrentControlSet\textbackslash Control\textbackslash Lsa}
            \item Öffnen Sie das Kontextmenü mit der rechten Maustaste und
            wählen Sie \enquote{Neu} $\rightarrow$ \enquote{DWORD-Wert (32-Bit)}
            \bild{Anlegen eines DWORD-Schlüssels in der
            Registry}{regedit_new_key_1}{1.3}
            \item Geben Sie dem neuen Wert den Namen
            \texttt{SuppressExtendedProtection}.
            \bild{Der Schlüssel
            Suppress\-Extended\-Protection}{regedit_new_key_2}{1.3}
            \item Der Schlüssel muss den Wert 0 haben.
            \item Öffnen Sie das Kontextmenü mit der rechten Maustaste und
            wählen Sie \enquote{Neu} $\rightarrow$ \enquote{DWORD-Wert (32-Bit)}
            \item Geben Sie dem neuen Wert den Namen
            \texttt{LmCompatibilityLevel}.
            \bild{Der Schlüssel
            LmCompa\-tibility\-Level}{regedit_new_key_3}{1.3}
            \item Doppelklicken Sie auf den Wert \texttt{LmCompatibilityLevel}. 
            \item Geben Sie den Wert 3 ein und klicken Sie auf OK.
            \bild{Der Schlüssel
            LmCompa\-tibility\-Level}{regedit_new_key_4}{1.3}
            \item Starten Sie den Windows Server neu.
          \end{enumerate}
        \subsubsection{Den Erweiterten Schutz für SQL Server aktivieren}
          Der Erweiterte Schutz für den SQL Server wird mit Hilfe von drei
          Einstellungen im SQL Server-Konfigurations-Manager verwaltet:
          \begin{itemize}
            \item \textbf{Erzwingen der Verschlüsselung}: Durch die
            Konfiguration dieser Eigenschaft auf den Wert \texttt{Ein} wird die
            Kanalbindung erzwungen, da der SQL Server die Verschlüsselung einer
            jeden Verbindung erzwingt.
            \item \textbf{Erweiterter Schutz}: Diese Eigenschaft besitzt drei
            unterschiedliche Werte: \texttt{Aus}, \texttt{Zulässig} und
            \texttt{Erforderlich}.
            
            Durch die Einstellung \texttt{Zulässig} wird die Dienstbindung bei
            allen Clients benutzt die dies unterstützen. Sollte ein Client keine
            Dienstbindung zur Verfügung stellen wird die Verbindung ohne
            Dienstbindung aufgebaut.
            
            Durch die Einstellung \texttt{Erforderlich} wird die Dienstbindung
            für alle Clients erzwungen. Sollte ein Client keine Dienstbindung
            zur Verfügungstellen wird keine Verbindung aufgebaut.
            \item \textbf{Akzeptierte NTLM-SPNs}: Diese Eigenschaft wird nur
            dann benötigt, wenn ein Datenbankserver durch mehrere SPN
            identifiziert wird. In diesem Fall kann hier eine durch Semikola
            separierte Liste von Service Principal Names angegeben werden.
          \end{itemize}
          Zufinden sind diese Eigenschaften durch einen Rechtsklick auf den
          Punkt \enquote{Protokolle für 'MSSQLSERVER'}
          \bild{Aktivieren des
          Erweiterten
          Schutzes}{sql_server_extended_protection_properties_1}{1.3}
          Die Einstellung \enquote{Verschlüsselung erzwingen} muss auf den Wert
          \enquote{Ja} eingestellt werden, um die Kanalbindung zu ermöglichen.
          \bild{Aktivieren des
          Erweiterten
          Schutzes}{sql_server_extended_protection_properties_2}{1.3}
          Auf der Registerkarte \enquote{Erweitert} muss die Einstellung
          \enquote{Erweiterter Schutz} auf den Wert \enquote{Erforderlich}
          eingestellt werden, um Kanal- und Dienstbindung zu erzwingen.
          \bild{Aktivieren des
          Erweiterten
          Schutzes}{sql_server_extended_protection_properties_3}{1.3}
          \begin{literaturinternet}
            \item \cite{ff487261}
          \end{literaturinternet}
    
    
