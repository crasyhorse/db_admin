\clearpage
    \section{Lösungen - Verwalten einer SQL Server-Instanz}
      \begin{enumerate}
              \item Ermitteln Sie, ob der serverseitige Result Cache aktiviert ist!

          
          Fügen Sie Ihrem Computer die \enquote{Richtlinienverwaltung} und die
          \enquote{Remoteverwaltungstools} hinzu!
          
              \item Welchen temporären Tablespace nutzt alice jetzt und welcher View können Sie diese Angabe entnehmen?

          \begin{lstlisting}[language=powershell, caption={Aufnehmen eines
          Computers in eine Windows Domäne}, label=admin_03_loesung_01]
~$ou~ = "OU=Computers,OU=Serverxx,OU=EXER,OU=IT-SysAdmin,DC=MS-C-IX-04-DC=FUS"
Add-Computer |-ComputerName| FEA11-119SRVxx `
|-LocalCredential| FEA11-119SRVxx\Administrator `
|-DomainName| MS-C-IX-04.FUS `
|-Credential| MS-C-IX_04\Administrator `
|-OUPath| ~$ou~
|-Restart| |-Force|
          \end{lstlisting}
          
              \item Erstellen Sie den Nutzer \identifier{bob} mit dem Passwort \enquote{Pass1/gH,3word}.

          \bild{Den Shared Memory-Zugriff
          deaktivieren}{deactivate_shared_memory}{2}
\clearpage          
              \item Welchen Default Tablespace benutzt \identifier{bob} und aus welcher View können Sie diese Angabe entnehmen?

          \bild{IP-Adressen deaktivieren}{deactivate_ip_addresses}{2}
          
              \item F\"uhren Sie ein Recovery bei Verlust einer Kontrolldatei durch.
      \begin{enumerate}
        \item Starten Sie das Skript \oscommand{lab\_delete\_ctlfl.sql}. Es l\"oscht eine Ihrer Kontrolldateien.
          \begin{lstlisting}[language=terminal]
SQL> @/home/oracle/labs/lab_delete_ctlfl.sql
          \end{lstlisting}
        \item Analysieren Sie das Problem und f\"uhren Sie ein geeignetes Recovery durch.
      \end{enumerate}

          \begin{lstlisting}[language=powershell, caption={Die notwendigen
          gMSAs erstellen}, label=admin_03_loesung_02]
~$gou~ = "OU=GROUPS,OU=Serverxx,OU=EXER,OU=IT-SysAdmin,DC=MS-C-IX-04-DC=FUS"
New-ADGroup |-Name| SQLServerSVCxx |-GroupScope| Global `
|-GroupCategory| Security `
|-Path| ~$gou~

~$cou~ = "OU=Computers,OU=Serverxx,OU=EXER,OU=IT-SysAdmin,DC=MS-C-IX-04-DC=FUS"
~$identity~ = "CN=SQLServerSVCxx," + ~$cou~
~$member~ = "CN=FEA11-119SRVxx," + ~$cou~

Add-ADGroupMember |-Identity| ~$identity~ |-Members| $member
          \end{lstlisting}
\clearpage
          \begin{lstlisting}[language=powershell]           
~$mou~ = "OU=gMSAs,OU=Serverxx,OU=EXER,OU=IT-SysAdmin,DC=MS-C-IX-04-DC=FUS"

New-ADServiceAccount |-Name| MSSQLSRVRxx `
|-DNSHostname| FEA11-119SRVAD.MS-C-IX-04.FUS `
|-Path| ~$mou~ `
|-PrincipalsAllowedToRetrieveManagedPassword| ~$identity~ `
|-Enabled| ~$true~

New-ADServiceAccount |-Name| MSSQLAGENTxx `
|-DNSHostname| FEA11-119SRVAD.MS-C-IX-04.FUS `
|-Path| ~$mou~ `
|-PrincipalsAllowedToRetrieveManagedPassword| ~$identity~ `
|-Enabled| ~$true~

New-ADServiceAccount |-Name| MSDTSSRVRxx `
|-DNSHostname| FEA11-119SRVAD.MS-C-IX-04.FUS `
|-Path| ~$mou~ `
|-PrincipalsAllowedToRetrieveManagedPassword| ~$identity~ `
|-Enabled| ~$true~

New-ADServiceAccount |-Name| MSRPTSRVRxx `
|-DNSHostname| FEA11-119SRVAD.MS-C-IX-04.FUS `
|-Path| ~$mou~ `
|-PrincipalsAllowedToRetrieveManagedPassword| ~$identity~ `
|-Enabled| ~$true~

New-ADServiceAccount |-Name| MSSQLBRWSRxx `
|-DNSHostname| FEA11-119SRVAD.MS-C-IX-04.FUS `
|-Path| ~$mou~ `
|-PrincipalsAllowedToRetrieveManagedPassword| ~$identity~ `
|-Enabled| ~$true~

#Execute these steps as local admin on your computer

~$identity~ = "CN=MSSQLSRVRxx," + ~$mou~
Install-ADServiceAccount |-Identity| ~$identity~

~$identity~ = "CN=MSSQLAGENTxx," + ~$mou~
Install-ADServiceAccount |-Identity| ~$identity~

~$identity~ = "CN=MSDTSSRVRxx," + ~$mou~
Install-ADServiceAccount |-Identity| ~$identity~

~$identity~ = "CN=MSRPTSRVRxx," + ~$mou~
Install-ADServiceAccount |-Identity| ~$identity~

~$identity~ = "CN=MSSQLBRWSRxx," + ~$mou~
Install-ADServiceAccount |-Identity| ~$identity~
          \end{lstlisting}        
            \item Machen Sie alle Auditingeinstellungen r\"uckg\"angig und l\"oschen Sie den Inhalt des Auditingtrails!

        \begin{lstlisting}[language=powershell, caption={Stoppen und
        umkonfigurieren von Diensten}, label=admin_03_loesung_03]
Set-Service |-DisplayName| 'SQL Server Integration Services 12.0' `
|-StartupType| Manual

Set-Service |-DisplayName| 'SQL Server Reporting Services (MSSQLSERVER)' `
|-StartupType| Manual

Stop-Service |-DisplayName| 'SQL Server Integration Services 12.0'
Stop-Service |-DisplayName| 'SQL Server Reporting Services (MSSQLSERVER)'     
        \end{lstlisting}
        \item Konfigurieren Sie Ihre Instanz MSSQLSERVER so, dass Sie immer
mindestens 200M Arbeitsspeichern zur Verfügung hat und nie mehr als 2G
benutzt!

        \begin{lstlisting}[language=ms_sql, caption={Ändern der
        Servereigenschaften}, label=admin_03_loesung_04]
EXEC sp_configure 'min server memory (MB)', '200'
EXEC sp_configure 'max server memory (MB)', '2048'
RECONFIGURE
        \end{lstlisting}
        \item Legen Sie einen nicht gruppierten Index auf die Tabelle
\identifier{eigenkunde} (Spalte \identifier{ablaufdatum}). Der Index soll nur
die Zeilen enthalten, die einen Personalausweis zeigen, der vor dem 01.01.2016
abgelaufen ist. Schließen Sie die Spalte \identifier{personalausweisnr} als
Nichtschlüsselspalte in den Index ein.
        \begin{lstlisting}[language=ms_sql, caption={Ändern der
        Servereigenschaften}, label=admin_03_loesung_05]
SELECT name, value, value_in_use
FROM   sys.configurations
WHERE  configuration_id IN (1543, 1544)
GO
        \end{lstlisting}
        \begin{center}
          \begin{small}
            \changefont{pcr}{m}{n}
            \tablefirsthead {
              \multicolumn{1}{l}{\textbf{name}} &
              \multicolumn{1}{l}{\textbf{value}} &
              \multicolumn{1}{l}{\textbf{value\_in\_use}} \\
              \cmidrule(l){1-1}\cmidrule(l){2-2}\cmidrule(l){3-3}
            }
            \tablehead{}
            \tabletail {
              \multicolumn{1}{l}{\textbf{2 Zeilen ausgew\"ahlt}} \\
            }
            \tablelasttail {
              \multicolumn{1}{l}{\textbf{2 Zeilen ausgew\"ahlt}} \\
            }
            \begin{mssql}
              \begin{supertabular}{lll}
                min server memory (MB) & 200 & 200 \\
                max server memory (MB) & 2048 & 2048 \\
              \end{supertabular}
            \end{mssql}
          \end{small}
        \end{center}
            \item Konfigurieren Sie zwei Net Service Names, einen f\"ur eine Dedicated Server Verbindung und einen f\"ur eine DRCP-Verbindung. Testen Sie beide Verbindungen (Hinweis: Um die DRCP-Verbindung testen zu k\"onnen, muss vorher das Connection Pooling aktiviert werden, siehe: \myhref{http://docs.oracle.com/cd/B28359_01/server.111/b28318/process.htm#CIHBIFCI}{Database Resident Connection Pooling, Database Concepts 11g Release 2, Kapitel 9})

        \begin{lstlisting}[language=ms_sql, caption={Ändern der
        Servereigenschaften}, label=admin_03_loesung_06]
USE [master]
GO

EXEC xp_instance_regwrite N'HKEY_LOCAL_MACHINE', 
     N'Software\Microsoft\MSSQLServer\MSSQLServer', 
     N'AuditLevel', REG_DWORD, 0
GO
        \end{lstlisting}
        
            \item Schalten Sie das SQL-Autotracing und das Timing wieder aus: \languagesqlplus{set autotrace off} und \languagesqlplus{set timing off}

        \begin{lstlisting}[language=ms_sql, caption={Ändern der
        Servereigenschaften}, label=admin_03_loesung_07]
USE [master]
GO

CREATE DATABASE [Bank 2014]
  CONTAINMENT = NONE
  ON  PRIMARY 
  ( NAME = N'bank_2014_primary_01', 
    FILENAME = N'D:\u01\bank_2014\data\bank_2014_primary_01.mdf', 
    SIZE = 10 MB, MAXSIZE = 500 MB, FILEGROWTH = 10 MB ), 
  FILEGROUP [CRM] 
  ( NAME = N'bank_2014_crm_01', 
    FILENAME = N'E:\u02\bank_2014\data\bank_2014_crm_01.ndf', 
    SIZE = 50 MB, MAXSIZE = 100 MB, FILEGROWTH = 10 MB ), 
  FILEGROUP [HR] 
  ( NAME = N'bank_2014_hr_01', 
    FILENAME = N'D:\u01\bank_2014\data\bank_2014_hr_01.ndf', 
    SIZE = 100 MB, MAXSIZE = 500 MB, FILEGROWTH = 20 MB ),
  ( NAME = N'bank_2014_hr_02', 
    FILENAME = N'D:\u01\bank_2014\data\bank_2014_hr_02.ndf', 
    SIZE = 100 MB, MAXSIZE = 500 MB, FILEGROWTH = 20 MB ), 
  FILEGROUP [KAM] 
  ( NAME = N'bank_2014_kam_01', 
    FILENAME = N'E:\u02\bank_2014\data\bank_2014_kam_01.ndf', 
    SIZE = 100 MB, MAXSIZE = 500 MB, FILEGROWTH = 100 MB ), 
        \end{lstlisting}
\clearpage
        \begin{lstlisting}[language=ms_sql]        
  FILEGROUP [MAIL] 
  ( NAME = N'bank_2014_mail_01', 
    FILENAME = N'D:\u01\bank_2014\data\bank_2014_mail_01.ndf', 
    SIZE = 500 MB, MAXSIZE = 1 GB, FILEGROWTH = 100 MB ),
  ( NAME = N'bank_2014_mail_02', 
    FILENAME = N'D:\u01\bank_2014\data\bank_2014_mail_02.ndf', 
    SIZE = 500 MB, MAXSIZE = 1 GB, FILEGROWTH = 100 MB ),
  ( NAME = N'bank_2014_mail_03', 
    FILENAME = N'D:\u01\bank_2014\data\bank_2014_mail_03.ndf', 
    SIZE = 500 MB, MAXSIZE = 1 GB, FILEGROWTH = 100 MB )
  LOG ON 
  ( NAME = N'Bank 2014_log', 
    FILENAME = N'F:\u03\bank_2014\log\bank_2014_log.ldf', 
    SIZE = 768 MB, MAXSIZE = 3 GB, FILEGROWTH = 768 MB )
GO
        \end{lstlisting}
        
            \item Weisen Sie den Nutzern \identifier{alice}, \identifier{bob} und \identifier{chloe} das Profil \identifier{p\_clerk} zu und testen Sie die Auswirkungen!

        \begin{lstlisting}[language=ms_sql, caption={Hinzufügen einer
        Dateigruppe mit Datendatei}, label=admin_03_loesung_08]
USE [master]
GO

ALTER DATABASE [Bank 2014] 
ADD FILEGROUP [STAGING]
GO

ALTER DATABASE [Bank 2014] 
ADD FILE (NAME = N'bank_2014_staging', 
          FILENAME = N'E:\u02\bank_2014\data\bank_2014_staging_01.ndf', 
          SIZE = 800 MB, MAXSIZE = 2 GB, FILEGROWTH = 40 MB) 
TO FILEGROUP [STAGING]
GO
        
        \end{lstlisting}
\clearpage        
            \item Melden Sie sich als Nutzer \identifier{alice} (Passwort: Welcome01\#) an Ihrer Datenbank an und versuchen Sie das folgende SQL-Statement abzusetzen:
    \begin{lstlisting}[language=oracle_sql]
&SQL>& INSERT INTO full
  2 VALUES (1,1,1);
    \end{lstlisting}
    Die Ausf\"uhrung des Statements wird scheitern. Pr\"ufen Sie im Alert.log, warum das Statement gescheitert ist.

        \begin{lstlisting}[language=ms_sql, caption={Hinzufügen einer
        Datendatei zu einer bestehenden Dateigruppe}, label=admin_03_loesung_09]
USE [master]
GO

ALTER DATABASE [Bank 2014] 
ADD FILE (NAME = N'bank_2014_crm_02', 
          FILENAME = N'E:\u02\bank_2014\data\bank_2014_crm_02.ndf', 
          SIZE = 50 MB, MAXSIZE = 500 MB, FILEGROWTH = 10 MB )
TO FILEGROUP [CRM]
GO        
        \end{lstlisting}
        
          \item Ändern Sie die Sperrdauer des Accounts im Profil \identifier{p\_clerk} auf unbegrenzt!

      \begin{lstlisting}[language=ms_sql, caption={Ändern der
      Datendateieigenschaften}, label=admin_03_loesung_10]
USE [master]
GO

ALTER DATABASE [Bank 2014] 
MODIFY FILE ( NAME = N'bank_2014_hr_01', MAXSIZE = 1 GB )
GO

ALTER DATABASE [Bank 2014] 
MODIFY FILE ( NAME = N'bank_2014_hr_02', MAXSIZE = 1 GB )
GO

ALTER DATABASE [Bank 2014] 
MODIFY FILE ( NAME = N'bank_2014_kam_01', SIZE = 500 MB, MAXSIZE = 2 GB )
GO

ALTER DATABASE [Bank 2014]
MODIFY FILE ( NAME = N'bank_2014_primary_01', MAXSIZE = 768 MB )
GO
      \end{lstlisting}
\clearpage
          \item L\"oschen Sie das Nutzerprofil \identifier{p\_clerk} in einem Arbeitsschritt!

      \begin{lstlisting}[language=ms_sql, caption={Abfragen der
      Datendateieigenschaften}, label=admin_03_loesung_11]
USE [Bank 2014]
GO

SELECT name, size * 8 / 1024 AS Size, 
       growth * 8 / 1024 AS Growth, 
       max_size * 8 / 1024 AS Max_Size
FROM   sys.database_files
      \end{lstlisting}
      
      \item Lesen Sie den Artikel \parencite{utilUDPaC} und notieren Sie sich
mindestens drei sinnvolle Fragen!

  \rule{0.94\textwidth}{0.5pt}

  \rule{0.94\textwidth}{0.5pt}

  \rule{0.94\textwidth}{0.5pt}

  \rule{0.94\textwidth}{0.5pt}

  \rule{0.94\textwidth}{0.5pt}

  \rule{0.94\textwidth}{0.5pt}

  \rule{0.94\textwidth}{0.5pt}

  \rule{0.94\textwidth}{0.5pt}

      \begin{lstlisting}[language=ms_sql, caption={Verschieben einer
      Datendatei}, label=admin_03_loesung_12]
USE [master]
GO

ALTER DATABASE [Bank 2014]
SET OFFLINE;

-- Move Datafile on filesystem
-- Change filesystemrights

ALTER DATABASE [Bank 2014]
MODIFY FILE ( NAME = N'bank_2014_kam_01',
FILENAME = N'D:\u01\bank_2014\data\bank_2014_kam_01.ndf' )
GO

ALTER DATABASE [Bank 2014]
SET ONLINE;     
      \end{lstlisting}

          \item Verschieben Sie die Datendatei \oscommand{uebungs\_ts02.dbf} nach \oscommand{/u02/oradata/orcl/}.

      \begin{lstlisting}[language=ms_sql, caption={Ändern der
      Standarddateigruppe}, label=admin_03_loesung_12]
USE [Bank 2014]
GO

IF NOT EXISTS (
   SELECT name 
   FROM sys.filegroups 
   WHERE is_default=1 AND name = N'CRM')
     ALTER DATABASE [Bank 2014]
     MODIFY FILEGROUP [CRM] DEFAULT
GO
      \end{lstlisting}
            
            \item Um die weiteren Übungsaufgeben bearbeiten zu können, führen Sie bitte das Skript \oscommand{lab\_java.sql} als Nutzer \identifier{SYS} aus. Das SQL-Skript befindet sich im Verzeichnis \oscommand{/home/oracle/labs}.
        \begin{lstlisting}[language=terminal]
SQL> @/home/oracle/labs/lab_java.sql
        \end{lstlisting}

      \begin{lstlisting}[language=ms_sql, caption={Die Datenbankoption
      quoted\_identifier ändern}, label=admin_03_loesung_13]
USE [master]
GO

ALTER DATABASE [Bank 2014] 
SET QUOTED_IDENTIFIER ON 
WITH NO_WAIT
GO 
      \end{lstlisting}
      
      \item Ermitteln Sie welchen Wert die Datenbankoption \identifier{auto\_close}
hat und recherchieren Sie, welche Bedeutung diese Option hat bzw. was sie
bewirkt.

      
      Die Datenbankoption \identifier{auto\_close} bewirkt, dass die Datenbank
      geschlossen wird, sobald sich der letzte noch aktive Nutzer abgemeldet
      hat. Dies hat zur Folge, dass alle im Arbeitspeicher befindlichen Teile
      der Datenbank von dort entfernt werden, wodurch die Performance dieser
      Datenbank sehr starkt negativ beeinflusst wird.
      
            \item Passen Sie das Setup Ihrer Redo Log Gruppen so an, dass Sie drei Gruppen mit je 3 Membern haben. Verteilen Sie die Member sinnvoll \"uber die vorhandenen Datentr\"ager \oscommand{/u01} bis \oscommand{/u05} und notieren Sie sich die \"Anderungen in der obigen Tabelle.

      \begin{lstlisting}[language=ms_sql, caption={Verschieben der
      Systemdatenbank tempdb}, label=admin_03_loesung_14]
USE [master]
GO

ALTER DATABASE [tempdb]
MODIFY FILE ( NAME = N'tempdev',
              FILENAME = N'G:\u04\tempdb\data\tempdb.mdf' )
GO

ALTER DATABASE [tempdb]
MODIFY FILE ( NAME = N'templog',
              FILENAME = N'F:\u03\tempdb\log\templog.ldf' )
GO

-- Shutdown the instance

-- Move file on filesystem to their new locations

-- Change filesystemrights

-- Startup the instance
      \end{lstlisting}
\clearpage      
          \item Pr\"ufen Sie im RMAN welche Backups nun nicht mehr zur Verf\"ugung stehen.

      \bild{Verschieben des errorlog}{move_error_log}{0.48}
      
            \item Ändern Sie die Parameter der Archivierung wie folgt:
        \begin{center}
          \begin{small}
            \tablefirsthead{
              \multicolumn{1}{c}{\textbf{Optional}} &
              \multicolumn{1}{c}{\textbf{/u02}} &
              \multicolumn{1}{c}{\textbf{/u03}} &
              \multicolumn{1}{c}{\textbf{/u04}} &
              \multicolumn{1}{c}{\textbf{/u05}} \\
              \hline
            }
            \tablefirsthead{
              \multicolumn{1}{c}{\textbf{Optional}} &
              \multicolumn{1}{c}{\textbf{/u02}} &
              \multicolumn{1}{c}{\textbf{/u03}} &
              \multicolumn{1}{c}{\textbf{/u04}} &
              \multicolumn{1}{c}{\textbf{/u05}} \\
              \hline
            }
            \tabletail{
              \hline
            }
            \tablelasttail {
              \hline
            }
            \begin{supertabular}{| >{\centering\arraybackslash}m{1cm}| >{\centering\arraybackslash}m{2cm}| >{\centering\arraybackslash}m{2cm}| >{\centering\arraybackslash}m{2cm}| >{\centering\arraybackslash}m{2cm}|}
               Ja &  dest 1 &  - &  - & - \\
              \hline
               Nein &  - &  dest 2 &  - & - \\
              \hline
               Ja &  - &  - &  dest 3 & - \\
            \end{supertabular}
          \end{small}
        \end{center}
        \begin{itemize}
          \item Die Archivierung muss an mindestens zwei Speicherorten erfolgreich sein.
          \item Das Dateinamensformat der Archive Log Files muss wie folgt aufgebaut sein: \oscommand{archive\_\%d\_\%t\_\%s\_\%r.arc}
        \end{itemize}

      \begin{lstlisting}[language=ms_sql, caption={Umbennen einer Datenbank},
      label=admin_03_loesung_15]
USE [master]
GO

ALTER DATABASE [Bank 2014]
Modify Name = Bank_2014
GO
      
      \end{lstlisting}
    \end{enumerate}
