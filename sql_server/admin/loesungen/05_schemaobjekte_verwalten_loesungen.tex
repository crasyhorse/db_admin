\clearpage
    \section{Lösungen - Schemaobjekte verwalten}
      Benutzen Sie für die Durchführung der Übungen die Datenbank
      \identifier{Bank\_2014}.
      \begin{enumerate}
        %Erstellen Sie die Tabelle \identifier{bank} gemäß Ihrem ER-Modell. 
        %Machen Sie das Attribut \identifier{bank\_id} zum Primärschlüssel und
        % achten sie darauf, dass die Tabelle als Heap erstellt wird.
            \item Ermitteln Sie, ob der serverseitige Result Cache aktiviert ist!

        \begin{lstlisting}[language=ms_sql]
USE [Bank_2014]
GO

CREATE TABLE bank (
  Bank_ID         NUMERIC NOT NULL,
  BIC             VARCHAR(11) NOT NULL,
  Name            VARCHAR(50) NOT NULL,
  Strasse         VARCHAR(50),
  Hausnummer      VARCHAR(20),
  PLZ             CHAR(5),
  Ort             VARCHAR(20),
  Rating          VARCHAR(3),
  CONSTRAINT bank_pk PRIMARY KEY NONCLUSTERED (Bank_ID)
)
ON KAM;
GO
        \end{lstlisting}
        
        %Ist es sinnvoll, die Tabelle \identifier{bank} als Heap anzulegen?
        %Begründen Sie Ihre Aussage!
            \item Welchen temporären Tablespace nutzt alice jetzt und welcher View können Sie diese Angabe entnehmen?

        
        Die Nutzung einer Tabelle in Form eines Heaps, anstatt eines gruppierten
        Indizes, kann sinnvoll sein, wenn die Tabelle sehr klein ist und somit
        das Lesen der gesamten Tabelle kostengünstiger ist, als das Durchsuchen
        eines clustered Index. 
        
        In der Praxis kann es hierfür noch weitere Gründe geben, die allerdings
        den Rahmen dieser Unterrichtsunterlage deutlich sprengen würden.

\clearpage
        %Legen Sie ein \UNIQUE-Constraint auf die Spalte \identifier{bic} der
        %Tabelle \identifier{bank}. Die Tabelle muss weiterhin in Form eines
        %Heaps gespeichert bleiben.
            \item Erstellen Sie den Nutzer \identifier{bob} mit dem Passwort \enquote{Pass1/gH,3word}.

        \begin{lstlisting}[language=ms_sql]
USE [Bank_2014]
GO

ALTER TABLE bank
ADD CONSTRAINT bic_uk UNIQUE NONCLUSTERED (BIC);
GO
        \end{lstlisting}
        
        %Fügen Sie Ihrer Datenbank die Dateigruppe \identifier{in\_memory} hinzu
        %und bereiten Sie alles vor, um eine SOT darin anzulegen.
            \item Welchen Default Tablespace benutzt \identifier{bob} und aus welcher View können Sie diese Angabe entnehmen?

        \begin{lstlisting}[language=ms_sql]
USE [Bank_2014]
GO

-- Memory optimized filegroup anlegen
ALTER DATABASE [Bank_2014] 
ADD FILEGROUP [in_memory]
CONTAINS MEMORY_OPTIMIZED_DATA; 
GO

-- Es muss eine Filestream Datendatei geben,
-- um Tabellen anlegen zu können.
ALTER DATABASE [Bank_2014]
ADD FILE (
  NAME =     'bank_2014_in_memory_01',
  FILENAME = 'D:\u01\bank_2014\filestream\in_memory'
)
TO FILEGROUP [in_memory]
GO
        \end{lstlisting}
\clearpage        
        %Legen Sie die Tabelle \identifier{buchung} gemäß Ihrem ER-Modell als SOT
        %an und fügen Sie zwischen die beiden Spalten
        %\identifier{buchungsdatum} und \identifier{konto\_id} die Spalte
        % \identifier{buchungstext} VARCHAR(200) ein! Legen Sie auf die
        %Spalte \identifier{buchungsdatum} einen Index. Legen Sie auch
        %auf die Spalte \identifier{buchungstext} einen Index.
            \item F\"uhren Sie ein Recovery bei Verlust einer Kontrolldatei durch.
      \begin{enumerate}
        \item Starten Sie das Skript \oscommand{lab\_delete\_ctlfl.sql}. Es l\"oscht eine Ihrer Kontrolldateien.
          \begin{lstlisting}[language=terminal]
SQL> @/home/oracle/labs/lab_delete_ctlfl.sql
          \end{lstlisting}
        \item Analysieren Sie das Problem und f\"uhren Sie ein geeignetes Recovery durch.
      \end{enumerate}
        
        \begin{lstlisting}[language=ms_sql]
USE [Bank_2014]
GO

CREATE TABLE Buchung (
  Buchungs_ID       NUMERIC NOT NULL,
  Betrag            NUMERIC(12,2),
  Buchungsdatum     DATETIME2 NOT NULL,
  Buchungstext      VARCHAR(200) COLLATE LATIN1_GENERAL_BIN2 NOT NULL,
  Konto_ID          NUMERIC,
  Transaktions_ID   NUMERIC,
  CONSTRAINT Buchung_pk PRIMARY KEY NONCLUSTERED (Buchungs_ID),
  INDEX      idx_Buchungsdatum NONCLUSTERED (Buchungsdatum),
  INDEX      idx_Buchungstext NONCLUSTERED (Buchungstext)
)
WITH (
  MEMORY_OPTIMIZED = ON
)
GO
        \end{lstlisting}
        %Erstellen Sie die Tabelle \identifier{mitarbeiter} gemäß Ihrem
        % ER-Modell und legen Sie ein \UNIQUE-Constraint auf die Spalte \identifier{sozversnr}. Die
        %Sortierung des Indexes, der zu diesem Constraint gehört soll absteigend
        %sein.
            \item Machen Sie alle Auditingeinstellungen r\"uckg\"angig und l\"oschen Sie den Inhalt des Auditingtrails!

        \begin{lstlisting}[language=ms_sql]
USE [Bank_2014]
GO

CREATE TABLE Eigenkunde (
  Kunden_ID            NUMERIC NOT NULL,
  Geburtsdatum         DATETIME2,
  Strasse              VARCHAR(50),
  Hausnummer           VARCHAR(15),
  PLZ                  CHAR(5),
  Ort                  VARCHAR(20),
  PersonalausweisNr    VARCHAR(9),
  Ausstellungsdatum    DATETIME2,
  Ablaufdatum          DATETIME2,
  Staatsangehoerigkeit VARCHAR(20),
  Geburtsort           VARCHAR(50),
  Telefon              VARCHAR(15),
  Emailadresse         VARCHAR(25),
  CONSTRAINT eigenkunde_pk PRIMARY KEY (Kunden_ID),
  CONSTRAINT personalausweisNr_uk UNIQUE (PersonalausweisNr DESC)
)
ON KAM;
GO
        \end{lstlisting}
      
        %Erstellen Sie die Tabelle \identifier{mitarbeiter} gemäß Ihrem
        % ER-Modell und legen Sie ein \UNIQUE-Constraint auf die Spalte
        %\identifier{sozversnr}. Schließen Sie die Spalten \identifier{nachname}
        % und \identifier{vorname} in den Index als Nichtschlüsselspalten in.
        \item Konfigurieren Sie Ihre Instanz MSSQLSERVER so, dass Sie immer
mindestens 200M Arbeitsspeichern zur Verfügung hat und nie mehr als 2G
benutzt!

        \begin{lstlisting}[language=ms_sql]
USE [Bank_2014]
GO

CREATE TABLE Mitarbeiter (
  Mitarbeiter_ID   NUMERIC NOT NULL,
  Vorname          VARCHAR(30),
  Nachname         VARCHAR(35),
  Vorgesetzter_ID  NUMERIC,
  Bankfiliale_ID   NUMERIC,
  Geburtsdatum     DATETIME2,
  SozVersNr        VARCHAR(20),
  Gehalt           NUMERIC(12,2),
  Strasse          VARCHAR(50),
  Hausnummer       VARCHAR(20),
  PLZ              CHAR(5),
  Ort              VARCHAR(20),
  Provision        NUMERIC,
  CONSTRAINT Mitarbeiter_pk PRIMARY KEY (Mitarbeiter_ID)
)
ON KAM;
GO

CREATE UNIQUE NONCLUSTERED INDEX idx_SozVersNr
ON dbo.Mitarbeiter (SozVersNr)
INCLUDE (Vorname, Nachname);
GO
        \end{lstlisting}

        %Legen Sie einen nicht gruppierten Index auf die Tabelle
        %\identifier{eigenkunde} (Spalte \identifier{ablaufdatum}). Der Index
        %soll nur die Zeilen enthalten, die einen Personalausweis zeigen, der vor dem 01.01.2016
        %abgelaufen ist. Schließen Sie die Spalte \identifier{personalausweisnr}
        % als Nichtschlüsselspalte in den Index ein.
        \item Legen Sie einen nicht gruppierten Index auf die Tabelle
\identifier{eigenkunde} (Spalte \identifier{ablaufdatum}). Der Index soll nur
die Zeilen enthalten, die einen Personalausweis zeigen, der vor dem 01.01.2016
abgelaufen ist. Schließen Sie die Spalte \identifier{personalausweisnr} als
Nichtschlüsselspalte in den Index ein.
        \begin{lstlisting}[language=ms_sql]
USE [Bank_2014]
GO

CREATE NONCLUSTERED INDEX idx_ablaufdatum
ON Eigenkunde (Ablaufdatum)
INCLUDE (PersonalausweisNr)
WHERE  Ablaufdatum < CONVERT(DATETIME2, '2016-01-01');
GO
        \end{lstlisting}
\clearpage
        %Ermitteln Sie den Prozentsatz der logischen Indexfragmentierung der
        % beiden Indizes \identifier{mitarbeiter\_pk} und \identifier{girokonto\_pk}.
            \item Konfigurieren Sie zwei Net Service Names, einen f\"ur eine Dedicated Server Verbindung und einen f\"ur eine DRCP-Verbindung. Testen Sie beide Verbindungen (Hinweis: Um die DRCP-Verbindung testen zu k\"onnen, muss vorher das Connection Pooling aktiviert werden, siehe: \myhref{http://docs.oracle.com/cd/B28359_01/server.111/b28318/process.htm#CIHBIFCI}{Database Resident Connection Pooling, Database Concepts 11g Release 2, Kapitel 9})

        \begin{lstlisting}[language=ms_sql]
USE [Bank_2014]
GO

SELECT a.index_id, name, avg_fragmentation_in_percent
FROM   sys.dm_db_index_physical_stats(DB_ID(N'Bank'), 
       OBJECT_ID('dbo.mitarbeiter'), NULL, NULL, NULL) AS a
       JOIN sys.indexes AS b
     ON (a.object_id = b.object_id AND a.index_id = b.index_id);
GO

/* 1 mitarbeiter_pk 80 -> REBUILD */ 

SELECT a.index_id, name, avg_fragmentation_in_percent
FROM   sys.dm_db_index_physical_stats(DB_ID(N'Bank'), 
       OBJECT_ID('dbo.girokonto'), NULL, NULL, NULL) AS a
       JOIN sys.indexes AS b
     ON (a.object_id = b.object_id AND a.index_id = b.index_id);
GO

/* 1 girokonto_pk 25 -> REORGANIZE */
        \end{lstlisting}

        %Ergreifen Sie für jeden der beiden Indizes die geeignete Maßnahme, um
        %deren Fragmentierung zu reduzieren.
            \item Schalten Sie das SQL-Autotracing und das Timing wieder aus: \languagesqlplus{set autotrace off} und \languagesqlplus{set timing off}

        \begin{lstlisting}[language=ms_sql]
USE [Bank_2014]
GO

/* 1 mitarbeiter_pk 80 -> REBUILD */ 

ALTER INDEX mitarbeiter_pk 
ON Mitarbeiter
REBUILD;

/* 1 girokonto_pk 25 -> REORGANIZE */

ALTER INDEX girokonto_pk
ON Girokonto
REORGANIZE;
        \end{lstlisting}
\clearpage
        %Erstellen Sie die Tabelle \identifier{girokonto} gemäß Ihrem ER-Modell.
        %Legen Sie den Primärschlüssel dieser Tabelle mit einem clustered Index
        % an.
            \item Weisen Sie den Nutzern \identifier{alice}, \identifier{bob} und \identifier{chloe} das Profil \identifier{p\_clerk} zu und testen Sie die Auswirkungen!

        \begin{lstlisting}[language=ms_sql]
USE [Bank_2014]
GO

CREATE TABLE Girokonto (
  Konto_ID       NUMERIC NOT NULL,
  Gebuehr        NUMERIC(12,2),
  Guthaben       NUMERIC(12,2),
  Habenzins      NUMERIC(5,2),
  Kreditrahmen   NUMERIC(12,2),
  CONSTRAINT girokonto_pk PRIMARY KEY CLUSTERED (Konto_ID)
)
ON KAM;
        \end{lstlisting}

        %Erstellen Sie einen nicht gruppierten Index aufl der Spalte
        %\identifier{guthaben} der Tabelle \identifier{girokonto}.
            \item Melden Sie sich als Nutzer \identifier{alice} (Passwort: Welcome01\#) an Ihrer Datenbank an und versuchen Sie das folgende SQL-Statement abzusetzen:
    \begin{lstlisting}[language=oracle_sql]
&SQL>& INSERT INTO full
  2 VALUES (1,1,1);
    \end{lstlisting}
    Die Ausf\"uhrung des Statements wird scheitern. Pr\"ufen Sie im Alert.log, warum das Statement gescheitert ist.

        \begin{lstlisting}[language=ms_sql]
USE [Bank_2014]
GO

CREATE NONCLUSTERED INDEX idx_guthaben
ON Girokonto (Guthaben);
GO
        \end{lstlisting}

        %Deaktivieren Sie den gruppierten Index der Tabelle
        % \identifier{girokonto} und beantworten Sie die folgenden Fragen:
            \item Ändern Sie die Sperrdauer des Accounts im Profil \identifier{p\_clerk} auf unbegrenzt!

        \begin{lstlisting}[language=ms_sql]
USE [Bank_2014]
GO

ALTER INDEX girokonto_pk
ON Girokonto
DISABLE;
        \end{lstlisting}
\clearpage
        %Suchen Sie einen Weg, den gruppierten und den nicht gruppierten Index
        % der Tabelle \identifier{girokonto} in einem Zug zu reaktivieren.
            \item L\"oschen Sie das Nutzerprofil \identifier{p\_clerk} in einem Arbeitsschritt!

        \begin{lstlisting}[language=ms_sql]
USE [Bank_2014]
GO

ALTER INDEX ALL
ON Girokonto
REBUILD;
        \end{lstlisting}

        %Lesen Sie den Artikel \parencite{utilUDPaC} und notieren Sie sich
        %mindestens drei sinnvolle Fragen!
        \item Lesen Sie den Artikel \parencite{utilUDPaC} und notieren Sie sich
mindestens drei sinnvolle Fragen!

  \rule{0.94\textwidth}{0.5pt}

  \rule{0.94\textwidth}{0.5pt}

  \rule{0.94\textwidth}{0.5pt}

  \rule{0.94\textwidth}{0.5pt}

  \rule{0.94\textwidth}{0.5pt}

  \rule{0.94\textwidth}{0.5pt}

  \rule{0.94\textwidth}{0.5pt}

  \rule{0.94\textwidth}{0.5pt}

      \end{enumerate}