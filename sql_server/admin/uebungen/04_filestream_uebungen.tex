\clearpage
    \section{Projekt - Benutzen von Filestream und FileTables}
      \begin{enumerate}
        %Führen Sie das Skript \identifier{add_foto_column.sql} aus. Dieses
        % Skript fügt der Tabelle \identifier{mitarbeiter} in der Datenbank
        %\identifier{bank\_2014} eine Spalte namens \identifier{foto} hinzu.
            \item Ermitteln Sie, ob der serverseitige Result Cache aktiviert ist!


        %Führen Sie das Skript \identifier{add\_foto\_column.sql} aus. Dieses
        %Skript fügt der Tabelle \identifier{mitarbeiter} in der Datenbank
        %\identifier{bank\_2014} eine Spalte namens \identifier{foto} hinzu.
            \item Welchen temporären Tablespace nutzt alice jetzt und welcher View können Sie diese Angabe entnehmen?

        
        %Legen Sie die Filestream-Dateigruppe \identifier{externalresources} an. Das Filestream-Verzeichnis
        % soll in \oscommand{D:\textbackslash u01\textbackslash bank\_2014\textbackslash
        %filestream\textbackslash fsstorage} liegen. Der logische Name für das
        %Filestream-Verzeichnis soll \identifier{bank\_2014\_externalresources}
        % sein.
            \item Erstellen Sie den Nutzer \identifier{bob} mit dem Passwort \enquote{Pass1/gH,3word}.

        
        %Erstellen Sie die FileTable \identifier{fotos}. Benutzen Sie
        %\identifier{mitarbeiter\_fotos} als FileTable-Verzeichnis.
            \item Welchen Default Tablespace benutzt \identifier{bob} und aus welcher View können Sie diese Angabe entnehmen?

        
        %Benutzen Sie die die \enquote{OpenRowset}-Funktionalität des SQL
        % Servers, um die Fotos der Mitarbeiter in die FileTable \identifier{fotos} zu importieren.
        %Verknüpfen Sie die Inhalt der Tabelle \identifier{mitarbeiter} mit den
        %Fotos in der Tabelle \identifier{fotos}. Welches Attribut der FileTable
        % eignet sich dafür?
            \item F\"uhren Sie ein Recovery bei Verlust einer Kontrolldatei durch.
      \begin{enumerate}
        \item Starten Sie das Skript \oscommand{lab\_delete\_ctlfl.sql}. Es l\"oscht eine Ihrer Kontrolldateien.
          \begin{lstlisting}[language=terminal]
SQL> @/home/oracle/labs/lab_delete_ctlfl.sql
          \end{lstlisting}
        \item Analysieren Sie das Problem und f\"uhren Sie ein geeignetes Recovery durch.
      \end{enumerate}

      \end{enumerate}