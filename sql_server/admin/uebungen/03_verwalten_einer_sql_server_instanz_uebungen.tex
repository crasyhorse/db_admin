\clearpage
    \section{\"Ubungen - Verwalten einer SQL Server-Instanz}
      \begin{enumerate}
        %Richten Sie auf Ihrem Übungsserver alle notwendigen Werkzeuge ein, um
        % mit der Domäne \identifier{MS-C-IX-04.FUS} arbeiten zu können.
            \item Ermitteln Sie, ob der serverseitige Result Cache aktiviert ist!


        %Nehmen Sie Ihren Übungsserver in die Domäne
        %\identifier{MS-C-IX-04.FUS} auf. Achten Sie darauf, dass das
        %Computerkonto Ihres Datenbankservers in der 
        %Organisationseinheit \oscommand{IT-SysAdmin\textbackslash
        %EXER\textbackslash Serverxx\textbackslash Computers} abgelegt wird,
        % bzw. verschieben Sie es dorthin.
            \item Welchen temporären Tablespace nutzt alice jetzt und welcher View können Sie diese Angabe entnehmen?

        
        %Deaktivieren Sie den Zugriff mittels Shared Memory auf Ihre SQL
        %Server-Instanz!
            \item Erstellen Sie den Nutzer \identifier{bob} mit dem Passwort \enquote{Pass1/gH,3word}.

        
        %Der Zugriff auf Ihre SQL Server-Instanz soll nur noch mittels der 
        %öffentlichen IP-Adresse 192.168.111.xx erfolgen können! Alle anderen
        % IPs müssen deaktiviert werden!
            \item Welchen Default Tablespace benutzt \identifier{bob} und aus welcher View können Sie diese Angabe entnehmen?

        
        %Richten Sie die im Folgenden aufgeführten gMSAs in der
        %Organisationseinheit \oscommand{IT-SysAdmin\textbackslash
        %EXER\textbackslash Serverxx\textbackslash MSAs} für die Dienste Ihrer
        % SQL Server-Instanz ein. Ersetzen Sie dabei \enquote{xx} durch Ihre Platznummer!
            \item F\"uhren Sie ein Recovery bei Verlust einer Kontrolldatei durch.
      \begin{enumerate}
        \item Starten Sie das Skript \oscommand{lab\_delete\_ctlfl.sql}. Es l\"oscht eine Ihrer Kontrolldateien.
          \begin{lstlisting}[language=terminal]
SQL> @/home/oracle/labs/lab_delete_ctlfl.sql
          \end{lstlisting}
        \item Analysieren Sie das Problem und f\"uhren Sie ein geeignetes Recovery durch.
      \end{enumerate}


        %Stoppen Sie die beiden Dienste \identifier{SQL Server Reporting
        %Services} und \identifier{SQL Server Integration Services 12.0}.
        %Konfigurieren Sie beide so, dass Sie manuell gestartet werden müssen.
            \item Machen Sie alle Auditingeinstellungen r\"uckg\"angig und l\"oschen Sie den Inhalt des Auditingtrails!


        %Konfigurieren Sie Ihre Instanz MSSQLSERVER so, dass Sie immer
        %mindestens 200M Arbeitsspeichern zur Verfügung hat und nie mehr als 2G
        %benutzt!
        \item Konfigurieren Sie Ihre Instanz MSSQLSERVER so, dass Sie immer
mindestens 200M Arbeitsspeichern zur Verfügung hat und nie mehr als 2G
benutzt!


        %Schreiben Sie eine SQL-Abfrage, die anzeigen, ob die Einstellungen
        %aus der vorangegangenen Aufgabe bereits wirksam geworden sind oder
        %nicht!
        \item Legen Sie einen nicht gruppierten Index auf die Tabelle
\identifier{eigenkunde} (Spalte \identifier{ablaufdatum}). Der Index soll nur
die Zeilen enthalten, die einen Personalausweis zeigen, der vor dem 01.01.2016
abgelaufen ist. Schließen Sie die Spalte \identifier{personalausweisnr} als
Nichtschlüsselspalte in den Index ein.
        
        %Deaktivieren Sie die Überwachung von fehlerhaften Anmeldungen. Es
        %soll keine Anmeldeüberwachung geben!
            \item Konfigurieren Sie zwei Net Service Names, einen f\"ur eine Dedicated Server Verbindung und einen f\"ur eine DRCP-Verbindung. Testen Sie beide Verbindungen (Hinweis: Um die DRCP-Verbindung testen zu k\"onnen, muss vorher das Connection Pooling aktiviert werden, siehe: \myhref{http://docs.oracle.com/cd/B28359_01/server.111/b28318/process.htm#CIHBIFCI}{Database Resident Connection Pooling, Database Concepts 11g Release 2, Kapitel 9})


        %Fügen Sie Ihre Instanz MSSQLSERVER die Datenbank \identifier{Bank
        %2014} hinzu. Der Aufbau der Datenbank kann der folgenden Tabelle
        %entnommen werden. Die Namen der Dateigruppen sind in den logischen
        % Namen der Datendateien enthalten (Primary, HR, CRM, KAM und MAIL).
            \item Schalten Sie das SQL-Autotracing und das Timing wieder aus: \languagesqlplus{set autotrace off} und \languagesqlplus{set timing off}

        
        %Fügen Sie der \identifier{Bank 2014} die Dateigruppe
        %\identifier{staging} hinzu. Benutzen Sie dazu die folgenden Angaben:
            \item Weisen Sie den Nutzern \identifier{alice}, \identifier{bob} und \identifier{chloe} das Profil \identifier{p\_clerk} zu und testen Sie die Auswirkungen!

        
        %Fügen Sie der Dateigruppe \identifier{crm} eine weitere Datendatei
        %hinzu. Verwenden Sie dazu die folgenden Angaben:
            \item Melden Sie sich als Nutzer \identifier{alice} (Passwort: Welcome01\#) an Ihrer Datenbank an und versuchen Sie das folgende SQL-Statement abzusetzen:
    \begin{lstlisting}[language=oracle_sql]
&SQL>& INSERT INTO full
  2 VALUES (1,1,1);
    \end{lstlisting}
    Die Ausf\"uhrung des Statements wird scheitern. Pr\"ufen Sie im Alert.log, warum das Statement gescheitert ist.

        
        %Nehmen Sie die im Folgenden beschriebenen Verändernungen an den
        %Datendateien Ihrer Datenbank \identifier{Bank 2014} vor.
            \item Ändern Sie die Sperrdauer des Accounts im Profil \identifier{p\_clerk} auf unbegrenzt!

      
        %Schreiben Sie eine Abfrage, um zu überprüfen, ob die Datendatei
        %ordnungsgemäß angefügt wurde.
            \item L\"oschen Sie das Nutzerprofil \identifier{p\_clerk} in einem Arbeitsschritt!

        
        %Verschieben Sie die Datendatei mit dem Namen
        %\identifier{bank\_2014\_kam\_01} nach \oscommand{D:\textbackslash
        %u01\textbackslash bank\_2014\textbackslash data}.
        \item Lesen Sie den Artikel \parencite{utilUDPaC} und notieren Sie sich
mindestens drei sinnvolle Fragen!

  \rule{0.94\textwidth}{0.5pt}

  \rule{0.94\textwidth}{0.5pt}

  \rule{0.94\textwidth}{0.5pt}

  \rule{0.94\textwidth}{0.5pt}

  \rule{0.94\textwidth}{0.5pt}

  \rule{0.94\textwidth}{0.5pt}

  \rule{0.94\textwidth}{0.5pt}

  \rule{0.94\textwidth}{0.5pt}

        
        %Machen Sie die Dateigruppe \identifier{CRM} zur Standarddateigruppe
        %der Datenbank \identifier{Bank 2014}.
            \item Verschieben Sie die Datendatei \oscommand{uebungs\_ts02.dbf} nach \oscommand{/u02/oradata/orcl/}.

        
        %Konfigurieren Sie für die Datenbankoption
        %\identifier{quoted\_identifier} den Wert \identifier{true}.
              \item Um die weiteren Übungsaufgeben bearbeiten zu können, führen Sie bitte das Skript \oscommand{lab\_java.sql} als Nutzer \identifier{SYS} aus. Das SQL-Skript befindet sich im Verzeichnis \oscommand{/home/oracle/labs}.
        \begin{lstlisting}[language=terminal]
SQL> @/home/oracle/labs/lab_java.sql
        \end{lstlisting}

        
        %Ermitteln Sie welchen Wert die Datenbankoption
        %\identifier{auto\_close} hat und recherchieren Sie, welche Bedeutung
        % diese Option hat bzw. was sie bewirkt.
        \item Ermitteln Sie welchen Wert die Datenbankoption \identifier{auto\_close}
hat und recherchieren Sie, welche Bedeutung diese Option hat bzw. was sie
bewirkt.


\clearpage
        %\item Verschieben Sie Ihre Datenbank \identifier{tempdb} gemäß den
        %folgenden Angaben:
              \item Passen Sie das Setup Ihrer Redo Log Gruppen so an, dass Sie drei Gruppen mit je 3 Membern haben. Verteilen Sie die Member sinnvoll \"uber die vorhandenen Datentr\"ager \oscommand{/u01} bis \oscommand{/u05} und notieren Sie sich die \"Anderungen in der obigen Tabelle.

      
        %Verschieben Sie das Fehlerprotokoll Ihrer Instanz nach
        %\oscommand{D:\textbackslash u01\textbackslash errorlog}.
            \item Pr\"ufen Sie im RMAN welche Backups nun nicht mehr zur Verf\"ugung stehen.

        
        %Beim Anlegen der Datenbank ist Ihnen ein Fehler unterlaufen. Statt
        %die Datenbank \identifier{Bank\_2014} zu nennen, haben Sie sie
        %\identifier{Bank 2014} genannt. Recherchieren Sie, wie eine
        %Datenbank umbenannt werden kann und ersetzten Sie das Leerzeichen im
        %Datenbanknamen durch einen Unterstrich!
              \item Ändern Sie die Parameter der Archivierung wie folgt:
        \begin{center}
          \begin{small}
            \tablefirsthead{
              \multicolumn{1}{c}{\textbf{Optional}} &
              \multicolumn{1}{c}{\textbf{/u02}} &
              \multicolumn{1}{c}{\textbf{/u03}} &
              \multicolumn{1}{c}{\textbf{/u04}} &
              \multicolumn{1}{c}{\textbf{/u05}} \\
              \hline
            }
            \tablefirsthead{
              \multicolumn{1}{c}{\textbf{Optional}} &
              \multicolumn{1}{c}{\textbf{/u02}} &
              \multicolumn{1}{c}{\textbf{/u03}} &
              \multicolumn{1}{c}{\textbf{/u04}} &
              \multicolumn{1}{c}{\textbf{/u05}} \\
              \hline
            }
            \tabletail{
              \hline
            }
            \tablelasttail {
              \hline
            }
            \begin{supertabular}{| >{\centering\arraybackslash}m{1cm}| >{\centering\arraybackslash}m{2cm}| >{\centering\arraybackslash}m{2cm}| >{\centering\arraybackslash}m{2cm}| >{\centering\arraybackslash}m{2cm}|}
               Ja &  dest 1 &  - &  - & - \\
              \hline
               Nein &  - &  dest 2 &  - & - \\
              \hline
               Ja &  - &  - &  dest 3 & - \\
            \end{supertabular}
          \end{small}
        \end{center}
        \begin{itemize}
          \item Die Archivierung muss an mindestens zwei Speicherorten erfolgreich sein.
          \item Das Dateinamensformat der Archive Log Files muss wie folgt aufgebaut sein: \oscommand{archive\_\%d\_\%t\_\%s\_\%r.arc}
        \end{itemize}

        
    \end{enumerate}