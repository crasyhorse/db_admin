\clearpage
    \section{\"Ubungen - Schemaobjekte verwalten}
      Benutzen Sie für die Durchführung der Übungen die Datenbank
      \identifier{Bank\_2014}.
      \begin{enumerate}
        %Erstellen Sie die Tabelle \identifier{bank} gemäß Ihrem ER-Modell. 
        %Machen Sie das Attribut \identifier{bank\_id} zum Primärschlüssel und
        % achten sie darauf, dass die Tabelle als Heap erstellt wird.
            \item Ermitteln Sie, ob der serverseitige Result Cache aktiviert ist!

        
        %Ist es sinnvoll, die Tabelle \identifier{bank} als Heap anzulegen?
        %Begründen Sie Ihre Aussage!
            \item Welchen temporären Tablespace nutzt alice jetzt und welcher View können Sie diese Angabe entnehmen?

        
        %Legen Sie ein \UNIQUE-Constraint auf die Spalte \identifier{bic} der
        %Tabelle \identifier{bank}. Die Tabelle muss weiterhin in Form eines
        %Heaps gespeichert bleiben.
            \item Erstellen Sie den Nutzer \identifier{bob} mit dem Passwort \enquote{Pass1/gH,3word}.

        
        %Fügen Sie Ihrer Datenbank die Dateigruppe \identifier{in\_memory} hinzu
        %und bereiten Sie alles vor, um eine SOT darin anzulegen.
            \item Welchen Default Tablespace benutzt \identifier{bob} und aus welcher View können Sie diese Angabe entnehmen?

        
        %Legen Sie die Tabelle \identifier{buchung} gemäß Ihrem ER-Modell als SOT
        %an und fügen Sie zwischen die beiden Spalten
        %\identifier{buchungsdatum} und \identifier{konto\_id} die Spalte
        % \identifier{buchungstext} VARCHAR(200) ein! Legen Sie auf die
        %Spalte \identifier{buchungsdatum} einen Index. Legen Sie auch
        %auf die Spalte \identifier{buchungstext} einen Index.
            \item F\"uhren Sie ein Recovery bei Verlust einer Kontrolldatei durch.
      \begin{enumerate}
        \item Starten Sie das Skript \oscommand{lab\_delete\_ctlfl.sql}. Es l\"oscht eine Ihrer Kontrolldateien.
          \begin{lstlisting}[language=terminal]
SQL> @/home/oracle/labs/lab_delete_ctlfl.sql
          \end{lstlisting}
        \item Analysieren Sie das Problem und f\"uhren Sie ein geeignetes Recovery durch.
      \end{enumerate}

        
        %Erstellen Sie die Tabelle \identifier{eigenkunde} gemäß Ihrem ER-Modell
        %und legen Sie ein \UNIQUE-Constraint auf die Spalte
        %\identifier{personalausweisnr}. Die Sortierung des Indexes, der zu
        % diesem Constraint gehört soll absteigend sein.
            \item Machen Sie alle Auditingeinstellungen r\"uckg\"angig und l\"oschen Sie den Inhalt des Auditingtrails!

        
        %Erstellen Sie die Tabelle \identifier{mitarbeiter} gemäß Ihrem
        % ER-Modell und legen Sie ein \UNIQUE-Constraint auf die Spalte
        %\identifier{sozversnr}. Schließen Sie die Spalten \identifier{nachname}
        % und \identifier{vorname} in den Index als Nichtschlüsselspalten in.
        \item Konfigurieren Sie Ihre Instanz MSSQLSERVER so, dass Sie immer
mindestens 200M Arbeitsspeichern zur Verfügung hat und nie mehr als 2G
benutzt!


        %Legen Sie einen nicht gruppierten Index auf die Tabelle
        %\identifier{eigenkunde} (Spalte \identifier{ablaufdatum}). Der Index
        %soll nur die Zeilen enthalten, die einen Personalausweis zeigen, der vor dem 01.01.2016
        %abgelaufen ist. Schließen Sie die Spalte \identifier{personalausweisnr}
        % als Nichtschlüsselspalte in den Index ein.
        \item Legen Sie einen nicht gruppierten Index auf die Tabelle
\identifier{eigenkunde} (Spalte \identifier{ablaufdatum}). Der Index soll nur
die Zeilen enthalten, die einen Personalausweis zeigen, der vor dem 01.01.2016
abgelaufen ist. Schließen Sie die Spalte \identifier{personalausweisnr} als
Nichtschlüsselspalte in den Index ein.

        %Ermitteln Sie den Prozentsatz der logischen Indexfragmentierung der
        % beiden Indizes \identifier{mitarbeiter\_pk} und \identifier{girokonto\_pk}.
            \item Konfigurieren Sie zwei Net Service Names, einen f\"ur eine Dedicated Server Verbindung und einen f\"ur eine DRCP-Verbindung. Testen Sie beide Verbindungen (Hinweis: Um die DRCP-Verbindung testen zu k\"onnen, muss vorher das Connection Pooling aktiviert werden, siehe: \myhref{http://docs.oracle.com/cd/B28359_01/server.111/b28318/process.htm#CIHBIFCI}{Database Resident Connection Pooling, Database Concepts 11g Release 2, Kapitel 9})


        %Ergreifen Sie für jeden der beiden Indizes die geeignete Maßnahme, um
        %deren Fragmentierung zu reduzieren.
            \item Schalten Sie das SQL-Autotracing und das Timing wieder aus: \languagesqlplus{set autotrace off} und \languagesqlplus{set timing off}


        %Erstellen Sie die Tabelle \identifier{girokonto} gemäß Ihrem ER-Modell.
        %Legen Sie den Primärschlüssel dieser Tabelle mit einem clustered Index
        % an.
            \item Weisen Sie den Nutzern \identifier{alice}, \identifier{bob} und \identifier{chloe} das Profil \identifier{p\_clerk} zu und testen Sie die Auswirkungen!


        %Erstellen Sie einen nicht gruppierten Index aufl der Spalte
        %\identifier{guthaben} der Tabelle \identifier{girokonto}.
            \item Melden Sie sich als Nutzer \identifier{alice} (Passwort: Welcome01\#) an Ihrer Datenbank an und versuchen Sie das folgende SQL-Statement abzusetzen:
    \begin{lstlisting}[language=oracle_sql]
&SQL>& INSERT INTO full
  2 VALUES (1,1,1);
    \end{lstlisting}
    Die Ausf\"uhrung des Statements wird scheitern. Pr\"ufen Sie im Alert.log, warum das Statement gescheitert ist.


        %Deaktivieren Sie den gruppierten Index der Tabelle
        % \identifier{girokonto} und beantworten Sie die folgenden Fragen:
            \item Ändern Sie die Sperrdauer des Accounts im Profil \identifier{p\_clerk} auf unbegrenzt!


        %Suchen Sie einen Weg, den gruppierten und den nicht gruppierten Index
        % der Tabelle \identifier{girokonto} in einem Zug zu reaktivieren.
            \item L\"oschen Sie das Nutzerprofil \identifier{p\_clerk} in einem Arbeitsschritt!


        %Lesen Sie den Artikel \parencite{utilUDPaC} und notieren Sie sich
        mindestens drei sinnvolle Fragen!
        \item Lesen Sie den Artikel \parencite{utilUDPaC} und notieren Sie sich
mindestens drei sinnvolle Fragen!

  \rule{0.94\textwidth}{0.5pt}

  \rule{0.94\textwidth}{0.5pt}

  \rule{0.94\textwidth}{0.5pt}

  \rule{0.94\textwidth}{0.5pt}

  \rule{0.94\textwidth}{0.5pt}

  \rule{0.94\textwidth}{0.5pt}

  \rule{0.94\textwidth}{0.5pt}

  \rule{0.94\textwidth}{0.5pt}

      \end{enumerate}