\item Fügen Sie der \identifier{Bank 2014} die Dateigruppe
\identifier{staging} hinzu. Benutzen Sie dazu die folgenden Angaben:
\begin{center}
  \begin{small}
    \changefont{pcr}{m}{n}
    \tablefirsthead {
      \multicolumn{1}{c}{\textbf{Name}} &
      \multicolumn{1}{c}{\textbf{Größe}} &
      \multicolumn{1}{c}{\textbf{Wachstum}} &
      \multicolumn{1}{c}{\textbf{G Max.}} &
      \multicolumn{1}{c}{\textbf{Pfad}} \\
    }
    \tablehead{
    }
    \tabletail {
    }
    \tablelasttail {
    }
    \begin{supertabular}{lrrrl}
      bank\_2014\_staging\_01 & 800 M &  + 40 M & 2048 M &
      E:\textbackslash u02\textbackslash bank\_2014\textbackslash data \\
    \end{supertabular}
  \end{small}
\end{center}
