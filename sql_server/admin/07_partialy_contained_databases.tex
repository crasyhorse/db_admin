  \chapter{Partially contained databases}
    \setcounter{page}{1}\kapitelnummer{chapter}
    \minitoc
\newpage
    \section{Partially contained databases - Teilweise eigenständige
    Datenbanken} 
      Die Architektur des Microsoft SQL Server sieht vor, dass Einstellungen,
      die zum Betrieb einer Datenbank notwendig sind, an verschiedenen Orten
      gespeichert werden. Beispielsweise werden Logins in der
      \identifier{master}-Datenbank gespeichert, um instanzweit zur Verfügung zu
      stehen, während Benutzerkonten direkt in der betreffenden Datenbank
      abgelegt werden. Ein anderes Beispiel sind Jobs des SQL Server Agent.
      Diese werden in der \identifier{msdb}-Datenbank abgelegt, können sich aber
      auf Datenbanken auswirken.
      
      Die durch diese Umstände bewirkte Verflechtung von Instanz und Datenbanken
      macht es schwierig eine Datenbank auf einen anderen Server / in eine
      andere Instanz zu migrieren. Mit Hilfe des Konzeptes der
      \enquote{Teilweise eigenständigen Datenbank} (engl. Partially contained
      databases) versucht Microsoft diesem Problem entgegen zu treten.

      In einer Partially contained database werden verschiedene Metadaten
      in der Datenbank selbst und nicht in der \identifier{master}-Datenbank
      gespeichert. Der Fachbegriff dafür, dass bestimmte Objekte ohne
      Verbindung zur Instanz oder zu einer anderen Datenbank existieren können
      lautet \enquote{Kapselung}
      \begin{merke}
        Von großer Bedeutung ist das Wort \enquote{Partially}, da nicht alle
        Metadaten in der Datenbank selbst abgelegt werden. Diese Art von
        Datenbank ist tatsächlich nur zu einem gewissen Teil unabhängig.
      \end{merke}
      \subsection{Was genau bedeutet Kapselung?}
        Bei der Kapselung geht es darum, dass Objekte in einer Datenbank
        keinerleie Bezüge/Verknüpfungen nach außen haben dürfen. Ein Beispiel
        hierfür ist die Systemview \identifier{sys.tables}. Die Basisobjekte,
        auf welche sich diese View bezieht, befinden sich alle in der
        betroffenen Datenbank selbst. Ein Gegenbeispiel wäre ein Objekt, wie
        z. B. ein SQL-Benutzer mit Anmeldename. Ein solches Benutzerkonto
        bezieht sich auf ein Windows-Login und dieses Login liegt in der
        \identifier{master}-Datenbank. Somit existiert ein Bezug nach außen.
        
        Man kann somit zwei Arten von Objekten unterscheiden:
        \begin{itemize}
          \item \textbf{Contained entities}: Ein Objekt wird dann als
          \enquote{vollständig enthalten} bezeichnet, wenn es die
          Datenbankbegrenzung nie überschreitet, sich also nur auf Funktionen
          stützt, welche in der Datenbank enthalten sind.
          \item \textbf{Uncontained entities}: Ein Objekt, welches die
          Datenbankgrenze überschreitet wird als \enquote{nicht enthalten}
          bezeichnet, da es Verknüpfungen zu Objekten außerhalb seiner eigenen
          Datenbank hat.
        \end{itemize}
        \begin{merke}
          Als \enquote{Datenbankbegrenzung} wird die Grenze zwischen einer
          Datenbank und der Instanz bzw. die Grenze zwischen zwei Datenbanken
          bezeichnet. Sie ist eine eindeutig definierbare Trennlinie zwischen
          den Funktionen der Datenbank (dem Datenmodell) und den Funktionen
          der Instanz (dem Verwaltungsmodell).
        \end{merke}
        \subsection{Welche Probleme werden gelöst?}
          \begin{itemize}
            \item Die Datenbank übernimmt die Authentifizierung von Benutzern
            komplett. Es sind keine Logins mehr von Nöten.
            \item Weitere Metadaten, z. B. eine Liste der Datenbanken, von denen
            die Partially contained database abhängig ist und weitere
            Systemeinstellungen, die zum Betrieb der Datenbank notwendig
            sind, werden in der Datenbank selbst gespeichert.
            \item Temporäre Tabellen werden direkt im Kontext der Partially
            contained database gespeichert, so dass es keine Konflikte mehr bei
            Sortierreihenfolgen zwischen der \identifier{tempdb} und der
            Datenbank selbst gibt.
          \end{itemize}
        \subsection{Einschränkungen für Partially contained databases}
          Das Konzept der teilweise eigenständigen Datenbanken bringt nicht nur
          Vorteile mit sich, es existieren auch Einschränkungen. Beispielsweise
          unterstützen diese Datenbank weder das Feature der Replikation noch
          das Change-Data-Capture. Eine vollständige Liste mit allen
          Einschränkungen ist in der MSDN enthalten.
          \begin{literaturinternet}
            \item \cite{ff929071}
          \end{literaturinternet}
      \section{Partially contained databases erzeugen}
        \subsection{Konfigurieren der Instanz}
          Bevor Partially contained databases erstellt werden können muss zuerst
          die Authentifizierung für eigenständige Datenbanken eingeschaltet
          werden. Dies wird durch eine einfache Instanzeinstellung vorgenommen.
          \bild{Eigen\-stän\-dige Daten\-banken aktivieren}{pcd_activate}{1.2}
          \begin{lstlisting}[language=ms_sql,caption={Eigen\-stän\-dige Daten\-banken
          aktivieren},label=sql20_01]
USE [master]
GO

EXEC §sp_configure§ 'contained database authentication', 1
RECONFIGURE
GO          
          \end{lstlisting}
        \subsection{Erstellen der Datenbank}
          \subsubsection{Eine neue Partially contained database erstellen}
            Eine teilweise eigenständige Datenbank wird mit Hilfe des
            \languagemssql{CREATE DATABASE}-Statements und dem Zusatz
            \languagemssql{CONTAINMENT = PARTIAL} erzeugt. Wahlweise kann dies
            aber auch im SSMS geschehen.
            \bild{Erzeugen einer teilweise eigen\-stän\-digen Daten\-bank}{create_pcd}{1.2}
            \begin{lstlisting}[language=ms_sql,caption={Erstellen einer teilweise eigen\-stän\-digen Daten\-bank},label=sql20_02]
USE [master]
GO

CREATE DATABASE [Bank_PCD]
  CONTAINMENT = PARTIAL
  ON  PRIMARY (
  NAME = N'Bank_PCD',
  FILENAME = N'D:\120\bank_pcd\data\Bank_pcd.mdf',
  SIZE = 10GB,
  MAXSIZE = 20GB,
  FILEGROWTH = 512MB
  )
  LOG ON (
  NAME = N'Bank_PCD_Log',
  FILENAME = N'D:\120\bank_pcd\log\Bank_pcd.ldf',
  SIZE = 100M,
  MAXSIZE = 2GB,
  FILEGROWTH = 10%
  )
GO
            \end{lstlisting}
          \subsubsection{Den Containmenttype einer Datenbank ändern}
          
          \subsubsection{Identifizieren der Datenbankkapselung}
          %https://msdn.microsoft.com/de-de/library/ff929071(v=sql.120).aspx#Identifizieren%20der%20Datenbankkapselung
        \subsection{Contained logins verwalten}
          \subsubsection{Das Passwort eines Benutzerkontos ändern}

        \subsection{Collations}
