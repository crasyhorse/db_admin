  \chapter{Partially contained databases}
    \setcounter{page}{1}\kapitelnummer{chapter}
    \minitoc
\newpage
    \section{Partially contained databases - Teilweise eigenständige
    Datenbanken} 
      Die Architektur des Microsoft SQL Server sieht vor, dass Einstellungen,
      die zum Betrieb einer Datenbank notwendig sind, an verschiedenen Orten
      gespeichert werden. Beispielsweise werden Logins in der
      \identifier{master}-Datenbank gespeichert, um instanzweit zur Verfügung zu
      stehen, während Benutzerkonten direkt in der betreffenden Datenbank
      abgelegt werden. Ein anderes Beispiel sind Jobs des SQL Server Agent.
      Diese werden in der \identifier{msdb}-Datenbank abgelegt, können sich aber
      auf Datenbanken auswirken.
      
      Die durch diese Umstände bewirkte Verflechtung von Instanz und Datenbanken
      macht es schwierig eine Datenbank auf einen anderen Server / in eine
      andere Instanz zu migrieren. Mit Hilfe des Konzeptes der
      \enquote{Teilweise eigenständigen Datenbank} (engl. Partially contained
      databases).
      \subsection{Die Idee dahinter}
        In einer Partially contained database werden verschiedene Metadaten
        in der Datenbank selbst und nicht in der \identifier{master}-Datenbank
        gespeichert.
        \begin{merke}
          Von großer Bedeutung ist das Wort \enquote{Partially}, da nicht alle
          Metadaten in der Datenbank selbst abgelegt werden. Diese Art von
          Datenbank ist tatsächlich nur zu einem gewissen Teil unabhängig.
        \end{merke}
        \subsubsection{Welche Probleme werden gelöst?}
          \begin{itemize}
            \item Die Datenbank übernimmt die Authentifizierung von Benutzern
            komplett. Es sind keine Logins mehr von Nöten.
            \item Weitere Metadaten, z. B. eine Liste der Datenbanken, von denen
            die Partially contained database abhängig ist und weitere
            Systemeinstellungen, die zum Betrieb der Datenban notwendig
            sind, werden in der Datenbank selbst gespeichert.
            \item Temporäre Tabellen werden direkt im Kontext der Partially
            contained database gespeichert, so dass es keine Konflikte mehr bei
            Sortierreihenfolgen zwischen der \identifier{tempdb} und der
            Datenbank selbst gibt.
          \end{itemize}
          
          \subsubsection{Das Passwort eines Benutzerkontos ändern}
%https://msdn.microsoft.com/de-de/library/ms176060.aspx?f=255&MSPPError=-2147217396